\chapter{Potions}

\key{Potioneering}, also known as the arcane art of \key{Alchemy}, is one of the most ancient of the \imp{artificing} practices.

Whilst the sister arts of \key{enchanting} and \key{crafting} focus on the creation and imbuement of physical items, \imp{potioneering} focuses on the mixing, brewing and steeping of magical fluids, balms and tinctures: \imp{potions} which imbue the user with magical effects when they are drunk, spilled, applied to the skin, or any other myriad forms of application. 

Unlike the other \imp{artificing arts}, the ingredients used for \imp{alchemy} are almost entirely natural, herbal or bestial in nature - rather than the refined and manufactured products required for smithing, for example. A \imp{potion master's} toolkit contains leaves, claws, eyeballs and all manner of other organic ingredients - the ingredients for the most powerful potions have to be harvested from rare and dangerous magical beasts - though some mundane plants also have some surprisingly powerful magical effects. 

\section{Mixing a Potion}

Mixing a \imp{potion} is far more than simply throwing ingredients into a bowl and hoping for the best - it is a delicate, magical art that involves imbuing the mages own power into the creation as it infuses - shaped by their understanding of the ingredients and the processes involved. 

Even if a muggle (or, indeed, simply an untrained wizard) were to follow a tried-and-tested recipe to the letter, they would not end up with a batch of \imp{felix felices} potion, but rather some upsetting bogwater creation. In order for the potion to work, the mage must have a level of understanding, an intuitive connection, to the potion. 

Potion mixing inherently requires the use of a set of \imp{tools} - specifically an \imp{Alchemy Set}, which contains the all-important cauldron, as well as tools for preparing and filtering samples, purifying chemicals 
and other such important aspects of potionmaking.

\subsection{Ingredients}

After deciding on the potion you wish to make, you must then select at least three ingredients you wish to include to induce the desired effect. 

Each ingredient found in the list below is described by a number of simple phrases, which indicate the magical properties possessed by the ingredient - you should simply choose a selection of ingredients, and construct a narrative for why the chosen ingredients would produce the desired effect. 

For example, if \imp{Sean} were trying to mix a powerful \imp{Healing potion}, he would note that the combination of \imp{Wiggentree bark}, \imp{Horklump Juice} and \imp{Moly} would be a powerful healing combination. When narrating the potion making, \imp{Sean}'s player would describe something along the lines of:

{\it I use my Wiggentree Bark and the Horklump Juice to produce a basic healing mixture, to which I then add a dash of my extremely rare Moly, thereby magnifying the effects of the potion.}

Equally, you can mix and match additional properties to create a `story' about the potion - the \imp{Weasley twins} were infamous for their stink bombs: an attempt to make a stink bomb might invoke the \imp{aerosol} properties of \imp{peppermint}, the \imp{smelly} and \imp{Disgusting} properties of a \imp{pungent onion}, tied together with the \imp{catalyst} property of \imp{octopus powder} - together these properties clearly bind together to create a foul-smelling gas. 

You may also choose to add in ingredients to counteract any unwanted effects - the polyjuice potion famously uses a lot of ingredients which have the \imp{disgusting} property, resulting in a truly vile-tasting beverage. The discerning potioneer might decide that a dash of \imp{ginger} (with the \imp{tasty} property) might help the potion go down a bit more smoothly. 



\subsubsection{Improvised Ingredients}

The players are perfectly free (in fact, encouraged!) to work with the \imp{GM} to expand the list of ingredients - if they defeat a \imp{hippogriff} in battle, and come away with a handful of feathers, they are welcome to try and work together to come up with some additional words to describe the alchemical effects of their bounty. Equally, new `words' to describe alchemical properties can be invoked at any time by the \imp{GM}, in conjunction with the players. 

A final note is that you may also elect to use ingredients as symbolic entities - a potion designed to transform something into gold would be made out of \imp{metamorphic} ingredients, and then some gold itself - even though this does not match with the `words' used to describe it. Equally, a polyjuice potion uses a part of the target individual - even though this has no intrinsic magical property. 

This system is designed to be largely freeform, allowing players to experiment and innovate with their own mixtures and recipes. - as long as a coherent narrative can be constructed as to why the chosen ingredients come together to give the specified effect, the \imp{GM} should be lenient.


\subsection{Preparing the Project}

The \imp{GM} should now attempt to determine the exact details of the potion you are making (and also perhaps dream up some complications or mitigating factors). They should ask the potion mixer to describe exactly the effects of the potion they are trying to make, the magnitude of the effects, and the number of doses intended. 

The \imp{difficulty} of the mixing is determined using the basic \imp{artificing rules} found on page \pageref{T:ArtificingDV}. In short, the \imp{GM} uses the properties of the potion mixing that is being attempted to divine a \imp{Rarity} for the potion.

As suggested on page \pageref{T:ArtificingDV}, the \imp{Rarity} also corresponds to the number of required successes for the potion to be completed. However, attempting to brew a larger dose may increase the number of required successes. Most \imp{alchemy} attempts would produce only a small number of samples of the desired potion - usually between 1 and 3 (at the \imp{GM}s discretion). This number can be increased by voluntarily increasing the \imp{complexity} of the project by one level - if a \imp{Fiddly} project would produce 2 samples, then you may brew 4 potions by making the mixing \imp{arduous}, or 6 potions by making it \imp{tough}



\def\AbundantList{
	\ingredient{Ash}{Burned and blackened organic matter.} {\imp{Absorbant}, \imp{Dry}, \imp{Insulating} and \imp{Suffocating}}
	\ingredient{Asphodel}{A mundane member of the lily family\comma{} used in sleeping potions} {\imp{Soothing} and \imp{Soporific}}
	\ingredient{Caterpillar}{Pupae form of a butterfly. A variety of species and colours.} {\imp{Colourful} and \imp{Metamorphic}}
	\ingredient{Coffee Beans}{Small brown berries with a high caffeine content. Used by muggles as a restorative.} {\imp{Stimulant}}
	\ingredient{Copper}{A chemical element with many intruiging properties.} {\imp{Conductive}, \imp{Hard} and \imp{Voltaic}}
	\ingredient{Daisy}{A small white and yellow flower familiar to muggles.} {\imp{Aphrodesiac} and \imp{Fragrant}}
	\ingredient{Dittany}{A mundane green leaf with powerful healing properties.} {\imp{Healing}}
	\ingredient{Flobberworm Mucous}{The green\minus{}grey goo extruded by the most useless of creatures.} {\imp{Irritant}, \imp{Smelly} and \imp{Sticky}}
	\ingredient{Ginger}{A pleasant smelling plant and foostuff. Gives life a bit of zing.} {\imp{Stimulant} and \imp{Tasty}}
	\ingredient{Honeywater}{A dilute form of honey. Useful as a potion base.} {\imp{Nutritious}, \imp{Stabilising} and \imp{Sticky}}
	\ingredient{Lavender}{A pleasant smelling purple plant with powerful calming effects.} {\imp{Fragrant}, \imp{Soothing} and \imp{Soporific}}
	\ingredient{Lemon Juice}{Cloudy\comma{} acidic juice – a good addition to many potions.} {\imp{Acidic}, \imp{Nutritious} and \imp{Tasty}}
	\ingredient{Moondew}{Dew gathered at midnight on a new moon. Absorbs all light that hits it.} {\imp{Absorbant} and \imp{Purifying}}
	\ingredient{Morning Dew}{Dew harvested by naked virgins from only the purest oak leaves\comma{} just as the first rays of morning infuse them.} {\imp{Slippery}}
	\ingredient{Nettles}{Stinging plant\comma{} but has restorative properties when brewed.} {\imp{Healing} and \imp{Irritant}}
	\ingredient{Peppermint}{A pleasant smelling and tasting herb\comma{} which produces a cloud of gas when heated with acids.} {\imp{Aerosol}, \imp{Fragrant} and \imp{Tasty}}
	\ingredient{Rose Petals}{Red petals that exude lust.} {\imp{Aphrodesiac} and \imp{Fragrant}}
	\ingredient{Tea Leaf}{A muggle plant that awakens the brain\comma{} and broadens the senses. Good with milk.} {\imp{Prescient} and \imp{Stimulant}}
	\ingredient{Vodka}{A strong mixture of ethanol and water\comma{} usually distilled from grain or potatoes.} {\imp{Confusing} and \imp{Resilient}}
}
\def\CommonList{
	\ingredient{Aconite}{The brilliant blue flower of a common\comma{} non\minus{}magical (but poisonous) plant.} {\imp{Floating} and \imp{Poisonous}}
	\ingredient{Asp Tail}{The tail of a poisonois European snake\comma{} used in potion making for thousands of years.} {\imp{Paralytic} and \imp{Poisonous}}
	\ingredient{Billywig Sting}{The venom inside this vicious barb causes giddiness and levitation.} {\imp{Floating} and \imp{Hallucinogenic}}
	\ingredient{Boomberry}{A small brown nut that explodes when disturbed.} {\imp{Explosive} and \imp{Fragrant}}
	\ingredient{Bubotuber Juice}{White  sap from the magic tree causes boils on contact.} {\imp{Acidic}, \imp{Irritant} and \imp{Paralytic}}
	\ingredient{Bulbadox Powder}{Volatile orange powder capable of causing boils and itching} {\imp{Disgusting} and \imp{Irritant}}
	\ingredient{Bundium Fluid}{A powerfully acidic\comma{} foul smelling grey secretion.} {\imp{Acidic}, \imp{Smelly} and \imp{Sticky}}
	\ingredient{Chizpurfle Fang}{The fang of the magic\minus{}absorbing insects is a powerful mental restorative.} {\imp{Stimulant} and \imp{Tasty}}
	\ingredient{Doxy Eggs}{The bright blue eggs of the trickster\minus{}fairies are mildly poisonous.} {\imp{Confusing} and \imp{Poisonous}}
	\ingredient{Eye of Newt}{A classic potion ingredient\comma{} these black orbs are often used to stabilise volatile potions.} {\imp{Disgusting}, \imp{Purifying} and \imp{Stabilising}}
	\ingredient{Fairy Wings}{Fairies regrow their iridescent wings regularly\comma{} though fresh\minus{}plucked wings are the most potent.} {\imp{Floating}, \imp{Hallucinogenic} and \imp{Lucky}}
	\ingredient{Fluxweed}{A magical plant known for its healing and transformative properties.} {\imp{Flexible}, \imp{Healing} and \imp{Metamorphic}}
	\ingredient{Hemlock Essence}{A well known poison\comma{} known for its purple hue.} {\imp{Poisonous}}
	\ingredient{Horklump Juice}{The deep red juice of the horklump is a healing agent.} {\imp{Healing} and \imp{Tasty}}
	\ingredient{Iron}{A plentiful\comma{} hard metal. Used as a base in alchemy.} {\imp{Conductive} and \imp{Hard}}
	\ingredient{Lacewing Flies}{A species of small green insects\comma{} known for their transparent wings.} {\imp{Invisible} and \imp{Metamorphic}}
	\ingredient{Leeches}{Animals that feed off blood. Powerful healing properties\comma{} but gross.} {\imp{Disgusting}, \imp{Healing} and \imp{Sapping}}
	\ingredient{Lobalug Venom}{This white fluid is a mild poison\comma{} often used to amplify other ingredients.} {\imp{Amplifying} and \imp{Poisonous}}
	\ingredient{Lovage}{A mundane plant with nausea inducing qualities.} {\imp{Disgusting}}
	\ingredient{Magnesium}{This lustrous metal is so reactive it must be stored in oil to prevent it reacting with air.} {\imp{Amplifying}, \imp{Conductive} and \imp{Explosive}}
	\ingredient{Mallowsweet}{The yellow berries of this plant have many beneficial properties.} {\imp{Flexible} and \imp{Tasty}}
	\ingredient{Murtlap Tentacles}{The pink tentacles have a soothing effect on the skin.} {\imp{Flexible}, \imp{Healing}, \imp{Slippery} and \imp{Soothing}}
	\ingredient{Nightshade}{A poisonous purple flower\comma{} used as a cosmetic by muggles throughout history.} {\imp{Aphrodesiac} and \imp{Poisonous}}
	\ingredient{Owl Feather}{Proximity to wizards mean that an owls feathers pick up many properties.} {\imp{Floating}}
	\ingredient{Pungent Onion}{A bright green onion with a powerfully repulsive odour.} {\imp{Aerosol}, \imp{Disgusting} and \imp{Smelly}}
	\ingredient{Slug Slime}{Horned slugs produce an acidic green\minus{}grey fluid that slow their targets down.} {\imp{Acidic} and \imp{Sticky}}
	\ingredient{Stinksap}{A foul smelling green sap that permeates all surfaces it touches.} {\imp{Penetrating}, \imp{Smelly} and \imp{Sticky}}
	\ingredient{Tormentil Tincture}{A bright yellow fluid extracted from a plant known for its soothing properties.} {\imp{Colourful} and \imp{Soothing}}
	\ingredient{Wartcap Powder}{A sickly yellow powder that causes boils and rashes to break out.} {\imp{Irritant} and \imp{Resilient}}
	\ingredient{Wiggentree Bark}{A thick lump of bark from a magical tree. Powerful restorative properties.} {\imp{Hard}, \imp{Healing} and \imp{Insulating}}
}
\def\ExtraordinaryList{
	\ingredient{Acheron Water}{Water from one of the rare magic rivers\comma{} the Acheron is the river of pain. Drinking this water is not advised.} {\imp{Painful} and \imp{Paralytic}}
	\ingredient{Acromantula Venom}{Thick\comma{} black venom of the giant spiders. Very rare and potent.} {\imp{Paralytic}, \imp{Poisonous} and \imp{Preserving}}
	\ingredient{Bicorn Horn}{The golden horn of a legendary beast\comma{} with many properties.} {\imp{Fortifying}, \imp{Metamorphic}, \imp{Penetrating} and \imp{Stimulant}}
	\ingredient{Cocytus Water}{Water from one of the rare magic rivers\comma{} the Cocytus is the river of wailing.} {\imp{Freezing}, \imp{Hallucinogenic} and \imp{Melancholic}}
	\ingredient{Dementor Cloak}{A cutting from the cloak of a dementor. Oozes cold\comma{} and saps your will.} {\imp{Cursed}, \imp{Freezing} and \imp{Melancholic}}
	\ingredient{Erumpet Horn}{A grey\comma{} twisted horn that has a nasty habit of exploding.} {\imp{Amplifying}, \imp{Explosive} and \imp{Penetrating}}
	\ingredient{Lethe Water}{Water from one of the rare magic rivers\comma{} the Lethe is the river of forgetfulness\comma{} and so the water is a powerful amnesiac.} {\imp{Amnesic}, \imp{Confusing} and \imp{Stabilising}}
	\ingredient{Occamy Egg}{Seemingly made of solid silver\comma{} yet constantly growing in size.} {\imp{Aphrodesiac}, \imp{Colourful}, \imp{Lucky} and \imp{Metamorphic}}
	\ingredient{Phlegiston Water}{Water from one of the rare magic rivers\comma{} the Phlegiston is the river of fire\comma{} and the water drawn from this river is unusually reactive and explosive.} {\imp{Amplifying}, \imp{Burning} and \imp{Explosive}}
	\ingredient{Quintaped Leg}{A brown\comma{} hairy leg from a magic abomination. Filled with hatred and power.} {\imp{Amplifying}, \imp{Enraging} and \imp{Ugly}}
	\ingredient{Re\apos{}em Blood}{A vibrant yellow fluid that imbues the drinker with immense strength.} {\imp{Amplifying} and \imp{Fortifying}}
	\ingredient{Sphinx Saliva}{Used to keep the sphynx cool in the hot deserts\comma{} this fluid is also incredibly acidic.} {\imp{Acidic} and \imp{Freezing}}
	\ingredient{Styx Water}{Water from one of the rare magic rivers\comma{} the Styx is the river of hatred\comma{} but is also rumoured to provide near\minus{}invulnerability.} {\imp{Aquatic}, \imp{Enraging} and \imp{Resilient}}
	\ingredient{Unicorn Hair}{A pure\minus{}white hair with many beneficial properties\comma{} if taken politely.} {\imp{Amplifying}, \imp{Aphrodesiac} and \imp{Aphrodesiac}}
}
\def\MythicalList{
	\ingredient{Angel’s Feather}{A feather of the purest white\comma{} glowing with incandescent magical power. Some say that they come from actual\comma{} real angels – whilst others are more doubtful about their origins.} {\imp{Lucky}, \imp{Magical}, \imp{Purifying} and \imp{Radiant}}
	\ingredient{Basilisk Venom}{Potent purple venom from the fangs of a monstrous snake.} {\imp{Acidic}, \imp{Paralytic} and \imp{Poisonous}}
	\ingredient{Manticore Skin}{The manticore{\apos}s magic resistance resides within its tanned skin.} {\imp{Absorbant}, \imp{Insulating} and \imp{Resilient}}
	\ingredient{Moly}{A golden\comma{} glowing plant that helps to heal the wounded and break curses. It can only be picked by an immortal at the exact moment of dawn\comma{} else it shrivels and dies.} {\imp{Aphrodesiac}, \imp{Fragrant}, \imp{Healing} and \imp{Purifying}}
	\ingredient{Nundu Venom Sac}{A black lump of flesh responsible for producing the poisonous aura of the nundu.} {\imp{Aerosol}, \imp{Amplifying}, \imp{Poisonous} and \imp{Suffocating}}
	\ingredient{Pheonix Feather}{A scarlet feather with many wonderful magical properties.} {\imp{Burning}, \imp{Healing}, \imp{Inspiring}, \imp{Purifying} and \imp{Radiant}}
	\ingredient{Thunderbird Feather}{A pale\comma{} golden feather which seems to crackle with energy. Merely touching it causes your hair to stand on end.} {\imp{Accelerant}, \imp{Conductive}, \imp{Floating} and \imp{Voltaic}}
	\ingredient{Unicorn Blood}{Visibly similar to mercury\comma{} the blood of a unicorn carries a powerful curse.} {\imp{Cursed}, \imp{Healing}, \imp{Sapping} and \imp{Unlucky}}
}
\def\RareList{
	\ingredient{Ashwinder Eggs}{A clutch of the eggs of a fire\minus{}snake. They are red\minus{}hot\comma{} and are renowned in love potions.} {\imp{Aphrodesiac} and \imp{Burning}}
	\ingredient{Cyclops Eye}{The single eye torn from the thunder giant tribe\comma{} very rare and very dangerous} {\imp{Enraging}, \imp{Prescient} and \imp{Voltaic}}
	\ingredient{Demiguise Hair}{An invisible strand of hair\comma{} with many beneficial properties.} {\imp{Incorporeal} and \imp{Invisible}}
	\ingredient{Dragon Blood}{Dumbledore is said to have discovered 12 uses for this scarlet substance.} {\imp{Acidic}, \imp{Amplifying}, \imp{Burning}, \imp{Magical} and \imp{Radiant}}
	\ingredient{Dragon Claw}{The powdered claw of a dragon is said to provide a potent brain\minus{}boost.} {\imp{Hard}, \imp{Sapping} and \imp{Stimulant}}
	\ingredient{Dragon Fire Gland}{The red\minus{}hot glands that sit inside the mouth of a dragon\comma{} responsible for their fire\minus{}breathing.} {\imp{Aerosol} and \imp{Burning}}
	\ingredient{Dragon Liver}{The liver of a dragon takes on the qualities of the food that the dragon eats.} {\imp{Disgusting} and \imp{Healing}}
	\ingredient{Dragon Scale}{A hardened scale from the hide of a dragon \minus{} the colour varies depending on the species it was harvested from.} {\imp{Colourful}, \imp{Hard}, \imp{Insulating} and \imp{Resilient}}
	\ingredient{Fire Crab Shell}{A jewel\minus{}encrusted ruby shell that occaisionally emits a gout of flame.} {\imp{Burning}, \imp{Colourful}, \imp{Insulating}, \imp{Radiant} and \imp{Resilient}}
	\ingredient{Gold}{A rare and lustrous metal. The goal of alchemists throughout history.} {\imp{Aphrodesiac}, \imp{Conductive} and \imp{Magical}}
	\ingredient{Griffin Claw}{A magic raptor\minus{}like claw. Said to confer its great intelligence to the owner.} {\imp{Fortifying}, \imp{Prescient} and \imp{Stimulant}}
	\ingredient{Kelpie Hair}{The grey hair of the shapeshifter retains some of this magic.} {\imp{Aquatic} and \imp{Metamorphic}}
	\ingredient{Mackled Malaclaw Tail}{A powerful iridescent blue ingredient\comma{} useful but unstable.} {\imp{Amplifying}, \imp{Explosive} and \imp{Unlucky}}
	\ingredient{Runespoor Egg}{Deep blue eggs with an orange aura\comma{} they are said to focus the mind} {\imp{Magical} and \imp{Stimulant}}
	\ingredient{Sea\minus{}Serpent Spine}{Shed from the fins of aquatic beasts\comma{} these spines are used by poisoners worldwide\comma{} and are renowned for their ability to pierce even the toughest materials.} {\imp{Aquatic}, \imp{Penetrating} and \imp{Poisonous}}
}
\def\SingularList{
	\ingredient{Abyssinian Shrivelfig}{A purple fruit found in the African desert. Dries up and shrinks when picked.} {\imp{Dry} and \imp{Metamorphic}}
	\ingredient{Boomslang Skin}{The brown\comma{} sloughed of skin of a nonmagical snake.} {\imp{Disgusting} and \imp{Metamorphic}}
	\ingredient{Doxy Venom}{This clear fluid deeply affects the brain of the victim.} {\imp{Confusing} and \imp{Poisonous}}
	\ingredient{Dugbog Bark}{Very dense wood\minus{}like material from the back of a dugbog.} {\imp{Hard} and \imp{Resilient}}
	\ingredient{Fire Seed}{A seed that burns with a hot flame whilst growing. Takes hours to cool once picked.} {\imp{Aphrodesiac}, \imp{Burning} and \imp{Purifying}}
	\ingredient{Gillyweed}{A magical plant with the ability to confer the consumer with gills.} {\imp{Aquatic}, \imp{Disgusting} and \imp{Nutritious}}
	\ingredient{Glumbumble Treacle}{A melancholy inducing substance that looks like pink honey.} {\imp{Melancholic}, \imp{Sapping} and \imp{Sticky}}
	\ingredient{Grindylow Claw}{A grey talon used by the creature to suffocate its victims.} {\imp{Paralytic}, \imp{Sapping} and \imp{Suffocating}}
	\ingredient{Jarvey Fang}{A curved fang containing a venom that causes involuntary babbling.} {\imp{Aphrodesiac} and \imp{Babbling}}
	\ingredient{Knotgrass}{The result of magical experimentation on a muggle plant \minus{} the result is an unusually resilient weed which can grow almost anywhere.} {\imp{Metamorphic} and \imp{Resilient}}
	\ingredient{Mandrake Root}{Trimmings from a sentient plant that act as a powerful antidote.} {\imp{Babbling}, \imp{Nutritious} and \imp{Purifying}}
	\ingredient{Mercury}{A liquid silver metal that is constantly changing shape and form.} {\imp{Metamorphic}, \imp{Poisonous} and \imp{Slippery}}
	\ingredient{Moke Skin}{A green scaled pouch that shrinks at the sign of approaching danger.} {\imp{Flexible}, \imp{Resilient} and \imp{Resilient}}
	\ingredient{Moonstone}{A gemstone of unknown provenance. Glows with an inner light.} {\imp{Conductive} and \imp{Radiant}}
	\ingredient{Octopus Powder}{A disgusting orange powder\comma{} but a powerful catalyst.} {\imp{Amplifying} and \imp{Disgusting}}
	\ingredient{Salamander Blood}{Bright red fluid that emits huge amounts of heat. A powerful catalyst.} {\imp{Amplifying}, \imp{Burning} and \imp{Radiant}}
	\ingredient{Scarab Beetles}{Once considered sacred by the ancient egyptians\comma{} these contain a surprising amount of magical power for a mundane beetle.} {\imp{Healing} and \imp{Resilient}}
	\ingredient{Sloth Brain}{The diced brain of a sloth is said to contain the essence of the being.} {\imp{Confusing}, \imp{Sapping} and \imp{Soporific}}
	\ingredient{Squill Bulb}{The root of a non\minus{}magical plant found at high altitudes\comma{} often used to make potions palatable.} {\imp{Lucky}, \imp{Soothing} and \imp{Tasty}}
}
\def\UnusualList{
	\ingredient{Alihotsy Leaves}{Consuming the speckled leaves of the `hyena tree\apos{} results in uncontrollable laughter} {\imp{Hallucinogenic}}
	\ingredient{Bezoar}{A hard\comma{} brown lump formed in the stomach of a goat. Horrifying to look at\comma{} but said to be a powerful antidote.} {\imp{Purifying} and \imp{Ugly}}
	\ingredient{Centaur Hoof}{Shavings from the hoof is said to contain the wisdom of the mystical people.} {\imp{Prescient} and \imp{Stimulant}}
	\ingredient{Frost Salamander Blood}{The ice\minus{}cold blood of the frost salamander\comma{} a pleasant sky\minus{}blue colour.} {\imp{Freezing} and \imp{Stabilising}}
	\ingredient{Hippocampus Hair}{This multicoloured hair is said to help the memory.} {\imp{Babbling}, \imp{Colourful} and \imp{Stimulant}}
	\ingredient{Kneazle Claw}{When powdered\comma{} increases the consumer{\apos}s perception enormously.} {\imp{Hallucinogenic}, \imp{Prescient} and \imp{Stimulant}}
	\ingredient{Mooncalf Tears}{Glowing fluid that seems to calm you down just by looking at it.} {\imp{Fragrant}, \imp{Inspiring}, \imp{Soothing} and \imp{Soporific}}
	\ingredient{Nogtail Trotter}{The foot of the nogtail makes one as fleet as the beast itself.} {\imp{Accelerant} and \imp{Hallucinogenic}}
	\ingredient{Pearl Dust}{A lustrous powder that gleams with positive energy.} {\imp{Aphrodesiac} and \imp{Radiant}}
	\ingredient{Pogrebin Shell}{A lump of hardened flesh that resembles stone. Exudes an ominous aura.} {\imp{Hard} and \imp{Melancholic}}
	\ingredient{Raiju Shaving}{A clump of fur torn from the lightning\minus{}fast thunder dog\comma{} crackles with energy.} {\imp{Accelerant} and \imp{Voltaic}}
	\ingredient{Silver}{A rare and lustrous metal\comma{} second only to gold in its value. Feared by the undead.} {\imp{Conductive} and \imp{Purifying}}
	\ingredient{Troll Snot}{A thick grey goo that dulls the senses\comma{} but bolsters the muscles.} {\imp{Disgusting}, \imp{Fortifying} and \imp{Sticky}}
	\ingredient{Venemous Tentacula}{A green goo formed from the mashed plant. Highly toxic.} {\imp{Poisonous}}
}


\def\effectList{
	\effect{Absorbant}{Causes things to be swallowed up\comma{} or incorporated into the target} {\imp{Ash}, \imp{Manticore Skin} and \imp{Moondew}}
	\effect{Accelerant}{Speeds up the metabolic rate\comma{} and even the passage of time} {\imp{Nogtail Trotter}, \imp{Raiju Shaving} and \imp{Thunderbird Feather}}
	\effect{Acidic}{Corrosive\comma{} inflicts acid damage and degrades physical objects.} {\imp{Bundium Fluid}, \imp{Dragon Blood} and \imp{Lemon Juice}}
	\effect{Aerosol}{Produces a cloud of gas or air\comma{} ideal for producing long\minus{}lasting clouds.} {\imp{Pungent Onion}, \imp{Peppermint} and \imp{Nundu Venom Sac}}
	\effect{Amnesic}{Induces amnesia\comma{} causes a target to forget things} {\imp{Lethe Water}}
	\effect{Amplifying}{Increase the strength of the things it is applied to\comma{} either physically\comma{} magically or alchemically} {\imp{Re\apos{}em Blood}, \imp{Phlegiston Water} and \imp{Erumpet Horn}}
	\effect{Aphrodesiac}{Makes things appear beautiful\comma{} induces lust and desire} {\imp{Daisy}, \imp{Gold} and \imp{Jarvey Fang}}
	\effect{Aquatic}{Imbued with the essence of water\comma{} gives the consumer aspects of underwater creatures.} {\imp{Styx Water}, \imp{Kelpie Hair} and \imp{Sea\minus{}Serpent Spine}}
	\effect{Babbling}{Induces excessive talkativeness\comma{} usually with a sense of mania} {\imp{Hippocampus Hair}, \imp{Jarvey Fang} and \imp{Mandrake Root}}
	\effect{Burning}{Creates heat and fire\comma{} can ward off cold\comma{} protect from fire or outright burn things} {\imp{Dragon Blood}, \imp{Dragon Fire Gland} and \imp{Salamander Blood}}
	\effect{Colourful}{Brightly coloured\comma{} induces specific colour changes in the potion or its consumer} {\imp{Caterpillar}, \imp{Tormentil Tincture} and \imp{Fire Crab Shell}}
	\effect{Conductive}{Allows electricty and heat to pass through easily} {\imp{Thunderbird Feather}, \imp{Silver} and \imp{Magnesium}}
	\effect{Confusing}{Induces confusion in the minds of living beings} {\imp{Vodka}, \imp{Doxy Eggs} and \imp{Sloth Brain}}
	\effect{Cursed}{Contains powerful dark magic that bring about foul curses} {\imp{Dementor Cloak} and \imp{Unicorn Blood}}
	\effect{Disgusting}{Vile\comma{} likely to induce nausea and vomiting} {\imp{Eye of Newt}, \imp{Bulbadox Powder} and \imp{Dragon Liver}}
	\effect{Dry}{Rapidly absorbs liquids\comma{} helps mop up spillages} {\imp{Abyssinian Shrivelfig} and \imp{Ash}}
	\effect{Enraging}{Induces anger and rage\comma{} turning the afflicted into a mindless beast.} {\imp{Cyclops Eye}, \imp{Quintaped Leg} and \imp{Styx Water}}
	\effect{Explosive}{Powerfully unstable\comma{} liable to explode or detonate} {\imp{Phlegiston Water}, \imp{Boomberry} and \imp{Magnesium}}
	\effect{Flexible}{Unbrittle and able to deform and contort\comma{} but always able to return to original shape} {\imp{Mallowsweet}, \imp{Fluxweed} and \imp{Murtlap Tentacles}}
	\effect{Floating}{Flies or moves about in space of its own accord} {\imp{Fairy Wings}, \imp{Billywig Sting} and \imp{Aconite}}
	\effect{Fortifying}{Increases the positive attributes of things – increases strength or intelligence (when applied correctly).} {\imp{Troll Snot}, \imp{Bicorn Horn} and \imp{Griffin Claw}}
	\effect{Fragrant}{Has a pleasant aroma} {\imp{Peppermint}, \imp{Rose Petals} and \imp{Moly}}
	\effect{Freezing}{Cold to the touch: can ward off excess heat\comma{} protect from cold or freeze things solid} {\imp{Cocytus Water}, \imp{Sphinx Saliva} and \imp{Dementor Cloak}}
	\effect{Hallucinogenic}{Induces euphoria and brings about visions and false memories.} {\imp{Billywig Sting}, \imp{Nogtail Trotter} and \imp{Cocytus Water}}
	\effect{Hard}{Hard\comma{} unbending and immune to change – ideal for protective ointments} {\imp{Copper}, \imp{Dugbog Bark} and \imp{Dragon Scale}}
	\effect{Healing}{Induces healing to living beings} {\imp{Fluxweed}, \imp{Dragon Liver} and \imp{Scarab Beetles}}
	\effect{Incorporeal}{Not quite there\comma{} allows one to step briefly into the astral plane} {\imp{Demiguise Hair}}
	\effect{Inspiring}{Induces hope and bravery into things\comma{} pushing them to achieve their best and defy the odds.} {\imp{Mooncalf Tears} and \imp{Pheonix Feather}}
	\effect{Insulating}{Prevents heat and electricity from passing through easily} {\imp{Fire Crab Shell}, \imp{Wiggentree Bark} and \imp{Manticore Skin}}
	\effect{Invisible}{Causes things to become invisible or hard to see} {\imp{Demiguise Hair} and \imp{Lacewing Flies}}
	\effect{Irritant}{Causes itching\comma{} swelling and boils to erupt on the skin} {\imp{Bulbadox Powder}, \imp{Wartcap Powder} and \imp{Nettles}}
	\effect{Lucky}{Said to bring good luck to those who bear it} {\imp{Occamy Egg}, \imp{Angel’s Feather} and \imp{Squill Bulb}}
	\effect{Magical}{Acts as a conduit or a source of raw magical power} {\imp{Angel’s Feather}, \imp{Gold} and \imp{Runespoor Egg}}
	\effect{Melancholic}{Induces melancholy\comma{} despair and in extreme doses\comma{} terror} {\imp{Cocytus Water}, \imp{Pogrebin Shell} and \imp{Glumbumble Treacle}}
	\effect{Metamorphic}{Causes objects to alter their shape and change form} {\imp{Lacewing Flies}, \imp{Kelpie Hair} and \imp{Knotgrass}}
	\effect{Nutritious}{Supplies vitamins\comma{} minerals and energy for living beings and plants to thrive upon} {\imp{Honeywater}, \imp{Mandrake Root} and \imp{Lemon Juice}}
	\effect{Painful}{Creates incredible mind\minus{}numbing pain if consumed} {\imp{Acheron Water}}
	\effect{Paralytic}{Causes muscles to lock up and induce paralysis} {\imp{Grindylow Claw}, \imp{Bubotuber Juice} and \imp{Asp Tail}}
	\effect{Penetrating}{Seeps through defensive\comma{} or bursts through armour} {\imp{Stinksap}, \imp{Erumpet Horn} and \imp{Sea\minus{}Serpent Spine}}
	\effect{Poisonous}{Toxic to living beings\comma{} inflicts poison damage} {\imp{Sea\minus{}Serpent Spine}, \imp{Basilisk Venom} and \imp{Acromantula Venom}}
	\effect{Prescient}{Stimulates psychic abilities\comma{} allowing glimpses into the future\comma{} or into the minds of others} {\imp{Kneazle Claw}, \imp{Cyclops Eye} and \imp{Centaur Hoof}}
	\effect{Preserving}{Prevents rot and decay\comma{} holds things as they are} {\imp{Acromantula Venom}}
	\effect{Purifying}{Drives away disease\comma{} poisons and corruption} {\imp{Fire Seed}, \imp{Eye of Newt} and \imp{Moly}}
	\effect{Radiant}{Emits a glow} {\imp{Fire Crab Shell}, \imp{Salamander Blood} and \imp{Pearl Dust}}
	\effect{Resilient}{Unusually immune or resistive to external influence} {\imp{Manticore Skin}, \imp{Dugbog Bark} and \imp{Knotgrass}}
	\effect{Sapping}{Drains the positive attributes from things} {\imp{Dragon Claw}, \imp{Unicorn Blood} and \imp{Leeches}}
	\effect{Slippery}{Causes things to slide\comma{} induces movement} {\imp{Mercury}, \imp{Morning Dew} and \imp{Murtlap Tentacles}}
	\effect{Smelly}{Has an unpleasant aroma} {\imp{Stinksap}, \imp{Pungent Onion} and \imp{Flobberworm Mucous}}
	\effect{Soothing}{Calms the target of mental stress\comma{} prevents itchiness} {\imp{Tormentil Tincture}, \imp{Mooncalf Tears} and \imp{Asphodel}}
	\effect{Soporific}{Sends things off to sleep} {\imp{Lavender}, \imp{Asphodel} and \imp{Mooncalf Tears}}
	\effect{Stabilising}{Prevents instability\comma{} makes things more predictable and reliable} {\imp{Honeywater}, \imp{Eye of Newt} and \imp{Lethe Water}}
	\effect{Sticky}{Causes things to stick in place\comma{} prevents movement} {\imp{Stinksap}, \imp{Troll Snot} and \imp{Flobberworm Mucous}}
	\effect{Stimulant}{Brings about mental acuity\comma{} prevents sleepiness} {\imp{Centaur Hoof}, \imp{Chizpurfle Fang} and \imp{Bicorn Horn}}
	\effect{Suffocating}{Prevents target from breathing} {\imp{Ash}, \imp{Grindylow Claw} and \imp{Nundu Venom Sac}}
	\effect{Tasty}{Yummy to eat} {\imp{Lemon Juice}, \imp{Mallowsweet} and \imp{Ginger}}
	\effect{Ugly}{Horrible to look at\comma{} induces revulsion and hatred} {\imp{Bezoar} and \imp{Quintaped Leg}}
	\effect{Unlucky}{Said to bring about chronic bad luck} {\imp{Mackled Malaclaw Tail} and \imp{Unicorn Blood}}
	\effect{Voltaic}{Possessing or generating an electrical charge} {\imp{Copper}, \imp{Thunderbird Feather} and \imp{Cyclops Eye}}
}
\newcommand\effect[3]
{
	{\footnotesize \key{#1}}	&	\parbox[t]{8cm}{\scriptsize #2} &	\parbox[t]{6cm}{\scriptsize #3} \\
}
\begin{strip}
\section{Ingredient Properties}
Below is a list of the ingredient properties which are used for the `default' alchemical ingredients included in this guide. Each effect is described (deliberately vaguely) by a single word, which is elaborated upon in this list. 
{\scriptsize 
\begin{center}
\begin{rndtable}{l c c}
	{\normalsize \key{Effect Name}}	&	{\normalsize\key{Effect Description}}	&	{\normalsize \key{Sample Ingredients}}
	\\
	\effectList{}
\end{rndtable}
\end{center}
}
\end{strip}

\section{Ingredient List}

\newcommand\ingredient[3]
{
	\parbox[t]{2 cm}{\raggedright\small \key{#1}} &	\parbox[t]{4cm}{#2}	&	\parbox[t]{2.5 cm}{\raggedright #3} \\
}

\newcommand\ingredientList[2]
{
	\subsection{		#1}

	\scriptsize
\begin{rndtable}{l l l}
{\small \key{Ingredient}}	&	{\small \key{Description} } &	{\small \key{Effects}} \\
#2
\end{rndtable}
}

\ingredientList{Abundant Ingredients}{\AbundantList}
\ingredientList{Common Ingredients}{\CommonList}
\ingredientList{Singular Ingredients}{\SingularList}
\ingredientList{Unusual Ingredients}{\UnusualList}
\ingredientList{Rare Ingredients}{\RareList}
\ingredientList{Extraordinary Ingredients}{\ExtraordinaryList}
\ingredientList{Mythical Ingredients}{\MythicalList}

\section{Selected Recipes}
