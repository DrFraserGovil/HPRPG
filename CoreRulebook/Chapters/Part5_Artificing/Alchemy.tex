\chapter{Potions}

\key{Potioneering}, also known as the arcane art of \key{Alchemy}, is one of the most ancient of the \imp{artificing} practices.

Whilst the sister arts of \key{enchanting} and \key{crafting} focus on the creation and imbuement of physical items, \imp{potioneering} focuses on the mixing, brewing and steeping of magical fluids, balms and tinctures: \imp{potions} which imbue the user with magical effects when they are drunk, spilled, applied to the skin, or any other myriad forms of application. 

Unlike the other \imp{artificing arts}, the ingredients used for \imp{alchemy} are almost entirely natural, herbal or bestial in nature - rather than the refined and manufactured products required for smithing, for example. A \imp{potion master's} toolkit contains leaves, claws, eyeballs and all manner of other organic ingredients - the ingredients for the most powerful potions have to be harvested from rare and dangerous magical beasts - though some mundane plants also have some surprisingly powerful magical effects. 

\section{Mixing a Potion}

Mixing a \imp{potion} is far more than simply throwing ingredients into a bowl and hoping for the best - it is a delicate, magical art that infvolves imbuing the mages own power into the creation as it infuses - shaped by their understanding of the ingredients and the processes involved. 

Even if a muggle (or even an untrained wizard) were to follow a tried-and-tested recipe to the letter, they would not end up with a batch of \imp{felix felices} potion, but rather some upsetting bogwater creation. In order for the potion to work, the mage must have a level of understanding, an intuitive connection, to the ingredients which they are adding. 

After they have gained this level of understanding, they must gain the use of a \imp{Alchemy Set}, a \imp{tool} containing all the necessary equipment and doohickeys required for the delicate alchemical processes - without access to this tool, you are forced to \imp{improvise} the \imp{tool}, which makes mixing the potion much harder and liable to catastrophic failure. 

\subsection{Ingredients}

After deciding on the potion you wish to make, you must then select at least three ingredients you wish to include to induce the desired effect. 

Each ingredient found in the list below is described by a number of simple phrases, which indicate the magical properties possessed by the ingredient - you should simply choose a selection of ingredients, and construct a narrative for why the chosen ingredients would produce the desired effect. 

For example, if \imp{Sean} were trying to mix a powerful \imp{Healing potion}, he would note that the combination of \imp{Wiggentree bark} (described as {\it Healing, Hard} and {\it Insulating}), \imp{Horklump Juice} ({\it Healing, Tasty}), and \imp{Moly} ({\it Purifying, Healing, Fragrant, Beautiful}) would be a powerful healing combination. When narrating the potion making, \imp{Sean}'s player would describe something along the lines of:

{\it I use my Wiggentree Bark and the Horklump Juice to produce a basic healing mixture, to which I then add a dash of my extremely rare Moly, thereby magnifying the effects of the potion.}

Equally, you can mix and match additional properties to create a `story' about the potion - the \imp{Weasley twins} were infamous for their stink bombs: an attempt to make a stink bomb might invoke the \imp{aerosol} properties of \imp{peppermint}, the \imp{smelly} and \imp{Disgusting} properties of a \imp{pungent onion}, tied together with the \imp{catalyst} property of \imp{octopus powder} - together these properties clearly bind together to create a foul-smelling gas. 

You may also choose to add in ingredients to counteract any unwanted effects - the polyjuice potion famously uses a lot of ingredients which have the \imp{disgusting} property, resulting in a truly vile-tasting beverage. The discerning potioneer might decide that a dash of \imp{ginger} (with the \imp{tasty} property) might help the potion go down a bit more smoothly. 



\subsubsection{Improvised Ingredients}

The players are perfectly free (in fact, encouraged!) to work with the \imp{GM} to expand the list of ingredients - if they defeat a \imp{hippogriff} in battle, and come away with a handful of feathers, they are welcome to try and work together to come up with some additional words to describe the alchemical effects of their bounty. 

Equally, new `words' to describe alchemical properties can be invoked at any time by the \imp{GM}, in conjunction with the players. 

A final note is that you may also elect to use ingredients as symbolic entities, rather than simply invoking their magical properties - a potion designed to transform something into gold would be made out of \imp{metamorphic} ingredients, and then some gold itself - even though this does not match with the `words' used to describe it. Equally, a polyjuice potion uses a part of the target individual - even though this has no intrinsic magical property. 

This system is designed to be largely freeform, allowing players to experiment and innovate with their own mixtures and recipes.  


\subsection{Determining the Properties}

The \imp{GM} should now attempt to determine the exact details of the potion you are making (and also perhaps dream up some complications or mitigating factors). 

They should ask the potion mixer to describe exactly the effects of the potion they are trying to make, the magnitude of the effects, and the number of doses intended. All of this information is then used by the \imp{GM} to determine how difficult and how long the \imp{potion mixing} will be. 

The \imp{GM} first selects the difficulty of the mixture, which is determined by the \imp{rarity} of the potion being attempted, as compared to the potion-mixers mixing ability:




\section{Ingredients: Propeties \& Aquisition}
