
\chapter{Combat}

\section{The Combat Cycle}
Unlike most RPGs, which tend to use a turn-based system for combat, this game uses a simultaneous combat system. The reason for this is that whilst the turn-based combat fits in with how we play games (I have my turn, you have yours, etc.), it is not entirely realistic: in a fight, you don't wait patiently for everyone else to complete attacking you before finally returning fire: everybody is completing actions at once. 

After combat is initiated, a series of turn cycles occur. Each turn cycle allows every character in combat one major action, such as: a movement, casting a spell, or using an item. Before the turn is activated, there is a period of time (to be decided by your GM), during which you must decide on what you will do. Players may talk to each other during this time, but do be aware that discussing your tactics in front of the GM may give the game away, you wouldn't start shouting your plan out whilst fighting the enemy now, would you? 

After this time is up, each player writes down their action on a scrap of paper (to prevent last minute changes of heart), and then all players (including the GM) reveal their action simultaneously. 

Since all actions are considered to be simultaneous, the order in which the actions are resolved does not usually matter, recall that spells have a finite travel time, so it is entirely feasible for two players to stun each other simultaneously and it does not matter {\apos}who cast first{\apos}.

It might, of course, still be possible for actions to come into conflict with each other: if two characters attempt to occupy the same space, for example. It is up to the GM's discretion how to deal with edge cases like this - for the example given, it is recommended that this be treated as a `body slam', and both characters should recoil and take some damage. 

There might also be cases where two spells are cast simultaneously where the ordering does actually matter: for example, if you heal someone at the same time that someone casts a damaging spell that would take them below 50\% health, incurring the ``major injury{\apos \apos} status. If the healing action occurs first, then they are not taken below 50\% health, but if the damage action occurs first, then they do fall below 50\%, even if they are then brought back up over that threshold. The final health that the character ends up on might be the same, but the ordering of actions effects whether they have the {\it major injury} status at the end of the turn. 

In cases such as this it is useful to remember that it is the {\it casting} of the spell that is simultaneous: so the ordering in which the spell effects should take place can be inferred from the distance between the caster and the target. The issue above is resolved simply by looking at whoever is closest to the target. 
 
 \newpage
 
 \section{Major Actions}
 
 Major actions take an entire turn to complete, and as such are considered the main way to engage in combat. 
 
 \subsection{Attacking}
 
 Casting a spell, swinging a sword, or loosing an arrow takes (usually) a full turn to complete, and so you may decide to use your entire turn to cast a spell. Some skills and archetype abilities allow you to perform multiple such actions as part of a single major action. 
 
 Generally, to perform an attack, you first perform an accuracy check (see page \pageref{S:Accuracy}) against your targets {\it block} or {\it dodge} values, to determine if the attack hits. You must then perform the damage check, to determine the amount of damage your check does. A spell-attack has the additional step of requiringa `casting check', to determine if you can successfuly cast the spell. 
 
 \subsection{Movement}
 
 When used as a major action, movement allows you to move up to a distance given by: 
\small
$$ \text{metres travelled} = \text{Base Speed } + \frac{\text{ATH modifier + Speed Proficiency }}{2}  $$  
\normalsize
(See below for more)

All other non-standard movement (i.e. climbing, crawling etc, swimming) must be performed as a full major action. 

 \subsection{Using Items (sometimes)}
 
Obviously, some `uses' of items include using swords, wands and ranged weapons, which have already been covered by `attacking'. 

However, sometimes you might want to use an action to get something big done, outside of hitting somebody. Using a crowbar to pry open a door, changing your weapon, finding the right page of a book -- all of these take enough time to be considered major actions. 

Some uses might take multiple turns -- for instance, climbing into a full suit of armour takes more than 3 seconds to complete, and will therefore require multiple, consecutive major actions to complete. 

\subsection{Trading Items}

If two characters are standing within touching distance, they may trade items between them. 

Giving items to other people takes the major actions of both the giver and the receiver. 

\newpage
\section{Minor Actions}
You may perform two minor actions in place of a single major action, all minor movement actions occur first, but otherwise you may choose the order in which the actions are completed. Some important minor actions are listed. 

\subsection{Communicate}

Communicating vital information - such as the location of a hidden enemy or trap - to your comrades takes a minor action. Note that it is assumed that the enemy can hear you, unless you make an effort to not be understood. 

In the example of alerting a comrade to a hidden foe, the hidden enemy will hear your warning and will know it has been discovered, it may adjust its actions accordingly. 

\subsection{Using Items (sometimes)}

Generally speaking, using an item is considered a minor action. Consuming a potion, checking a rememberall, removing an item from your bag and so on would be considered `minor actions'. Anything that can be completed in around 1 second would fall into this category. 

\subsection{{\it Evade} or {\it Brace} }
 
 You may also choose to ready yourself against incoming attacks, by bolstering your ability to either {\it Dodge} or {\it Block}. This gives you a better chance of negating incoming effects.  

See page \pageref{S:Accuracy} for more details on this mechanic. 


\subsection{Quickspells}

A quickspell is a spell that is cast as a minor action. 

Spells require a clarity of focus, so casting whilst moving, or otherwise in a hurry is generally a bad idea, if you want it to actually work. However, if you are very comfortable with spell (or a very powerful spellcaster), then you may be able get away with it. 

When performing a quickspell, you {\it must} take check-disadvantage on the accuracy check, unless you have a skill which explicitly mentions {\it Quickspells}. 

~

\section{Conditional Actions}

The use of the simultaneous combat system raises some interesting opportunities with conditional actions, which are actions that depend on the actions that another character takes.

The actual action, as well as the condition, needs to be declared during the normal turn cycle -- but the action itself is not triggered until all other actions had been triggered. 

For example, it could be that you declare as your action \textit{if the troll attacks player A, then I cast a healing spell on player A}. You could also attempt to prevent the damage from being taken in the first place, by declaring \textit{if the troll attacks player A, then I cast the knockback charm on the troll}. The GM may ask for a check to determine if you are close enough and have fast enough reactions for your spell to interrupt the action, but if you pass this, then you may be able to save your friend. Please see below for more counterspell options.

You are only allowed a single conditional clause in your declaration, and if that conditional does not come to pass, then your character does not do anything: there is no \verb|if-then-else| in this game!

If a seemingly unbreakable condition-chain arises (i.e. player A says he will perform X if player B does Y, but player B says he will only perform Y if player A does X), it is up to the GM to resolve the conditionals -- in such cases the answer is usually \textit{nothing happens}, but there may be examples where the GM feels it is more appropriate that the action-chain is triggered. 


\section{Movement}

Moving is a very common action to take during combat, to avoid the enemy's attacks, or to maneouvre yourself to enable an attack on the enemy. Movement can be considered a major action, a minor action, or indeed, neither. To aid this distinction, movement is broken down into three types: minor movements, transport movements and considered movements. 




\subsection{Minor Movements}

A { minor movement} includes things such as turning to face an enemy, or taking a step out from behind cover. These actions do not constitute the entirety of a turn and you may still take a major action afterwards, \textbf{however, they are considered to happen at the very beginning of a turn cycle, and you cannot return to cover after emerging from it}. If you therefore emerge from cover to attack someone, and a character successfully guessed that this would happen and sent a spell in your direction, you will not be protected until you move back into cover in the next turn cycle. 


\subsection{Transport Movement}

{\bf Transport movements} are those designed to get you from point A to point B as quickly as possible. These actions do take up your entire turn: you can do nothing else except take a transport action. The distance that you can travel in a given trasnport action is calculated from:

\small
$$ \text{metres travelled} = \text{Base Speed } + \frac{\text{ATH modifier + Speed Proficiency }}{2}  $$  
\normalsize

This distance is rounded downwards  to the nearest half-metre, unless you are wearing ``heavy armour'' (i.e. anything more heavy than usual fabrics), in which case it is rounded downwards to the nearest integer. The direction that you are travelling in \textbf{must} be declared before performing this check. You may elect to not use all of the movement that you rolled for -- i.e. if you can move 1.5m in total, you may only use 1m, if you desire. 


\subsection{Considered Movement}

A { considered movement} is one in which your character is attempting to do something else, whilst moving. It is considered a minor action -- or `half' a major action. The check is performed exactly as above, but you then simply divi the distance by two. You may use the other half to perform another minor action, such as an evasion, or to prepare a counterspell. 

~

~

\section{Accuracy \& Instincts}\label{S:Accuracy}

When an attack is launched on a being, it is necessary to determine if the attack lands home, or if the attack goes wide, the target dodges underneath the swinging blade, or catches the attack on their armour. 

\subsection{Accuracy}

The attacker -- be they an archer, a gunman, a swordsman, or a spellcaster -- quantifies their ability to successfully hit their target through an {\it accuracy check}, a standard d20 FIN (Precision) check, plus any additional relevant bonuses or penalties. 

If the check is greater than or equal to the {\it instinct value} used by the target, then the attack lands true, and the associated effects are applied. If the accuracy check fails, then the attack misses, or is successfully blocked by the target. 

If the target is not a living being (or is restriced from moving), then hitting the target is much easier, but not totally guaranteed. The `dodge' DV of a stationary object is normally equal to 5, though modifiers may be added for targets which are particularly small, or (for ranged attacks), particularly far away. 

The additional penalty for hitting small/far away targets is:

$$ P = \frac{\text{distance}}{10 \times \text{size}} ~~~~ \text{(rounded down)}$$

Therefore, hitting a 1m target at a distance of up to 10m has a DV of 5, whilst the same target 30m away has a DV of 8, and hitting a 1cm target at a distance of of 1m has a DV of 15. 

\subsection{Instincts} \label{S:AC}

Beings either block or dodge instinctively, without having to devote conscious thought to their reaction. These two actions are therefore termed {\it instincts}, and occur inbetween turn cycles. Each action has a statistic associated with it, which is used to context the accuracy check of the attacker. The value of these statistics is:

\begin{align*} 
\text{Block} &= 8 + \text{ATH (Strength) modifier} 
\\
\text{Dodge} &= 8 + \text{FIN (Speed) modifier} 
\end{align*}

By default, characters instinctively use whichever of these values is the highest:
$$ \text{IV} = \max \left( \text{Block}, \text{Dodge} \right)$$

If a character successfully dodges, the attack whizzes by their ear and misses completely. If they successfully block the attack, then they catch the spell or weapon on a piece of armour (or, with the appropriate skill, they can {\it parry} the attack). 

Various items may improve either of these statistics. A pair of running shoes, for example, makes it easier to dodge out of the way, whilst a heavy shield makes defending yourself easier. Generally speaking, items will be a compromise: wearing heavy armour will bulk up your Block statistic, but will slow you down, reducing your Dodge value. 

\subsection{{\it Brace} and {\it Evade} }

Of course, not all defense happens instinctively -- you may make a conscious decision to brace yourself against an incoming attack, or prepare to dive out of the way. Such a decision is classified as a minor action. 

You may choose to either {\it brace} or {\it evade} (i.e. you do not have to use the highest-value statistic). 

Whichever you choose, you double your Expertise bonus (if applicable) on the chosen statistic and agressors take check-disadvantage on their accuracy checks against you this turn cycle. 


\subsection{Proficiencies}

Whether you are swinging a blade, or cowering behind a shield: you must know how to use your equipment, in order to be able to apply your Expertise to it. 

You may only add your Expertise bonus to an accuracy check, if you are proficient in the weapon you are using, or to an {\it Instinct} if you are proficient in the armour (or lack thereof) you are wearing. 

\subsection{Unblockable and Unavoidable Effects}

Some effects (usually those generated by certain spells) cannot be avoided or blocked: holding up a shield against an incoming cannonball isn't going to prevent much, and trying to dodge out of the way of a tsunami is rarely effective. 

Spells denote in their description if they can be blocked or dodged. For the (rarer) instances of non-spell effects which fall into one of these categories, the GM decides if it is reasonable to dodge or block the effect. 

If the `dominant' instinct (i.e. the one with the highest value) would be ineffective against a given effect, you may use the non-dominant one. However, if the character chose to, for example, use the evade action, they may not transfer the bonus to `block' if an evasion turns out to be ineffective. 

Note that even `unblockable' effects are stopped by `impenetrable' fields and spells which are `undodgeable' treat the target as stationary, and may still miss under those rules. 


\section{Stealth and Critical Strikes} \label{S:Stealth}

Being noticed by the enemy is generally regarded as a bad thing. It therefore often pays to be sneaky, to stay hidden from the enemy. Stealth is governed by the FIN attribute, via the Stealth proficiency. 

Every character and beast has a baseline level of awareness, even when not actively searching for hidden creatures or traps. This is your {\it passive perception}. It is calculated using an `average' dice roll (for a d20, this is 10), so: 
$$\text{Passive PER} = 10 + \text{bonuses}$$

To remain hidden, your sneak-check must exceed 

Every time you wish to take an action whilst remaining hidden, you will need to perform a FIN (stealth) check against the target, with the target performing a PE check -- if the sneak check exceeds the perception check, then you remain hidden. If it fails, then the target becomes aware of you, and probably initiates combat. 

Equally, some creatures might try to sneak up on you -- but the GM can't very well ask you to perform a perception check, as you would immediately know that something was there! In order to keep the surprise, each checktype has a `passive' value, which is simply equal to the average. Hence, for a d20 check, the passive value is 10 + relevant bonuses. The GM will use this value in private to determine if beings remain hidden or not. 

The same is true for illusion spells which are cast on you without your knowledge -- a passive SPR (endurance) check is used, with the same rules as before. The GM does not need to tell you about this spell, unless you actively perform a perception check to notice something wrong with the world. 

If you willingly choose to perform a perception check, this gets a +2 bonus. In combat, this would count as your major action. 

If you initiate combat whilst undetected (or have it initiated against you by an unseen opponent), then the attacked party must continue to attempt to percieve the enemy, until they can attack them in the usual way. You may attempt to wildly attack the enemy -- throwing a fireball {\it near} them is probably going to hurt, even if you don't know exactly where they are, but this might be a waste of resources. 

\subsection{Critical Strikes}\label{S:Sneak}

If you perform an attack on someone who is not aware that you are attacking them, or if you perform an attack on someone who has their mind elsewhere, then you have an opportunity to do large amounts of damage to the unwary target. 

A sneak attack is triggered when a character attacks another when they are not expecting it -- be it attacking someone who is not even aware that you pose a danger to them, or if you have snuck up behind an enemy whilst they are attacking someone else -- if they don't see an attack coming, you get an opportunity to surprise them!

An attack of opportunity is triggered when somebody is aware that they are in combat, but is doing something that opens them up to attack. For example, if someone was in close-quarters range and they attempt to cast a spell on you, you can quickly stab them with a knife, and there is nothing they could do about it. Equally, if they attempt to cast a spell on someone else, then their attention is not on you. If you had already commited to an attack on them, then it has a chance to be much more effective.

Whichever method is triggered, the effect is the same: you roll any (even-numbered) dice. If the result is an even number, then you multiply the damage by 2. If it is odd, then you just do the normal amount of damage\footnote{This assumes that the {\it catastrophic critical} is not in use -- if it is, use the rules detailed in that skill}. 

Critical attacks (i.e. triggered by a nat20, or otherwise through a skill) are mechanically identical to an Attack of opportunity. 

If you perform a critical {\it during} a Critical attack, then you do get to use two multipliers, but they are {\it added}. For example, a critical-opportunity attack would roll two dice, and use the following table to determine the dice:

\begin{center}
\begin{rndtable}{|c c c|}
\hline
~ & odd & even
\\ \hline 
\cellcolor{\tablecolorhead }odd & 2 & 3
\\ \hline
\cellcolor{\tablecolorhead}even & 3  & 4
\\ \hline
\end{rndtable}
\end{center}
