\documentclass[CoreRulebook.tex]{subfile}

\chapter{Taking Damage}

When you are attacked, or do something stupid, you must pay the price. This is called `taking damage'. There are different ways for you to take damage, and these have different effects associated with them. 

\subsection{Damage Types}



Many effects specify what kind of damage they do (for instance, a sword does 1d8 slashing damage). This helps the players and the GM work out how the damage is done, and also how it is affected by any weaknesses and resistances possessed by the target. 

{\bf Acid} 

A spray of acid burns through 1cm physical armour to reduce the HP of the being within. 

{\bf Bludgeoning}

The blunt-force of a hammer, or the force of falling on the ground is very difficult to protect against, reducing the HP by a large amount, and risking breaking a bone. 


{\bf Celestial}

A searing white light damages the undead, and banishes the unliving. Has no effect on living beings.  


{\bf Cold}

Cold damage seeps at your willpower, subtracting from FP rather than HP, until FP is zero, at which point it starts leaching HP.  Roll a 1d4, if 1 rolled, acquire Frostbite status. 


{\bf Concussive}

A shockwave from an explosion, passes through physical armour to reduce HP. In addition, target rolls a 1d4, if a 1 is rolled acquire the Deaf status for 3 turns. 

{\bf Electric}

A bolt of lightning can arc from one target to the next, damaging the HP of all it touches. Electric damage can pass through objects and beings which are in contact. 

{\bf Fatigue}

Fatigue damage drains FP from the afflicted. 

{\bf Fire}

Fire damage burns the flesh to reduce the HP. Roll 1d6, 1-2 gives a mild burn, 2-4 gets a moderate burn, 5-6, no effect. 

{\bf Force}

A pure magical energy that directly damages HP. 


{\bf Necrotic}

The evil energies of the undead withers your soul as it damages your body -- reducing HP and FP by equal amounts. 

{\bf Piercing}

Daggers, spears and teeth can puncture even the thickest armour to damage HP. Piercing attacks experience only 50\% the usual armour rating. 

{\bf Poison}

Venemous stings and poisoned weapons damage HP. Roll 1d6, 1-2, mild poisoned status for 3 turns, 3-6, no effect.  

{\bf Psychic}

Damage that originates not from the body, but from the mind. Bypasses all protections to damage HP, and 50\% again to FP. 

{\bf Slashing}

Swinging blades and flashing claws damage the HP of unprotected targets. 


\subsection{Falling}

You are considered to be ``falling'' if you have dropped more than 2 metres, or have been propelled (by an explosion or a spell effect) over any distance.  

For every metre that you fall, you take 1d4 bludgeoning damage, and upon landing you end up in the `prone' position on the ground. 

 If the surface upon which you fall has any additional hazards (i.e. spikes, caltrops, fire), the associated damage is applied in addition to the falling damage. 
 
\subsection{Air}

All living beings require air to breathe. The average human being requires approximately 6 litres of air (at 1atm) per minute in order to stay conscious. This scales approximately as $L^3$, so a house elf (at $\sim$1m tall) needs only 1 litre per minute to survive, whilst a giant at 5m will require over 100 litres per minute. 

The amount of time a being can go without oxygen is determined by 1 minute + 1 for every point of the ATH (survival) modifier, with a minimum of 30 seconds. After this time limit is up, the being enters into the Hypoxia status, where their brain begins to shut down, and if it is not cured, then they die. Once in the hypoxia state, it is not sufficient simply to reintroduce the being to a normal environment, you must actively cure the hypoxia with a spell or potion. 

Various beings are immune to these effects to a greater or lesser extent -- the undead do not generally require oxygen to survive, and creatures such as merpeople possess the ability to breath underwater (though they may still suffocate in other ways). 

\subsection{Armour Class}\label{S:AC}

All sapient beings and beasts have an associated ``Armour Class'' (AC), which denotes their natural resistance to damage, as well as any additional supplemental effects, such as wearing armour, or other magical effects. THe formula to calculate AC is given on page \pageref{S:AC}. 

AC acts to protect your HP from being damaged. If you are about to take damage from any of the following damage types: {\bf Acid, Bludgeoning, Electric, Force, Necrotic, Piercing, Poison or Slashing}, then you may subtract away your AC from the damage done (minimum of 0), thus protecting you. Generally speaking, your AC only defends against physical phenomena, and hence does not protect against heat, pyschic and otherworldly damage unless otherwise specified.  Concussive damage produces a shockwave which no physical AC can protect against. 

Some spells also provide a magical pseudo-AC. The spell effects should specify which damage effects they protect against, and any differences from the usual AC effects. 

\subsection{Resisting}

Not all effects of actions are cut and dried -- some effects can be {\bf Resisted}. For example, some spells, even after they have successfully been cast, can be resisted by the target, if they have a strong enough willpower to overpower the caster; spells such as {\it confundus}, and {\it stupefy}, as well as most illusion spells. Alternatively, somebody might try to restrain you, and your character can resist this action if they are strong enough. 

Resist actions, like normal checks, are assigned an attribute (and possibly Proficiencies) that may boost the Resist check. Unless otherwise specified, the Resist check is performed using the dice granted to your character by the {\bf Withstand} skill. 

This Resist check is then compared with the assigned DV (when resisting spells, or a direct action, this is often the CV of that action). If the Resist check is greater than the CV, then the action is either denied, or has a lesser effect. 

Successfully Resisting costs 2 FP. If you have fewer than 2 FP, then you cannot Resist.

You can perform multiple Resists over the course of a Turn Cycle, if multiple combatants attack you with spells that require one, for example. The only limit is when your FP runs out. However, each subsequent resist gets harder and harder: you suffer a 1 point penalty to your check for each Resist you have already performed this cycle. This counter resets at the end of the cycle.


\onecolumn
\section{Statuses}

Statuses are temporary effects applied to a character, usually due to a spell or a magic item, though sometimes mundane objects can confer statuses such as burns. Often these effects come with a built-in time constraint, after which time, the status is removed. It is possible to have more than one status effect at any given time. 

Some statuses come in 3 different levels: mild, moderate and severe; the effects for each are labelled using the notation mild/ moderate / severe: the three levels of burns are indicated to do 1/2/3 damage per turn, meaning that a mild burn does 1 damage per turn, whilst a severe burn does 3. 

A list of statuses is found on the next page.

\def\w{2}
\def\x{6}
\def\y{6}
\def\z{3}

\footnotesize
%% This table is automatically generated from the table in a .xlsx file, compiled into text by a matlab script
%%StatusBegin
\begin{center} \tablealternate\begin{longtable}{|m{\w cm} m{\x cm} m{\y cm} m{\z cm}|}\hline  \tablehead  \normalsize {\bf Status} & \normalsize \bf Description & \normalsize \bf Effect & \normalsize \bf Duration \\  \bf \begin{flushleft} Asleep\end{flushleft}  &  \parbox[t]{\x cm}{\begin{flushleft}Visiting the land of nod.\end{flushleft}}  &   \parbox[t]{\y cm}{\begin{flushleft}Can take no actions, but health and fortitude regenerate at a rate of 1d6 HP for every hour over 3 hours that they are asleep.
Character is unaware of what is going on around them.\end{flushleft}}  &   \parbox[t]{\z cm}{\begin{flushleft}Until waking\end{flushleft}}  \\\bf \begin{flushleft} Blinded\end{flushleft}  &  \parbox[t]{\x cm}{\begin{flushleft}Your eyes are temporarily overloaded by a bright light.\end{flushleft}}  &   \parbox[t]{\y cm}{\begin{flushleft}All checks that would normally require vision fail.\end{flushleft}}  &   \parbox[t]{\z cm}{\begin{flushleft}1 hour\end{flushleft}}  \\\bf \begin{flushleft} Broken Bone\end{flushleft}  &  \parbox[t]{\x cm}{\begin{flushleft}You have suffered an injury that has broken your bone.\end{flushleft}}  &   \parbox[t]{\y cm}{\begin{flushleft}Cannot use the limb in question until it is healed. This is a major injury (see below).\end{flushleft}}  &   \parbox[t]{\z cm}{\begin{flushleft}Until healed\end{flushleft}}  \\\bf \begin{flushleft} Broken Wand\end{flushleft}  &  \parbox[t]{\x cm}{\begin{flushleft}Your wand is broken, and cannot perform properly.\end{flushleft}}  &   \parbox[t]{\y cm}{\begin{flushleft}All spell checks get a -5 penalty, and spell failures are particularly severe.\end{flushleft}}  &   \parbox[t]{\z cm}{\begin{flushleft}Until wand is repaired\end{flushleft}}  \\\bf \begin{flushleft} Burned\end{flushleft}  &  \parbox[t]{\x cm}{\begin{flushleft}Heat has damaged your body, but the effects are ongoing.\end{flushleft}}  &   \parbox[t]{\y cm}{\begin{flushleft}Does 1 / 2 / 3 damage per turn (depending on the severity), unless cold water is applied. 
Even after water is applied, you are 50\% more vulnerable to fire damage.\end{flushleft}}  &   \parbox[t]{\z cm}{\begin{flushleft}10 turns.\end{flushleft}}  \\\bf \begin{flushleft} Calm Mind\end{flushleft}  &  \parbox[t]{\x cm}{\begin{flushleft}Your mind is calm and clear, you are undistracted.\end{flushleft}}  &   \parbox[t]{\y cm}{\begin{flushleft}All checks receive a +1 bonus.\end{flushleft}}  &   \parbox[t]{\z cm}{\begin{flushleft}Lasts for 1 hour, or until hurt.\end{flushleft}}  \\\bf \begin{flushleft} Check Advantage\end{flushleft}  &  \parbox[t]{\x cm}{\begin{flushleft}You have the upper hand\end{flushleft}}  &   \parbox[t]{\y cm}{\begin{flushleft}For each affected check type, you roll the dice twice � and take the highest of the two values\end{flushleft}}  &   \parbox[t]{\z cm}{\begin{flushleft}As specified\end{flushleft}}  \\\bf \begin{flushleft} Check Disadvantage\end{flushleft}  &  \parbox[t]{\x cm}{\begin{flushleft}A negative effect is stopping you performing at your best\end{flushleft}}  &   \parbox[t]{\y cm}{\begin{flushleft}For each affected check type, you roll the dice twice � and take the lowest of the two values\end{flushleft}}  &   \parbox[t]{\z cm}{\begin{flushleft}As specified\end{flushleft}}  \\\bf \begin{flushleft} Confused\end{flushleft}  &  \parbox[t]{\x cm}{\begin{flushleft}A fog descends upon your brain, and you are unable to think clearly,\end{flushleft}}  &   \parbox[t]{\y cm}{\begin{flushleft}After committing to an action, perform a 1d6 check. 5-6, the action is successful. 3-4, the action misses/doesn`t work. 1-2, the action backfires randomly to you or your allies.\end{flushleft}}  &   \parbox[t]{\z cm}{\begin{flushleft}3 turns\end{flushleft}}  \\\bf \begin{flushleft} Critical But Stable\end{flushleft}  &  \parbox[t]{\x cm}{\begin{flushleft}You were close to dying, but your condition is no longer degrading.\end{flushleft}}  &   \parbox[t]{\y cm}{\begin{flushleft}You are totally unable to act\end{flushleft}}  &   \parbox[t]{\z cm}{\begin{flushleft}Until healed above 0HP\end{flushleft}}  \\\bf \begin{flushleft} Critical Condition\end{flushleft}  &  \parbox[t]{\x cm}{\begin{flushleft}You are close to death, bleeding out.\end{flushleft}}  &   \parbox[t]{\y cm}{\begin{flushleft}You are totally unable to act. 1HP lost per turn. When reaching -10HP, you are dead.\end{flushleft}}  &   \parbox[t]{\z cm}{\begin{flushleft}Until stabilised or healed\end{flushleft}}  \\\bf \begin{flushleft} Deaf\end{flushleft}  &  \parbox[t]{\x cm}{\begin{flushleft}Your ears have been damaged, hopefully only temporarily!\end{flushleft}}  &   \parbox[t]{\y cm}{\begin{flushleft}All hearing-based checks fail\end{flushleft}}  &   \parbox[t]{\z cm}{\begin{flushleft}3 turns, or otherwise specified\end{flushleft}}  \\\bf \begin{flushleft} Diseased\end{flushleft}  &  \parbox[t]{\x cm}{\begin{flushleft}You have contracted a disease.\end{flushleft}}  &   \parbox[t]{\y cm}{\begin{flushleft}Specifics of the effects vary according to the disease. Knowledge checks are needed to learn more.\end{flushleft}}  &   \parbox[t]{\z cm}{\begin{flushleft}Never\end{flushleft}}  \\\bf \begin{flushleft} Exhaustion\end{flushleft}  &  \parbox[t]{\x cm}{\begin{flushleft}You have not slept in a long time, your mind and body are weary.\end{flushleft}}  &   \parbox[t]{\y cm}{\begin{flushleft}Every turn, perform a SPR (endurance) check (difficulty 15), if it fails, fortitude costs of actions are doubled, and regeneration of health and fortitude halt.\end{flushleft}}  &   \parbox[t]{\z cm}{\begin{flushleft}Until resting.\end{flushleft}}  \\\bf \begin{flushleft} Frostbite\end{flushleft}  &  \parbox[t]{\x cm}{\begin{flushleft}The cold has damaged your body, and it cannot function properly, but the biggest toll is on your sluggish thoughts.\end{flushleft}}  &   \parbox[t]{\y cm}{\begin{flushleft}Halts fortitude regeneration, and drains 1 / 2 / 3 fortitude per turn, unless warmth is applied.

Even after warmth is applied, you are 50\% more vulnerable to cold damage.\end{flushleft}}  &   \parbox[t]{\z cm}{\begin{flushleft}10 turns.\end{flushleft}}  \\\bf \begin{flushleft} Hypoxia\end{flushleft}  &  \parbox[t]{\x cm}{\begin{flushleft}Oxygen is not reaching your vital organs, you struggle to concentrate, but your brain is slowly shutting down.\end{flushleft}}  &   \parbox[t]{\y cm}{\begin{flushleft}All checks get a -5 penalty. If not cured within 2 minutes, death follows.\end{flushleft}}  &   \parbox[t]{\z cm}{\begin{flushleft}2 minutes\end{flushleft}}  \\\bf \begin{flushleft} Invisible\end{flushleft}  &  \parbox[t]{\x cm}{\begin{flushleft}Light passes straight through you; you are hidden from sight.\end{flushleft}}  &   \parbox[t]{\y cm}{\begin{flushleft}In adverse conditions (i.e. rain and snow), can still be visually detected. Does not stop noise.  Otherwise, visual perception checks to find you fail.\end{flushleft}}  &   \parbox[t]{\z cm}{\begin{flushleft}Various (depends on cause.)\end{flushleft}}  \\\bf \begin{flushleft} Lucky\end{flushleft}  &  \parbox[t]{\x cm}{\begin{flushleft}The result of a Felix Felicius potion, you become extra-ordinarily lucky.\end{flushleft}}  &   \parbox[t]{\y cm}{\begin{flushleft}All checks used by the player get a +5 bonus, and all checks against the player suffer a -3 hit.\end{flushleft}}  &   \parbox[t]{\z cm}{\begin{flushleft}1 hour\end{flushleft}}  \\\bf \begin{flushleft} Major Injury\end{flushleft}  &  \parbox[t]{\x cm}{\begin{flushleft}You have suffered a major injury.\end{flushleft}}  &   \parbox[t]{\y cm}{\begin{flushleft}Cannot heal above 50\% HP until the major injury is fixed.\end{flushleft}}  &   \parbox[t]{\z cm}{\begin{flushleft}Until healed\end{flushleft}}  \\\bf \begin{flushleft} Poisoned\end{flushleft}  &  \parbox[t]{\x cm}{\begin{flushleft}A nefarious chemical, a toxin, has been introduced into your system.\end{flushleft}}  &   \parbox[t]{\y cm}{\begin{flushleft}Does 2 / 3 / 5 damage per turn (unless otherwise directed).\end{flushleft}}  &   \parbox[t]{\z cm}{\begin{flushleft}10 turns.\end{flushleft}}  \\\bf \begin{flushleft} Silenced\end{flushleft}  &  \parbox[t]{\x cm}{\begin{flushleft}You find yourself unable to make any sounds.\end{flushleft}}  &   \parbox[t]{\y cm}{\begin{flushleft}Cannot speak, or cast verbal magic.\end{flushleft}}  &   \parbox[t]{\z cm}{\begin{flushleft}2 turns (unless otherwise directed).\end{flushleft}}  \\\bf \begin{flushleft} Stunned\end{flushleft}  &  \parbox[t]{\x cm}{\begin{flushleft}You have been knocked unconscious\end{flushleft}}  &   \parbox[t]{\y cm}{\begin{flushleft}As if you were asleep, but without the regeneration.\end{flushleft}}  &   \parbox[t]{\z cm}{\begin{flushleft}3 turns\end{flushleft}}  \\\bf \begin{flushleft} Terrified\end{flushleft}  &  \parbox[t]{\x cm}{\begin{flushleft}Your knees knock, your hands shake, and your mind turns inwards: you{\apos}re scared.\end{flushleft}}  &   \parbox[t]{\y cm}{\begin{flushleft}All checks get a -3 penalty. Cannot get closer to the cause of the fear.\end{flushleft}}  &   \parbox[t]{\z cm}{\begin{flushleft}5 turns, or until the cause is removed.\end{flushleft}}  \\\bf \begin{flushleft} Trapped\end{flushleft}  &  \parbox[t]{\x cm}{\begin{flushleft}Ropes, snares or magic are holding you back, preventing you from moving.\end{flushleft}}  &   \parbox[t]{\y cm}{\begin{flushleft}You are fixed in one place, and cannot move. Some traps may also immobilise the arms, in which case you may not perform actions which require your arms.\end{flushleft}}  &   \parbox[t]{\z cm}{\begin{flushleft}3 turns, or until the trap is released.\end{flushleft}}  \\\bf \begin{flushleft} Unlucky\end{flushleft}  &  \parbox[t]{\x cm}{\begin{flushleft}Things are just not going your way�\end{flushleft}}  &   \parbox[t]{\y cm}{\begin{flushleft}All checks get a -2 penalty\end{flushleft}}  &   \parbox[t]{\z cm}{\begin{flushleft}1 week\end{flushleft}}  \\ \hline \end{longtable} \end{center}%%StatusEnd
\normalsize
\twocolumn
