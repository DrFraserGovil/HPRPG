
\documentclass[CoreRulebook.tex]{subfile}

\chapter{Character Progression \& Skills}

\section{Character Progression}
Each character has a `level' associated with them\comma{} which denotes how far your character has progressed\comma{} and how powerful they are.  Levelling your character is key to progressing: it unlocks new skills\comma{} boosts your attributes\comma{} and gives access to new spells. A higher\minus{}level magic user is a stronger magic user. A stronger magic user is less likely to get eaten by a passing beast\comma{} which is generally considered a good thing. 

\subsection{Experience}

Increasing the level of your character (`levelling up') is achieved by accumulating experience. To progress from level 1 to level 2\comma{} you must accumulate 100 experience points (EP). When your character reaches 100EP\comma{} they ascend to level 2\comma{} and the counter is reset. To go from level 2 to level 3 you need to acquire another 200 EP\comma{} and so on and so forth. The EP needed to go from level $x$ to $x+1$ is calculated from:

$$ EP_{x \to x + 1} = 100 x $$

Experience is gained by completing actions and defeating enemies. Experience is awarded for completing difficult actions such as casting a spell\comma{} mixing a potion\comma{} defeating an enemy in combat\comma{} or convincing someone to give you something. The GM will instruct you to roll a dice\comma{} and you will gain that much experience from completing the action.

The dice you roll (and hence the amount of experience you gain) from such an action depends on your proficiency in that skill. For instance\comma{} a first year student gains far more knowledge and experience from casting wingardium leviosa than a seasoned auror does. Hence\comma{} as you progress\comma{} you will learn less experience from trivial actions. 

As a rough guide\comma{} performing an action (such as casting a spell) which is of the same proficiency level as you are will get a 2d20 roll\comma{} using one level below your proficiency is a 2d12\comma{} and so on:

\begin{center}
	\begin{rndtable}{|c c|}
	\hline \bf Relative Proficiency & \bf Experience Roll
	\\ 
	Same level 		& 	2d20
	\\ 
	1 level below 	&	2d12
	\\ 
	2 levels below	& 	2d8
	\\ 
	3 levels below	&	2d6
	\\ 
	4 levels below 	& 	2d4
	\\ 	\hline
\end{rndtable}
\end{center}

For example\comma{} a character with the Adept Battlemage (combat magic) skill would roll a 2d20 for successfully casting the Impediment Jinx (an adept level combat spell)\comma{} whilst if they were an Master Thaumaturge (transfiguration)\comma{} they would only get to roll a 2d8 for casting an Adept transfiguration spell\comma{} as this is 2 levels below Master. 

Experience is only awarded when an action is truly succesful (i.e. a spell has to hit its target\comma{} as well as be succesfully cast). 

\subsection{Levelling Up}
When your experience reaches the requisite amount\comma{} you may choose to rest and muse on what you have learned fromyour experiences\comma{} triggering the level\minus{}up process. You may only do this if not facing life\minus{}threating injury \minus{}\minus{} levelling up cannot heal a broken leg!

When you level up\comma{} you make the following changes to your character:

\begin{itemize}[itemsep=0em]
	\item Increase character level by 1\comma{} and reset EXP counter to zero (you may carry any excess EXP over)
	\item Increase Archetype level by one {\bf OR} choose a new archetype (see multiclassing rules on page \pageref{S:Multiclassing}). Add any new Features you gain at this point.
	\item You may choose one of the following:
	\begin{itemize}[itemsep=0em]
		\item Increase an attribute by 2\comma{} or two attributes by 1
		\item Choose a new Skill\comma{} if you meet the minimum prerequisites
	\end{itemize}
	\item Calculate new HP and FP ceilings
	\item Reset HP and FP to maximum
	\item Reset spell\minus{}learned counter
\end{itemize}

\subsection{Other Changes}

The GM may also decide that\comma{} during the normal course of play\comma{} you have done something that warrants a permanent bonus or penalty \minus{}\minus{} be it something you have learned from extensive practice\comma{} or a gift from some higher being \minus{}\minus{} the GM will grant you a bonus to your Attributes or Proficiencies. This will probably most commonly be used to penalise players for immoral actions \minus{}\minus{} by increasing their EVL level.

\newpage
\section{Skills}\label{S:Skills}

Skills are learned abilities that your character picks up along the way. They can be learned either by levelling up or given as gifts by external devices. Some skills are only temporary and will wear off after a while. Skills are key to learning new and more powerful magic\comma{} as well as ulocking useful abilities. Most skills come in 5 levels: Beginner\comma{} Novice\comma{} Adept\comma{} Expert and Master\comma{} which must be learned in that order. 

The 8 most important skills are the magic\minus{}school skills: skills which are each associated with one of the 7 schools of magic (and Resist checks). They are\comma{}

\begin{center}
\begin{rndtable}{|c c|}
\hline \bf  Magic School  &  \bf Associated Skill
\\ 
Malediction  	& 	Battlemage
\\
Transfiguration		& 	Thaumaturgus
\\
Charms				&	Sorcerer
\\
Recuperation	& 	Defender
\\
Illusion				&	Magician
\\
Divination			& 	Clairvoyant
\\
Dark Arts			& 	Necromancy
\\
Resist Checks		& Withstand
\\ \hline
\end{rndtable}
\end{center}

Learning more powerful spells in each school of magic requires more and more levels in the relevant skill. To cast Expert level Hexes and Curses\comma{} you need to be an Expert Battlemage\comma{} and so on. 

Other skills gives you access to more powerful abilities as well. 

\subsection{Prerequisites}

Some skills list a minimum ability score\comma{} or other threshold that your character must posses before they take that skill. If you do not meet the threshold\comma{} you cannot take the skill\comma{} unless you are provided it by external means\comma{} such as a Class Feature. 

In addition\comma{} for multi\minus{}level skills\comma{} when you take a skill you cannot take another level in that skill until you level up twice. If a character takes a skill at level $x$\comma{} then the prerequisite of the next level is that a character is level $x+2$. This includes skills that are given by Class Features \minus{}\minus{} if you are given a level in a skill by a Class Feature at level 8\comma{} you must wait until level 10 to level up again.

\subsection{Automatic Skill Aquiring} \label{S:Auto}

Some skills are acquired automatically through levelling up\comma{} without you having to choose. Sometimes\comma{} these skills may be givn to your by the GM for narrative reasons\comma{} or in order to further a stagnating game. 

The 8 Magic skills can indeed be levelled up by player choice in the usual fashion in order to get access to those spells slower. However\comma{} you may also gain these skills simply by reaching the appropriate level:

\begin{center}
\begin{rndtable}{|c|c|}
\hline \bf  Skill Level  &  \bf Acquiring Level
\\ 
Beginner 	&    1
\\
Novice		&    5
\\
Adept		&     10
\\
Expert         &     15   
\\ \hline
\end{rndtable}
\end{center}

When an auto\minus{}levelling is incurred\comma{} if you already have that spellcasting level thanks to taking the appropriate Skill manually\comma{} you may instead choose to take a +1 casting bonus to all spells within one discipline associated with each school you had already chosen. This bonus does stack. 

