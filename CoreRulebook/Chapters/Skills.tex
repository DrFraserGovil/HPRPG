\documentclass[CoreRulebook.tex]{subfile}

\chapter{Character Progression \& Skills}

\section{Character Progression}
Each character has a `level' associated with them, which denotes how far your character has progressed, and how powerful they are.  Levelling your character is key to progressing: it unlocks new skills, boosts your attributes, and gives access to new spells. A higher-level magic user is a stronger magic user. A stronger magic user is less likely to get eaten by a passing beast, which is generally considered a good thing. 

\subsection{Experience}

Increasing the level of your character (`levelling up') is achieved by accumulating experience. To progress from level 1 to level 2, you must accumulate 100 experience points (EP). When your character reaches 100EP, they ascend to level 2, and the counter is reset. To go from level 2 to level 3 you need to acquire another 200 EP, and so on and so forth. The EP needed to go from level $x$ to $x+1$ is calculated from:

$$ EP_{x \to x + 1} = 100 x $$

Experience is gained by completing actions and defeating enemies. Experience is awarded for completing difficult actions such as casting a spell, mixing a potion, defeating an enemy in combat, or convincing someone to give you something. The GM will instruct you to roll a dice, and you will gain that much experience from completing the action.

The dice you roll (and hence the amount of experience you gain) from such an action depends on your proficiency in that skill. For instance, a first year student gains far more knowledge and experience from casting wingardium leviosa than a seasoned auror does. Hence, as you progress, you will learn less experience from trivial actions. 

As a rough guide, performing an action (such as casting a spell) which is of the same proficiency level as you are will get a 2d20 roll, using one level below your proficiency is a 2d12, and so on:

\begin{center}
	\begin{rndtable}{|c c|}
	\hline \bf Relative Proficiency & \bf Experience Roll
	\\ 
	Same level 		& 	2d20
	\\ 
	1 level below 	&	2d12
	\\ 
	2 levels below	& 	2d8
	\\ 
	3 levels below	&	2d6
	\\ 
	4 levels below 	& 	2d4
	\\ 	\hline
\end{rndtable}
\end{center}

For example, a character with the Adept Battlemage (combat magic) skill would roll a 2d20 for successfully casting the Impediment Jinx (an adept level combat spell), whilst if they were an Master Thaumaturge (transfiguration), they would only get to roll a 2d8 for casting an Adept transfiguration spell, as this is 2 levels below Master. 

Experience is only awarded when an action is truly succesful (i.e. a spell has to hit its target, as well as be succesfully cast). 

\subsection{Levelling Up}
When your experience reaches the requisite amount, you immediately trigger the levelling up process. When you level up, you make the following changes to your character:

\begin{itemize}[itemsep=0em]
	\item Increase character level by 1, and reset EXP counter to zero (you may carry any excess EXP over)
	\item Increase Archetype level by one {\bf OR} choose a new archetype (see multiclassing rules on page \pageref{S:Multiclassing}). Add any new Features you gain at this point.
	\item You may choose one of the following:
	\begin{itemize}[itemsep=0em]
		\item Increase an attribute by 2, or two attributes by 1
		\item Choose a new Skill, if you meet the minimum prerequisites
	\end{itemize}
	\item Calculate new HP and FP ceilings
	\item Reset HP and FP to maximum
	\item Reset spell-learned counter
\end{itemize}

\subsection{Other Changes}

The GM may also decide that, during the normal course of play, you have done something that warrants a permanent bonus or penalty -- be it something you have learned from extensive practice, or a gift from some higher being -- the GM will grant you a bonus to your Attributes or Proficiencies. This will probably most commonly be used to penalise players for immoral actions -- by increasing their EVL level.

\newpage
\section{Skills}

Skills are learned abilities that your character picks up along the way. They can be learned either by levelling up or given as gifts by external devices. Some skills are only temporary and will wear off after a while. Skills are key to learning new and more powerful magic, as well as ulocking useful abilities. Most skills come in 5 levels: Beginner, Novice, Adept, Expert and Master, which must be learned in that order. 

The 8 most important skills are the magic-school skills: skills which are each associated with one of the 7 schools of magic (and Resist checks). They are,

\begin{center}
\begin{rndtable}{|c c|}
\hline \bf  Magic School  &  \bf Associated Skill
\\ 
Hexes \& Curses  	& 	Battlemage
\\
Transfiguration		& 	Thaumaturgus
\\
Charms				&	Sorcerer
\\
Recuperation	& 	Defender
\\
Illusion				&	Magician
\\
Divination			& 	Clairvoyant
\\
Dark Arts			& 	Necromancy
\\
Countervail		&	Resist Magic
\\ \hline
\end{rndtable}
\end{center}

Learning more powerful spells in each school of magic requires more and more levels in the relevant skill. To cast Expert level Hexes and Curses, you need to be an Expert Battlemage, and so on. 

Other skills gives you access to more powerful abilities as well. 

\subsection{Prerequisites}

Some skills list a minimum ability score, or other threshold that your character must posses before they take that skill. If you do not meet the threshold, you cannot take the skill, unless you are provided it by external means, such as a Class Feature. 

In addition, for multi-level skills, when you take a skill you cannot take another level in that skill until you level up twice. If a character takes a skill at level $x$, then the prerequisite of the next level is that a character is level $x+2$. This includes skills that are given by Class Features -- if you are given a level in a skill by a Class Feature at level 8, you must wait until level 10 to level up again.

\subsection{Automatic Skill Aquiring} \label{S:Auto}

Some skills are acquired automatically through levelling up, without you having to choose. Sometimes, these skills may be givn to your by the GM for narrative reasons, or in order to further a stagnating game. 

The 8 Magic skills can indeed be levelled up by player choice in the usual fashion in order to get access to those spells slower. However, you may also gain these skills simply by reaching the appropriate level:

\begin{center}
\begin{rndtable}{|c|c|}
\hline \bf  Skill Level  &  \bf Acquiring Level
\\ 
Beginner 	&    1
\\
Novice		&    5
\\
Adept		&     10
\\
Expert         &     15   
\\
Master    &    20
\\ \hline
\end{rndtable}
\end{center}

When an auto-levelling is incurred, if you already have that spellcasting level thanks to taking the apropriate Skill manually, you instead get a +1 increase to your arcane wisdom for each Skill you took. This only applies to manually-chosen skills, it does not apply to increases due to the Spellcasting Improvement archetype feature. 


\onecolumn
\section{Skill List}
\def\w{2.6}
\def\x{1}
\def\y{11.5}
\def\z{2}
\footnotesize

%%SkillBegin
\begin{center} \tablealternate \begin{longtable}{|m{\w cm}  m{\y cm} m{\x cm} m{\z cm}|}\hline \tablehead \normalsize \bf Name  & \normalsize \bf Effect & \bf \normalsize Levels & \normalsize \bf Prerequisite \\ \hline \hline \bf \begin{flushleft} Ambidextrous\end{flushleft}  &   \parbox[t]{\y cm}{\begin{flushleft}No penalty for using items in your non-dominant hand.\end{flushleft}}  &   \parbox[t]{\x cm}{\begin{center}1\end{center}}  &   \parbox[t]{\z cm}{\begin{center} \it FIN > 10\end{center}}  \\  \bf \begin{flushleft} Animagus\end{flushleft}  &   \parbox[t]{\y cm}{\begin{flushleft}Transform into an non-magical animal at will. This animal must be chosen at the moment you acquire this skill, and cannot be changed afterwards. Transforming costs 5FP and constitutes a major action.\end{flushleft}}  &   \parbox[t]{\x cm}{\begin{center}1\end{center}}  &   \parbox[t]{\z cm}{\begin{center} \it Expert Thaumaturgy\end{center}}  \\  \bf \begin{flushleft} Apparate\end{flushleft}  &   \parbox[t]{\y cm}{\begin{flushleft}Perform a SPR (willpower) check.
Apparition difficulty is 18 / 17 / 15 / 15 / 12 / 12.
If successful, character teleports to a region that they are intimately familiar with (Beginner), have visited before (Adept), or have heard of (Master).\end{flushleft}}  &   \parbox[t]{\x cm}{\begin{center}5\end{center}}  &   \parbox[t]{\z cm}{\begin{center} \it Adept Sorcerer\end{center}}  \\  \bf \begin{flushleft} Archer\end{flushleft}  &   \parbox[t]{\y cm}{\begin{flushleft}Use a 1d4/ 1d6 / 1d8  /1d10  /1d12 / 1d20 dice to determine your long range accuracy checks.\end{flushleft}}  &   \parbox[t]{\x cm}{\begin{center}5\end{center}}  &   \parbox[t]{\z cm}{\begin{center} \it \end{center}}  \\  \bf \begin{flushleft} Battlemage\end{flushleft}  &   \parbox[t]{\y cm}{\begin{flushleft}May use a 1d6/8/10/12/20 die to cast Hexes \& Curses
You are able to use Hexes \& Curses which match your level in this skill.\end{flushleft}}  &   \parbox[t]{\x cm}{\begin{center}5\end{center}}  &   \parbox[t]{\z cm}{\begin{center} \it \end{center}}  \\  \bf \begin{flushleft} Brawler\end{flushleft}  &   \parbox[t]{\y cm}{\begin{flushleft}Can perform non-magical melee attacks. Each level unlocks a different type of combat: \\ 1: Unarmed combat \\ 2: Small melee weapons (daggers, knives) \\ 3: One handed melee weapons (swords, spears, axes) \\ 4:  Two-handed melee weapons (warhammer, battleaxes, broadsword) \\ 5: Complex weapons\end{flushleft}}  &   \parbox[t]{\x cm}{\begin{center}5\end{center}}  &   \parbox[t]{\z cm}{\begin{center} \it \end{center}}  \\  \bf \begin{flushleft} Broomstick Lessons\end{flushleft}  &   \parbox[t]{\y cm}{\begin{flushleft}If you have a broomstick, can fly to other locations. Flight speed increases with each level. Mid-air dodging checks gets a + 1 / 2 / 3 / 4 / 5 boost.\end{flushleft}}  &   \parbox[t]{\x cm}{\begin{center}5\end{center}}  &   \parbox[t]{\z cm}{\begin{center} \it \end{center}}  \\  \bf \begin{flushleft} Catastrophic Critical\end{flushleft}  &   \parbox[t]{\y cm}{\begin{flushleft}Upon a critical hit opportunity, roll a 1d 4 / 6 / 8 / 10 / 20. Multiply the damage by the outcome of this dice roll. This overrides the usual critical procedure.\end{flushleft}}  &   \parbox[t]{\x cm}{\begin{center}5\end{center}}  &   \parbox[t]{\z cm}{\begin{center} \it \end{center}}  \\  \bf \begin{flushleft} Clairvoyant\end{flushleft}  &   \parbox[t]{\y cm}{\begin{flushleft}May use a 1d6/8/10/12/20 die to cast diviniation spells
At Master level, you may spontaneously get visions of what is about to occur.
You are able to use Divination spells which match your level in this skill.\end{flushleft}}  &   \parbox[t]{\x cm}{\begin{center}5\end{center}}  &   \parbox[t]{\z cm}{\begin{center} \it \end{center}}  \\  \bf \begin{flushleft} Countervail\end{flushleft}  &   \parbox[t]{\y cm}{\begin{flushleft}Resist checks may use a 1d6/8/10/12/20 dice.\end{flushleft}}  &   \parbox[t]{\x cm}{\begin{center}5\end{center}}  &   \parbox[t]{\z cm}{\begin{center} \it \end{center}}  \\  \bf \begin{flushleft} Defender\end{flushleft}  &   \parbox[t]{\y cm}{\begin{flushleft}May use a 1d6/8/10/12/20 die to cast Recuperationspells
You are able to use Recuperation magic which match your level in this skill.\end{flushleft}}  &   \parbox[t]{\x cm}{\begin{center}5\end{center}}  &   \parbox[t]{\z cm}{\begin{center} \it \end{center}}  \\  \bf \begin{flushleft} Emergency Care\end{flushleft}  &   \parbox[t]{\y cm}{\begin{flushleft}You may take a major action to perform a {\it stabilization} action. Perform an EMP(healing) check (DV 15), if successful, remove the {\it Critical Condition} status and apply the {\it Critical but Stable} condition\end{flushleft}}  &   \parbox[t]{\x cm}{\begin{center}1\end{center}}  &   \parbox[t]{\z cm}{\begin{center} \it \end{center}}  \\  \bf \begin{flushleft} Familiar\end{flushleft}  &   \parbox[t]{\y cm}{\begin{flushleft}You may have one of the following to accompany you: \\ Beginner:   Newt, Toad \\ Novice:       Cat, Rat, Snake \\ Adept:        Owl, Poltergeist  \\ Expert:       Hippogriff, Phoenix \\Master:      (negotiate with your GM!)\end{flushleft}}  &   \parbox[t]{\x cm}{\begin{center}5\end{center}}  &   \parbox[t]{\z cm}{\begin{center} \it Flora \& Fauna > 2\end{center}}  \\  \bf \begin{flushleft} Fast Caster\end{flushleft}  &   \parbox[t]{\y cm}{\begin{flushleft}You may cast two spells during a major action. If the first spell fails, so does the second.\end{flushleft}}  &   \parbox[t]{\x cm}{\begin{center}1\end{center}}  &   \parbox[t]{\z cm}{\begin{center} \it Adept Battlemage\end{center}}  \\  \bf \begin{flushleft} Linguist\end{flushleft}  &   \parbox[t]{\y cm}{\begin{flushleft}For each level of this skill, you can pick a new language to learn. Each language must be declared when levelling up. To learn Parseltongue, you require EVL to be greater than 4.\end{flushleft}}  &   \parbox[t]{\x cm}{\begin{center}5\end{center}}  &   \parbox[t]{\z cm}{\begin{center} \it \end{center}}  \\  \bf \begin{flushleft} Magician\end{flushleft}  &   \parbox[t]{\y cm}{\begin{flushleft}May use a 1d6/8/10/12/20 die to cast Illusion spells
You are able to use Illusion spells which match your level in this skill.\end{flushleft}}  &   \parbox[t]{\x cm}{\begin{center}5\end{center}}  &   \parbox[t]{\z cm}{\begin{center} \it \end{center}}  \\  \bf \begin{flushleft} Metamorphmagus\end{flushleft}  &   \parbox[t]{\y cm}{\begin{flushleft}Perform a CHR check. GM sets the difficulty dependent on the extent to which you must change your appearance, and the surrounding circumstances.\end{flushleft}}  &   \parbox[t]{\x cm}{\begin{center}1\end{center}}  &   \parbox[t]{\z cm}{\begin{center} \it Expert Thaumaturgy\end{center}}  \\  \bf \begin{flushleft} Mimicry\end{flushleft}  &   \parbox[t]{\y cm}{\begin{flushleft}You can mimic the voice of another sapient, or the call of an animal, provided you have heard them for at least 1 minute previously.\end{flushleft}}  &   \parbox[t]{\x cm}{\begin{center}1\end{center}}  &   \parbox[t]{\z cm}{\begin{center} \it CHR > 19\end{center}}  \\  \bf \begin{flushleft} Momentum Dodge\end{flushleft}  &   \parbox[t]{\y cm}{\begin{flushleft}If you begin the phase behind cover, you may emerge from cover, take an action, and still be considered to be in an evasion phase, taking a 4 / 3 / 2 / 1 / 0 point penalty to the evasion check. \\ You may not move the next turn.\end{flushleft}}  &   \parbox[t]{\x cm}{\begin{center}5\end{center}}  &   \parbox[t]{\z cm}{\begin{center} \it ATH > 10\end{center}}  \\  \bf \begin{flushleft} Musician\end{flushleft}  &   \parbox[t]{\y cm}{\begin{flushleft}Replace your wand with your music: perform all illusion \& protective magic with your instrument of choice, using a CHR (performance) check. Spells take 2 turns to cast in this fashion. 
Using Silent Casting negates this effect.\end{flushleft}}  &   \parbox[t]{\x cm}{\begin{center}5\end{center}}  &   \parbox[t]{\z cm}{\begin{center} \it Performance > 0\end{center}}  \\  \bf \begin{flushleft} Necromancer\end{flushleft}  &   \parbox[t]{\y cm}{\begin{flushleft}May use a 1d/6/8/10/12/20 die to cast Dark Arts spells.
You can use Dark Arts spells which match your level in this skill.\end{flushleft}}  &   \parbox[t]{\x cm}{\begin{center}5\end{center}}  &   \parbox[t]{\z cm}{\begin{center} \it \end{center}}  \\  \bf \begin{flushleft} On the Ball\end{flushleft}  &   \parbox[t]{\y cm}{\begin{flushleft}Get 10 / 15 / 20 / 25 / 30 seconds to make a decision about a counterspell, rather than the usual 5\end{flushleft}}  &   \parbox[t]{\x cm}{\begin{center}5\end{center}}  &   \parbox[t]{\z cm}{\begin{center} \it \end{center}}  \\  \bf \begin{flushleft} Overcome Resistance\end{flushleft}  &   \parbox[t]{\y cm}{\begin{flushleft}Choose a damage type: fire, cold, electric, necrotic and holy. You may ignore any resistance below 100\% to this type. Every time you take this skill again, choose a different damage type.\end{flushleft}}  &   \parbox[t]{\x cm}{\begin{center}5\end{center}}  &   \parbox[t]{\z cm}{\begin{center} \it Beginner Battlemage\end{center}}  \\  \bf \begin{flushleft} Parry\end{flushleft}  &   \parbox[t]{\y cm}{\begin{flushleft}Acts in place of a counterspell in close range physical combat. The attacker and defender both perform an ATH (strength) check (with weapon modifiers). If the defender succeeds, the attack fails.\end{flushleft}}  &   \parbox[t]{\x cm}{\begin{center}1\end{center}}  &   \parbox[t]{\z cm}{\begin{center} \it Novice Brawler\end{center}}  \\  \bf \begin{flushleft} Proficiency Boost\end{flushleft}  &   \parbox[t]{\y cm}{\begin{flushleft}Choose one proficiency associated with an attribute with a score > 10. Increase that proficiency by 1.\end{flushleft}}  &   \parbox[t]{\x cm}{\begin{center}5\end{center}}  &   \parbox[t]{\z cm}{\begin{center} \it Various\end{center}}  \\  \bf \begin{flushleft} Raw Power\end{flushleft}  &   \parbox[t]{\y cm}{\begin{flushleft}Once per day, may use perform a POW check to cast a spell, rather than the specified check, at the cost of doubling the FP cost of the cast. Cannot be used in the learning spell procedure or counterspells.\end{flushleft}}  &   \parbox[t]{\x cm}{\begin{center}1\end{center}}  &   \parbox[t]{\z cm}{\begin{center} \it \end{center}}  \\  \bf \begin{flushleft} Signature Spell\end{flushleft}  &   \parbox[t]{\y cm}{\begin{flushleft}You may have 1/2/3/4/5 spells that are your `signature�. These spells must be one level below your current level in their respective fields, but you get a +3 casting check on these spells.\end{flushleft}}  &   \parbox[t]{\x cm}{\begin{center}5\end{center}}  &   \parbox[t]{\z cm}{\begin{center} \it \end{center}}  \\  \bf \begin{flushleft} Silent Magic\end{flushleft}  &   \parbox[t]{\y cm}{\begin{flushleft}Spells that normally require an incantation can be used silently.
Silent magic checks suffer   -4 / - 3 / - 2 / - 1 / 0 penalty to all checks.\end{flushleft}}  &   \parbox[t]{\x cm}{\begin{center}5\end{center}}  &   \parbox[t]{\z cm}{\begin{center} \it \end{center}}  \\  \bf \begin{flushleft} Sorcerer\end{flushleft}  &   \parbox[t]{\y cm}{\begin{flushleft}May use a 1d6/8/10/12/20 die to cast Charms \\
You are able to use Charms which match your skill in this skill.\end{flushleft}}  &   \parbox[t]{\x cm}{\begin{center}5\end{center}}  &   \parbox[t]{\z cm}{\begin{center} \it \end{center}}  \\  \bf \begin{flushleft} Surge\end{flushleft}  &   \parbox[t]{\y cm}{\begin{flushleft}You may take one extra major action in your turn. This skill cannot be used again until you rest for 4+ hours.\end{flushleft}}  &   \parbox[t]{\x cm}{\begin{center}1\end{center}}  &   \parbox[t]{\z cm}{\begin{center} \it \end{center}}  \\  \bf \begin{flushleft} Thaumaturgus\end{flushleft}  &   \parbox[t]{\y cm}{\begin{flushleft}May use a 1d6/8/10/12/20 die to cast Transfiguration spells \\
You are able to use Transfiguration spells which match your level in this skill.\end{flushleft}}  &   \parbox[t]{\x cm}{\begin{center}5\end{center}}  &   \parbox[t]{\z cm}{\begin{center} \it \end{center}}  \\  \bf \begin{flushleft} Wandless Magic\end{flushleft}  &   \parbox[t]{\y cm}{\begin{flushleft}May cast a spell without using a wand. All wandless magic is also silent but is only 50\% effective compared to the same spell with a wand.
Can only cast spells which match your level in this skill.\end{flushleft}}  &   \parbox[t]{\x cm}{\begin{center}5\end{center}}  &   \parbox[t]{\z cm}{\begin{center} \it Adept Silent Magic\end{center}}  \\  \hline\end{longtable} \end{center}%%SkillEnd

\twocolumn
\normalsize
~
\clearpage
