\documentclass[CoreRulebook.tex]{subfile}

\chapter{Character Progression \& Skills}

\section{Character Progression}
Each character has a `level' associated with them, which denotes how far your character has progressed, and how powerful they are.  Levelling your character is key to progressing: it unlocks new skills, boosts your attributes, and gives access to new spells. A higher-level magic user is a stronger magic user. A stronger magic user is less likely to get eaten by a passing beast, which is generally considered a good thing. 

\subsection{Experience}

Increasing the level of your character (`levelling up') is achieved by accumulating experience. To progress from level 1 to level 2, you must accumulate 100 experience points (EP). When your character reaches 100EP, they ascend to level 2, and the counter is reset. To go from level 2 to level 3 you need to acquire another 200 EP, and so on and so forth. The EP needed to go from level $x$ to $x+1$ is calculated from:

$$ EP_{x \to x + 1} = 100 x $$

Experience is gained by completing actions and defeating enemies. Experience is awarded for completing difficult actions such as casting a spell, mixing a potion, defeating an enemy in combat, or convincing someone to give you something. The GM will instruct you to roll a dice, and you will gain that much experience from completing the action.

The dice you roll (and hence the amount of experience you gain) from such an action depends on your proficiency in that skill. For instance, a first year student gains far more knowledge and experience from casting wingardium leviosa than a seasoned auror does. Hence, as you progress, you will learn less experience from trivial actions. 

As a rough guide, performing an action (such as casting a spell) which is of the same proficiency level as you are will get a 2d20 roll, using one level below your proficiency is a 2d12, and so on:

\begin{center}
	\begin{rndtable}{|c c|}
	\hline \bf Relative Proficiency & \bf Experience Roll
	\\ 
	Same level 		& 	2d20
	\\ 
	1 level below 	&	2d12
	\\ 
	2 levels below	& 	2d8
	\\ 
	3 levels below	&	2d6
	\\ 
	4 levels below 	& 	2d4
	\\ 	\hline
\end{rndtable}
\end{center}

For example, a character with the Adept Battlemage (combat magic) skill would roll a 2d20 for successfully casting the Impediment Jinx (an adept level combat spell), whilst if they were an Master Thaumaturge (transfiguration), they would only get to roll a 2d8 for casting an Adept transfiguration spell, as this is 2 levels below Master. 

Experience is only awarded when an action is truly succesful (i.e. a spell has to hit its target, as well as be succesfully cast). 

\subsection{Levelling Up}
When your experience reaches the requisite amount, you immediately trigger the levelling up process. When you level up, you make the following changes to your character:

\begin{itemize}[itemsep=0em]
	\item Increase character level by 1, and reset EXP counter to zero (you may carry any excess EXP over)
	\item Increase Archetype level by one {\bf OR} choose a new archetype (see multiclassing rules on page \pageref{S:Multiclassing}). Add any new Features you gain at this point.
	\item You may choose one of the following:
	\begin{itemize}[itemsep=0em]
		\item Increase an attribute by 2, or two attributes by 1
		\item Choose a new Skill
	\end{itemize}
	\item Calculate new HP and FP ceilings
	\item Reset HP and FP to maximum
	\item Reset spell-learned counter
\end{itemize}

\subsection{Other Changes}

The GM may also decide that, during the normal course of play, you have done something that warrants a permanent bonus or penalty -- be it something you have learned from extensive practice, or a gift from some higher being -- the GM will grant you a bonus to your Attributes or Proficiencies. This will probably most commonly be used to penalise players for immoral actions -- by increasing their EVL level.

\newpage
\section{Skills}

Skills are learned abilities that your character picks up along the way. They can be learned either by levelling up or given as gifts by external devices. Some skills are only temporary and will wear off after a while. Skills are key to learning new and more powerful magic, as well as ulocking useful abilities. Most skills come in 5 levels: Beginner, Novice, Adept, Expert and Master, which must be learned in that order. 

The 8 most important skills are the magic-school skills: skills which are each associated with one of the 7 schools of magic (and Resist checks). They are,

\begin{center}
\begin{rndtable}{|c c|}
\hline \bf  Magic School  &  \bf Associated Skill
\\ 
Hexes \& Curses  	& 	Battlemage
\\
Transfiguration		& 	Thaumaturgus
\\
Charms				&	Sorcerer
\\
Recuperation	& 	Defender
\\
Illusion				&	Magician
\\
Divination			& 	Clairvoyant
\\
Dark Arts			& 	Necromancy
\\
Countervail		&	Resist Magic
\\ \hline
\end{rndtable}
\end{center}

Learning more powerful spells in each school of magic requires more and more levels in the relevant skill. To cast Expert level Hexes and Curses, you need to be an Expert Battlemage, and so on. 

\subsection{Aquiring Skills}

Skills can be acquired in three ways: those set at initial character creation, provided temporarily by the use of items, or acquired permanently by levelling up.

At character creation, all characters start at Beginner level in the 6 non-dark magic school-skills (i.e. Battlemage etc.), and with no skill in all others. You may then add your Racial and Background skills, before selecting any extra skills that you might be allowed. 

Some items that you find might grant you special skills whilst wearing themL a True Cloak of Invisibility, for example, not only confers the Invisible Status, but also confers the Camouflage skill whilst it is being worn. When the item is removed, the Skill is also removed from the character. 

Every time your character levels up, you get to pick a new skill to add to your character,  as long as you meet the minimum requirements. This skill point must be spent immediately and cannot be saved for later. 

As already noted, most Skills come in 5 levels: Beginner, Novice, Adept, Expert and Master. Unless otherwise stated, you must meet the minimum level requirement, and have the preceding level skill before you can acquire the next one. The skill list table gives the minimum requirements in order to get the next level in that skill. However, \textbf{as each skill is acquired (outside character creation), the minimum level requirement increases by 2. }

Skills that are aquired from character creation, but which you would not normally be able to access until much later (i.e. a House-Elf gets access to apparition, usually a LVL10 skill), follow this same pattern -- \textbf{however, the minimum LVL requirement is reduced by 4}, and it should be noted that this is the requirement to get the \textit{next} level up. 

For example, apparition as mentioned already is a level 10 skill, so a normal half-blood auror would need to reach LVL10 to become a beginner, and then LVL 12 to get novice, and so on. In contrast, a house elf is a beginner apparator at birth so the requirement to get the next level (i.e. LVL 10) is reduced by 4, and so {\bf only needs to get to LVL6 in order to meet the minimum requirements} to become a Novice apparator. 

\subsection{Automatic Skill Aquiring} \label{S:Auto}

Some skills are acquired automatically through levelling up, without you having to choose. Sometimes, these skills may be givn to your by the GM for narrative reasons, or in order to further a stagnating game. 

The 8 Magic skills can indeed be levelled up by player choice in the usual fashion in order to get access to those spells slower. However, you may also gain these skills simply by reaching the appropriate level:

\begin{center}
\begin{rndtable}{|c|c|}
\hline \bf  Skill Level  &  \bf Acquiring Level
\\ 
Beginner 	&    1
\\
Novice		&    5
\\
Adept		&     10
\\
Expert         &     15   
\\
Master    &    20
\\ \hline
\end{rndtable}
\end{center}

When an auto-levelling is incurred, if you already have that spellcasting level thanks to taking the apropriate Skill manually, you instead get a +1 increase to your arcane wisdom for each Skill you took. This only applies to manually-chosen skills, it does not apply to increases due to the Spellcasting Improvement archetype feature. 


\onecolumn
\section{Skill List}
\def\w{2.6}
\def\x{5.1}
\def\y{6.9}
\def\z{2}
\footnotesize

%%SkillBegin
\begin{center} \tablealternate \begin{longtable}{|m{\w cm} m{\x cm} m{\y cm} m{\z cm}|}\hline \tablehead \normalsize \bf Name & \normalsize \bf Description & \normalsize \bf Effect & \normalsize \bf Prerequisite \\ \hline \hline \bf \begin{flushleft} Ambidextrous\end{flushleft}  &  \parbox[t]{\x cm}{\begin{flushleft}You are able to use both hands as if they were your dominant hand.\end{flushleft}}  &   \parbox[t]{\y cm}{\begin{flushleft}No penalty for using items in your non-dominant hand. (Single level skill)\end{flushleft}}  &   \parbox[t]{\z cm}{\begin{center}LVL 5\end{center}}  \\  \bf \begin{flushleft} Animagus\end{flushleft}  &  \parbox[t]{\x cm}{\begin{flushleft}You can transform into an animal at will.\end{flushleft}}  &   \parbox[t]{\y cm}{\begin{flushleft}Transform into an non-magical animal at will. This animal must be chosen at the moment you acquire this skill, and cannot be changed afterwards. Transforming costs 5FP and constitutes a major action. \\ Single level skill.\end{flushleft}}  &   \parbox[t]{\z cm}{\begin{center}LVL 10 \\ Expert Thamaturgy \& Adept Animal Lover.\end{center}}  \\  \bf \begin{flushleft} Apparate\end{flushleft}  &  \parbox[t]{\x cm}{\begin{flushleft}You can disappear and re-appear at will. This spell works even without a wand.\end{flushleft}}  &   \parbox[t]{\y cm}{\begin{flushleft}Perform a SPR (willpower) check.
Apparition difficulty is 18 / 17 / 15 / 15 / 12 / 12.
If successful, character teleports to a region that they are intimately familiar with (Beginner), have visited before (Adept), or have heard of (Master).\end{flushleft}}  &   \parbox[t]{\z cm}{\begin{center}LVL 10 \\
Adept Sorcerer\end{center}}  \\  \bf \begin{flushleft} Archer\end{flushleft}  &  \parbox[t]{\x cm}{\begin{flushleft}You can strike targets from further away with your ranged weapons.\end{flushleft}}  &   \parbox[t]{\y cm}{\begin{flushleft}Use a 1d4/ 1d6 / 1d8  /1d10  /1d12 / 1d20 dice to determine your long range accuracy checks.\end{flushleft}}  &   \parbox[t]{\z cm}{\begin{center}LVL 2 \\ Brawler (Beginner)\end{center}}  \\  \bf \begin{flushleft} Battlemage\end{flushleft}  &  \parbox[t]{\x cm}{\begin{flushleft}You become better at using your magic in combat situations\end{flushleft}}  &   \parbox[t]{\y cm}{\begin{flushleft}May use a 1d6/8/10/12/20 die to cast Hexes \& Curses
You are able to use Hexes \& Curses which match your level in this skill.\end{flushleft}}  &   \parbox[t]{\z cm}{\begin{center}LVL 2\end{center}}  \\  \bf \begin{flushleft} Blood Magic\end{flushleft}  &  \parbox[t]{\x cm}{\begin{flushleft}By killing an innocent, you get a massive boost to your powers.\end{flushleft}}  &   \parbox[t]{\y cm}{\begin{flushleft}After murdering an innocent, take one turn to perform the Blood Ritual.
Once the ritual is performed, get a + 4 / 6 / 8 / 10 / 12 bonus to all magic rolls for 1 day.\end{flushleft}}  &   \parbox[t]{\z cm}{\begin{center}LVL 13\end{center}}  \\  \bf \begin{flushleft} Brawler\end{flushleft}  &  \parbox[t]{\x cm}{\begin{flushleft}You are used to getting into physical altercations.\end{flushleft}}  &   \parbox[t]{\y cm}{\begin{flushleft}Can perform non-magical melee attacks. Each level unlocks a different type of combat: \\ 1: Unarmed combat \\ 2: Small melee weapons (daggers, knives) \\ 3: One handed melee weapons (swords, spears, axes) \\ 4:  Two-handed melee weapons (warhammer, battleaxes, broadsword) \\ 5: Complex weapons\end{flushleft}}  &   \parbox[t]{\z cm}{\begin{center}LVL 1\end{center}}  \\  \bf \begin{flushleft} Broomstick Lessons\end{flushleft}  &  \parbox[t]{\x cm}{\begin{flushleft}You can fly a broomstick\end{flushleft}}  &   \parbox[t]{\y cm}{\begin{flushleft}If you have a broomstick, can fly to other locations. Flight speed increases with each level. Mid-air dodging checks gets a + 1 / 2 / 3 / 4 / 5 boost.\end{flushleft}}  &   \parbox[t]{\z cm}{\begin{center}LVL 4\end{center}}  \\  \bf \begin{flushleft} Catastrophic Critical\end{flushleft}  &  \parbox[t]{\x cm}{\begin{flushleft}When you land a surprise attack on an enemy, the effects are truly devastating\end{flushleft}}  &   \parbox[t]{\y cm}{\begin{flushleft}Upon a critical hit opportunity, roll a 1d 4 / 6 / 8 / 10 / 20. Multiply the damage by the outcome of this dice roll. This overrides the usual critical procedure.\end{flushleft}}  &   \parbox[t]{\z cm}{\begin{center}LVL 7\end{center}}  \\  \bf \begin{flushleft} Clairvoyant\end{flushleft}  &  \parbox[t]{\x cm}{\begin{flushleft}Your vision begins to penetrate the veil that separates this world from the powers beyond.\end{flushleft}}  &   \parbox[t]{\y cm}{\begin{flushleft}May use a 1d6/8/10/12/20 die to cast diviniation spells
At Master level, you may spontaneously get visions of what is about to occur.
You are able to use Divination spells which match your level in this skill.\end{flushleft}}  &   \parbox[t]{\z cm}{\begin{center}LVL 2\end{center}}  \\  \bf \begin{flushleft} Cleave\end{flushleft}  &  \parbox[t]{\x cm}{\begin{flushleft}Your powerful swings cut through multiple enemies in one go.\end{flushleft}}  &   \parbox[t]{\y cm}{\begin{flushleft}If a melee attack kills an opponent, you may perform a second attack using the same weapon on one adjacent opponent, with a 4/3/2/1/0 point penalty to the damage (cannot go below zero).\end{flushleft}}  &   \parbox[t]{\z cm}{\begin{center}LVL 4\end{center}}  \\  \bf \begin{flushleft} Countervail\end{flushleft}  &  \parbox[t]{\x cm}{\begin{flushleft}You fight against the spells cast against you with greater and greater fervour, improving your chances of resisting their effects.\end{flushleft}}  &   \parbox[t]{\y cm}{\begin{flushleft}Resist magic checks may use a 1d6/8/10/12/20 dice for the check.\end{flushleft}}  &   \parbox[t]{\z cm}{\begin{center}LVL 2\end{center}}  \\  \bf \begin{flushleft} Curse-Breaker\end{flushleft}  &  \parbox[t]{\x cm}{\begin{flushleft}You can remove the negative effects of spells\end{flushleft}}  &   \parbox[t]{\y cm}{\begin{flushleft}Countercurse and curse-identification checks get a + 2 / 4 /6 /8 / 8 boost.
At Master, curses on yourself last only 2 turns, before being removed automatically.\end{flushleft}}  &   \parbox[t]{\z cm}{\begin{center}LVL 5 \\
Adept Battlemage\end{center}}  \\  \bf \begin{flushleft} Defender\end{flushleft}  &  \parbox[t]{\x cm}{\begin{flushleft}You become better at using your magic to help and heal others.\end{flushleft}}  &   \parbox[t]{\y cm}{\begin{flushleft}May use a 1d6/8/10/12/20 die to cast Recuperationspells
You are able to use Recuperation magic which match your level in this skill.\end{flushleft}}  &   \parbox[t]{\z cm}{\begin{center}LVL 2\end{center}}  \\  \bf \begin{flushleft} Eagle-Eyed\end{flushleft}  &  \parbox[t]{\x cm}{\begin{flushleft}Your eyes are sharper, and you can aim more clearly. You can cast spells at targets that ae further away.\end{flushleft}}  &   \parbox[t]{\y cm}{\begin{flushleft}Increases the effective range of your magic by 10 / 20 / 30 / 40 / 50\%.
At Adept, you get a permanent Night Vision bonus.\end{flushleft}}  &   \parbox[t]{\z cm}{\begin{center}LVL 8\end{center}}  \\  \bf \begin{flushleft} Familiar\end{flushleft}  &  \parbox[t]{\x cm}{\begin{flushleft}You have an animal companion who accompanies you.\end{flushleft}}  &   \parbox[t]{\y cm}{\begin{flushleft}You may have one of the following to accompany you: \\ Beginner:   Newt, Toad \\ Novice:       Cat, Rat, Snake \\ Adept:        Owl, Poltergeist  \\ Expert:       Hippogriff, Phoenix \\Master:      (negotiate with your GM!)\end{flushleft}}  &   \parbox[t]{\z cm}{\begin{center}LVL 3 \\
Beginner Animal Lover\end{center}}  \\  \bf \begin{flushleft} Golden Touch\end{flushleft}  &  \parbox[t]{\x cm}{\begin{flushleft}Where most people find a gold coin, you somehow manage to find more.\end{flushleft}}  &   \parbox[t]{\y cm}{\begin{flushleft}Get 10 / 20 / 30 / 40 / 50\% more gold from transactions, and get the same decrease in costs.\end{flushleft}}  &   \parbox[t]{\z cm}{\begin{center}LVL 4\end{center}}  \\  \bf \begin{flushleft} Holy Aura\end{flushleft}  &  \parbox[t]{\x cm}{\begin{flushleft}You are imbued with the power of the Light: the undead fear you.\end{flushleft}}  &   \parbox[t]{\y cm}{\begin{flushleft}Undead attacks on you are 10 / 20 / 30 / 40 / 50\% less effective.
Undead creatures perform a SPR check, and compare to your EMP check ( + 1/ 2 / 3 / 4 / 5). If your check is greater than theirs, they become {\it scared}.\end{flushleft}}  &   \parbox[t]{\z cm}{\begin{center}LVL 7 \\

EVL <  3\end{center}}  \\  \bf \begin{flushleft} Improvise\end{flushleft}  &  \parbox[t]{\x cm}{\begin{flushleft}You can make small adjustments to the effects of spells, and can use spells in unusual ways.\end{flushleft}}  &   \parbox[t]{\y cm}{\begin{flushleft}Unconventional uses of spells (as judged by GM) get + 1 / 2 / 3 / 4 / 5 bonus to casting checks
From Adept, you can (with GMs consent) make small adjustments to the outcome of spells. i.e. make specific exceptions to wards.\end{flushleft}}  &   \parbox[t]{\z cm}{\begin{center}LVL 11 \\Adept Clairvoyant\end{center}}  \\  \bf \begin{flushleft} Linguist\end{flushleft}  &  \parbox[t]{\x cm}{\begin{flushleft}You can speak different languages\end{flushleft}}  &   \parbox[t]{\y cm}{\begin{flushleft}For each level of this skill, you can pick a new language to learn. Each language must be declared when levelling up. To learn Parseltongue, you require EVL to be greater than 4.\end{flushleft}}  &   \parbox[t]{\z cm}{\begin{center}LVL 2\end{center}}  \\  \bf \begin{flushleft} Magician\end{flushleft}  &  \parbox[t]{\x cm}{\begin{flushleft}Your illusion spells gain more power, and you can hoodwink people with your magic.\end{flushleft}}  &   \parbox[t]{\y cm}{\begin{flushleft}May use a 1d6/8/10/12/20 die to cast Illusion spells
You are able to use Illusion spells which match your level in this skill.\end{flushleft}}  &   \parbox[t]{\z cm}{\begin{center}LVL 2\end{center}}  \\  \bf \begin{flushleft} Metamorphmagus\end{flushleft}  &  \parbox[t]{\x cm}{\begin{flushleft}You can change your appearance at will.\end{flushleft}}  &   \parbox[t]{\y cm}{\begin{flushleft}Perform a CHR check. GM sets the difficulty dependent on the extent to which you must change your appearance, and the surrounding circumstances. (Single level skill)\end{flushleft}}  &   \parbox[t]{\z cm}{\begin{center}LVL 12 \\
Expert at Thamaturgus\end{center}}  \\  \bf \begin{flushleft} Momentum Dodge\end{flushleft}  &  \parbox[t]{\x cm}{\begin{flushleft}You are able to emerge from cover firing,\end{flushleft}}  &   \parbox[t]{\y cm}{\begin{flushleft}If you begin the phase behind cover, you may emerge from cover, take an action, and still be considered to be in an evasion phase, taking a 4 / 3 / 2 / 1 / 0 point penalty to the evasion check. \\ You may not move the next turn.\end{flushleft}}  &   \parbox[t]{\z cm}{\begin{center}LVL 6\end{center}}  \\  \bf \begin{flushleft} Musician\end{flushleft}  &  \parbox[t]{\x cm}{\begin{flushleft}The joining of magic and music is as old as time itself, allowing you to manipulate powerful primal forces.\end{flushleft}}  &   \parbox[t]{\y cm}{\begin{flushleft}Performance check gets +1/2/3/4/5 bonus. Replace your wand with your music: perform all illusion \& protective magic with your instrument of choice, using a CHR (performance) check. Spells take 2 turns to cast in this fashion. 
Using Silent Casting negates this effect.\end{flushleft}}  &   \parbox[t]{\z cm}{\begin{center}LVL 5\end{center}}  \\  \bf \begin{flushleft} Necromancer\end{flushleft}  &  \parbox[t]{\x cm}{\begin{flushleft}You are a plight on this world, dark magic has corrupted you, and you are corrupting life itself.\end{flushleft}}  &   \parbox[t]{\y cm}{\begin{flushleft}May use a 1d/6/8/10/12/20 die to cast Dark Arts spells.
You can use Dark Arts spells which match your level in this skill.\end{flushleft}}  &   \parbox[t]{\z cm}{\begin{center}LVL 6 \\EVL $\geq$ 6\end{center}}  \\  \bf \begin{flushleft} On the Ball\end{flushleft}  &  \parbox[t]{\x cm}{\begin{flushleft}Your reactions are well honed, and you can think clearly under pressure.\end{flushleft}}  &   \parbox[t]{\y cm}{\begin{flushleft}Get 10 / 15 / 20 / 25 / 30 seconds to make a decision about a counterspell, rather than the usual 5\end{flushleft}}  &   \parbox[t]{\z cm}{\begin{center}LVL 1\end{center}}  \\  \bf \begin{flushleft} Parry\end{flushleft}  &  \parbox[t]{\x cm}{\begin{flushleft}You are able to retaliate when somebody attacks you in close-quarters physical combat\end{flushleft}}  &   \parbox[t]{\y cm}{\begin{flushleft}Acts in place of a counterspell in close range physical combat. The attacker and defender both perform an ATH (strength) check (with weapon modifiers). If the defender succeeds, the attack fails. 
(Single level skill)\end{flushleft}}  &   \parbox[t]{\z cm}{\begin{center}LVL 4 \\Novice Brawler\end{center}}  \\  \bf \begin{flushleft} Protected\end{flushleft}  &  \parbox[t]{\x cm}{\begin{flushleft}You know how to use armour to maximise your defence, and how not to let it slow you down.\end{flushleft}}  &   \parbox[t]{\y cm}{\begin{flushleft}Armour gets a  +1/2/3/4/5 point boost to strength whilst you are wearing it. At Adept, you are no longer carry out the rounding down procedure during movement checks, whilst wearing heavy armour.\end{flushleft}}  &   \parbox[t]{\z cm}{\begin{center}LVL 2\end{center}}  \\  \bf \begin{flushleft} Raw Power\end{flushleft}  &  \parbox[t]{\x cm}{\begin{flushleft}Sometimes, raw magical power can be a substitute for magical skill. Throw enough power and conviction into a spell, and you can cast anything.\end{flushleft}}  &   \parbox[t]{\y cm}{\begin{flushleft}1 /2 / 3 / 4 / 5 times per day, may use perform a POW check to cast a spell, rather than the specified check, at the cost of doubling the FP cost of the cast. Cannot be used in the learning spell procedure or counterspells.\end{flushleft}}  &   \parbox[t]{\z cm}{\begin{center}LVL 4\end{center}}  \\  \bf \begin{flushleft} Regenerative\end{flushleft}  &  \parbox[t]{\x cm}{\begin{flushleft}You can re-gather your thoughts, and recover yourself quickly.\end{flushleft}}  &   \parbox[t]{\y cm}{\begin{flushleft}Fortitude regeneration gets a + 2 / 3 / 4 / 5 / 6 boost per turn where magic is not used.\end{flushleft}}  &   \parbox[t]{\z cm}{\begin{center}LVL 6 \\ Adept Defender\end{center}}  \\  \bf \begin{flushleft} Signature Spell\end{flushleft}  &  \parbox[t]{\x cm}{\begin{flushleft}You have a number of spells that you use all the time, and as such, they come easily to you.\end{flushleft}}  &   \parbox[t]{\y cm}{\begin{flushleft}You may have 1/2/3/4/5 spells that are your `signature�. These spells must be one level below your current level in their respective fields, but you get a +3 casting check on these spells.\end{flushleft}}  &   \parbox[t]{\z cm}{\begin{center}LVL 3\end{center}}  \\  \bf \begin{flushleft} Silent Magic\end{flushleft}  &  \parbox[t]{\x cm}{\begin{flushleft}Can perform magic without speaking\end{flushleft}}  &   \parbox[t]{\y cm}{\begin{flushleft}Spells that normally require an incantation can be used silently.
Silent magic checks suffer   -4 / - 3 / - 2 / - 1 / 0 penalty to all checks.\end{flushleft}}  &   \parbox[t]{\z cm}{\begin{center}LVL 8\end{center}}  \\  \bf \begin{flushleft} Sorcerer\end{flushleft}  &  \parbox[t]{\x cm}{\begin{flushleft}You become better at using charms, and can use more powerful magic.\end{flushleft}}  &   \parbox[t]{\y cm}{\begin{flushleft}May use a 1d6/8/10/12/20 die to cast Charms
You are able to use Charms which match your skill in this\end{flushleft}}  &   \parbox[t]{\z cm}{\begin{center}LVL 2\end{center}}  \\  \bf \begin{flushleft} Surge\end{flushleft}  &  \parbox[t]{\x cm}{\begin{flushleft}You push yourself beyond your normal limits, just for a moment, allowing you to take an extra action\end{flushleft}}  &   \parbox[t]{\y cm}{\begin{flushleft}You may take one extra major action in your turn. This skill cannot be used again until you rest for 4+ hours. Single Level Skill\end{flushleft}}  &   \parbox[t]{\z cm}{\begin{center}LVL 3\end{center}}  \\  \bf \begin{flushleft} Thaumaturgus\end{flushleft}  &  \parbox[t]{\x cm}{\begin{flushleft}You become better at using Transfiguration spells\end{flushleft}}  &   \parbox[t]{\y cm}{\begin{flushleft}May use a 1d6/8/10/12/20 die to cast Transfiguration spells
You are able to use Transfiguration spells which match your level in this skill.\end{flushleft}}  &   \parbox[t]{\z cm}{\begin{center}LVL 2\end{center}}  \\  \bf \begin{flushleft} Trickster\end{flushleft}  &  \parbox[t]{\x cm}{\begin{flushleft}Your traps become harder to detect.\end{flushleft}}  &   \parbox[t]{\y cm}{\begin{flushleft}Traps (magical and otherwise) placed by you have a + 3 / 4 / 5 / 6 / 7 bonus to their detection difficulty.\end{flushleft}}  &   \parbox[t]{\z cm}{\begin{center}LVL 6\end{center}}  \\  \bf \begin{flushleft} Undead Benefactor\end{flushleft}  &  \parbox[t]{\x cm}{\begin{flushleft}Though the undead are not alive, you can still restore their health.\end{flushleft}}  &   \parbox[t]{\y cm}{\begin{flushleft}Healing spells work on the undead. 
(single level skill)\end{flushleft}}  &   \parbox[t]{\z cm}{\begin{center}LVL 7\end{center}}  \\  \bf \begin{flushleft} Wandless Magic\end{flushleft}  &  \parbox[t]{\x cm}{\begin{flushleft}Can perform magic without a wand.\end{flushleft}}  &   \parbox[t]{\y cm}{\begin{flushleft}All wandless magic is also silent but is only 50\% effective compared to the same spell with a wand.
Can only cast spells which match your level in this skill.\end{flushleft}}  &   \parbox[t]{\z cm}{\begin{center}LVL 14 \\Silent Magic (Adept)\end{center}}  \\  \hline\end{longtable} \end{center}%%SkillEnd

\twocolumn
\normalsize
~
\clearpage
