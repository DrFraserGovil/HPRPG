\onecolumn
\begin{landscape}
\small
\section{Magic Cheat Sheet}

This section contains a brief summary of the tables needed for spellcasting. This serves as a useful printout to have on hand, to avoid arduous page-flipping. 

\begin{multicols}{3}
\def\xS{1.8}
\def\wS{2}

\vbox{
\subsubsection{How to Cast}

To cast a spell, either in combat or in day-to-day life, you must declare the spell which you are about to cast. You must be holding your wand in your dominant hand, hand be able to speak the incantaiton aloud, unless you have a skill or character trait which negates these rules.

You must then perform a check by rolling the relevant dice, and then compare this value with the Difficulty Value (DV) of the spell, and then finally deduct the appropriate FP cost. 

\subsubsection{Memory}


}

\vbox{
\subsubsection{Check Type}
Every spell belongs to one of the Disciplines, which determines the character attribute to use when casting that spell. Appropriate Proficiencies may be added onto spellcasting checks as determined by character skills, or by GM's consent that it is applicable to the spell being cast.
\begin{center}
	\begin{rndtable}{c m{\xS cm} p{\wS cm}}
	\bf School	&	\bf Discipline	&	\bf Attribute
	\\
	\school{Charms}{Elemental}{\ElCheck}{Kinesis}{\KinCheck}
	\\
	\school{Divination}{Telepathy}{\TelCheck}{Temporal}{\TemCheck}
	\\
	\school{Illusion}{Bewitchment}{\BewCheck}{Psionics}{\PsiCheck}
	\\
	\school{Malediction}{Hexes}{\HexCheck}{Curses}{\CurCheck}
   \\ 
   \school{Recuperation}{Healing}{\HeaCheck}{Warding}{\WarCheck}
	\\
	\school{Transfiguration}{Alteration}{\AltCheck}{Conjuration}{\ConCheck}
	\\
	\school{Dark Arts}{Necromancy}{\NecCheck}{Occultism}{\OccCheck}
	\end{rndtable}
\end{center}


}
\vbox{
\subsubsection{Check Difficulty}

For a cast to be successful, the result of the casting check must be equal to or larger than the value given in this table:
\begin{center}
	\begin{rndtable}{c c c c c}
		~	&	Instant	&	Focus	&	Ward	&	Ritual
		\\
		\cellcolor{\tablecolorhead} Beginner & 	\DVBegI	&	\DVBegF	&	\DVBegW	&	\DVBegR
		\\
		\cellcolor{\tablecolorhead} Novice	& 	\DVNovI	&	\DVNovF	&	\DVNovW	&	\DVNovR
		\\
		\cellcolor{\tablecolorhead} Adept	&	\DVAdpI	&	\DVAdpF	&	\DVAdpW	&	\DVAdpR	
		\\
		\cellcolor{\tablecolorhead} Expert	& \DVExpI	&	\DVExpF	&	\DVExpW	&	\DVExpR
		\\
		\cellcolor{\tablecolorhead} Master	&	\DVMasI	&	\DVMasI	&	\DVMasW	&	\DVMasR
		
	\end{rndtable}
\end{center}
}

\subsubsection{FP Costs}
Spells `cost' FP to cast. Failed spells cost half the amount of a successful spell and Resisting a spell costs 2FP. The FP cost of a spell is numerically equal to the difficulty of a spell, prior to any skill modifications (i.e. a skill which reduces the difficulty of a certain spell does not reduce the FP of it, and vice versa), unless the spell is being book-cast, in which case use the bracketed values.  


\small
\begin{center}
	\begin{rndtable}{p {1.2cm} |c m{\wFP cm} | c m{\wFP cm} |c m{\wFP cm}| c m{\wFP cm}}
		~ & \multicolumn{2}{c}{\bf Instant} & \multicolumn{2}{c}{\bf Focus} & \multicolumn{2}{c}{\bf Ward} & \multicolumn{2}{c}{\bf Ritual}
		\\
		\cellcolor{\tablecolorhead} Beginner & 	\FPEntry{\DVBegI}  & 	\FPEntry{\DVBegF}	& 	\FPEntry{\DVBegW} 	& 	\FPEntry{\DVBegR}
		\\
		\cellcolor{\tablecolorhead} Novice	& 	\FPEntry{\DVNovI}  & 	\FPEntry{\DVNovF}	& 	\FPEntry{\DVNovW} 	& 	\FPEntry{\DVNovR}
		\\
		\cellcolor{\tablecolorhead} Adept	&	\FPEntry{\DVAdpI}  & 	\FPEntry{\DVAdpF}	& 	\FPEntry{\DVAdpW} 	& 	\FPEntry{\DVAdpR}
		\\
		\cellcolor{\tablecolorhead} Expert	&	\FPEntry{\DVExpI}  & 	\FPEntry{\DVExpF}	& 	\FPEntry{\DVExpW} 	& 	\FPEntry{\DVExpR}
		\\
		\cellcolor{\tablecolorhead} Master	&	\FPEntry{\DVMasI}  & 	\FPEntry{\DVMasF}	& 	\FPEntry{\DVMasW} 	& 	\FPEntry{\DVMasR}
	\end{rndtable}
\end{center}
\normalsize
FP Regenerates at a rate of 2FP per turn cycle in which no FP was deducted. 

\subsubsection{Resisting}

\end{multicols}
\end{landscape}
