\documentclass[../CoreRulebook.tex]{subfile}
\twocolumn
\chapter{Manipulate Spells}
\label{S:Manipulate}
\newcommand\PPDifference[2]
{
(+ {\the\numexpr(#1 -#2)\relax} PP)
}

The `Manipulate' class of spells are unusual in that their description is left vague, relating only to the physical effect the spell has, and not how it relates directly to the game mechanics. This is deliberate, as the use of these spells is inherently limited only by the caster\apos{}s imagination. 

In this section, some possible uses for how to tie these spells into the game mechanics are given, and examples of how the spells are stratified by difficulty. This list is not meant to be exhaustive, and your GM may determine the difficulty associated with a proposed use of a `Manipulate' spell. 

The PP costs given here are given relative to the beginner-level casting. A master-level spell is therefore assumed to have \PPDifference{\DVMasF}{\DVBegF} assigned to it already. Additional PP dedicated to the spell are given on top of this. 


\subsection{Manipulate Air}\label{S:ManipulateAir}
\begin{spellitemize}
\item {\bf Beginner} 

\airB
\begin{spellitemize}
\item {\it Gust}: Cause a localised light breeze within a 5m radius. This breeze is strong enough to divert the path of a light projectile by approximately half a metre.
\item {\it Distract}: Summon a breeze to cause a commotion behind an opponent such that, on a failed DV 5 INT Resist, they are open to an Attack of Opportunity. 
\end{spellitemize}
\item {\bf Novice \PPDifference{\DVNovF}{\DVBegF} }

\airN

\begin{spellitemize}
\item {\it Windtunnel}: direct a powerful blast of air in a line up to 5m long. On a failed Resist, any being or object in the path is blown back to the edge of the range of the spell, taking (2+PP)d4 concussive damage. 
\item {\it Airboost}: use air currents to give a target within range a boost of 1m to their base speed and allow them to jump up to 3m in a single bound. Alternatively, use this against a foe: Slow a being by 1m per turn, or cause any acrobatics to fail,  leaving them prone. Negated on a successful Resist.   
\end{spellitemize}
\item {\bf Adept: \PPDifference{\DVAdpF}{\DVBegF}} 

\airA

\begin{spellitemize}
\item {\it Updraft}: a powerful blast of air lifts everything within a (3+PP) radius of the caster heavier than (5+PP)$\times$ your caster level (in kg) to be thrown 10m into the air, doing 5d4 bludgeoning damage. 
\item {\it Downdraft}: a wall of air slams into everything within a cylinder  (3+PP) in radius around the caster and 5m in height. All airborne objects (but not spell-bolts) slam into the ground and take double the usual falling damage. 

\item {\it Cloudmove}: by maintaining focus for 1 minute, you may summon a brisk breeze over an area 1km in size. You may use this to move a raincloud out of the way, or to summon a mild drizzle over the targeted area. This does not work in conditions with a strong wind already present: instead you simply slow that wind down. 

\end{spellitemize}

\item {\bf Expert:\PPDifference{\DVExpF}{\DVBegF}}

\airE

\begin{spellitemize}
\item {\it Arctic Wind}: cool the air you control to freezing. All beings affected by other effects of this spell (except large-scale weather manipulation) take PPd4 cold damage per turn in addition to any other effects. 
\item {\it Vortex Field}: summon a powerfull, swirling wall of air to act as a shield around you in a (1+PP)m radius. Physical objects and people entering the  field must Resist, or be flung 10m in a random direction. The field is opaque in both directions, so spells cast through the field must succeed an accuracy check to hit something on the other side of the field. 
\end{spellitemize}
\item {\bf Master: \PPDifference{\DVMasF}{\DVBegF}}

\airM
\begin{spellitemize}
\item {\it Tempest}: Summon a terrifying storm over an area 1km$^2$. The storm limits visibility to 10\% and deals PPd4 bludgeoning damage, and half as much again cold damage to all targets in this radius. In addition, the caster summon PPd4 lightning bolts per turn, each of which does 3d8 electrical damage. This spell costs half the Master-level FP to maintain per turn. 
\item {\it Flight}: Use precision manipulation to lift one being of up to (20$\times$PP)kg into the air and move it freely at a speed of up to 10$\times$PP mph within a radius of 200m of the caster. The caster may cast this on themselves to mimic the effects of true flight.  
\end{spellitemize}

\end{spellitemize}

\subsection{Manipulate Earth}\label{S:ManipulateEarth}
\begin{spellitemize}


\item {\bf Beginner} 

\earthB
\begin{spellitemize}
\item {\it Tremor}: Cause the ground to shake and emit a low rumble. All beings in a 10m radius must make a DV (5+PP) FIN Resist check to maintain their balance, or take check disadvantage next turn. 
\item {\it Pebbledash}: Cause a number of small stones to hurl themselves at a target within a range of 5m of the caster, doing (1+PP)d4 bludgeoning damage. 
\end{spellitemize}



\item {\bf Novice \PPDifference{\DVNovF}{\DVBegF} }

\earthN
\begin{spellitemize}
\item {\it Excavate:} target an unnocupied area of loose or packed earth up to (1+PP)m in radius. You can instantly excavate it down to a depth of PP/2 metres, and move it up to 4m per turn. Excavated `packed' earth is considered `loose' after being excavated. 
\item {\it Mold:} target an area of loose earth of a cube up to PP/2 metres in length and manipulate it into taking on any shape you desire. This shape may `defy physics' during the molding, but as soon as your concentration is broken, the shape is liable to crumble. Whilst being manipulated, constructions are not strongly bound, and so cannot be weaponised, and a normal human could break them apart with ease. 
\item {\it Holdfast:} root yourself or a target into the Earth, trapping you in position, but rendering you immune to forced-movement effects. Can be broken by a DV (3 + PP) ATH (Strength) check.  
\end{spellitemize}


\item {\bf Adept: \PPDifference{\DVAdpF}{\DVBegF}} 

\earthA
\begin{spellitemize}
\item {\it Erupt}: target a region (1+PP)m in radius. A fountain of churned earth erupts vertically upwards in that region, damaging all those inside the region for (1+PP)d12 bludgeoning damage. This region is considered `loose footing' until cleared, which takes $5 \times$PP turns to do. 
\item {\it Animate Earth}: as with {\it mold}, but the earth is considered `packed' during motion and you may manipulate `packed' as well as `loose' Earth. You may clumsily animate the manipulated material, and, for example, create an animated hand or club out of the manipulated earth to strike at an enemy, doing (1+PP)d8 damage (either bludgeoning or piercing, depending on the shape of the creation). 
\end{spellitemize}



\item {\bf Expert:\PPDifference{\DVExpF}{\DVBegF}}

\earthE
\begin{spellitemize}
\item {\it Grand Manipulation}: 
\item {\it Raise Wall}:
\item {\it Clad Being}:
\end{spellitemize}


\item {\bf Master: \PPDifference{\DVMasF}{\DVBegF}}

\earthM
\begin{spellitemize}
\item {\it Fix Molds}: At the end of a 
\end{spellitemize}

\end{spellitemize}

\subsection{Manipulate Fire}\label{S:ManipulateFire}


\subsection{Manipulate Water}\label{S:ManipulateWater}