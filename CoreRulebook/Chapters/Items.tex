\documentclass[../CoreRulebook.tex]{subfile}

\chapter{Items}

Items are, as you would expect, physical objects that you can have in your possession. Items may be stored in one of two places: in your backpack, or on your person, in which case they are said to be equipped. 

\section{Equipped Items}

An item that is equipped can be used immediately. In combat, this would count as your major action. Simply tell your GM that you are using a certain item, and you may then carry out the effect that the item has. 

Some items must be equipped before they can be used; you can't whack someone with your magical sword, if your magical sword is in your bag, after all. Generally speaking, getting items out of storage is not a major action; you may retrieve and then use a health potion in a single motion, for example. Some items, however, might take longer to equip: strapping on a suit of armour, for instance, clearly takes some time!

You only have a finite number of `slots' that an item can be equipped into. Generally speaking, items fall into one of the following major categories:

\begin{itemize}
\item Headwear
\item Jacket / robe 
\item Trousers
\item Shoes
\item Jewellery
\item Right hand held item 
\item Left hand held item
\end{itemize}

You can only have 1 of each item class equipped into these slots at any given time (with the exception of jewellery: you may wear one necklace, two earings and 2 rings). If you want to equip an item into a slot, you must first unequip any items that already occupy that slot. If an item has an effect, it must be equipped for that effect to be used (unless otherwise explicitly stated). 



\section{Storing Items}

Items that are not currently equipped are stored in your backpack, which you should probably try to keep on you at all times. Losing it would be bad!

Unlike most other RPGs, in this game the weight of the items in your backpack is not a strict numerical figure that is kept track of (that gets a bit dull after a while), but your GM may enquire if it is reasonable for you to be carrying 13 different cauldrons around with you, along with 50 tonnes of gold. If you cannot justify how your character is able to move the equipment around, you may be forced to jettison some equipment until you can justify it. Sturdy, enchanted backpacks are your friend!

Items may be transferred between members of a party at any time, if they are within 1m (or you may use a spell such as accio). In combat, switching an item counts as a major action for both characters. 

\section{Physical Weapons}
 
Magical combat is covered in detail elsewhere in this guide, but what happens when you just want to hit the bad guys with big sticks? Most wizards are inexperienced in the art of physical combat, but those with the {\it Brawler} and {\it Archer} skills can attack people with their fists, with steel, or with longer ranged weapons. 

Physical combat is underrated in the magical world, but it can be used to devastating effect. When you have moved in close enough to someone, they do not have the time or room to cast an effective counterspell, and attempts to do so trigger an `attack of opportunity'. Hence, your enemy is effectively at the mercy of you and your big stick...unless they have one of their own. In addition to this, many magical defences do not defend against physical objects, so throwing a rock through a shield charm can often be a good tactic.

Physical weapons come in two types: melee , and ranged. Melee weapons are close-quarters weapons like swords, daggers and so on, and can only be used within a 1m radius of the target. Ranged weapons are bows and arrows and even guns, and can be used from larger distances. 

Weapon usage does not cost any Fortitude points, and so is often a last resort if your character has no more magic spells remaining. 

\subsection*{Melee Weapons}

To perform a melee attack, you must have the item equipped in one of your hands (or both)\footnote{There is a 2 point penalty on any checks for weapons in your non-dominant hand} and be stood adjacent to the target. Some weapons (such as spears and battleaxes) have a longer reach. 

Melee weapons are so simple that they are automatically assumed to hit their target, unless the target is actively dodging, in which case the usual evasion rules apply. Unarmed strikes do 1HP of damage, and strikes with weapons use a specified weapon check (usually an ATH (strength) check, with a variable die size). 

Because a melee attack is up close and personal, it does not usually give spellcasters enough time to retaliate with a counterspell. A non-conditional spell will still be cast before you land your blow, however, though it will trigger an attack of opportunity on the spellcaster. 

All melee weapons can be used from the beginning of the game -- however you are not considered proficient in them until you have the relevant {\it Brawler} skill. Using weapons that you are not proficient in means that you cannot apply any positive modifiers (and negative weapon modifiers are doubled) on all weapon-related checks (included evasion and anti-evasion checks), and always open you up to attacks of opportunity. 

The table below gives a rough overview of the weapons available, and how other effects. 

\subsection*{Ranged Weapons}

Unlike melee weapons, missing the target entirely is a rather real prospect. Ranged weapons cannot be used on any target any closer than 5m, and you need to have the Archer skill to make use of long ranged weapons. 

After selecting your target, you must then check if the projectile hits its target. The projectile check uses a varying dice depending on the level of the Archery skill. The base level Archery skill gets you a 1d4 dice to use. The projectile hits its target if the distance to the target is \textbf{less than 5 times the dice roll} 

Therefore if you roll a 6 to hit a target that is 30 metres away, the projectile misses, as $6 \times 5 = 30$ m, and we need the dice roll to be \textbf{larger}. If the target had been 1 metre closer, it would indeed have succeeded. 

Increasing the Archery skill gets you access to larger dice, and hence increases the distance that you can reach, and the liklihood of success at lower distances.  If the projectile accuracy check succeeds, the relevant evasion checks are applied, and then the damage check is performed to determine how much damage is done. 

\subsection*{Weapon Types \& Improvements}

The table on the next page gives the statistics for a handful of the most common weapon types, including the generalised damage checks. 

However, there are of course different qualities of weapons -- a finely crafted sword is going to be a more formiddable weapon than a hastily thrown together blade. Different materials can also hold an edge for longer, and hence do more damage, and last longer. 

The weapon list is given assuming the weapon is a base-level iron weapon. Use the following table to account for better (or worse) quality weapons. Weapon damage cannot go below 0. 
\def\y{2.6}
\small
\begin{center}
\begin{rndtable}{|c c c m {\y cm}|}
\hline
\bf Material & \bf Damage & \bf Blunting & \bf Notes
\\
Wood & -3  & 10 uses  & \parbox[t]{\y cm}{\raggedright Illusion magics bind strongly to wood}
\\
Bone & -1  & 20 uses &  \parbox[t]{\y cm}{\raggedright Dark Arts bind strongly to bone}
\\
Iron & +0 & 30 uses & 
\\
Steel & +1 & 50 uses & 
\\
Meteorite-iron & +2 & 100 uses &  \parbox[t]{\y cm}{\raggedright Especially powerful enchantments can be bound to meteorite-iron.}
\\
Adamantium & + 3 & Does not blunt &  \parbox[t]{\y cm}{\raggedright Cannot be forged or enchanted }
\\ 
Silver & +1 & 30 uses & \parbox[t]{\y cm}{\raggedright Does double damage to undead}
\\ 
\end{rndtable}
\end{center}

Other materials and/or bonuses may be introduced as is story appropriate. 

Weapons may also be modified by being enchanted (see below), or having a chemical/potion applied to them, in order to add an extra effect to the weapon. This does not generally affect the other properties of the weapon, with the exception of things such as strong acid, which would obviously impinge the integrity of a metal sword!


\section{Armour}

As discussed in section \ref{S:AC}, wearing armour will help protect your character from taking damage. 

\subsection*{Destroying Armour}

Of course, armour cannot protect you indefinitely -- it will break down at some point. Acid and Piercing damage are the usual way of reducing the armour of an individual. 

If the piercing damage in a {\it single hit} exceed the AC of a given piece of armour, that armour is said to be damaged, and its AC is set to zero until it is repared. You cannot damage more than one piece of armour in a given attack. On the other hand, acid damage is cumulative -- the acid burns away the armour making it steadily worse and worse, until it burns through the armour to your skin.  Each point of acid damage reduces the AC by 1, until it is equal to zero. 

Of course -- it is not just sapients that have `armour', large creatures such as dragons have exceedingly strong scales that act as armour. The exact same rules apply to animal AC as to human AC, with the exception that each 1m$^2$ section of the beaing is considered an individual `piece' of armour. Breaking the armour on the tail, and then attacking the chest will obviously not work!

\subsection*{Restoring Armour}

Armour may be restored to working order by spending 3 hours reparing it (with a repair kit), or by using a suitable magic spell.

You cannot repair your own armour whilst it is being worn: you must take it off, then repair it, and then place it back on. This means that battlefield repairs take3 turns, unless you have an ally who can repair the armour for you in a single turn.  

\onecolumn
\section{Weapon List}
\small
\def\l{4}
\begin{center}
%%WeaponsBegin
 \tablealternate\begin{rndtable}{|c c c c c m {\l cm}|}\hline \tablehead \normalsize \bf Weapon & \normalsize \bf Type & \normalsize \bf Brawler & \normalsize \bf Damage Check & \normalsize \bf Damage Type & \bf Notes\\  \bf Club &  Melee & 2 & 1d4 ATH (Strength)  & Bludgeoning& \parbox[t]{\l cm}{Does not blunt}\\  \bf Dagger &  Melee & 2 & 1d6 ATH (Speed)  & Piercing& \parbox[t]{\l cm}{Can be thrown 10m, damage gets disadvantage}\\  \bf Improvised &  Melee & 2 & 1d2 ATH (Strength)  & Various& \parbox[t]{\l cm}{(i.e. sticks, household objects)}\\  \bf Quarterstaff &  Melee & 2 & 1d6 ATH (Dexterity)  & Bludgeoning& \parbox[t]{\l cm}{Can be used 2 handed (use 2d4 check), does not blunt}\\  \bf Scythe &  Melee & 2 & 1d6 ATH (Speed)  & Slashing& \parbox[t]{\l cm}{}\\  \bf Light Axe &  Melee & 3 & 1d6 ATH (Strength)  & Slashing& \parbox[t]{\l cm}{Can be thrown 5m}\\  \bf Longsword &  Melee & 3 & 2d4 ATH (Strength)  & Slashing& \parbox[t]{\l cm}{}\\  \bf Mace &  Melee & 3 & 1d8 ATH (Strength)  & Bludgeoning& \parbox[t]{\l cm}{Does not go blunt}\\  \bf Rapier &  Melee & 3 & 1d8 FIN (Precision)  & Piercing& \parbox[t]{\l cm}{}\\  \bf Shortsword &  Melee & 3 & 1d6 ATH (Speed)  & Slashing& \parbox[t]{\l cm}{}\\  \bf Spear &  Melee & 3 & 1d8 ATH (Strength)  & Piercing& \parbox[t]{\l cm}{Can be thrown 20m (disadvantage), melee reach 2m}\\  \bf Greataxe &  Melee & 4 & 1d10 ATH (Strength)  & Slashing& \parbox[t]{\l cm}{Two-handed}\\  \bf Greatsword &  Melee & 4 & 2d6 ATH (Strength)  & Slashing& \parbox[t]{\l cm}{Two-handed}\\  \bf Trident &  Melee & 4 & 1d8 ATH (Strength)  & Piercing& \parbox[t]{\l cm}{}\\  \bf Warhammer &  Melee & 4 & 4d4 ATH (Strength)  & Bludgeoning& \parbox[t]{\l cm}{Two-handed}\\  \bf Glaive &  Melee & 5 & 1d20 ATH (Precision)  & Slashing& \parbox[t]{\l cm}{Two-handed, reach 2m}\\  \bf Lance &  Melee & 5 & 1d12 ATH (Precision)  & Piercing& \parbox[t]{\l cm}{Requires mount, reach 2m}\\  \bf Pike &  Melee & 5 & 1d12 ATH (Strength)  & Piercing& \parbox[t]{\l cm}{Two-handed, reach 2m}\\  \bf Whip &  Melee & 5 & 1d4 ATH (Precision)  & Slashing& \parbox[t]{\l cm}{Reach 5m}\\  \bf Blowdart &  Ranged &   & 1d6 FIN (Precision)  & Poison& \parbox[t]{\l cm}{Can be coated in a variety of toxins}\\  \bf Crossbow &  Ranged &   & 1d12 FIN (Precision)  & Piercing& \parbox[t]{\l cm}{Max range 20m, requires bolts}\\  \bf Improvised &  Ranged &   & 1d4 FIN (Precision)  & Various& \parbox[t]{\l cm}{(i.e. thrown rocks)}\\  \bf Longbow &  Ranged &   & 1d20 ATH (Strength)  & Piercing& \parbox[t]{\l cm}{Requires arrows. Minimum strength 15 to use.}\\  \bf Shortbow &  Ranged &   & 1d10 ATH (Strength)  & Piercing& \parbox[t]{\l cm}{Max range 30m, requires arrows}\\  \bf Sling &  Ranged &   & 1d6 FIN (Strength)  & Bludgeoning& \parbox[t]{\l cm}{Max range 20m, can use rocks as ammunition}\\  \hline\end{rndtable} %%WeaponsEnd
\end{center}
\normalsize
\twocolumn
\section{Wands}

All witches and wizards start off with their very own magic wand. The wand chooses the wizard, not the other way around,  so the process for selecting your wand is to roll two d6 successively. The first roll determines the wood your wand is made of, the second determines the core. 

Different materials have an affinity with different kinds of magic, and make casting those spells easier. Wood makes the spell type easier to cast (+1 to checks), and the core reduces the mental strain of casting that class of spell (-1 FP cost). 
 \footnotesize
 \begin{center}

 \begin{rndtable}{|c c c c|}

 \hline
 \bf Roll & \bf Magic School & \bf Wood& \bf Core
 \\
 1 & Defensive & Apple & Pheonix feather
 \\
 2 & Hexes \& Curses & Holly & Dragon heartstring
 \\ 
 3 & Divination & Beech & Unicorn Tail hair
 \\ 
 4 & Transifguration & Oak & Thunderbird feather
 \\ 
 5 & Charms & Hawthorn & Kelpie hair
 \\ 
 6 & Illusion & Hazel & Veela hair
 \\ 
 - & Dark Arts & Human Bone & Dementor Robe
 \\\hline
 \end{rndtable}
 \end{center}
\normalsize
 If your original wand is destroyed or lost, you need to find someone who can sell (or make) you a new one, and perform the selection process anew. 
 
The only way to access the 7th and final category of wand is to have an EVL greater than 8. This then bypasses all other wand selection checks, and your wand is necessarily evil. It should of course be noted that wandmakers aren{\apos}t too happy to sell these evil objects -- you might have to cut a few bits off in order to sufficiently motivate them.  
