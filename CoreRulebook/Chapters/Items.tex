\documentclass[../CoreRulebook.tex]{subfile}

\chapter{Item Basics}

\section{Currency \& the Economy}

The wizarding currency is commonly broken up into 3 coins: the bronze {\bf Knut}\comma{} the silver {\bf Sickle} and the golden {\bf Galleon}. Because the system was designed by goblins \minus{} who have a different intrinsic idea about mental arithmetic \minus{} the coinage has an unusual exchange rate.

\subsubsection{Knut}

The bronze knut\comma{} denoted by the symbol \knut{}\comma{} is the lowest denomination coin in the wizarding world. Typically considered `loose change'\comma{} individual exchanges rarely occur with Knuts\comma{} though a veritable fortune in knuts is estimated to be found down the sides of sofas of the wizarding world. 

\subsubsection{Sickle}

The silver sickle (\sickle{}) is the primary currency used by most wizards. Prices for everyday items are generally listed on the order of tens of sickles. An low\minus{}skilled worker could be expected to be earning around 10 sickles for a full days work. 

There are 29\knut{} in one sickle. 

\subsubsection{Galleon}

The galleon\comma{} \galleon{}\comma{} is the largest denomination of currency\comma{} consisting as it does of 17\sickle{}\comma{} or 493\knut{}. 

Most wizards rarely handle or carry around actual galleons \minus{} purchases that occur on this scale are often directed through Gringotts \minus{} though it is not unheard of for rich wizards to flash their golden coins around town.

\subsection{Muggle Exchange Rate}

The exchange rate between muggle and wizarding currencies can be hard to pin down\comma{} as their respective economies bear very little resemblance to each other. What is scarce in one world is often common in the other. 

However\comma{} since the economic crash in 1929\comma{} Gringotts has agreed to establish a fixed exchange rate. Under the current scheme\comma{} Gringotts will purchase £20 for 50\galleon. This works out to give 1 GBP to be equal to 10 knuts\comma{} or just under £3 to a sickle.



\def\cc{\cellcolor{\tablecolorhead}}
\begin{center}
	\begin{rndtable}{c c c c c}
	~	&	\multicolumn{4}{c}{\bf Value}
	\\
	\multirow{\negTwo}{*}{\cc \bf Coin}	&	\cc \knut	&\cc	\sickle	& \cc \galleon & \cc £
	\\
	\cc	\knut		& 1		&	0.034	&	0.002	&	0.1	
	\\
	\cc\sickle		&29		&	1		&	0.059	&	2.94
	\\
	\cc\galleon		&493	&	17		&	1		&	50
	\\
	\cc  £		&	9.86	&	0.34	&	0.02	&	1
	\end{rndtable}
\end{center}

\subsection{Prices \& Availability}

Many items in this guide are listed with an associated price. This is the `standard purchase price' (SPP)\comma{} and is the price one could expect to pay for the item in a large population centre\comma{} during normal economic times\comma{} without excessive bartering. 

However\comma{} this price may increase or decrease for certain items\comma{} depending on the location and the adventure you are undergoing. 

If\comma{} for example\comma{} you had {\it accidentally} triggered a worldwide famine\comma{} then food items could become exceptionally expensive and cost far more than the SPP. Conversely\comma{} if you manage to rid a local lake of the hippocampus that had been terrorising it\comma{} you may find the bountiful fishing harvest reduces the price of fish for a few days. 

Some items may also simply be unavailable \minus{} either because you are speaking to the wrong person (don't go to a bookstore for potions!)\comma{} because of outside influences\comma{} or simply because the item is so rare that none of the available merchants possess it to sell to you. 

\subsection{Selling \& Bartering}

You may also sell your own found or manufactured items to amenable vendors. Items generally sell for 50\% of their SPP\comma{} and no amount of bartering will raise it to 100\%\comma{} unless you can demonstrate your wares are of a significantly higher quality\comma{} and hence not subject to the `standard' price. 

As with purchasing your items\comma{} your ability to sell is dependent on you finding a willing (even enthusiastic) buyer\comma{} as well as the surrounding economic circumstances. 

Note that since 1692 is has been a crime in the wizarding world to allow magical items to fall into the hands of muggles \minus{} a crime which\comma{} in the most egregious of circumstances\comma{} has a punishment of death. 


\section{Equipped Items}

An item that is equipped can be used immediately. In combat\comma{} this would count as your major action. Simply tell your GM that you are using a certain item\comma{} and you may then carry out the effect that the item has. 

Some items must be equipped before they can be used; you can't whack someone with your magical sword\comma{} if your magical sword is in your bag\comma{} after all. Generally speaking\comma{} getting items out of storage is not a major action; you may retrieve and then use a health potion in a single motion\comma{} for example. Some items\comma{} however\comma{} might take longer to equip: strapping on a suit of armour\comma{} for instance\comma{} clearly takes some time!



\section{Storing Items}

Items that are not currently equipped are stored in your backpack\comma{} which you should probably try to keep on you at all times. Losing it would be bad!

Items may be transferred between members of a party at any time\comma{} if they are within 1m (or you may use a spell such as accio). In combat\comma{} switching an item counts as a major action for both characters. 



\section{Item Weight}

Rather than keeping track of the exact weight of each individual item in your backback\comma{} this game opts for a more free\minus{}form approach to tracking item weight. Each item is categorised as either `Light'\comma{} `Medium'\comma{} `Heavy' or `Very Heavy'. 

A `Light item' can be picked up without thinking. They can typically easily fit into your pocket; a sheaf of paper\comma{} some candles and a wizard's wand are all `light'. 

A `medium' weight item has a reasonable amount of heft to it\comma{} but can be held comfortably without strain; most weapons are categorised as `medium'. 

A `heavy' item requires two hands to carry without strain; medium and heavy armour\comma{} as well as cumbersome objects such as the bludger are classified as `heavy'.

A `very heavy' item cannot be carried by one person alone: multiple individuals are required. A chest full of gold and jewels would be `very heavy'. 
\newpage
\chapter{Weapons \& Wands}
 



\section{Wands}

The most important tool of any witch or wizard is their wand.

Unlike with other items\comma{} you don't get to choose your wand\comma{} since it is well known that the wand chooses the wizard\comma{} not the other way around. The process for selecting your wand is to roll two d6 successively. The first roll determines the wood your wand is made of\comma{} the second determines the core. 

Different materials have an affinity with different kinds of magic\comma{} and make casting those spells easier. Wood makes the spell type easier to cast (+1 to checks)\comma{} and the core reduces the mental strain of casting that class of spell (\minus{}1 FP cost). 
 \footnotesize
 \begin{center}

 \begin{rndtable}{|c c c c|}

 \hline
 \bf Roll & \bf Magic School & \bf Wood& \bf Core
 \\
 1 & Defensive & Apple & Pheonix feather
 \\
 2 & Hexes \& Curses & Holly & Dragon heartstring
 \\ 
 3 & Divination & Beech & Unicorn Tail hair
 \\ 
 4 & Transifguration & Oak & Thunderbird feather
 \\ 
 5 & Charms & Hawthorn & Kelpie hair
 \\ 
 6 & Illusion & Hazel & Veela hair
 \\ 
 \minus{} & Dark Arts & Human Bone & Dementor Robe
 \\\hline
 \end{rndtable}
 \end{center}
 
\normalsize
 
If your original wand is destroyed or lost\comma{} you need to find someone who can sell (or make) you a new one\comma{} and perform the selection process anew. 
 
The only way to access the 7th and final category of wand is to have an EVL greater than 8. This then bypasses all other wand selection checks\comma{} and your wand is necessarily evil. It should of course be noted that wandmakers aren{\apos}t too happy to sell these evil objects \minus{}\minus{} you might have to cut a few bits off in order to sufficiently motivate them.  


\newpage


% \section*{Melee Weapons}

% Magical combat and the use of wands is covered in detail elsewhere in this guide\comma{} but what happens when you just want to hit the bad guys with big sticks? Most wizards are inexperienced in the art of physical combat\comma{} but those with the {\it Brawler} and {\it Archer} skills can attack people with their fists\comma{} with steel\comma{} or with longer ranged weapons. 

% Physical combat is underrated in the magical world\comma{} but it can be used to devastating effect. When you have moved in close enough to someone\comma{} they do not have the time or room to cast an effective counterspell\comma{} and attempts to do so trigger an `attack of opportunity'. Hence\comma{} your enemy is effectively at the mercy of you and your big stick...unless they have one of their own. In addition to this\comma{} many magical defences do not defend against physical objects\comma{} so throwing a rock through a shield charm can often be a good tactic.

% Physical weapons come in two types: melee \comma{} and ranged. Melee weapons are close\minus{}quarters weapons like swords\comma{} daggers and so on\comma{} and can only be used within a 1m radius of the target. Ranged weapons are bows and arrows and even guns\comma{} and can be used from larger distances. 

% Weapon usage does not cost any Fortitude points\comma{} and so is often a last resort if your character has no more magic spells remaining. 

% To perform a melee attack\comma{} you must have the item equipped in one of your hands (or both)\footnote{There is a 2 point penalty on any checks for weapons in your non\minus{}dominant hand} and be stood adjacent to the target. Some weapons (such as spears and battleaxes) have a longer reach. 

% Melee weapons are so simple that they are automatically assumed to hit their target\comma{} unless the target is actively dodging\comma{} in which case the usual evasion rules apply. Unarmed strikes do 1HP of damage\comma{} and strikes with weapons use a specified weapon check (usually an ATH (strength) check\comma{} with a variable die size). 

% Because a melee attack is up close and personal\comma{} it does not usually give spellcasters enough time to retaliate with a counterspell. A non\minus{}conditional spell will still be cast before you land your blow\comma{} however\comma{} though it will trigger an attack of opportunity on the spellcaster. 

% All melee weapons can be used from the beginning of the game \minus{}\minus{} however you are not considered proficient in them until you have the relevant {\it Brawler} skill. Using weapons that you are not proficient in means that you cannot apply any positive modifiers (and negative weapon modifiers are doubled) on all weapon\minus{}related checks (included evasion and anti\minus{}evasion checks)\comma{} and always open you up to attacks of opportunity. 

% The table below gives a rough overview of the weapons available\comma{} and how other effects. 

% \section*{Ranged Weapons}

% Unlike melee weapons\comma{} missing the target entirely is a rather real prospect. Ranged weapons cannot be used on any target any closer than 5m\comma{} and you need to have the Archer skill to make use of long ranged weapons. 

% After selecting your target\comma{} you must then check if the projectile hits its target. The projectile check uses a varying dice depending on the level of the Archery skill. The base level Archery skill gets you a 1d4 dice to use. The projectile hits its target if the distance to the target is \textbf{less than 5 times the dice roll} 

% Therefore if you roll a 6 to hit a target that is 30 metres away\comma{} the projectile misses\comma{} as $6 \times 5 = 30$ m\comma{} and we need the dice roll to be \textbf{larger}. If the target had been 1 metre closer\comma{} it would indeed have succeeded. 

% Increasing the Archery skill gets you access to larger dice\comma{} and hence increases the distance that you can reach\comma{} and the liklihood of success at lower distances.  If the projectile accuracy check succeeds\comma{} the relevant evasion checks are applied\comma{} and then the damage check is performed to determine how much damage is done. 

% \newpage
% \subsection*{Weapon Types \& Improvements}

% The table on the next page gives the statistics for a handful of the most common weapon types\comma{} including the generalised damage checks. 

% However\comma{} there are of course different qualities of weapons \minus{}\minus{} a finely crafted sword is going to be a more formiddable weapon than a hastily thrown together blade. Different materials can also hold an edge for longer\comma{} and hence do more damage\comma{} and last longer. 

% The weapon list is given assuming the weapon is a base\minus{}level iron weapon. Use the following table to account for better (or worse) quality weapons. Weapon damage cannot go below 0. 
% \def\y{2.6}
% \small
% \begin{center}
% \begin{rndtable}{|c c c m {\y cm}|}
% \hline
% \bf Material & \bf Damage & \bf Blunting & \bf Notes
% \\
% Wood & \minus{}3  & 10 uses  & \parbox[t]{\y cm}{\raggedright Illusion magics bind strongly to wood}
% \\
% Bone & \minus{}1  & 20 uses &  \parbox[t]{\y cm}{\raggedright Dark Arts bind strongly to bone}
% \\
% Iron & +0 & 30 uses & 
% \\
% Steel & +1 & 50 uses & 
% \\
% Meteorite\minus{}iron & +2 & 100 uses &  \parbox[t]{\y cm}{\raggedright Especially powerful enchantments can be bound to meteorite\minus{}iron.}
% \\
% Adamantium & + 3 & Does not blunt &  \parbox[t]{\y cm}{\raggedright Cannot be forged or enchanted }
% \\ 
% Silver & +1 & 30 uses & \parbox[t]{\y cm}{\raggedright Does double damage to undead}
% \\ 
% \end{rndtable}
% \end{center}

% \normalsize
% Other materials and/or bonuses may be introduced as is story appropriate. 

% Weapons may also be modified by being enchanted (see below)\comma{} or having a chemical/potion applied to them\comma{} in order to add an extra effect to the weapon. This does not generally affect the other properties of the weapon\comma{} with the exception of things such as strong acid\comma{} which would obviously impinge the integrity of a metal sword!


\onecolumn
\section{Weapon List}\label{S:WeaponList}
\small
\def\l{8}
\begin{center}
%%WeaponsBegin
\begin{rndtable}{|l l c c l |}\hline \normalsize \bf Weapon & \normalsize \bf Cost & \normalsize \bf Modifier &  \normalsize \bf Damage & \normalsize \bf Properties \\ \hline{ \it Unarmed Weapons} & & & & \\ 
\bf ~~~~~Unarmed Strike	&		&	\attPhys	&	1~Bludgeoning	&	\parbox[t]{\l cm}{}\\ 
\bf ~~~~~Improvised Weapons	&		&	?	&	1d4~	&	\parbox[t]{\l cm}{(GM fiat takes precedence: use similarity to existing weapons)}\\ 
{ \it Simple Weapons} & & & & \\ 
\bf ~~~~~Club	&	\sickle{1}~	&	\attPhys	&	1d4~Bludgeoning	&	\parbox[t]{\l cm}{}\\ 
\bf ~~~~~Dagger	&	\sickle{10}~	&	Versatile	&	1d4~Piercing	&	\parbox[t]{\l cm}{Can be thrown, range: 5m}\\ 
\bf ~~~~~Quarterstaff	&	\sickle{4}~	&	Versatile	&	1d6~Bludgeoning	&	\parbox[t]{\l cm}{Multi-handed (1d8)}\\ 
\bf ~~~~~Spear	&	\sickle{10}~	&	\attPhys	&	1d8~Piercing	&	\parbox[t]{\l cm}{Can be thrown, range: 10m}\\ 
{ \it Bladed Weapons} & & & & \\ 
\bf ~~~~~Greatsword	&	\galleon{6}~	&	\attPhys	&	2d6~Slashing	&	\parbox[t]{\l cm}{Two-handed}\\ 
\bf ~~~~~Longsword	&	\galleon{5}~	&	\attPhys	&	2d4~Slashing	&	\parbox[t]{\l cm}{}\\ 
\bf ~~~~~Rapier	&	\galleon{3}~	&	\attFin	&	1d8~Piercing	&	\parbox[t]{\l cm}{}\\ 
\bf ~~~~~Shortsword	&	\galleon{3}~	&	Versatile	&	1d6~Slashing	&	\parbox[t]{\l cm}{}\\ 
{ \it Brutish Weapons} & & & & \\ 
\bf ~~~~~Greataxe	&	\galleon{2}~	&	\attPhys	&	1d12~Slashing	&	\parbox[t]{\l cm}{Two-handed}\\ 
\bf ~~~~~Light Axe	&	\galleon{1}~	&	\attPhys	&	1d6~Slashing	&	\parbox[t]{\l cm}{Can be thrown, range: 5m}\\ 
\bf ~~~~~Mace	&	\galleon{1}~	&	\attPhys	&	1d6~Bludgeoning	&	\parbox[t]{\l cm}{}\\ 
\bf ~~~~~Warhammer	&	\galleon{3}~	&	\attPhys	&	2d4~Bludgeoning	&	\parbox[t]{\l cm}{Two-handed}\\ 
{ \it Reach Weapons} & & & & \\ 
\bf ~~~~~Glaive	&	\galleon{2}~\sickle{10}~	&	\attPhys	&	2d6~Slashing	&	\parbox[t]{\l cm}{Two-handed, reach 2m}\\ 
\bf ~~~~~Lance	&	\galleon{2}~\sickle{5}~	&	\attPhys	&	1d12~Piercing	&	\parbox[t]{\l cm}{Requires mount, reach 2m}\\ 
\bf ~~~~~Pike	&	\galleon{1}~\sickle{10}~	&	\attPhys	&	1d10~Piercing	&	\parbox[t]{\l cm}{Two-handed, reach 2m}\\ 
{ \it Exotic Weapons} & & & & \\ 
\bf ~~~~~Scythe	&	\sickle{10}~	&	Versatile	&	1d4~Slashing	&	\parbox[t]{\l cm}{}\\ 
\bf ~~~~~Trident	&	\galleon{1}~\sickle{10}~	&	Versatile	&	1d8~Piercing	&	\parbox[t]{\l cm}{}\\ 
\bf ~~~~~Whip	&	\sickle{10}~	&	\attFin	&	1d4~Slashing	&	\parbox[t]{\l cm}{Reach 5m}\\ 
\bf ~~~~~Chakram	&	\galleon{2}~	&	\attFin	&	2d4~Slashing	&	\parbox[t]{\l cm}{Max range 200m.}\\ 
\bf ~~~~~Fan	&	\sickle{8}~	&	\attFin	&	1d6~Slashing	&	\parbox[t]{\l cm}{}\\ 
\bf ~~~~~Net	&	\sickle{8}~	&	Versatile	&	~	&	\parbox[t]{\l cm}{Applies {\it Incapacitated} status on a failed DV10 Strength Resist check. Can be thrown: range 5m.}\\ 
{ \it Simple Ranged Weapons} & & & & \\ 
\bf ~~~~~Blowdart	&	\knut{5}	&	\attFin	&	1d4~Poison	&	\parbox[t]{\l cm}{Range: 10m. Ammuniion: Darts}\\ 
\bf ~~~~~Sling	&	\sickle{2}~	&	\attFin	&	1d4~Bludgeoning	&	\parbox[t]{\l cm}{Max range: 50m (rocks), 100m (lead shot). Ammunition: lead shot, or improvised.}\\ 
{ \it Ranged Weapons} & & & & \\ 
\bf ~~~~~Crossbow	&	\galleon{4}~	&	\attFin	&	1d12~Piercing	&	\parbox[t]{\l cm}{Max range 20m. Ammunition: Bolts. Reload time:1 turn.}\\ 
\bf ~~~~~Longbow	&	\galleon{2}~	&	Versatile	&	2d6~Piercing	&	\parbox[t]{\l cm}{Max range: 150m. Use a \attFinShort{} check to aim, but \attPhysShort{} for damage check. Ammunition: Arrows.}\\ 
\bf ~~~~~Shortbow	&	\galleon{1}~	&	\attFin	&	1d6~Piercing	&	\parbox[t]{\l cm}{Max range 30m, Ammunition: Arrows.}\\ 
{ \it Firearms Weapons} & & & & \\ 
\bf ~~~~~Pistol	&	\galleon{8}~	&	\attFin	&	2d12~Piercing	&	\parbox[t]{\l cm}{Max range: 30m (accurate). Ammunition: Bullets. Cartridge: 8, reload time: 1 turn.}\\ 
\bf ~~~~~Rifle	&	\galleon{12}~	&	\attFin	&	5d6~Piercing	&	\parbox[t]{\l cm}{Max range: 40m (standing), 100m (standing, 2 turn aim), 500m (prone, 3 turn aim). Ammunition: Bullets, Cartridge: 1, reload time: 1 turn.}\\ 
\bf ~~~~~Shotgun	&	\galleon{16}~	&	\attFin	&	10d4~Piercing	&	\parbox[t]{\l cm}{Max range: 10m (full damage), 1d4 removed for every subsequent metre. Ammunition: Bullets, Cartridge: 2, Reload time: 2 turns.}\\ 
\hline\end{rndtable} %%WeaponsEnd
\end{center}
\normalsize
\twocolumn


\chapter{Clothing \& Armour} \label{S:Armour}

The clothing and protective gear you wear can have a dramatic impact on your ability to defend yourself\comma{} or run away from problems. 

\section{Wearing Armour}

\subsection{Outfits}

Wearing thicker armour protects you\comma{} by increasing your {\it Block} statistic by a specified amount. Most sets of clothing are considered to come in a `full set'\comma{} and thus cover the entire torso\comma{} arms\comma{} legs\comma{} feet \minus{} and possibly comes with some headwear. 

For the sake of simplicity\comma{} you are generally discouraged from `mix and matching' various types of armour. You are allowed to switch out various pieces of armour for magical equivalents\comma{} or simply for a cooler aesthetic. However\comma{} your Block value is determined by whatever type of protection you are wearing {\it most} of \minus{} and if in doubt\comma{} the lower value will be used. 

If Gunter the half\minus{}giant wishes to wear a full suit of knight's armour\comma{} but swap the gloves out for her cotton {\it Gloves of Pugilism}\comma{} she can do so without altering the total Block value. However\comma{} if she also swapped out the helmet for a jaunty hat\comma{} and the footwear for some running shoes\comma{} the GM may step in and decree a penalty to her Block statistic. 

\def\y{1.3}
\def\w{3.3}
\def\x{2.6}
\def\u{0.8}
\newcommand\armour[4]
{
\\
	\parbox[t]{\y cm}{\raggedright\textbf{\footnotesize\textit{#1}}} & \parbox[t]{\w cm}{\footnotesize#2} &  \parbox[t]{\x cm}{\footnotesize\raggedright #3}	&	\parbox[t]{\u cm}{\centering #4}
	
}

\subsection{Proficiencies}

Armour comes in 4 categories: clothing\comma{} light armour\comma{} medium armour and heavy armour\comma{} in order of increasing protection. 

he first two of these (clothing \& light armour) can be worn by anyone\comma{} without penalty. However\comma{} wearing medium or heavy armour requires skill to be able to do\comma{} without it becoming a severe distraction. These armours require you to be proficient (either through a class bonus\comma{} or through the relevant Skill). If you attempt to wear armour you are not proficient in\comma{} you take the {\it Encumbered} status effect and check\minus{}disadvantage on any accuracy checks made.

\newpage
\subsection{Clothing}

Everyday clothes offer no additional protection against the attacks of malevolent forces. It is\comma{} however\comma{} comfy and easy to wear. 

You require no proficiencies in order to wear clothing. 

\small
\begin{center}
\begin{rndtable}{p{\y cm} p{\w cm} p{\x cm} p{\u cm}}
\bf Type   &	\bf Description	&	\bf Effect	& \bf Cost

\armour{Casual outfit}{Jeans and a t\minus{}shirt. Cheap\comma{} comfy and practical}{No effect}{\sickle{10}}
\armour{Formal Wear}{Extra suave look for the discerning witch or wizard. Ball gowns and tuxedoes are impractical\comma{} but you look amazing!}{\minus{}2 Dodge\comma{} \\ +2 Charisma}{\galleon{2}}
\armour{Sports clothes}{Specially designed clothing for taking part in physical activity.}{+2 Dodge}{\galleon{1}}
\armour{Wizards Robes}{Once the everyday clothes of all wizardkind\comma{} now usually seen as the typical school uniform of a Hogwarts student.}{+1 to spellcasting checks}{\sickle{7}}

\end{rndtable}
\end{center}
\normalsize
\subsection{Light Armour}

Light armour is the crossing point between what we typically think of as armour (knights clanking around in metal)\comma{} and everyday clothes. Light and flexible\comma{} it grants only limited protection. 

You require no proficiencies in order to wear light armour. 

\small
\begin{center}
\begin{rndtable}{p{\y cm} p{\w cm} p{\x cm} p{\u cm}}
\bf Type   &	\bf Description	&	\bf Effect	& \bf Cost
\armour{Padded}{Formed from multiple layers of soft fabric and padding}{+2 Block\comma{} \\\minus{}1 Dodge\comma{} \\ Conspicuous}{\sickle{25}}
\armour{Leather Jacket}{A simple leather jacket offers a surprising amount of protection. Plus it looks cool.}{+1 Block}{\sickle{10}}
\armour{Warded Cloth}{A recent magical invention\comma{} this expensive material hardens on impact\comma{} providing extra protection\comma{} whilst not impeding your movement.}{+2 Block}{\galleon{12}}
\end{rndtable}
\end{center}
\normalsize

\newpage
\subsection{Medium Armour}

\small
\begin{center}
\begin{rndtable}{p{\y cm} p{\w cm} p{\x cm} p{\u cm}}
\bf Type   &	\bf Description	&	\bf Effect	& \bf Cost
\armour{Bulletproof Vest}{A muggle invention\comma{} this weaved kevlar material offers a good amount of protection.}{+3 Block\comma{}\\ \minus{}1 Dodge\comma{}\\ Resistance to Ranged Weapon attacks}{\galleon{3}}
\armour{Hardened Furs}{A primitive\minus{}appearing armour often worn by giants and other isolated peoples. Layers of hardened leather and treated hides protects against the cold\comma{} as well as from weapons.}{+2 Block\comma{}\\ \minus{}1 Dodge \\ Resistance to Cold damage}{\sickle{15}}
\armour{Tactical Armour}{The armour of the Auror class\comma{} thought to strike the correct balance between hardened and fortified plates inserted between layers of flexible fabric.}{+4 Block \\ \minus{}2 Dodge\comma{} \\ Conspicuous}{\galleon{8}}
\armour{Warrior Robe}{Magical armies are rare\comma{} but Battlemages often wore specially warded robes which offered improved protection\comma{} though hampered movement.}{+3 Block\comma{}\\\minus{}1 Dodge}{\galleon{3}}
\end{rndtable}
\end{center}
\normalsize

\subsection{Heavy Armour}


\small
\begin{center}
\begin{rndtable}{p{\y cm} p{\w cm} p{\x cm} p{\u cm}}
\bf Type   &	\bf Description	&	\bf Effect	& \bf Cost
\armour{Bomb Suit}{Specially designed suit that one must climb inside. Used by professionals who frequently find themselves at risk of incineration or detonation}{+5 Block\comma{}\\ \minus{}6 Dodge \\ Resistance to Fire \& Concussive damage\comma{} \\Conspicuous.}{\galleon{15}}
\armour{Runic Mail}{Enchanted scales of metal fit together to provide full physical and magical protection over your body\comma{}.}{+7 Block\comma{}\\ \minus{}5 Dodge\comma{} }{\galleon{100}}
\armour{Steel Plate}{It is said that modern problems require modern solutions. Steel plate is proof that\comma{} maybe\comma{} this isn't always the case}{+4 Block\comma{} \\ \minus{}5 Dodge\comma{} \\Conspicuous\comma{} \\ Resistance to Piercing \& Slashing damage.}{\galleon{10}}
\armour{Special Response Set}{The bigger\comma{} badder brother of the Tactical armour. Used only when overwhelming firepower needs to be withstood\comma{} as it is much more cumbersome}{+5 Block\comma{} \\ \minus{}4 Dodge \\ Conspicuous}{\galleon{12}}
\end{rndtable}
\end{center}
\normalsize

\newpage

\section*{Damaging Armour}\label{S:DestroyArmour}

Of course\comma{} armour is not a panacea\comma{} and it cannot protect the squishy meat inside indefinitely. 

When a {\it Critical Strike} is performed with one of the damage types mentioned in the table below\comma{} the attacker may choose to forgo inflicting damage and instead damage the armour of the target. 

\begin{center}
\begin{rndtable}{ c c}
\bf Damage Type	&	\bf Armour Damage
\\
Acid	&	1d4
\\
Bludgeoning	&	1d2
\\
Piercing	&	1d4
\\
Slashing	&	1d2
\end{rndtable}
\end{center}

Roll the associated {\it Armour Damage Dice} for the damage type\comma{} and deduct this total from the current Block bonus provided by the being's protective layer. This is a permanent deduction in the Block statistic\comma{} until the armour is repaired. 


If the block\minus{}bonus reaches zero\comma{} the armour is considered `destroyed'\comma{} and is automatically `de\minus{}equipped' as it falls to shreds around you. 

\section*{Restoring Armour}

Damaged Armour may be restored by spending 1 hours repairing it (with a repair kit) for one hour per {\it Block} bonus that must be restored\comma{} or by using a suitable magic spell.

Armour that has been `destroyed' cannot be repaired without proficiency with a {\it repair kit}.


\chapter{Adventuring Gear}

Adventuring gear is the set of (usually non\minus{}magical) items that you would need to use to survive on a day\minus{}to\minus{}day basis on an adventure.
\newcommand\generic[2]
{
	\vspace{2 ex}
	\small
	\vbox{
	{\bf #1:}~~#2
	}
	\normalsize
}


\newcommand\pack[3]
{
	\vspace{2 ex}
	\small
	\vbox{
	{\bf #1}
	
	\begin{tabular}{l p{6cm}}
		Cost: & #2 gold
	\\
		Contains: & #3
	\end{tabular}
	}
	\normalsize
}


\def\q{1.6}
\newcommand\itemEntry[3]
{
	 \parbox[t]{\q cm}{\bf \raggedright \footnotesize #1}	&	#2	&#3
}

\newcommand\doubleRow[2]
{
	\itemEntry#1 & ~ & \itemEntry#2 \\
}
\newcommand\singleRow[1]
{
	\itemEntry#1 \\
}

\def\cw{0.6}
\def\t{0.01}

\footnotesize

 \begin{center}\begin{rndtable}{|p{\q cm} l p{\cw cm} p{\t cm} p{\q cm} p{\cw cm} l|}\hline \tablehead 
 \bf Name & \bf Weight & \bf Cost & ~ &\bf Name & \bf Weight & \bf Cost\\ 
  \hline
 %%ItemsBegin
\doubleRow{{ Acid}{Light}{\sickle{4}~}}{{ Jewellery (luxurious)}{Light}{\galleon{20}~}} 
\doubleRow{{ Arrows (10)}{Light}{\sickle{4}~}}{{ Ladder (2m)}{Heavy}{\sickle{15}~}} 
\doubleRow{{ Backpack (25L)}{Light}{\sickle{8}~\knut{15}}}{{ Lamp}{Light}{\sickle{2}~}} 
\doubleRow{{ Backpack (65L)}{Medium}{\galleon{1}~\sickle{5}~}}{{ Magnifying Glass}{Light}{\sickle{2}~}} 
\doubleRow{{ Ball bearings}{Light}{\sickle{1}~\knut{20}}}{{ Manacles (Iron)}{Medium}{\sickle{10}~}} 
\doubleRow{{ Bedroll}{Light}{\sickle{5}~}}{{ Mirror (handheld)}{Light}{\sickle{2}~}} 
\doubleRow{{ Blowgun Needles (20)}{Light}{\sickle{2}~}}{{ Oil (flask of)}{Light}{\sickle{1}~}} 
\doubleRow{{ Caltrops}{Medium}{\sickle{6}~\knut{25}}}{{ Paper (20 sheets A4)}{Light}{\knut{10}}} 
\doubleRow{{ Candle}{Light}{\knut{9}}}{{ Parchment (5 sheets A3)}{Light}{\sickle{2}~}} 
\doubleRow{{ Case (map or scroll)}{Light}{\sickle{3}~\knut{10}}}{{ Perfume (vial)}{Light}{\sickle{10}~}} 
\doubleRow{{ Chain (5m)}{Heavy}{\sickle{12}~}}{{ Pole (3 m)}{Light}{\sickle{8}~\knut{15}}} 
\doubleRow{{ Chalk}{Light}{\knut{19}}}{{ Potion: Antidote}{Light}{\galleon{1}~}} 
\doubleRow{{ Chest}{Heavy}{\galleon{1}~}}{{ Potion: Pepper\minus{}Up}{Light}{\sickle{10}~\knut{5}}} 
\doubleRow{{ Crossbow Bolts (10)}{Light}{\sickle{4}~}}{{ Potion: Poison}{Light}{\galleon{2}~}} 
\doubleRow{{ Crowbar}{Medium}{\sickle{3}~\knut{10}}}{{ Potion: Wiggenweld}{Light}{\sickle{10}~\knut{5}}} 
\doubleRow{{ Drinking Flask}{Light}{\sickle{3}~\knut{10}}}{{ Rations (1 day)}{Light}{\sickle{2}~\knut{20}}} 
\doubleRow{{ Firearm Bullets (20)}{Light}{\sickle{10}~}}{{ Rope (20m)}{Light}{\sickle{4}~\knut{20}}} 
\doubleRow{{ Fishing Rod}{Light}{\sickle{13}~\knut{15}}}{{ Sack}{Light}{\knut{19}}} 
\doubleRow{{ Floo Powder Pouch}{Light}{\galleon{1}~}}{{ Shovel}{Light}{\sickle{4}~}} 
\doubleRow{{ Food (1 warm meal)}{Light}{\sickle{3}~\knut{10}}}{{ Slingshot Bullets (10)}{Light}{\sickle{1}~}} 
\doubleRow{{ Glass Vial}{Light}{\sickle{2}~\knut{20}}}{{ Soap}{Light}{\knut{19}}} 
\doubleRow{{ Hammer}{Medium}{\sickle{5}~}}{{ Tea Set}{Light}{\sickle{1}~\knut{20}}} 
\doubleRow{{ Holy Water}{Light}{\galleon{2}~}}{{ Tent (two\minus{}person)}{Light}{\galleon{1}~\sickle{5}~}} 
\doubleRow{{ Hourglass}{Light}{\sickle{1}~\knut{20}}}{{ Tinderbox}{Light}{\sickle{1}~}} 
\doubleRow{{ Hunting Trap}{Light}{\galleon{1}~}}{{ Torch}{Light}{\knut{19}}} 
\doubleRow{{ Ingredient Pouch}{Light}{\sickle{5}~}}{{ Whetstone}{Light}{\sickle{1}~\knut{20}}} 
\doubleRow{{ Ink}{Light}{\knut{19}}}{{ ~}{~}{~}} 
\doubleRow{{ Ink pen}{Light}{\knut{19}}}{{ ~}{~}{~}} 
\doubleRow{{ Jewellery (cheap)}{Light}{\sickle{10}~}}{{ ~}{~}{~}} 
\doubleRow{{ Jewellery (fine)}{Light}{\galleon{3}~}}{{ ~}{~}{~}} 
\hline
\end{rndtable}
\end{center} 

 
 \generic{Acid}{May be splashed on a melee opponent\comma{} or used as an improvised Ranged weapon\comma{} following the normal rules. In either case\comma{} do 3d4 acid damage.}
 
 \generic{Backpack (25L)}{The primary storage for most adventurers. A medium\minus{}sized backpack suitable for adventuring\comma{} with a volume of around 25L}
 
 \generic{Ball bearings}{As a major action\comma{} spill these on the floor covering up to 5 square metres. Any creature passing through this region must succeed on a DV10 FIN Resist check\comma{} or fall prone.}
 
 \generic{Bedroll}{Comfy enough to get a decent night\apos{}s sleep on when out on an adventure.}
 
 \generic{Caltrops}{As a major action\comma{} spill these on the floor covering up to 2 square metres. Any creature passing through this region must succeed on a DV12 FIN check\comma{} or stop moving and take 1d4 piercing damage.}
 
 \generic{Candle}{For 1 hour\comma{} shed bright light 1m radius\comma{} and dim light for a further 1m.}
 
 \generic{Case (map or scroll)}{Safely protects up to 10 large sheets of paper from the elements.}
 
 \generic{Chain (5m)}{A set of large metal links. Can be broken by a DV18 ATH (Strength) check\comma{} or by taking more than 10 physical damage.}
 
 \generic{Chest}{A large wooden structure\comma{} bound with iron bars. Useful for storage\comma{} with an interior volume of 150L.}
 
 \generic{Crowbar}{Gives advantage on Strength checks when leverage can be applied.}
 
 \generic{Drinking Flask}{Contains enough water for one person for one day}
 
 \generic{Floo Powder Pouch}{Can be used to navigate from one fireplace on the Floo network to another. Each pouch contains enough powder for 5 journeys.}
 
 \generic{Holy Water}{May be splashed on a melee opponent\comma{} or used as an improvised Ranged weapon\comma{} following the normal rules. In either case\comma{} do 3d4 Celestial damage.}
 
 \generic{Hunting Trap}{Requires 2 major actions to set\comma{} and forms a ring 0.5m in radius. Any creature that steps into this ring must succeed a DV15 FIN(Speed) check\comma{} or become Trapped\comma{} and taking 1d4 piercing damage. Trap may be broken via a DV10 ATH(Strength) check\comma{} but each failed attempt does a further 1d4 piercing damage.}
 
 \generic{Ingredient Pouch}{Used to keep potion ingredients safe from the elements.}
 
 \generic{Lamp}{For 6 hours\comma{} casts a bright light in a 4m radius\comma{} and dim light for a further 3m.}
 
 \generic{Manacles (Iron)}{Can be broken via a DV15 ATH(Strength check)\comma{} but otherwise immobilises the hands of the wearer.}
 
 \generic{Oil (flask of)}{Contains enough oil to refill a lamp once.}
 
 \generic{Potion: Antidote}{Cures up to 5 points of poison damage.}
 
 \generic{Potion: Pepper\minus{}Up}{Restores 10FP}
 
 \generic{Potion: Poison}{Does 5 Poison damage per turn for 5 turns.}
 
 \generic{Potion: Wiggenweld}{Restores 10HP}
 
 \generic{Rations (1 day)}{Not particularly nourishing\comma{} but enough to fill you up and keep you alive and kicking.}
 
 \generic{Tea Set}{Contains all the ingredients to make a decent cup of tea}
 
 \generic{Tinderbox}{Contains a flint and some tinder\comma{} necessary to create a non\minus{}magical fire.}
 
 \generic{Torch}{Burns for 1 hour\comma{} casting bright light for 2m\comma{} and dim light for a further 2. May be used as an improvised weapon\comma{} where it does an additional 1d4 fire damage.}
 
 \generic{Whetstone}{Useful for sharpening a dulled weapon.}%%ItemsEnd
\normalsize


\section{Artefacts}

Artefacts are items which are more magical in nature\comma{} and generally cannot be synthesised directly\comma{} though they may be recreated through enchanting. Some artefacts are incredibly rare and powerful\comma{} and can be hard to track down. The list below contains only some artefacts which are commonly found in the wizarding world\comma{} and should by no means be thought of as extensive. 
\def\q{4}

 \begin{center}\begin{rndtable}{|p{\q cm} l l |}\hline \tablehead \bf Name & \bf Weight & \bf Cost \\ \hline 	  
 
 %%ArtefactBegin
\singleRow{{Bludger}{Heavy}{\galleon{5}~}}
\singleRow{{Broomstick (cheap)}{Medium}{\galleon{40}~}}
\singleRow{{Broomstick (fine)}{Medium}{\galleon{500}~}}
\singleRow{{Crystal Ball}{Light}{\galleon{1}~}}
\singleRow{{Darkandles}{Light}{\sickle{4}~}}
\singleRow{{Deluminator}{Light}{\galleon{60}~}}
\singleRow{{Extending Stachel}{Light}{\galleon{100}~}}
\singleRow{{Gobstone (Set of 30)}{Light}{\sickle{5}~}}
\singleRow{{Golden Snitch}{Light}{\galleon{10}~}}
\singleRow{{Howler}{Light}{\sickle{1}~}}
\singleRow{{Invisibility Cloak}{Medium}{\galleon{240}~}}
\singleRow{{Mokeskin Pouch}{Light}{\galleon{1}~}}
\singleRow{{Obsidian Manacles}{Medium}{\galleon{150}~}}
\singleRow{{Omnioculars}{Light}{\sickle{15}~}}
\singleRow{{Pensieve}{Heavy}{\galleon{400}~}}
\singleRow{{Portkey}{(Varies)}{\galleon{16}~}}
\singleRow{{Quick\minus{}Quotes Quill}{Light}{\galleon{1}~\sickle{10}~}}
\singleRow{{Rememberall}{Light}{\galleon{1}~}}
\singleRow{{Self\minus{}Erecting Tent}{Heavy}{\galleon{24}~}}
\singleRow{{Sneakoscope}{Light}{\galleon{1}~}}
\singleRow{{Spellotape}{Light}{\sickle{4}~}}
\singleRow{{Talking Portrait}{Heavy}{\galleon{15}~}}
\singleRow{{Time\minus{}Turner}{Light}{\galleon{100000}~}}
\singleRow{{Wand}{Light}{\galleon{6}~}}
\hline
\end{rndtable}
\end{center} 

 
 \generic{Bludger}{An enchanted iron ball\comma{} weighing approximately 80kg\comma{} and yet able to fly. They possess a malicious streak\comma{} and will target any flying entity within 100m and attempt to smash into them\comma{} before moving onto their next target\comma{}}
 
 \generic{Broomstick (cheap)}{A low\minus{}range broomstick that can get off the ground\comma{} but not much more than that. Capable of carrying one passenger at speeds of up to 100mph\comma{} thought with very clumsy handling at high speeds.}
 
 \generic{Broomstick (fine)}{A high\minus{}end broomstick capable of high\minus{}speed precision flying. Capable of carrying one passenger at speeds of up to 250mph\comma{} with the handling only limited by the pilot\apos{}s rection time.}
 
 \generic{Crystal Ball}{A sphere of perfect crystal � the manufacture of these objects is a carefully curated secret. A properly trained mind can use a crystal ball to peer through the mystic veil and learn about the universe.}
 
 \generic{Darkandles}{The exact opposite of a candle\comma{} a darkandle emits darkness\comma{} rather than light. Within a 5m radius\comma{} there is total darkness\comma{} and within 5m there is only dim light\comma{} regardless of any other light sources nearby.}
 
 \generic{Deluminator}{A device designed by Dumbledore\comma{} the deluminator may be targeted at a specific light\minus{}source\comma{} at which point the light is extinguished\comma{} and absorbed by the deluminator. This light source is permanently disabled until the deluminator restores the light to it.}
 
 \generic{Extending Stachel}{A normal backpack that has been enchanted with the {\it internal extension charm}\comma{} making it several times larger on the inside than the outside. This backpack can hold up to 300L\comma{} and makes the contents 10 times lighter than normal.}
 
 \generic{Gobstone (Set of 30)}{A small set of stones used in the titular game. When knocked by another gobstone\comma{} they eject a squirt of corrosive liquid into the eyes of the target.}
 
 \generic{Golden Snitch}{A small golden\comma{} metal orb which sprouts wings when activated. The snitch then immediately attempts to evade all living beings\comma{} though sometimes it will taunt them by floating in front of their faces\comma{} before quickly vanishing.}
 
 \generic{Howler}{A magical letter that\comma{} when opened by the recipient\comma{} unfolds itself\comma{} floats and begins to scream the enclosed message\comma{} before incinerating itself. If left unopened after being delivered\comma{} it will explode violently.}
 
 \generic{Invisibility Cloak}{A cloak that renders whatever is concealed within it invisible\comma{} though external factors such as sound\comma{} or the presence of rain or snow may still give away the location. The cloak also offers no protection against spells.}
 
 \generic{Mokeskin Pouch}{A small coin pouch that can only be accessed by its owner.}
 
 \generic{Obsidian Manacles}{A magical set of handcuffs which\comma{} when firmly closed around the wrist\comma{} prevent a witch or wizard from casting any magic.}
 
 \generic{Omnioculars}{A pair of high\minus{}resolution binoculars\comma{} capable of pausing\comma{} rewinding and replaying previous events. Omnioculars are capable of 10x magnification\comma{} and a playback up to 3x slower than original of up to 1 hour of recorded footage.}
 
 \generic{Pensieve}{A large stone bowl\comma{} engraved with ancient runes and inlaid with previous stones. A pensieve can be filled with memories\comma{} in the form of a silvery glowing fluid\comma{} which can then be viewed and relived in real time.}
 
 \generic{Portkey}{A one\minus{}use device with a teleportation charm embedded in it. A portkey resembles a random piece of junk\comma{} but when activated (either by touch\comma{} or at a specific time)\comma{} teleports to a preset location.}
 
 \generic{Quick\minus{}Quotes Quill}{A quill which automatically writes down whatever is said within a 2m radius.}
 
 \generic{Rememberall}{A small glass orb filled with grey smoke\comma{} which turns bright scarlet whenever the bearer forgets something. The colour reverts to grey when the target remembers.}
 
 \generic{Self\minus{}Erecting Tent}{Outwardly a three\minus{}man tent\comma{} but due to an extension charm\comma{} its interior is large enough for around 10 adults to live comfortably. It erects and disassembles itself upon hearing a double\minus{}clap.}
 
 \generic{Sneakoscope}{A sneakoscope is a type of dark\minus{}detector that looks like a glass spinning top. It lights up\comma{} spins and whistles whenever someone in a 2m radius is doing something untrustworthy. In practice\comma{} this grants the bearer a +5 bonus to passive perception against `untrustworthy� actors.}
 
 \generic{Spellotape}{An alternative to the {\it Stick} charm\comma{} when dealing with delicate magical objects\comma{} spellotape is an adhesive material.}
 
 \generic{Talking Portrait}{A portrait of a witch or wizard which is imbued with their personality and (to an extent) their memories\comma{} as interpreted by the artist. These paintings can wander in and out of any nearby paintings\comma{} and also visit other paintings of themselves anywhere in the world.}
 
 \generic{Time\minus{}Turner}{An incredibly powerful item\comma{} that takes the form of an hourglass on a necklace\comma{} a time turner allows one to travel backwards in time one hour for every turn of the hourglass. Attempting to travel back more than 5 hours\comma{} or otherwise interfering with the past can cause irreparable damage to the space\minus{}time continuum.}
 
 \generic{Wand}{The cornerstone of wizarding life\comma{} though magic is possible without a wand\comma{} it is much harder � nearly every witch and wizard possess one. Wands bond to their owner through a complex and unknown process\comma{} so it is vital that you only use your own wand\comma{} or one you have bonded with.}%%ArtefactEnd

\newpage
\section{Packs}

Packs are pre\minus{}arranged sets of equipment. Where a pack leaves the precise nature of an item unclear (i.e. `a book')\comma{} you may choose the exact item within the following bounds:
\begin{itemize}
	\item A book may not cost more than 50gp.
	\item A set of tools may not cost more than 30gp.
\end{itemize}

%%PackBegin

\pack{Basic Pack}{30}{A normal backpack\comma{} some normal clothes\comma{} a small dagger\comma{} a candle\comma{} and a healing potion.}

\pack{Explorer Pack}{30}{A set of adventuring clothes\comma{} a climbing set\comma{} a torch\comma{} a map case (with map)\comma{} 10 days of rations\comma{} a water flask\comma{} a bedroll and a tent.}

\pack{Fighter Pack}{30}{A weapon (your choice)\comma{} a basic set of armour\comma{} and a healing potion.}

\pack{Scholar Pack}{30}{A normal backpack\comma{} some normal clothes\comma{} 2 books\comma{} 10 sheets of paper\comma{} ink and pen\comma{} a magnifying glass and 1 set of tools.}

\pack{Student Pack}{30}{A normal backpack\comma{} 1 book\comma{} a set of potion equipment and a set of student robes.}

\pack{Thief Pack}{30}{A set of (dark) clothes\comma{} lockpicking tools\comma{} a set of ball bearings\comma{} a torch\comma{} and a set of rope.}

%%PackEnd


\chapter{Books}\label{S:Books}

\newcommand\book[2]
{
	\vspace{2 ex}
	\small
	\vbox{
	{\bf #1}
	
	#2
	}
	\normalsize
}
\def\spellIntro{Spellbooks contain within them the information needed to cast spells. The rules for casting from spellbooks are detailed on page \pageref{S:Memory}.

For each topic\comma{} 5 books are listed in descending order. Each of these 5 books corresponds to one block of spells listed on page \pageref{S:SpellList}. {\it The Forbidden Arts}\comma{} the second Dark\minus{}Arts spellbook therefore contains all the level\minus{}2 Dark Arts spells\comma{} but not the level one spells. 
}
\def\normalIntro{Normal books fall into many different categories\comma{}. The list below contains an example of some of the most common topics of wizarding books\comma{} and a few examples of the most famous texts within those categories\comma{} where relevant. }

A book is a compendium of knowledge\comma{} contained between two pages. As wizards\comma{} words and knowledge are power \minus{}\minus{} so all good wizards are familiar with their literature! Despite this\comma{} books can be rather heavy (classified as `medium' weight)\comma{} and hence a normal witch or wizard will struggle to carry more than 3 books on them during everyday life. 
\vspace{-3 ex}
\def\w{8}
%%BooksBegin
\\subsection{Normal Books} \normalIntro \begin{center} \footnotesize \begin{rndtable}{|p{\w cm} l |}\hline \normalsize \bf Name & \normalsize \bf Cost \\ \hline	\bf Ancient Runes	&	50 \\ 
	\bf Artificing	&	\\
	~~{\it From Twigs to Flight: A Broommaking Guide}	&	35\\
	~~{\it Avoiding Mishaps When Making Things}	&	20\\
	~~{\it Steel\comma{} Stone \& Sorcery: A Guide to Golems}	&	1000\\
	\bf Astronomy	&	\\
	~~{\it The Stars and Why They Matter}	&	25\\
	~~{\it Galactic Dynamics\comma{} Second Edition}	&	80\\
	~~{\it The Magical Effects of Stars}	&	20\\
	\bf Biographies	&	\\
	~~{\it Wizarding Biographies}	&	30\\
	~~{\it Muggle Biographies}	&	10\\
	\bf Herbology	&	\\
	~~{\it One Thousand Magical Herbs and Fungi}	&	40\\
	~~{\it Flesh\minus{}Eating Trees of the World}	&	30\\
	\bf History of Magic	&	\\
	~~{\it A History of Magic}	&	30\\
	~~{\it Great Wizards Through History}	&	25\\
	~~{\it Non\minus{}European Magic and its History}	&	40\\
	~~{\it Hogwarts a History}	&	15\\
	~~{\it Sites of Historical Sorcery}	&	80\\
	\bf Magical Creatures Book	&	\\
	~~{\it Fantastic Beasts and Where to Find Them: A Guide to Common Magical Creatures}	&	20\\
	~~{\it Studies on Sapient Creatures}	&	20\\
	~~{\it The Unlife\comma{} and How to Avoid Them}	&	40\\
	~~{\it Monster Book of Monsters}	&	60\\
	~~{\it Rare and Dangerous Magical Creatures Around the World}	&	100\\
	\bf Maps	&	\\
	~~{\it Local\minus{}Scale Maps}	&	10\\
	~~{\it Large\minus{}Scale Maps}	&	40\\
	\bf Mathematics	&	10 \\ 
	\bf Muggle Literature	&	5 \\ 
	\bf Muggle Studies	&	25 \\ 
	\bf Periodicals	&	\\
	~~{\it Daily Prophet}	&	4\\
	~~{\it The Quibbler}	&	10\\
	~~{\it Witch Weekly}	&	5\\
	\bf Potions	&	\\
	~~{\it Magical Drafts and Potions}	&	30\\
	~~{\it Advanced Potion Making}	&	80\\
	\bf Quidditch	&	\\
	~~{\it Quidditch Through the Ages}	&	15\\
	~~{\it Handbook of Do\minus{}It\minus{}Yourself Broomcare}	&	35\\
\hline
\end{rndtable}
\end{center} \vfill \normalsize 
\subsection{Spell Books} \spellIntro \begin{center} \footnotesize \begin{rndtable}{|p{\w cm} l |}\hline \normalsize \bf Name & \normalsize \bf Cost \\ \hline	\bf Spellbook: Charms	&	\\
	~~{\it The Standard Book of Spells}	&	30\\
	~~{\it Achievements in Charming}	&	60\\
	~~{\it The Standard Book of Spells (Grade 2)}	&	100\\
	~~{\it Charms: An Expert\apos{}s Guide}	&	200\\
	~~{\it Extreme Incantations}	&	500\\
	\bf Spellbook: Dark Arts	&	\\
	~~{\it An A\minus{}Z of Spooky Spells}	&	100\\
	~~{\it The Forbidden Arts}	&	200\\
	~~{\it Necromancy: A Misunderstood Skill}	&	300\\
	~~{\it Magick Moste Evile}	&	500\\
	~~{\it Spelles Moste Vyle}	&	800\\
	\bf Spellbook: Divination	&	\\
	~~{\it The Dream Oracle}	&	30\\
	~~{\it The Future is an Open Book (And So is This)}	&	60\\
	~~{\it Unfogging the Future}	&	100\\
	~~{\it Death Omens: What to Do When You Know the Worst is Coming}	&	200\\
	~~{\it Time and its Mysteries}	&	500\\
	\bf Spellbook: Hexes \& Curses	&	\\
	~~{\it Basic Hexes for the Busy and Vexed}	&	30\\
	~~{\it A Compendium of Common Curses}	&	60\\
	~~{\it Curses \& Counter\minus{}Curses}	&	100\\
	~~{\it Dark Forces: A Guide to Self Protection}	&	200\\
	~~{\it An Auror\apos{}s Toolkit}	&	500\\
	\bf Spellbook: Illusion	&	\\
	~~{\it Easy Spells to Fool Muggles}	&	30\\
	~~{\it Jiggery\minus{}Pokery \& Hocus\minus{}Pocus}	&	60\\
	~~{\it On the Mysteries of the Human Mind}	&	100\\
	~~{\it Merlin\apos{}s Tricks and Incantations}	&	200\\
	~~{\it Light and Perception: The Magician\apos{}s Mastery}	&	500\\
	\bf Spellbook: Recuperation	&	\\
	~~{\it Self\minus{}Defensive Spellwork}	&	30\\
	~~{\it How To Not Be Killed: A Guide}	&	60\\
	~~{\it Defensive Spells to Save Your Skin}	&	100\\
	~~{\it An Anthology of Safeguarding Measures}	&	200\\
	~~{\it Life\comma{} and How to Preserve It}	&	500\\
	\bf Spellbook: Transfiguration	&	\\
	~~{\it A Beginner\apos{}s Guide to Transfiguration}	&	30\\
	~~{\it Transmutation and  other Transformative Tricks}	&	60\\
	~~{\it Theories of Transubstantial Transfiguration}	&	100\\
	~~{\it Conjuring and Summoning for the Experienced Witch}	&	200\\
	~~{\it The True Art of Transfiguration}	&	500\\
\hline
\end{rndtable}
\end{center} \vfill \normalsize 
%%BooksEnd



\chapter{Tools}

\newcommand\tool[2]
{
	\vspace{2 ex}
	\small
	\vbox{
	{\bf #1}
	
	#2
	}
	\normalsize
}

A tool helps you to do something you couldn't otherwise do with your bare hands \minus{}\minus{} or even with your wand \minus{}\minus{} such as craft or repair an item\comma{} forge a document\comma{} or pick a lock. Anyone can use a tool\comma{} but only someone who is proficient in it will be able to use a tool to its full potential. Proficiency in a tool is granted through Racial or Archetype abilities\comma{} or by taking the Tool\minus{}User Skill. 

A common list of tools is presented below:


%%ToolsBegin

 \begin{center}\begin{rndtable}{|l l l|}\hline \tablehead \normalsize \bf Name & \normalsize \bf Weight & \normalsize \bf Cost \\ \hline 	\bf Climbers Kit	&	Medium	&	25 gold  
 \\ 
	\bf Cooking Utensils	&	Medium	&	10 gold  
 \\ 
	\bf Disguise Kit	&	Light	&	10 gold  
 \\ 
	\bf Forgery Tools	&	Light	&	15 gold  
 \\ 
	\bf Gaming Set	&	Medium	&	10 gold  
 \\ 
	\bf Herbology Tools	&	Medium	&	10 gold  
 \\ 
	\bf Jeweller\apos{}s Tools	&	Light	&	35 gold  
 \\ 
	\bf Lockpicking Tools	&	Light	&	20 gold  
 \\ 
	\bf Musical Instrument	&	Various	&	50 gold  
 \\ 
	\bf Navigator\apos{}s Tools	&	Light	&	10 gold  
 \\ 
	\bf Potion Equipment	&	Medium	&	15 gold  
 \\ 
	\bf Protective Gear	&	Medium	&	30 gold  
 \\ 
	\bf Repair Kit	&	Medium	&	15 gold  
 \\ 
	\bf Runic Tools	&	Light	&	25 gold  
 \\ 
	\bf Smithing Tools	&	Heavy	&	15 gold  
 \\ 
	\bf Surgeon\apos{}s Tools	&	Light	&	25 gold  
 \\ 
\hline
\end{rndtable}
\end{center} 

 
 \tool{Climbers Kit}{Required for scaling up vertical faces\comma{} or abseiling down them. You move at one\minus{}quarter your base speed\comma{} unless you have proficiency in this tool\comma{} in which case you move at half\minus{}speed.}
 
 \tool{Cooking Utensils}{Useful for producing life\minus{}sustaining nutrition out in the wilderness. Proficiency in this toolset means meals cooked restore one level of exhaustion when eaten.}
 
 \tool{Disguise Kit}{A pouch of minor cosmetics\comma{} dyes and small props allows you to alter your appearance through non\minus{}magical means. Proficiency allows you to add your Deception proficiency to all related checks.}
 
 \tool{Forgery Tools}{This kit of parchments\comma{} papers\comma{} inks and wax seals enables you to attempt to produce convincing fakes and forgeries. Proficiency allows you to add your Precision bonus to forgery checks.}
 
 \tool{Gaming Set}{A set of a mundane or magical game\comma{} such as Wizard chess. Proficiency in this toolset allows you to take check\minus{}advantage on all checks whilst playing that game. Each subsequent game requires a new proficiency.}
 
 \tool{Herbology Tools}{Tools required to grow your own plants – pruning shears\comma{} plant nutrients and so on. Proficiency allows you to add your Flora \& Fauna proficiency to herbology checks.}
 
 \tool{Jeweller\apos{}s Tools}{A set of tools used to determine the authenticity and nature of mundane and magical objects\comma{} a Jeweller\apos{}s set bears a visual similarity to the muggle tools from which they get their name – an eyeglass and some simple alchemical equipment. Proficiency allows you to add your Arcane bonus to checks.}
 
 \tool{Lockpicking Tools}{Whilst a wizard often relies on magical means to get past locks\comma{} powerful and intricate magics often require additional help. Locking tools can help you investigate both mundane and magical locks\comma{} and a proficiency with them enables you to add your Dexterity proficiency to any checks.}
 
 \tool{Musical Instrument}{Merely possessing a musical instrument allows you to make crude noises\comma{} but proficiency in an individual instrument allows you to work wonders with it. Each subsequent instrument requires a new proficiency.}
 
 \tool{Navigator\apos{}s Tools}{A requirement for long\minus{}distance navigation. Allows you to chart a course for a broomstick flight or ship\apos{}s course for journeys greater than 1 hour. Proficiency allows you to add your Perception bonus to avoid getting lost.}
 
 \tool{Potion Equipment}{A must\minus{}have for the budding alchemist\comma{} potion equipment typically consists of a cauldron\comma{} an alembic\comma{} and other useful bits of equipment. Proficiency in this set of equipment allows you to add your Flora \& Fauna proficiency to potion making attempts.}
 
 \tool{Protective Gear}{A heavy\minus{}duty set of protective robes\comma{} eye protection and gloves provide protection against alchemy and enchanting mishaps\comma{} but at the cost of a 4\minus{}point penalty to FIN and ATH whilst wearing them. Proficiency in this set of tools reduces this to a 1\minus{}point penalty.}
 
 \tool{Repair Kit}{A set of tools for repairing armour and clothing. Proficiency allows you to halve the time required to repair a set to full strength.}
 
 \tool{Runic Tools}{Runic tools allow you to engrave intricate runes on a surface with exquisite accuracy. They are therefore invaluable to powerful Enchanting and Rune\minus{}placing endeavours. Proficiency in these skills grants you check\minus{}advantage in all Enchanting and Rune\minus{}binding checks.}
 
 \tool{Smithing Tools}{Required to forge new items out of raw material. Proficiency grants you the ability to add your Strength bonus onto all forging checks.}
 
 \tool{Surgeon\apos{}s Tools}{A set of surgeon\apos{}s tools allows you to perform delicate medical procedures\comma{} when simple healing spells do not do the trick. Proficiency with these tools allows you to add your Healing proficiency onto any related checks.}%%ToolsEnd


