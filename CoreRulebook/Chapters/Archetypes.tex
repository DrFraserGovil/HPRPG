\documentclass[CoreRulebook.tex]{subfile}
     
     
     
     
     
      \chapter{Character Archetype}\label{C:Archetype}
      
      Whilst your character is a unique individual, an adventuring soul destined for greatness, most questers find themslves falling into one of many {\it archetypes} -- are they the headstrong hero who needs to learn humilty? The academic who's quest for knowledge has led to unforseen consequences, or the plucky underdog trying to quit their life of crime? 
      
      The archetype (also known as the {\it class}) of your character is a way of formalising these character types. The role of your character is more than simply the job they perform, it is the prism through which they see the world -- it guides their very essence, how they perceieve themselves and others. 
      
      
\define@key{archetype}{arcaneI}{\def\arcaneI{#1}}
\define@key{archetype}{featureI}{\def\featureI{#1}}
\define@key{archetype}{arcaneII}{\def\arcaneII{#1}}
\define@key{archetype}{featureII}{\def\featureII{#1}}
\define@key{archetype}{arcaneIII}{\def\arcaneIII{#1}}
\define@key{archetype}{alphaFeatureIII}{\def\alphaIII{#1}}
\define@key{archetype}{betaFeatureIII}{\def\betaIII{#1}}
\define@key{archetype}{arcaneIV}{\def\arcaneIV{#1}}
\define@key{archetype}{alphaFeatureIV}{\def\alphaIV{#1}}
\define@key{archetype}{betaFeatureIV}{\def\betaIV{#1}}
\define@key{archetype}{arcaneV}{\def\arcaneV{#1}}
\define@key{archetype}{alphaFeatureV}{\def\alphaV{#1}}
\define@key{archetype}{betaFeatureV}{\def\betaV{#1}}
\define@key{archetype}{arcaneVI}{\def\arcaneVI{#1}}
\define@key{archetype}{alphaFeatureVI}{\def\alphaVI{#1}}
\define@key{archetype}{betaFeatureVI}{\def\betaVI{#1}}
\define@key{archetype}{arcaneVII}{\def\arcaneVII{#1}}
\define@key{archetype}{alphaFeatureVII}{\def\alphaVII{#1}}
\define@key{archetype}{betaFeatureVII}{\def\betaVII{#1}}
\define@key{archetype}{arcaneVIII}{\def\arcaneVIII{#1}}
\define@key{archetype}{alphaFeatureVIII}{\def\alphaVIII{#1}}
\define@key{archetype}{betaFeatureVIII}{\def\betaVIII{#1}}
\define@key{archetype}{arcaneIX}{\def\arcaneIX{#1}}
\define@key{archetype}{alphaFeatureIX}{\def\alphaIX{#1}}
\define@key{archetype}{betaFeatureIX}{\def\betaIX{#1}}
\define@key{archetype}{arcaneX}{\def\arcaneX{#1}}
\define@key{archetype}{alphaFeatureX}{\def\alphaX{#1}}
\define@key{archetype}{betaFeatureX}{\def\betaX{#1}}
\define@key{archetype}{arcaneXI}{\def\arcaneXI{#1}}
\define@key{archetype}{alphaFeatureXI}{\def\alphaXI{#1}}
\define@key{archetype}{betaFeatureXI}{\def\betaXI{#1}}
\define@key{archetype}{arcaneXII}{\def\arcaneXII{#1}}
\define@key{archetype}{alphaFeatureXII}{\def\alphaXII{#1}}
\define@key{archetype}{betaFeatureXII}{\def\betaXII{#1}}
\define@key{archetype}{arcaneXIII}{\def\arcaneXIII{#1}}
\define@key{archetype}{alphaFeatureXIII}{\def\alphaXIII{#1}}
\define@key{archetype}{betaFeatureXIII}{\def\betaXIII{#1}}
\define@key{archetype}{arcaneXIV}{\def\arcaneXIV{#1}}
\define@key{archetype}{alphaFeatureXIV}{\def\alphaXIV{#1}}
\define@key{archetype}{betaFeatureXIV}{\def\betaXIV{#1}}
\define@key{archetype}{arcaneXV}{\def\arcaneXV{#1}}
\define@key{archetype}{alphaFeatureXV}{\def\alphaXV{#1}}
\define@key{archetype}{betaFeatureXV}{\def\betaXV{#1}}
\define@key{archetype}{arcaneXVI}{\def\arcaneXVI{#1}}
\define@key{archetype}{alphaFeatureXVI}{\def\alphaXVI{#1}}
\define@key{archetype}{betaFeatureXVI}{\def\betaXVI{#1}}
\define@key{archetype}{arcaneXVII}{\def\arcaneXVII{#1}}
\define@key{archetype}{alphaFeatureXVII}{\def\alphaXVII{#1}}
\define@key{archetype}{betaFeatureXVII}{\def\betaXVII{#1}}
\define@key{archetype}{arcaneXVIII}{\def\arcaneXVIII{#1}}
\define@key{archetype}{alphaFeatureXVIII}{\def\alphaXVIII{#1}}
\define@key{archetype}{betaFeatureXVIII}{\def\betaXVIII{#1}}
\define@key{archetype}{arcaneXIX}{\def\arcaneXIX{#1}}
\define@key{archetype}{alphaFeatureXIX}{\def\alphaXIX{#1}}
\define@key{archetype}{betaFeatureXIX}{\def\betaXIX{#1}}
\define@key{archetype}{arcaneXX}{\def\arcaneXX{#1}}
\define@key{archetype}{alphaFeatureXX}{\def\alphaXX{#1}}
\define@key{archetype}{betaFeatureXX}{\def\betaXX{#1}}

%%keyEnd    
          
          
      \def\comma{,}
      \def\dash{-}
      \newcommand{\tableLong}
      {
      
      %%tableBegin

1  &  + \arcaneI & \multicolumn{2}{c}{\featureI} 
\\
2  &  + \arcaneII & \multicolumn{2}{c}{\featureII} 
\\
3  &  + \arcaneIII & \parbox[t]{\w cm}{\centering\alphaIII}  &  \parbox[t]{\w cm}{\centering\betaIII}
 \\
4  &  + \arcaneIV & \parbox[t]{\w cm}{\centering\alphaIV}  &  \parbox[t]{\w cm}{\centering\betaIV}
 \\
5  &  + \arcaneV & \parbox[t]{\w cm}{\centering\alphaV}  &  \parbox[t]{\w cm}{\centering\betaV}
 \\
6  &  + \arcaneVI & \parbox[t]{\w cm}{\centering\alphaVI}  &  \parbox[t]{\w cm}{\centering\betaVI}
 \\
7  &  + \arcaneVII & \parbox[t]{\w cm}{\centering\alphaVII}  &  \parbox[t]{\w cm}{\centering\betaVII}
 \\
8  &  + \arcaneVIII & \parbox[t]{\w cm}{\centering\alphaVIII}  &  \parbox[t]{\w cm}{\centering\betaVIII}
 \\
9  &  + \arcaneIX & \parbox[t]{\w cm}{\centering\alphaIX}  &  \parbox[t]{\w cm}{\centering\betaIX}
 \\
10  &  + \arcaneX & \parbox[t]{\w cm}{\centering\alphaX}  &  \parbox[t]{\w cm}{\centering\betaX}
 \\
11  &  + \arcaneXI & \parbox[t]{\w cm}{\centering\alphaXI}  &  \parbox[t]{\w cm}{\centering\betaXI}
 \\
12  &  + \arcaneXII & \parbox[t]{\w cm}{\centering\alphaXII}  &  \parbox[t]{\w cm}{\centering\betaXII}
 \\
13  &  + \arcaneXIII & \parbox[t]{\w cm}{\centering\alphaXIII}  &  \parbox[t]{\w cm}{\centering\betaXIII}
 \\
14  &  + \arcaneXIV & \parbox[t]{\w cm}{\centering\alphaXIV}  &  \parbox[t]{\w cm}{\centering\betaXIV}
 \\
15  &  + \arcaneXV & \parbox[t]{\w cm}{\centering\alphaXV}  &  \parbox[t]{\w cm}{\centering\betaXV}
 \\
16  &  + \arcaneXVI & \parbox[t]{\w cm}{\centering\alphaXVI}  &  \parbox[t]{\w cm}{\centering\betaXVI}
 \\
17  &  + \arcaneXVII & \parbox[t]{\w cm}{\centering\alphaXVII}  &  \parbox[t]{\w cm}{\centering\betaXVII}
 \\
18  &  + \arcaneXVIII & \parbox[t]{\w cm}{\centering\alphaXVIII}  &  \parbox[t]{\w cm}{\centering\betaXVIII}
 \\
19  &  + \arcaneXIX & \parbox[t]{\w cm}{\centering\alphaXIX}  &  \parbox[t]{\w cm}{\centering\betaXIX}
 \\
20  &  + \arcaneXX & \parbox[t]{\w cm}{\centering\alphaXX}  &  \parbox[t]{\w cm}{\centering\betaXX}
 \\

%%tableEnd	
      }
      
      \newcommand{\tableShort}[1]
{

%%smallTableBegin


1  &  + \arcaneI & \multicolumn{2}{c}{\featureI} 
\\
2  &  + \arcaneII & \multicolumn{2}{c}{\featureII} 
\\
3  &  + \arcaneIII & \parbox[t]{\w cm}{\centering\alphaIII}  &  \parbox[t]{\w cm}{\centering\betaIII}
 \\
4  &  + \arcaneIV & \parbox[t]{\w cm}{\centering\alphaIV}  &  \parbox[t]{\w cm}{\centering\betaIV}
 \\
5  &  + \arcaneV & \parbox[t]{\w cm}{\centering\alphaV}  &  \parbox[t]{\w cm}{\centering\betaV}
 \\

%%smallTableEnd

      }   
          
          
     \newcommand{\archetype}[5]
     {
     \def\mode{#4}
    
     \begingroup
     
     %%defaultBegin

\setkeys{archetype}{ arcaneI=0, featureI= --, arcaneII=0, featureII= --, arcaneIII=0, alphaFeatureIII= -- , betaFeatureIII= --, arcaneIV=0, alphaFeatureIV= -- , betaFeatureIV= --, arcaneV=1, alphaFeatureV= -- , betaFeatureV= --, arcaneVI=1, alphaFeatureVI= -- , betaFeatureVI= --, arcaneVII=1, alphaFeatureVII= -- , betaFeatureVII= --, arcaneVIII=1, alphaFeatureVIII= -- , betaFeatureVIII= --, arcaneIX=1, alphaFeatureIX= -- , betaFeatureIX= --, arcaneX=2, alphaFeatureX= -- , betaFeatureX= --, arcaneXI=2, alphaFeatureXI= -- , betaFeatureXI= --, arcaneXII=2, alphaFeatureXII= -- , betaFeatureXII= --, arcaneXIII=2, alphaFeatureXIII= -- , betaFeatureXIII= --, arcaneXIV=2, alphaFeatureXIV= -- , betaFeatureXIV= --, arcaneXV=3, alphaFeatureXV= -- , betaFeatureXV= --, arcaneXVI=3, alphaFeatureXVI= -- , betaFeatureXVI= --, arcaneXVII=3, alphaFeatureXVII= -- , betaFeatureXVII= --, arcaneXVIII=3, alphaFeatureXVIII= -- , betaFeatureXVIII= --, arcaneXIX=3, alphaFeatureXIX= -- , betaFeatureXIX= --, arcaneXX=4, alphaFeatureXX= -- , betaFeatureXX= --}

%%defaultEnd
     \def\w{6.2}
       \setkeys{archetype}{#5}
 
     
     \begin{tcolorbox}[  before skip=7pt plus 2pt,
     	  boxrule=0pt,
     	  boxsep=0pt,
     	  toptitle=4pt,
     	  left=0pt,
     	  right=0pt,
     	  bottom=11pt,
     	  arc=0.5mm,
     	  oversize=0pt,
     	  colback=papyrus,
     	 colbacktitle=titlered,
     	colframe=titlered, title=\vspace{-4ex}
     	]
     		
     		{\large \begin{center} \bf #1\end{center}}
     		\vspace{-1.3ex}
     		
     		\dndlineLong
     	
     		\begin{center}
     	\small
     		      		 \begin{tabular}{c cc  c }
\bf Level 	&	\bf Arcane Wisdom	&	\bf #2 Features	&	\bf #3 Features
\\ 
     		\if\mode0
     			\tableLong
		\fi
		\if\mode1
			\tableShort
		\fi
		     		\end{tabular}
     		\end{center}
     		\normalsize
     		\dndlineLong
     
     	\end{tcolorbox}
     	\endgroup
     }
     
     
   
     
     There are 12 Archetypes, each with two branches. 
     
      \begin{center}
      \begin{rndtable}{c c c}
      \hline
     	Archetype	&	Alpha Branch	&	Beta Branch
     	\\
     	Artificer	&	Enchanter		&	Potioneer
     	\\
     	Auror		&	Soldier			&	Warder
     	\\
     	Brute		&	Beserker		&	Bodyguard
     	\\
     	Empath		&	Healer			&	Seer
     	\\
     	Fighter		&	Melee			&	Ranged
     	\\
     	Sage	&	Researcher		&	Teacher
     	\\
     	Investigator	&	Detective	&	Journalist
     	\\
     	Naturalist	&	Magizoologist	&	Druid
     	\\
     	Oathkeeper	&	Knight	&	Paladin
     	\\
     	Outlaw		&	Assassin	&	Thief
     	\\
     	Performer	&	Bard	&	Acrobat
     	\\
     	Ranger	&	Scout	&	Hunter
     	\\
     	\hline
     \end{rndtable} 
      \end{center}
     
     
     CHaracters who are students, however, are much less likely to know what their roll in life is yet. They are much more likely to be defined and shaped by their school environment, so there are four special Archetypes, dedicated to the 4 Houses at Hogwarts:
     
     \begin{center}
      \begin{rndtable}{c c c}
      \hline
     	Archetype	&	Alpha Branch	&	Beta Branch
     	\\
     	Gryffindor Student	&	Sportsman	&	Rebel
     	\\
     	Hufflepuff Student	&   Hard-Worker	&	??
     	\\
     	Ravenclaw Student	&	Nerd	&	Prodigy
     	\\
     	Slytherin	&	Student Politician	&	Schemer
     	\\
     	\hline
     \end{rndtable} 
      \end{center}
     
     
\clearpage
\subfile{Data/Archetypes/Gryffindor}
\clearpage
\subfile{Data/Archetypes/Hufflepuff}
\clearpage
\subfile{Data/Archetypes/Ravenclaw}
\clearpage
\subfile{Data/Archetypes/Slytherin}
\clearpage
\subfile{Data/Archetypes/Empath}
