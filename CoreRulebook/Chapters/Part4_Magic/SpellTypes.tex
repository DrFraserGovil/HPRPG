

\section{Spell Types} 

In addition to falling into one of the seven Schools (a taxonomy based on the spell effect), every spell can also be categorised as a {\it type}, which is based on how the spell is cast. 

\subsection{Instant}

An instant spell is `cast and forget': as soon as you complete the requisite casting checks, the spell is `launched' (usually in the form of a magical bolt of light) towards the target. These bolts travel at speeds of 100m per cycle, which means in most cases, the effect is applied between the successful casting and the beginning of the next turn cycle.

Instant spells are denoted by the symbol \instSymb.

\subsection{Focus}

A focus spell is cast like an Instant spell, but may then be continued indefinitely, repeating the initial effects once per turn as long as you keep the spell active. No further checks are needed to continue the spell, but you must keep your mind focussed on the task at hand. Unless stated otherwise, Focus spells {\bf do not} cost additional FP after the first round in which they are cost. 

Because you must remain focussed, no further spells can be cast for the duration of this spell. In combat, maintaining a Focus spell takes your entire major action.    

Whilst maintaining a {\it Focus} spell you are considered {\it Distracted} and take the associated status effect. This renders you vulnerable to Critical Strikes, and upon taking damage you must pass a Willpower Resist check to maintain your concentration. 

You may end the spell effect at any time without it counting as an action. 

Focus spells are denoted by the symbol \concSymb.

\subsection{Ward}
A ward is a spell that affects a large area, and typically lasts for a long time after being cast. Most wards are centred on a single `focal point', which is selected at the time of casting. Some wards limit the kind of target that a valid focal point can be attached to. 

Unless stated otherwise, a ward spell is assumed to move as the object the focal point is attached to moves - a warded individual, therefore, does not need to stay still to remain protected. 

Ward spells are denoted by the symbol \wardSymb.

\subsection{Ritual}

A Ritual spell is a spell that requires a large amount of preparation -- be it meditation, drawing a summoning circle upon the ground, or performing a special dance. Each Ritual spell has a designated time that the ritual takes to complete, to cast a ritual spell you must spend this length of time preparing for the spell, and after the requisite time has passed, {\it then} you perform the check, and the spell effect is activated. If you fail the check, or choose to stop the ritual, i.e. to take another action, you must restart the ritual spell from the beginning. 

As with a focus spell, concentration is key to completing a ritual, and whilst performing a ritual, you are considered {\it Distracted}. 

Ritual spells are denoted by the symbol \ritSymb.


\subsection{Other Spell Types}

\subsubsection{Runic}



\subsubsection{Beast}

Beast spells are denoted by the symbol \beastSymb.
