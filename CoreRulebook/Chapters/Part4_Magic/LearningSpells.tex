\chapter{Learning Spells}\label{S:SpellLearn}

Learning more spells is key for a witch or wizard to expand their magical portfolio - though individual spells can be bent and shaped by the will of their caster, no amount of firebending will help you teleport through space, after all. 

There are three ways for a witch to learn a new spell - some much more easy than others. 

Throughout the learning process you will need to perform spellcasting checks for the relevant spell. As you are learning new information, the \imp{Intelligence} base attribute is to be assumed as the default, unless a {\it very} convincing reason for any other attribute is given. 

\section{Teachers}

Hogwarts is, after all, a school for Witches and Wizards. It wouldn't be a very good school if the teachers couldn't teach you spells, would it? 

If you can find an expert in the spell you wish to learn, they may impart their knowledge to you, and walk you through the process of learning a new spell. 

This typically takes a number of hours for you to truly understand the use of the spell - and you must dedicate the relevant time to completing your schoolwork and paying attention in classes for the knowledge to stick. 

Once the GM is satisfied that you have gone through the relevant learning processes and dedicated enough time to absorbing your teacher's knowledge, they will reward you with a new memorised spell. If you are in downtime, they may also offer you a choice of a number of spells which are being taught over that period - you may choose to dedicate your practice time to learn one of this selection. This is encouraged because it allows your group to diversify the spells that they know as a group. 


\section{Books}

Of course, teachers at Hogwarts teach their lessons according to a strict lesson plan - and even then, some of them are just not very good teachers. 

What is a student to do when they wish to strike out on their own, and acquire knowledge without the help of a teacher? Luckily, over the centuries, wizardkind has transcribe dmuch of their knowledge into \key{Spellbooks}, arcane tomes which hold deep magical secrets. 

Each spell contained in the list on page \pageref{S:SpellList} can be found within a given spellbook - if you wish to learn a given spell, you must locate the book, and begin to study it.

Studying an unusued \imp{spellbook} for 6 hours is enough for the spell to become learned (subject to the final bit of learning). Once the spellbook has been used in this fashion, it must `recharge' before it can be memorised by another person. Full information about \key{spellbooks} and their magical properties can be found on page \pageref{S:Spellbooks}. 


\section{Trial and Error}

If you have no outside help - or you are attempting to `learn' a spell which has not been discovered yet, then you must strike out on your own in order to uncover the arcane secrets, and to work out the correct ritualistic elements and frame of mind which allow you to manifest the desired effect. 

Doing so, however, is an incredibly difficult and time-consuming effort, and can generally only be achieved by extraordinarily powerful or clever spellcasters. 

In order to complete this process, you must attempt an long-term project action, as discussed on page \pageref{S:Extended}. 

You must first fully articulate (to the GM) what spell you are attempting to create. If the GM decides that this meets the criteria of one of the existing spells, or if it is an appropriate new spell to be bringing into the setting, they may allow you to continue with the process. 

Every day where you dedicate at least 4 hours to the process of trial and error, you may perform a mock-spellcasting action for your desired spell. The DV of the spell is set by the GM, but the recommended DV takes the form DV = $15 - $\imp{Affinity}. A spellcaster capable of casting \levelSeven{} spells in the relevant discipline would therefore face a DV of 7 to complete this action. 

Once you have accumulated 7 sets of 7 successes (i.e. a total of 49), then you have successfully learned the new spell. As this spell was designed by you, you may decide upon the relevant incantation and other ritualistic elements which accompany the spell.  



\section{Finalising the Learning}

Once you have acquired the base level of knowledge through one of these three means you have the {\it theoretical} ability to cast the spell. 

Before you can truly be able to say that you have memorised it, you must attempt a single spellcasting action, with a \imp{DV} of 7. If you get a single success, you may transcribe the spell onto your character sheet, and use it freely in future. 

If you fail, you must go back to your notes for a further bit of study, refine your method, and then try again. You may attempt another spellcasting action one hour later, with the DV increased by 1. If you fail a second time, you must take at least another hour to memorise the spell, and attempt again with the DV increased again. When the DV reaches 12, you realise that you completely misunderstood the theory, and must start the learning process again. 
