
\chapter{Understanding Magic}

Magic is an all-encompassing supernatural force within the universe, with the ability to alter reality at a most basic level. Incredibly powerful and difficult to control, magic is - at its heart - formless, chaotic and without boundaries. 

However, over the centuries, some humans have been born with the ability to touch this immense reservoir of power: witches and wizards. These magic-users have attempted to tame and define magic, and to shepherd into easily understood forms. 

The greatest discovery in wizarding history - comparable to the discovery of fire, or the wheel in the muggle world - was the discovery of the magical \key{spell}, followed by the discovery of the \key{wand}. 

Prior to this discovery, witches and wizards had attempted to harness the infinite force of magic through sheer mental effort. Of course, this meant that a single stray thought at an inopportune moment could lead to blowing up a city, rather than lighting a candle. 

Magical spells however, combine a ritualistic element (usually in the form of an incantation and a physical movement) to condition and focus the mind into the correct shape. The discovery of wands to focus and channel magical energies helped popularise this new way of casting magic - and it is now the utterly dominant way for magical folk to use their skills.  

Even with this focussing and harnessing, the theory of magic remains a field shrouded in mystery. The spell \imp{incendio} is well known to manipulate the primal, elemental force of fire and heat - however the effect is rarely the same between two people: a first-year student using all of their might could probably ignite a campfire. However, mere hours before his death, Albus Dumbledore used the exact same spell to bring about a fiery maelstrom capable of incinerating an entire army. 

Clearly, merely speaking the words and waving your wand is not enough to cast magic - it still requires channeling through the mage's internal reserves of magical power and understanding in order to shape the resulting magic into the desired effect. 


\section{Schools, Disciplines \& \imp{Affinities}}\label{S:Schools}

A witch or wizard's ability to wield immense magical powers is limited only by their ability to understand and shape their spells. A higher level of understanding of the mystical forces allows them to access more powerful magic and to wield their spells more effectively. 

Of course, magic is not a cohesive whole, it is a far-reaching field, which encompasses many different areas and skills -- some of which require vastly different skillsets to use. The human desire to categorise, quantify and codify has led to the popularity of a Taxonomy of magical spells, a way of categorising spells which require a similar level of understanding. 

The current paradigm suggests that there are Seven distinct \key{Schools} of magic (which broadly affect reality in the same way), each of which contains two \key{Disicplines}. Each \imp{Discipline} describes some facet of magic which behave, more or less, in the same fashion.

The seven schools of magic are Charms, Divination, Illusion, Malediction, Recuperation, Transfiguration and the Dark Arts. A description of each of the 7 schools and their contained disciplines is found below.

\subsection{\key{Affinities}}\label{S:Affinities}
Because all \imp{Spells} within a given \imp{Discipline} require a similar kind of mystical knowledge, a person who is very good at a given \imp{Conjuration} spell is highly likely to excel at another spell from that discipline. This is known as having an \key{Affinity} with this discipline. 

An \imp{Affinity} functions similarly to an \imp{Ability} for spellcasting efforts, and equally has a number of dots assigned to it at character creation, and subsequently improved as your character evolves. 

When you wish to cast a spell belonging to a certain Discipline, you use the combined dice from your \imp{Aspect} and \imp{Affinity} to form the dice pool. 

The assignment of \imp{Affinities} at character creation is determined by the wand-choosing process discussed on page \pageref{S:Wandchoosing}.


\section{Discipline Descriptions}\label{S:DiscDescs}

	\schoolDescribe{Charms}{The Charms school of magic relies on manipulating the material world, harnessing the power of movement and speed, as well as manipulating the basic building blocks of reality: earth, air, wind and fire.  Those who are proficient in Charms are known as {\it Sorcerers}.}{Elemental}{Elemental magic studies the manipulation and invocation of very primal forces -- heat, light, energy, matter, and the classical elements.}{Kinetic}{Kinetics is a discipline which relies on moving and manipulating physical objects, and often forms the basis of `everyday' magic.}

	\schoolDescribe{Divination}{The Divination school encompasses magic which taps into forces which exist beyond the physical world to discern knowledge that would have previously remained hidden - entering the domain of the senses, memory, and the spiritual realms. Those who are proficient in the field of Divination are known as {\it Clairvoyants}.}{Cerebral}{Cerebral magic is the study of peering into the human mind, extending the senses beyond their normal range and detecting the undetectable. }{Temporal}{One of the most mysterious disciplines, temporal magic allows one to see beyond concerns such as time and space, casting your vision across vast distances, peering into the distant past, and observe (and perhaps manipulate) the universe at an extraplanar level}
	
	\schoolDescribe{Illusion}{The Illusion school of magic is, as the name might suggest, focussed on magic which produces false images, and tweaks the mind into seeing things which are not really there.  Witches and Wizards who excel in Illusion magics are known as {\it Magicians}. }{Bewitchment}{This discipline focusses on the gentle persuasion of the mind and the manipulation and conjuring of images to convince the target of something which is not true.}{Psionic}{A darker side of illusion magics, psionics is the art of imposing your will over that of your target -- forcing your way into their mind and altering it as you see fit.}
	
	\schoolDescribe{Malediction}{The Malediction school of magic contains those spells which have the primary intent to hurt, inflict harm on and otherwise incapacitate others.   Those who are experts in the field of Malediction are known as {\it Battlemages}. }{Hexes}{Hexes are a field which focusses on magic that directly harms the targeted person or object.}{Curses}{Unlike hexes, curses do not directly harm the target but instead incapacitates them, inhibits their capabilities, or otherwise reduces the threat they pose.}
   
   \schoolDescribe{Recuperation}{The Recuperation school of magic is often considered unglamourous, but those who can look past that can see that the ability to heal and protect yourself and others from harm is utterly invaluable.   Those who are proficient in the use of Recuperation magic are known as {\it Aegistes}. }{Healing}{Healing is, unsurprisingly, the study of magic used to heal the sick and wounded, break curses and project powerful positive energies.}{Warding}{Warding magic is almost entirely defensive in nature, allowing the caster to protect themselves and others from harm by casting powerful and long lasting shields and force-fields.}
	
	\schoolDescribe{Transfiguration}{The Transfiguration school of magic is focused on the transformation of the natural order - either by altering and reshaping the form of existing objects, or by summoning entirely new matter from thin air.  Those who excel in Transfiguration are known as {\it Thaumaturges}.}{Alteration}{The alteration discipline studies the ability to change things from one form into another.}{Conjuration}{Conjuration magic is concerned with the ability to summon new objects and beings out of thin air, or to banish objects from existence.}
	
	\schoolDescribe{Dark Arts}{The Dark Arts school of magic encompasses magic which is frowned on in polite society, either because it involves truly evil spells - those which cannot be used without leaving scars on the soul, or those which tap into the dangerous and unfathomable energies of the dark and unspeakable things which lie just out of sight - under your bed and in the corner of your eye... Those who wield this forbidden magic are known as {\it Warlocks}.}{Necromancy}{A taboo discipline which contains deeply unpleasant spells which can only be cast by beings corrupted by evil - torture, death and worse lie in the domain of necromancy.}{Occultism}{Occultism is a rarely studied discipline that accesses and manipulates otherworldly energies originating from the Eldritch domain -- powerful, yet highly unpredictable.}




