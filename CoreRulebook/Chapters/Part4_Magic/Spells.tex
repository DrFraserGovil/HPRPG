\chapter{Spells}\label{S:SpellList}

Magic is invoked through the casting of a magic spell. 


\newcommand\spExample[2]
{
	\textit{\textbf{#1:}} #2
}

\makeatletter
\define@key{spell}{name}{\def\name{#1}}
\define@key{spell}{incantation}{\def\incantation{#1}}
\define@key{spell}{description}{\def\description{#1}}
\define@key{spell}{book}{\def\book{#1}}
\define@key{spell}{zero}{\def\zero{#1}}
\define@key{spell}{one}{\def\one{#1}}
\define@key{spell}{two}{\def\two{#1}}
\define@key{spell}{three}{\def\three{#1}}
\define@key{spell}{four}{\def\four{#1}}
\define@key{spell}{five}{\def\five{#1}}
\define@key{spell}{six}{\def\six{#1}}
\define@key{spell}{seven}{\def\seven{#1}}
\define@key{spell}{haszero}{\def\haszero{#1}}
\define@key{spell}{hasone}{\def\hasone{#1}}
\define@key{spell}{hastwo}{\def\hastwo{#1}}
\define@key{spell}{hasthree}{\def\hasthree{#1}}
\define@key{spell}{hasfour}{\def\hasfour{#1}}
\define@key{spell}{hasfive}{\def\hasfive{#1}}
\define@key{spell}{hassix}{\def\hassix{#1}}
\define@key{spell}{hasseven}{\def\hasseven{#1}}
\define@key{spell}{hasExamples}{\def\hasExamples{#1}}
\makeatother



\newcommand\spell[1]
{
	\begingroup
	\setkeys{spell}{name= None, incantation = None, description = None, zero = , one = ,  two = , three = , four = , five = , six = , seven = , book = ,hasExamples = 0} 
	
	\setkeys{spell}{#1}
	\index{Spells!\name{}}	
		\subsubsection*{\key{\name} (\textit{\incantation})}
		
		\description
		
		This spell is found in the textbook \imp{\book}.
		
		%~ \if\hasExamples1
			%~ \spellTable{}
		%~ \fi
	\endgroup
}

\newcommand\discipline[2]
{
	\section{#1 Spells}
	
	#2
}

\newcommand\summaryRow[2]
{
	\key{#1}	&	\parbox[t]{6.5cm}{\raggedright \imp{#2}} \\
}

\newcommand\spellList[1]
{
	\section{Spell Summary}
	
	Below is a list of each spell, grouped by the Disciplines they belong to. 
	
	\begin{rndtable}{l l}
		\key{Discipline}	&	\key{Spells} \\
		#1
	\end{rndtable}

}


%%Begin
\spellList{
	\summaryRow{Alteration}{Animate, Degrade, Refine, Transmute}
	\summaryRow{Bewitchment}{Charm, Conceal, Distract, Mirage}
	\summaryRow{Cerebral}{Communicate, Inspire, Sense, Slumber}
	\summaryRow{Conjuration}{Bind, Forge, Manifest, Vanish}
	\summaryRow{Curse}{Corrupt, Disarm, Infect, Stun}
	\summaryRow{Elemental}{Burn, Flood, Freeze, Gust, Illuminate, Sculpt, Shock}
	\summaryRow{Hermetics}{Heal, Purify, Restore, Sustain}
	\summaryRow{Hex}{Disintegrate, Force, Jinx, Strike}
	\summaryRow{Kinesis}{Apparate, Halt, Move, Repair, Seek}
	\summaryRow{Necromancy}{Blight, Defile, Drain, Kill, Raise}
	\summaryRow{Occultism}{Attune, Consort, Eclipse}
	\summaryRow{Psionics}{Compel, Delude, Remind, Torture}
	\summaryRow{Temporal}{Identify, Prophesy, Scry, Traverse}
	\summaryRow{Warding}{Abjure, Bypass, Resist, Shield, Trap}
}

\discipline{Alteration}
{
	\spell{name = Animate, incantation = Piertotum Locomotum, book = The Subtle Nuances of Transmogrification\comma{} Transformation and Transfiguration, description = Breathe a modicum of life into a target and cause it to move and take actions as if were alive: from animating simple origami swans\comma{} to calling forth legions of statues and articulated trees. 

Animation spells typically required continued \imp{concentration} from the caster in order to sustain\comma{} though more powerful casters can imbue their creations with purpose without needing to focus upon them., haszero=1, zero=\spExample{Jumping Bean}{Cause a tiny object\comma{} such as a grain of rice\comma{} to hop and jump around of its own accord.}, hasone=0, hastwo=0, hasthree=0, hasfour=0, hasfive=0, hassix=0, hasseven=1, seven=\spExample{Army of Statues}{Animate a large number of stone statues\comma{} trees of similar monolithic objects\comma{} causing them to come alive and fight for your cause.}, hasExamples = 1}

	\spell{name = Degrade, incantation = Abissio, book = Transmutation and Transformative Tricks, description = When you \imp{degrade} a target\comma{} you cheapen and deteriorate it\comma{} weaken its structure and make it less appealing \minus{} though without fundamentally altering its nature. 

Physical objects become tarnished and pockmarked\comma{} losing their effectiveness\comma{} whilst living beings find their muscles temporarily weakened\comma{} and their mind clouded over., haszero=1, zero=\spExample{Dirty}{Add a layer of dirt of grime to the target}, hasone=1, one=\spExample{Blemish}{Create small imperfections in the surface of an object. }

\spExample{Minor Deteriorate}{Suppress a foe’s abilities\comma{} temporarily weakening their muscles or clouding their mind\comma{} making their next action just a bit harder.}, hastwo=0, hasthree=0, hasfour=0, hasfive=0, hassix=0, hasseven=1, seven=\spExample{True Degredation}{Make the target into the worst of its form; the worst\comma{} weakest and most ugly of its type.}, hasExamples = 1}

	\spell{name = Refine, incantation = Meliorus, book = Theories of Transubstantial Transfiguration, description = A \imp{refined} target finds its quality improved\comma{} kinks worked out and generally become their `best selves’\comma{} without fundamentally altering their nature. 

Physical objects regain their shine\comma{} dents reform and vanish\comma{} and unpleasant food takes on delicious flavour. Living beings under the effect of this spell find themselves with a sudden burst of energy\comma{} able to do more than they were before., haszero=1, zero=\spExample{Clean}{Remove a thin layer of dirt or grime from the target.}, hasone=1, one=\spExample{Embellish}{Remove small imperfections in the surface of an object\comma{} or improve the taste of a certain food.}

\spExample{Minor Fortify}{Make a target quicker\comma{} stronger or more resilient\comma{} and hence more likely to succeed at a single action.}, hastwo=0, hasthree=0, hasfour=0, hasfive=0, hassix=0, hasseven=1, seven=\spExample{True Refinement}{Make the target into the ultimate of its form: the most beautiful or the strongest of its type. }, hasExamples = 1}

	\spell{name = Transmute, incantation = Mutatum, book = A Beginner\apos{}s Guide to Transfiguration, description = \imp{Transmutation} is the ultimate act of \imp{Alteration}: changing the nature of an object\comma{} at a fundamental nature. With sufficient control\comma{} you can completely change one thing into another. 

The \imp{transmutation} of humans is a notoriously difficult feat (\imp{Quartum} at least)– especially when the target is unwilling. In such circumstances\comma{} the change of form usually lasts only as long as the caster is able to maintain \imp{concentration}., haszero=1, zero=\spExample{Hue}{Change the colour of an object}, hasone=1, one=\spExample{Simple Object Transformation}{Turn a simple object\comma{} such as a cup\comma{} into another object\comma{} such as a plate\comma{} or alter the material from which the object was created – turning the cup from iron to silver.}, hastwo=1, two=\spExample{Simple Animal Transformation}{Turn a simple creature\comma{} such as a rat or a cat\comma{} into another creature\comma{} or turn them into a simple object.}, hasthree=0, hasfour=0, hasfive=1, five=\spExample{Humanoid Transfiguration}{Turn a humanoid (willing or otherwise) into another form\comma{} forcing them to take on the characteristics (both physical and mental) of their chosen form.}, hassix=0, hasseven=0, hasExamples = 1}


}


\discipline{Bewitchment}
{
	\spell{name = Charm, incantation = Amicus, book = On the Mysteries of the Human Mind, description = You may cause a being to calm\comma{} become positive to you\comma{} or perceive you as a friend and ally\comma{} making them much more likely to listen to you and bend to your words. \imp{Charming} is not mind control\comma{} however: you cannot use this spell to convince someone to do something which is utterly against their moral code\comma{} or to deliberately cause harm to themselves., haszero=1, zero=\spExample{Smile}{Cause a target to experience a moment of pure happiness\comma{} maybe eliciting a smile.}, hasone=1, one=\spExample{Animal Friendship}{A target \imp{Beast} becomes \imp{Charmed} for a short time}, hastwo=0, hasthree=0, hasfour=0, hasfive=0, hassix=0, hasseven=0, hasExamples = 1}

	\spell{name = Conceal, incantation = Intervisio, book = Merlin\apos{}s Tricks and Incantations, description = A \imp{Conceal} spell can make an object or being much harder to detect. At the more powerful end\comma{} you can render entire buildings – cities even – totally invisible\comma{} though less skilled users might have to rely on a more chameleon\minus{}like approach to remaining obscured\comma{} rather than totally hidden., haszero=1, zero=\spExample{Invisible Ink}{Write a message which remains hidden until the spell is dispelled.}, hasone=1, one=\imp{Chameleon}{Make a medium\minus{}sized object such as a book incredibly hard to spot\comma{} though not truly invisible}, hastwo=0, hasthree=1, three=\imp{Fragile Invisibility}{Place a fragile field over a person which renders them perfectly invisible. Care must be taken not to break this field – rapid actions such as running or attacking are liable to cause it to shatter.}, hasfour=0, hasfive=1, five=\imp{Invisibility}{Make a person or object truly invisible\comma{} even when taking actions that would dissolve a weaker spell.}, hassix=0, hasseven=1, seven=\spExample{Inivisibility Dome}{Shroud a large group of people\comma{} or an entire region in an invisibility field\comma{} rendering it impossible to detect.}, hasExamples = 1}

	\spell{name = Distract, incantation = Confundo, book = Jiggery Pockery \& Hocus Pocus, description = A \imp{Distracted} foe is easy to bypass – perhaps you cloud their mind with \imp{confusion} as your allies leap to attack\comma{} or you cause them to lose focus at a crucial moment as you sneak past them. 

A \imp{Distraction} can cause your target to momentarily lose focus\comma{} and even have brief visual or auditory hallucinations., haszero=1, zero=\spExample{Ventriloquism}{Cause your voice to appear from a nearby location.}, hasone=1, one=\spExample{Crash}{Create a loud noise appear to come from some distance away.}, hastwo=1, two=\spExample{Hypnotism}{A being becomes totally focussed on an innocuous detail and loses focus briefly.}, hasthree=0, hasfour=0, hasfive=0, hassix=0, hasseven=0, hasExamples = 1}

	\spell{name = Mirage, incantation = Allucianato, book = Light and Perception: The Magician\apos{}s Mastery, description = A |imp{Mirage} allows you to bend and twist light to create an illusory image with the power of your mind. These images appear to be real on a cursory glance\comma{} but are incorporeal and immaterial\comma{} passing through solid matter. 

At a basic level\comma{} you can create only simple\comma{} static images – with only visual stimuli. A more powerful \imp{illusionist} can (with continued \imp{concentration}) cause their illusions to move and to mimic sounds and smells., haszero=1, zero=\spExample{Sparkle}{Create a small sparkle\comma{} a momentary flicker.}, hasone=1, one=\spExample{Simple Image}{Create a small\comma{} static and temporary illusion\comma{} such as an illusory weapon in your hand.}, hastwo=1, two=\spExample{Illusory Terrain}{Create an image which covers a region of the floor\comma{} making it appear different. Hide away tracks in dust\comma{} or make it appear as if a layer of water covers the floor.}, hasthree=1, three=\spExample{Moving Pictures}{Create a larger illusion\comma{} up to the size of a normal human. You can cause this image to move and respond at your direction. }, hasfour=0, hasfive=0, hassix=0, hasseven=1, seven=\spExample{Illusory World}{Create a True Illusion\comma{} which covers a wide area\comma{} possesses sounds\comma{} smells and is populated by realistic individuals\comma{} controlled at the caster’s command. }, hasExamples = 1}


}


\discipline{Cerebral}
{
	\spell{name = Communicate, incantation = Sermo Colloquius, book = Communing with Others\comma{} and with Yourself, description = The ability to \imp{Communicate} is vital for an adventurer – whether you are establishing a mental link with your allies so that you may communicate at great distances or sending an urgent message halfway across the globe. 

You may also come across beings with whom you share no language\comma{} or encounter runes or pictograms which are meaningless to you: the \imp{communicate} spell might not be flashy\comma{} but it is a cornerstone of magical relations., haszero=0, hasone=1, one=\spExample{Whispercast}{You can whisper a message to a target you can see. They hear the message as if you were next to them.}, hastwo=0, hasthree=1, three=\spExample{Tongues}{You can speak and understand the language of a target living being.}, hasfour=0, hasfive=0, hassix=0, hasseven=0, hasExamples = 1}

	\spell{name = Inspire, incantation = Virtus Animus, book = Mind Beyond Body, description = Project positive energy into your targets\comma{} burning away negative thoughts and bolstering their ability to be brave and commit acts of heroism. When the chips are down\comma{} and your allies have lost the ability to \imp{Endure} your foes attacks\comma{} a little \imp{inspiration} can go a long way., haszero=0, hasone=0, hastwo=1, two=\spExample{Dissolve Terror}{Make a target temporarily immune to \imp{Terror}\comma{} fortifying their bravery.}, hasthree=0, hasfour=0, hasfive=0, hassix=0, hasseven=0, hasExamples = 1}

	\spell{name = Sense, incantation = Revelio, book = Detection is the Best Defense, description = Nothing is ever really as it seems – the ability to broaden your \imp{Senses} to detect the presence of things hidden from your view can be incredibly useful. 

You can use this spell to \imp{sense} the presence of magic spells and traps that might ensnare you\comma{} or feel the presence of nearby humanoids. The most powerful Seers use their \imp{senses} to see beyond all facades and gain `truesight’\comma{} seeing through all forms of invisibility\comma{} deception and concealment., haszero=0, hasone=1, one=\spExample{Detect Magic}{Sense the presence of active enchantments and magical spells.}, hastwo=1, two=\spExample{Detect Traps}{Detect traps\comma{} pitfalls and other mechanisms placed with intent to harm\comma{} delay or alert.}, hasthree=1, three=\spExample{Detect Humanoids}{Sense the presence of human life in your vicinity}, hasfour=0, hasfive=0, hassix=0, hasseven=1, seven=\spExample{Truesight}{You see absolutely everything – you see simultaenously into the astral plane\comma{} as well as through all forms of invisibility\comma{} deception and other forms of concealment. }, hasExamples = 1}

	\spell{name = Slumber, incantation = Somnus, book = The Dream Oracle, description = The world of dreams is a mysterious place – no wizard can truly understand it. With the \imp{Slumber} spell\comma{} however\comma{} a wizard has the ability to view\comma{} enter and manipulate the dreams of a sleeping entity.  

With a sufficient force of will\comma{} you may also force living beings to enter the land of nod – though larger and more intelligent beings requires significantly more power to send to sleep., haszero=1, zero=\spExample{Dreamsense}{Know when a nearby being is dreaming. }, hasone=0, hastwo=1, two=\spExample{Sleep}{If the target fails to resist\comma{} they plunge into a deep sleep.}, hasthree=0, hasfour=0, hasfive=0, hassix=0, hasseven=1, seven=\spExample{Architecture of Nightmares}{You conjure a terrifying nightmare scenario\comma{} trapping a vast number of people in a terrifying fictional world of your own design. All those trapped must find a way to wake themselves\comma{} or continually suffer psychic damage.}, hasExamples = 1}


}


\discipline{Conjuration}
{
	\spell{name = Bind, incantation = Impedimentia, book = Conjuring and Summoning for the Experienced Witch, description = A \imp{Binding} spell allows one to attempt to prevent a foe from escaping – be it by summoning snares to slow their movement\comma{} or ropes to bind them in one place\comma{} rendering them \imp{Trapped}. A more inventive user might also use it to stick two objects together. 

However\comma{} \imp{Binding} is not merely a physical act: a powerful mage may use this spell to \imp{Bind} a summoned creature to their will\comma{} gaining the ability to control powerful extraplanar beings., haszero=1, zero=\spExample{Miniscule Binding}{Hold a small animal such as an insect\comma{} and no larger than a rat\comma{} with magical bindings/}, hasone=1, one=\spExample{Slow}{Summon impediments into the path of a target\comma{} slowing them to half speed.}, hastwo=1, two=\spExample{Hold}{Summon ropes to hold a person in place\comma{} immobilising them.}, hasthree=0, hasfour=0, hasfive=0, hassix=0, hasseven=1, seven=\spExample{Dimensional Binding}{Bind a summoned creature or object to this plane of existence – permanently tying them to your current plane of existence. This spell ends only when dispelled\comma{} or when the creature dies.}, hasExamples = 1}

	\spell{name = Forge, incantation = Confabricor, book = Summoning Your Desires, description = \imp{Forging} is the magical art of creating an entirely new object\comma{} either by assembling existing materials\comma{} or by simply willing it into existence. You might summon a shiny trinket to distract an animal\comma{} or simply conjure a rock in mid\minus{}air above your foe’s head – though larger and more complex objects require more power to \imp{Forge}. 

Summoned objects are almost always lesser in quality than those that they attempt to mimic\comma{} and typically disintegrate into dust after mere minutes to hours., haszero=0, hasone=1, one=\spExample{Simple Object}{Summon small\comma{} simple\comma{} and crude objects – such as a cup or a quill.}, hastwo=1, two=\spExample{Larger objects}{Summon larger objects\comma{} such as tables or chairs}, hasthree=1, three=\spExample{Summon Weapon}{Summon an object with enough structural integratity to be used as a weapon.}, hasfour=1, four=\spExample{Dimensional Tether}{Bind a \imp{Manifested} being to you for a longer period of time.}, hasfive=0, hassix=0, hasseven=0, hasExamples = 1}

	\spell{name = Manifest, incantation = Sortia, book = The Demons Beyond the Veil, description = A \imp{Manifestation} is a magical spell which conjures a living being out of nothing\comma{} fabricating them from raw magic\comma{} or pulling an existing creature from some other plane of existence. You might be forced to start with pulling bunnies out of a metaphorical hat\comma{} but before you know it\comma{} you can summon a swarm of \imp{Dementors} to chase your foes to hell and back. 

Summoned beings are generally created with a positive attitude towards you\comma{} and will obey your commands (to a limited extent)\comma{} as if they were a trained animal or friendly person. Summonings generally disintegrate within a few minutes unless bolstered by a \imp{Bind}\comma{} or maintained through \imp{Concentration}., haszero=1, zero=\spExample{Summon Tiny Beast}{Summon a tiny animal such as a sparrow or an insect.}, hasone=1, one=\spExample{Summon Small Beast}{Summon a larger animal\comma{} such as a cat\comma{} a niffler or a snake.}, hastwo=0, hasthree=0, hasfour=0, hasfive=0, hassix=0, hasseven=1, seven=\spExample{Summon Magnificent Beings}{Summon the most powerful magical creatures: giant \imp{Draconids}\comma{} otherworldly \imp{Celestials} and despicable \imp{Monsters} and \imp{Abominations}.}, hasExamples = 1}

	\spell{name = Vanish, incantation = Evanesco, book = Making and Unmaking: The Art of Conjuration, description = When you \imp{vanish} an object or creature\comma{} you force it through the cracks in reality; sending them into nothingness and oblivion\comma{} or back to whatever \imp{Realm} they originally hailed from. 

For all but the most powerful wizards\comma{} this spell only works on entities which have been \imp{Manifested} or \imp{Forged} by their own hand– to vanish a creature summoned by another requires a certain amount of power\comma{} and banishing a truly\minus{}real thing of any particular size or power is a very difficult act indeed., haszero=1, zero=\spExample{Self\minus{}Dismissal}{Dismiss a summoned creature which remains under your control.}, hasone=0, hastwo=0, hasthree=0, hasfour=0, hasfive=0, hassix=0, hasseven=0, hasExamples = 1}


}


\discipline{Curse}
{
	\spell{name = Corrupt, incantation = Vitiosus, book = The Bumper Book of Crooked Curses, description = A being afflicted with a \imp{Corrupting} curse finds that their own abilities have been turned against them: the magical effects of an enchanted item have been warped into something darker\comma{} a healing potion can become tainted with acid\comma{} and maybe a living being’s hands disobey them\comma{} or attempt to throttle their owner.

The \imp{Corrupt} spell is the vindictive mage’s most powerful tool\comma{} as it allows one to take revenge in as imaginative way as possible., haszero=0, hasone=0, hastwo=0, hasthree=0, hasfour=0, hasfive=0, hassix=0, hasseven=0, hasExamples = 0}

	\spell{name = Disarm, incantation = Expelliarmus, book = A Compendium of Common Curses, description = With a \imp{Disarm} spell you may de\minus{}fang a target by removing their means of attack: enemies will drop their wands\comma{} or a beasts claws would retract or become blunted., haszero=0, hasone=0, hastwo=1, two=\spExample{Arms}{Any weapons being carried by the target are immediately dropped at their feet as their hands spasm}., hasthree=0, hasfour=1, four=\spExample{Improved Arms}{Held weapons and wands are thrown away a great distance.}, hasfive=0, hassix=0, hasseven=0, hasExamples = 1}

	\spell{name = Infect, incantation = Ictus, book = Voodoo and Vomiting: A Study in Curses, description = An \imp{infecting} curse is a particularly unpleasant spell which causes a target to suffer from a creeping disease\comma{} poison\comma{} contagion or illness which gradually impedes their abilities and their senses. A weaker \imp{Infect} might mimic a vicious insect sting\comma{} but in the hands of an evil wizard\comma{} necrotic\comma{} flesh\minus{}eating diseases are not unheard of., haszero=1, zero=\spExample{Sting}{A minor moment of discomfort\comma{} and a small amount of swelling\comma{} as might be expected from an insect sting.}, hasone=0, hastwo=1, two=\spExample{Disease}{Infect the target with a novel disease of your choosing}, hasthree=0, hasfour=0, hasfive=0, hassix=0, hasseven=0, hasExamples = 1}

	\spell{name = Stun, incantation = Stupefy, book = Curses \& Counter Curses, description = A victim of a \imp{Stunning} curse finds their ability to function severely impaired. A weaker \imp{Stun} might cloud their mind for a moment\comma{} \imp{Confusing} them for a while\comma{} or simply \imp{trap} them in one place\comma{} but a more powerful stun might render them \imp{immobilised} or \imp{unconscious} for a significant period of time., haszero=0, hasone=1, one=\spExample{Stun}{A target becomes dazed and confused for a shrot time (1 round)\comma{} and takes a 2d penalty to any actions they take.}, hastwo=0, hasthree=1, three=\spExample{Knockout}{A living being up to the size of a human is knocked unconscious for a short period of time (2 rounds). During this time they can take no actions.}, hasfour=0, hasfive=0, hassix=0, hasseven=1, seven=\spExample{Dragonsleep}{Stun and immobilise even the largest and most powerful of magical creatures.}, hasExamples = 1}


}


\discipline{Elemental}
{
	\spell{name = Burn, incantation = Incendio, book = Igniting the Spark: An Introduction to Elemental Magic, description = Manipulate\comma{} create and extinguish the primal force of fire and heat. You may summon jets of fire from your wand\comma{} or fling it at your opponent as a flaming ball of heat and combustion. 

You might also \imp{concentrate} on an existing fire and manipulate it\comma{} shape it\comma{} douse it or control it as you desire., haszero=1, zero=\spExample{Spark}{Summon a tiny spark of flame\comma{} perhaps enough to catch dry tinder.}

\spExample{Smother}{Douse tiny embers and sparks.}

\spExample{Flicker}{Cause mundane fires to flicker\comma{} dance or glow brighter.}, hasone=1, one=\spExample{Flame}{A small\comma{} continuous jet of flame extends from your wand\comma{} causing burns on contact.}

\spExample{Sense Fire}{Sense the presence of fires.}

\spExample{Minor Manipulation}{Control and manipulate the shape of nearby fire\comma{} causing it to take on a chosen shape\comma{} or to direct it to slowly move in a certain direction.}, hastwo=1, two=\spExample{Far\minus{}Flung Flames}{Ignite a stationary target from a distance.}

\spExample{Extinguish}{Suppress and extinguish a fire over a small region.}, hasthree=1, three=\spExample{Firebolt}{Fire a bolt of flame at a foe\comma{} dealing increased \imp{Harm}}

\spExample{Red\minus{}hot}{Summon fire which is hot enough to ignite even resistant materials such as damp wood}, hasfour=1, four=\spExample{Fireball}{Launch a fireball which explodes on contact\comma{} hurting nearby foes with fiery concussive force}

\spExample{Firebending}{Rapidly control an existing fire at great distance\comma{} causing it to float through space or lash out at a foe like a weapon.}, hasfive=1, five=\spExample{Firewall}{Create an enormous wall of fire as a barrier across a defined space\comma{} incinerating those who would pass through it.}, hassix=1, six=\spExample{White\minus{}hot}{Summon and project fire hot enough to melt through metal and liquefy stone. }, hasseven=1, seven=\spExample{Firestorm}{Summon a fiery maelstrom to envelop a large area such as an entire building\comma{} or devastate huge numbers of foes}, hasExamples = 1}

	\spell{name = Flood, incantation = Augamente, book = Laughable Liquidation, description = With the \imp{Flood} spell\comma{} one gains the ability to manipulate elemental water – as well as any fluidic substance which contains a large amount of water. You can also summon jets of pure water from the end of your wand – either for drinking purposes\comma{} to put out a raging inferno\comma{} or to blast an enemy in the face with., haszero=1, zero=\spExample{Mist}{Summon a fine mist from the tip of your wand.}, hasone=1, one=\spExample{Minor Manipulation}{Control and manipulate the shape of nearby bodies of water\comma{} causing the\comma{} to take on a chosen shape\comma{} or to direct it to flow in a certain direction.}, hastwo=0, hasthree=0, hasfour=0, hasfive=0, hassix=0, hasseven=1, seven=\spExample{Tsunami}{An enormous tidal wave surges from your wand\comma{} enough to destroy buildings and decimate armies.}, hasExamples = 1}

	\spell{name = Freeze, incantation = Glacius, book = Secrets of Elemental Sorcery, description = Project and control freezing blasts of air\comma{} and manipulate ice and other frozen objects\comma{} or summon a blizzard of snow to obscure and protect your allies from view. 

Few witches are as feared as those who fling gigantic icicles at their foe as they ride a glacier galloping down a valley., haszero=1, zero=\spExample{Snowflake}{Cause a small shower of snowflakes from your wand}

\spExample{Cool}{Cause a region to cool down slightly\comma{} like putting an ice cube in a drink.}, hasone=1, one=\spExample{Frostbite}{A beam of freezing cold energy causes a small amount of liquid to freeze\comma{} and deals \imp{Cold} damage to a foe.}, hastwo=1, two=\spExample{Insta\minus{}freeze}{Cause a small amount of liquid\comma{} up to a small pond\comma{} to instantly freeze solid.}, hasthree=0, hasfour=0, hasfive=0, hassix=0, hasseven=0, hasExamples = 1}

	\spell{name = Gust, incantation = Ventillio, book = Storms\comma{} Seas and Seismic Shocks, description = Bend the winds to your command\comma{} generate mighty gusts of wind to hamper your foes\comma{} distract opponents\comma{} or summon gigantic storms to tear entire buildings to the ground. 

It is said that the \imp{Gust} spell was the secret to \imp{Voldemort}’s human\minus{}flight spell\comma{} though this is understandably a taboo subject., haszero=1, zero=\spExample{Murmuration}{Cause a small gust of wind to rush through an area from no apparent source.}, hasone=0, hastwo=0, hasthree=0, hasfour=0, hasfive=0, hassix=0, hasseven=1, seven=\spExample{Tornado}{Summon and control a mighty pillar of twisting air\comma{} which picks up\comma{} traps and eventually hurls anything it comes into contact with.}, hasExamples = 1}

	\spell{name = Illuminate, incantation = Lumos, book = Light in the Darkness: The Forgotten Element, description = At its most basic level\comma{} \imp{Illumination} allows one to see in the dark – either by igniting the tip of your wand\comma{} activating disused lights\comma{} or floating globules of sunlight. You might even stretch yourself to do the opposite\comma{} and turn out the lights\comma{} or bend and redirect the light at your will. 

However\comma{} the ability to direct bright sunlight into your foes is surprisingly useful – rendering someone \imp{blind} is not to be sniffed at\comma{} and with sufficient focus one can focus the beams of light to deal \imp{Incandescent} damage\comma{} which is particularly harmful to the evil and undead creatures which inhabit the world., haszero=1, zero=\spExample{Glowing Wand}{Cause the tip of your wand to glow with light.}, hasone=1, one=\spExample{Blinding Lance}{Target a creature and send a bolt of bright light towards them. If it strikes\comma{} they are blinded for a short time (1 turn).}, hastwo=0, hasthree=1, three=\spExample{Daylight}{Summon a radiance from your wand or a nearby surface\comma{} which illuminates the room as if it were midday. Vampires and other creatures affected by sunlight treat it as real sunlight\comma{} and react accordingly.}, hasfour=0, hasfive=0, hassix=0, hasseven=0, hasExamples = 1}

	\spell{name = Sculpt, incantation = Gaius, book = Further Elemental Studies, description = Manipulate the ever\minus{}present Earth beneath your feet\comma{} digging mighty trenches or raising enormous walls\comma{} or simply causing nearby stones to pelt your foe in the face. 

Earth\minus{}wizards have been known to create warrens of tunnels\comma{} or move through the earth like one might walk on land – there are even whispers of mages bringing entire cities to ruin with an earthquake., haszero=0, hasone=1, one=\spExample{Pebbledash}{Cause  stones and loose earth to fling themselves at your foe\comma{} dealing \imp{bludgeoning damage}.}, hastwo=1, two=\spExample{Move Earth}{Move large amounts of earth and loose stone\comma{} digging trenches and building simple earthworks.}, hasthree=1, three=\spExample{Raise Pillars}{Instantly cause a small number of stone pillars to rise out of the Earth\comma{} to a height of 10 metres.}, hasfour=0, hasfive=0, hassix=0, hasseven=1, seven=\spExample{Earthquake}{Cause the earth in an enourmous radius around the target to shake\comma{} destroying buildings and dealing devastating damage.}, hasExamples = 1}

	\spell{name = Shock, incantation = Baubilius, book = The Fundamental Power, description = There are few joys in the world as great as summoning a gigantic bolt of lightning from the end of your wand – one might call the experience {\it electric}. 

The ability to manipulate the force of electricity is somewhat more nuanced\comma{} however – from the tiniest spark\comma{} to the mightiest thunderstorm. Perhaps one of the most unused aspects of this spell is its ability to manipulate muggle electronic equipment\comma{} by manipulating the flow of energy in their devious contraptions., haszero=1, zero=\spExample{Jolt}{Cause a harmless contact shock.}, hasone=1, one=\spExample{Redirect Current}{You allow a current to pass through something without harm\comma{} or redirect an electrical flow at your will.}

\spExample{Thunderhands}{Release a powerful jolt when your wand touches something\comma{} dealing \imp{Electric} damage.}, hastwo=0, hasthree=1, three=\spExample{Lightning Bolt}{A jagged bolt of electrical energy surges from the end of your wand.}, hasfour=0, hasfive=0, hassix=0, hasseven=1, seven=\spExample{Thunderstrom}{Summon a mighty electrical storm overhead\comma{} and direct bolts of lightning to attack your foes.}, hasExamples = 1}


}


\discipline{Hermetics}
{
	\spell{name = Heal, incantation = Enervate, book = Cures\comma{} Cantrips and Coughs, description = A \imp{Healing} spell restores life to the target\comma{} causing their wounds to knit shut\comma{} broken bones to set and repair and torn muscles to heal. Basic healing spells simply heal the surface wounds – more complex injuries require a more powerful version of the spell\comma{} and the ability to regrow limbs or repair organs from 

A basic \imp{Heal} spell removes one level of harm for each point of \imp{Power}\comma{} but cannot remove \imp{status} effects\comma{} and has no effect on patients who are in a \imp{critical condition} or worse\comma{} \imp{dead}., haszero=1, zero=\spExample{Kind Words}{Sooth a patient\comma{} heal tiny scrapes and bruises\comma{} but not restoring any \imp{Damage}.}, hasone=1, one=\spExample{Minor Healin}{Channel life\minus{}giving energy into a patient\comma{} giving their body the energy to heal itself\comma{} restoring one level of \imp{Harm}. }, hastwo=0, hasthree=1, three=\spExample{Knit Bones}{Cause broken bones to rapidly repair themselves\comma{} healing larger amounts of harm.}, hasfour=0, hasfive=0, hassix=1, six=\spExample{Regeneration}{Regenerate lost limbs\comma{} regrow bones\comma{} and otherwise vastly accelerate the natural healing process.}, hasseven=1, seven=\spExample{Spark of Life}{Nurture a being which has suffered so much damage it seems dead\comma{} reignite the final sparks of life in a being which died only a few moments ago. Cannot truly revive the dead\comma{} but merely save those who are about to pass the threshold.}, hasExamples = 1}

	\spell{name = Purify, incantation = Expecto Patronum, book = Manipulating the Forces of Life, description = The \imp{Purify} spell\comma{} also known as the `Patronus charm’\comma{} projects powerful positive energies which protect against necrotic or necromantic forces. 

This energy can burn away rot and decay from food\comma{} counter powerful curses and protect against defiling spells which raise \imp{Inferi}. 

Beings which are an antithesis to life\comma{} such as demons and dementors\comma{} find this energy repulsive and harmful – some mages have been known to summon a powerful corporeal guardian using this spell\comma{} though this is recognised as an incredibly advanced feat., haszero=0, hasone=0, hastwo=0, hasthree=0, hasfour=0, hasfive=0, hassix=0, hasseven=0, hasExamples = 0}

	\spell{name = Restore, incantation = Episkey, book = Journals of St. Mungo\comma{} the Master Healer, description = Not all afflictions take the form of cuts and bruises – for more complex afflictions you will need to rely on the \imp{Restore} spell. This spell helps cure poisons and diseases\comma{} remove the effects of magical terror and counter other negative status effects originating from within the body., haszero=0, hasone=1, one=\spExample{Minor Restoration}{Remove minor status effects such as weak poison\comma{} weak burns etc.}, hastwo=0, hasthree=1, three=\spExample{Sustain Life}{Remove the \imp{Critical Condition} status from a being which has been unconscious for 30 seconds or less.}, hasfour=0, hasfive=0, hassix=0, hasseven=1, seven=\spExample{Master Reset}{Remove all ailments from a target\comma{} restoring their body to perfect health in every possible way – even down to correcting vision\comma{} straightening spines. The target is left as perfect as possible.}, hasExamples = 1}

	\spell{name = Sustain, incantation = Omnium, book = Life\comma{} and How to Preserve it, description = The \imp{Sustain} spell allows one to project nourishing energy into a target. 

Herbologists and gardeners use this spell to rapidly grow plants\comma{} pushing the life\minus{}giving energy into them\comma{} shaping and controlling their growth. The great herbologist Linneus was said to be able to grow an oak from an acorn in mere minutes\comma{} and once shaped a tree into an obscene gesture as an act of defiance. 

When directed at a human (infinitely more complex than a plant)\comma{} this typically allows them to forgo the need to breath oxygen for a while\comma{} or replaces the need to eat – their lifeforce being sustained through magic rather than mundane means\comma{} for a short while., haszero=0, hasone=1, one=\spExample{Thermoregulate}{Allow a being to exist in exceedingly hot or cold environments without becoming unduly unvomfortable.}, hastwo=0, hasthree=1, three=\spExample{Waterbreathing}{Allow a target to go without air for a short time.}, hasfour=0, hasfive=0, hassix=0, hasseven=0, hasExamples = 1}


}


\discipline{Hex}
{
	\spell{name = Disintegrate, incantation = Reducto, book = Basic Hexes for the Busy \& Vexed, description = When one wishes to tear into a target with great magical ferocity\comma{} or reduce an entity to ash and dust\comma{} the \imp{Disintegrate} spell is surely the tool to use. 

A very dangerous tool in the wrong hands\comma{} this is a very powerful offensive weapon., haszero=0, hasone=0, hastwo=1, two=\spExample{Tearing Pulse}{A blast of energy attempts to rip a target apart.}, hasthree=0, hasfour=0, hasfive=0, hassix=0, hasseven=1, seven=\spExample{Disintegrating Blast}{A shockwave of energy emerges around you for dozens of metres. Everything which is touched by the shockwave is instantly reduced to ash and dust.}, hasExamples = 1}

	\spell{name = Force, incantation = Flipendo, book = Hexes to Make Your Head Spin (Literally), description = A \imp{Force} spell allows one to manipulate and create shockwaves and walls of ethereal force which push and crush your enemies. 

A \imp{Force} spell often has two effects (harming a foe and pushing them around)\comma{} it is up to the caster to decide how to much power is delegated to each effect., haszero=0, hasone=1, one=\spExample{Knockback}{A small shockwave extends from your wand\comma{} harming a foe and pushing them backwards.}, hastwo=1, two=\spExample{Forcewave}{A shockwave extends from all around you\comma{} slamming into  nearby beings\comma{} harming them and pushing them outwards and away.}, hasthree=0, hasfour=0, hasfive=1, five=\spExample{Bonecrusher}{Summon a giant wall of force to pin a foe against a wall\comma{} floor\comma{} or ceiling\comma{} dealing immense damage and breaking their bones.}, hassix=0, hasseven=0, hasExamples = 1}

	\spell{name = Jinx, incantation = Verdimillius, book = Hexing Your Problems \minus{} the Healthy Coping Mechanism, description = A \imp{Jinx} is a multi\minus{}purpose offensive spell\comma{} with the exact effect greatly affected by the intent of the caster. 

Whilst many \imp{Jinxes} focus on hurting the target\comma{} it is common for the caster to imbue the spell with other side effects – perhaps the spell forces them to roll around on the floor laughing as they are tickled\comma{} or their hands reverse on their wrists to face the wrong direction. The exact effects imbued\comma{} and their relative power\comma{} is up to the caster – the trick to winning a duel is often in using your \imp{Jinxes} effectively!, haszero=0, hasone=1, one=\spExample{Rainbow Sparks}{A jet of coloured sparks emerges from the end of your wand\comma{} striking the foe. The colour of the sparks determines the kind of damage done (I.e red = fire\comma{} blue = ice). }, hastwo=1, two=\spExample{Tickling Jinx}{The target is harmed by the blast\comma{} and must resist the urge to laugh uncontrollably.}, hasthree=0, hasfour=1, four=\spExample{Bat\minus{}Bogeys}{Cause the snot in a targets nose to take on the form of a bat\comma{} and attack them\comma{} dealing \imp{necrotic damage.}}, hasfive=0, hassix=0, hasseven=0, hasExamples = 1}

	\spell{name = Strike, incantation = Sectumsempra, book = Dark Forces: A Guide to Self\minus{}Defense, description = A spell designed by the erstwhile potions master of Hogwarts\comma{} the \imp{Strike} spell allows a \imp{battlemage} to land mighty hammer blows\comma{} or vicious sword strike on their foes – without the difficulty of needing a weapon\comma{} or knowing how to use it. 

The \imp{Strike} spell allows one to slice\comma{} stab or cut into the flesh of your opponent – though a more peaceful minded mage might repurpose this spell to cut down trees\comma{} or hammer in nails., haszero=1, zero=\spExample{Pinprick}{Inflict a small\comma{} short pain on your target – more annoying than harmful.}, hasone=1, one=\spExample{Swordstroke}{Mimic a swordstrike with your wand.}, hastwo=0, hasthree=1, three=\spExample{Bloody Slash}{Rake deep\comma{} bloody wounds into your target\comma{} dealing immense physical damage. }, hasfour=0, hasfive=0, hassix=0, hasseven=0, hasExamples = 1}


}


\discipline{Kinesis}
{
	\spell{name = Apparate, incantation = Cruratele, book = The Matter of Mass, description = \imp{Apparating} is a form of teleportation\comma{} and can cause an object to vanish and reappear at another point in space. Teleporting only really works if the caster is incredibly familiar with the target location – preferably having spent more than a trivial amount of time there\comma{} or being currently able to see it.

Apparating can often be quite dangerous\comma{} especially when trying to transport living beings\comma{} as the risk of \key{splinching} oneself is quite high\comma{} so it is recommended that this only be attempted by trained individuals., haszero=1, zero=\spExample{Tiny Beaming}{Cause a small object\comma{} such as a pebble\comma{} to teleport to another location within 1m.}, hasone=1, one=\spExample{Sightcast}{Target a small object (i.e. enough that you could comfortably lift)\comma{} and teleport it to a region that you can see.}, hastwo=0, hasthree=1, three=\spExample{Apparate}{Teleport yourself to a location that you are intimitely familiar with.}, hasfour=0, hasfive=0, hassix=0, hasseven=1, seven=\spExample{Fargate}{Create a long\minus{}lasting portal\comma{} anchored at some other point in space. Beings which enter the portal are instantly transported to the other end. }, hasExamples = 1}

	\spell{name = Halt, incantation = Stabit, book = The Standard Book of Spells, description = Is something\comma{} or someone\comma{} moving and you don’t want it to? The \imp{Halt} spell is here to help. 

From catching projectiles\comma{} or halting a fleeing foe in their tracks\comma{} the \imp{Halt} spell is bound to have endless uses. It is rumoured that a particularly powerful version of this spell is used at \imp{Gringotts} – when an intruder is detected\comma{} a field is projected which makes it impossible for anyone or anything to move within a certain region., haszero=0, hasone=1, one=\spExample{Catch Projectile}{Prevent a single moving object such as an arrow\comma{} rock or similar projectile from moving\comma{} holding it in mid\minus{}air.}, hastwo=0, hasthree=0, hasfour=0, hasfive=0, hassix=0, hasseven=1, seven=\spExample{Immobility Field}{Imbue a huge region with an enormous amount of energy which renders it impossible for anything to move in the area. }, hasExamples = 1}

	\spell{name = Move, incantation = Wingardium Leviosa, book = Achievements in Charming, description = When one \imp{Move}s an object\comma{} you use magic to float it around in space. The most basic version of this spell simply allows you to lift objects vertically upwards – with more finesse and control\comma{} you gain the ability to telekinetically move objects around in 3D space – though heavier objects require much more power to lift., haszero=0, hasone=1, one=\spExample{Simple Levitation}{Cause a light object to move up and down vertically\comma{} or hover in place.}, hastwo=1, two=\spExample{Telekinesis}{Move an object around\comma{} freely\comma{} in 3D space. }, hasthree=0, hasfour=0, hasfive=0, hassix=0, hasseven=1, seven=\spExample{Mass Kinesis}{Cause many thousands of objects to dance and move at your command}, hasExamples = 1}

	\spell{name = Repair, incantation = Reparo, book = Motion: A Treatise, description = The \imp{Repair} spell allows one to reassemble a broken object and restore its non\minus{}magical functionality\comma{} the broken remains slot back into their respective places and seal any cracks or breaks. 

Generally\comma{} the caster must have a rough idea what the broken object is before they can attempt this spell\comma{} and the vast majority of the object must be present for it to be fixed – more powerful \imp{Repair} jobs can sidestep this issue\comma{} and can repair things even if it has been totally eviscerated and reduced to ash., haszero=0, hasone=1, one=\imp{Seal Cracks}{Cause the cracks in an object to reseal to their original condition.}, hastwo=0, hasthree=1, three=\imp{Magical Ressembly}{Cause a broken and shattered object to reassemble itself}, hasfour=1, four=\spExample{Morphic memory}{Cause an item to repair itself beyond simple physical form: burned books recover their text\comma{} for example\comma{} technological devices have their hard\minus{}drives restored and so on.}, hasfive=0, hassix=0, hasseven=1, seven=\spExample{Architectural Restoration}{Restore entire buildings\comma{} towns and cities\comma{} healing them of the effects of mass destruction.}, hasExamples = 1}

	\spell{name = Seek, incantation = Accio, book = Extreme Incantations, description = A \imp{Seek} spell imbues the target with purpose: it will strive to reach the specified target\comma{} through any means necessary. Some wizards find this spell useful to summon objects to them\comma{} whilst others use it to create snowballs which unerringly seek out their target., haszero=0, hasone=1, one=\spExample{Homing Projectile}{Target a projectile\comma{} and imbue it with a target and purpose: the  projectile becomes much more likely to hit its next target.}, hastwo=0, hasthree=1, three=\spExample{Summon}{Speak aloud the name of a nearby object you are familiar with\comma{} and whose location you know\comma{} and cause it to fly towards you.}, hasfour=0, hasfive=0, hassix=0, hasseven=1, seven=\spExample{Global Summons}{Holding a picture of an item your are familiar with in your mind\comma{} cause it to instantly appear at your location\comma{} from anywhere in the world.}, hasExamples = 1}


}


\discipline{Necromancy}
{
	\spell{name = Blight, incantation = Carnes Mortis, book = Defying Nature, description = \imp{Blight} spells are evil acts which project awful necrotic energies which sap at the life force of all living beings. 

Plants subjected to this spell will wither and die\comma{} and the wounds on living beings fester and become gangrenous as the \imp{Necrotic} damage seeps into their flesh., haszero=0, hasone=1, one=\spExample{Necrosis}{A bolt of necromantic energy launches at a foe\comma{} causing their flesh to rot and dealing \imp{Necrotic damage}.}, hastwo=0, hasthree=0, hasfour=0, hasfive=0, hassix=0, hasseven=1, seven=\spExample{Death Zone}{Create a cursed region\comma{} so filled with necromantic energy that all living beings are damaged simply by their presence: weaker beings and plants instantly die off\comma{} whilst stronger beings suffer \imp{harm} as long as they remain in the zone.}, hasExamples = 1}

	\spell{name = Defile, incantation = Sanguinus, book = Magick Moste Evile, description = When one uses a spell to \imp{Defile}\comma{} you conduct a profane blood\minus{}ritual which allows you to place powerful life\minus{}long curses on entire bloodlines\comma{} or corrupt powerful enchantments on a place\comma{} turning them to darkness for your own end. 

The most powerful curses are cast using a \imp{Defile} spell., haszero=0, hasone=0, hastwo=1, two=\spExample{Bloodcurse}{Cause the blood of a target to boil like acid in their veins\comma{} causing immeasurable pain.}, hasthree=0, hasfour=0, hasfive=0, hassix=0, hasseven=0, hasExamples = 1}

	\spell{name = Drain, incantation = Exbibo Maledictum, book = Necromancy: A Misunderstood Skill, description = The \imp{Drain} spell uses vampiric energy to channel life force and other magical energies out of a living being\comma{} and into another being. 

This energy can be used to fortify their own health\comma{} or used to power obscene acts of magic., haszero=1, zero=\spExample{Vampric Snack}{Drain the life out of a small animal\comma{} such as an insect\comma{} weaking (but not killing) it. Use its life force to sustain you in place of consuming food or drink.}, hasone=0, hastwo=0, hasthree=0, hasfour=0, hasfive=0, hassix=0, hasseven=0, hasExamples = 1}

	\spell{name = Kill, incantation = Avada Kedavra, book = Spelles Moste Vyle, description = The \imp{Killing Curse} is one of the \imp{Unforgivable Curses}\comma{} which severs the link between the soul and the body\comma{} causing death wherever it leads. 

If a \imp{Kill} spell would reduce a target to \imp{Critical Condition}\comma{} they are instead instantly killed. If the spell does not kill them\comma{} the target instead takes no harm\comma{} but may bleed from the eyes or nose., haszero=1, zero=\spExample{Kill Insects}{End the life of tiny creatures such as insects.}, hasone=1, one=\spExample{Kill Small Animals}{Kill small mammals and magical creatures such as mice\comma{} nifflers and small birds.}, hastwo=1, two=\spExample{Kill Medium Animals}{Kill medium\minus{}sized animals such as cats\comma{} dogs\comma{} nogtails and other animals of a similar size.}, hasthree=1, three=\spExample{Kill Large Animals}{End the lives of large animals\comma{} such as horses\comma{} cattle\comma{} fire crabs\comma{} kappas and suchlike.}, hasfour=1, four=\spExample{Kill Humanoids}{Instantly execute undefended humanoids.}, hasfive=1, five=\spExample{Kill Lesser Monsters}{Kill powerful magical creatures such as runespoorts\comma{} wolf\minus{}form werewolves\comma{} occamy\comma{} trolls and so on.}, hassix=1, six=\spExample{Kill Greater Monsters}{Destroy some of the most powerful beasts around: dragons\comma{} acromantula\comma{} nundu and so on.}, hasseven=1, seven=\spExample{Kill Celestials}{Kill powerful\comma{} otherworldly creatures which normally transcend petty concerns of mortality.}, hasExamples = 1}

	\spell{name = Raise, incantation = Inferi pareo, book = Secrets of the Darkest Art, description = Mighty \imp{Necromancers} of yore \imp{Raised} entire armies of undead servants to follow them into battle. Thuogh when first attempted you might struggle to animate even a mouse’s skeleton\comma{} powerful necromancers can \imp{Raise} thousands upon thousands of skeletons\comma{} dozens of \imp{Inferi} and other more powerful undead., haszero=0, hasone=1, one=\spExample{Skeleton}{Animate the bones of a long\minus{}dead humanoid\comma{} becoming a disgusting undead servant.}, hastwo=0, hasthree=1, three=\spExample{Inferi}{Raise a zombie\minus{}like creature which will mindlessly attack your foes.}, hasfour=0, hasfive=0, hassix=0, hasseven=1, seven=\spExample{Army of Undeath}{Raise thousands upon thousands of skeletons and inferi and other horrifying creatures to serve you.}, hasExamples = 1}


}


\discipline{Occultism}
{
	\spell{name = Attune, incantation = Thanatos, book = Theories on the World Beyond, description = When you use magics which allow you to \imp{Attune}\comma{} you focus your mind on the realm of the spirits and ghosts. You can use this spell to see them\comma{} and manipulate spiritual creatures on the \imp{Mortal Realm} and beyond to your desire\comma{} compelling or persuading them to follow your requests., haszero=0, hasone=1, one=\spExample{Ghostcall}{Speak the name of a ghost aloud\comma{} and they will feel the urge to come and talk to you.}, hastwo=0, hasthree=0, hasfour=0, hasfive=0, hassix=0, hasseven=0, hasExamples = 1}

	\spell{name = Consort, incantation = Profundo Diabolus, book = The Outer Planes, description = \imp{Consorting} is a dangerous set of magic which allows one to summon\comma{} command and converse with demonic entities. 

You may call on them to serve your will\comma{}  or answer your questions – but beware that without great \imp{concentration}\comma{} these beasts will often try to turn on their master., haszero=0, hasone=0, hastwo=0, hasthree=0, hasfour=0, hasfive=0, hassix=0, hasseven=0, hasExamples = 0}

	\spell{name = Eclipse, incantation = Umbra, book = An A\minus{}Z of Spooky Spells, description = The \imp{Eclipse} spell gives one control over darkness and shadows\comma{} allowing one to manipulate them to your will. You can use this to plunge a region into darkness\comma{} sneak in shadow whilst the sun beats down\comma{} or to hurl necrotic blasts of shadow energy at your foe., haszero=0, hasone=1, one=\spExample{Darkness}{Create a magical darkness to shroud a small region\comma{} extinguishing other light sources.}, hastwo=1, two=\spExample{Shadowblast}{Hurl  shadows at your foes to tear and their flesh.}, hasthree=0, hasfour=0, hasfive=0, hassix=0, hasseven=1, seven=\spExample{Blot the Sun}{Summon a darkness so complete it blots out the sun and the moon: Vampires\comma{} Werewolves and other such creatures gain full control over their abilities.}, hasExamples = 1}


}


\discipline{Psionics}
{
	\spell{name = Compel, incantation = Imperio, book = The Immoral Art of Psionics, description = When you \imp{Compel} someone\comma{} your force a them to obey your commands\comma{} layering your words with power.  The target will unthinkingly obey your words as long as you maintain \imp{Concentration}. 

Where there is conflict between the effectiveness of a \imp{Compel} vs a \imp{Delude}\comma{} remember that the effects of a \imp{Compel} tend to be larger and more dramatic than a \imp{Delude}\comma{} but in turn a shorter lived., haszero=0, hasone=1, one=\spExample{Command}{Give a one word command\comma{} which the target will attempt to follow for a short period of time (1 round)}, hastwo=0, hasthree=1, three=\spExample{Enrage}{Force fury and anger into the mind of a target\comma{} and cause them to attack everything in sight.}, hasfour=0, hasfive=0, hassix=0, hasseven=1, seven=\spExample{Willing Slave}{You take complete control of the target\comma{} bending them to your will so utterly that they cannot resist. The target will act as your most loyal follower\comma{} taking every command and executing it to best serve you.}, hasExamples = 1}

	\spell{name = Delude, incantation = Credo Apina, book = Cool Cantrips to Make You Crazy, description = A \imp{Delude} spell allows you to sneak an idea into a being’s very mind\comma{} causing them to believe it as absolute truth. A benevolent mage might use this to give a being confidence in their own abilities\comma{} or distract someone with false images inside their head\comma{} whilst a malicious one might convince their enemy that they can fly by jumping off a tall building…, haszero=0, hasone=1, one=\spExample{Ghostly lights}{Make the target see floating lights\comma{} or some other simple illusion\comma{} which no one else can perceive. }, hastwo=0, hasthree=1, three=\spExample{False Opponent}{Cause a target to see a creature or being. This being can interact with them at the caster’s command\comma{} and (unless they resist) the target will act as if the other being was there. If the target causes harm to them\comma{} they take \imp{Level One} psychic damage. }, hasfour=0, hasfive=1, five=\spExample{Cognitive Dissonance}{Using a single sentence\comma{} describe an idea. The target will believe this idea with all their heart\comma{} even if it totally contradicts everything else they know about reality. They will act on this idea as if it were their own\comma{} deeply held belief.}, hassix=0, hasseven=0, hasExamples = 1}

	\spell{name = Remind, incantation = Obliviate, book = The Apotheosis of the Psionics, description = The \imp{Remind} spell covers the domain of memory – with this spell you can force a target to relive a specific memory\comma{} or you can tear it from their skull to wipe their mind. 

With sufficient control you can fabricate enter reams of memory\comma{} giving a target a completely new recollection of events., haszero=1, zero=\spExample{Fortify Memory}{Help a target remember a fact – grant +1d to a knowledge check.}, hasone=0, hastwo=0, hasthree=0, hasfour=0, hasfive=0, hassix=0, hasseven=1, seven=\spExample{Personality Rewrite}{Completely alter the memories of the target\comma{} sculpting them to your will. Give them an entirley fictional set of memories\comma{} or simply alter their existing ones to better suite your needs.}, hasExamples = 1}

	\spell{name = Torture, incantation = Crucio, book = Mindmakers\comma{} Mindbreakers, description = The \imp{Torture} spell allows you to inflict huge amounts of pain on a target\comma{} without leaving a visible mark anywhere. 

The target feels immense amounts of pain as you direct\comma{} feeling as if they are on fire\comma{} or being stabbed with a dozen swords. More powerful versions of this spell allow one to \imp{immobilise} a foe whilst you maintain \imp{Concentration}\comma{} or rend their brain with \imp{psychic} damage., haszero=0, hasone=1, one=\spExample{Illusory Pain}{The target feels a deep seated pain and reacts accordingly\comma{} but takes no \imp{Harm}}, hastwo=0, hasthree=0, hasfour=0, hasfive=0, hassix=0, hasseven=0, hasExamples = 1}


}


\discipline{Temporal}
{
	\spell{name = Identify, incantation = Dicemi, book = Knowledge From the Realms Beyond, description = The \imp{Identify} spell wraps a targeted object or being with magic\comma{} extracting information about its past and its present\comma{} allowing one to learn a great deal of information about what the object might be\comma{} and what it does. 

More powerful attempts to \imp{Identify} can frustrate attempts at deception and concealment\comma{} as well as reveal more in\minus{}depth information about a target., haszero=1, zero=\spExample{Naming}{Learn the name of a target humanoid or simple non\minus{}magical object.}, hasone=0, hastwo=0, hasthree=0, hasfour=0, hasfive=0, hassix=0, hasseven=0, hasExamples = 1}

	\spell{name = Prophesy, incantation = Providentia, book = Unfogging the Future, description = A \imp{Prophesy} allows one to get a brief snapshot into the past\comma{} present of future\comma{} learning what is soon to come\comma{} what is\comma{} or what was. Be it a major event on the horizon\comma{} the secret history of your family\comma{} or merely the hidden assassins around the corner\comma{} you can use this to inform your actions\comma{} or to prepare your reflexes for when disaster strikes. 

The most effective acts of \imp{Prophesy} are through rituals – reading of tea leaves\comma{} peering into crystal balls. These acts can focus the mind and give a clearer picture of the events you are looking in upon., haszero=0, hasone=0, hastwo=0, hasthree=0, hasfour=0, hasfive=0, hassix=0, hasseven=0, hasExamples = 0}

	\spell{name = Scry, incantation = Videro, book = The Third Eye And You, description = A \imp{Scrying} spell allows one to cast their senses over vast distances\comma{} gaining the ability to see\comma{} hear\comma{} smell and experience what is going on around. 

In order to scry on a person or a location\comma{} you must generally know their name (and their {\it real} name)\comma{} or be very familiar with the location\comma{} though with signifiant power you can bypass these steps. 

A beginner \imp{scry} spell might be used to look around corners without revealing oneself\comma{} as the distance over which you can \imp{scry} is a function of how powerful the spell is., haszero=0, hasone=0, hastwo=0, hasthree=0, hasfour=0, hasfive=0, hassix=0, hasseven=0, hasExamples = 0}

	\spell{name = Traverse, incantation = Astra, book = Death Omens: What to Do When You Know the Worst is Coming, description = Glimpse the majesty of the multiverse: with a \imp{Traverse} spell you gain the ability to sense the \imp{Realms} which exist beyond the mortal coil\comma{} and with enough control\comma{} you may travel there\comma{} and even erect \imp{portals} allowing easy transport – though this is an ancient and long\minus{}lost art. 

At low levels\comma{} you are restricted to brief forays into the \imp{Astral Realm} alone\comma{} even gaining a glimpse of realms further afield a significant act of power., haszero=0, hasone=0, hastwo=0, hasthree=0, hasfour=0, hasfive=0, hassix=0, hasseven=1, seven=\spExample{Gateway}{Establish a permanent\comma{} stable portal to a \imp{Realm} of your choosing. You can choose if the gateway is traversible by all\comma{} or only by you and your allies. }, hasExamples = 1}


}


\discipline{Warding}
{
	\spell{name = Abjure, incantation = Finite Incantatem, book = An Anthology of Safeguarding Measures, description = The \imp{Abjure} spell is also known as the {\it common counterspell}\comma{} as it allows you to drain a magical effect of its power\comma{} breaking the enchantment and ending the spell. Advanced practitioners can also use this spell to prevent a spell from ever being cast in the first place\comma{} halting the magic before it can ever be summoned. 

The more powerful the magic you are targeting\comma{} the more powerful the \imp{abjure} spell you must use in order to counter or dispel it., haszero=0, hasone=0, hastwo=0, hasthree=0, hasfour=0, hasfive=0, hassix=0, hasseven=0, hasExamples = 0}

	\spell{name = Bypass, incantation = Alohamora, book = Sidestepping Those Who Oppose You, description = The \imp{Bypass} spell allows one to sidestep security measures\comma{} temporarily unlocking doors\comma{} disabling alarms and so on. With a particularly powerful \imp{bypass} spell\comma{} you may find yourself able to walk through walls\comma{} completely bypassing all forms of magical and physical defenses.

Any magical or mundane measures usually reactivate when the spell wears off\comma{} as if you were never there. This can be a blessing\comma{} but you must also be careful when retracing your steps!, haszero=0, hasone=1, one=\spExample{Mundane Lock}{Open a non\minus{}magical lock. }, hastwo=0, hasthree=1, three=\spExample{Magical Lock}{Attempt to bypass a magical lock. The lock resets 1 minute after this spell ends.}, hasfour=0, hasfive=0, hassix=0, hasseven=1, seven=\spExample{Ghoststep}{Enchant a group of people with powerful magic such that they do not trigger traps or alarms\comma{} and locks simply fall open for them. They can still be  detected visually\comma{} but most mechanical and magical means of detection and obstruction are ineffective.}, hasExamples = 1}

	\spell{name = Resist, incantation = Adverso, book = Defending Against  the Undefendable, description = A ward of \imp{Resistance} doesn’t inherently protect its target in the same way that a \imp{shield} spell might\comma{} but it provides its target with an additional layer of protection to bolster their own natural defences – under the protection of a \imp{Resist} spell\comma{} their sidesteps and dodges are just a bit quicker\comma{} and their armour just that bit tougher. 

A powerful \imp{warder} might also use this spell to provide a target with \imp{Resistance}\comma{} or even \imp{Immunity} to a certain kind of damage\comma{} though this is quite a powerful act., haszero=0, hasone=0, hastwo=0, hasthree=0, hasfour=0, hasfive=0, hassix=0, hasseven=0, hasExamples = 0}

	\spell{name = Shield, incantation = Protego, book = Self\minus{}Defensive Spellwork, description = The \imp{Shield} spell does exactly what it says on the tin: it produces an ethereal magic shield which can protect an individual or a location from harm\comma{} repel intruders and otherwise offer protection to those under its protection. 

\imp{Shield} spells have also been known to be used to place walls of near\minus{}impenetrable force over doorways to prevent access\comma{} or to lock doors to deter intruders., haszero=1, zero=\spExample{Lock}{Magically reinforce a lock\comma{} preventing it from easily being opened.}, hasone=1, one=\spExample{Combat Shield}{A simple shield which flashes into existence\comma{} absorbing 1 level of harm.}, hastwo=0, hasthree=0, hasfour=0, hasfive=0, hassix=0, hasseven=0, hasExamples = 1}

	\spell{name = Trap, incantation = Dolus, book = How Not to be Killed: A Guide for the Discerning Wizard, description = A \imp{Trap} spell allows you to hide a nasty (or perhaps merely surprising) effect\comma{} lying in wait for another to activate it. 

From a simple proximity alarm\comma{} to a glyph which detonates when it detects the presence of a dragon\comma{} the \imp{Trap} spell encourages its users to be as inventive\comma{} sneaky and cunning as they possible can., haszero=0, hasone=1, one=\spExample{Alarm}{Lay a hidden tripwire which releases an awful wailing when triggered}, hastwo=0, hasthree=0, hasfour=0, hasfive=0, hassix=0, hasseven=0, hasExamples = 1}


}



%%End
