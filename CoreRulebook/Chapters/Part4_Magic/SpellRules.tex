\chapter{Laws of Magic}\label{S:Laws}

Magic is a complex and nebulous thing to pin down. However, there are a few known hard-and-fast rules which magic must obey.


\section{The Prime Law}

The \key{Prime Law of Magic} is a law which states that there are several immutable aspects of reality. Attempts to warp or sidestep these immutable aspects almost inevitably leads to a universal, karmic backlash. The more that a mage attempts to bend these laws, the more catastrophic this backlash is. 

The exact nature of the immutable aspects of reality is unknown, though \imp{Unspeakables} at the \imp{Ministry} have spent decades attempting to divine them. The known applications of the \imp{Prime Law} are:
\begin{enumerate}
	\boldItem{Death}{Death is a constant. It cannot be cheated or reversed, only delayed. When someone is truly dead, they can never be brought back. Attempts to defeat death are only ever temporary and have horrifying consequences, i.e. the creation of \imp{Horcruxes} destroys and fractures the humanity of the subject, rendering it impossible for the soul to move on and condemning the mage to an eternity in a horrifying \imp{Limbo}.} 
	\boldItem{Love}{Whilst temporary infatuation and lust can be invoked through magic, \imp{True Love} can never be created artificially. Attempts to truly sway the heart of another will inevitably backfire, leaving the witch or wizard permanently alone.}
	\boldItem{Time}{Whilst it is possible for the discerning witch or wizard to wander through time, to revisit moments in the recent past or to allow you to complete several tasks at once, it is categorically impossible to alter the flow of time without destroying the very fabric of reality. Those who attempt to alter time inevitably find that they were themselves the cause of the very event they are attempting to stop. } 
\end{enumerate} 

\section{Fundamental Law of Conjuration}

\imp{Ulick Gamp} is most famous as being the original Minister for Magic (1707-1718) and the Wizard responsbile for imposing the \imp{International Statute of Secrecy}. However, he was also a renowned \imp{Scholar} and \imp{Thaumaturge}, specialising in \imp{Conjuration}. 

Before his time in office, formulated \imp{Fundamental Law of Conjuration}, also known as \imp{Gamp's Law of Elemental Transfiguration}. 

This law states that the those who excel in the \imp{Conjuration} discipline can conjure anything out of thin air, with only 5 exceptions. These exceptions are:



\begin{enumerate}
	\boldItem{Alien}{A conjurer can only summon objects that they can clearly visualise, understand and know of. A conjurer cannot summon a snargle if they have no idea what a snargle is. }
	\boldItem{Nebulous}{A conjurer can only summon physical, material things, not nebulous concepts. You cannot summon `love', `knowledge' or `happiness'}
	\boldItem{Magic}{A conjurer cannot summon an item already imbued with magic - potions, magical wands and wizarding currency must be acquired through conventional means. Most potion ingredients are also inherently magical, and so cannot be summoned out of nothing.}
	\boldItem{Sentience}{A conjurer can never summon another sentient being from nothingness, nor imbue a non-sentient object with a soul}
	\boldItem{Sustenance}{Objects summoned by a conjurer can never provide nutrition, turning to ash as soon as they are consumed. Any summoned, living beings turn to ash (or return to their original dimension) when they are killed, as do bits which are chopped off or removed from them. Conjured water seems to be an exception to this exception.}  
\end{enumerate}

\section{Tenets of Shapeshifting}

Changing the shape of oneself and anothers is a time-honoured tradition in the wizarding world. However, whilst we would all like to spend an hour soaring around as an eagle every now and then, this kind of powerful magic is not without its limits. These limits are known as the \imp{Tenets of Shapeshifting}, unlike the \imp{Fundamental Law of Conjuration}, these are not always applicable, and a sufficiently powerful spellcaster can bypass several of the Tenets, though not without incurring signifiant risks. 


\begin{enumerate}
	\boldItem{Accoutrements}{When being is transformed, generally all objects, items and clothing on their person are transformed along with them. This often leads to a physical marking on the transformed being: Headmistress McGonnagal's cat-form famously had a ring of black fur around its eyes from her spectacles. When the transformation is reversed, all items and equipment return to their original position.}
	\boldItem{Permanence}{No shapeshifting effort on a living being can ever be permanent. A being which has its body destroyed whilst in an altered shape always reverts back to its true form.}
	\boldItem{Sentience-to-non-sentience}{Transforming a being possessing a sentient mind is generally a tricky thing. Transforming a human into a non-sentient form (such as a weasel) gives them the brain of a weasel, and those who suffer a human-to-animal transformation often only have very rough memories and emotions of their time in this form. This can be sidestepped by a particularly powerful spellcaster, with care they would be able to give said weasel the brainpower of a human, allowing the animal to act with humanoid intelligence. Turning a sentient being into an inanimate object requires an incredibly skilled spellcaster. }
	\boldItem{Non-sentience-to-sentience}{The reverse of the above is attempting to turn a non-sentient being such as a cat or a table into a sentient being. This is generally regarded as impossible. Whilst you may form them into a humanoid body, it will inevitably be either brain dead, or literally a corpse. Turning a table into a cat, and vice-verse, however, is perfectly possible. } 
\end{enumerate}


\section{Wandlore}\ref{S:Wandlore}


\section{Magic \& Technology}

The intersection between Muggle technological prowess and wizarding magic is a rarely studied field - mostly because the two worlds so rarely interact. 

However, under the stewardship of \imp{Arthur Weasley}, the Department for Muggle Affairs has conducted a number of limited studies into the interaction between these opposed forces. 

\begin{enumerate}
	\boldItem{Powercell Saturation Inversion}{The unusual potio-static cells which muggles use to power their electronic devices interact incredibly badly with strong magical fields, such as those found in permanent wizarding settlements. Within places like Hogwarts, Godric's Hollow or Diagon Alley, these power cells are depleted in mere moments, leaving the devices useless. This is true on a smaller scale too - magical devices summoned by the \imp{Seeking charm}, for example, suffer an anomalous battery drain whilst the magical spell is maintained.}
	\boldItem{Technological fluxes}{The inverse of this does not seem to be true - magic is unaffected when in the presence of large amounts of technology. However, the Ministry notes that wizards should still be careful that any magical feats in such areas go unnoticed, lest they be recorded breaking the \imp{International Statute of Secrecy}.}
	\boldItem{Technological Detection of Magic}{Thus far, Muggle technology seems to be incapable of detecting magic. Their cameras can record images of feats of magic, but the furrowed brows of muggle scientists at scene of a magical outbreak leads the ministry to believe that there is no magical residue which can be detected by muggle sensors.}
\end{enumerate}
