\chapter{Negating Spells}

Of course, most beings do not simply wait for a spell to hit them - they will often try to take some action to mitigate the effect. This might take the form of \key{Dodging}, \key{Blocking} or \key{Enduring} the spell effect, or if they have the ability, casting some other spell to shield them or deflect the attack. 

The rules for Resisting are discussed in more detail in the Combat Actions section, so those discussed here are specific to spells. 

The DV of the Resist is determined by your relative ability (i.e. the number of dots) in the chosen Resist action to those associated with the spell effect:
$$ \text{DV}  = 8 + \text{Incoming Spell level} + \text{Increased DV} - \text{Resist dots} $$

Each success rolled reduces the \imp{Power} of the spell by one. If the \imp{Power} reaches zero, then the spell effect is nullified. 


Of course, some attempts at Resisting are not going to work - when an enormous meteor comes screaming out of the sky towards you, holding up a metal shield is not going to be much help. A Resist only reduces the power of a spell if the GM actually rules that it would help nullify the spell effects as described by the caster. If an inappropriate Resist attempt is made, the GM may limit its effectiveness, or rule it as totally innappropriate and therefore have no effect. 

\subsubsection{Example Negations}

For example, in the example above, Michael cast a \levelOne{} spell at close range. If his sparring partner (with a \key{Dodge} rating of 3) attempts to leap out of the way, his DV would be 6, as he has two more dots in dodge than the spell level. However, Michael used some of his successes to make the spell harder to dodge, so the DV is increased to 8. 

The spell Michael cast has a \imp{Power} of 1, which he increased to 2 using a success. Hence, his partner needs two successes to evade the spell, using a \imp{Fitness (Dodge)} pool of 4 dice. 

Rolling \imp{3-5-6-10} gives only a single success. The \imp{Power} of the spell is reduced to one, and so Michael's spell takes hold, but only for a single round. 

Conversely, Jane was attempting to attack the Whomping Willow, she had cast a \levelThree{} fire blast towards it, and augmented it to have a \imp{Power} of 4. The Whomping Willow is coated a thick bark, and so has a \imp{Block} rating of 5, and will use its immense Vitality (rating of 6) to withstand the fiery blast. This gives it 11 dice to roll against a DV of 6. With 8 successes rolled and a single botch, the Whomping Willow is able to reduce the \imp{Power} of the spell to zero, and hence takes no damage. 


