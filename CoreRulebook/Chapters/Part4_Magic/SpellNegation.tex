\chapter{Negating Spells}

Of course, most beings do not simply wait for a spell to hit them, and wizards rarely rely on thick shields or cumbersome armour to protect them against incoming harm. 

The general rules for \imp{Resisting} harm and other status effects can be found on page \pageref{S:Defence} - the rules here are specific to resiting magic, and using magic for defensive purposes.

\section{Resisting Against Magic}

Resisting an incoming magical attack is, in most cases, identical to negating an attack from a mundane weapon: performing a DV 7 check using one of your \imp{Defence Statistics}, and reducing the \imp{Power} of the incoming attack by the number of successes. 

However, unlike a simple weapon attack, magical attacks are most likely to have complicating factors from the simple normal cases - some magical attacks are such that any number of \imp{defensive} actions are obviously unfeasable - you cannot \imp{dodge} out of the way of a \imp{homing} fireball, and trying to \imp{block} as a house is dropped on your head isn't going to do much. 

There is therefore much narrative leeway for complications when magic is involved, whilst you may attempt to be creative in how your character gets around the problems presented to them, the \imp{GM}s word on the matter is final.

In addition, spellcasters may choose to use the \imp{Defy} option, whereby they sacrifice a level of \imp{power} in exchange for making an attack harder to negate.
 
\section{Using Magic to Resist}

The expert magical duelist is equally as adept at using their spells for defensive purposes as offensive, and so you may use your spells to counter incoming attacks. 

This is considered a \imp{Full-Round Defence} (so additional \imp{quickdefense} actions do not incur drain), however you still perform the action as a standard spellcasting action, selecting from your normal set of \imp{Aspects}. This means that you are not impacted by \imp{Drain}, but you do suffer the effects of \imp{Harm} on such checks. 

If you choose to manifest a spell effect for defensive purposes, the successes are used to subtract away from your foes attacks - either against yourself, or against one of your allies. This is of course subject to your \imp{GM}s approval that the selected spell would be suitable for defensive purposes - you can be as inventive as you like here. 

As with a \imp{full round defense}, you may allocate your successes between any number of opponents, as long as the selected defence would generalise suitably, or distribute across a large region of space.

\subsection{Limitations}

Note that the above rules for using magic to negate an incoming attack violate one of the key paradigms of the \imp{Combat Cycle} mechanic - namely that all actions are near-simultaneous, with all effects applied at the end of the round. In the normal course of events, therefore, it is perfectly possible for two characters to simultaneously stun each other if they cast spells on each other. 

However, if you use the \imp{Force} spell to attempt to push back an opponent rushing at you with a sword, your spell manages to get in {\it just} before their attack does, thereby preventing their attack from landing. 

For this reason, {\bf it is impossible for a defensive spell to deal \imp{harm} or impose any long-lasting status effects.} Any magical effects manifested for the spell last only for the instant necessary to interrupt and negate the incoming attack, before quickly wearing off. The \imp{Force} spell might normally fling a target across the room, but if used for defensive purposes, it simply staggers them enough to prevent the attack being formulated. 

From a narrative standpoint, this is because the spellcasting was `rushed' - pushed out in a hurry in order to interrupt an incoming negative effect, and therefore does not quite have the same {\it oomph} that a normal spell does. 
