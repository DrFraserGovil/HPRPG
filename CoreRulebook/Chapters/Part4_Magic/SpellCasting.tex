

\chapter{Casting Spells}

Of course, knowing the theory of magic is one thing, it is quite another summon an ethereal spirit from the end of your wand, or to blast a foe with a wall of energy. In order to truly understand magic, you must go about actually {\it casting} magic. 

Spellcasting is the process by which a witch or wizard harness the infinite, chaotic and formless power of {\it magic}, shape it through their intellect or force of will, and project it into the world around them. For most wizards, this is achieved through the use of an incantation, a movement of the wand, and deep concentration, though some extremely powerful magic might require a ritual be conducted before the magic can be executed. 

Some powerful wizards understand that these are simply crutches, guiding tools for the weaker mind - and can cast magic both silently, and without their wand to focus the magical energies. This, however, is an advanced feat and is not to be taken lightly. 

\section{Choosing the Spell}

When a magic user wants to go about using magic, they must first decide what it is they would like to do - do they want to stun a foe, teleport across a nation, or simply turn out the lights?

After deciding the effect they would like, they must then select a spell that they have memorised, and decide how to use it to achieve their goals. There is often more than one way to achieve a given effect, though some may be more obvious than others - turning out the lights would probably fall under the domain of \imp{Lumos}, the illumination spell, though there is no doubt that using \imp{Reducto} to reduce the light to dust would also be an effective solution. 

That there are multiple ways around a given obstacle is not a problem - in fact, it is the very nature of magic that there are many ways to achieve a single goal. \imp{Greta the Arsonist} was a famous and powerful 15th-century witch who never learned a single spell beyond \imp{Incendio}, but who wielded her fire magic with such variety and finesse she could overcome most anything in her path. 

Of course, the problem lies in the aftermath of the spellcasting. 

A locked door might pose no challenge to a wizard armed with \imp{Reducto}, the disintegrating jinx. However, the presence of a blackened, smoking hole where there once was a door might indicate to anyone passing by that an intruder had been through here - not to mention the noise that would likely ensue! In this instance, the Bypass charm, \imp{Alohomora} might have been appropriate as a silent yet sneaky way past a lock, though even this is not foolproof, as it is easily blocked by magic. 

When selecting a spell, you should therefore talk through with your teammates and the GM what the likely side-effects of the spell are. This phase should hopefully become less frequent as both the player and character become more experienced with individual spells, but you should never be afraid to ask! Equally, if a character is not likely to know the side-effects, the GM is perfectly allowed to refrain from the discussion (or ask for an Intelligence (Arcane) check to allow the characters to solve the problem). 


\subsection{Spell Level}

The caster and the GM must then work together to determine how much magical energy must be dedicated towards the spell, in order to achieve the stated goals. For example, the water-manipulating spell \imp{Auguamente} can summon a small stream of drinking water, or raise a cataclysmic tsunami - the caster's description of their spellcasting should indicate where in this spectrum their intended spell lies. 

Each spell is therefore split into a number of `power levels', which determine how much change a spell can induce:

\extendedRatingTable
{{\levelZero}{The weakest possible manifestation of the spell, with truly miniscule effects}}
{{\levelOne}{A weak manifestation, with limited control. Able to deal a small amount of damage.}}
{{\levelTwo}{A spell which requires more control and finesse, though the effect and harm inflicted remains limited}}
{{\levelThree}{Typically the limit of magic taught at Hogwarts, spells which can pack a bit of a punch}}
{{\levelFour}{An adept level spell, requiring significant power and skill, but with increased effectiveness}}
{{\levelFive}{A powerful effect that would turn the heads of most wizards and deal significant damage or other powerful effects}}
{{\levelSix}{Extremely powerful effect, requiring masterful levels of control and dealing life-threatening damage.}}
{{\levelSeven}{God-like levels of effectiveness. If you can imagine doing it with a spell, at \levelSeven{} you can.}}

Conjuring a small stream of water from the tip of your wand would require \imp{Augumente \levelOne}, whilst summoning a tidal wave would require \imp{Aguamente \levelSix} or even \imp{Aguamente \levelSeven}. The description of a given magical spell gives examples of how various effects would be classified. As usual, it is up to the GM to determine what level a proposed spell effect would fall into.

The maximum `spell level' that a caster can access is determined by their \imp{Affinity} in the associated \imp{Discipline}. Casting \imp{Aguamente \levelSix} would therefore require a six-dot rating in the \imp{Elemental} discipline. 

A spellcaster with \imp{Affinity} rating of three in \imp{Bewitchment}, for example, would therefore be able to cast a \imp{Bewitchment} spell anywhere between \levelZero{} and \levelThree{} without significant effort. You can expend a \imp{Fortitude} point in order to access a higher level spell, but beware doing this too often, as it can leave a wizard feeling weak and drained at a crucial moment. 


\subsubsection{A Note on Roleplaying}

Deciding on the effect your spell has is a prime opportunity for roleplaying, as it is directly shaped by your character's own understanding of magic and their perception of reality. Equally, you may have as much fun as necessary deciding, for example, exactly where on your target and how the spell is going to strike. 

The examples of effects given in the spell descriptions are intended as exactly that: examples. You should not feel bound by these descriptions, and be as inventive as you wish. After all, which leads to a more compelling story?

{\it I use \imp{Incendio} \levelThree{} to blast the troll with fire}

Or:

{\it I summon a gigantic blast of flame, which erupts from the end of my wand in the shape of a unicorn to impale the troll through the eye with its fiery horn.}

Mechanically speaking, these would have exactly the same effect and require the same dice rolls, but clearly one is much more fun! Of course, this comes with a warning: don't overdo it. Find a healthy balance between moving the game along and exploring the infinite potential that magic offers.

\section{Casting Checks}

After working with the GM to determine the magnitude of the spell, this determines the DV of the spellcasting action, using the following formula:

$$ \text{DV} = 8 + \text{Spell Level} - \text{Affinity} $$

In practice, this means that the most powerful spells you have access to have a DV of 8, with the DV decreasing by one for every subsequent lower level of spell. Hence, casting a \levelTwo{} spell in a discipline in which you have an \imp{Affinity} of 4 would have a DV of 6, whilst (if you expended the Fortitude to allow you access to it), a \levelFive{} spell would have a DV of 9. 

You must then build a dice pool with which to perform the casting check: the relevant \imp{Affinity} is always included, and cannot be changed. However, the \imp{Aspect} which is being used depends on how you described the spell as being cast. 

For instance, the fiery-unicorn described above could probably benefit from a \imp{Precision} aspect being used, as the spell was specifically mentioned to be targeting a specific part of the target. A hex or psionic attack would find \imp{Willpower} to be useful as you try to impose your will over that of your target. A particularly novel or clever use of a spell coulduse \imp{Intelligence}. Though rare, if you were trying to fortify your own strength as you desperately tried to hold open a door, for example, a \imp{Fitness}-casting could be appropriate. 

The aspect used should be informed by the spellcaster's description of the spell which they are trying to cast, though if no choice is made particularly obvious by the caster's description, you may use the following table to determine the basline aspect:
\begin{center}
	\begin{rndtable}{c p{\xS cm} p{\wS cm}}
	\bf School	&	\bf Discipline	&	\bf Attribute
	\\
	\school{Charms}{Elemental}{Willpower}{Kinesis}{Precision}
	\\
	\school{Divination}{Telepathy}{Insight/Perception}{Temporal}{Intelligence}
	\\
	\school{Illusion}{Bewitchment}{Charm/Deception}{Psionics}{Willpower}
	\\
	\school{Malediction}{Hexes}{Willpower}{Curses}{Intelligence}
   \\ 
   \school{Recuperation}{Healing}{Insight}{Warding}{Intelligence}
	\\
	\school{Transfiguration}{Alteration}{Precision}{Conjuration}{Intelligence}
	\\
	\school{Dark Arts}{Necromancy}{Willpower}{Occultism}{Charm}
	\end{rndtable}
\end{center}

You may also gain or lose extra dice depending on the situation you find yourself in - if you are a novice trying out the spell for the first time, you might gain an additional dice due to the presence of the teacher, or perhaps because you have the spellbook open in front of you the DV is reduced. 

Equally, if you are currently under the effects of {\it Terror} or some other negative effect, your spellcasting efforts may be hindered, either through an increase in the DV, or the confiscation of a dice. If you are really pushing it with the capabilities of spell, you may also suffer an increased DV - using a shield made out of fire is clearly going to be more difficult than just using a \imp{Shield} spell. 

The GM rules on what bonuses or penalties are appropriate in a given moment. 

After assigning the DV and the dice pool, you then perform a normal ability check, following the rules discussed on page \ref{C:Checks}

\subsection{Casting Failure}

If there are no successes on the casting check, then the magic effect fails to materialise, and you suffer a \key{Casting Failure}. You shout the incantation and wave your wand, but nothing quite seems to happen beyond a few sparks. You suffer no negative consequences inherent to the failure of a spell, though you may suffer incidental misfortune such as having given away your position by speaking aloud. 

If the cast was a \key{Catastrophe}, however, due to the rolled 1s outnumbering the rolled successes, something pretty bad is probably going to happen. As usual, it is up to the GM to decide what form this takes, though it would be fairly common for a damaging spell to be inflicted on you or your allies, and likewise for a beneficial spell to accidentally bounce onto an opponent. 



\subsection{Success}

If your casting was a success, then you successfully summon the desired magical effect. However, the actual effectiveness of the spell must now be evaluated by accounting for the number of successes. 

\subsubsection{Inherent Effect}

The inherent effectiveness of a spell is determined by the level used to cast it. Casting a \imp{Incendio Tria} spell is obviously going to be more intrinsically powerful than an \imp{Incendio Dua} spell.

You may think of a spell has having a \key{Power} rating, the inherent \imp{Power} of a spell is equal to the spell level used to cast it, but can be modified by additional successes. 

For a damage-causing spell, for example, each point of Power deals 1 point of damage to the target, so a \levelOne{} spell deals 1 level of harm, whilst a \levelSeven{} spell deals 7 levels, enough to instantly kill a normal person. Equally, Power can be used to calculate the duration of a given spell effect - a stunning spell cast at \levelThree{} would be enough to knock a being out for 3 rounds. 

\subsubsection{Required Successes}

Before the effects of a spell can be applied, however, you must first successfully ensure that the magic hits its target. Projecting magic out of yourself is a difficult task - even when you successfully cast the spell, there are certain feats that require a minimum degree of success before they actually work. 

For example, casting a spell an another person requires not only that the spell is cast, but that you can successfully target them and {\it push} the effect out towards them. Targeting more individuals would equally require a larger number of successes, as there is also a difference between casting a spell on a target within wandreach and at the edge of your vision. 

Use the following table to determine the minimum number of successes required for given spellcasting effects:


\newcommand\rrow[3]
{
	\key{#1}	&	\parbox[t]{4cm}{\small \raggedright #2}	&	\parbox[t]{2cm}{\small\centering  \raggedright#3} \\
}
\begin{center}
	\begin{rndtable}{c l c}
	\bf Effect	&	\bf Example	&	\bf Successes
	\\
	\rrow{Self}{Casting a spell on yourself}{0}
	\rrow{Wandtip}{Casting a spell on a target you can place your wand or hands upon}{+1 per target}
	\rrow{Ranged}{Cast a spell on a target at a distance}{+2 per target}
	\rrow{Mass}{Cast a spell on a large area, affecting everyone in the region}{+4 and up}
	\end{rndtable}
\end{center}

With \key{Mass} spells, the required successes is determined by the GM depending on the magnitude of the effect required, but generally such spells are rather difficult. 

If you do not meet the minimum number of required successes for the effect the manifest, you may either reduce the number of targets selected (i.e. blast only 1 of your 2 targets) or reduce the targeted area you have chosen until you have enough successes, but you cannot materially alter the nature of the spell - i.e. you could not change a \imp{Mass} spell to a \imp{Ranged} spell. 

Alternatively, you may simply abandon the casting, treating it as a \imp{Spellcasting Failure}, or you may attempt to use \imp{Extended Casting} (discussed below) in order to focus your mind and complete the casting after gathering your thoughts again. 

\subsubsection{Optional Successes}

After meeting your required successes, the spell effect materialises. Your remaining successes may then be allocated to increase the effectiveness of the manifested spell by increasing the magnitude, decreasing the possibility of a foe avoiding or negating the spell, or even making the spell effect last longer. 

\begin{center}
	\begin{rndtable}{c l p{2 cm}}
	\bf Effect	&	\bf Example	&	\bf Successes
	\\
	\rrow{Overpower}{Increase the \imp{Power} of the spell, increasing the magnitude, damage or healing of a spell by one point}{+1 per increase}
	\rrow{Defy}{Increase the DV an opponent must defeat in order to negate the }{+1 per increase}
	\rrow{Extend}{Increase the duration of the spell effect}{+ original duration per increase}
	\end{rndtable}
\end{center}

Note that the \imp{Extend} option does not apply to spells where the duration is set by the \imp{Power} (i.e. stunning spells). Instead it could be used to increase the duration of a shield or ward which has an explicit duration. 

The maximum number of additional dice that can be assigned to a given optional effect is equal to twice the spell's level (or 1, for \levelZero{} spells). 

You may also choose to leave some successes unused - if you are attempting to subdue a wild beast, you may not want to risk killing it. Equally, some spells are all-or-nothing, and so as long as the minimum requirements are met, additional successes may not do anything beyond increasing the flavour of the GM's descriptions. 

\subsubsection{Example Spellcasting}

Jane has gotten herself into a confrontation with the Whomping Willow and is trying to set it on fire to teach it a lesson. She has an \imp{Affinity} of 3 in the Elemental discipline, so she decides to fire a bolt of flaming energy at her foe (a \levelThree{} feat). This means that she must defeat a DV of 8, and must have at least 2 successes in order to fire the bolt at such a distance. 

Jane's player reasons that since Jane is doing this out of anger, \imp{Willpower} seems like an appropriate \imp{Aspect} to use, and the GM agrees, building her dice pool up to a total of 7. There are no other bonuses relevant to this moment, so Jane performs the dice check. 

With a roll of \imp{1-4-5-8-8-9-11}, Jane has three successes, two of which are used to form the blast of fire. Since Jane is angry, she uses her additional success to deal an additional point of damage. This spell therefore deals 4 points of fire damage to the Whomping Willow. 

Meanwhile, Michael is busy practicing for duelling class. He casts \imp{Stupefy} to attempt to temporarily confound his opponent (a \levelOne{} effect). Michael has an affinity of 3 in the \imp{curses} discipline, and since he is training, an \imp{Intelligence} (rating 3) check seems relevant. Michael therefore has 6 dice to roll against a DV of 6. Michael decides that to give himself the edge, he's going to try and get in close to his opponent and force the spell at a \imp{Wandtip} range. 

After rolling \imp{2-3-7-8-9-11}, he has four successes, one of which is required for the spell to work. Since this spell is a very basic one, Michael knows that it is easy to Resist, so he expends two of his successes to \imp{Defy} and hence increase the Resist DV by 2, the maximum amount for a \levelOne{} spell, and so uses the remaining point to \imp{Overpower}, and so increase the power of the spell.

\section{Negating Spells}

Of course, most beings do not simply wait for a spell to hit them - they will often try to take some action to mitigate the effect. This might take the form of \key{Dodging}, \key{Blocking} or \key{Enduring} the spell effect, or if they have the ability, casting some other spell to shield them or deflect the attack. 

The rules for Resisting are discussed in more detail in the Combat Actions section, so those discussed here are specific to spells. 

The DV of the Resist is determined by your relative ability (i.e. the number of dots) in the chosen Resist action to those associated with the spell effect:
$$ \text{DV}  = 8 + \text{Incoming Spell level} + \text{Increased DV} - \text{Resist dots} $$

Each success rolled reduces the \imp{Power} of the spell by one. If the \imp{Power} reaches zero, then the spell effect is nullified. 


Of course, some attempts at Resisting are not going to work - when an enormous meteor comes screaming out of the sky towards you, holding up a metal shield is not going to be much help. A Resist only reduces the power of a spell if the GM actually rules that it would help nullify the spell effects as described by the caster. If an inappropriate Resist attempt is made, the GM may limit its effectiveness, or rule it as totally innappropriate and therefore have no effect. 

\subsubsection{Example Negations}

For example, in the example above, Michael cast a \levelOne{} spell at close range. If his sparring partner (with a \key{Dodge} rating of 3) attempts to leap out of the way, his DV would be 6, as he has two more dots in dodge than the spell level. However, Michael used some of his successes to make the spell harder to dodge, so the DV is increased to 8. 

The spell Michael cast has a \imp{Power} of 1, which he increased to 2 using a success. Hence, his partner needs two successes to evade the spell, using a \imp{Fitness (Dodge)} pool of 4 dice. 

Rolling \imp{3-5-6-10} gives only a single success. The \imp{Power} of the spell is reduced to one, and so Michael's spell takes hold, but only for a single round. 

Conversely, Jane was attempting to attack the Whomping Willow, she had cast a \levelThree{} fire blast towards it, and augmented it to have a \imp{Power} of 4. The Whomping Willow is coated a thick bark, and so has a \imp{Block} rating of 5, and will use its immense Vitality (rating of 6) to withstand the fiery blast. This gives it 11 dice to roll against a DV of 6. With 8 successes rolled and a single botch, the Whomping Willow is able to reduce the \imp{Power} of the spell to zero, and hence takes no damage. 


\section{Extended Casting}

Sometimes spellcasting is not over and done with in a flash - there are often incidents where the spellcaster must continue to process and power the magic long after the incantation has been uttered.

\subsection{Concentration}

Many spells continue to have an effect after the spell is cast - the most basic such spell would be \imp{Lumos \levelZero{}}, which ignites the end of the spellcaster's wand until they dismiss the effect, though there are many other spell effects which would require a continual source of magic to support the continuation of the effect.

A basic effect such as the \imp{Lumos} example can continue, even without the spellcaster's full attention. In fact, with a spell this basic, the wand will probably remain illuminated even if the witch drops her wand! Simpl, self-sustaining effects such as this only mildly hinder the witch or wizard if they attempt to cast a second spell effect - you take a single dice penalty to a subsequent spellcasting check for every such effect which is active. 

If Neha has \imp{Lumos \levelOne{}} active to produce a glowing ball of light, as well as \imp{Aguamente \levelThree{}} allowing her to walk on water, she would take a 2-die penalty to her next spellcasting check. 

However, more complex feats of magic require continual input and focus from the spellcaster: holding back a tidal wave, or mind-controlling a troll requires constant attention and manipulaiton from the caster in order to maintain the effect. A spellcaster must devote almost all their energy to this effort and can usually take no other actions besides moving or talking. Your GM should inform you when a spell meets these conditions. 

Generally, you may choose to end any continued magical effect as an instantaneous action, taking no additional penalty. 

\subsection{Catch-Casting}

When casting a spell, if you successfully cast the spell, but fail to meet the minimum requirements of the spell on your first attempt you may choose to either abort the attempt, or choose to \imp{ Catch-Cast}. You do so either on your next turn or, if you spend a \imp{Fortitude} point to get an extra action, immediately afterwards. 

A \imp{Catch-Cast} is an attempt to refocus your mind and {\it force} the incomplete spell to completion. To an outside observer, it appears as if you had to spend more time than usual focussing on the spell, maybe taking several attempts at the required wandwork. 

When performing the \imp{Catch-Cast}, you re-perform the casting check with a one-dice penalty. Any successes are added to those already achieved until you have enough to fully materialise the spell effect. You may \imp{Catch-Cast} multiple times, taking an additional 1-die penalty each time but continually adding to your number of successes. If at any point you fail to get any successes, or suffer a \imp{Catastrophe}, the spellcasting effort fails and you cannot \imp{Catch-Cast} any further.

\subsection{Rituals}

Rituals are a form of deliberately-invoked extended casting. When doing a ritual you perform a long meditation, lay out a region of magic totems, or draw arcane symbols upon the floor. All this serves to help focus your mind and so perform more powerful feats of magic with ease. Some forms of magic (especially those belonging to \imp{Divination}, \imp{Occult} and \imp{Conjuration} disciplines) find themselves well suited to ritual-casting. 

One of the key aspects of magic is allowing your mind to enter into the correct state, which sometimes means that imagery and ritual is, though not inherently magical or powerful, necessary to complete the task. Reaching across the veil to question a deceased soul in an instant, whilst in broad daylight requires one to have great mastery of the art, whilst the same act would be much easier to do in a darkened room, surrounded by flickering candles and clutching an item belonging to the deceased individual - simply by virtue of it {\it feeling} like that is how you are supposed to summon the dead. 

Performing a Ritual, especially a well-described ritual, can grant you a number of automatic successful dice rolls, as if you had expended \imp{Fortitude}, at the behest of the GM.  

\section{Learning Spells}

