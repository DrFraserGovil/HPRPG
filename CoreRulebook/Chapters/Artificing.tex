\documentclass[../CoreRulebook.tex]{subfile}

\chapter{Artificing}\label{S:Artificing}

\section{Enchanting}
\label{S:Enchanting}
Enchanting is the process whereby magical items are made – imbuing them with extraordinary abilities. 

To enchant an item, you must first work out what effect you want to imbue it with – for example, you might want to imbue a sword with a fire spell to turn it into a flaming sword. You must them identify if you have a spell which performs this effect (in this case, {\it incendio} would work). 

If you are able to cast this spell, and you have an unenchanted version of the item (i.e. a sword), then you may proceed with the enchanting. You must first cast the ‘effect spell’, and then, after that has been successfully cast, you must then perform the Enchanting Ritual spell, to transfer that spell into the item. This ritual takes 3 hours to complete, during which time, no other magic may be performed.  The ‘effect spell’ and the enchanting ritual need not be carried out by the same person, though the effects of the enchanting will be better if they are performed by the same person. 

The enchanting ritual requires approximately 6 hours to complete, and the item needs to be immersed in a vat of liquid precious metal (silver or gold will suffice, these metals are consumed in the process), and then finally the 1d20 FIN (arcane) casting check must be performed, though POW points may be dedicated towards it in the same fashion as a ‘power dependent spell’. 


The GM will tell you if the enchanting was a success, and how powerful the enchanted item is. The GM will also determine any limits the item has – i.e. the number of uses that you may get out of it before it needs recharging, for example. 

\subsection{Enchanting Mishaps}

If your enchanting fails, you may suffer an Enchanting Mishap, which requires you roll on the following table:

\begin{center}
\begin{rndtable}{c p{6 cm}}
\bf Roll 	&	\bf Mishap
\\
1		&	Nothing happens.
\\
2		&	The item crumbles into dust
\\
3		&	The item becomes `Cursed'. GM rolls for curse effect.
\\
4		&	The item is enchanted with the exact opposite effect to the target.
\\
5		&	The item explodes. If target enchantment was damage causing, apply that damage for 2d10, else damage is `Force'. 	
\\
6		&	Another random magical item in your inventory is drained of all charges (but not disenchanted). 
\end{rndtable}
\end{center}

\section{Potion Making}

Potion making is the art of mixing together ingredients into a magic potion. Potions can have a large variety of effects, from healing the drinker, to causing immense pain, invisibility, or even conferring superhuman good luck. 

To mix a potion requires a number of things:

\begin{itemize}
\item 	A safe place to mix it
\item 	A fire to brew it
\item 	A cauldron to brew it in 
\item 	Between 2 and 5 ingredients
\item 	An empty container to store the potion.
\end{itemize}


%%PotionBegin

\potion{name =Pepperup Potion, description =Bright blue gel\comma{} with a strong\comma{} spicy odour., cost =5\sickle, effect =Restores FP by 10 points, difficulty =10, time =1 hour, doses =3~doses, essential =Chizpurfle Fang\comma{} Ginger\comma{} Tea Leaf, optional = Dragon Liver & +100\% & 3\\ Honeywater & +25\% & 1\\ Runespoor Egg & +150\% & 5\\ Salamander Blood & +50\% & 2\\ ,othereffect  =Causes smoke to issue from the ears with a loud whistling noise.}
\potion{name =Polyjuice Potion, description =The colour\comma{} scent and taste of this potion reflect the target transformation., cost =10\galleon , effect =Transfigure yourself into another human for 1 hour, difficulty =15, time =1 day, doses =1~dose, essential =Boomslang Skin\comma{} DNA of target\comma{} Fluxweed\comma{} Lacewing Flies, optional = Bicorn Horn & +100\% & 2\\ Knotgrass & +25\% & 1\\ Leeches & +50\% & 1\\ ,othereffect  =The transformation is randomly warped\comma{} and you end up with an ear for a mouth\comma{} and a mouth for an ear (for example).}
\potion{name =Viper\apos{}s Venom, description =A blue liquid with a slight acrid odour., cost =10\sickle, effect =Applies the {\it Poisoned: Mild} status effect and immediately deals 5 Poison Damage, difficulty =10, time =2 hours, doses =3~doses, essential =Asp Tail\comma{} Lobalug Venom\comma{} Nightshade, optional = Acromantula Venom & +150\% & 4\\ Basilisk Venom & +150\% & 4\\ Doxy Eggs & +50\% & 1\\ Venemous Tentacula & +75\% & 2\\ ,othereffect  =Triggers immune response so target is Resistant to poison damage for 24 hours.}
\potion{name =Wiggenweld Potion, description =Vibrant red fluid with a pleasant\comma{} herbal aroma., cost =5\sickle, effect =Restores HP 10 points, difficulty =10, time =1 hour, doses =3~doses, essential =Dittany\comma{} Horklump Juice\comma{} Wiggentree Bark, optional = Leeches & +50\% & 1\\ Moly & +200\% & 5\\ Murtlap Tentacles & +25\% & 1\\ Nettles & +25\% & 1\\ ,othereffect  =Injuries heal improperly\comma{} leaving the drinker Vulnerable to fire damage.}


%%PotionEnd


