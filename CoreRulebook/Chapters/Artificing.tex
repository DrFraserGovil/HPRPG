\documentclass[../CoreRulebook.tex]{subfile}

\chapter{Artificing}

\section{Enchanting}
\label{S:Enchanting}
Enchanting is the process whereby magical items are made – imbuing them with extraordinary abilities. 

To enchant an item, you must first work out what effect you want to imbue it with – for example, you might want to imbue a sword with a fire spell to turn it into a flaming sword. You must them identify if you have a spell which performs this effect (in this case, {\it incendio} would work). 

If you are able to cast this spell, and you have an unenchanted version of the item (i.e. a sword), then you may proceed with the enchanting. You must first cast the ‘effect spell’, and then, after that has been successfully cast, you must then perform the Enchanting Ritual spell, to transfer that spell into the item. This ritual takes 3 hours to complete, during which time, no other magic may be performed.  The ‘effect spell’ and the enchanting ritual need not be carried out by the same person, though the effects of the enchanting will be better if they are performed by the same person. 

The enchanting ritual requires approximately 6 hours to complete, and the item needs to be immersed in a vat of liquid precious metal (silver or gold will suffice, these metals are consumed in the process), and then finally the 1d20 FIN (arcane) casting check must be performed, though POW points may be dedicated towards it in the same fashion as a ‘power dependent spell’. 


The GM will tell you if the enchanting was a success, and how powerful the enchanted item is. The GM will also determine any limits the item has – i.e. the number of uses that you may get out of it before it needs recharging, for example. 


\section{Potion Making}

Potion making is the art of mixing together ingredients into a magic potion. Potions can have a large variety of effects, from healing the drinker, to causing immense pain, invisibility, or even conferring superhuman good luck. 

To mix a potion requires a number of things:

\begin{itemize}
\item 	A safe place to mix it
\item 	A fire to brew it
\item 	A cauldron to brew it in 
\item 	Between 2 and 5 ingredients
\item 	An empty container to store the potion.
\end{itemize}

Of course, mixing a potion is not as simple as mixing the ingredients in a vat and hoping for the best -- it is a magical process. You must therefore perform the Potion Mixing spell, which requires a 1d20 INT (arcane) check, to determine how successful (and hence how potent) the resulting potion is. The difficulty of this check, and the effects of the potion are determined by the ingredients that you put in to the cauldron.

Each ingredient has associated with it a number of alchemical effects and their strengths, for example:


\begin{center}
\begin{rndtable}{|c| c c|}
\hline Name	&	Ashwinder Eggs	&	Fire Seed
\\ 
Category	& Animal	&	Plant
\\ 
Effect 1 	&	Hot		&	Lust
\\
Effect 2	&	Lust	&  Hot
\\
Effect 3	&	Glow	&	Awareness
\\
Effect 4	&	Concentrate	&	Anger
\\
Difficulty	&	6	&	4
\\ \hline
\end{rndtable}
\end{center}
We can see here that both ingredients have in common the `hot' and `lust' effects, and therefore mixing these together will reult in a potion with strong effects in those two areas (most probably, giving a Fire Weakness and a deep, burning passion to the consumer). Mixing ingredients with effects in common multiplies the strength of that effect exponentially -- adding another ingredient with `lust' effects would increase the power of the lust even further. 

On the other hand, you might decide that you {\it only} want the lust effect, so you might add a cold ingredient (say, Frost Salamander Blood) to negate the `hot' effect. Having competing effects exponentially {\it decreases} the strength of that effect, so even a tiny amount of `cold' added would drastically alter the potion effects.

The `difficulty' of the potion is the sum of the mixing difficulties of the ingredients. Getting below the mixing difficulty doesn't mean that the potion automatically fails -- but you might start to discover some unpleasant side effects. Conversely, getting significantly above the check will result in a more powerful version of the potion.


Of course, you won't necessarily know how powerful the potion is....until you get somebody to drink it (or find some other means of probing the effects of the potion). 

Learning a potion is not as set in stone as learning a spell, as there is not always only one correct way to do it. Instead, you must research the effects of individual ingredients. If you spend 6 hours with a textbook, you may learn up to three ingredients which have the same effect in common (you may specify this effect when beginning to learn). Alternatively, you may research a single ingredient, and learn up to 3 of its properties.

By cross-referencing your knowledge of ingredients and their effects, you may then devise your own potion recipes, modifying their effects by adding or removing ingredients. Ingredients can either be purchased from a vendor, or can be searched for out in the wild, if you know where to look for them. Some ingredients may be grown in your own greenhouse, if you so desire.  

