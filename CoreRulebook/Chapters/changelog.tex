\documentclass[../CoreRulebook.tex]{subfile}
\twocolumn
\chapter{Changelog}

\section{Changes in V4.0$\beta$}

\subsection{Changes to Character Attributes}

\subsubsection{EMP $\rightarrow$ PER}

The Empathy attribute seemed a bit...weird. Shifting it to Perception means a better balanced game + makes it an actually useful skill!

\subsubsection{Dodge + Block}

The single AC value was shifted into two complimentary systems: the {\it dodge} and {\it block} stats, which were integrated with the new {\it instinct} system and the associated minor actions. 

\subsubsection{Character Sheet}

The character sheet was altered to include room for an accuracy, a dodge and a block attriubute. There has also beena general restructuring of the character sheet in order to make it more user-friendly. 

\subsubsection{Resist Dice Removal}

Previously `Resist' checks used a dice that increased in size, using the same system as the spellcasting dice. This was deemed needlessly complex, and didn't fit quite as well after `Resisting' was made a generic ability (akin to 5e Saving Throws), rather than a magic-specific ability. Reisting now uses a default d20 check, and spell resst DVs have been aggregated into a single statistic (see below) to make it easier to use. 




\subsection{Changes to the Spellcasting System}

\subsubsection{Spell Damage Reworked}

``cvdv" spells (i.e. those who did more damage when a casting check exceeded the DV by a higher amount) have been replaced by equivalent spells with standard ndx damage checks. This stabilises spell effects + balances the system better than before. 

\subsubsection{General effect balancing}

Spell effects were quantified across levels, and were balanced to ensure that lower level spells remain relevant at higher levels, but do not outclass their higher level compatriots. 

\subsubsection{Power Point Refocus}

Previous iterations relied on drastic changes to spell effects at higher levels (i.e. switching out damage-causing dice at higher levels). This not only required strict adherence to the spelllist page, it also led to unbalanced spells and a discontinous power slope (cheap spells suddenly gained significant power without commeasurate increases in DV/FP). 

This flaw was fixed by altering these spells to use the Power Point system introduced in V1 but poorly utilised. This integrates with the effect balancing: it allows spells to gradually gain power whilst also increasing the costs of that spell. 

\subsubsection{Disciplines Introduced}

The introduction of the 14 Spell Disciplines and the associated fixed check-type for each of them removes the need for each spell to have a DV/checktype listed. It therefore streamlines gameplay by removing the need for constant pageflipping, and simplifies the magic system. It also synergises well with the bimodal nature many of the Schools already possessed (i.e. `Hexes \& Curses', `Healing \& Warding') in previous editions. 

\subsubsection{Some Renaming}

Some small things changed names.

`Hexes and Curses' spell school became `Malediction'. 

`Concentration' spell types became `Focus' (this was purely for aesthetic / space saving reasons!)

\subsubsection{Cheat Sheet}

The `cheat sheet' has been added, allowing for a simple printout to serve as a reference for nearly all spellcasting. 

\subsubsection{Manipulate Spells}

It was felt that the `manipulate' spells were badly done, and were either overpowered (when considered a Novice or Adept spell), or held back simple manipulation abilities until too late in the game (when considered an Expert or Master spell). Therefore 20 new spells: 5 for each of the 4 classical elements were added in a successive manner. This hopefully provides a more integrated and flexible experience for the budding benders out there. 

\subsubsection{Spells alterations}

The following spells have been added:

\begin{itemize}
	\item Control Air/Earth/Fire/Water 1/2/3/4/5 (20 Elemental spells)
	
	See above for the justification for these additions
	
	
	\item Disrupt Connection (Adept Telepathy)
	\item Invert Connection (Master Telepathy)
	\item Timeslip (Expert Temportal)
	\item Drain Fortitude (Adept Psionics)
	\item Howl (Beginner Curse, Beast)
	\item Delayed Effect (Adept Curse)
	\item Break Concentration (Expert Curse)
	\item Feign Death (Adept Healing)
	\item Shimmering Confetti (Beginner Conjuration)
	\item Bind Being (Expert Conjuration)
	\item  Blood Moon (Adept Necromancy)
	\item Use Ancient Knowledge (Beginner Occultism)
	\item Shadowsight (Novice Occultism)
	\item Unfathomable Visage (Novice Occultism)
	\item Coven's Protection (Expert Occultism)
	\item Summoning Circle (Expert Occultism)
\end{itemize}

The following spells have been removed:
\begin{itemize}
	\item Mantle Element (Master Elemental)
	
	This spell becaeme redundant with the improved Control spells
	
	\item Manipulate Earth/Air/Fire/Water
	
	(As above)
	
	\item Summon Weak/Capable Avatar
	
	Too skyrim-y! This is part of the removal of the three-tiered system implemented for many creatures in V1.0
\end{itemize}

The following spells have been changed:
\begin{itemize}
	\item Flame Dart was renamed Ignite Being. 
\end{itemize}
(Almost all damage-causing spells had their effects altered as part of previously mentioned rebalancing, and are not listed here)


\subsection{Skill \& Proficiency Changes}

As a result of some of the wider structural changes implemented in this version, the skills \& proficiencies have seen a drastic overhaul in many areas. 


\subsubsection{Skills}
\begin{itemize}
	\item Several weapon-proficiency and armour-proficiency skills have been introduced. 
	\item The 8 dice-determining skills have been renamed to block them all together
	\item Metamorphmagus has been changed from a learnable skill to a new species, to better reflect in-universe lore. 
\end{itemize}
\subsection{Expertise}

Previously calld `Arcane Wisdom', the Expertise mechanics has been brought in as a similar device to the DnD 5e `proficiency bonus'. Previously, I was trying to do something different for the sake of being different. But why fix what isn't broken?

\subsubsection{Proficiencies}
\begin{itemize}
	\item Previously, proficiencies were treated as additional bonuses, each with their own numerical score. In line with the Expertise mechanic being brought in, they are instead replaced with checkboxes: you add your proficiency bonus to related checks, rather than having individual values associated with each. 
	\item I debated renaming `proficiencies' to remove confusion with the DnD mechanic of the same name, but decided against it. 
	\item The old EMP proficiencies were removed and replaced with {\it Observation} and {\it Compassion} to reflect the new PER system. 
\end{itemize}



\subsection{Changes to Items}

\subsubsection{Prices}

More realistic prices were added to the basic items and artefacts lists. 

\subsubsection{Weapons}

Weapons were overhauled, in line with the new proficiency


\subsection{Meta-Changes}

A meta-change: this changelog was included!


