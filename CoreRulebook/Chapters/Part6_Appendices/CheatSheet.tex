\onecolumn
\begin{landscape}
\footnotesize
\chapter{Cheat Sheet}
\begin{multicols}{3}
\def\xS{1.8}
\def\wS{2}


\subsection{Actions}

A turn cycle lasts 3 seconds has 5 phases. During {\bf Reasoning} everyone (including GM) decides on their actions, which are then {\bf Revealed}. You may get a chance to {\bf React} to avoid or modify the results. The actions are then {\bf Resolved} to determine which actions were successful, calculate damage etc, and then the {\bf Results} are applied, before continuing.  

\subsubsection{Actions}

Every turn, you are allowed to use the following actions:

\begin{itemize}
	\item One minor-action movement
	\item One major action or two minor actions of any kind
\end{itemize}


\def\wA{3.5}
\newcommand\actionEntry[2]{ \parbox[t]{\wA cm}{\raggedright #1} & \parbox[t]{\wA cm}{\raggedright #2} \\ }
{\scriptsize
\begin{center}
\begin{rndtable}{l l}
\bf Minor Actions & \bf Major Actions \\
\actionEntry{Making a {\it Quickattack} }{Making an attack or casting a spell}
\actionEntry{Using a large or complex item}{Using a simple item, such as drinking a potion}
\actionEntry{Moving up to half your movement speed}{Moving your full movement speed}
\actionEntry{{\it Evading} or {\it Bracing} }{}
\end{rndtable}
\end{center}
}
\subsubsection{Reactions}

After the {\bf Reveal} phase you may take an additional action change the outcome. You are generally limited to casting a spell to negate the effects (such as a Shield), or using {\it Evade} or {\it Brace}. 

After using a reaction, you roll a 1d6 at the end of every cycle. You can use a Reaction again after rolling a 5-6.

\subsubsection{Accuracy \& Instincts }

An {\it Accuracy} check is needed whenever there is a chance an attack may miss a target. Most beings instinctively either {\it Dodge} or {\it Block} an incoming attack, using whichever of their respective stats is highest:
\begin{align*}
 \text{Dodge} & = 10 + \text{\attFin{} modifier+ bonus}
 \\
 \text{Block} &= 10 + \text{\attPhys{} modifier+ bonus}
\end{align*}
These attributes set the DV of an {\it accuracy check}, performed using any relevant spellcasting or weapon modifiers. In combat, you may also choose to {\it Evade} or {\it Brace} as a minor action:

\def\w{3}
\def\c{6}
{\scriptsize
\begin{center}
\begin{rndtable}{p{1.2 cm} c | c}
~	&	\bf Brace	&	\bf Evade
\\
\cellcolor{\tablecolorhead}\bf Resist:	&	\parbox[t]{\w cm}{\raggedright Advantage on \attPhysShort{}, \attSprShort{} \& \attPowShort{} Resist checks.}	&	\parbox[t]{\w cm}{\raggedright  Advantage on \attFinShort{}, \attIntShort{} \& \attPerShort{} Resist checks}
\\
\cellcolor{\tablecolorhead}\bf Accuracy:	&	\multicolumn{2}{c}{\parbox[t]{\c cm}{\cellcolor{\tablecolorlight}\centering Agressors take disadvantage on accuracy checks made against you this turn}} 
\end{rndtable}
\end{center} 
}
\subsection{Magic}

To cast a spell, you must be holding your wand in your dominant hand, and be able to speak the incantaiton aloud, unless you have {\it Silent Casting} or {\it Wandless Casting} skills. Declare the spell you are about to cast	

\subsubsection{Casting}

For a memorised spell, you may automatically cast it at any level equal to or below the maximum memorised level, without needing an additional casting check (unless told otherwise).

For a non-memorised spell, you must equip a spellbook containing the spell and `book cast', which takes an entire combat cycle. In addition, you must perform a spellcasting check, using the specified check, against the DV of the spell. 

After book casting a spell $5 - $\attIntShort{} times (min 1), you memorise the spell. 

\subsubsection{Upcasting}

When memorised, some spells may be cast at a higher level than normal, giving them increased effect. For these purposes the spell acts like one of the chosen level. You must perform a spellcasting check at the appropriate level to upcast, until you have `memorised' it, though you do not require a spellbook to do so.



\subsubsection{Spellcasting Checks}
Every spell belongs to a Discipline, which determines the attribute modifier used for spellcasting checks.
\begin{center}
	\begin{rndtable}{c m{\xS cm} p{\wS cm}}
	\bf School	&	\bf Discipline	&	\bf Attribute
	\\
	\school{Charms}{Elemental}{\ElCheck}{Kinesis}{\KinCheck}
	\\
	\school{Divination}{Telepathy}{\TelCheck}{Temporal}{\TemCheck}
	\\
	\school{Illusion}{Bewitchment}{\BewCheck}{Psionics}{\PsiCheck}
	\\
	\school{Malediction}{Hexes}{\HexCheck}{Curses}{\CurCheck}
   \\ 
   \school{Recuperation}{Healing}{\HeaCheck}{Warding}{\WarCheck}
	\\
	\school{Transfiguration}{Alteration}{\AltCheck}{Conjuration}{\ConCheck}
	\\
	\school{Dark Arts}{Necromancy}{\NecCheck}{Occultism}{\OccCheck}
	\end{rndtable}
\end{center}

\subsubsection{Spellcasting DV}

For a cast to be successful, the result of the casting check must be equal to the spellcasting DV. This is determined by the caster level, and the level of the spell as shown in the table below, where a {\it Beginner} spell is considered level-1, {\it Novice} a level-2, and so on. 
\def\cc{\cellcolor{\tablecolorhead}\bf}
\def\vcol{\multirow{-6}{*}{\rotatebox[origin=c]{90}{\cc \bf Caster Level}}}
{\scriptsize
\begin{center}
\begin{rndtable}{c c c c c c c c c}
~	& ~ &	\multicolumn{7}{c}{\bf Spell Level}
\\
\cc	&	\cc	&	 \cc 0 & \cc 1 &\cc 2&\cc 3&\cc 4&\cc 5&\cc 6	
\\
\cc~	&	\cc1	&	8 & 	15~&~&~&~&~&
\\
\cc&\cc	2	&			5 & 	12	&	15~&~&~&~&
\\
\cc&	\cc3	& 		3 &		8	&	12	&	15~&~&~&~
\\
\cc&	\cc4	&		3 &		5	&	8	&	12	&	15~&~&~
\\
\cc&	\cc5	&		1 &		3	&	5	&	8	&	12	&	15 & 
\\
\vcol&	\cc 6 &			1 & 	3	&	5	&	8	&   12 	&15	&20
\end{rndtable}
\end{center}
}



\subsubsection{FP Costs}
Spells `cost' FP to cast. Failed spells cost half the amount of a successful spell:
{
\scriptsize
\def\wFP{1}
\begin{center}
	\begin{rndtable}{c c c    c  c c  c}
		{\bf Trivial} & {\bf Beginner}	&	{\bf Novice}	&	{\bf Adept}	&	{\bf Expert}	&	{\bf Master}	& {\bf Ascendant}
		\\
	0 & 2	&	4	&	8	&	16	&	32	&	64
	\end{rndtable}
\end{center}
}

\subsection{Resisting}

Some spells and other effects are not avoided by dodging or blocking, but instead by {\it Resisting}. A resist check is a normal d20 using the specified modifier. If not specified, the DV of a spell is given by a character\apos{}s {\it Arcane Subjugation} value for that spell:

$$ \text{AS} = 8 + \text{Power modifier + Spellcasting Modifier} $$

Some Resist checks are can be made using different modifiers depending on the situation, two common examples are:
\begin{itemize}
	\boldItem{Attention}{Observation or Investigation (GM\apos{}s choice)}
	\boldItem{Reflex}{Speed or Acrobatics (character\apos{}s choice)}
\end{itemize}

\end{multicols}
\end{landscape}
