
\chapter{Everyday Actions}

Within the framework of the game, there are broadly two classes of actions: {\it everyday} and {\it combat}. Everyday actions are things such as traveling between two cities, getting some sleep, talking to a friend, sitting in the library and so on. Combat, however, involves things trying to hurt you, and you trying to hurt them back. 

This section is concerned with the everyday, and is by no means meant to be an exhaustive list of things you may do. Instead, it merely provides some guidelines as to how to perform some common actions, and the effects that they can have. 

\section{Roleplaying}



\section{Movement}

Out of combat, wandering around the environment is very natural -- you simply tell the GM that you want to go over there, and you do - barring unforeseen circumstances such as traps. You needn't calculate the exact time taken for each individual movement (that would get dull), but it is generally presumed to occur on the scale of seconds to a few minutes. 


However, sometimes you might wish to travel over distances which will take more than a handful of minutes. If you are travelling by foot more than 10 minutes, then you need to decide how rapidly and carefully you are moving.


\begin{center}
\begin{rndtable}{|c c c m {3 cm}|}
\hline
Pace & Speed & Duration & Effect
\\
\hline 
Slow & 2km/h & 8 hours & Can remain hidden, or draw a map
\\ 
Normal & 4 km/h & 7 hours & Can draw a map
\\ 
Rapid & 6 km/h & 5 hours & -5 penalty to all checks made whilst moving. Costs 5 FP per hour.
\\ 
Breakneck & 10km/h & 1 hour & {\raggedright- 10 penalty to all checks checks made whilst moving. Costs 2 FP per minute and 5 HP per hour.}
\\ \hline
\end{rndtable}
\end{center}

If you attempt to travel for longer than the `duration' of the selected pace, you risk exhausting yourself. After the first additional kilometre travelled, all members of the party must succeed a DV 10 ATH (endurance) check. This check must be repeated after every subsquent kilometre travelled, with the DV increasing by 1 each time. After failing this check, you must halt, and take an additional level of exhaustion. 

This timer resets after a rest of more than 8 hours, after which time you can take up your pace again. 

\subsection{Vehicles \& Mounts}

Of course, the discerning wizard rarely travels too far on foot - they may prefer to use a broomstick, tame and ride a griffin or simply apparate or portkey around. 

Each of this modes of transport has their own limitations, specified by the relevant item, beast or spell effects. 

\subsection{Actions while moving}

It is possible to perform other actions whilst on the move, though unless you are travelling in a luxury carriage, you may be somewhat restricted in what exactly you can achieve. 

You may make checks to navigate, to track a foe keep or to keep an eye out for enemies (these all use variations on the PER attribute), or you may leverage your knowledge of Flora & Fauna to forage for food and water. The faster you travel, the heavier a penalty you suffer for these checks. 

Whilst travelling at a slow pace, you may make an effort to remain hidden, the rules for which are elaborated on more on page \pageref{S:Stealth}. 

If the Slow or Normal pace is used, a member of your part may elect themselves as a map-maker, if they have the {\it Observation} proficiency. Having a map makes it impossible to get lost (unless the scenery is magically altered, of course), and you can always retrace your steps. 

\subsection{Special Movement}

Walking and running are not the only kinds of movement out there: navigating a dangerous environment often requires other ways of exploring the space. 

\subsubsection{Climbing}

You may navigate slopes of up to 20 degrees (2 in 5) without difficulty. Between 30 degrees and 50 degrees you must move at half speed, but can walk normally. For slopes between 50 degrees and 90 you must use your hands to climb. If you wish to use a hand to perform an action, you must halt, check you have a secure handhold with the other hand, and then continue. Failure to do so will lead to a (possibly fatal) fall.


\subsection{Resting}

Resting is an important action that can only occur when not in combat. Attempts to rest during combat are highly likely to get you killed on the spot. 

When in safe territory, you may set up camp, and get a few hours shut-eye to recover from your ordeals (see the Asleep status effect for details). But be warned, the night is dark and full of terrors, and who knows what might sneak up on you whilst you are resting…

You may take rests whilst delving deep into unfriendly territory, but note that resting after every single encounter is generally frowned upon, and the GM might start throwing more and more unpleasant random encounters at you if you begin to take things to the extremes. 

You should only rest in a place where it makes sense to rest – it does not makes sense, for example, to take a quick nap in whilst delving through the dungeons of an evil warlord, even if you have cleared the immediate area of enemies. Of course, if you kill the Warlord and claim his castle as your own, then it is a different matter...

\section{Social Actions}

\section{Downtime}