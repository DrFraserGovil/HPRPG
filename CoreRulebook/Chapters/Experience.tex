\documentclass[CoreRulebook.tex]{subfile}

\chapter{Character Progression}

Each character has a `level' associated with them, which denotes how far your character has progressed, and how powerful they are.  Levelling your character is key to progressing – it unlocks new skills, boosts your attributes, and gives access to new spells. A higher-level magic user is a stronger magic user. A stronger magic user is less likely to get eaten by a passing beast, which is generally considered a bad thing. 

Increasing the level of your character (`levelling up') is achieved by accumulating experience. To progress from level 1 to level 2, you must accumulate 100 experience points (EP). When your character reaches 100EP, they ascend to level 2, and the counter is reset. To go from level 2 to level 3 you need to acquire another 200 EP, and so on and so forth. The EP needed to go from level $x$ to $x+1$ is calculated from:

$$ EP_{x \to x + 1} = 100 x $$

Experience is gained by completing actions and defeating enemies, and is gained differently inside and outside of combat. 
Outside of combat, experience is awarded for completing difficult actions – such as casting a spell, mixing a potion, or convincing someone to give you something. The GM will instruct you to roll a dice, and you will gain that much experience from completing the action.

The dice you roll (and hence the amount of experience you gain) from such an action depends on your proficiency in that skill. For instance, a first year student gains far more knowledge and experience from casting wingardium leviosa than a seasoned auror does. Hence, as you progress, you will learn less experience from trivial actions. 

As a rough guide, casting a spell which is of the same proficiency level as you are will get a d20 roll, casting a spell one level below your proficiency is a d12, and so on:

\begin{center}
	\begin{tabular}{|c|c|}
	\hline \bf Relative Proficiency & \bf Experience Roll
	\\ \hline \hline
	Same level 		& 	d20
	\\ \hline
	1 level below 	&	d12
	\\ \hline
	2 levels below	& 	d8
	\\ \hline
	3 levels below	&	d6
	\\ \hline
	4 levels below 	& 	d4
	\\ 	\hline
\end{tabular}
\end{center}

For example, a character with the Adept Battlemage (combat magic) skill would roll a d20 for successfully casting the Impediment Jinx (an adept level combat spell), whilst if they were an Master Thaumaturge (transfiguration), they would only get to roll a d8 for casting an Adept transfiguration spell, as this is 2 levels below Master. 

Other actions will follow a similar pattern of experience awarding, at the GM’s discretion. Experience is only awarded when an action is truly succesful (i.e. a spell has to hit its target, as well as be succesfully cast). In addition to this action-based experience, you can gain experience by defeating enemies: more difficult enemies award more experience. 

When your experience reaches the requisite amount, you immediately trigger the levelling up process: you get 1 attribute point to allocate at will (see below) and you get 1 skill point to spend on a new skill. Skill points must be spent immediately, and cannot be saved for later. Equally, levelling up is an immediate action when you reach the required EP – you cannot choose to delay this process!

You also reset your spell-learned counter back to zero – you may start to learn more spells again – as well as resetting your HP and FP to maximum (you may need to adjust your max HP and FP levels to accommodate your new attributes). Your EP counter is also reset, though you may carry over any extra EP from the previous level (i.e. if you were a level 2 character on 180EP and got 30 experience points, you could start level three on 10EP). 

After levelling up, you may increase one of your attributes by 1 -- with the exception of your EVL stat. EVL is not levelled up by choice, but directly by the actions that you take. If EVL < 5, you gain one point per innocent that you slaughter. As you EVL increases, you need to perform larger and more grotesque atrocities to increase your EVL stat, at the whim of the GM.

The GM may also decide that, during the normal course of play, you have done something that warrants a permanent bonus -- be it something you have learned from extensive practice, or a gift from some higher being -- the GM will grant you a bonus to your Proficiencies. Outside of those granted in skills, this is the usual way to gain proficiency in these areas. 