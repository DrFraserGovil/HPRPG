
\breaklesschapter{Character Progression} \label{S:Progression}\index{Character Progression|see{Progression}}\index{Progression}\index{Experience|see{Progression (Experience)}}

	
	As a character progresses through the world, they gain experience, knowledge and new skills. This allows them to develop their abilities and hence allow them to become more powerful. A \imp{Warrior} hits harder, a \imp{Scholar}'s brain is sharpened to a razor's edge, and the \imp{Artificer} learns new designs and techniques of their craft. 
	
	This character progression is enumerated through {\imp Experience Points} (\imp{Exp}). 
	
	\section{Earning Experience}\index{Progression!Experience}\index{Progression!Additional Triggers}
	
	You earn \imp{Exp} by solving problems - overcoming obstacles, defeating foes, learning new things and otherwise growing as a character. 
	
	At the end of a long day, just before you head off to sleep, you should always think back and reflect upon what you have achieved during that time: \imp{Exp} is awarded at the culmination of an adventure, in a pause in the frantic adventure, or when a character has a moment to breathe and reflect. If the \imp{GM} has timed their session well, this will often occur at the end of a session - though if a natural point is reached in the middle of a session this should not be shyed away from. 
	
	When the GM decides it is time to distribute \imp{Exp}, they will take into consideration the following:
	\begin{itemize}
		\boldItem{How many serious obstacles did they overcome?}{Did the group face a problem and neutralise it? What issues did they face? Each significant task can be awarded 1\imp{Exp}}
		\boldItem{Were these tasks new and challenging?}{Defeating a basilisk, or being chosen as prefect for the 5th time is probably less instructive than it was the 1st. If the task was especially new or novel, you may be granted an additional \imp{Exp}}
		\boldItem{Did they use their abilities to solve them?}{Each \imp{Archetype} states that they gain additional \imp{Exp} whenever they use a certain ability or approach to solve a problem.}
		\boldItem{Did they grow as a character?}{This final question is used to reward good roleplaying - overcoming internal, personal challenges, as much as those imposed by the GM. If the GM feels that a player went out of their way to inhabit and develop a character, this may be rewarded with \imp{Exp}}. 
	\end{itemize}
	
	You may negotiate with the GM and remind them of what you have accomplished and overcome since you last reflected on your achievements, but their ruling on this is final. After being awarded \imp{Exp} you may store it in the \imp{Exp Rail} on your character sheet. 
	
	
	\section{Expending Experience}\index{Aspects}\index{Abilities}\index{Affinities}\index{Abilities!Major \& Minor Abilities}\index{Feats}\index{Health}\index{Fortitude}
	
	During these moments of reflection and growth, you may also spend these experience points to increase your abilities as a character. 
	
	The available options, and the associated \imp{Exp} cost for each of these is shown below:
	
	\begin{center}
		\begin{rndtable}{r l}
			\bf Ability	&	\bf Exp Cost
			\\
			\progressionCosts
		\end{rndtable}
	
	\end{center}
	
	Therefore if Simone has a rating of three in \imp{Fitness}, she would need $2\times4 = 8$ experience to increase it to a 4-rating ability. However, given she has a rating of 3 in \imp{Elemental}, she would need 12 points to increase this to a 4-level rating. 


	\section{Additional Feats} \label{S:AllFeats}
	
	In addition to feats specific to each \imp{Archetype}, below a number of feats that any character can take are presented. 
	
	\newcommand\animal[2]{\imp{#1} & \parbox[t]{5cm}{\raggedright #2} \\}
	\def\ArtificerFeats
{
	\feat{Adept Alchemist }{Whenever you undertake a potion\minus{}mixing effort\comma{} you gain one additional auto\minus{}success. You may also `discover’ one common ingredient which has up to three properties of your choosing\comma{} discussing this with the GM.}{0}{}
	\feat{Expert Enchanter }{Whenever you undertake an enchanting effort\comma{} you gain one additional auto\minus{}success. You may also learn new \imp{Runes} with only 1 hour of study.}{0}{}
	\feat{Hidden Work }{When you complete an \imp{Imbuing} or \imp{Crafting} project\comma{} you
may expend an additional hour to make your work completely hidden from inspection.
Runes are hidden and alchemical creations can appear as mundane
fluids. Only upon activation or a spell such as Identify can the true
nature be divined.}{0}{}
	\feat{Idiosyncrasies }{You know every oddity and quirk of your own creations: when using them\comma{} you gain one additional auto\minus{}success.}{0}{}
	\feat{Master Mechanist }{Whenever you undertake a tinkering or mechanical manufacturing effort\comma{} you gain one additional auto\minus{}success. You may also stray further from `realistic’ or scientific constructions and may handwave slightly more vigorously over the workings of your constructions.}{0}{}
	\feat{Quick Worker }{You take only half the normal time to perform feats of \imp{Crafting} and \imp{Imbuing} – you may perform checks every 3 hours\comma{} rather than every 6.}{0}{}
	\feat{Siege Master }{When dealing damage to or attempting to bypass a building\comma{} structure\comma{} wall\comma{} door or other such solid object\comma{} you gain one additional auto\minus{}success and deal an additional level of harm.}{0}{}
	\feat{Thick Skin }{Years of accidents and lab mishaps have left you with a superhuman level of resilience: choose from \imp{Fire} and \imp{Concussive} or \imp{Acid} and \imp{Poison} damage: any rolls to \imp{Resist} damage of the chosen types has a DV that is 2 lower than normal.}{0}{}
	\feat{Wandmaker }{You have perfected the art of crafting magical focusses\comma{} and can create a new wand for yourself or others. This takes up to 3 days\comma{} but you can craft the new `wand’ into any form you like – a mighty oaken staff\comma{} or a bejewelled necklace\comma{} but must remain an item in their possession which the spellcaster can focus on when using their magic.}{0}{}
}

\def\WarriorFeats
{
	\feat{All Guns Blazing }{When making an attack against a small group of people\comma{} you can truly throw yourself into the attack\comma{} expending two \imp{Fortitude} points to take them all on at once. Make attacks against a number of beings (up to twice your \imp{Rage} score) within range. You cannot use this ability on consecutive turns.}{0}{}
	\feat{Blind Rage }{When using a \imp{Rage} action to attack\comma{} you ignore all dice penalities due to injuries.}{0}{}
	\feat{Bloodlust }{On any turn in which you successfully incapacitate (lethally or not) a foe\comma{} you may take an additional free action to perform another attack\comma{} with a 2\minus{}dice penalty on the check.}{0}{}
	\feat{Duelist }{When fighting against a single foe\comma{} you gain one auto\minus{}success on all attack rolls and Resist actions.}{0}{}
	\feat{Furious Spellcaster }{Up to three times a day\comma{} you may elect to use a \imp{Rage} check\comma{} rather than the associated \imp{Affinity} to perform a spellcasting check. The maximum spell level is still determined by your \imp{Affinity}.}{0}{}
	\feat{Holistic Tactics }{Whenever a \imp{Tactics} check is successful in determining the tactics of a foe\comma{} you may use this knowledge to infer any Resistances\comma{} Immunities or Susceptabilities the target has.}{0}{}
	\feat{Lightning Strikes }{At the end of each long rest\comma{} perform a DV 7 check using just your \imp{Speed} pool. For each success gained (min 1) you may perform one additional attack at some point over the next day without expending a \imp{Fortitude} point.}{0}{}
	\feat{Savage Attacker }{Your attacks\comma{} both physical and magical\comma{} deal one additional point of \imp{Harm}.}{0}{}
	\feat{Student of War }{If you study\comma{} read up on and otherwise prepare for a target before engaging them in combat\comma{} the DV of all attacks you make against them is reduced by 1.}{0}{}
}

\def\AurorFeats
{
	\feat{Ambush }{When you attack from hiding\comma{} spring a trap or successfully orchestrate
an ambush\comma{} your first attack is particularly powerful: when you complete your first attack roll\comma{} if the total number of successes is below  half the number of dice rolled\comma{} you may instead use that number.}{0}{}
	\feat{Cold Cases }{When performing a Knowledge check\comma{} if you can relate the information
you seek to a historical or past case you reduce the DV by 3.}{0}{}
	\feat{De\minus{}escalation Training }{You are trained specifically to capture and contain\comma{} not to kill. When you take an action to contain\comma{} constrain\comma{} bind\comma{} trap or disarm a foe\comma{} rather than inflict pain or damage upon them\comma{} you gain +1d to the effort.}{0}{}
	\feat{Familiar Terrain }{Choose a favoured terrain such as \imp{Grasslands}\comma{} \imp{Forests}\comma{} \imp{Urban Areas}\comma{} \imp{Caverns}\comma{} or name a specific region\comma{} such as \imp{Hogwarts}. Whilst in your favoured terrain you gain +1d on every action which utilises the surroundings such as a \imp{Tracking} or \imp{Covert} check.}{0}{}
	\feat{Fancy Footwork }{When fighting more than one foe\comma{} you may use an action to expend a \imp{Fortitude} point to confuse your foes with some feat of athletics and maneouvering\comma{} causing them to attack each other. Nominate two enemies within range – next turn cycle\comma{} the first of these two to take an action will attack the other\comma{} instead of their intended target.}{0}{}
	\feat{Lie Detector }{You can automatically detect when someone is lying to you by telling you deliberate falsehoods}{0}{}
	\feat{Mental Training }{You have trained your mind to resist the effects of external manipulation. You gain +3d against all checks to resist unnatural mental manipulation\comma{} and may expend a \imp{Fortitude} point to end an ongoing mental effect such as \imp{Charmed}.}{0}{}
	\feat{Rapid Reflexes }{When performing a \key{Reflex} roll\comma{} you may roll the dice twice and take the largest value.}{0}{}
	\feat{Unwavering Focus }{Once per day you may expend a \imp{Fortitude} point to reroll all \imp{Catastrophe} dice you rolled\comma{} declaring this action after the roll has been performed\comma{} but before the outcome has been narrated.}{0}{}
}

\def\AllFeats
{
	\feat{Armour Piercing }{When a target attempts to \imp{quickblock} your attacks\comma{} their armour takes two levels of \imp{Drain}.}{0}{}
	\feat{Attuned Attacks }{When fighting unarmed\comma{} or with a non\minus{}magical weapon\comma{} you can channel your very life\minus{}force into your attacks. Such attacks deal an additional level harm\comma{} and bypass any resistance to physical attacks.}{0}{}
	\feat{Course 101 }{You study a crash course in a selection of 10 abilities you previously had no skill in\comma{} giving you a basic level of knowledge. Choose up to 10 \imp{Abilities} with a \emptyCape{} rating\comma{} and gain 1 dot in each of them. If\comma{} when you take this ability\comma{} it costs more than 9 dots\comma{} pay only 9 dots.}{0}{}
	\feat{Elemental Attunement }{You feel a particular affinity for one of the elements (Fire\comma{} Water\comma{} Ice\comma{} Earth\comma{} Air\comma{} Lightning\comma{} etc.) deep within your bones. When casting a spell to manipulate\comma{} create or otherwise effect your chosen element\comma{} you gain +1d. You also gain +1d to any check to resist damage caused by your element.}{1}{\imp{Elemental} (\threeCape{})}
	\feat{Helping Hand }{You are so proficient in helping out your allies that your `help’ action gives +3d\comma{} rather than +1.}{0}{}
	\feat{Innate Trick }{As a witch or wizard\comma{} the chaotic force of magic flows within your veins. You have learned to harness this magic in some innate way beyond the usual spellcasting. This effect is usually minor (something a Muggle could put down to an act of trickery or showmanship)\comma{} and often forms the basis of a parlour trick. 

You might be able to summon a small flame from your finger\comma{} make your eyes into burning coals or deep black voids\comma{} play a stirring soundtrack whenever they engage in a fight\comma{} know the name of every individual you meet\comma{} or some other marvellous but ultimately slightly inconsequential feat that you could imagine being the focus of conversation at a party. 

No rolls are needed to use this ability\comma{} and the GM has a veto if this tool is being used in an inappropriate fashion.}{0}{}
	\feat{Jack\minus{}of\minus{}all\minus{}Trades }{You have a surprising amount of miscellaneous skills\comma{} knowledge and abilities that you have acquired over your life\comma{} and are often able to surprise your allies with something pulled from your sleeve. 

Each day\comma{} you get 4 free dots\comma{} which you may temporarily allocate to \imp{Abilities} as and when you need them\comma{} though you may not increase any ability to more than 5 dots. You may use this enhanced ability for the next hour\comma{} before the effect wears off. You regain your dots when you complete a \imp{Long Rest}.}{0}{}
	\feat{Light Sleeper }{You need much less sleep than others\comma{} and can go from asleep to awake in a blink of an eye. You gain the benefits of a \imp{Long Rest} after only 4 hours\comma{} and may ignore the effects of \imp{Level One Exhaustion}. Any \imp{Alertness} checks called for whilst asleep have the DV reduced by 3.}{0}{}
	\feat{Linguist }{You have studied another language enough to be considered fluent in it. When conversing in this language\comma{} gain +2d to all social checks. You may take this ability multiple times\comma{} learning a new language each time.}{0}{}
	\feat{Loyal Companion }{You have an animal ally which is eternally loyal and devoted to you\comma{} and can carry out simple tasks: a `familiar’. The most common animals are owls\comma{} ravens\comma{} cats\comma{} rats and toads\comma{} though you may ask your GM for a different choice.}{1}{Kinship (\twoCape)}
	\feat{Martial Arts }{You are a master of unarmed combat\comma{} making your hands into lethal weapons. Unarmed strikes deal damage equal to the number of successes (DV 6). You may expend a Fortitude point to reduce the DV of an unarmed strike to 3.}{1}{Brawl (\threeCape)}
	\feat{Moving Target }{On a turn during which you move more than half your movement (without doubling back)\comma{} your \imp{quickdodge} checks do not incur \imp{Drain}.}{0}{}
	\feat{Numbed to Pain }{When you expend a \imp{Fortitude} point to ignore the negative effects of \imp{Harm}\comma{} the effect lasts for one hour\comma{} rather than just the next round.}{1}{Vitality (\threeCape)}
	\feat{Psychic Awareness }{Your mind is especially attuned to those of others\comma{} and you can naturally sense the shift induced when a psychic power alters or interacts with minds. Whenever a psychic effect such as mind reading\comma{} memory modification\comma{} or magic which alters emotions and allegiances is used on a target within 5m of you\comma{} you are automatically aware of this\comma{} though you are not aware of the source.}{1}{\imp{Kindness} (\threeCape{})}
	\feat{Ritualist }{You are a strong believer that the most powerful magic is performed with large groups\comma{} in elaborate rituals\comma{} with chanting\comma{} incense and possibly a pentagram or two. Whenever you invoke a \imp{Ritual} to cast a spell\comma{} you gain one automatic success for every 3 members of the ritual (max +5d).}{1}{\imp{Occultism} (\threeCape{})}
	\feat{Second Chances }{Once per day\comma{} you may re\minus{}roll any number of dice on a single check\comma{} but must keep the new result.}{0}{}
	\feat{Signature Spell }{You have a spell which is considered your `signature move’\comma{} chosen when you take this feat. When casting this spell\comma{} you gain one additional auto\minus{}success. You may change your `signature spell’ only with GM consent that your old choice no longer represents your character’s go\minus{}to move.}{0}{}
	\feat{Silent Casting }{You do not need to perform the verbal component of a spellcasting action. Efforts to silence you do not impact your spellcasting efforts\comma{} and reactions to your spells take a 1d penalty.}{0}{}
	\feat{Wandless Casting }{You are able to perform limited feats of magic without needing the crutch of a wand or ritualistic movements\comma{} so attempts to disarm your or bind you in place do not affect your spellcasting efforts. You take a 1d penalty on all wandless spellcasting efforts. All wandless actions are also silent.}{1}{Silent Casting}
}

\def\DruidFeats
{
	\feat{Asteria’s Eyes }{When you cast a spell from the \imp{Divination} school of magic (\imp{Cerebral} or \imp{Temporal})\comma{} if you can see the stars\comma{} you gain +2d to the check.}{0}{}
	\feat{Cloak of Seasons }{You are magically protected from the effects of the weather and the natural environment. You are perfectly comfortable in winter's chill or summer's blazing heat regardless of your clothing (or lack thereof). You do not suffer from sunstroke or exposure. You're not even bitten by insects or other vermin. Your senses are still limited by the elements (including fog\comma{} rain and snow)\comma{} and you're not protected from either hunger or thirst.}{0}{}
	\feat{Dryad’s Embrace }{When you cast a spell on or attempt to \imp{Commune} with plant\minus{}based beings\comma{} or attempt to use a \imp{Knowledge} check to learn about such an entity\comma{} you gain one additional auto\minus{}success.}{0}{}
	\feat{Exuding Aura }{You may expend a \imp{Fortitude} point to attune yourself to your favoured aspect of nature\comma{} exuding an aura which influences the minds of others – perhaps a sweet pine smell calms them\comma{} or animal pheremones send them into a frenzy. You gain +1d on all \imp{Social} checks made against people within 2m of you for the next hour.}{0}{}
	\feat{Green Thumb }{If you so choose\comma{} you can become a beacon of vibrant plant life. Flowers spring up in your footsteps and trees burst into bloom at your touch. Your hands are always warm and comforting\comma{} and plants will avoid hurting you\comma{} blunting their thorns\comma{} or dulling their poison as you pass. This effect is somewhat limited (you cannot heal a field of necromantic blight\comma{} for example\comma{} and plants may retaliate if under sustained injury)\comma{} but plants will recognise you as a source of light and life.}{0}{}
	\feat{Nymph’s Fury }{Channeling the power of primal\comma{} elemental spirits grants you additional power. When casting a spell from the \imp{Elemental} discipline\comma{} you gain one additional auto\minus{}success.}{0}{}
	\feat{Organic Repose }{Once per day\comma{} you may expend a \imp{Fortitude} point to recover 3 levels of Health}{0}{}
	\feat{Satyr Spirit }{When casting a spell or attempt to \imp{Commune} with a \imp{Beast}\comma{} or attempt to use a \imp{Knowledge} check to learn about such a creature\comma{} you gain one additional auto\minus{}success.}{0}{}
}

\def\OutlawFeats
{
	\feat{Black Market }{You know just where to acquire forbidden items\comma{} and source unscrupulous materials\comma{} and your experience with dealing with such people grants you +1d on all bartering checks.}{0}{}
	\feat{Cover Identity }{Given enough time – perhaps a day or two – you can forge yourself a completely new identity\comma{} with the necessary paperwork and credentials to reasonably pass as whoever you desire. This may not stand up to high\minus{}level scrutiny\comma{} but most people should be easily fooled.}{0}{}
	\feat{Hidden Weapon }{Up to three times a day\comma{} you can draw a previously unknown small blade from a fold in your robes\comma{} or a hidden pocket\comma{} and then use it. This is an instantaneous action.}{0}{}
	\feat{Innocent Face }{You are always thought of as an honest and good soul. If you do something wrong which isn’t immediately attributed to you\comma{} it will most likely be blamed on something else. As long as you’re not caught red\minus{}handed killing puppies\comma{} people will try to excuse your actions and move on from your misdeeds.}{0}{}
	\feat{Move in Shadow }{Whilst outside of bright light\comma{} all attempts to percieve you have a DV 2 higher than normal.}{0}{}
	\feat{Naturally Shifty }{Doing unscrupulous deeds comes as naturally to you as breathing – gain one additional auto\minus{}success on any \imp{Covert} action.}{0}{}
	\feat{Sly Action }{At the end of a turn cycle\comma{} if you have not been directly targeted for an attack\comma{} you may take an additional action at the end of the cycle to move\comma{} use an item\comma{} or otherwise attempt to hide}{0}{}
	\feat{Surprise Attack }{Whenever you attack a target from a position where they cannot see you\comma{} you deal an additional level of harm}{0}{}
	\feat{Unobtrusive }{You don’t stand out in a crowd\comma{} and can make yourself\comma{} if not invisible\comma{} just socially {\it absent}. People don’t necessarily remember your face or your name (if they remember you at all)\comma{} as you make very little impression on people – until you are stealing their wallet\comma{} or knifing them in the back\comma{} that is. All checks made by enemies to notice you in a crowd\comma{} or remember details about you have a DV 3 higher than normal.}{0}{}
}

\def\SophisticateFeats
{
	\feat{Brazen }{You are so brash and bold in your approach that you can simply breeze past an error\comma{} playing it for laughs\comma{} or simply ignoring it altogether. Whenever you perform a \imp{Social} check\comma{} you may treat one \imp{Catastrophe} as a normal\comma{} benign failure.}{0}{}
	\feat{Burn Book }{When using your skills to spread rumours\comma{} misinformation or gossip\comma{} or when trying to discredit an individual\comma{} gain +2 dice}{0}{}
	\feat{Mesmerising Presence }{Once per day\comma{} you may use your alluring charm to slightly hypnotise a person\comma{} gaining +1d to social checks with them. They remember talking to you\comma{}  but are slightly starstruck and overpowered by your personality\comma{} such that they cannot recall what you talked about.}{0}{}
	\feat{Natural Leader }{You're a natural born leader. While not everyone will simply surrender authority to you\comma{} they'll consent to "follow your lead." Reduce the DV of any check directly related to leadership by 3.}{0}{}
	\feat{One for you\comma{} two for me }{Whenever your or your allies gain an amount of \imp{Galleons}\comma{} you gain one additional coin.}{0}{}
	\feat{Poker Face }{You are an expert at hiding your true feelings – beings cannot rely on the usual cues to read your demeanour or true|motivation. Any \imp{Insight} attempts against you have a DV 2 higher than normal.}{0}{}
	\feat{Sue For Peace }{You excel at halting violence when it breaks out. Whenever you \imp{Surrender}\comma{} it is automatically accepted. You may also expend a \imp{Fortitude} point to try to convince your foe to surrender to you – force your opponents to perform a resist check such as \imp{Willpower (Bravery)} with the DV set by the value of the dice pool you would use to sue for peace \minus{} (\imp{Charm (Eloquence)} for a rational argument\comma{} or perhaps \imp{Willpower (Intimidation)} to cow them into submission). On a failure\comma{} they will \imp{Surrender} or \imp{Flee}.}{0}{}
	\feat{Unbreakable Vow }{When you willingly shake on a deal or contract with another sapient being both you and your partner are bound together by a magical oath. If either of you breaks the contract\comma{} the offending party takes the maximum amount of \imp{Harm} and falls into a \imp{Critical Condition}\comma{} alerting the other.}{0}{}
}

\def\ScholarFeats
{
	\feat{Expertise }{Choose a spell discipline\comma{} or a viable target of a spell\comma{} which is associated with your area of research or specialty. When casting a spell of this school\comma{} or a spell on your chosen target\comma{} you gain one additional auto\minus{}success.}{0}{}
	\feat{Healer }{Whenever you restore Health to a being\comma{} heal an additional level of \imp{Harm}.}{0}{}
	\feat{Lightning Mind }{You can perform complex mathematical calculations at the speed of a computer\comma{} even when under life\minus{}threatening stress. The player controlling this character may use a calculator and refer to the statistical tables to evaluate the outcome of an action at any time\comma{} even when in\minus{}game time is short.}{0}{}
	\feat{Master of the Mind }{When an action would interrupt you casting a spell requiring continuing concentration\comma{} the DV to remain focussed is 2 lower than normal.}{0}{}
	\feat{Novel Technique }{Whenever you use a spell in a new and novel fashion\comma{} you gain +2d for the spellcasting effort.}{0}{}
	\feat{Quick Learner }{You need to spend half the usual time in order to learn a new spell or potion recipe.}{0}{}
	\feat{Well Read }{You hold in your brain a simply incredible amount of information\comma{} which leads to sudden flashes of inspiration and insight. Once per \imp{Long Rest}\comma{} you may use one of these flashes to gain +5d on one \imp{Knowledge} check.}{0}{}
}


	\AllFeats
