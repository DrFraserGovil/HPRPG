
\chapter{Character Progression} \label{S:Progression}

	
	As a character progresses through the world, they gain experience, knowledge and new skills. This allows them to develop their abilities and hence allow them to become more powerful. A \imp{Warrior} hits harder, a \imp{Scholar}'s brain is sharpened to a razor's edge, and the \imp{Artificer} learns new designs and techniques of their craft. 
	
	This character progression is enumerated through {\imp Experience Points} (\imp{Exp}). 
	
	\section{Earning Experience}
	
	You earn \imp{Exp} by solving problems - overcoming obstacles, defeating foes, learning new things and otherwise growing as a character. 
	
	At the end of a long day, just before you head off to sleep, you should always think back and reflect upon what you have achieved during that time: \imp{Exp} is awarded at the culmination of an adventure, in a pause in the frantic adventure, or when a character has a moment to breathe and reflect. If the \imp{GM} has timed their session well, this will often occur at the end of a session - though if a natural point is reached in the middle of a session this should not be shyed away from. 
	
	When the GM decides it is time to distribute \imp{Exp}, they will take into consideration the following:
	\begin{itemize}
		\boldItem{How many serious obstacles did they overcome?}{Did the group face a problem and neutralise it? What issues did they face? Each significant task can be awarded 1\imp{Exp}}
		\boldItem{Were these tasks new and challenging?}{Defeating a basilisk, or being chosen as prefect for the 5th time is probably less instructive than it was the 1st. If the task was especially new or novel, you may be granted an additional \imp{Exp}}
		\boldItem{Did they use their abilities to solve them?}{Each \imp{Archetype} states that they gain additional \imp{Exp} whenever they use a certain ability or approach to solve a problem.}
		\boldItem{Did they grow as a character?}{This final question is used to reward good roleplaying - overcoming internal, personal challenges, as much as those imposed by the GM. If the GM feels that a player went out of their way to inhabit and develop a character, this may be rewarded with \imp{Exp}}. 
	\end{itemize}
	
	You may negotiate with the GM and remind them of what you have accomplished and overcome since you last reflected on your achievements, but their ruling on this is final. After being awarded \imp{Exp} you may store it in the \imp{Exp Rail} on your character sheet. 
	
	
	\section{Expending Experience}
	
	During these moments of reflection and growth, you may also spend these experience points to increase your abilities as a character. 
	
	The available options, and the associated \imp{Exp} cost for each of these is shown below:
	
	\newcommand\expRow[2]{{\bf #1}	&	\parbox[t]{4cm}{#2}	\\}
	\begin{center}
		\begin{rndtable}{r l}
			\bf Ability	&	\bf Exp Cost
			\\
			\expRow{Increase Aspect Rating}{$=~2\times$ new attribute score}
			\expRow{Increase Affinity Rating}{$=~3\times$ new affinity score}
			\expRow{Increase Major Ability Rating}{=~New \imp{Ability} score}
			\expRow{Increase Minor Ability Rating}{=~ 1 + New \imp{Ability} score}
			\expRow{Swap Major and Minor Ability}{ =~ 1 \imp{Exp}}
			\expRow{Gain New Feat}{ =~ 7 + Current number of feats}
			\expRow{Increase Health or Fortitude}{ = 10 + 4 for each previous purchase}
		\end{rndtable}
	
	\end{center}
	
	Therefore if Simone has a rating of three in \imp{Fitness}, she would need $2\times4 = 8$ experience to increase it to a 4-rating ability. Since she has already gained 1 \imp{Feat} previously, she could also purchase a second for 8 \imp{Experience} points. However, given she has a rating of 2 in \imp{Elemental}, she would need 9 points to increase this to a 3-level rating. 
%~ Each character has a `level' associated with them\comma{} which denotes how far your character has progressed\comma{} and how powerful they are.  Levelling your character is key to progressing: it unlocks new skills\comma{} boosts your attributes\comma{} and gives access to new spells. A higher\minus{}level magic user is a stronger magic user. A stronger magic user is less likely to get eaten by a passing beast\comma{} which is generally considered a good thing. 

%~ \section{Experience}

%~ Increasing the level of your character (`levelling up') is achieved by accumulating experience. To progress from level 1 to level 2\comma{} you must accumulate 100 experience points (EP). When your character reaches 100EP\comma{} they ascend to level 2\comma{} and the counter is reset. To go from level 2 to level 3 you need to acquire another 200 EP\comma{} and so on and so forth. The EP needed to go from level $x$ to $x+1$ is calculated from:

%~ $$ EP_{x \to x + 1} = 100 x $$

%~ Experience is gained by completing actions and defeating enemies. Experience is awarded for completing difficult actions such as casting a spell\comma{} mixing a potion\comma{} defeating an enemy in combat\comma{} or convincing someone to give you something. The GM will instruct you to roll a dice\comma{} and you will gain that much experience from completing the action.

%~ The dice you roll (and hence the amount of experience you gain) from such an action depends on your proficiency in that skill. For instance\comma{} a first year student gains far more knowledge and experience from casting wingardium leviosa than a seasoned auror does. Hence\comma{} as you progress\comma{} you will learn less experience from trivial actions. 

%~ As a rough guide\comma{} performing an action (such as casting a spell) which is of the same proficiency level as you are will get a 2d20 roll\comma{} using one level below your proficiency is a 2d12\comma{} and so on:

%~ \begin{center}
	%~ \begin{rndtable}{|c c|}
	%~ \hline \bf Relative Proficiency & \bf Experience Roll
	%~ \\ 
	%~ Same level 		& 	2d20
	%~ \\ 
	%~ 1 level below 	&	2d12
	%~ \\ 
	%~ 2 levels below	& 	2d8
	%~ \\ 
	%~ 3 levels below	&	2d6
	%~ \\ 
	%~ 4 levels below 	& 	2d4
	%~ \\ 	\hline
%~ \end{rndtable}
%~ \end{center}

%~ For example\comma{} a character with the Adept Battlemage (combat magic) skill would roll a 2d20 for successfully casting the Impediment Jinx (an adept level combat spell)\comma{} whilst if they were an Master Thaumaturge (transfiguration)\comma{} they would only get to roll a 2d8 for casting an Adept transfiguration spell\comma{} as this is 2 levels below Master. 

%~ Experience is only awarded when an action is truly succesful (i.e. a spell has to hit its target\comma{} as well as be succesfully cast). 

%~ \section{Levelling Up}
%~ When your experience reaches the requisite amount\comma{} you may choose to rest and muse on what you have learned fromyour experiences\comma{} triggering the level\minus{}up process. You may only do this if not facing life\minus{}threating injury \minus{}\minus{} levelling up cannot heal a broken leg!

%~ When you level up\comma{} you make the following changes to your character:

%~ \begin{itemize}[itemsep=0em]
	%~ \item Increase character level by 1\comma{} and reset Exp counter to zero (you may carry any excess Exp over)
	%~ \item Increase Archetype level by one {\bf OR} choose a new archetype (see multiclassing rules on page \pageref{S:Multiclassing}). Add any new Features you gain at this point.
	%~ \item You may choose one of the following:
	%~ \begin{itemize}[itemsep=0em]
		%~ \item Increase an attribute by 2\comma{} or two attributes by 1
		%~ \item Choose a new Skill\comma{} if you meet the minimum prerequisites
	%~ \end{itemize}
	%~ \item Calculate new HP and FP ceilings
	%~ \item Reset HP and FP to maximum
	%~ \item Reset spell\minus{}learned counter
%~ \end{itemize}

%~ \subsection{Other Changes}

%~ The GM may also decide that\comma{} during the normal course of play\comma{} you have done something that warrants a permanent bonus or penalty \minus{}\minus{} be it something you have learned from extensive practice\comma{} or a gift from some higher being \minus{}\minus{} the GM will grant you a bonus to your Attributes or Proficiencies. This will probably most commonly be used to penalise players for immoral actions \minus{}\minus{} by increasing their EVL level.

%~ \section{Skills}\label{S:Skills}

%~ Skills are learned abilities that your character picks up along the way. They range from the mechanical (i.e. {\it Ambidextrous}, which allows your character to use both hands without penalty, and to dual-wield weapons with ease), to the magical (i.e. {\it Animagus}, which allows you to take on the form of an animal).


%~ They can be learned either by levelling up or given as gifts by external devices. Some skills may be taken multiple times, which increases the bonus provided by the skills. 

%~ The full list of skills is found on page \pageref{S:SkillList}.

%~ \subsection{Prerequisites}

%~ Some skills list a minimum ability score\comma{} or other threshold that your character must posses before they take that skill. If you do not meet the threshold\comma{} you cannot take the skill\comma{} unless you are provided it by external means.


