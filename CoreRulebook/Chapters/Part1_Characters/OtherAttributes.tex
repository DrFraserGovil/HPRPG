\chapter{Other Attributes}




\section{Health}\index{Health}\index{Harm}\label{S:Health}

The \key{Health} of a character is an abstracted representation of their physical wellbeing. As you suffer \key{harm} and take \key{damage} (discussed more on page \pageref{S:Damage}), you lose health and begin to suffer negative consequences. 

These consequences increase in severity as you take more and more harm, until eventually you are left in a \imp{Critical Condition}, and either fall unconscious or approach death. 

Most characters start with 8 levels of health, starting at \key{Unharmed}, progressing through \key{Injured} eventually to \key{Critical Condition}. The higher levels of harm are associated with a decrease in your ability to function on both a mental and a physical level. This is enumerated by a dice-penalty to all checks, as enumerated in the table below:

\dotTable{\key{Unharmed}}{\key{Sore}}{\key{Bruised} - take a 1d penalty}{\key{Hurt} - take a 2d penalty}{\key{Injured} - take a 3d penalty}{\key{Wounded} - take a 4d penalty}{\key{Mangled} - take a 5d penalty}{\key{Critical Condition} - you are unconscious, in a coma, or nearing death}{{Health Rating}}

The character sheet provided on page \pageref{S:CharacterSheet} contains seven diamonds to mark off as you take harm. 


\subsection{Increasing or Decreasing Maximum Health}

The character sheet also possesses space for additional health diamonds to be added. This is because magical effects can increase the vitality of a being, allowing it to soak up more damage before becoming impaired. Equally, the physical status of a character might require that they have additional health either from physical conditioning and training (see page \pageref{S:Progression}), or from their innate resilience - a \imp{Half-Giant} can take a lot more of a beating than a normal human, for example. 

When an additional \imp{Health} point is gained, it is added to the {\bf top} of the health-stack. These additional diamonds are therefore marked off first. If the effect granting addtional health is only temporary, when the effect is removed, you simply remove any additional diamonds and leave the rest of the \imp{Health} untouched. 

For example, if an unharmed character (health = \emptyCape) gains an additional health point by drinking a potion, their new health is $\mdwhtdiamond$ + \emptyCape, with the additional dot at the front. 

If the character then takes level three harm, they fill in the first three diamonds, giving a health rating of $\mdblkdiamond$ + \twoCape. When the potion wears off, the additional diamond is removed, leaving the character on a health rating of \twoCape. Drinking the health potion therefore allows the character to reduce their level of harm by one level\footnote{Of course, you might wonder why this is different to just healing by 1 point after the damage was dealt. Note that if a character gained an additional 2 health dots, and then took level 7 \imp{harm}, this would leave them awake and \imp{Wounded} when the effect wore off - however if they had not increased their health, they would have been knocked unconscious by taking the \imp{critical condition} status, at which point simple healing does not help.}.

If a character has their maximum health level reduced - say by a crippling \imp{blood curse} imposed upon them by a Dark Wizard, you remove the dots from the top (excluding \imp{Critical Condition}). A character suffering from a 2-point health drain would jump straight from \imp{Injured} to \imp{Critical Condition}. If this effect is removed, their health dots return, and they may take additional damage as normal.

\section{Fortitude}\index{Fortitude}

The \key{Fortitude} of a character is a measure of their psychological fitness, their ability to push themselves and defy the odds as things go wrong around them. A character with lots of \imp{Fortitude} remaining is viacious, awake and ready to take on the world, whilst when \imp{Fortitude} reserves are running low, characters have run themselves ragged and have very little left to give before they collapse into exhaustion. 

Fortitude can be expended willingly by a character in order to gain one of the following effects:
\begin{enumerate}
	\boldItem{Increase Odds}{Gain one automatic success on a check (equal to a roll of 12). This increases the odds of success in the check and, in most cases, rules out the possibility of rolling a \imp{Catastrophic Failure}. You must expend the \imp{Fortitude} point before the check is made.}
	\boldItem{Negate Catastrophe}{Turn a \imp{Catastrophic Failure} in a normal, more benign, failure. You may use this abiliy after the dice are rolled.}
	\boldItem{Extra Action}{Gain an additional action in a combat, jumping back into the initiative order.}
	\boldItem{Magic Surge}{Cast a magic spell one level higher than your current \imp{affinity} would normall allow. You must expend the \imp{Fortitude} point before the check is made.}
	\boldItem{Enhanced Endurance}{Negate the effects of your current \imp{Harm} for one turn. You must expend the \imp{Fortitude} point before the check is made.}
\end{enumerate}


Every time you expend \imp{Fortitude}, shade in one of the 7 \imp{Fortitude} dots on your character sheet. When all 7 are shaded in, you cannot expend any further Fortitude points. Additional levels of Fortitude can be added and removed following the same rules as those for \imp{Health} discussed above.

\subsection{Losing Fortitude}

Other effects can also decrease your fortitude - the \imp{Torture spell} ({\it crucio}), for example, breaks the mind of an opponent without touching their physical form, thereby reducing their \imp{Fortitude}. 

Equally, if one acquires the \imp{Exhausted} status by failing to look after one's physical and mental health, you suffer penalties to your \imp{Fortitude}. 

\subsection{Regaining Fortitude}\index{Fortitude!Regeneration}

Expending fortitude causes one become tired, irritable and run down -- in order for a character recover their normal state, they must relax, rest or perform an activity which soothes their soul. 

Every character has a number of actions that they can take to increase their fortitude. these actions are known as \imp{Nourishment Actions}.

\begin{itemize}
	\boldItem{Proper Rest}{A proper night's rest of at least 7 hours sleep, accompanied by a hot meal is enough to restore 2 points of \imp{Fortitude}}\index{Sleep}
	\boldItem{Personality Indulgences}{Every personality grants an action or a set of actions which you can indulge in. When the GM rules that you have met the conditions for this action, you may restore 1 \imp{Fortitude Point}. You can do this up to twice per day.}
\end{itemize}

As the game progresses, you may also encounter some narrative points where it is clear that you are undertaking an action for no other reason than to relax and collect your thoughts - perhaps after emerging from the giant's lair, covered in filth, grime and other unspeakable fluids, you go for a swim in a nearby lake to clean the muck off you. Going out of the way to roleplay narratively interesting actions such as this can be rewarded with a fortitude point. 

As you roleplay and inhabit your character, you may discover additional facets of your character's personality - what makes them tick, and how they would react and behave within the world you are exploring. If you so desire, you can work with the GM to expand and refine your \imp{Personality Indulgences} to better reflect the character that you have built up. 


\section{Heroism \& Villainy}\index{Heroism}\index{Villainy}

The universe is a potent thing, and whilst it may seem at a first glance to be an infinite expanse of formless chaos, there is actually far more to it than that. 

The twisting strands of fate interweave through space and time, forming a web of possibility, purpose and cauasilty. Whilst most people find themselves passively pushed through life by the ebbing and flowing of the tides of fate, a few rare people have the ability to grab fate by the scruff of the neck, and take matters into their own hands. 

As \imp{Player Characters}, you are all such people, and so you must learn that the most powerful laws which govern reality are not magical or physical in nature -- they are {\it narrative}. 

The universe {\it loves} a good story, which is why (in the words of the late muggle author Terry Pratchett): {\it 1-in-a-million odds happen 9 times out of 10}, and why a heavily armed group of people should {\it never} fight against a single old man - especially if he is unarmed, and {\it especially} if he is smiling. 

By positioning yourself in the way which turns yourself into an archetypal valiant hero, or the despicable villain to be defeated at all costs, you may find that reality itself caves to your will just a little bit easier than you might have expected. 

As you perform heroic or villainous acts, the universe begins to bend around your will further and further, allowing you to achieve even more insane actions, and execute ridiculous plans - as long as you are continuing to further the cause of your narrative. 

This ability to bend the strands of fate is determined by the dual statistic of \key{ Heroism \& Villainy}. These two statistics are kept track on a single 7-dot line called the \key{Moral Track}, with \imp{Heroism} being filled in from the left, and \imp{Villainy} being filled in from the right. 

\subsection{Gaining Heroism \& Villainy}

Whenever you perform an act which is spectacular, ridiculous, and motiviated by either pure \imp{Heroism} or \imp{Villainy}, your \imp{GM} may reward (or indeed, punish) you by increasing your rating in either of these fields. 

Generally speaking, there has to be a level of escalation with each level granted - you must be attempting to do {\it more} than you have ever achieved, pushing the boundaries of what is possible and plausible in order to grab hold of more of the fabric of reality. 

\subsection{Utilising Heroism \& Villainy}

After have earned a point in either \imp{Heroism} or \imp{Villainy}, you may attempt to utilise your newfound sway over the narrative course of reality to improve your odds of achieving an action. 

Whenever you attempt in action, you may ask the \imp{GM} if it qualifies as suitably noble (or indeed, depraved and evil enough). If they agree that, should it occur on film, this action would be accompanied by a rising crescendo in the score, a heartstopping moment within a scene, you may add your current score of \imp{Heroism} or \imp{Villainy} (as appropriate) into your dice pool. 

This gives you a greater chance of succeeding in the action, and whatever the result of the roll, the outcome will be suitably cinematic. 

\subsubsection{Limitations}

Actions which dip into this bending and warping of fate should be done only rarely. You are, after all, attempting a monumental act of hubris by attempting to wield this power - doing so too often could be catastrophic. 

If the GM feels that you are attempting to abuse this power too often without it adding anything interesting the the game, they may invoke \key{Narrative Slip}. When a character undergoes \imp{Narrative Slip}, the fabric of reality lashes out at the thing which has been distorting it. When the character next attempts to utilise their ability to warp reality, they will find that they are no longer a `main character' in reality, and the action will fail. 

In other words, though a character may be attempting to be the \imp{Lone Hero to Defeat The Villain}, they accidentally cast themself as the \imp{Guy Who Was Mortally Wounded So The Villain Could Make A Point}.

When this happens, the character loses half of the dots present on their \imp{Moral Track} (at the GM's discretion) representing their demotion from the cast of the cosmic narrative. Over time, they may rebuild their abilities, but \imp{Narrative Slip} is something to be avoided at all costs.

From an out-of-world perspective, this mechanic is present to allow you to have a better chance at doing truly insane and awesome actions - because those are the most {\it fun}. \imp{Narrative Slip} is a sledgehammer, an approach of last resort if a player is attempting to use this ability to the point that the game is becoming less fun. A \imp{GM} and the players should always try to resolve this out-of-character before it becomes an issue. 

If the GM wishes to make a point (or just thinks it would be funny), they may invoke the first part of \imp{Narrative slip} (the comical fail of a supposedly legendary action), without imposing the permanent penality of losing points along the \imp{Moral Track}.

\subsection{Clashing Morals}\index{Moral Crisis}\imp{Moral Track}

Sentient beings are hugely complex social creatures, and Morality is rarely clear cut in all cases, so it is not surprising that many characters can have both \imp{Heroism} and \imp{Villainy} points. At a certain point however, these conflicting roles causes them to enter into a \key{Moral Crisis}. 

A \imp{Moral Crisis} occurs when your two scores `touch' each other on the combined track, leaving no unfilled dots. When this occurs, you must go on an introspective journey and lose a dot in either your \imp{Heroism} or \imp{Villainy}, such that the two tracks are not touching. Usually you would lose the dot associated opposite to the one you just raised, though you may be able to roleplay a scenario through with your \imp{GM} where the opposite occurs. 

For example, Tara is hunting down a pack of witchhunters - a nominally valiant act that has earned her 5 \imp{Hero} points. However, recently she has become especially vicious and uncompromising in her quest - after an unfortunate amount of collateral damage killed an innocent bystander, her GM awarded her a point of \imp{Villainy}. Her \imp{Moral Track} would therefore be:
$$ \text{Hero}~~\mdblkdiamond\mdblkdiamond\mdblkdiamond\mdblkdiamond\mdblkdiamond\mdwhtdiamond\mdblkdiamond~~\text{Villain}$$

A few days later, one of the witchhunters is using a group of muggles as a human shield - leading Tara to again kill a large number of them as she quested for vengeance. The \imp{GM} tells Tara that she has increased her \imp{Villainy} score again, leading to the ambiguous \imp{Moral Track}:
$$ \text{Hero}~~\mdblkdiamond\mdblkdiamond\mdblkdiamond\mdblkdiamond\mdblkdiamond\mdblkdiamond\mdblkdiamond~~\text{Villain}$$
This induces a \imp{Moral Crisis}, as Tara can no longer reconcile her heroic urges with the evil acts that she has comitted: she can no longer tell good from evil. The Players and the \imp{GM} are encouraged to roleplay this scenario in whatever way is most interesting for them - perhaps Tara must seek out a mentor for advice, as she feels lost and unsure of herself. 

As it was an act of \imp{Villainy} that introduced this moral conflict, Tara will most likely realise the evil of her actions and so suffer a drop in her \imp{Heroism} as a result, resolving the \imp{Moral Track} to:
$$ \text{Hero}~~\mdblkdiamond\mdblkdiamond\mdblkdiamond\mdblkdiamond\mdwhtdiamond\mdblkdiamond\mdblkdiamond~~\text{Villain}$$

Unless you are a \imp{True Hero}, or an \imp{Irredemable Evil} (7 dots in either \imp{Heroism} or \imp{Villainy} respectively), there must always be at least one unfilled dot on your \imp{Moral Track}. 
