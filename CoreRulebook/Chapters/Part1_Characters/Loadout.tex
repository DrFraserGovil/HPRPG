\chapter{Initial Loadout}

In addition to determining your initial statistics, you must also determine the initial set of items and \imp{Spells} that you possess at the beginning of the game. 

This of course varies somewhat - if you wish to start a game from year 1 at \imp{Hogwarts}, this will vary hugely from if you start off with adult characters who already have careers. Notes are given on ideas of how to modify these ideas somewhat to accomodate different initial stages of progression. 

\section{Items}

Most characters will start off with only a limited item loadout, consisting of the basics:

\begin{itemize}
	\item \imp{A wand} (see page \pageref{S:Wandchoosing})
	\item \imp{A set of wizards robes \& pointed hat}
	\item \imp{A set of muggle clothing}
	\item \imp{Two non-spell books} (see \pageref{S:Books})
	\item \imp{A tool} (see \pageref{S:Tools})
	\item \imp{3 galleons} (usually with 1 in pocket, 2 in the \imp{vault}, see \pageref{S:Money})
\end{itemize}

Depending on the `entry point' of a character, the \imp{GM} may instead elect to let the first part of gameplay be the \imp{Diagon Alley} shopping trip as they acquire these items and explore the beginnings of the magical world. Approximately \galleon{5} can be spent on the initial set of items if this is option is chosen (this includes the initial 3 \imp{galleons}.sssssss

If you wish to have any specific additional non-magical items at the beginning of the game, you may discuss this with your \imp{GM}, though you should be wary that this is intended only for `flavour' purposes, not for gaining a mechanical advantage from the start.

\subsubsection{Advanced Start}

If you are taking part in a campaign that uses characters who are not only just starting out in life, you may choose some additional items, subject to your \imp{GM}'s approval. If you have thought of a reasonable background for your character, you may assemble a set of equipment based on their profession or experiences. Some ideas are presented below:
\begin{itemize}
	\item \imp{A weapon} (see page \pageref{S:Weapons})
	\item \imp{Some armour} (see page \pageref{S:Armour})
	\item \imp{A magical item or two} (of \imp{Unusual} rarity or less)
	\item \imp{Adventuring gear} (i.e. rope, rations, potions etc.)
\end{itemize}

As a rough guideline, you may have up to \galleon{20} to spend on your initial items, or save in your \imp{Vault}



\section{Spells}

As with the \imp{items}, some \imp{GM}s may choose to roleplay your first few days at \imp{Hogwarts} explicitly, in which case it is entirely reasonable that you enter into the game with zero spells memorised. However, if you do not very rapdily learn up to three spells, you may be within your rights to complain, as the standard recommendation is that players enter the game with three spells memorised from the following list:

\newcommand\impIt[2]{\keyItem{#1}{\imp{#2}}}
\begin{itemize}
	\impIt{Alteration}{Refine, Transmute}
	\impIt{Bewtichment}{Charm, Distract, Mirage}
	\impIt{Cerebral}{Communicate, Sense}
	\impIt{Conjuration}{Bind, Forge}
	\impIt{Curse}{Disable, Disarm}
	\impIt{Elemental}{Burn, Freeze, Gust, Illuminate, Soak}
	\impIt{Hermetics}{Heal, Restore}
	\impIt{Hex}{Force, Jinx}
	\impIt{Kinesis}{Move, Repair}
	\impIt{Temporal}{Identify}
	\impIt{Warding}{Abjure, Shield, Trap}
\end{itemize}

\subsubsection{Advanced Start}

If you are starting from a more advanced level, you may instead choose up to 7 spells to memorise from the full spell list.  
