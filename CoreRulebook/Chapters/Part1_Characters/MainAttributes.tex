%% DEFINITIONS

\newcommand\abilityRow[2]
{
	\imp{#1} & \parbox[t]{6.8 cm}{\raggedright #2} \\
}

\newcommand\abilityTable[1]
{
	\small
	\begin{center}
		\begin{rndtable}{r l}
		\bf Ability	& \bf Description \\
		#1
		\end{rndtable}
	\end{center}
}
\def\cc{\cellcolor{\tablecolorhead}\bf}
\newcommand\fourRow[4]{{\cc \bf \key{#1}}	&	\imp{#2}	&	\imp{#3}	&	\imp{#4} \\}


%% BEGIN CHAPTER

\chapter{Capabilities}\label{C:Aspects}

A character's ability to function in the world is defined by their capabilities across a wide number of areas. These capabilities are split into 3 categories: \key{Aspects}, \key{Abilities} and \key{Affinities}. 

This section deals with the first two of these. \key{Affinities} are discussed in detail in the Magic section starting on page \pageref{C:Magic}.

\section{Capability Dots}

Each one of the 9 Aspects and myriad Abilities and Affinities represents a way for a character to interact with the world. How {\it well} they can do so depends on their competence in that field. 

To this end, each and every one of the Aspects and Affinities is represented by between 0 and 7 `dots'. Each dot represents a 12-sided dice that can be rolled when that capability is used. 

Zero dots means that you are absolutely useless in the field, totally untrained and with no idea what you are doing. Five dots, on the other hand, represents the peak of human achievement: perhaps a dozen people in the entire world have 7 dots in a given area. Almost everyone finds themselves somewhere in the region of 1-4. 

A character can never gain more than 5 dots as part of their normal life, however magic is a crazy and fickle thing: once in a blue moon you may temporarily find yourself with more than 7 dice allocated to a given capability as the result of a spell. This is a rare and wondrous event. Maybe you should sing a song. 

You have already been granted dots in certain fields by your species and your Archetype: you also get to allocate a larger number of additional points, as described in this chapter. 

Do not fret if there are gaps in your abilities, as your character will continue to grow and improve as the game progresses. 

\section{Aspects}

Aspects are the fundamental characteristics of a character: every action that is performed finds one of the 9 Aspects at its root.

\subsection{Aspects Classification}

There are nine core Aspects: Fitness, Precision, Vitality, Charm, Deception, Insight, Intelligence, Willpower and Perception. Each of these aspects is classified in two ways, once by the Aspect's \key{Type} and then by the Aspect's \key{Method}. 

~

\vbox{
	The Type determines which of three key attributes of a character is being used:
	\begin{itemize}[itemsep = 0em]
		\keyItem{Physical}{The capacity to use your body to interact with the material world.}
		\keyItem{Social}{The capacity to interact and understand others.}
		\keyItem{Mental}{The capacity to use your mind and process information.}
	\end{itemize}
}

\vbox{
	The Method determines how that ability is used:
	\begin{itemize}[itemsep = 0em]
		\keyItem{Project}{The capacity to use the {Type} to its maximum possible level, pushing and striving for great effects.}
		\keyItem{Manipulate}{The capacity to use the {Type} in a careful and refined fashion, to maintain control of the situation.}
		\keyItem{Absorb}{The capacity to resist or take in the {Type}.}
	\end{itemize}
}

The 9 aspects therefore lie on a 3x3 grid:

\begin{center}
	\begin{rndtable}{r c c c}
		\fourRow{~}{\key{Physical}}{\key{ Social}}{\key{ Mental}}
		\fourRow{Project}{Fitness}{Charm}{Intelligence}
		\fourRow{Manipulate}{Precision}{Deception}{Willpower}
		\fourRow{Absorb}{Vitality}{Insight}{Perception}
	\end{rndtable}
\end{center}

A full description of the Attributes, and the situations in which they are used, can be found on page \ref{S:Proficiencies}.

\subsection{Assigning Aspects}

Every character starts off with a baseline of a 1-dot rating in each of the 9 Aspects, and you must then decide which areas they are skilled in. When creating a character, you need to have a clear idea of their role in the story: their personality and basic abilities. This will stem naturally from their backstory and origins, so you should spend time thinking about who your character is, before diving into the numbers. 

Upon making this decision, you then rank the 3 Aspect Types (Physical, Social \& Mental) in order of importance to the character. Into the most Important Type, you may distribute 7 dots. The secondary type gets 5 dots to allocate, and the least important type gets only three dots. 

\begin{center}
	\small
	\begin{rndtable}{c c}
	Ranking &  Allocated Dots \\
	Highest & 7
	\\
	Middle 	&	5
	\\
	Lowest & 3
	\end{rndtable}
\end{center}

During this stage, you may not allocate more than 3 additional dots to a single Aspect, though you may end up with more than this limiting number of dots in a field if you benefit from some a racial or Archetype bonus. 



\section{Abilities}

Although your \key{Aspects} inform the broad approach used to complete an action, it is your \key{Abilities} which determine exactly how you will go about doing so, narrowing down the specific kind of skills you will be using. 

Each of the 30 aspects is classified as either being \key{Innate}, \key{Practical} or \key{Learned}. These differ in how the skills are acquired and used, with the primary mechanical difference being how a `zero-dot' rating is treated in each field. 

For an Innate ability, having no experience is no barrier to attempting an action as the actions represent natural extensions of your Aspects. Practical abilities, however, you may still attempt an action without training, but the action is much more difficult as you lack any proper training into how to undertake the action. For an action relying on a knowledge ability, having no training makes using the action impossible in all but the rarest of circumstances.

The sections below elaborate on each of these skills, along with a brief summary of each ability. A full description of each ability can be found in Part \ref{C:Actions}.


\subsection{Innate}

An \key{Innate} ability is one which represents some aspect of a character's intrinsic social, mental or physical abilities. Though many people are born being particularly good in one or more of these areas (hence `innate'), they are still areas that can be worked on and improved. 


\abilityTable
{
	\abilityRow{Alertness}{Rapidly detect and identify threats and miniscule clues.}
	\abilityRow{Bravery}{Defy worry and terror and stare down foes much stronger than yourself}
	\abilityRow{Conviction}{Understand your own reality, and the moral and intellectual positions you hold dear.}
	\abilityRow{Eloquence}{Express yourself appropriately for the situation at hand}
	\abilityRow{Intimidation}{Inflict terror into the hearts of your foes, assert authority and command people to follow your directions}
	\abilityRow{Kindness}{Show your gentle side, making others feel loved and safe.}
	\abilityRow{Kinship}{Befriend and control animals}
	\abilityRow{Logic}{Solve puzzles, spot clever solutions and use reason to solve your problems.}
	\abilityRow{Speed}{Get from A to B as quickly as possible}
	\abilityRow{Strength}{Exert physical force to lift and move heavy objects and beings}
}


\subsection{Practical}

A \key{Practical} ability is one which you have learned through hands-on experience, laborious training and practice. Though they rely on an Aspect to direct the task, they are separate from your intrinsic abbilities and often requires some special tool or equipment to complete. 

Generally, anyone can attempt to perform a practical action, even if they have no training (0 dots), relying instead on their instinctive Aspects to get a lucky break. However, the DV of the associated action is increased by 2, to reflect the complete lack of training. 
 
\abilityTable
{
	\abilityRow{Acrobatics}{Leap, flip, tumble and contort yourself}
	\abilityRow{Brawl}{Punch, kick, bite and otherwise wrestle your way to dominance}
	\abilityRow{Covert}{Use stealth and slight of hand to move and act without being spotted}
	\abilityRow{Craft}{Tinker with devices, form armour and produce masterful artwork}
	\abilityRow{Imbue}{Perform delicate acts of magical creation, mixing potions or enchanting items}
	\abilityRow{Marksmanship}{Hit your target, either throwing objects, or using firearms}
	\abilityRow{Performance}{Embody another character, either as a disuise or for theatrical purposes}
	\abilityRow{Pilot}{Effectively handle and drive vehicles such as cars and broomsticks.}
	\abilityRow{Skirmish}{Use blades, axes and other close-combat weapons effectively} 
	\abilityRow{Survival}{Survive in the hostile environment of the wild}
}

\subsection{Knowledge}

A \key{Knowledge} ability is one which has been learned through intensive study, attending classes and days spent in the library. A knowledge ability can be used either to recall information, or to weave that information into another action. 

It is generally impossible to use a knowledge action in which a character has zero experience: if the knowledge is not there, it cannot be used. There may be cases where you can appeal to some lower-level knowledge to try an Aspect-Only roll, but this is entirely at the discretion of the GM.
	
\abilityTable
{
	\abilityRow{Arcane}{Understand the effects and abilities of magic and other supernatural phenomena}
	\abilityRow{General}{Small amounts of aggregate knowledge on a variety of topics}
	\abilityRow{History}{Recall prominant names, places and events throughout history}
	\abilityRow{Investigation}{Meta-knowledge: you know how to learn and uncover new information, by closely inspecting both books and the physical world}
	\abilityRow{Medicine}{Understand the functioning of the humanoid body, and how to heal its ailments.}
	\abilityRow{Muggle}{Understand the Muggle world, and know what's going on in the world of media, TV and celebrity}
	\abilityRow{Nature}{Familiarity with the behaviour and life cycle of plants and beasts, both magical and mundane.}
	\abilityRow{Science}{Understand the mundane science behind the natural world: physics, chemistry, biology and beyond} 
	\abilityRow{Technology}{Comprehension of the goings-on in computers and other technologcal marvels}
	\abilityRow{World}{Knowledge of the geography of the Earth, and the people in it on both a macro and a micro level}
} 


\subsection{Additional Abilities}

In addition to the 3$\times$10 standard ability array, characters can gain access to additional actions and abilities, personalised to them. 

This will most commonly arise from the choice of Archetype and Race: each Archetype grants a number of ``Special Actions'', which are manifest through three additional Abilities, and some Races also gain addiitonal abilities. These actions and the associated abilities are discussed in more detail in the relevant Archetypes and Races sections. 


\subsubsection{Custom Abilities}
You may also work with the GM if you feel that a given character should have additional abilities because of their background and previous experiences. This is encouraged only within the limits that it keeps the game fun and interesting and is sufficiently differentiated from the existing abilities.

It would be perfectly acceptable to give a character from a circus a special ``Juggle" or ``Tightrope" ability, as this opens up alternative and interesting actions for them to take without drastically altering the balance of the game. A super-dedicated warrior asking for an insta-kill move, or a scholar claiming to have access to an infinite library, however, would drastically alter the flow of the game without making it more fun and interesting. 

You are not required to come up with an additional ability, but if you have a fun idea, you should ask your GM if this is OK. As always, they have a veto, but may supply an alternative idea which works better in their world. 

\subsection{Assigning Abilities}

Similarly to the assignment of the Aspects, you must first decide which form of Ability your character will have spent most of their time becoming proficient in: ranking the three choices of Innate, Practical and Knowledge in order of importance. To the first of these you may allocate 12 dots, 8 dots into the second most important, and 4 into your lowest rated class:


\begin{center}
	\small
	\begin{rndtable}{c c}
	Ranking &  Allocated Dots \\
	Highest & 12
	\\
	Middle 	&	8
	\\
	Lowest & 4
	\end{rndtable}
\end{center}


During this stage, you may not allocate more than 3 dots to a single Ability, though you may end up with more than this limiting number of dots in a field if you benefit from some a racial or Archetype bonus. 

\section{Health}


\section{Fortitude}

\section{Heroism \& Villainy}
