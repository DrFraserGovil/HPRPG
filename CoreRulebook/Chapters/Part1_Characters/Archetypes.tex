
\chapter{Character Archetype}\label{C:Archetype}

Whilst your character is a unique individual, an adventuring soul destined for greatness, most questers find themslves falling into one of many {\it archetypes} -- are they the headstrong hero who needs to learn humilty? The academic who's quest for knowledge has led to unforseen consequences, or the plucky underdog trying to quit their life of crime? 

The archetype (also known as the {\it class}) of your character is a way of formalising these character types. The role of your character is more than simply the job they perform, it is the prism through which they see the world -- it guides their very essence, how they perceieve themselves and others. The Archetype of a character therefore has a drastic impact on the roleplaying aspect of the game.
   
As well as informing what kind of person your character is, the Archetype serves to provide them with some unique skills ({\it Features}) that they acquire as they progress through the archetype. Each time they level up, their archetype abilities increase in power. Your choice of path also provides you with information about the character's starting equipment and any proficiencies they may already have. 

Most (though not all) Archetypes have multiple paths that can be followed, to further differentiate between characters, these are known as `subtypes', or `subclasses'. The Archetype descriptions provide you with instructions on the available choices and when you must make such a choice. 

The commonly used Archetypes are as follows:

\newcommand\archEntry[3]
{
	{\bf #1} &	\parbox[t]{2.5 cm}{\raggedright #2}	& \parbox[t]{4.5 cm}{\raggedright#3}\\
}

{
\small
\begin{center}
	\begin{rndtable}{l l l}
		\bf Archetype	&	\bf Subtypes	&	\bf Description
		\\
		\archEntry{Acolyte}{-}{A follower of a higher power, gaining additional powers through their service.}
		\archEntry{Artificer}{Potioneer\\Enchanter\\Mechanist}{A person trained in the delicate arts of creating and producing new items.}
		\archEntry{Auror}{-}{A dedicated warrior-investigator, who seeks out evil and brings it to justice.} 
		\archEntry{Berserker}{Noble Savage\\Primal Warrior\\Mystic Champion}{A warrior who relies on a burning rage to negate harm and bolster their physical or magical power.}
		\archEntry{Druid}{Asteria \\Dryad \\Nymph \\Satyr}{A person dedicated to some primal aspect of nature, earning nature-related powers and gifts.}
		\archEntry{Outlaw}{Thief\\Assassin}{Someone who works outside the law, a trickster, not bound by conventional ideas of how to behave.}
		\archEntry{Scholar}{-}{Someone dedicated to knowledge, delving deep into the inner mysteries of the universe.}
		\archEntry{Warrior}{Way of the Blade\\Way of the Shield\\Way of the Wand}{A powerful fighter, trained in all forms of combat. They excel in kicking ass, and taking names.}
		\archEntry{Zealot}{-}{Someone who has dedicated themself to some personal quest. Their single-minded devotion gives them unprecedented control over their own bodies.}
		
	\end{rndtable}
\end{center}
\normalsize
}
\newpage 
\section{Students}



Characters who are students, however, are much less likely to know what their roll in life is yet. They are much more likely to be defined and shaped by their school environment, so there are four special Archetypes, dedicated to the 4 Houses at Hogwarts. Note that these 4 Archetypes only have 3 levels of features, so these classes are intended only to be used as introductory Archetypes, until the students are able to decide for themselves what they wish to dedicate their lives to.  

Only human wizards (muggleborns, halfbloods and purebloods) may take these Archetypes, as Hogwarts does not (yet) accept non-human students.

\renewcommand\archEntry[2]
{
{\bf #1} &	\parbox[t]{6.5 cm}{\raggedright#2}\\
}
\begin{center}
	\begin{rndtable}{l l}
		\hline
		\bf Archetype	&	\bf Description
		\\
		\archEntry{Gryffindor}{The house of the noble and the brave. Prize a high \attSpr{} score.}
		\archEntry{Hufflepuff}{The house of the kind and the dedicated. Prize a high \attPer{} score.}
		\archEntry{Ravenclaw}{The house of the smart and the wise. Prize a high \attInt{} score.}
		\archEntry{Slytherin}{The house of the charming and ambitious. Prize a high \attChr{} score.}
	\end{rndtable} 
\end{center}

\section{Detailed Information}

Each Archetype has a detailed entry in the Archetype List, found in the appendices, starting on page \pageref{A:ArchetypeList}. When playing a character, you should have the Archetype Entry close to hand so that you may refer to it. You don't want to forget about the abilities that it provides!

\section{Initial Archetype}

When starting the game, you should choose your initial Archetype. For games that start out with the PCs as students at Hogwarts, this will probably involve picking a house. You may choose a house by determining which of your \attSprShort{}, \attPerShort{}, \attIntShort{} or \attChrShort{} scores are highest and allowing this to `sort' you, or you may simply ask the Sorting Hat to place you in a given house. 

Your initial class will give you your `starting statistics', such as your Initial HP and FP. This is detailed in each archetype entry. In addition, your initial Archetype determines what items and spells you start the game knowing.  

\subsection*{Starting Equipment}

Each archetype details a basic set of equipment that you can assume your character has in their possession at the start of the game. 

For Hogwarts students, this is typically standard student garb, a wand and a handful of basic supplies. For `professional' Archetypes, it represents the standard tools of the trade. Your GM might also allow you to pick up an extra item, or swap out one of the prescribed ones for one which better fits into your character's backstory. 

The most important item is your magical wand (if you are a species that can use one!). To determine the composition of your wand, roll on the Wand Table, as shown on page \pageref{S:Wands}. 


\subsection*{Starting Spells}\label{S:BasicSpells}

At the start of the game, all characters are given a number of spells that they have `memorised', and can begin to use immediately. This is typically determined by choosing a number of spells from the `Starting Spells' table, presented below:

\begin{center}
	\small
	\begin{rndtable}{c m{\xS cm} p{\wS cm}}
		\bf School	&	\bf Discipline	&	\bf Available Spells
		\\
		\school{Charms}{Elemental}{Create Fire, Create Water, Gust, Contact Shock}{Kinesis}{Levitation, Mage Hands, Mark Surface}
		\\
		\school{Divination}{Telepathy}{Assist Ally, Induce Anxiety}{Temporal}{Identify Object, Receive Omen}
		\\
		\school{Illusion}{Bewitchment}{Glamour, Aura of Kindness, Throw Voice}{Psionics}{Piercing Wail, False Friendship}
		\\
		\school{Malediction}{Hexes}{Knockback Jinx, Rainbow Sparks, Sting}{Curses}{Confound, Curse of the Bogies}
		\\ 
		\school{Recuperation}{Healing}{Minor Healing, Boost Health}{Warding}{Force Shield, Silent Step}
		\\
		\school{Transfiguration}{Alteration}{Change Colour, Transmutation}{Conjuration}{Prank, Conjure Flowers, Silver Shield}
		\\
		\school{Dark Arts}{Necromancy}{-}{Occultism}{Blood Pacts}
	\end{rndtable}
	\normalsize
\end{center}

\section{Multiclassing}\label{S:Multiclassing}
Although it is perfectly possible to progress with only one archetype, sometimes you might want to dip your toes into another set of abilities. This is called {\it multiclassing}. Whenever you next level up, you may decide to take a new Archetype. Rather than increasing your level in your current Archetype, you may instead choose to become a Level 1 in a new class. In an ideal world, this should only be done because of a profound change in either the character, or their circumstances. 

For example, a Level 6 Warrior might decide, after their ordeal at the hands of an evil cult, to dedicate their life to eradicating all cults everywhere. This all consuming quest means that they decide to swear fealty to a powerful being and become an Acolyte. Next time the character progresses, she becomes a Level 6 Warrior/Level 1 Acolyte. They may decide to focus on their Acolyte abilities until they are a level 6/5 Warrior/Acolyte -- at which point they may take another level in Warrior. You do not necessarily abandon your original archtype. 

The sum of your archetypes should simply be the total character level (and it is this character level that determines when you next level up). 

Your abilities in a given archetype are based on your level {\it in that archetype}, not your total character level. Our 6/5 Warrior/Acolyte is a level 11 character, but only has access to Level 6 Fighter features, and so on. Abilities such as your available spells and your Expertise bonus are determined by your total character level. 

If you are playing a student character, you may not multiclass into a different House. Equally, a non-student may not multiclass into a House. 

You may multiclass as many times as you like -- though you will find yourself with considerably fewer abilities than a character who has stuck with a single archetype.


\subsection*{Multiclass Equipment}

Note that the equipment detailed in each archetype is the {\it starting} equipment. If you multiclass, however, you do not automatically acquire these items, except where it makes narrative sense. 

\subsection{Multiclass Proficiencies}

Each Archetype provides an initial set of proficiencies that you gain when you start down that path, both attribute proficiencies (i.e. Persuasion), and weapon proficiencies. 

When you multiclass, however, you do not gain all of these abilities. You take all proficiencies which you are directly told to take, however when a choice is presented, you must take fewer than directed. For every subsequent multiclass, you take 2-fewer proficiencies from the pool than directed (min 0). 
