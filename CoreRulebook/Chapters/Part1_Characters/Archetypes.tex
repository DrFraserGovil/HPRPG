
\chapter{Character {Archetype}}\label{C:Archetype}\index{Archetype}\index{Class|see{Archetype}}

Whilst your character is a unique individual, an adventuring soul destined for greatness, most questers find themslves falling into one of \key{Archetypes} which helps define their abilities and goals-- are they the academic who's quest for knowledge has led to unforseen consequences, or the plucky underdog trying to quit their life of crime? 

The \imp{Archetype} (also known as the {\it class}) of your character is a way of formalising these character types. The role of your character is more than simply the job they perform, it is the prism through which they see the world. Along with their personality, it guides their very essence, how they perceieve themselves and others. The \imp{Archetype} of a character therefore has a drastic impact on the roleplaying aspect of the game.
   
As well as helping to inform what kind of person your character is, the \imp{Archetype} serves to provide them with some unique skills ({\it Features}) that they acquire and improve as they grow in power, as well as some unique special actions. 

Each \imp{Archetype} is elaborated on in more detail on their own pages. A summary is found below:
\newcommand\archEntry[2]
{
	\key{ #1} &	 \parbox[t]{5 cm}{\raggedright#2}\\
}

{
\small
\begin{center}
	\begin{rndtable}{l l l}
		\bf \imp{Archetype}		&	\bf Description
		\\
		\archEntry{Artificer}{A person trained in the delicate arts of creating and producing new items, both magical and mundane.}
		\archEntry{Auror}{A dedicated warrior-investigator, who seeks out evil and brings it to justice.} 
		\archEntry{Druid}{A person dedicated to some primal aspect of nature, earning nature-related powers and gifts.}
		\archEntry{Noble}{Someone who moves in high society, excelling in using their social graces to achieve their aims.}
		\archEntry{Outlaw}{Someone who works outside the law, employing subterfuge and deception to achieve their aims}
		\archEntry{Scholar}{Someone dedicated to knowledge, delving deep into the inner mysteries of the universe.}
		\archEntry{Warrior}{A powerful fighter, trained in all forms of combat. They excel in kicking ass, and taking names.}
	\end{rndtable}
\end{center}
\normalsize
}

\section{\imp{Archetype} Abilities} \index{Abilities!Archetype Abilities}

Each \imp{Archetype} provides an three additional \key{Abilities}, one in each of \imp{Innate}, \imp{Practical} and \imp{Knowledge} which a character can use as normal. 

Often these abilities could be duplicated by a sufficiently high roll in another field - the \imp{Pickpocket} ability associated with the \imp{Outlaw}, for example, could be achieved through a \imp{Precision (Covert)} check. However, these skills are highly tailored and even a low dice roll represents a high degree of training in this particular skill - the same as the difference between the ugly brute-force strength required to \imp{Brawl} and the weapon skills required to \imp{Skirmish}.

A character using \imp{Pickpocket} would therefore find the same action much easier than using \imp{Covert}. 

\subsection{Assigning \imp{Archetype} Abilities} \index{Character Creation!Archetype Points}

When creating a character, you automatically gain 1 dot in each of the three \imp{Archetype} abilities, and gain another 5 dots to assign freely between them.  You cannot go above a 4-dot rating in any \imp{Ability} at this stage. 

\section{\imp{Archetype} Feats} \index{Feats!Archetype Feats}

As well as granting dice-pool \imp{Abilities}, an \imp{Archetype} also grants you the choice of a number of \key{Feats}, which are powerful unique skills that a character unlocks as they progress. 

You generally do not start with any \imp{Feats} (unless your GM allows it). 


\section{Changing Archetype}

Since an \imp{Archetype} represents some fundamental aspect of a character's view of themselves and their role within the world it takes something truly monumental to alter their \imp{Archetype}. 

However, there are narrative scenarios where it makes sense for a character to switch roles as a result of events within the story - perhaps an \imp{Auror} character has been wrongly framed for a crime, and after being on the run for months they have picked up aspects of an \imp{Outlaw}'s skills. 

Such an event is rare, and should only happen if driven by a compelling narrative. When this happens, you should work with your GM to determine the nature of the change. 

Perhaps you gradually shift your abilities over a period of time - the \imp{Auror} loses his \imp{Interrogate} ability but gains the \imp{Pickpocket} ability, and after another few weeks gains knowledge of the \imp{Underworld}, until eventually they are fully an \imp{Outlaw}. Perhaps after they clear their name, they must go on a redemption arc to recover their old abilities and emerge from their life of crime.

Alternatively, the nature of the change could be dramatic and sudden - a Captain America-esque transformation turns a weedy \imp{Scholar} into a mighty \imp{Warrior} overnight, the player simply transfering the character onto a new playsheet with their new abilities and moving on from their old life. 

This is a rare and momentous undertaking, and should not be treated lightly!

