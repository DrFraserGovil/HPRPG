
\chapter{Character {Archetype}}\label{C:Archetype}\index{Archetype}\index{Class|see{Archetype}}

Whilst your character is a unique individual, an adventuring soul destined for greatness, most questers find themslves falling into one of \key{Archetypes} which helps define their abilities and goals-- are they the academic who's quest for knowledge has led to unforseen consequences, or the plucky underdog trying to quit their life of crime? 

The \imp{Archetype} (also known as the {\it class}) of your character is a way of formalising these character types. The role of your character is more than simply the job they perform, it is the prism through which they see the world. Along with their personality, it guides their very essence, how they perceieve themselves and others. The \imp{Archetype} of a character therefore has a drastic impact on the roleplaying aspect of the game.
   
As well as helping to inform what kind of person your character is, the \imp{Archetype} serves to provide them with some unique skills ({\it Features}) that they acquire and improve as they grow in power, as well as some unique special actions. 

Each \imp{Archetype} is elaborated on in more detail on their own pages. A summary is found below:
\newcommand\archEntry[2]
{
	\key{ #1} &	 \parbox[t]{5 cm}{\raggedright#2}\\
}

{
\small
\begin{center}
	\begin{rndtable}{l l l}
		\bf \imp{Archetype}		&	\bf Description
		\\
		\archEntry{Artificer}{A person trained in the delicate arts of creating and producing new items, both magical and mundane.}
		\archEntry{Auror}{A dedicated warrior-investigator, who seeks out evil and brings it to justice.} 
		\archEntry{Druid}{A person dedicated to some primal aspect of nature, earning nature-related powers and gifts.}
		\archEntry{Noble}{Someone who moves in high society, excelling in using their social graces to achieve their aims.}
		\archEntry{Outlaw}{Someone who works outside the law, employing subterfuge and deception to achieve their aims}
		\archEntry{Scholar}{Someone dedicated to knowledge, delving deep into the inner mysteries of the universe.}
		\archEntry{Warrior}{A powerful fighter, trained in all forms of combat. They excel in kicking ass, and taking names.}
	\end{rndtable}
\end{center}
\normalsize
}

\section{\imp{Archetype} Abilities} \index{Abilities!Archetype Abilities}

Each \imp{Archetype} provides an three additional \key{Abilities}, one in each of \imp{Innate}, \imp{Practical} and \imp{Knowledge} which a character can use as normal. 

Often these abilities could be duplicated by a sufficiently high roll in another field - the \imp{Pickpocket} ability associated with the \imp{Outlaw}, for example, could be achieved through a \imp{Precision (Covert)} check. However, these skills are highly tailored and even a low dice roll represents a high degree of training in this particular skill - the same as the difference between the ugly brute-force strength required to \imp{Brawl} and the weapon skills required to \imp{Skirmish}.

A character using \imp{Pickpocket} would therefore find the same action much easier than using \imp{Covert}. 

\subsection{Assigning \imp{Archetype} Abilities} \index{Character Creation!Archetype Points}

When creating a character, you automatically gain 1 dot in each of the three \imp{Archetype} abilities, and gain another 5 dots to assign freely between them.  You cannot go above a 4-dot rating in any \imp{Ability} at this stage. 

\section{\imp{Archetype} Feats} \index{Feats!Archetype Feats}

As well as granting dice-pool \imp{Abilities}, an \imp{Archetype} also grants you the choice of a number of \key{Feats}, which are powerful unique skills that a character unlocks as they progress. 

You generally do not start with any \imp{Feats} (unless your GM allows it). 


\section{Changing Archetype}

Since an \imp{Archetype} represents some fundamental aspect of a character's view of themselves and their role within the world it takes something truly monumental to alter their \imp{Archetype}. 

However, there are narrative scenarios where it makes sense for a character to switch roles as a result of events within the story - perhaps an \imp{Auror} character has been wrongly framed for a crime, and after being on the run for months they have picked up aspects of an \imp{Outlaw}'s skills. 

Such an event is rare, and should only happen if driven by a compelling narrative. When this happens, you should work with your GM to determine the nature of the change. 

Perhaps you gradually shift your abilities over a period of time - the \imp{Auror} loses his \imp{Interrogate} ability but gains the \imp{Pickpocket} ability, and after another few weeks gains knowledge of the \imp{Underworld}, until eventually they are fully an \imp{Outlaw}. Perhaps after they clear their name, they must go on a redemption arc to recover their old abilities and emerge from their life of crime.

Alternatively, the nature of the change could be dramatic and sudden - a Captain America-esque transformation turns a weedy \imp{Scholar} into a mighty \imp{Warrior} overnight, the player simply transfering the character onto a new playsheet with their new abilities and moving on from their old life. 

This is a rare and momentous undertaking, and should not be treated lightly!

\def\Feats
{
	\feat{}{}{0}{}
	\feat{}{}{0}{}
}

\def\DruidFeats
{
	\feat{A Friend to All}{Whenever a magical effect  or spell specifies that it works on \imp{Beasts}\comma{} you may also use this ability on all \imp{non\minus{}sapient} creatures\comma{} including \imp{Monsters} and \imp{Elementals}. Such creatures are also much less likely to attack you on sight\comma{} but will retaliate if provoked.}{0}{}
	\feat{Asteria’s Eyes}{At the beginning of every dawn\comma{} you perform a small ritual underneath the fading stars to divine the future. Over the next day\comma{} you use this knowledge to gain a +1d bonus\comma{} or impose a 1d penalty\comma{} on any roll performed by a creature you can see. You may use this ability a number of times equal to your highest \imp{Affinity} rating in the \imp{Divination} school}{0}{}
	\feat{Barkskin}{Until your \imp{Block} skill becomes \imp{Fully\minus{}drained}\comma{} additional levels of \imp{Drain} do not reduce the number of dice rolled on \imp{Block} checks.}{0}{}
	\feat{Cloak of Seasons}{You are magically protected from the effects of the weather and the natural environment. You are perfectly comfortable in winter's chill\comma{} summer's blazing heat or the insect\minus{}ridden swamp regardless of your clothing (or lack thereof). At the end of each \imp{Long Rest}\comma{} you may focus this aura further to gain resistance to \imp{Fire}\comma{} \imp{Electric} or \imp{Cold} damage.}{0}{}
	\feat{Dryad’s Embrace}{When you cast a spell on or attempt to \imp{Commune} with plant\minus{}based beings\comma{} or attempt to use a \imp{Knowledge} check to learn about such an entity\comma{} you gain one additional auto\minus{}success.}{0}{}
	\feat{Exuding Aura}{You may expend a \imp{Fortitude} point to attune yourself to your favoured aspect of nature\comma{} exuding an aura which influences the minds of others – perhaps a sweet pine smell calms them\comma{} or animal pheremones send them into a frenzy. You gain +2d on all \imp{Social} checks made against people within 2m of you for the next hour.}{0}{}
	\feat{Freedom of the Wild}{You are able to move freely within natural environments – your climb speed and swim speed are equal to your normal movement speed\comma{} and you are not slowed down by difficult terrain. If a creature attempts to use natural features to gain \imp{Cover}\comma{} you consider it as one level less than normal.}{0}{}
	\feat{Fungal Spore}{You know the secrets of a magical fungal spore\comma{} and can call it forth when you need it. As an action\comma{} you may summon these spore to either inflict the \imp{Poison} status effect (1 damage\comma{} requires 10 successes) on a target you can see in range\comma{} or to perform a DV 8 \imp{Insight (Commune)} check\comma{} healing the target equal to the number of successes. You may do this a number of times per day equal to your \imp{Commune} rating.}{0}{}
	\feat{Nymph’s Fury}{Channeling the power of primal\comma{} elemental spirits grants you additional power. When casting a spell from the \imp{Elemental} discipline\comma{} you gain one additional auto\minus{}success.}{0}{}
	\feat{Organic Repose}{When completing a \imp{Short Rest}\comma{} you can choose a single being to gain the benefits of a \imp{Long Rest}. You can use this ability a number of times per day equal to your \imp{Nurture} rating\comma{} but each being can only benefit from this ability once per day.}{0}{}
	\feat{Satyr Spirit}{When casting a spell or attempt to \imp{Commune} with a \imp{Beast}\comma{} or attempt to use a \imp{Knowledge} check to learn about such a creature\comma{} you gain one additional auto\minus{}success.}{0}{}
}

\def\SophisticateFeats
{
	\feat{A Little Bird Told Me….}{Whilst in conversation with someone\comma{} provided that you have met\comma{} or have heard of them in the past\comma{} you may recall an embarrassing bit of information\comma{} or incriminating secret about them. You may choose to reveal this information\comma{} or keep it secret for later. You may use this ability a number of times per day equal to your \imp{Society} rating.}{0}{}
	\feat{Brazen}{You are so brash and bold in your approach that you can simply breeze past an error\comma{} playing it for laughs\comma{} or simply ignoring it altogether. Whenever you perform a \imp{Social} check\comma{} you may treat one \imp{Catastrophe} as a normal\comma{} benign failure. By expending a \imp{Fortitude} point\comma{} you may extend this to all \imp{Catastrophes} rolled.}{0}{}
	\feat{Burn Book}{When using your skills to spread rumours\comma{} misinformation or gossip\comma{} or when trying to discredit an individual\comma{} gain +3 dice}{0}{}
	\feat{Distracting Shout}{Your forceful presence extends onto the battlefield – whilst in combat you can use a \imp{minor action} to select a target which has not yet declared their action within 10m and perform some action to get their attention. You may either force them to focus all attacks on you this turn\comma{} {\bf or} distract them\comma{} imposing a 1d penalty on any action they take.}{0}{}
	\feat{Magical Conversation}{You are able to modify the incantations required for your spells such that you can slip them into conversation without anyone recognising them as such. You may cast spells on a conversation partner without the spellcasting being noticed or considered an aggressive act.}{0}{}
	\feat{Mesmerising Presence}{You may use your alluring charm to slightly hypnotise a person\comma{} gaining +1d to social checks with them. They remember talking to you\comma{}  but are slightly starstruck and overpowered by your personality\comma{} such that they cannot recall what you talked about. You may use this ability a number of times equal to your \imp{Bamboozle} rating\comma{} before requiring a \imp{Long Rest}.}{0}{}
	\feat{Natural Leader}{You're a natural born leader. While not everyone will simply surrender authority to you\comma{} they'll consent to "follow your lead." Reduce the DV of any check directly related to leadership by 2.}{0}{}
	\feat{One for you\comma{} two for me}{Whenever your or your allies gain an amount of \imp{Galleons}\comma{} you gain one additional coin.}{0}{}
	\feat{Poker Face}{You are an expert at hiding your true feelings – beings cannot rely on the usual cues to read your demeanour or true motivation. Any \imp{Insight} attempts against you have a DV 2 higher than normal\comma{} and the \imp{Auror} `lie detector’ ability does not work on you (though they are aware that they cannot read you).}{0}{}
	\feat{Sue For Peace}{You excel at halting violence when it breaks out. Whenever you \imp{Surrender}\comma{} it is automatically accepted. When you request a foe surrender or flee\comma{} your DV is 2 lower than normal.}{0}{}
	\feat{Takes One to Know One}{You are able to automatically spot people with a similar skillset to your own\comma{} and are aware when someone attempts to \imp{Bamboozle} or otherwise manipulate you or your allies in a social situation. Your additional scrutiny means that attempts to do so have a DV 2 higher than normal.}{0}{}
	\feat{Unbreakable Vow}{When you willingly shake on a deal or contract with another sapient being both you and your partner are bound together by a magical oath. If either of you breaks the contract\comma{} the offending party takes the maximum amount of \imp{Harm} and falls into a \imp{Critical Condition}\comma{} alerting the other.}{0}{}
}

\def\ArtificerFeats
{
	\feat{Adept Alchemist}{Whenever you undertake a potion\minus{}mixing effort\comma{} you gain one additional auto\minus{}success. You may also `discover’ one common ingredient which has up to three properties of your choosing\comma{} discussing this with the GM.}{0}{}
	\feat{Defensive Gadgets}{At the end of every \imp{Long Rest}\comma{} you conceal a number of defensive doohickeys on your sleeve equal to your \imp{Danger} rating. You may expend one of these devices as an instantaneous action to gain the maximum number of successes on a single \imp{Resist} roll.}{0}{}
	\feat{Discerning Eyes}{At the end of every \imp{Long Rest}\comma{} roll a DV 7 check using only your \imp{Analyse} pool (min 1 success). You may expend one of these successes as an instantaneous action to learn about a target as if you had cast the \imp{Identify} spell at \levelThree{} level (though this does not count as a \imp{spellcasting} effort\comma{} and cannot be percieved by others). You may increase the level of the effect by expending additional successes.}{0}{}
	\feat{Expert Enchanter}{Whenever you undertake an enchanting effort\comma{} you gain one additional auto\minus{}success. You may also learn new \imp{Runes} with only 1 hour of study.}{0}{}
	\feat{Fireworks Guru}{Whenever you attempt to cause an explosion\comma{} either magical or mundane\comma{} you gain +1d to the roll. A number of times per day equal to your \imp{Modify} rating\comma{} you may also choose to double the radius of the explosion\comma{} alter its form to take on a specific shape or form\comma{} or turn it into an \imp{Implosion}.}{0}{}
	\feat{Hidden Work}{When you complete an \imp{Imbuing} or \imp{Crafting} project\comma{} you
may expend an additional hour to make your work completely hidden from inspection.
Runes are hidden and alchemical creations can appear as mundane
fluids. Only upon activation\comma{} or a spell such as \imp{Identify}\comma{} can the true
nature be divined.}{0}{}
	\feat{Idiosyncrasies}{You know every oddity and quirk of your own creations: when using them\comma{} you gain one additional auto\minus{}success. When completing an artificing project\comma{} you may also choose to make it impossible for anyone other than you to use the item.}{0}{}
	\feat{Master Mechanist}{Whenever you undertake a tinkering or mechanical manufacturing effort\comma{} you gain one additional auto\minus{}success. You may also stray further from `realistic’ or scientific constructions and may handwave slightly more vigorously over the workings of your constructions.}{0}{}
	\feat{Quick Worker}{You take only half the normal time to perform feats of \imp{Crafting} and \imp{Imbuing} – you may perform checks every 3 hours\comma{} rather than every 6.}{0}{}
	\feat{Siege Master}{When dealing damage to or attempting to bypass a building\comma{} structure\comma{} wall\comma{} door or other such solid object\comma{} you gain two additional auto\minus{}successes.}{0}{}
	\feat{Thick Skin}{Years of accidents and lab mishaps have left you with a superhuman level of resilience: choose from \imp{Fire} and \imp{Crushing} or \imp{Acid} and \imp{Poison} damage: any rolls to \imp{Resist} damage of the chosen types has a DV that is 2 lower than normal.}{0}{}
	\feat{Wandmaker}{You have perfected the art of crafting magical focusses\comma{} and can create a new wand for yourself or others. This takes up to 3 days\comma{} but you can craft the new `wand’ into any form you like – a mighty oaken staff\comma{} or a bejewelled necklace\comma{} anything conceivable as a magical focus (\imp{GM} veto)}{0}{}
}

\def\GuruFeats
{
	\feat{Alien Thoughts}{Through some weird means\comma{} or simply deep meditation\comma{} you have attuned your mind to a different way of thinking\comma{} making it harder to infiltrate your mind. You are considered \imp{Resistant} to psychic damage\comma{} and effects to alter your mind have a DV two higher than normal.}{0}{}
	\feat{Chi Siphon}{Whenever you deal damage to a being using a \imp{Brawl} action (or a melee spell attack)\comma{} you heal yourself by 1 level of harm. You may also allocate a number of your successes to increasing this effect by 1 level each (thereby reducing the damage dealt). If the attack damage is reduced to zero by a successful \imp{Resist}\comma{} the healing effect is negated.}{0}{}
	\feat{Like the Wind}{Perform a DV 6 \imp{Fitness (Glide)} check. For every success\comma{} you can run\comma{} leap or parkour 2m across a body of water\comma{} a gaping chasm\comma{} or up a vertical surface.}{0}{}
	\feat{Melee Mage}{When you cast a spell with a range of \imp{Self} which would impact your combat ability\comma{} you may immediately take a melee attack on a foe in range.}{0}{}
	\feat{Open Mind}{The way that things have always been done is not always the only (or even the best) way. A number of times per day equal to your \imp{Wisdom} rating\comma{} you may use a completely different skill\comma{} tool or ingredient for a given task. You may substitute almost any \imp{Ability} for a check\comma{} use any \imp{Tool} to complete a task\comma{} or use a random \imp{Ingredient} in a potion\comma{} confident that alternative methods can get the job done.}{0}{}
	\feat{Out of Body Experience}{Whilst unconscious or asleep\comma{} your spirit remains awake and aware of your surroundings even as you gain the benefits of rest. Your spectral self can move up to 10m away from your body\comma{} and can see and hear as well as you normally can. You can expend a \imp{Fortitude} point to instantly end the \imp{asleep} status and take an action.}{0}{}
	\feat{Piercing Insight}{You have the ability to see through people’s facade to the person beneath. When you gain at least one success on a \imp{Insight} check against a person\comma{} you learn enough hidden information about them that all further checks against them have a DV 1 lower than normal.}{0}{}
	\feat{Self Improvement}{Whenever you cast a spell with a range of \imp{Self}\comma{} gain one additional auto\minus{}success.}{0}{}
	\feat{Silent Step}{If you know the \imp{Teleport} spell\comma{} you may cast it as a silent\comma{} wandless action using your \imp{movement}\comma{} rather than a \imp{major action}. You may appear in an unnocupied space that you can see within a distance equal to 5$\times$ the power of the spell (in metres). You can do this a number of times per day equal to your \imp{Glide} ability.}{0}{}
	\feat{Stillness of Mind and Body}{You may expend a \imp{Fortitude} point as an instantaneous action to enter a meditative state\comma{} ending all ongoing status effects on your body: \imp{Poisons} are purged\comma{} \imp{Fear} is banished\comma{} and \imp{Paralysis} wears off. This does not effect status effects imposed by physical phenomena (i.e. you cannot untie ropes giving the \imp{Trapped} status)\comma{} and you must be conscious to use this ability. You can do this a number of times per day equal to your \imp{Self} rating.}{0}{}
	\feat{Wise Teacher}{You are always on the lookout to help others. As a minor action\comma{} you take a second to talk to and inspire your ally: donating your \imp{Widsom} pool to another character within 2m. They may add this pool to any check they make next turn cycle. You may use this action multiple times per cycle\comma{} but each character can only benefit from the effect once.}{0}{}
}

\def\WarriorFeats
{
	\feat{All Guns Blazing}{When making an attack against a small group of people\comma{} you can truly throw yourself into the attack\comma{} expending two \imp{Fortitude} points to take them all on at once. As a \imp{Major Action}\comma{} take a single attack against a number of beings (up to twice your \imp{Rage} score) within range.}{0}{}
	\feat{Blind Rage}{When using a \imp{Rage} action to attack\comma{} you ignore all dice penalities due to injuries.}{0}{}
	\feat{Bloodlust}{On any turn in which you successfully incapacitate (lethally or not) a foe\comma{} you may take an additional free action to perform another attack\comma{} with a 2\minus{}dice penalty on the check.}{0}{}
	\feat{Chivalrous Defence}{At any point prior to or during your action in the turn cycle\comma{} you may donate up to two dice to a creature within 2m of you that you can see. This creature may use this additional dice on any \imp{Resist} actions they take this turn. You take a dice penalty on any attacks or skill checks (but not your own \imp{resists}) you make this turn equal to the number of dice donated.}{0}{}
	\feat{Duelist}{When fighting against a single foe\comma{} you gain one auto\minus{}success on all attack rolls and Resist actions.}{0}{}
	\feat{Furious Spellcaster}{Up to three times a day\comma{} you may elect to use a \imp{Rage} check\comma{} rather than the associated \imp{Affinity} to perform a spellcasting check. The maximum spell level is still determined by your \imp{Affinity}.}{0}{}
	\feat{Holistic Tactics}{Whenever a \imp{Tactics} check is successful in determining the tactics of a foe\comma{} you may use this knowledge to infer any Resistances\comma{} Immunities or Susceptabilities the target has.}{0}{}
	\feat{Lightning Strikes}{At the end of each long rest\comma{} perform a DV 6 check using just your \imp{Speed} pool. For each success gained (min 1) you may perform one additional attack at some point over the next day without expending a \imp{Fortitude} point.}{0}{}
	\feat{Mind over Matter}{You may \imp{Endure} the harm caused by physical attacks. Whilst your \imp{Endure} rating remains above zero\comma{} penalties due to \imp{Harm} are one level less serious.}{0}{}
	\feat{Never Give In}{As a minor action\comma{} you take a moment to clear your head\comma{} suppress the pain in your limbs\comma{} or readjust your armour. Reduce the level of \imp{Drain} in a defence of your choice to 1. You can do this a number of times per day equal to your \imp{Vitality} rating.}{0}{}
	\feat{Proud Leader}{Whenever you use a \imp{Command} action to issue an instruction or provide guidance\comma{} if that action is a success\comma{} you regain a \imp{Fortitude} point. You may do this a number of times per day equal to your \imp{Command} rating.}{0}{}
	\feat{Student of War}{If you study\comma{} read up on and otherwise prepare for a target before engaging them in combat\comma{} the DV of all attacks you make against them is reduced by 1. If you use a \imp{Command} action this turn\comma{} then this bonus is shared amongst all allies within 20m of you until the end of the next \imp{turn cycle}.}{0}{}
}

\def\AurorFeats
{
	\feat{Ambush}{When you attack from hiding\comma{} spring a trap or successfully orchestrate
an ambush\comma{} if the total number of successes on your first roll is less than half the number of dice rolled\comma{} you may instead use that number.}{0}{}
	\feat{Arcane Detective}{When performing an investigation\comma{} you notice if a spell has been cast nearby in the last week and may determine which spell school it belonged to. By expending a \imp{Detective’s Toolset}\comma{} you may learn exactly what spell\comma{} and how long ago it was cast.}{0}{}
	\feat{Cold Cases}{When performing a Knowledge check\comma{} if you can relate the information
you seek to a historical or past case you reduce the DV by 3.}{0}{}
	\feat{De\minus{}escalation Training}{Having been trained to capture and contain\comma{} when you take an action to contain\comma{} constrain\comma{} bind\comma{} trap or disarm a foe\comma{} rather than inflict damage\comma{} you gain +1d to the effort.}{0}{}
	\feat{Deductive Senses}{Your logical mind allows you to keep track of hidden foes. Whenever a creature that you can see goes invisible\comma{} or otherwise becomes undetectable\comma{} perform a DV 8 \imp{Insight (Intuition)} check as an instantaneous action. For every success\comma{} you remain aware of the position of that creature for an additional \imp{Combat Cycle}\comma{} and do not treat it as invisible.}{0}{}
	\feat{Familiar Terrain}{Choose a favoured terrain such as \imp{Grasslands}\comma{} \imp{Forests}\comma{} or \imp{Urban Areas} or name a specific region\comma{} such as \imp{Hogwarts}. Whilst in your favoured terrain you gain +1d on every action which utilises the surroundings such as a \imp{Tracking} or \imp{Covert} check.}{0}{}
	\feat{Fancy Footwork}{When fighting more than one foe\comma{} you may use an action to expend a \imp{Fortitude} point to confuse your foes with some feat of athletics and maneouvering\comma{} causing them to attack each other. Nominate two enemies within range – next turn cycle\comma{} the first of these two to take an action will attack the other\comma{} instead of their intended target.}{0}{}
	\feat{Heightened Senses}{For 5 minutes after performing an investigation check\comma{} your senses remain heightened such that you do not incur \imp{Drain} on your \imp{Dodge} ability.}{0}{}
	\feat{Lie Detector}{You can automatically detect when someone is lying to you by telling you deliberate falsehoods. You may expend a \imp{Fortitude} point to get a glimpse at what they are hiding or lying about.}{0}{}
	\feat{Mental Training}{You gain +3d against all checks to resist unnatural mental manipulation\comma{} and may expend a \imp{Fortitude} point to end an ongoing mental effect such as \imp{Charmed}.}{0}{}
	\feat{Rapid Reflexes}{When performing a \key{Reflex} roll\comma{} you may roll the dice twice and take the largest value.}{0}{}
	\feat{Unwavering Focus}{Once per day you may expend a \imp{Fortitude} point to reroll all \imp{Catastrophe} dice you rolled\comma{} declaring this action after the roll has been performed\comma{} but before the outcome has been narrated.}{0}{}
}

\def\AllFeats
{
	\feat{Animagus}{As an exceptionally powerful Thaumaturge\comma{} you have learned to shift your shape into that of an animal. By expending a \imp{Fortitude} point\comma{} you may instantly assume the physical attributes of your `spirit animal’. This animal is (usually) the same as your \imp{Patronus} and can only be changed if your undergo a profound change on a spiritual level. Whilst in this form\comma{} you retain your mental state\comma{} but all your physical statistics and abilities are replaced by those of the form you take. 

You may choose to revert back to your human form at any time\comma{} and you do so automatically if you are reduced to a \imp{Critical Condition} whilst in bestial form.}{1}{\imp{Alteration} (\fiveCape{})}
	\feat{Armour Piercing}{When a target attempts to block your attacks\comma{} their armour takes an additional level of \imp{Drain}: a \imp{full\minus{}round block} incurs one level of drain\comma{} and \imp{quickblocking} incurs two levels.}{0}{}
	\feat{Crash Course}{You study a crash course in a selection of 10 abilities you previously had no skill in\comma{} giving you a basic level of knowledge. Choose up to 10 \imp{Abilities} with a \emptyCape{} rating\comma{} and gain 1 dot in each of them.}{0}{}
	\feat{Elemental Attunement}{You feel a particular affinity for one of the elements (Fire\comma{} Water\comma{} Ice\comma{} Earth\comma{} Air\comma{} Lightning\comma{} etc.) deep within your bones. When casting a spell to manipulate\comma{} create or otherwise effect your chosen element\comma{} you gain +1d. You also gain +1d to any check to \imp{resist }damage caused by your element.}{1}{\imp{Elemental} (\threeCape{})}
	\feat{Emergency Care}{You know the basics of helping others and mainting life. Whenever a being falls into the \imp{critical condition} status within 3m of you\comma{} you may expend a \imp{Fortitude} point to allow them to perform emergency care. Perform a DV 8 \imp{Insight (Kindness)} check\comma{} and heal them by an amount equal to the number of successes\comma{} removing the \imp{Critical Condition} status. If you have a \imp{First Aid Kit}\comma{} you may expend that rather than a \imp{Fortitude} point.}{0}{}
	\feat{Helping Hand}{You are so proficient in helping out your allies that your `help’ action gives +3d\comma{} rather than +1.}{0}{}
	\feat{Innate Trick}{As a witch or wizard\comma{} the chaotic force of magic flows within your veins. You have learned to harness this magic in some innate way beyond the usual spellcasting. This effect is usually minor (something a Muggle could put down to an act of trickery or showmanship)\comma{} and often forms the basis of a parlour trick. 

You might be able to summon a small flame from your finger\comma{} make your eyes into burning coals or deep black voids\comma{} play a stirring soundtrack whenever they engage in a fight\comma{} know the name of every individual you meet\comma{} or some other marvellous but ultimately slightly inconsequential feat that you could imagine being the focus of conversation at a party. 

No rolls are needed to use this ability\comma{} and the GM has a veto if this tool is being used in an inappropriate fashion. {\bf This ability costs only 4 \imp{EXP}.}}{0}{}
	\feat{Instinctive Defence}{Whilst you have a \imp{wand} equipped and are able to cast spells\comma{} you gain a +1 bonus to all \imp{base defence bonuses}.}{0}{}
	\feat{Jack\minus{}of\minus{}all\minus{}Trades}{You have a surprising amount of miscellaneous skills\comma{} knowledge and abilities that you have acquired over your life\comma{} and are often able to surprise your allies with something pulled from your sleeve. 

Each day\comma{} you get 4 free dots\comma{} which you may temporarily allocate to \imp{Abilities} as and when you need them\comma{} though you may not increase any ability to more than 5 dots. You may use this enhanced ability for the next hour\comma{} before the effect wears off. You regain your dots when you complete a \imp{Long Rest}.}{0}{}
	\feat{Light Sleeper}{You need much less sleep than others\comma{} and can go from asleep to awake in a blink of an eye. You gain the benefits of a \imp{Long Rest} after only 4 hours\comma{} and may ignore the effects of \imp{Level One Exhaustion}. Any \imp{Alertness} checks called for whilst asleep have the DV reduced by 3.}{0}{}
	\feat{Loyal Companion}{You have an animal ally which is eternally loyal and devoted to you\comma{} and can carry out simple tasks: a `familiar’. This familiar is a well trained pet and will follow simple orders\comma{} though they cannot communicate back. You maintain a weak psychic link allowing you to know their current physical and emotional condition. The most common animals are owls\comma{} ravens\comma{} cats\comma{} rats and toads\comma{} though you may ask your GM for a different choice.}{1}{Kinship (\twoCape)}
	\feat{Martial Arts}{You are a master of unarmed combat\comma{} making your hands into lethal weapons. Unarmed strikes now deal an additional level of damage (1 + \# Successes)\comma{} with a reduced DV of 4. In addition\comma{} before making an attack roll\comma{} you may expend a \imp{Fortitude} point to automatically roll all your \imp{Brawl} dice as successes.}{1}{Brawl (\threeCape)}
	\feat{Moving Target}{On any turn during which you move more than half your movement (without doubling back)\comma{} your \imp{quickdodge} checks do not incur \imp{Drain}.}{0}{}
	\feat{Numbed to Pain}{When you expend a \imp{Fortitude} point to ignore the negative effects of \imp{Harm}\comma{} the effect lasts for one hour\comma{} rather than just the next round.}{1}{Vitality (\threeCape)}
	\feat{Parry Master}{Carrying a melee weapon grants a +2 to your base \imp{block} bonus\comma{} in addition\comma{} if you \imp{Block} an incoming melee attack to a power of \imp{zero}\comma{} you may expend a \imp{fortitude} point to \imp{disarm} your attacker.}{0}{}
	\feat{Psychic Awareness}{Your mind is especially attuned to those of others\comma{} and you can naturally sense the shift induced when a psychic power alters or interacts with minds. Whenever a psychic effect such as mind reading\comma{} memory modification\comma{} or magic which alters emotions and allegiances is used on a target within 5m of you\comma{} you are instantly aware of this – you may expend a \imp{fortitude} point to learn the target of the spell\comma{} as well as the caster and the intent of the spell.}{1}{\imp{Kindness} (\threeCape{})}
	\feat{Ritualist}{You are a strong believer that the most powerful magic is performed with large groups\comma{} in elaborate rituals\comma{} with chanting\comma{} incense and possibly a pentagram or two. Whenever you invoke a \imp{Ritual} to cast a spell\comma{} you gain one automatic success for every 3 members of the ritual (max +5d).}{1}{\imp{Occultism} (\threeCape{})}
	\feat{Second Chances}{Once per day\comma{} you may re\minus{}roll any number of dice on a single check\comma{} but must keep the new result.}{0}{}
	\feat{Signature Spell}{You have a spell which is considered your `signature move’\comma{} chosen when you take this feat. When casting this spell\comma{} \imp{catastrophes} are counted as normal failures. You may change your `signature spell’ only with GM consent that your old choice no longer represents your character’s go\minus{}to move.}{0}{}
	\feat{Silent Casting}{You do not need to perform the verbal component of a spellcasting action. Efforts to silence you do not impact your spellcasting efforts\comma{} and the lack of an alerting incantation means efforts to \imp{Resist} your spells take a 1d penalty.}{0}{}
	\feat{Wandless Casting}{You are able to perform limited feats of magic without needing the crutch of a wand or ritualistic movements\comma{} so attempts to disarm your or bind you in place do not affect your spellcasting efforts. You take a 1d penalty on all wandless spellcasting efforts. All wandless actions are also silent.}{1}{Silent Casting}
}

\def\ResponderFeats
{
	\feat{Apothecary}{Whenever you take a \imp{Short Rest} whilst in possession of a \imp{First Aid Kit} or a \imp{Alchemy Set}\comma{} you can brew a quick healing potion\comma{} gaining a base\minus{}level \imp{Wiggenweld Potion} (see page \pageref{P:Wiggenweld}). You cannot store more than 5 such potions using this ability.}{0}{}
	\feat{Beacon of Life}{Your \imp{Hermetics} spells now channel so much positive energy that they deal \imp{harm} to unliving and evil creatures such as \imp{Demons}\comma{} \imp{Abominations}\comma{} \imp{Phantasms} and \imp{Undead}. When you deal harm using this ability\comma{} you restore health to yourself equal to half the damage dealt.}{0}{}
	\feat{Emergency Procedure}{When performing a \imp{Healing} check\comma{} or a \imp{Defensive} check you may choose to automatically roll the maximum number of successes. If an enemy reduces the power of this action to zero\comma{} you (or your patient\comma{} as appropriate)\comma{} fall into the \imp{Critical Condition} status. You cannot use this ability again until you take a \imp{Long Rest}.}{0}{}
	\feat{Healer}{Whenever you restore Health to a being\comma{} or remove a negative status effect\comma{} gain one additional auto\minus{}success.}{0}{}
	\feat{Matyr}{Whenever a being which you have successfully cast a \imp{Hermetics} spell on takes damage (but before they \imp{Resist})\comma{} you may elect to transfer the damage to yourself\comma{} performing the \imp{Resist} yourself. The list of targets for this ability resets whenever you take a \imp{short} or \imp{long} rest.}{0}{}
	\feat{Plague Doctor}{You release a virulent pathogen around you\comma{} infecting all those within 5m of your current position (you may expend a \imp{Fortitude} point to exclude any number of allies). You may choose the effect of the pathogen:
\begin{itemize}
\item \key{Poisoned}: 2 damage per turn\comma{} power = your \imp{Pathology} rating
\item \key{Blinded}: power = your \imp{Willpower} rating
\item \key{Enraged}: power = your \imp{Deception} rating
\end{itemize}
Targets make a check to \imp{Resist} at the onset of this effect\comma{} and then again at the end of every subsequent cycle. You can use this ability a number of times equal to your \imp{Pathology} rating before requiring a \imp{Long Rest}.}{0}{}
	\feat{Preventative Treatment}{You gain +2d to any \imp{Defensive} actions used to reduce the effectiveness of an incoming attack. If you reduce the \imp{Power} of an attack to zero\comma{} the target(s) of the attack gain 1 point of healing.}{0}{}
	\feat{Red Cross}{By expending a \imp{Fortitude} point\comma{} you may use a \imp{Minor Action} to designate a target within \imp{Wandtip} range as a non\minus{}combatant. Any being which attempts to harm the designated target (either with a targeted attack\comma{} or an Area\minus{}of\minus{}Effect attack) must first perform a \imp{Willpower (Conviction)} check (or similar) with a DV equal to 5 + your \imp{Willpower} rating. On a failure\comma{} the attack falters and the attacker’s turn ends. This effect lasts until the target re\minus{}enters combat.}{0}{}
	\feat{Stabilising Presence}{When a target within 10m falls into the \imp{Critical Condition} status\comma{} you may expend a \imp{Fortitude} point to instead place them into \imp{Critical But Stable}.}{0}{}
	\feat{Warden}{Whenever you use a \imp{Major Action} to restore \imp{Health} to a being\comma{} or remove a negative  status effect\comma{} you may use an \imp{instantaneous} reaction to cast a defensive spell from the \imp{Warding} discipline. You can do this a number of times equal to your \imp{Triage} rating\comma{} before needing a \imp{Long Rest}}{0}{}
}

\def\OutlawFeats
{
	\feat{Cover Identity}{Given enough time – perhaps a day or two – you can forge yourself a completely new identity\comma{} with the necessary paperwork and credentials to reasonably pass as whoever you desire. This may not stand up to high\minus{}level scrutiny\comma{} but most people should be easily fooled.}{0}{}
	\feat{Eyes of Greed}{You have a knack for spotting things of value. You gain +2d on any checks to look for valuable items and can automatically know the value of non\minus{}magical items\comma{} and some \imp{Common} magical items.}{0}{}
	\feat{Fight Dirty}{If you use a \imp{Called Shot} to target a foe’s tender bits\comma{} pull hair\comma{} or throw sand in their eyes\comma{} etc. and the attack is successful\comma{} the target takes a dice penalty on all attacks taken next turn\comma{} equal to your \imp{Brawl} rating.}{0}{}
	\feat{Freerunning}{Your base movement speed is increased to 5\comma{} your climb speed is equal to your movement speed and your jump distance is doubled. You gain +1d on any \imp{Resist} checks against effects to slow you down or trap you.}{0}{}
	\feat{Hidden Weapon}{You can draw a previously unknown small blade from a fold in your robes\comma{} or a hidden pocket\comma{} and then use it as an instantaneous action. You can use this ability a number of times per day equal to your \imp{Covert} rating.}{0}{}
	\feat{Innocent Face}{You are always thought of as an honest and good soul. If you do something wrong which isn’t immediately attributed to you\comma{} it will most likely be blamed on something else. As long as you’re not caught red\minus{}handed killing puppies\comma{} people will try to excuse your actions and move on from your misdeeds.}{0}{}
	\feat{Move in Shadow}{Whilst you are outside of bright light\comma{} all \imp{Covert} actions have a DV one lower than normal.}{0}{}
	\feat{Naturally Shifty}{Doing unscrupulous deeds comes as naturally to you as breathing – gain one additional auto\minus{}success on any \imp{Covert} action.}{0}{}
	\feat{Play the System}{You know every trick and loophole in the rules and regulations. Whenever you perform a \imp{Social} check against a figure of authority\comma{} you may gain a +2d bonus. You can use this ability a number of times equal to your \imp{Savvy} rating before you need a long rest.}{0}{}
	\feat{Sly Action}{At the end of a turn cycle\comma{} if you have not been directly targeted for an attack\comma{} you may take an additional \imp{minor} action at the end of the cycle to move\comma{} use an item\comma{} or otherwise attempt to hide}{0}{}
	\feat{Surprise Attack}{Whenever you attack a target from a position where they cannot see you\comma{} the target is considered \imp{suscpetible} to the attack.}{0}{}
	\feat{Unobtrusive}{You don’t stand out in a crowd\comma{} and can make yourself\comma{} if not invisible\comma{} just socially {\it absent}. People don’t necessarily remember your face or your name (if they remember you at all)\comma{} as you make very little impression on people – until you are stealing their wallet\comma{} or knifing them in the back\comma{} that is. All checks made by enemies to notice you in a crowd\comma{} or remember details about you have a DV 3 higher than normal.}{0}{}
}

\def\ScholarFeats
{
	\feat{Flexible Knowledge}{You may spend a day completely immersing yourself in a new set of knowledge and expertise. Convert a number of your \imp{Knowledge} dots back into experience points. You must then reassign at least this number of experience points back into your \imp{Knowledge} abilities in any new combination you wish.}{0}{}
	\feat{Learn From Failure}{When performing a check of any kind\comma{} for every consectutive previous roll which failed\comma{} you gain +2d (up to a maximum of your \imp{Stubbornness} rating).}{0}{}
	\feat{Library Lover}{When you perform an intelligence\minus{}based check whilst in a library\comma{} or similar repository of knowledge\comma{} the dice associated with your \imp{Intelligence} are all counted as successes. You need only roll your \imp{Ability} dice.}{0}{}
	\feat{Magical Prodigy}{When initiating a \imp{Spellcasting} action\comma{} you may automatically reduce the DV required to cast it by 2. You may do this a number of times equal to your \imp{Intelligence} rating before you require a \imp{Long Rest}.}{0}{}
	\feat{Master of the Mind}{When an action would interrupt you casting a spell requiring continuing concentration\comma{} the DV to remain focussed is 2 lower than normal. You also gain +1d on all \imp{Resist} checks against mind\minus{}altering effects.}{0}{}
	\feat{Navigator’s Knack}{You are able to perfectly keep track of time even whilst asleep\comma{} cannot get lost except by magical means and can always pinpoint your location anywhere on the globe given reasonable information i.e. access to the stars. In addition\comma{} if any magical effect removes you from your normal plane of existence\comma{} you are able to feel and notice this happening.}{0}{}
	\feat{Novel Technique}{Whenever you use a spell in a new or novel fashion\comma{} you gain +2d for the spellcasting effort.}{0}{}
	\feat{Quick Learner}{You need to spend half the usual time in order to learn a new spell\comma{} or memorise an enchanting rune\comma{} and automatically succeed on the final memorisation roll for spells.}{0}{}
	\feat{Subject Expertise}{Choose a spell discipline\comma{} or a viable target of a spell\comma{} which is associated with your area of research or specialty. When casting a spell of this school\comma{} or a spell on your chosen target\comma{} you gain one additional auto\minus{}success.}{0}{}
	\feat{Well Read}{You hold in your brain a simply incredible amount of information\comma{} you may use some of this vast knowledge to gain +3d on one \imp{Knowledge} check. You can use this ability a number of times equal to your \imp{General Knowledge} rating before needing a long rest.}{0}{}
}


