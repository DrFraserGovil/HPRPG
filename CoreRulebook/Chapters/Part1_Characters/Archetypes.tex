
\chapter{Character {Archetype}}\label{C:Archetype}\index{Archetype}\index{Class|see{Archetype}}

Whilst your character is a unique individual, an adventuring soul destined for greatness, most questers find themslves falling into one of \key{Archetypes} which helps define their abilities and goals-- are they the academic who's quest for knowledge has led to unforseen consequences, or the plucky underdog trying to quit their life of crime? 

The \imp{Archetype} (also known as the {\it class}) of your character is a way of formalising these character types. The role of your character is more than simply the job they perform, it is the prism through which they see the world. Along with their personality, it guides their very essence, how they perceieve themselves and others. The \imp{Archetype} of a character therefore has a drastic impact on the roleplaying aspect of the game.
   
As well as helping to inform what kind of person your character is, the \imp{Archetype} serves to provide them with some unique skills ({\it Features}) that they acquire and improve as they grow in power, as well as some unique special actions. 

Each \imp{Archetype} is elaborated on in more detail on their own pages. A summary is found below:
\newcommand\archEntry[2]
{
	\key{ #1} &	 \parbox[t]{5 cm}{\raggedright#2}\\
}

{
\small
\begin{center}
	\begin{rndtable}{l l l}
		\bf \imp{Archetype}		&	\bf Description
		\\
		\archEntry{Artificer}{A person trained in the delicate arts of creating and producing new items, both magical and mundane.}
		\archEntry{Auror}{A dedicated warrior-investigator, who seeks out evil and brings it to justice.} 
		\archEntry{Druid}{A person dedicated to some primal aspect of nature, earning nature-related powers and gifts.}
		\archEntry{Guru}{A person possessing great wisdom and knowledge of themselves, leveraging this for feats of self-improvement and inner strength.}
		\archEntry{Outlaw}{Someone who works outside the law, employing subterfuge and deception to achieve their aims}
		\archEntry{Responder}{A healer and specialist in reversing and preventing the cause of harm.}
		\archEntry{Scholar}{Someone dedicated to knowledge, delving deep into the inner mysteries of the universe.}
		\archEntry{Sophisticate}{Someone who moves in high society, excelling in using their social graces to achieve their aims.}
		\archEntry{Warrior}{A powerful fighter, trained in all forms of combat. They excel in kicking ass, and taking names.}
	\end{rndtable}
\end{center}
\normalsize
}

\section{\imp{Archetype} Abilities} \index{Abilities!Archetype Abilities}

Each \imp{Archetype} provides an three additional \key{Abilities}, one in each of \imp{Innate}, \imp{Practical} and \imp{Knowledge} which a character can use as normal. 

Often these abilities could be duplicated by a sufficiently high roll in another field - the \imp{Pickpocket} ability associated with the \imp{Outlaw}, for example, could be achieved through a \imp{Precision (Covert)} check. However, these skills are highly tailored and even a low dice roll represents a high degree of training in this particular skill - the same as the difference between the ugly brute-force strength required to \imp{Brawl} and the weapon skills required to \imp{Skirmish}.

A character using \imp{Pickpocket} would therefore find the same action much easier than using \imp{Covert}. 

\subsection{Assigning \imp{Archetype} Abilities} \index{Character Creation!Archetype Points}

When creating a character, you automatically gain 1 dot in each of the three \imp{Archetype} abilities, and gain another 3 dots to assign freely between them.  

\section{\imp{Archetype} Equipment \& Spells}

Each archetype also grants a list of equipment which is placed into your \imp{inventory} upon creating such a character. Where a choice is indicated (by the key word `\key{or}'), you must choose only one item from the list, unless otherwise specified. 


Included in the initial selection criteria is a selection of spells that you have memorised. This usually takes the form of a number of spells memorised from the \key{Basic Spells Table} (see below), and a choice of at least one additional spell from a smaller selection related ot the abilities of the \imp{archetype}. You may immediately transcribe these spells into the \imp{Spellbook} section of your character sheet. 

If your \imp{GM} wishes to run your characters starting from the {\it very} beginnings of their magical stories, it may be appropriate for you to start with no equipment and no spells memorised - you may roleplay purchasing your first wand, and your first lessons at Hogwarts. By the end of the first session, however, you should have acquired all of your basic skills and equipment. 


\subsubsection{Basic Spells Table} \label{T:BasicSpells}

\newcommand\impIt[2]{\key{#1}	&	{\imp{#2}} \\}
\begin{center}
\begin{rndtable}{l l}
	\key{School}	&	\key{Spells} \\
	\impIt{Alteration}{Refine, Transmute}
	\impIt{Bewtichment}{Charm, Distract, Mirage}
	\impIt{Cerebral}{Communicate, Sense}
	\impIt{Conjuration}{Bind, Manifest}
	\impIt{Curse}{Disable, Disarm}
	\impIt{Elemental}{Burn, Freeze, Gust, Illuminate, Soak}
	\impIt{Hermetics}{Heal, Restore}
	\impIt{Hex}{Force, Jinx}
	\impIt{Kinesis}{Move, Repair}
	\impIt{Temporal}{Identify}
	\impIt{Warding}{Abjure, Shield}
\end{rndtable}
\end{center}


\section{\imp{Archetype} Feats} \index{Feats!Archetype Feats}

As well as granting dice-pool \imp{Abilities} and starting equipment, an \imp{Archetype} also grants you the choice of a number of \key{Feats}, which are powerful unique skills that a character unlocks as they progress. A general list of feats accessible to all characters can be found on page \pageref{S:AllFeats}, and each \imp{Archetype} grants a number of feats unique to characters following that path.

You generally start with one \imp{Feat} chosen from your \imp{Archetype} list, though your \imp{GM} may allow you to choose a more general feat. 

\section{Advanced Start}

The default assumptions listed here for the initial \imp{skills}, \imp{feats} and \imp{spells} possessed by characters assumes that your campaign is centred around novice characters, most likely characters who are just starting at \imp{Hogwarts}. If, however, your \imp{GM} wishes to use a campaign in which characters are already matured into their powers (perhaps more akin to the \imp{Fantastic Beasts} movies), then they may decide to grant you the following additional starting bonuses:

\begin{itemize}
	\item 15 \imp{Experience} points to spend as you wish, following the normal character progression rules
	\item Two additional spells from the basic spells table, \key{or} one spell from the full spell list, \key{or} 3 enchanting runes
	\item \galleon{5} to spend on items and ingredients.
\end{itemize}

\section{Changing Archetype}

Since an \imp{Archetype} represents some fundamental aspect of a character's view of themselves and their role within the world it takes something truly monumental to alter their \imp{Archetype}. 

However, there are narrative scenarios where it makes sense for a character to switch roles as a result of events within the story - perhaps an \imp{Auror} character has been wrongly framed for a crime, and after being on the run for months they have picked up aspects of an \imp{Outlaw}'s skills. 

Such an event is rare, and should only happen if driven by a compelling narrative. When this happens, you should work with your GM to determine the nature of the change. 

Perhaps you gradually shift your abilities over a period of time - the \imp{Auror} loses his \imp{Interrogate} ability but gains the \imp{Pickpocket} ability, and after another few weeks gains knowledge of the \imp{Underworld}, until eventually they are fully an \imp{Outlaw}. Perhaps after they clear their name, they must go on a redemption arc to recover their old abilities and emerge from their life of crime.

Alternatively, the nature of the change could be dramatic and sudden - a Captain America-esque transformation turns a weedy \imp{Scholar} into a mighty \imp{Warrior} overnight, the player simply transfering the character onto a new playsheet with their new abilities and moving on from their old life. 

This is a rare and momentous undertaking, and should not be treated lightly!

\section{Archetype Inspirations}

It should go without saying that the \imp{Archetypes} presented here are based on the roles played by characters within the \imp{Harry Potter} novels and related films - with a {\it healthy} dose of modification to fit in with common RPG norms and tropes. 

It might be helpful for players' understanding of the \imp{Archetype}, and their ability to roleplay them,  to understand which characters inspired the creation of the \imp{Archetypes}:

\subsubsection{Artificer}

\imp{Artificers} are masters of creating things - perhaps the most prominent examples of \imp{Artificers} in the series are the \imp{Weasley} twins. Though their personality would seem to indicate a leaning towards \imp{Outlaws}, they are repeatedly praised for their ingenious creations, most of which are of an \imp{Alchemical} nature, though the creation of \imp{Shield Hats} and \imp{Trick Wands} shows they have a flair for \imp{Enchanting} as well. 

\imp{Severus Snape}, in his guise as the `half-blood prince' also showed an incredible ability to tweak and alter established potion recipes to new ends. He would therefore be an \imp{Alchemist-Artificer} with a high \imp{Modify} rating.

\subsubsection{Auror}

Actual \imp{Aurors} abound within the \imp{Harry Potter} novels, so it is not hard to find the inspiration for such characters. 

\imp{Harry Potter} himself ends up becoming the prototype for the \imp{Aurors} - his thirst for righteousness and justice, and his preference for less-lethal methods of combat were particular inspirations. \imp{Harry} himself is not quite as level-headed, controlled and trained as a `true' \imp{Auror} might be: \imp{Harry} probably has a very high \imp{Intuition} rating, as he seems to use his gut feelings above logic.

Because of the nature of the \imp{Harry Potter} series as it reached its conclusion, most characters' classified as Aurors tended towards being a \imp{Warrior} - to make a starker difference between these two professions, the \imp{Auror} class focusses more on the problem-solving and mystery-hunting aspects of the \imp{Auror} profession.

\subsubsection{Druid}

Druids care about, and care for, the natural order of things, the natural places in the world, and the beings which live there. Perhaps the most prominent (though perhaps not the wisest or the most skilled) druid would be \imp{Rubeus Hagrid}, who has a particular affinity for the more monstrous creatures. \imp{Newt Sacamander}, author of {\it Fantastic Beasts and Where to Find Them} is also a prominent influence. 

Of course, not everything natural is plants and beasts: the stars in the night sky and the fates they guard, as well as the primal elements of Fire, Air, Earth and Water would also fall under a Druid's domain. For this reason, \imp{Sybil Trewlany} would be a celestial druid, whilst the infamous \imp{Merlin} was recognised as a powerful elemental-druid.

\subsubsection{Guru}

It might seem that the \imp{Guru} is primarily based on the \imp{monk}-like classes found in many RPG's. In fact, the main inspiration is \imp{Luna Lovegood}. 

\imp{Luna} is known for her ditsy attitude and insane beliefs, which hides an ability to dispense great wisdom. Her unconventional beliefs also inspired the idea that \imp{Gurus} are focussed on finding new and unusual ways to do things - usually by finding new and undiscovered aspects of their own abilities. 


\subsubsection{Outlaw}

Perhaps the only true `Outlaw' within the books was \imp{Mundungus Fletcher}, a reprobate lowlife and a coward - but who certainly had great knowledge of the underworld. 

However, few would doubt that \imp{Gilderoy Lockhart} was a con artist and a crook, and so would probably end up as an \imp{Outlaw} too - though he made good use of the \imp{Alternative Profession} and \imp{Play the System} feats to appear as a well-respected \imp{Sophisticate}.

\subsubsection{Responder}

\imp{Madam Poppy Pomfrey} is perhaps the only character within the series who could be classified as a \imp{Responder}, given the relative scarcity of healers or doctors within the series. 

However, note that \imp{Responders} are far more than simple doctors and nurses waiting for the injured to arrive: they also excel in the protection of others and the negation of harm. \imp{Filius Flitwick} notably cast perhaps the most powerful protective enchantments within the series (just before the \imp{Battle of Hogwarts}), and so should probably be considered a \imp{Responder}. Equally, \imp{Molly Weasley} threw herself into battle in order to protect her children from \imp{Bellatrix Lestrange}, so she could be a \imp{Responder} with the \imp{Matyr} feat.

\subsubsection{Scholar}

Most of the teachers at \imp{Hogwarts} would obviously fall under the domain of a \imp{Scholar}, but perhaps the most famous \imp{Scholar}-character would be \imp{Hermione Granger}. 

\imp{Hermione} is well-known for her vast knowledge and near-savant level of understanding in magical fields, as well as her ability to learn new things at a rapid rate. 

\imp{Albus Dumbledore} would probably also be a \imp{Scholar} - his most famous feats include the discovery of 12 different uses for Dragons blood, and hte invention of several magical doohickeys (which could conceivably make him a \imp{Artificer} - though the distinction here is the same as between a scientist and an engineer). 


\subsubsection{Sophisticate}

Sophisticates wield their social graces to achieve their own ends, as well as wealth and power (or at least the image of it). The \imp{Malfoys} would probably be \imp{Sophisticates} - though given how rarely `I'll tell my father' worked out for \imp{Draco}, relatively poor ones. 

\imp{Rita Skeeta} weaponised her journalism career to great ends, and so would be a great \imp{Sophisticate}.  

\subsubsection{Warrior}

Many characters, by the end of the \imp{Second Wizarding War} ended up as \imp{Warriors} - people like \imp{Kingsley Shacklebolt} were forceful presences on the battlefield, and many of the \imp{Death Eaters} excelled at putting down their foes as quickly as possible. 

Characters such as \imp{Grawp} and the other giants, as well as the \imp{Centaurs} that faught \imp{Umbridge} would also be warriors - though focussed on weapons and non-magical combat. 

\def\ArtificerFeats
{
	\feat{Adept Alchemist }{Whenever you undertake a potion\minus{}mixing effort\comma{} you gain one additional auto\minus{}success. You may also `discover’ one common ingredient which has up to three properties of your choosing\comma{} discussing this with the GM.}{0}{}
	\feat{Expert Enchanter }{Whenever you undertake an enchanting effort\comma{} you gain one additional auto\minus{}success. You may also learn new \imp{Runes} with only 1 hour of study.}{0}{}
	\feat{Hidden Work }{When you complete an \imp{Imbuing} or \imp{Crafting} project\comma{} you
may expend an additional hour to make your work completely hidden from inspection.
Runes are hidden and alchemical creations can appear as mundane
fluids. Only upon activation or a spell such as Identify can the true
nature be divined.}{0}{}
	\feat{Idiosyncrasies }{You know every oddity and quirk of your own creations: when using them\comma{} you gain one additional auto\minus{}success.}{0}{}
	\feat{Master Mechanist }{Whenever you undertake a tinkering or mechanical manufacturing effort\comma{} you gain one additional auto\minus{}success. You may also stray further from `realistic’ or scientific constructions and may handwave slightly more vigorously over the workings of your constructions.}{0}{}
	\feat{Quick Worker }{You take only half the normal time to perform feats of \imp{Crafting} and \imp{Imbuing} – you may perform checks every 3 hours\comma{} rather than every 6.}{0}{}
	\feat{Siege Master }{When dealing damage to or attempting to bypass a building\comma{} structure\comma{} wall\comma{} door or other such solid object\comma{} you gain one additional auto\minus{}success and deal an additional level of harm.}{0}{}
	\feat{Thick Skin }{Years of accidents and lab mishaps have left you with a superhuman level of resilience: choose from \imp{Fire} and \imp{Concussive} or \imp{Acid} and \imp{Poison} damage: any rolls to \imp{Resist} damage of the chosen types has a DV that is 2 lower than normal.}{0}{}
	\feat{Wandmaker }{You have perfected the art of crafting magical focusses\comma{} and can create a new wand for yourself or others. This takes up to 3 days\comma{} but you can craft the new `wand’ into any form you like – a mighty oaken staff\comma{} or a bejewelled necklace\comma{} but must remain an item in their possession which the spellcaster can focus on when using their magic.}{0}{}
}

\def\WarriorFeats
{
	\feat{All Guns Blazing }{When making an attack against a small group of people\comma{} you can truly throw yourself into the attack\comma{} expending two \imp{Fortitude} points to take them all on at once. Make attacks against a number of beings (up to twice your \imp{Rage} score) within range. You cannot use this ability on consecutive turns.}{0}{}
	\feat{Blind Rage }{When using a \imp{Rage} action to attack\comma{} you ignore all dice penalities due to injuries.}{0}{}
	\feat{Bloodlust }{On any turn in which you successfully incapacitate (lethally or not) a foe\comma{} you may take an additional free action to perform another attack\comma{} with a 2\minus{}dice penalty on the check.}{0}{}
	\feat{Duelist }{When fighting against a single foe\comma{} you gain one auto\minus{}success on all attack rolls and Resist actions.}{0}{}
	\feat{Furious Spellcaster }{Up to three times a day\comma{} you may elect to use a \imp{Rage} check\comma{} rather than the associated \imp{Affinity} to perform a spellcasting check. The maximum spell level is still determined by your \imp{Affinity}.}{0}{}
	\feat{Holistic Tactics }{Whenever a \imp{Tactics} check is successful in determining the tactics of a foe\comma{} you may use this knowledge to infer any Resistances\comma{} Immunities or Susceptabilities the target has.}{0}{}
	\feat{Lightning Strikes }{At the end of each long rest\comma{} perform a DV 7 check using just your \imp{Speed} pool. For each success gained (min 1) you may perform one additional attack at some point over the next day without expending a \imp{Fortitude} point.}{0}{}
	\feat{Savage Attacker }{Your attacks\comma{} both physical and magical\comma{} deal one additional point of \imp{Harm}.}{0}{}
	\feat{Student of War }{If you study\comma{} read up on and otherwise prepare for a target before engaging them in combat\comma{} the DV of all attacks you make against them is reduced by 1.}{0}{}
}

\def\AurorFeats
{
	\feat{Ambush }{When you attack from hiding\comma{} spring a trap or successfully orchestrate
an ambush\comma{} your first attack is particularly powerful: when you complete your first attack roll\comma{} if the total number of successes is below  half the number of dice rolled\comma{} you may instead use that number.}{0}{}
	\feat{Cold Cases }{When performing a Knowledge check\comma{} if you can relate the information
you seek to a historical or past case you reduce the DV by 3.}{0}{}
	\feat{De\minus{}escalation Training }{You are trained specifically to capture and contain\comma{} not to kill. When you take an action to contain\comma{} constrain\comma{} bind\comma{} trap or disarm a foe\comma{} rather than inflict pain or damage upon them\comma{} you gain +1d to the effort.}{0}{}
	\feat{Familiar Terrain }{Choose a favoured terrain such as \imp{Grasslands}\comma{} \imp{Forests}\comma{} \imp{Urban Areas}\comma{} \imp{Caverns}\comma{} or name a specific region\comma{} such as \imp{Hogwarts}. Whilst in your favoured terrain you gain +1d on every action which utilises the surroundings such as a \imp{Tracking} or \imp{Covert} check.}{0}{}
	\feat{Fancy Footwork }{When fighting more than one foe\comma{} you may use an action to expend a \imp{Fortitude} point to confuse your foes with some feat of athletics and maneouvering\comma{} causing them to attack each other. Nominate two enemies within range – next turn cycle\comma{} the first of these two to take an action will attack the other\comma{} instead of their intended target.}{0}{}
	\feat{Lie Detector }{You can automatically detect when someone is lying to you by telling you deliberate falsehoods}{0}{}
	\feat{Mental Training }{You have trained your mind to resist the effects of external manipulation. You gain +3d against all checks to resist unnatural mental manipulation\comma{} and may expend a \imp{Fortitude} point to end an ongoing mental effect such as \imp{Charmed}.}{0}{}
	\feat{Rapid Reflexes }{When performing a \key{Reflex} roll\comma{} you may roll the dice twice and take the largest value.}{0}{}
	\feat{Unwavering Focus }{Once per day you may expend a \imp{Fortitude} point to reroll all \imp{Catastrophe} dice you rolled\comma{} declaring this action after the roll has been performed\comma{} but before the outcome has been narrated.}{0}{}
}

\def\AllFeats
{
	\feat{Armour Piercing }{When a target attempts to \imp{quickblock} your attacks\comma{} their armour takes two levels of \imp{Drain}.}{0}{}
	\feat{Attuned Attacks }{When fighting unarmed\comma{} or with a non\minus{}magical weapon\comma{} you can channel your very life\minus{}force into your attacks. Such attacks deal an additional level harm\comma{} and bypass any resistance to physical attacks.}{0}{}
	\feat{Course 101 }{You study a crash course in a selection of 10 abilities you previously had no skill in\comma{} giving you a basic level of knowledge. Choose up to 10 \imp{Abilities} with a \emptyCape{} rating\comma{} and gain 1 dot in each of them. If\comma{} when you take this ability\comma{} it costs more than 9 dots\comma{} pay only 9 dots.}{0}{}
	\feat{Elemental Attunement }{You feel a particular affinity for one of the elements (Fire\comma{} Water\comma{} Ice\comma{} Earth\comma{} Air\comma{} Lightning\comma{} etc.) deep within your bones. When casting a spell to manipulate\comma{} create or otherwise effect your chosen element\comma{} you gain +1d. You also gain +1d to any check to resist damage caused by your element.}{1}{\imp{Elemental} (\threeCape{})}
	\feat{Helping Hand }{You are so proficient in helping out your allies that your `help’ action gives +3d\comma{} rather than +1.}{0}{}
	\feat{Innate Trick }{As a witch or wizard\comma{} the chaotic force of magic flows within your veins. You have learned to harness this magic in some innate way beyond the usual spellcasting. This effect is usually minor (something a Muggle could put down to an act of trickery or showmanship)\comma{} and often forms the basis of a parlour trick. 

You might be able to summon a small flame from your finger\comma{} make your eyes into burning coals or deep black voids\comma{} play a stirring soundtrack whenever they engage in a fight\comma{} know the name of every individual you meet\comma{} or some other marvellous but ultimately slightly inconsequential feat that you could imagine being the focus of conversation at a party. 

No rolls are needed to use this ability\comma{} and the GM has a veto if this tool is being used in an inappropriate fashion.}{0}{}
	\feat{Jack\minus{}of\minus{}all\minus{}Trades }{You have a surprising amount of miscellaneous skills\comma{} knowledge and abilities that you have acquired over your life\comma{} and are often able to surprise your allies with something pulled from your sleeve. 

Each day\comma{} you get 4 free dots\comma{} which you may temporarily allocate to \imp{Abilities} as and when you need them\comma{} though you may not increase any ability to more than 5 dots. You may use this enhanced ability for the next hour\comma{} before the effect wears off. You regain your dots when you complete a \imp{Long Rest}.}{0}{}
	\feat{Light Sleeper }{You need much less sleep than others\comma{} and can go from asleep to awake in a blink of an eye. You gain the benefits of a \imp{Long Rest} after only 4 hours\comma{} and may ignore the effects of \imp{Level One Exhaustion}. Any \imp{Alertness} checks called for whilst asleep have the DV reduced by 3.}{0}{}
	\feat{Linguist }{You have studied another language enough to be considered fluent in it. When conversing in this language\comma{} gain +2d to all social checks. You may take this ability multiple times\comma{} learning a new language each time.}{0}{}
	\feat{Loyal Companion }{You have an animal ally which is eternally loyal and devoted to you\comma{} and can carry out simple tasks: a `familiar’. The most common animals are owls\comma{} ravens\comma{} cats\comma{} rats and toads\comma{} though you may ask your GM for a different choice.}{1}{Kinship (\twoCape)}
	\feat{Martial Arts }{You are a master of unarmed combat\comma{} making your hands into lethal weapons. Unarmed strikes deal damage equal to the number of successes (DV 6). You may expend a Fortitude point to reduce the DV of an unarmed strike to 3.}{1}{Brawl (\threeCape)}
	\feat{Moving Target }{On a turn during which you move more than half your movement (without doubling back)\comma{} your \imp{quickdodge} checks do not incur \imp{Drain}.}{0}{}
	\feat{Numbed to Pain }{When you expend a \imp{Fortitude} point to ignore the negative effects of \imp{Harm}\comma{} the effect lasts for one hour\comma{} rather than just the next round.}{1}{Vitality (\threeCape)}
	\feat{Psychic Awareness }{Your mind is especially attuned to those of others\comma{} and you can naturally sense the shift induced when a psychic power alters or interacts with minds. Whenever a psychic effect such as mind reading\comma{} memory modification\comma{} or magic which alters emotions and allegiances is used on a target within 5m of you\comma{} you are automatically aware of this\comma{} though you are not aware of the source.}{1}{\imp{Kindness} (\threeCape{})}
	\feat{Ritualist }{You are a strong believer that the most powerful magic is performed with large groups\comma{} in elaborate rituals\comma{} with chanting\comma{} incense and possibly a pentagram or two. Whenever you invoke a \imp{Ritual} to cast a spell\comma{} you gain one automatic success for every 3 members of the ritual (max +5d).}{1}{\imp{Occultism} (\threeCape{})}
	\feat{Second Chances }{Once per day\comma{} you may re\minus{}roll any number of dice on a single check\comma{} but must keep the new result.}{0}{}
	\feat{Signature Spell }{You have a spell which is considered your `signature move’\comma{} chosen when you take this feat. When casting this spell\comma{} you gain one additional auto\minus{}success. You may change your `signature spell’ only with GM consent that your old choice no longer represents your character’s go\minus{}to move.}{0}{}
	\feat{Silent Casting }{You do not need to perform the verbal component of a spellcasting action. Efforts to silence you do not impact your spellcasting efforts\comma{} and reactions to your spells take a 1d penalty.}{0}{}
	\feat{Wandless Casting }{You are able to perform limited feats of magic without needing the crutch of a wand or ritualistic movements\comma{} so attempts to disarm your or bind you in place do not affect your spellcasting efforts. You take a 1d penalty on all wandless spellcasting efforts. All wandless actions are also silent.}{1}{Silent Casting}
}

\def\DruidFeats
{
	\feat{Asteria’s Eyes }{When you cast a spell from the \imp{Divination} school of magic (\imp{Cerebral} or \imp{Temporal})\comma{} if you can see the stars\comma{} you gain +2d to the check.}{0}{}
	\feat{Cloak of Seasons }{You are magically protected from the effects of the weather and the natural environment. You are perfectly comfortable in winter's chill or summer's blazing heat regardless of your clothing (or lack thereof). You do not suffer from sunstroke or exposure. You're not even bitten by insects or other vermin. Your senses are still limited by the elements (including fog\comma{} rain and snow)\comma{} and you're not protected from either hunger or thirst.}{0}{}
	\feat{Dryad’s Embrace }{When you cast a spell on or attempt to \imp{Commune} with plant\minus{}based beings\comma{} or attempt to use a \imp{Knowledge} check to learn about such an entity\comma{} you gain one additional auto\minus{}success.}{0}{}
	\feat{Exuding Aura }{You may expend a \imp{Fortitude} point to attune yourself to your favoured aspect of nature\comma{} exuding an aura which influences the minds of others – perhaps a sweet pine smell calms them\comma{} or animal pheremones send them into a frenzy. You gain +1d on all \imp{Social} checks made against people within 2m of you for the next hour.}{0}{}
	\feat{Green Thumb }{If you so choose\comma{} you can become a beacon of vibrant plant life. Flowers spring up in your footsteps and trees burst into bloom at your touch. Your hands are always warm and comforting\comma{} and plants will avoid hurting you\comma{} blunting their thorns\comma{} or dulling their poison as you pass. This effect is somewhat limited (you cannot heal a field of necromantic blight\comma{} for example\comma{} and plants may retaliate if under sustained injury)\comma{} but plants will recognise you as a source of light and life.}{0}{}
	\feat{Nymph’s Fury }{Channeling the power of primal\comma{} elemental spirits grants you additional power. When casting a spell from the \imp{Elemental} discipline\comma{} you gain one additional auto\minus{}success.}{0}{}
	\feat{Organic Repose }{Once per day\comma{} you may expend a \imp{Fortitude} point to recover 3 levels of Health}{0}{}
	\feat{Satyr Spirit }{When casting a spell or attempt to \imp{Commune} with a \imp{Beast}\comma{} or attempt to use a \imp{Knowledge} check to learn about such a creature\comma{} you gain one additional auto\minus{}success.}{0}{}
}

\def\OutlawFeats
{
	\feat{Black Market }{You know just where to acquire forbidden items\comma{} and source unscrupulous materials\comma{} and your experience with dealing with such people grants you +1d on all bartering checks.}{0}{}
	\feat{Cover Identity }{Given enough time – perhaps a day or two – you can forge yourself a completely new identity\comma{} with the necessary paperwork and credentials to reasonably pass as whoever you desire. This may not stand up to high\minus{}level scrutiny\comma{} but most people should be easily fooled.}{0}{}
	\feat{Hidden Weapon }{Up to three times a day\comma{} you can draw a previously unknown small blade from a fold in your robes\comma{} or a hidden pocket\comma{} and then use it. This is an instantaneous action.}{0}{}
	\feat{Innocent Face }{You are always thought of as an honest and good soul. If you do something wrong which isn’t immediately attributed to you\comma{} it will most likely be blamed on something else. As long as you’re not caught red\minus{}handed killing puppies\comma{} people will try to excuse your actions and move on from your misdeeds.}{0}{}
	\feat{Move in Shadow }{Whilst outside of bright light\comma{} all attempts to percieve you have a DV 2 higher than normal.}{0}{}
	\feat{Naturally Shifty }{Doing unscrupulous deeds comes as naturally to you as breathing – gain one additional auto\minus{}success on any \imp{Covert} action.}{0}{}
	\feat{Sly Action }{At the end of a turn cycle\comma{} if you have not been directly targeted for an attack\comma{} you may take an additional action at the end of the cycle to move\comma{} use an item\comma{} or otherwise attempt to hide}{0}{}
	\feat{Surprise Attack }{Whenever you attack a target from a position where they cannot see you\comma{} you deal an additional level of harm}{0}{}
	\feat{Unobtrusive }{You don’t stand out in a crowd\comma{} and can make yourself\comma{} if not invisible\comma{} just socially {\it absent}. People don’t necessarily remember your face or your name (if they remember you at all)\comma{} as you make very little impression on people – until you are stealing their wallet\comma{} or knifing them in the back\comma{} that is. All checks made by enemies to notice you in a crowd\comma{} or remember details about you have a DV 3 higher than normal.}{0}{}
}

\def\SophisticateFeats
{
	\feat{Brazen }{You are so brash and bold in your approach that you can simply breeze past an error\comma{} playing it for laughs\comma{} or simply ignoring it altogether. Whenever you perform a \imp{Social} check\comma{} you may treat one \imp{Catastrophe} as a normal\comma{} benign failure.}{0}{}
	\feat{Burn Book }{When using your skills to spread rumours\comma{} misinformation or gossip\comma{} or when trying to discredit an individual\comma{} gain +2 dice}{0}{}
	\feat{Mesmerising Presence }{Once per day\comma{} you may use your alluring charm to slightly hypnotise a person\comma{} gaining +1d to social checks with them. They remember talking to you\comma{}  but are slightly starstruck and overpowered by your personality\comma{} such that they cannot recall what you talked about.}{0}{}
	\feat{Natural Leader }{You're a natural born leader. While not everyone will simply surrender authority to you\comma{} they'll consent to "follow your lead." Reduce the DV of any check directly related to leadership by 3.}{0}{}
	\feat{One for you\comma{} two for me }{Whenever your or your allies gain an amount of \imp{Galleons}\comma{} you gain one additional coin.}{0}{}
	\feat{Poker Face }{You are an expert at hiding your true feelings – beings cannot rely on the usual cues to read your demeanour or true|motivation. Any \imp{Insight} attempts against you have a DV 2 higher than normal.}{0}{}
	\feat{Sue For Peace }{You excel at halting violence when it breaks out. Whenever you \imp{Surrender}\comma{} it is automatically accepted. You may also expend a \imp{Fortitude} point to try to convince your foe to surrender to you – force your opponents to perform a resist check such as \imp{Willpower (Bravery)} with the DV set by the value of the dice pool you would use to sue for peace \minus{} (\imp{Charm (Eloquence)} for a rational argument\comma{} or perhaps \imp{Willpower (Intimidation)} to cow them into submission). On a failure\comma{} they will \imp{Surrender} or \imp{Flee}.}{0}{}
	\feat{Unbreakable Vow }{When you willingly shake on a deal or contract with another sapient being both you and your partner are bound together by a magical oath. If either of you breaks the contract\comma{} the offending party takes the maximum amount of \imp{Harm} and falls into a \imp{Critical Condition}\comma{} alerting the other.}{0}{}
}

\def\ScholarFeats
{
	\feat{Expertise }{Choose a spell discipline\comma{} or a viable target of a spell\comma{} which is associated with your area of research or specialty. When casting a spell of this school\comma{} or a spell on your chosen target\comma{} you gain one additional auto\minus{}success.}{0}{}
	\feat{Healer }{Whenever you restore Health to a being\comma{} heal an additional level of \imp{Harm}.}{0}{}
	\feat{Lightning Mind }{You can perform complex mathematical calculations at the speed of a computer\comma{} even when under life\minus{}threatening stress. The player controlling this character may use a calculator and refer to the statistical tables to evaluate the outcome of an action at any time\comma{} even when in\minus{}game time is short.}{0}{}
	\feat{Master of the Mind }{When an action would interrupt you casting a spell requiring continuing concentration\comma{} the DV to remain focussed is 2 lower than normal.}{0}{}
	\feat{Novel Technique }{Whenever you use a spell in a new and novel fashion\comma{} you gain +2d for the spellcasting effort.}{0}{}
	\feat{Quick Learner }{You need to spend half the usual time in order to learn a new spell or potion recipe.}{0}{}
	\feat{Well Read }{You hold in your brain a simply incredible amount of information\comma{} which leads to sudden flashes of inspiration and insight. Once per \imp{Long Rest}\comma{} you may use one of these flashes to gain +5d on one \imp{Knowledge} check.}{0}{}
}


