%%% definitions
\newcounter{itemlist}
\newcommand\itemlist{\stepcounter{itemlist} \theitemlist)}


\chapter{Creating A Character}

The first step in playing the game is to create your own character. Your character can be whatever or whoever you want it to be -- this is your story after all. 

Your character is manifest in the game through your imagination, but in order to quantify the events occurng in the story, a character is formed from a mixture of several ingredients, from which we can generate statistics and determine how proficient a character is at various actions. 

Before you begin, it is helpful if you have an idea of the kind of character you wish to create -- your GM should tell you the rough outlines of the setting, which should help guide the type of character that will work well in the story. Do you want to play a powerfully destructive mage bent on crushing their enemies; or an investigator, pursuing the truth behind a mystery? 

You should also think about the backstory of your character -- what has led them to this point in their lives? Why are they going on this adventure?  

Once you have a good idea of the kind of character you wish to create, follow these steps to generate you character, and record the results on the Character Sheet.

\subsubsection*{\itemlist{} Choose a (sub)Species}

Every character belongs to one of the Sapient races present in this world -- be they a human, a goblin, or a centaur. Some of the species (notably the humans) have several "sub-species" which take into account variation within the species. 

Belonging to a species confers your most basic characteristics: what do you look like? What magics -- if any -- do you have access to? 

Some species will also find themselves having a natural aptitude for certain skills, so it can be useful to think about how best to pair up your species and archetypes. The species available, and the abilities that they possess are discussed in Chapter \ref{C:Species}

You should also take into account your setting whilst making this decision: Wizarding society is, unfortunately, not the most accepting of other sapient races, so a game which takes primarily in Hogwarts would necessitate all characters being as close to fully-human as possible.  

\subsubsection*{\itemlist{} Choose your Personality}

Every character has a unique personality, the combination of qualities that defines them as a social being. You must decide on what kind of person your character will be, and what actions they must take in order to soothe their soul. 

This is also the point where, if you are a Hogwarts student, you will decide which House you will be sorted into, based on the personality you have chosen. More information about personalities can be found in chapter \ref{C:Personality}, starting on page \pageref{C:Personality}.

\subsubsection*{\itemlist{} Choose an Archetype}

An archetype broadly defines what your character does for a living -- but it is also much more than that. The archetype defines what role your character plays in the story, how they perceieve and interact with others and (perhaps more importantly) what skills they can develop as they progress. 

Your character recieves new skills and abilities by virtue of their archetype, so look ahead and see which skills you think will be the most useful (or, the most fun!) to develop along with your character. Archetypes are discussed in detail in Chapter \ref{C:Archetype}, starting on page \pageref{C:Archetype}.

\subsubsection*{\itemlist{} Allocate Capabilities}

Every character is either strong or weak, on a varying scale, across a number of fields governing potential actions: 
\begin{enumerate}[itemsep=0em]
\item \key{Aspects}, fundamental skills which form the basis of every action,
\item \key{Abilites}, cultivated and learned talents which give them a proficiency in a more narrow field
\item \key{Affinities}, their ability to cast certain types of spells. 
\end{enumerate}

Every action is assigned a number of die, usually represented as \key{Dots}, or simply as numbers. These dots/numbers encode how many dice are rolled when a check is required. A character's class and archetype will provide a base level of abilities in these area, but all characters then get a choice of how to allocate some additional points:


\begin{enumerate}
\item \key{Aspects}, start with 1 in each field, allocate an additional 8 dots freely (max 4)
\item \key{Abilites}, choose 5 major skills in each category. Then rank the categories in order, distributing 10 / 5 / 3 dots to each (max 4)  
\item \key{Affinities}, choose 2 disciplines to get 2 dot rating in, then 5 further to get 1 dot rating
\end{enumerate} 

A low score in a given attribute will have a long-term effects on your character's abilities (though they can develop with time), so think carefully about how your abilities mesh with your character's personality and archetype. A particularly shy character, you might decide, will not be very brave, and thus will have a low Willpower. 

Attributes are discussed in more detail in Chapter \ref{C:Aspects}, starting on page \pageref{C:Aspects}.

\subsubsection*{\itemlist{} Gather Your Equipment}

Your character will probably gain some supplies by virtue of their archetype, but you will also acquire some cash, as well as perhaps the most important item in your inventory: your wand. The item system is presented in Part III.

\subsubsection*{\itemlist{} Go adventuring!}

At this point, you will hopefully have a fully formed character, possibly working within a party of other characters. 

You will now be ready to set of on your adventure!
