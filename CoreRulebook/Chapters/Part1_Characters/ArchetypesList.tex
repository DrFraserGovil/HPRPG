
\newcommand\ability[4]
{
	\subsubsection{\imp{#1 Ability: #2}}
	
	#3

	%#4

}

\newcommand\bname
{
	\imp{\name}
}



\newcommand\bonus[2]
{
	\imp{#1}	&	#2 \\
}

\makeatletter
\define@key{archetype}{name}{\def\name{#1}}
\define@key{archetype}{article}{\def\article{#1}}
\define@key{archetype}{bonuses}{\def\bonuses{#1}}
\define@key{archetype}{experience}{\def\experience{#1}}
\define@key{archetype}{feats}{\def\feats{#1}}
\define@key{archetype}{bonusWidth}{\def\bonusWidth{#1}}
\define@key{archetype}{description}{\def\description{#1}}
\define@key{archetype}{innateAbility}{\def\innateAbility{#1}}
\define@key{archetype}{innateDescription}{\def\innateDescription{#1}}
\define@key{archetype}{innateNil}{\def\innateNil{#1}}
\define@key{archetype}{innateI}{\def\innateI{#1}}
\define@key{archetype}{innateII}{\def\innateII{#1}}
\define@key{archetype}{innateIII}{\def\innateIII{#1}}
\define@key{archetype}{innateIV}{\def\innateIV{#1}}
\define@key{archetype}{innateV}{\def\innateV{#1}}
\define@key{archetype}{innateVI}{\def\innateVI{#1}}
\define@key{archetype}{innateVII}{\def\innateVII{#1}}
\define@key{archetype}{knowledgeAbility}{\def\knowledgeAbility{#1}}
\define@key{archetype}{knowledgeDescription}{\def\knowledgeDescription{#1}}
\define@key{archetype}{knowledgeNil}{\def\knowledgeNil{#1}}
\define@key{archetype}{knowledgeI}{\def\knowledgeI{#1}}
\define@key{archetype}{knowledgeII}{\def\knowledgeII{#1}}
\define@key{archetype}{knowledgeIII}{\def\knowledgeIII{#1}}
\define@key{archetype}{knowledgeIV}{\def\knowledgeIV{#1}}
\define@key{archetype}{knowledgeV}{\def\knowledgeV{#1}}
\define@key{archetype}{knowledgeVI}{\def\knowledgeVI{#1}}
\define@key{archetype}{knowledgeVII}{\def\knowledgeVII{#1}}
\define@key{archetype}{practicalAbility}{\def\practicalAbility{#1}}
\define@key{archetype}{practicalDescription}{\def\practicalDescription{#1}}
\define@key{archetype}{practicalNil}{\def\practicalNil{#1}}
\define@key{archetype}{practicalI}{\def\practicalI{#1}}
\define@key{archetype}{practicalII}{\def\practicalII{#1}}
\define@key{archetype}{practicalIII}{\def\practicalIII{#1}}
\define@key{archetype}{practicalIV}{\def\practicalIV{#1}}
\define@key{archetype}{practicalV}{\def\practicalV{#1}}
\define@key{archetype}{practicalVI}{\def\practicalVI{#1}}
\define@key{archetype}{practicalVII}{\def\practicalVII{#1}}
\makeatother

\newcommand\archetype[1]
{
	\begingroup
	\setkeys{archetype}{name= None,article = A, bonuses = , description= None,innateAbility= None,innateDescription= None,innateNil= None,innateI= None,innateII= None,innateIII= None,innateIV= None,innateV= None,innateVI= None,innateVII= None,knowledgeAbility= None,knowledgeDescription= None,knowledgeNil= None,knowledgeI= None,knowledgeII= None,knowledgeIII= None,knowledgeIV= None,knowledgeV= None,knowledgeVI= None,knowledgeVII= None,practicalAbility= None,practicalDescription= None,practicalNil= None,practicalI= None,practicalII= None,practicalIII= None,practicalIV= None,practicalV= None,practicalVI= None,practicalVII= None,bonusWidth = 3, feats = , experience = \item Do something } 

	\setkeys{archetype}{#1}
	
	
	\chapter*{\name}
	\index{Archetype!\name}
	\index{Feats!Archetype Feats!\name{} Feats}
	\addcontentsline{toc}{section}{\name}
	
	\small 
	\description
	
	\section*{\imp{\name{}} {Capabilities} }
	
	\article{} \imp{\name} gets the following bonuses to their \imp{Aspects}, \imp{Abilities} and \imp{Affinities}. Where a choice is given, you cannot make the same choice twice. Note that these are {\it bonuses} on top of those granted by other abilities and natural starting values. 
	
	\begin{center}
	\begin{rndtable}{r c}
		\bf Capability	&	\bf Bonus Rating \\
		\bonuses
	\end{rndtable}
	\end{center}
	
	
	
	%\section*{\name{} Experience}
	\index{Progression!Additional Triggers}
	\article{} \imp{\name} gains additional experience when they:
	\begin{itemize}[itemsep = 0cm]
		\experience
	\end{itemize}
	
	\section*{\name{} Special \imp{Abilities}}
	\index{Abilities!Archetype Abilities}
	A character following the path of the \imp{\name{}} can use the following special abilities: \key{\innateAbility}, \key{\practicalAbility} and \key{\knowledgeAbility}. At Character Creation each of these skills has a rating of one, with a further three dots distributed amongst them at your design.
	
	\ability{Innate}{\innateAbility}{\innateDescription}{\ratingTable{\innateNil}{\innateI}{\innateII}{\innateIII}{\innateIV}{\innateV}{\innateVI}{\innateVII}}
	\ability{Practical}{\practicalAbility}{\practicalDescription}{\ratingTable{\practicalNil}{\practicalI}{\practicalII}{\practicalIII}{\practicalIV}{\practicalV}{\practicalVI}{\practicalVII}}
	\ability{Knowledge}{\knowledgeAbility}{\knowledgeDescription}{\ratingTable{\knowledgeNil}{\knowledgeI}{\knowledgeII}{\knowledgeIII}{\knowledgeIV}{\knowledgeV}{\knowledgeVI}{\knowledgeVII}}
	
	
	\section*{\imp{\name} Feats}
	
	\article{} \bname{} choose to take some of the following feats as they increase their abilities:
	
	\feats
	\endgroup
	
}


\def\auror{\imp{Auror}}


%%Begin
\archetype
{
	name = Artificer,
	description = {\bname{}}s are those individuals who revel in the act of creation\comma{} in making things more than the sum of their parts. 

Coming in many varieties – from the magic\minus{}item driven \imp{Enchanters} to the potion\minus{}mixing \imp{Alchemists} and even the more muggle\minus{}minded \imp{Technicians}\comma{} an \imp{Artificer} is both creative and technical\comma{} addressing challenges with a little something they prepared earlier. 

Many \imp{Artificers} seal themselves away in their laboratories\comma{} only surfacing for air when absolutely necessary – but it is not unusual for some to take a more libearl appraoch to `field testing’ their equipment. What’s the point of creating the ultimate magical weapon if you don’t get to shoot it at things?,
	experience = \item Solve a problem using technical knowledge\comma{} innovation or one of their creations
\item Create something new and impressive,
	bonuses = \bonus{Imbue or Craft}{\twoCape}
\bonus{Arcane or Technology}{\twoCape}
\bonus{Investigation}{\twoCape}
\bonus{Logic}{\oneCape},
	innateAbility = Danger,
	innateDescription = As experimentalists\comma{} most often using themselves as their test subjects\comma{} many {\bname{}}s have had to develop a sixth sense for an incoming explosion when things are suddenly going south. 

This ability allows artificers to sense that something very bad is about to happen – be it the minute ticking of a trap mechanism\comma{} the whiff of the acidic poition behind a hidden compartment\comma{} and allows them to get more advanced warning than their less experienced compatriots.,
	practicalAbility = Modify,
	practicalDescription = The ability to take something and quickly hack it in order to make it into something else\comma{} or extend its current capabilities\comma{} is a hallmark of the \bname{}’s craft. 

From small aesthetic jobs\comma{} such as changing the colour of an alchemical solution from red to blue\comma{} to (at the highest possible levels)\comma{} re\minus{}enchanting an already imbued artefact\comma{} the ability to \imp{Modify} the properties of an object is hugely valuable.,
	knowledgeAbility = Analyse,
	knowledgeDescription = Artificers are adept at discerning the function of objects\comma{} both magical and mundane\comma{} which is reflected in the \imp{analyse} ability. 

This ability may be used in order to determine the purpose or properties of a target for the maximum effect outside of an \imp{Identify} spell.,
	article = An,
	bonusWidth = 3, feats = \ArtificerFeats
}

\archetype
{
	name = Auror,
	description = As a profession\comma{} the \auror{}s are a group of highly\minus{}trained law enforcement officials working for the \imp{Ministry of Magic}\comma{} as well as a catchall term for those dedicated to catching bad guys and making them pay.

\imp{Aurors} (or even those who merely wish to emulate them) seek out their target with a single minded zeal\comma{} dedicated to the cause of finding the truth and bringing villains to justice. They adore solving mysteries and puzzles\comma{} and abhore those who would bring harm to others. 

Their pursuit of justice often puts them in harm's way\comma{} and so the budding \auror{} is encouraged to focus on magic which allows them to protect themselves from harm\comma{} as well as incapacitate their foes. 

However\comma{} the defining trait of an \auror{} is not their combat abilities but instead their ability to discover clues\comma{} intuit motives and hunt down their foes.,
	experience = \item Solve a problem using investigation\comma{} tracking and detective work
\item Track down or hunt an elusive target or solve a significant mystery,
	bonuses = \bonus{Insight}{\twoCape}
\bonus{Investigation}{\twoCape}
\bonus{Perception}{\oneCape}
\bonus{Brawl}{\oneCape}
\bonus{Survival}{\oneCape},
	innateAbility = Intuition,
	innateDescription = \imp{\innateAbility} is the inherent\comma{} instinctive understanding of the minds of others possessed by an insightful and trained mind. Bypassing all \imp{Logic} and conscious reasoning\comma{} \imp{intuition} allows an \name{} to make great strides in their understanding of people and their actions by getting inside their heads and understanding the way that they think. Though not useful for solving traditional intellectual puzzles\comma{} \imp{\innateAbility} can allow an \bname{} to suddenly have a flash of insight into the motives\comma{} aims or drive of another being. 

If you wish to know why someone would behave in a given way\comma{} why a certain shop was robbed and not another\comma{} or where a target might head next \minus{} an \name{}'s \imp{\innateAbility} is surely the best tool,
	practicalAbility = Interrogate,
	practicalDescription = The art of extracting information out of a target\comma{} either unwilling to divulge or unaware they're being questioned\comma{} is a key skill for an \bname{} to master.   Whilst the untrained would have to rely on raw \imp{Charm}\comma{} \imp{Eloquence}\comma{} \imp{Deception} or even \imp{Intimidation} to try and convince them to give up their information\comma{} the skill of \imp{Interrogation} allows you to dance delicately between all of these skills\comma{} executing known psychological tricks and even shrouding your true questions behind layers of misdirection so your target does not even know when they're giving up valuable information.,
	knowledgeAbility = Tracking,
	knowledgeDescription = Hunting down a foe is a key part of being an \bname{}\comma{} and part of that is being able to survey a scene and see where they were\comma{} what they did\comma{} and where they're going next.

Whilst \imp{Intuition} relies on a general understanding of the target's thought pattern\comma{} when you \imp{Track} a target you look for the trail that they have left \minus{} scuffs in the dirt\comma{} broken twigs in the forest and even more abstract trails such as an online presence or a paper trail. Whatever evidence you need to find your target\comma{} \imp{Tracking} can help you out.,
	article = An,
	bonusWidth = 3, feats = \AurorFeats
}

\archetype
{
	name = Druid,
	description = A {\bname{}} is a witch or wizard who has dedicated their life to the understanding\comma{} protection and preservation of the natural order of things. 

From the smallest fungus\comma{} to the most vicious of dragons\comma{} as well as the very bones of the Earth\comma{} and the stars in the sky – all {\bname{}}s feel a deep connection to them all. From this connection to nature\comma{} the \bname{}s draw their powers their understanding of all forms of magic is shaped into how it affects and relates to nature.

In the popular mythology of \bname{}s\comma{} even in the Wizarding world\comma{} they are seen as eccentric pacifists\comma{} a pushover afraid to even hurt a fly. However\comma{} a true \bname{} understands that death and destruction are a part of the every day cycle of nature. If an old tree must burn so that a dozen new flowers may blossom\comma{} a \imp{Druid} is often more than happy to oblige.,
	experience = \item Solve a problem by using their connection to nature
\item Protect some aspect of nature from signifiant harm,
	bonuses = \bonus{Willpower}{\twoCape}
\bonus{Kinship}{\twoCape}
\bonus{Nature}{\twoCape}
\bonus{Insight}{\oneCape},
	innateAbility = Belonging,
	innateDescription = A \bname{} with a high innate sense of \imp{Belonging} has an intuitive\comma{} almost supernatural ability to determine when the natural\comma{} organic\comma{} order of things has been disrupted or influenced.

By looking at a lone tree in an underground cave\comma{} such a character may attempt to discover if it \imp{Belongs} here\comma{} simply growing naturally\comma{} or if it was placed there and forced to grow by other means\comma{} or if a pack of dogs attacked out of natural instinct\comma{} or trained instructions. 

The sense of \imp{Belonging} is not an exact art\comma{} but merely gives a \bname{} an additional insight into disruptions and alterations of nature.,
	practicalAbility = Nurture,
	practicalDescription = The ability to nurture life\comma{} in all its forms\comma{} is critical to the role of a \bname{}. A high \imp{Nuture} score allows a \bname{} to care for plants\comma{} animals and nature in general\comma{} providing them with the support\comma{} nutrition and guidance they need.

Less useful on humans (\imp{Kindness} is probably more useful)\comma{} a successful \imp{Nurture} check ensures that life will continue and thrive where you set your mind to it. Those that you successfully \imp{Nurture} will owe you gratitude and become positive towards you.,
	knowledgeAbility = Commune,
	knowledgeDescription = It is said that\comma{} in ages past\comma{} the \bname{}s could speak to the winds\comma{} the trees\comma{} the beasts and even the stars themselves to seek answers. A \imp{Commune} check allows you to communicate – in a very rough fashion – with the natural world. You may attempt to commune with a wounded Hippogriff to learn what happened to it\comma{} or with a scorched tree to learn how the forest fire started. 

The way in which nature responds is often esoteric and open to interpretation\comma{} but a high \imp{Commune} skill represents an ability to interpret these vague signs.,
	article = A,
	bonusWidth = 4, feats = \DruidFeats
}

\archetype
{
	name = Outlaw,
	description = An \bname{} is someone who sits outside the normal constraints of the law (or \comma{} at Hogwarts\comma{} the rules laid down by teachers). Eternally at conflict between their own desires and those of society\comma{} many \bname{}s end up starting into a life of crime\comma{} putting their skills to more nefarious use. 

Others turn this defiance of law and order to become perennial tricksters\comma{} revelling in chaos and uncertainty. 

An \bname{}\comma{} whichever path they take in life\comma{} lives and dies by their preparedness and ability to surprise those would ensnare and imprison them. Those who would catch an \bname{} should prepare to have every trick in the book thrown at them.,
	experience = \item Solve a problem using trickery\comma{} deception or subterfuge
\item Further their own aims or self\minus{}interest at the expense of the law or local rules,
	bonuses = \bonus{Covert}{\threeCape}
\bonus{Precision}{\twoCape}
\bonus{Perception or Deception}{\oneCape}
\bonus{Acrobatics or Skirmish}{\oneCape},
	innateAbility = Savvy,
	innateDescription = You don’t get far in life as an \bname{} if you don’t develop a sixth sense when something is awry – is this a trap? Were those footsteps I heard? 

\imp{Savvy} is this inherent level of constant awareness (some would call it paranoia)\comma{} which allows an \bname{} to stay one step ahead of their enemies. 

A high Savvy check can be used to evade and detect problems based on a pure gut instinct that something is amiss. Whilst rarely perfect\comma{} \imp{Savvy} is an invaluable tool.,
	practicalAbility = Pickpocket,
	practicalDescription = Rifling through the pockets of an unsuspecting mark is a highly specialised skill\comma{} moreso than a simple \imp{Covert} action can normally achieve. 

As an \bname{}\comma{} you have some experience in this field however. On a successful \imp{Pickpocket} check you may use some stealthy method to distract a target whilst you quickly nab something from their very person. 

You must take care that you do this unnoticed\comma{} as people tend to get a bit antsy about theft…,
	knowledgeAbility = Underworld,
	knowledgeDescription = The local \bname{} always knows a few secrets – where to get your hands on a certain black market item\comma{} and the location of secret passages and escape routes. 

As an adult\comma{} they are probably familiar with the workings of the penal system (and how to exploit it)\comma{} whilst in Hogwarts\comma{} this knowledge can be used when questions about Detention come up.,
	article = An,
	bonusWidth = 3, feats = \OutlawFeats
}

\archetype
{
	name = Scholar,
	description = A \bname{} seeks to discover new knowledge\comma{} solve ancient mysteries and otherwise absorb as much information as possible\comma{} in order to further the totality of knowledge about all facets of the universe. 

Whilst the conventional \bname{} is most at home with a chalkboard covered in symbols\comma{} or ensconced in a dusty library\comma{} most scholars these days appreciate that information\comma{} both new and old\comma{} requires stepping outside your traditional comfort zones. You can’t exactly study the behaviour of dragons without going and poking a few dragons\comma{} after all. 

Whilst they prefer to term their adventures `field work’\comma{} there is no doubt that scholarship can sometimes be a dangerous experience. Most \bname{}s will stop at nothing\comma{} however\comma{} to help further their research. 

A \bname{} is typically a hyper\minus{}specialist\comma{} rather than a generalist\comma{} and so chooses one area of knowledge to focus specifically on\comma{} which informs much of their remaining knowledge.,
	experience = \item Use intelligence and knowledge to overcome a significant obstacle
\item Discover something new or undiscovered relating to your field of expertise.,
	bonuses = \bonus{Investigation}{\twoCape}
\bonus{Intelligence}{\twoCape}
\bonus{Any \imp{Knowledge} ability}{\twoCape}
\bonus{Willpower}{\oneCape},
	innateAbility = Stubbornness,
	innateDescription = It is often said that \imp{Intelligence} is the most important characteristic for a \imp{Scholar} to have. However\comma{} those who have spent any time banging their head against a seemingly unknowable\comma{} unsolvable problem will tell you that \imp{Stubbornness} is the only true requirement. 

Whenever repeated failures\comma{} constant letdowns and deadends would cause most others to give up and move on\comma{} a \bname{} can use their innate desire to {\it know} to make just a few more attempts.,
	practicalAbility = Collaboration,
	practicalDescription = The work of a \bname{} is often considered to be solitary\comma{} but a wise muggle once said that scholarship is `built on the shoulders of giants’ \minus{} those who wish to see further must rely on the work of others. 

A \imp{Collaboration} check allows you to work effectively with others\comma{} and boost the efficiency of the group as a whole\comma{} and as such can be substituted in a group or combined check for most other skills.,
	knowledgeAbility = Speculation,
	knowledgeDescription = A \bname{}\comma{} by their very nature\comma{} spends most of their day confronted with problems to which no one knows the solution. 

When there is no actual knowledge to be found\comma{} the only thing left to do is \imp{Speculate}. 

Speculation allows you to draw disperate threads of knowledge\comma{} in order to make conclusions about things you otherwise have no knowledge of. The risk is\comma{} of course\comma{} that you get things completely and utterly incorrect – but a scholar knows the limits of this guesswork\comma{} and with a high \imp{Speculation} can draw remarkable conclusions with only limited knowledge.,
	article = A,
	bonusWidth = 3, feats = \ScholarFeats
}

\archetype
{
	name = Sophisticate,
	description = A \bname{} is a refined person\comma{} proficient in using their wits\comma{} formenting gossip and rumours\comma{} and weaponising the force of their personality to get their way in this world – the renown of their family\comma{} or the weight of coin in their pocket is entirely incidental\comma{} of course. 

Many people believe that \name{}s are born into their suave\comma{} charm and connections to those in the upper echelons of society\comma{} but there are of those who have struggled up the ranks of class and affluence to attain the status of \bname{}. Though these people may be privately regarded as interlopers by the old\minus{}guard\comma{} they are every bit as affluent and influential. 

There are still others who are simply bluffing their way through\comma{} running the long\minus{}con and hoping nobody catches on…,
	experience = \item Overcome a  problem using rumours\comma{} gossip or skills of persuasion and deception
\item Turn an enemy into a friend\comma{} or otherwise significantly manipulate someone,
	bonuses = \bonus{Eloquence}{\threeCape}
\bonus{Charm or Deception}{\twoCape}
\bonus{Deception or Charm}{\oneCape},
	innateAbility = Wealth,
	innateDescription = \imp{Wealth} is not merely a measure of how much money you have\comma{} it is how you project an aura of wealth\comma{} confidence and belonging in high places. 

A character with a high \imp{Wealth} lives a charmed life in society – they can cruise past security in clubs and political institutions\comma{} they can gain a favour and otherwise bend those around them to their will by flashing some cash (imaginary or not). Money opens many doors\comma{} and \imp{Wealth} allows you access to that world\comma{} even without actually having to spend the coins.,
	practicalAbility = Bamboozle,
	practicalDescription = Whilst \imp{Eloquence}\comma{} \imp{Charm} and \imp{Intimidation} are undoubtedly useful skills\comma{} a \bname{} often values simply overcoming their targets with sheer force of personality. 

Whether it is through fast\minus{}talking jargon\comma{} or simply through a formiddable force of will\comma{} a character with a high \imp{Bamboozle} score cuts a swathe of confusion through a crowd.Taking and doing what they want not because they’ve earned it\comma{} but because everyone else is too shocked to really try to stop them.,
	knowledgeAbility = Society,
	knowledgeDescription = Sometimes\comma{} knowing the right people is knowing everything. 

A high \imp{Society} knowledge means that you know everyone who is anyone. You are up on all the gossip and know who is talking to who. You are aware of the feuds and alliances\comma{} as well as some of the more sordid rumours….,
	article = A,
	bonusWidth = 3, feats = \SophisticateFeats
}

\archetype
{
	name = Warrior,
	description = A \bname{} is someone who is dedicated to the martial arts\comma{} trained in the use of both physical and magical combat to defeat your foes. 

Warriors range from delicate and refined duelists\comma{} to axe\minus{}wielding maniacs. Being a warrior is more than just being handy with a weapon\comma{} however. A great warrior never goes into battle unprepared\comma{} and study and knowledge of tactics\comma{} history and the weaknesses of your foes is vital in achieving victory. 

No matter what weapons they wield\comma{} a warrior carefully balances their combat skills\comma{} their tactical knowledge and a deep\minus{}seated rage against those who would defy them.,
	experience = \item Overcome a problem with combat and martial prowess.
\item Defeat a worthy opponent in battle
Or execute a novel or interesting stratagem,
	bonuses = \bonus{Brawl\comma{} Skirmish or Marksmanship}{\twoCape}
\bonus{Fitness}{\twoCape}
\bonus{Perception}{\oneCape}
\bonus{Skirmish\comma{} Marksmanship or Brawl}{\oneCape}
\bonus{Intimidation}{\oneCape},
	innateAbility = Rage,
	innateDescription = \imp{Rage} is the deep seated anger that lies within the hearts of most people\comma{} even the most benevolent of us. A \bname{}\comma{} however\comma{} has learned to weaponise their rage\comma{} either by letting it out in an unbridled fury\comma{} or harnessing it\comma{} fuelling their cold\comma{} calculated actions. 

Whilst in combat\comma{} \imp{Rage} can be substituted for almost any physical act\comma{} and often requires far fewer successes to achieve – attacks fuelled by rage deal an additional point of \imp{Rage}. However\comma{} on any turn on which you use \imp{Rage}\comma{} you take a 1d penalty on all resist attempts.,
	practicalAbility = Command,
	practicalDescription = A \bname{} is trained not only in their own combat\comma{} but in the leadership of others. Using a \imp{Command} allows a \bname{} to influence the tide of battle on a large scale\comma{} giving advice\comma{} issuing orders and otherwise taking control of the situation. 

A \bname{} with a high \imp{Command} is respected as a warrior\comma{} and others will leap to follow your orders. Those who follow the issued command will find that the action is easier than expected\comma{} being buoyed and inspired by the \imp{Command}.,
	knowledgeAbility = Tactics,
	knowledgeDescription = A \bname{} lives and dies by their knowledge of tactics. Whether it is trying to discern a viable approach to defeating a seemingly implacable foe\comma{} or recognising a strategy employed by the enemy\comma{} a \imp{Tactics} check can help reveal how an opponent functions\comma{} and what the best way to defeat them is.,
	article = A,
	bonusWidth = 4.7, feats = \WarriorFeats
}


%%End
