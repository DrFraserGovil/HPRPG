
\renewcommand\ability[4]
{
	\subsubsection{\imp{#1 Ability: #2}}
	
	#3

	%#4

}

\newcommand\feat[2]
{
	\vspace{\minus{}0.4cm}
	\subsubsection{\key{#1}}
	
	#2

}

\newcommand\bonus[2]
{
	\parbox[t]{\bonusWidth cm}{\hfill \raggedright\imp{#1}}	&	#2 \\
}

\makeatletter
\define@key{archetype}{name}{\def\name{#1}}
\define@key{archetype}{article}{\def\article{#1}}
\define@key{archetype}{bonuses}{\def\bonuses{#1}}
\define@key{archetype}{experience}{\def\experience{#1}}
\define@key{archetype}{feats}{\def\feats{#1}}
\define@key{archetype}{bonusWidth}{\def\bonusWidth{#1}}
\define@key{archetype}{description}{\def\description{#1}}
\define@key{archetype}{innateAbility}{\def\innateAbility{#1}}
\define@key{archetype}{innateDescription}{\def\innateDescription{#1}}
\define@key{archetype}{innateNil}{\def\innateNil{#1}}
\define@key{archetype}{innateI}{\def\innateI{#1}}
\define@key{archetype}{innateII}{\def\innateII{#1}}
\define@key{archetype}{innateIII}{\def\innateIII{#1}}
\define@key{archetype}{innateIV}{\def\innateIV{#1}}
\define@key{archetype}{innateV}{\def\innateV{#1}}
\define@key{archetype}{innateVI}{\def\innateVI{#1}}
\define@key{archetype}{innateVII}{\def\innateVII{#1}}
\define@key{archetype}{knowledgeAbility}{\def\knowledgeAbility{#1}}
\define@key{archetype}{knowledgeDescription}{\def\knowledgeDescription{#1}}
\define@key{archetype}{knowledgeNil}{\def\knowledgeNil{#1}}
\define@key{archetype}{knowledgeI}{\def\knowledgeI{#1}}
\define@key{archetype}{knowledgeII}{\def\knowledgeII{#1}}
\define@key{archetype}{knowledgeIII}{\def\knowledgeIII{#1}}
\define@key{archetype}{knowledgeIV}{\def\knowledgeIV{#1}}
\define@key{archetype}{knowledgeV}{\def\knowledgeV{#1}}
\define@key{archetype}{knowledgeVI}{\def\knowledgeVI{#1}}
\define@key{archetype}{knowledgeVII}{\def\knowledgeVII{#1}}
\define@key{archetype}{practicalAbility}{\def\practicalAbility{#1}}
\define@key{archetype}{practicalDescription}{\def\practicalDescription{#1}}
\define@key{archetype}{practicalNil}{\def\practicalNil{#1}}
\define@key{archetype}{practicalI}{\def\practicalI{#1}}
\define@key{archetype}{practicalII}{\def\practicalII{#1}}
\define@key{archetype}{practicalIII}{\def\practicalIII{#1}}
\define@key{archetype}{practicalIV}{\def\practicalIV{#1}}
\define@key{archetype}{practicalV}{\def\practicalV{#1}}
\define@key{archetype}{practicalVI}{\def\practicalVI{#1}}
\define@key{archetype}{practicalVII}{\def\practicalVII{#1}}
\makeatother

\newcommand\archetype[1]
{
	\begingroup
	\setkeys{archetype}{name= None,article = A, bonuses = , description= None,innateAbility= None,innateDescription= None,innateNil= None,innateI= None,innateII= None,innateIII= None,innateIV= None,innateV= None,innateVI= None,innateVII= None,knowledgeAbility= None,knowledgeDescription= None,knowledgeNil= None,knowledgeI= None,knowledgeII= None,knowledgeIII= None,knowledgeIV= None,knowledgeV= None,knowledgeVI= None,knowledgeVII= None,practicalAbility= None,practicalDescription= None,practicalNil= None,practicalI= None,practicalII= None,practicalIII= None,practicalIV= None,practicalV= None,practicalVI= None,practicalVII= None,bonusWidth = 3, feats = , experience = \item Do something } 

	\setkeys{archetype}{#1}
	
	
	\chapter*{\name}
	\addcontentsline{toc}{section}{\name}
	
	\small 
	\description
	
	\section{\name{} \imp{Capabilities} }
	
	\article{} \imp{\name} gets the following bonuses to their \imp{Aspects}, \imp{Abilities} and \imp{Affinities}. Where a choice is given, you cannot make the same choice twice. Note that these are {\it bonuses} on top of those granted by other abilities and natural starting values. 
	
	\begin{center}
	\begin{rndtable}{c c}
		\bf Capability	&	\bf Bonus Rating \\
		\bonuses
	\end{rndtable}
	\end{center}
	
	
	
	\section{\name{} Experience}
	
	\article{} \imp{\name} gains additional experience when they:
	\begin{itemize}[itemsep = 0cm]
		\experience
	\end{itemize}
	
	\section{\name{} Special \imp{Abilities}}
	
	A character following the path of the \imp{\name{}} can use the following special abilities: \key{\innateAbility}, \key{\practicalAbility} and \key{\knowledgeAbility}.
	
	\ability{Innate}{\innateAbility}{\innateDescription}{\ratingTable{\innateNil}{\innateI}{\innateII}{\innateIII}{\innateIV}{\innateV}{\innateVI}{\innateVII}}
	\ability{Practical}{\practicalAbility}{\practicalDescription}{\ratingTable{\practicalNil}{\practicalI}{\practicalII}{\practicalIII}{\practicalIV}{\practicalV}{\practicalVI}{\practicalVII}}
	\ability{Knowledge}{\knowledgeAbility}{\knowledgeDescription}{\ratingTable{\knowledgeNil}{\knowledgeI}{\knowledgeII}{\knowledgeIII}{\knowledgeIV}{\knowledgeV}{\knowledgeVI}{\knowledgeVII}}
	
	
	\section{\imp{\name} Feats}
	
	As an Auror progresses, they may choose to take some of the following feats:
	
	\feats
	\endgroup
	
}


\addcontentsline{toc}{Chapter}{Archetypes List}
\def\auror{\imp{Auror}}


%%Begin
\archetype
{
	name = Auror,
	description = As a profession\comma{} the \auror{}s are a group of highly\minus{}trained law enforcement officials working for the \imp{Ministry of Magic}\comma{} as well as a catchall term for those dedicated to catching bad guys and making them pay.

\imp{Aurors} (or even those who merely wish to emulate them) seek out their target with a single minded zeal\comma{} dedicated to the cause of finding the truth and bringing villains to justice. They adore solving mysteries and puzzles\comma{} and abhore those who would bring harm to others. 

Their pursuit of justice often puts them in harm's way\comma{} and so the budding \auror{} is encouraged to focus on magic which allows them to protect themselves from harm\comma{} as well as incapacitate their foes. 

However\comma{} the defining trait of an \auror{} is not their combat abilities but instead their ability to discover clues\comma{} intuit motives and hunt down their foes.,
	experience = \item Track down or hunt a target
\item Solve a mystery 
\item Prevent a crime,
	bonuses = \bonus{Insight}{\twoCape}
\bonus{Investigation}{\twoCape}
\bonus{Hexes or Warding}{\twoCape}
\bonus{Perception}{\oneCape}
\bonus{Brawl}{\oneCape}
\bonus{Warding or Hexes}{\oneCape},
	innateAbility = Intuition,
	innateDescription = \imp{\innateAbility} is the inherent\comma{} instinctive understanding of the minds of others possessed by an insightful and trained mind. Bypassing all \imp{Logic} and conscious reasoning\comma{} \imp{intuition} allows an \name{} to make great strides in their understanding of people and their actions by getting inside their heads and understanding the way that they think. Though not useful for solving traditional intellectual puzzles\comma{} \imp{\innateAbility} can allow an \imp{\name} to suddenly have a flash of insight into the motives\comma{} aims or drive of another being. 

If you wish to know why someone would behave in a given way\comma{} why a certain shop was robbed and not another\comma{} or where a target might head next \minus{} an \name{}'s \imp{\innateAbility} is surely the best tool,
	practicalAbility = Interrogate,
	practicalDescription = The art of extracting information out of a target\comma{} either unwilling to divulge or unaware they're being questioned\comma{} is a key skill for an \imp{\name} to master.   Whilst the untrained would have to rely on raw \imp{Charm}\comma{} \imp{Eloquence}\comma{} \imp{Deception} or even \imp{Intimidation} to try and convince them to give up their information\comma{} the skill of \imp{Interrogation} allows you to dance delicately between all of these skills\comma{} executing known psychological tricks and even shrouding your true questions behind layers of misdirection so your target does not even know when they're giving up valuable information.,
	knowledgeAbility = Tracking,
	knowledgeDescription = Hunting down a foe is a key part of being an \imp{\name}\comma{} and part of that is being able to survey a scene and see where they were\comma{} what they did\comma{} and where they're going next.

Whilst \imp{Intuition} relies on a general understanding of the target's thought pattern\comma{} when you \imp{Track} a target you look for the trail that they have left \minus{} scuffs in the dirt\comma{} broken twigs in the forest and even more abstract trails such as an online presence or a paper trail. Whatever evidence you need to find your target\comma{} \imp{Tracking} can help you out.,
	feats = \feat{Ambush}{When you attack from hiding\comma{} spring a trap or successfully orchestrate an ambush\comma{} you gain +2 dice to your first attack roll}

\feat{Cold Cases}{When performing a \imp{Knowledge} check\comma{} if you can relate the information you seek to a historical or past case you reduce the DV by 2.}

\feat{Familiar Terrain}{Choose a favoured terrain such as \imp{Grasslands}\comma{} \imp{Forests}\comma{} \imp{Urban Areas}\comma{} \imp{Caverns}\comma{} or name a specific region\comma{} such as \imp{Hogwarts}. Whilst in your favoured terrain you gain an additional dice on every action which utilises the surroundings such as a \imp{Tracking} or \imp{Covert} check.}

\feat{Lie Detector}{You can automatically detect when someone is lying to you by telling you deliberate falsehoods.}

\feat{Rapid Reflexes}{When performing a \key{Reflex} roll\comma{} you may roll the dice twice and take the largest value. }

\feat{Unwavering Focus}{Once per day you may expend a \imp{Fortitude} point to reroll all \imp{Catastrophe} dice you rolled\comma{} declaring this action after the roll has been performed.},
	article = An,
	bonusWidth = 3
}

\archetype
{
	name = Druid,
	innateAbility = Belonging,
	practicalAbility = Nurture,
	knowledgeAbility = Commune
}

\archetype
{
	name = Scholar,
	innateAbility = Eureka,
	practicalAbility = Collaboration,
	knowledgeAbility = Speculation
}

\archetype
{
	name = Warrior,
	innateAbility = Rage,
	practicalAbility = Command,
	knowledgeAbility = Tactics
}

\archetype
{
	name = Artificer,
	innateAbility = Complexity,
	practicalAbility = Hack,
	knowledgeAbility = Analyse
}

\archetype
{
	name = Outlaw,
	innateAbility = Savvy,
	practicalAbility = Pickpocket,
	knowledgeAbility = Underworld
}

\archetype
{
	name = Noble,
	innateAbility = Wealth,
	practicalAbility = Inspire,
	knowledgeAbility = Society
}


%%End
