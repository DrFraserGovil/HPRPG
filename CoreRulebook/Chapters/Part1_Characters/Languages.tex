\breaklesschapter{Languages} \label{S:Languages}

There are many different languages spoken across the wizarding world - \imp{Muggles} alone have more than 6,500 active languages, not to mention the dozens of dialects of \imp{gobbledegook} (the alien goblin language), \imp{bartog} (the brutish language of the giants), and more sinister constructs such as the taboo \imp{parseltongue}.

Whilst magic has often been used to bridge the language barrier throughout history, it is often important to know who exactly you share a common language with. 

At character creation, most characters begin speaking only a single language (most likely {\it English}), but can learn new languages either by increasing their \imp{linguistics} skill, or by a prolongued period of study or immersion in the culture of that language (at your \imp{GM}'s discretion).

\section{Human Languages}

There are many thousands of languages that have been used by humans throughout history - most of these do not need a description here: you may freely choose to learn English, French, German, Mandarin or any myriad `real' language as you choose. These languages are useful for communicating with both muggles and wizards alike. 

Some creatures and magical beasts are noted as understanding `human languages'. This should be interpreted as them knowing the local language(s) of the region, and perhaps a smattering of other languages useful for communicating in that area. A sphynx found in Russia will speak Russian, whilst a Sphynx found in Brazil would speak Portugese.

Of more interest, perhaps, are the more ancient and arcane languages. Various cultures throughout history have been particularly focussed on or adept at certain kinds of magic. Historical research (as well as some modern research, either due to linguistic inertia within the field, or a belief in the intrinsic power of the verbiage ) into these areas are often found in tomes written in these languages.

\newcommand\languageName[3]
{
	\subsubsection{\key{#1}}
	
	#2
	
	\key{Script:} #3

}

\languageName{Ancient Greek}{The \imp{Mystics} of {Ancient Greece} were particularly focussed on research into the base building blocks of the universe, the nature of reality and the cosmos and the intersection between mathematics, reality and philosophy. Much research into the \imp{Elemental} school of magic are written on Ancient Greek, as are tomes on \imp{Multiverse Cosmology}, \imp{Numerology} and \imp{Arithmancy}. }{Greek Alphabet}

\languageName{Canaanite}{Unusually amongst the languages here, \imp{Hebrew} has been revived as an everday language, however, a modern Hebrew speaker might struggle to understand its ancient precursor - also known as Canaanite, or `Biblical Hebrew'. Canaanite scholars were often focussed on the relationship between mankind and the divine realms - as well as \imp{Cosmology}, magic of the \imp{Temporal} domain is often written in Canaanite.}{Hebrew Script}

\languageName{Celtic}{The mother of modern Welsh, Scottish Gaelic and Irish tongues, \imp{Celtic} has a strong tradition within the British Isles - most strongly associated with the great \imp{Alchemists} who have thrived in Britain and Ireland throughout the centuries, reflected in the fact that many \imp{alchemists} and \imp{potioneers} used extinct varieties of this tongue to encript their notes. }{Ogham, Latin Alphabet}

\languageName{Khoisan}{A precursor to many of the `click consonant' languages of Southern Africa, \imp{Khoisan} was the language spoken by the great African Transmogrifiers - mighty wizards who could perform the most powerful \imp{Alteration} magics without the need for paltry crutches such as wands or incantations. To this day, the best minds in transmutative theorisation use \imp{Khoisan} as their {\it lingua franca}}{Khoekhoe Script}

\languageName{Latin}{It was \imp{Roman} scholars who first began to standardise the incantations used to cast magical spells (reflecting that most incantations are derived from \imp{Latin}), and to consider the nature of magic itself to be a viable subject of study. As such many of the most esoteric tomes on magical power are written in \imp{Latin} - one who studies \imp{Latin} also has an advantage on discerning the nature of a spell from its incantation alone.}{Latin Alphabet}

\languageName{Nahuatl}{Unlike most other cultures anywhere else in the world, the mesoamerican people accepted magicians into everyday life - a double-edged sword which ultimately led to the rise of mighty \imp{Warrior-Mages} who would lead their armies into battle. With magic becoming so essential to the preservation of each kingdom, these \imp{Warrior Priests} developed the finest understanding of both offensive \imp{Hexes} and defensive \imp{Wards} - the very script used to encode the information also demonstrated proper footwork and tactical positions in the heiroglyphic-like nature. Though the kingdoms eventually fell to the advance of colonialism, the language and practices they taught have seen a revival in modern \imp{battlemage} education. }{Maya Logograms}

\languageName{Norse}{The most prevalant enchanting runes taught in the Western World are derived from the Nordic Runes - so being able to speak \imp{Norse} is almost a precondition of being an excellent \imp{Enchanter}. Less well known, however, is that the art of \imp{Curses} was being refined by village witches and travelling `medicine men', and that modern tomes on such knowledge are often more useful to those who know this tongue.}{Norse Runes}

\languageName{Sanskrit}{An ancient language which evolved on the Indian subcontinent, Sanksrit is one of the oldest known languages which is still spoken today. The oldest surviving sanksrit texts, the \imp{Vedas}, contain discussions of the nature of life, the mind, the soul and the connection of living beings to the deific positive forces which shape the world. Ancient texts on \imp{Hermetics} and \imp{Cerebral} magic, including various rituals and meditative techniques are still being uncovered and translated to this day.}{Brahmi Script, Devanagari Script and descendants}

\languageName{Sumerian}{\imp{Sumerian} and its eventual successor, \imp{Akkadian}, are ancient languages which died out long ago. The language, however, survived in the form of mystical cults throughout history, who found it the ideal language with which to conduct rites of \imp{demon summoning} and other such acts of power - eventually becoming near-synonymous with the act of devil worship and occult practices. Taboo tomes of \imp{Occultism} and \imp{Conjuration} are often written in \imp{Sumerian} in order to hide their forbidden nature.}{Sumerian Cuneiform}

\section{Sapient and Bestial Languages}

Humans are, of course, not the only peoples to have developed a language. Though it is rare for a human to learn another species' language, it is not unheard of - accessing the information or people necessary to learn the language, however, can often be troublesome. 

Learning one of these languages can often be non-trivial. You should work with your \imp{GM} if you wish to learn one of them - learning \imp{gartog} whilst camping out in the \imp{Forbidden Forest} is obviously not viable - camping out in a \imp{Giant Village}, however, is a different story...

\languageName{Bartog}{A rough and brutish language, similar in sound to monosyllabic imitations of cavemen, \imp{Bartog} is the language of the \imp{Giants} and the \imp{Giantkin}. It is said that it is impossible to whisper in \imp{Bartog}: the language naturally requires one to yell at full volume. The language is named for \imp{Bartog the Brainy}, the scholar who invented the writing system that giants use - most commonly to list their great victories and acts of valour upon the village record.}{Bartog, Latin Script (rarely)}

\languageName{Fey}{The language of elven creatures: the \imp{Br{\'u}nb{\'a}su}, \imp{Elleng\ae{}st} and \imp{Dun\ae{}lf}. A concerted effort to exterminate this language was part of the enslavement of the \imp{Br{\'u}nb{\'a}su} into the \imp{House Elves}, and over 1000 years this has resulted in the language becoming almost extinct - or at least {\it very} well hidden from human ears. This has started to change in recent years, with many of the free-elves attempting to revive the language.}{Telerin Alphabet}

\languageName{Gobbledegook}{The language of the \imp{Goblin clans}. Though many different dialects of \imp{gobbledegook} exist, they are all more-or-less mutually intelligible. This language sounds, to human ears, as utterly garbled, like someone making up a language on the spot: there is no consistent phonetic rhythm, words, or even recognisable repeated sounds. It is perhaps remarkable, therefore, that this alien language has the distinction of being the only non-human language to have a word enter standard English - that being {\it gobbledegook} itself, which has been used to mean "nonsense". Perhaps not the most complimentary thing humans have done to goblins (though far from the worst).}{Goblin Script}

\languageName{Haggish}{Spoken by \imp{Hags} the world over, many do not realise that it is a language at all, for it is composed almost entirely of maniacal laughter. Different pitches and tones of laughter are used to convey complex meaning - though the most common message is ``go away, or I will eat you and your plump little babies''.}{None} 

\languageName{Mermish}{The bubbling and whistling language of the Merpeople - it is almost impossible to speak this language above water, as various parts of the language require one to blow bubbles of a certain size, and to shoot or inhale jets of water of different speeds. As a fun fact: the noise of sucking on a near-empty drink through a straw translates into \imp{Mer} as ``I would like to perform horrifying acts upon your mother, your brothers and your domesticated animals", and is the cause of at least three beachside mer-human conflicts.}{Hydric Pictograms}

\languageName{Parseltongue}{The inherent language of serpents and serpent-like creatures such as \imp{Basilisks} and \imp{Runespoors}. For most who speak it, this ability is innate and inherented: all snakes are born knowing this language, and human speakers tend to inherit it from their bloodline (or from botched soul transplants, in the case of \imp{Harry Potter}). A verbal only language (snakes find it difficult to hold a pen), \imp{parseltongue} sounds like the hissing of a snake: low and ominous, and easily missed by non-speakers. Due to its association with \imp{Salazar Slytherin} and \imp{Tom Riddle}, the ability to speak this language is considered a marker of a \imp{Dark Wizard} and is thus considered taboo.}{None (verbal only)} 

\languageName{Spiderling}{A horrifying click-based language which sends chills down the back of even the most arachnid-obsessed of humanoids, this is the native tongue of the \imp{Acromantula}, though all spiders seem to have an inherent ability to understand this language, and often instinctively obey orders given in it. Human speakers who have learned this language claim that the trick is learning how to trap air between the tongue and the roof of the mouth, in order to make a convincing `click', which is usually made via the spider's pedipalps.}{Web-based symbols}

\languageName{Troll}{Wizards often look down on \imp{Troll} as consisting only of `pointing and grunting'. Whilst they are mostly correct in this assessment, trolls are nevertheless able to convey relatively complex ideas through this language. In 2005, a wizarding troupe performed a rendition of {\it Romeo and Juliet} entirely in \imp{Troll}, to rave reviews from critics, though the fact that both Romeo and Juliet were stark-naked \imp{Veela} was probably besides the point.}{None}

\languageName{Umbra, \imp{Nubes} \& Aurora}{Technically three entirely different languages when spoken, \imp{Umbra}, \imp{Nubes} and \imp{Aurora} are the languages spoken by centaurs. \imp{Umbra} is reserved for when the stars are visible above one's head, whilst \imp{Aurora} is used when one basks under the suns rays. At all other times, whether indoors or beneath a clouded sky, one uses \imp{Nubes}. Using the wrong language is considered not only highly offensive, but also a highly cursed act, akin to smashing a mirror. Centaurs believe that words carry great power over the past, the present and the future and are thus always incredibly careful with the words they choose.}{Astrograms}  
\section{Hidden Languages}

These languages are particularly rare - and are almost utterly unknown within the wizarding world. Learning such a language requires a great deal of investigation, and probably travel and exploration of unknown lands. The rare human wizard which knows one of these languages usually gleaned its knowledge from some unadvisable act of staring across the boundaries of time and space - or was born with some ingrained knowledge from beyond the stars. 

\languageName{Empyrean}{When spoken, \imp{Empyrean} is not recognisable to most as a language - instead it sounds like a humming of harmonious chords, of a hidden choir whispering of grace and holiness: Even speaking in \imp{Empyrean} is enough to calm down the most frenzied agressor. The myths say that this is the language of powerful extra-dimensional beings, often mistaken by humans as \imp{angels}. }{Empyrean script}

\languageName{Cthonic}{Wizards drove out or destroyed most of the powerful \imp{demons} in ages long past. All that remains of these powerful forces are lesser shadows, mimicries of the awful powers that once stalked the Earth - and their language: \imp{Cthonic}. When spoken by mortal mouths, this twisted tongue causes the speaker to bleed from the eyes, and those listening might swear they hear the tortured screams emananting from one of the more awful \imp{Hells}.}{Satanic Runes}



\section{Non-Traditional Languages}

Not all languages are verbal, some cannot be written down, and others have unique properties that distinguish them - some common examples include \imp{Sign Language}, \imp{Morse Code} and even \imp{Semaphore}. A more unusual interpretaion might be \imp{Cockney Rhyming Slang} or the near-mythical \imp{Thieves' Cant}, which seem to use normal English, but which bury hidden messages through clever subsitutions, allusions and metaphor. Though perhaps much more situational, you may learn these languages as well. 

You could also initiate an \imp{Extended Action} (see page \pageref{S:Extended}) to create your own language with which you may communicate amongst your allies. Speak to your \imp{GM}!
