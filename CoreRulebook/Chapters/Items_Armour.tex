\chapter{Clothing \& Armour} \label{S:Armour}

The clothing and protective gear you wear can have a dramatic impact on your ability to defend yourself\comma{} or run away from problems. 

\section{Wearing Armour}

\subsection{Outfits}

Wearing thicker armour protects you\comma{} by increasing your {\it Block} statistic by a specified amount. Most sets of clothing are considered to come in a `full set'\comma{} and thus cover the entire torso\comma{} arms\comma{} legs\comma{} feet \minus{} and possibly comes with some headwear. 

For the sake of simplicity\comma{} you are generally discouraged from `mix and matching' various types of armour. You are allowed to switch out various pieces of armour for magical equivalents\comma{} or simply for a cooler aesthetic. However\comma{} your Block value is determined by whatever type of protection you are wearing {\it most} of \minus{} and if in doubt\comma{} the lower value will be used. 

If Gunter the half\minus{}giant wishes to wear a full suit of knight's armour\comma{} but swap the gloves out for her cotton {\it Gloves of Pugilism}\comma{} she can do so without altering the total Block value. However\comma{} if she also swapped out the helmet for a jaunty hat\comma{} and the footwear for some running shoes\comma{} the GM may step in and decree a penalty to her Block statistic. 

\def\y{1.3}
\def\w{3.3}
\def\x{2.6}
\def\u{0.8}
\newcommand\armour[4]
{
\\
	\parbox[t]{\y cm}{\raggedright\textbf{\footnotesize\textit{#1}}} & \parbox[t]{\w cm}{\footnotesize#2} &  \parbox[t]{\x cm}{\footnotesize\raggedright #3}	&	\parbox[t]{\u cm}{\centering #4}
	
}

\subsection{Proficiencies}

Armour comes in 4 categories: clothing\comma{} light armour\comma{} medium armour and heavy armour\comma{} in order of increasing protection. 

he first two of these (clothing \& light armour) can be worn by anyone\comma{} without penalty. However\comma{} wearing medium or heavy armour requires skill to be able to do\comma{} without it becoming a severe distraction. These armours require you to be proficient (either through a class bonus\comma{} or through the relevant Skill). If you attempt to wear armour you are not proficient in\comma{} you take the {\it Encumbered} status effect and check\minus{}disadvantage on any accuracy checks made.

\newpage
\subsection{Clothing}

Everyday clothes offer no additional protection against the attacks of malevolent forces. It is\comma{} however\comma{} comfy and easy to wear. 

You require no proficiencies in order to wear clothing. 

\small
\begin{center}
\begin{rndtable}{p{\y cm} p{\w cm} p{\x cm} p{\u cm}}
\bf Type   &	\bf Description	&	\bf Effect	& \bf Cost

\armour{Casual outfit}{Jeans and a t\minus{}shirt. Cheap\comma{} comfy and practical}{No effect}{\sickle{10}}
\armour{Formal Wear}{Extra suave look for the discerning witch or wizard. Ball gowns and tuxedoes are impractical\comma{} but you look amazing!}{\minus{}2 Dodge\comma{} \\ +2 Charisma}{\galleon{2}}
\armour{Sports clothes}{Specially designed clothing for taking part in physical activity.}{+2 Dodge}{\galleon{1}}
\armour{Wizards Robes}{Once the everyday clothes of all wizardkind\comma{} now usually seen as the typical school uniform of a Hogwarts student.}{+1 to spellcasting checks}{\sickle{7}}

\end{rndtable}
\end{center}
\normalsize
\subsection{Light Armour}

Light armour is the crossing point between what we typically think of as armour (knights clanking around in metal)\comma{} and everyday clothes. Light and flexible\comma{} it grants only limited protection. 

You require no proficiencies in order to wear light armour. 

\small
\begin{center}
\begin{rndtable}{p{\y cm} p{\w cm} p{\x cm} p{\u cm}}
\bf Type   &	\bf Description	&	\bf Effect	& \bf Cost
\armour{Padded}{Formed from multiple layers of soft fabric and padding}{+2 Block\comma{} \\\minus{}1 Dodge\comma{} \\ Conspicuous}{\sickle{25}}
\armour{Leather Jacket}{A simple leather jacket offers a surprising amount of protection. Plus it looks cool.}{+1 Block}{\sickle{10}}
\armour{Warded Cloth}{A recent magical invention\comma{} this expensive material hardens on impact\comma{} providing extra protection\comma{} whilst not impeding your movement.}{+2 Block}{\galleon{12}}
\end{rndtable}
\end{center}
\normalsize

\newpage
\subsection{Medium Armour}

\small
\begin{center}
\begin{rndtable}{p{\y cm} p{\w cm} p{\x cm} p{\u cm}}
\bf Type   &	\bf Description	&	\bf Effect	& \bf Cost
\armour{Bulletproof Vest}{A muggle invention\comma{} this weaved kevlar material offers a good amount of protection.}{+3 Block\comma{}\\ \minus{}1 Dodge\comma{}\\ Resistance to Ranged Weapon attacks}{\galleon{3}}
\armour{Hardened Furs}{A primitive\minus{}appearing armour often worn by giants and other isolated peoples. Layers of hardened leather and treated hides protects against the cold\comma{} as well as from weapons.}{+2 Block\comma{}\\ \minus{}1 Dodge \\ Resistance to Cold damage}{\sickle{15}}
\armour{Tactical Armour}{The armour of the Auror class\comma{} thought to strike the correct balance between hardened and fortified plates inserted between layers of flexible fabric.}{+4 Block \\ \minus{}2 Dodge\comma{} \\ Conspicuous}{\galleon{8}}
\armour{Warrior Robe}{Magical armies are rare\comma{} but Battlemages often wore specially warded robes which offered improved protection\comma{} though hampered movement.}{+3 Block\comma{}\\\minus{}1 Dodge}{\galleon{3}}
\end{rndtable}
\end{center}
\normalsize

\subsection{Heavy Armour}


\small
\begin{center}
\begin{rndtable}{p{\y cm} p{\w cm} p{\x cm} p{\u cm}}
\bf Type   &	\bf Description	&	\bf Effect	& \bf Cost
\armour{Bomb Suit}{Specially designed suit that one must climb inside. Used by professionals who frequently find themselves at risk of incineration or detonation}{+5 Block\comma{}\\ \minus{}6 Dodge \\ Resistance to Fire \& Concussive damage\comma{} \\Conspicuous.}{\galleon{15}}
\armour{Runic Mail}{Enchanted scales of metal fit together to provide full physical and magical protection over your body\comma{}.}{+7 Block\comma{}\\ \minus{}5 Dodge\comma{} }{\galleon{100}}
\armour{Steel Plate}{It is said that modern problems require modern solutions. Steel plate is proof that\comma{} maybe\comma{} this isn't always the case}{+4 Block\comma{} \\ \minus{}5 Dodge\comma{} \\Conspicuous\comma{} \\ Resistance to Piercing \& Slashing damage.}{\galleon{10}}
\armour{Special Response Set}{The bigger\comma{} badder brother of the Tactical armour. Used only when overwhelming firepower needs to be withstood\comma{} as it is much more cumbersome}{+5 Block\comma{} \\ \minus{}4 Dodge \\ Conspicuous}{\galleon{12}}
\end{rndtable}
\end{center}
\normalsize

\newpage

\section*{Damaging Armour}\label{S:DestroyArmour}

Of course\comma{} armour is not a panacea\comma{} and it cannot protect the squishy meat inside indefinitely. 

When a {\it Critical Strike} is performed with one of the damage types mentioned in the table below\comma{} the attacker may choose to forgo inflicting damage and instead damage the armour of the target. 

\begin{center}
\begin{rndtable}{ c c}
\bf Damage Type	&	\bf Armour Damage
\\
Acid	&	1d4
\\
Bludgeoning	&	1d2
\\
Piercing	&	1d4
\\
Slashing	&	1d2
\end{rndtable}
\end{center}

Roll the associated {\it Armour Damage Dice} for the damage type\comma{} and deduct this total from the current Block bonus provided by the being's protective layer. This is a permanent deduction in the Block statistic\comma{} until the armour is repaired. 


If the block\minus{}bonus reaches zero\comma{} the armour is considered `destroyed'\comma{} and is automatically `de\minus{}equipped' as it falls to shreds around you. 

\section*{Restoring Armour}

Damaged Armour may be restored by spending 1 hours repairing it (with a repair kit) for one hour per {\it Block} bonus that must be restored\comma{} or by using a suitable magic spell.

Armour that has been `destroyed' cannot be repaired without proficiency with a {\it repair kit}.
