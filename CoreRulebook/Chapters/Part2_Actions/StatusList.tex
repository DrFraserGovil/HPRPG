\status{Asleep}{Whether by choice\comma{} or by magical influence\comma{} a being is generally completely helpless whilst asleep\comma{} and unable to take any form of action.}{\item No actions or movement can be taken\item After 7 hours\comma{} a \imp{Long Rest} has been completed.\item Status is terminated upon taking \imp{harm}\comma{} or if a suitable stimulus is present.}

\status{Blinded}{Physical trauma to the eyeballs\comma{} as well as overloading them with a bright light leads to the optic centres shutting down. 

Status typically lasts for one round per \imp{Power} used\comma{} though in exceptional circumstances\comma{} may be permanent.}{\item All \imp{Attack} checks by the afflicted are considered \imp{Fighting Blind}\item \imp{Resist} checks are performed at a 2d penalty\item Most checks which rely on vision (i.e. \imp{perception} and \imp{insight} etc.) will automatically fail.}

\status{Burned}{Prolonged contact with a heat source can leave one with severe tissue damage\comma{} and leaves the victim particularly susceptible to changes in temperature.}{\item Target is considered \imp{Susceptible} to \imp{Fire} and \imp{Cold} damage.}

\status{Charmed}{Almost always imposed by magical or hypnotic means\comma{} the \imp{charmed} status means that the target percieves their charmer as their dearest\comma{} friend\comma{} and an ally to be protected at all costs.}{\item A charmed being cannot attack or otherwise target their charmer with negative effects.\item Charmer gains +3d on all \imp{social} checks made against the charmed being}

\status{Confused}{Meddling with the minds of your foe can leave them briefly distracted\comma{} and unable to formulate their thoughts properly.}{\item All checks take a 1d penalty\item Target speaks in a dazed or slurred voice}

\status{Critical But Stable}{A character takes this condition after being \imp{Stabilised} from the \imp{Critical Condition} status. These characters remain gravely wounded\comma{} but their condition is no longer degrading\comma{} and they will eventually recover.}{\item Character remains mostly \imp{Unconscious}\item Taking any amount of damage places them back into \imp{Critical Condition}\item This condition is removed when the victim has cleared one full \imp{Health Diamond}\item Recovering health through any means other than a \imp{Long Rest} imposes a level of exhaustion.}

\status{Critical Condition}{A character takes this status after filling up their \imp{Health Track}. Depending on the \imp{GM}s decision\comma{} this might represent a simple knockout\comma{} or it might represent the target clinging on for dear life as they bleed out. Whatever the narrative position: the target is out of action for the time being\comma{} and needs help from their allies if they want any chance at recovering.}{\item Character falls \imp{unconscious}\comma{} and can take no action.\item At \imp{GM} fiat\comma{} the victim’s condition may begin to degrade\comma{} eventually leading to \key{Death}. This may be rapidly hasted by any subsequent attacks on the victim.\item The target cannot regain health until they are \imp{Stabilised}\item \imp{Stabilising} a patient requires a \imp{Restore} spell of \levelThree{} or greater\comma{} or a \imp{First Aid Kit}.\item Every 30 seconds spent in the critical condition increases the difficulty of ending the condition}

\status{Deaf}{As with blindness\comma{} both magical and mundane means can lead to deafness – overloading the eardrums is particularly paindful.}{\item Can only communicate through vague gestures or written word\comma{} unless both parties know sign language.\item Most checks which rely on hearing (i.e. \imp{perception} and \imp{insight} etc.) will automatically fail.}

\status{Enraged}{Become mindlessly furious\comma{} and perceive all beings as hostile to you.}{\item All actions must be spent performing attacks on the nearest living (or unliving) being to you\comma{} or moving into a position where you can attack them.\item The GM reserves the right to take control of your character for the duration of the effect}

\status{Exhausted}{Exhaustion is a measure of how tired a being is\comma{} and comes in multiple degrees of severity. A being gains levels in Exhaustion through magical means\comma{} or through failing to look after themselves\comma{} and going more than 24 hours without rest.}{\item \key{Level 1: Distracted} 1d penalty to all \imp{Mental} checks. You cannot recover \imp{Fortitude} points by taking a \imp{Long Rest}\item \key{Level 2: Tired}  1d penalty to all checks (so 1d on \imp{mental} checks)\item \key{Level 3: Lethargic} \imp{Speed} reduced to zero\item \key{Level 4: Drained} HP and FP maximum reduced by 1\item \key{Level 5: Catatonic}  HP and FP maximum reduced by 3\item \key{Level 6: Dead} Character Death\item These effects are compounding\comma{} so a Lethargic character has a 1d penalty to all checks (increased to 2d on mental checks)\comma{} and has their speed drastically reduced.\item You lose a level of \imp{Exhaustion} for every \imp{Long Rest} you take.}

\status{Frostbite}{Cold temperatures can affect not just the body\comma{} but also the mind.}{\item A character reduces their maximum \imp{Fortitude} rating by one\comma{} and cannot recover \imp{Fortitude} through long rests.\item If the \imp{Fortitude} track is already filled\comma{} instead take an additional level of \imp{Exhaustion}. This remains even after the \imp{Frostbite} status is removed.}

\status{Immobilised}{An \imp{immobilised} creature has been paralyzed by some force\comma{} or bound and chained up such that they cannot take actions.}{\item An \imp{Immobilised} creature can take no actions\comma{} and may not move until the effect wears off\item They remain aware of their surroundings and can see and hear as normal\item All \imp{Dodge} checks fail\item \imp{Block} checks use only the rating provided by equipped armour}

\status{Poisoned}{A toxin coursing around the bloodstream does continual \imp{poison} damage to a creature\comma{} which cannot be \imp{resisted}. Effects which impose this condition state a level of harm\comma{} and a number of successes that must be reached.}{\item The condition deals a stated amount of \imp{poison} damage at the end of every turn cycle\item At the end of each combat cycle\comma{} the victim may perform an additional DV 7 \imp{Vitality} check\comma{} reducing the number of successes needed to remove the status}

\status{Prone Position}{Beings entering into the \imp{Prone} status are lying on their bellies (or backs)\comma{} having been knocked off their feet. Beings may also choose to voluntarily enter into the \imp{prone} position.}{\item A prone creature can only move via crawling\comma{} at half speed.\item Condition can be ended by taking {\it either} a full movement\comma{} or a minor action to stand up.\item Beings may willingly enter the \imp{prone} status as a free action.}

\status{Terrified}{When soul\minus{}sapping fear affects you\comma{} you become \imp{terrified} of the source of your fear\comma{} and are greatly impaired until you can stop yourself from shaking.}{\item Upon gaining the condition\comma{} targets must either flee\comma{} hide or otherwise seek cover from the source of their fear.\item Once this is completed\comma{} they cannot willingly move back closer to the object of their fear\item Actions against the source of fear take 2d penalty.}

\status{Trapped}{Either due to a person holding you in place\comma{} or a physical binding\comma{} you are unable to move from your current location. You may still take actions\comma{} but cannot move.}{\item Your \imp{movement speed} is set to zero.\item \imp{Dodge} actions take a 2d penalty}

\status{Unconscious}{An unconscious creature is totally incapacitated\comma{} and can take no actions. They are totally unaware of their surroundings.}{\item The creature drops whatever they were holding and takes the \imp{prone position}.\item All subsequent resist checks fail.}

