
\chapter{Defence}\label{S:Accuracy}\label{S:Defence}

A good fighter knows that all-out attack is rarely the path to victory: defending onself against incoming attacks is just as important. 

\section{Defence Statistics}

Your ability to negate incoming attacks is governed by three additional statistics: \key{Block}, \key{Dodge} and \key{Endure}, which govern the three ways that you may prevent an attack from harming you. 

\newcommand\defendRow[3]{\key{#1}	&	\imp{#2}	&	\parbox[t]{4.4 cm}{\raggedright #3} \\}
	
\begin{center}
	\begin{rndtable}{c c c}
		\bf Defence Type	&	\bf Aspect	&	\bf Description 
		\\
		\defendRow{Block}{Fitness}{Prevent an attack from making contact, using your armour, or deflecting it with your fists or weapon.}
		\defendRow{Dodge}{Precision}{Evade an incoming attack, diving out of the way, shifting your body away from the attack or simply moving away from the region under threat.}
		\defendRow{Endure}{Willpower}{Attacks may hit you, but through sheer force of will you shrug off the effects of the attack and keep on going.  Most useful against effects that manipulate your mind, or try to impede you in some way.}
	\end{rndtable}
\end{center}

When trying to negate an attack, you must determine which of these abilities you will be using. 

However, note that note every form of defence is going to be applicable and appropriate - if a cannonball is hurtling towards you, holding up your puny metal shield isn't going to help. Equally, trying to \imp{Endure} as a Hippogriff sinks its claws into your flesh isn't going to do much to mitigate the harm you are suffering. If you attempt to use an inapporpriate mitigation technique, the \imp{GM} may either impose a higher DV, or simply rule it as inneffective. 

\subsection{Associated Aspects \& Defense Ratings}

As noted in the above table, each of the 3 defense tactics is tethered to an associated \imp{Aspect}. Your base \key{Block} rating is therefore equal to your \imp{Fitness} aspect - as the \imp{Aspects} increase and decrease, so do the \imp{Defense} ratings tethered to them. 

However, some effects may increase or decrease your \imp{Defence} ratings, without altering the underlying \imp{Aspects}. For example, wearing a heavy set of armour drastically increases your \imp{Block}, but imposes a heavy penalty to \imp{Dodge}. 

Therefore, whilst the \imp{Aspects} serve as a basis for the \imp{Defense} statistics, they may diverge fairly significantly. 

\section{Defending from Attacks}

When an attack is performed
