
\section{Defence}\label{S:Accuracy}

A good fighter knows that all-out attack is rarely the path to victory: defending onself against incoming attacks is just as important. 

\subsection{Instincts} \label{S:AC}

Most beings either block or dodge, without having to devote conscious thought to their reaction. These two actions are therefore termed {\it instincts}. It is these reactions which set the difficulty of an attacker's accuracy check. A higher {\it dodge} or {\it block} statistic makes it harder for an attack to actually hit you. 

The values associated with each statistic are:

\begin{align*} 
\text{Block} &= 10 + \text{\attPhys{} modifier} 
\\
\text{Dodge} &= 10 + \text{\attFin{} modifier} 
\end{align*}

By default, characters instinctively use whichever of these values is the highest:
$$ \text{IV} = \max \left( \text{Block}, \text{Dodge} \right)$$

If a character successfully dodges, the attack whizzes by their ear and misses completely. If they successfully block the attack, then they catch the spell or weapon on a piece of armour (or, with the appropriate skill, they can {\it parry} the attack with a weapon). 

\subsubsection{Clothing \& Armour}

Various items may improve either of these statistics. A pair of running shoes, for example, makes it easier to dodge out of the way, whilst a heavy shield makes defending yourself easier. 

Generally speaking, items will be a compromise: wearing heavy armour will bulk up your Block statistic, but will slow you down, reducing your Dodge value. 

Armour is discussed more in the Items chapter, on page \pageref{S:Armour}.

\subsubsection{Bolstering Defences }

Of course, not all defence happens instinctively -- you may make a conscious decision to brace yourself against an incoming attack, or prepare to dive out of the way. Such a decision is classified as a minor action. 

Though by default you automatically use whichever value is highest, when making a conscious decision, you may choose to bolster either statistic by {\it bracing} or {\it evading}. 

Whichever action is chosen, enemies take check-disadvantage on accuracy rolls against you for this turn cycle. In addition, you gain check-advantage on certain Resist checks this round, depending on which action you took. 

\def\w{3.6}
\def\c{7}
\small
\begin{center}
\begin{rndtable}{p{1.2 cm} c | c}
~	&	\bf Brace	&	\bf Evade
\\
\cellcolor{\tablecolorhead}\bf Resist:	&	\parbox[t]{\w cm}{\raggedright Advantage on \attPhysShort{}, \attSprShort{} \& \attPowShort{} Resist checks.}	&	\parbox[t]{\w cm}{\raggedright  Advantage on \attFinShort{}, \attIntShort{} \& \attPerShort{} Resist checks}
\\
\cellcolor{\tablecolorhead}\bf Accuracy:	&	\multicolumn{2}{c}{\parbox[t]{\c cm}{\cellcolor{\tablecolorlight}\centering Agressors take disadvantage on accuracy checks made against you this turn}} 
\end{rndtable}
\end{center}
\normalsize
\subsection{Cover}

Standing out in the open is a sure-fire way to get hurt quickly. Hiding behind something, be it a tree, a low wall, or even just your ally will make you safer and harder to hit. 

A target which is concealed in this fashion is said to be {\it under cover}. It is up to the GM to determine to what extent a target is hidden from view. This can usually be achieved through the `additional difficulty' mechanics discussed in the {\it Accuracy} section above. 

If a 2m tall target is 15m away, the penalty to hit is zero. However, if they were covered such that only their head ($\sim 30$cm) could be seen, you can estimate that the penalty to hit them would be -5.

Alternatively, you may use the simpler rules that `half cover' (i.e. half of the target is concealed) gives a -2 penalty to the accuracy check, and `three-quarter cover' gives -5, in addition to any other distance penalties. 

\subsection{Undefendable Effects}

Some effects cannot be avoided or blocked: holding up a shield against an incoming cannonball isn't going to prevent much, and trying to dodge out of the way of a tsunami is rarely effective. 

Spells denote in their description if they can be blocked or dodged. For the (rarer) instances of non-spell effects which fall into one of these categories, the GM decides if it is reasonable to dodge or block the effect. 

If the `dominant' instinct (i.e. the one with the highest value) would be ineffective against a given effect, you may use the non-dominant one. However changing your active instinct negates the effect of both the {\it Evade} and {\it Brace} actions for this turn cycle. Therefore, if a being is attacked by multiple effects in one cycle, it may be beneficial to allow one effect to land home, to keep the bonuses against the remainder of the effects. 

Note that even `unblockable' effects are stopped by `impenetrable' fields. 
