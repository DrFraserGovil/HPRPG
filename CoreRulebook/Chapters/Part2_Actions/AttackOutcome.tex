

\section{Doing Damage} \label{S:Damage}





\subsection{Damage Types}



Many effects specify what kind of damage they do (for instance, a sword does 1d8 slashing damage). This helps the players and the GM work out how the damage is done, and also how it is affected by any weaknesses and resistances possessed by the target. 

Some damage types do damage in unusual ways - draining Fortitude instead of Health, for example. 

\newcommand\damage[2]
{
\textbf{ \textit{#1}}: #2
}

\damage{Acid}{A corrosive spray of acid attacks the HP of a target, and weakens their armour.}

\damage{Bludgeoning}{The blunt-force of a hammer, or the force of falling on the ground deals bone-breaking bludgeoning HP damage.}

\damage{Celestial}{Celestial damage is dealt by pure-otherworldly energy, and damages the HP of Unliving and celestials, but does no harm to living beings.}

\damage{Cold}{Freezing temperatures seep at both your willpower and your health. Damages both the HP of a target, and half as much damage again to FP. } 

\damage{Concussive}{A concussive blast from an explosion or a shockwave causes deafening concussive HP damage.}

\damage{Electric}{Bolts of lightning, or simply touching a high-voltage wire, can lead to electrical HP damage. Electrical damage conducts through water and metal, harming all those in contact.}

\damage{Fatigue}{A magical will-sapping force damages only your FP.}

\damage{Fire}{Fire damage burns the flesh to reduce the HP of a target, and can often lead to long-lasting burns.}

\damage{Force}{A pure magical energy that directly damages HP.}

\damage{Necrotic}{The evil energies of the undead withers your soul as it damages your body -- reducing HP and FP by equal amounts.}

\damage{Piercing}{Daggers, spears and teeth can puncture even the thickest armour to damage HP.}

\damage{Poison}{Venomous stings and poisoned weapons damage HP, and may lead to some other unpleasant side effects}

\damage{Psychic}{Damage that originates not from the body, but from the mind, to damage your HP. You often cannot block psychic damage, you must instead rely on Resisting it.}

\damage{Slashing}{Swinging blades and flashing claws damage the HP of unprotected targets.} 



%~ \section{Statuses}

%~ In addition to dealing damage, you can also inflict negative statuses on your foes, or conversely you may gain a positive status from an item or spell. Statuses are (often temporary) effects and conditions which alter a being's capability for the duration of their effect. Statuses can arise as a result of an enemies attack, a magic spell, or from an interaction with the environment. 

%~ The majority of statuses are negative - they impair the character. However, a few statuses such as {\it Invisible} and {\it Calm Mind} are beneficial. 

%~ Most conditions are only temporary, and will wear off after a certain amount of time - or can be ended by a simple character action. Some Statuses, however, are more serious and can only be removed by magical or medical intervention. The effect which causes a status should specify the termination condition, if any. 

%~ A being can be afflicted by multiple statuses at once, and the effects do stack. However, if you have multiple effects which knock you `unconscious', for example, you are not {\it more} unconscious than if you only had the one effect. 

%~ Some Statuses, such as {\it Burned} and {\it Frostbite} have multiple levels of severity, which are listed as separate statuses.

%~ The full list of Statuses, and the effects they have on a being can be found on page \pageref{S:StatusList}.


%~ \subsection{Critical Strikes}\label{S:Sneak}

%~ A {\it Critical Strike} is an attack which is especially devastating. 

%~ A critical strike can be triggered in a number of ways. Common triggers are: attacking a target you are Hidden from, rolling a `natural 20' on an accuracy check, attacking an entity with the {\it Distracted} status effect. 

%~ When a critical strike happens, you double the number of dice used in the damage roll. For instance, a critical strike with a shortsword normally does 1d6 damage + modifiers. On a critical strike, however, you would do 2d6 + modifiers. 

%~ Alternatively, the attacker may choose to forgo doing damage to the target and damage their armour, using the rules discussed on page \pageref{S:DestroyArmour}.

%~ \section{Immunities \& Weaknesses}

%~ Some beings are more or less effected by certain damage types. This is quantified through one of three descriptors: {\it Immune}, {\it Resistant} and {\it Susceptible}. 

%~ A being which is {\it Immune} to a particular damage type takes no damage when it is inflicted upon them. Most dragons, for instance, are totally immune to Fire damage and the fearsome Basilisk is immune to all forms of Poison damage. Some beings may also be stated to be immune to given status effects (the Basilisk would be immune to the {\it Poisoned} status effect). This means that effect cannot be applied to them. 

%~ A being which is {\it Resistant} is not quite immune, but requires significantly more {\it oomph} to get the same effect. When taking damage of the specified type, the {\it damage check} is performed with disadvantage. 

%~ {\it Susceptible} is the inverse of {\it Resistant}: a being which is susceptible can easily be damaged by a certain damage type. The wood-based dugbog and bowtruckle would be particularly susceptible to taking fire damage, for instance. Damage checks associated with this type are performed with check-advantage. 

%~ \section{Resisting}

%~ Not all effects of actions are cut and dried -- some effects can be {\bf Resisted}. 

%~ Many spells, for example, can be resisted by the target. This occurs if they have a strong enough willpower to overpower the caster; spells such as {\it confundus}, and {\it stupefy}, as well as most illusion spells. Alternatively, somebody might try to restrain you, and your character can perform a physical Resist to break free, if they are strong enough. 

%~ Resist actions, like normal checks, are assigned an attribute (and possibly Proficiencies) that may boost the Resist check. Unless otherwise specified, the Resist check is performed using the standard d20 dice. 

%~ This Resist check is then compared with the assigned or contested DV. If the Resist check is greater than the CV, then the action is either denied, or has a lesser effect. 

%~ Successfully Resisting costs 2 FP. If you have fewer than 2 FP, then you cannot Resist.

%~ You can perform multiple Resists over the course of a Turn Cycle, if multiple combatants attack you with spells that require one, for example. The only limit is when your FP runs out. However, each subsequent resist gets harder and harder: you suffer a 1 point penalty to your check for each Resist you have already performed this cycle. This counter resets at the end of the cycle.

%~ \section{Stealth} \label{S:Stealth}

%~ Being noticed by the enemy is generally regarded as a bad thing. It therefore often pays to be sneaky, to stay hidden from the enemy. Stealth is governed by the FIN attribute, via the Stealth proficiency. 

%~ \subsection{Hiding}

%~ If you are not currently being observed by a being, you may take a major action to {\it Hide}, by performing a d20 Finesse (Stealth) check. This stealth value will then be contested by any hostile beings around you. 

%~ Whilst you are hidden you are considered an `unseen' foe, with the bonuses that come with that (see \pageref{S:Unseen}), and you are not a valid target for an attack. However, you may still take damage from area of effects that include you in their area. 

%~ The GM may ask you to re-perform the sneak check if there is a material change in circumstance. For instance, if you performed the check in a dingy room, and suddenly the lights are turned up, then you may need to re-perform the check, in line with your character altering their strategy for the new environment. Equally, if you take damage whilst hidden, you must perform a DV 15 Spirit (Endurance) check to grit your teeth and avoid shouting out and revealing yourself. 

%~ You remain hidden until you to do something to give away your position: shouting to your allies, or jumping from the shadows, sword in hand. 

%~ If an individual enemy does manage to spot you, but their allies fail to, they can use a {\it communication} action to alert everyone else to your presence. 


%~ \subsection{Being Discovered}

%~ Every character and beast has a baseline level of awareness, even when not actively searching for hidden creatures or traps. This is your {\it passive perception}, discussed on page \pageref{S:PassivePerception}.
%~ Alternatively, the beings might decide to take a major action to survey their surroundings, in which case they may perform an active Perception check, which may increase their perception value for this turn. 

%~ If a being's perception value exceeds your sneak value (and it is reasonable for them to be able to percieve you), then they have spotted you, and you are no longer hidden from that creature.  



