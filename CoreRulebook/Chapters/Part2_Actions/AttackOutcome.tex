\chapter{Effects, Damage \& Statuses}


After the \imp{Combat Cycle} has completed, all the attacks and responses have been declared, the effects are then applied to the respective characters. 



\section{Dealing Harm}

Each character and being possesses a \key{Health Rating}, a measure of their physical wellbeing, which is discussed in more detail on page \pageref{S:Health}. This \imp{Health} rating encodes a being's ability to absorb \imp{harm}.

Each level of \imp{Health} corresponds to a \imp{health-diamond} on the character sheet, and are numbered from \imp{level one}, up to the maximum number. For most normal humans this is 7, though more advanced and powerful may have a higher maximum health. Non-human beings may have a higher or lower maximum \imp{Health} depending on their hardiness. 

Whenever an attack or a spell would do damage to a character, the \imp{Power} of the attack is converted into \imp{harm}, with each point of \imp{power} increasing the \imp{harm level} by one.  

\subsection{Basic Harm}

When a character takes an amount of \imp{harm}, a character would immediately fill in the number of diamonds up to the level of harm that was dealt. 

If a character takes \imp{level three} harm, they would therefore fill in the first 3 diamonds on their \imp{health track}

\newcommand\df[1]
{
	\begin{tikzpicture}
		\def\r{0.1}
		\def\theta{45}
		\def\dfill{white}
		\if#15
			\def\dfill{black}
		\fi
		\draw[rotate =\theta,fill = \dfill] ({-\r},{-\r}) rectangle ({\r},{\r});
		
		\ifnum#1>0
			\draw[rotate =\theta,fill = \dfill] ({-\r},{-\r}) -- ({\r},{\r});
			
			\ifnum#1>1
				
				\draw[rotate =\theta,fill = \dfill] ({-\r},{\r}) -- ({\r},{-\r});
				\ifnum#1>2
					
					\draw[rotate =\theta,fill = \dfill] ({-\r},{0}) -- ({\r},{0});
					
					\ifnum#1>3
						
						\draw[rotate =\theta] ({0},{-\r}) -- ({0},{\r});
					\fi
					
				\fi
			\fi
		\fi
		
	\end{tikzpicture}
}

\begin{center}
	\df{0}\df{0}\df{0}\df{0}\df{0}\df{0}\df{0} + \imp{3 harm}$~~\Longrightarrow~~$\df{5}\df{5}\df{5}\df{0}\df{0}\df{0}\df{0}
\end{center}

Harm dealt in this fashion is not additive, so if the 3-\imp{harm} above had been dealt to a character with one diamond filled in, they would still end up on \imp{3-harm}:

\begin{center}
	\df{5}\df{0}\df{0}\df{0}\df{0}\df{0}\df{0} + \imp{3 harm}$~~\Longrightarrow~~$\df{5}\df{5}\df{5}\df{0}\df{0}\df{0}\df{0}
\end{center}


\subsection{Cumulative Harm}

If an attack would deal \imp{harm} below the amount already taken by a creature, you evidently cannot fill in the required diamonds:

\begin{center}
	\df{5}\df{5}\df{5}\df{0}\df{0}\df{0}\df{0} + \imp{2 harm}$~~\Longrightarrow~~$ ??
\end{center}

When this occurs, instead convert the \imp{harm} into \key{cumulative tallies}. These take the form of a number of marks placed into the next highest \imp{health diamond} - one mark for each level of \imp{harm}. 

\begin{center}
	\df{5}\df{5}\df{5}\df{0}\df{0}\df{0}\df{0} + \imp{1 harm}$~~\Longrightarrow~~$\df{5}\df{5}\df{5}\df{1}\df{0}\df{0}\df{0}
	
	\df{5}\df{5}\df{5}\df{0}\df{0}\df{0}\df{0} + \imp{2 harm}$~~\Longrightarrow~~$\df{5}\df{5}\df{5}\df{2}\df{0}\df{0}\df{0}
	
	\df{5}\df{5}\df{5}\df{1}\df{0}\df{0}\df{0} + \imp{2 harm}$~~\Longrightarrow~~$\df{5}\df{5}\df{5}\df{3}\df{0}\df{0}\df{0}
	
	\df{5}\df{5}\df{5}\df{3}\df{0}\df{0}\df{0} + \imp{1 harm}$~~\Longrightarrow~~$\df{5}\df{5}\df{5}\df{4}\df{0}\df{0}\df{0}
\end{center}

A diamond is considered filled in when it contains five tallies:
\begin{center}
	\df{5}\df{5}\df{5}\df{4}\df{0}\df{0}\df{0} + \imp{1 harm}$~~\Longrightarrow~~$\df{5}\df{5}\df{5}\df{5}\df{0}\df{0}\df{0}
\end{center}

If you deal enough damage to a character that it would fill in a new diamond {\it but} a character would end up with a higher level of harm if considered cumulatively, use the cumulative rules instead. 

For example, the character below has taken \imp{level 4} harm in total, and then gained an additional 4 tallies in the next diamond, such that taking anything above \imp{level one harm} would push them up to 5 filled in diamonds, plus some additional cumulative marks. However, taking \imp{level 5} harm would {\it also} place them onto five filled diamonds, without the `spare change'. This is obviously silly, so the harm is treated as five \imp{cumulative tallies}:


\begin{center}
	\df{5}\df{5}\df{5}\df{5}\df{4}\df{0}\df{0} + \imp{5 harm}$~~\Longrightarrow~~$\df{5}\df{5}\df{5}\df{5}\df{5}\df{4}\df{0}
\end{center}



\section{Healing}

The restoration of \imp{health} through magical, mundane or alchemical means follows a similar pattern - albeit in reverse. 

A \imp{healing} spell with a \imp{power} of three would clear the \imp{health} track of any character who had taken less than 3 \imp{harm}:

\begin{center}
	\df{5}\df{5}\df{3}\df{0}\df{0}\df{0}\df{0} + \imp{3 healing}$~~\Longrightarrow~~$\df{0}\df{0}\df{0}\df{0}\df{0}\df{0}\df{0}
\end{center}

However, if you have more \imp{harm} than \imp{healing}, the effect is converted into (negative) tallies, and removed one at a time:

\begin{center}
	\df{5}\df{5}\df{5}\df{5}\df{0}\df{0}\df{0} + \imp{3 healing}$~~\Longrightarrow~~$\df{5}\df{5}\df{5}\df{2}\df{0}\df{0}\df{0}
\end{center}

\section{Effects of Harm}

After \imp{harm} has been applied to the victim, they usually begin to suffer the consequences of that harm as their arms shake, blood drips down their face and their physical condition begins to degrade. 

For basic humanoids, for every level of harm they take after \imp{level one}, they begin to suffer a 1d penalty to rolls made for \imp{ability} checks and \imp{attack rolls}, increasing to a maximum of a 5d penalty at \imp{level 6} harm. 

If using the \imp{character sheet} on page \pageref{S:CharacterSheet}, the current penalty a character is suffering is detailed next to their health track. 

Note that this penalty only applies when a \imp{health-diamond} has been completely filled in - if a diamond only has \imp{tallies} in it, you do not suffer the associated penalty. If a character is healed such that the \imp{health diamond} is no longer completely filled in (or indeed, empty), they also immediately reduce their dice penalty to match their new level of harm. 

\subsection{Avoiding Penalties}

A character may temporarily negate the effects of their harm by expending a \imp{fortitude} point. This removes all harm-related penalties from a being for the duration of this \imp{combat cycle} - though some \imp{feats} expand this period somewhat. 

In addition, \imp{harm}-related penalties are {\bf not} applied to \imp{Resist} rolls using the \imp{defence statistics}. 

\subsection{Additional Health}

If a \imp{character} gains an additional level of \imp{health}, either through the \imp{progression rules} (page \pageref{S:Progression}), or through magical means, this point goes at the {\it front} of the health track, and is removed before the `normal' \imp{health} is touched. For this reason, such characters do not begin to suffer a dice penalty until they reach \imp{level 3} harm. 

If this additional level of health is conferred whilst a character has already suffered \imp{harm}, readjust the health scale  as if they had {\it always} had that level of \imp{health}. 

\section{Damage Types}



Many effects specify what kind of damage they do: for instance, a punch deals \imp{Bashing} damage, whilst summoning a bolt of lightning onto a foe deal \imp{Electric} damage. 

Whilst this is often merely just for flavour, enabling your group to differentiate between the burns you gained fighting the fire demons, vs the cuts and slashes gained from annoying a hippogriff, some creatures have a natural ability to resist certain kinds of attacks, and others are much more vulnerable to them. 

Magical effects, potions and enduring status effects can also provide a vulnerability to, or protection from, certain kinds of attack. If you know you are going up against a powerful foe, you may therefore tailor your attacks towards those that they are weak against, whilst bolstering your defences against their preferred form of attack. 

Each damage type falls into one of three categories: \key{Physical}, \key{Energetic} and \key{Vitriolic}.


\newcommand\damCat[3]
{
	\subsection{#1}
	
	#2
	
	\begin{itemize}
	#3
	\end{itemize}
}
\newcommand\damage[2]
{

\keyItem{#1}{#2}
}


\damCat{Physical}
{
	\key{Physical damage} causes cuts and bruises, breaks bones and pierces vital organs. Almost all normal weaponry deals \imp{Physical} damage, and many magical effects mimic the effects by summoning walls of force to crush a foe. 
}
{
	\damage{Bashing}{This kind of damage arises when someone is struck forcefully with a blunt object, such as a fist, a club, or (in extreme cases) a warhammer. Commonly leaves bruising and broken bones.}
	\damage{Crushing}{Whilst similar to \imp{Bashing}, \imp{crushing} damage is dealt when something falls from a great height, or is trapped between two heavy objects. Normally very painful, and very hard to evade or negate.}
	\damage{Cutting}{When attacked with a sharp blade, expect to suffer \imp{Cutting} damage. \imp{Cutting} damage causes targets to bleed profusely.}
	\damage{Stabbing}{\imp{Stabbing} damage is caused by attacks that pierce the skin, and cause trouble on the inside of your body. Almost all \imp{ranged} weaponry relies on stabbing damage from their projectiles. }
}

\damCat{Energetic}
{
	\key{Energetic} damage is caused by an interaction with some kind of elemental or energetic force - sticking ones hand into a burning flame, or subjecting yourself to freezing cold temperatures. Whilst \imp{Muggles} are certainly familiar with most types of \imp{Energetic} damage, it is normally only wizards who can manipulate these forces on such an individual and powerful level. 
}
{

	\damage{Cold}{Freezing temperatures seep into your flesh, causing frostbite and even freezing your limbs solid. }
	\damage{Fire}{\imp{Fire damage} burns of a the victim, and can often lead to long-lasting burns. In extreme cases, the target may catch on fire and continue to take fire damage until the fire is extinguished.}	
	\damage{Electric}{Bolts of lightning, or simply touching a high-voltage wire, can lead to \imp{Electic damage}. Known for conducting itself through all manner of materials such as metal and water, \imp{electric} damage can often be used to control large crowds of foes.}
	\damage{Incandesence}{A rare and unusual form of damage, imposed by pure radiant light. Whilst many living beings have some form of resistance to it, beings of darkness and shadow find it repulsive. As such, it is usually associated with life and goodness.}
}


\damCat{Vitriolic}
{
	\key{Vitriolic} damage is an unusual form of damage, and is usually classified as that which saps, degrades and destroys its target. 
}
{
	\damage{Acid}{When coming into contact with a corrosive or caustic substance, you are likely to suffer \imp{Acid} damage....physical items are also very likely to get harmed as well!}
	\damage{Necrotic}{An evil, soul-sapping, rotting force degrades the body of living beings, causing it to blacken and die. If \imp{Incandesence} is associated with light and life, \imp{Necrotic} is the exact opposite.}
	\damage{Poison}{When injected with a toxic substance, the venom courses around the victim's system, dealing awful \imp{Poison} damage to their internal organs. }
	\damage{Psychic}{A rare form of damage, where a psychic force infiltrates a target's brain and causes them to suffer such mental anguish that they near death.}
}


\section{Immunity, Resistance \& Susceptibility}

Some creatures inherently possess a particular aversion to a particular kind of \imp{harm}, whilst others are particularly resistant to it. A tree-dwelling \imp{Bowtruckle}, for instance, is very vulnerable to \imp{fire Damage}, whilst Hagrid's \imp{Blast-Ended Skrewts} could brush off a \imp{Dragon}'s fiery breath with nary a scratch. 

Equally, a witch or wizard might find themselves under the influence of a magical artefact or a spell which confers a similar affinity or aversion to a particular kind of harm. 

\subsection{Immunity}

As the name suggests, a being which is \key{Immune} to a particular damage \imp{type} suffers no \imp{harm} when such an attack is directed at them: the effect is simply cancelled out. 

Such a being is also \imp{immune} to any associated \imp{status effects} (see below). For instance, a being which is \imp{immune} to \imp{poison damage} cannot be afflicted with the \imp{poisoned} status effect. 

\subsubsection{Resistance}

A being with \key{Resistance} to a particular type of damage is able to take much more of a beating through that channel - though not quite at the level of true \imp{immunity}. 

Such a being suffers one level less harm from such an attack, and the \imp{DV} associated with all \imp{Resistance} checks is 3 lower than normal. This applies both to direct attacks, and to associated \imp{status effects}. 

\subsubsection{Susecptibility}

At the opposite end of the spectrum is \key{susceptibility}: a being is said to be \imp{susceptible} to a damage \imp{type} if they are particularly vulnerable to it. 

A \imp{susceptibility} imposes an additional level of \imp{harm} from an attack of that type, and the \imp{DV} associated with all \imp{Resistance} checks is 3 higher than normal.


\subsection{Odd Weaknesses}

Whilst most creatures encountered in the \imp{Game Master's Guide} have \imp{Immunities, Resistances} and/or \imp{Weaknesses} drawn solely from the list of \imp{damage types} and \imp{status effects}, others have more strange, specific or esoteric strengths and weaknesses. 

Some have a weakness to a particular spell (i.e. \imp{Dementors} are considered \imp{susceptible} to the patronus), whilst others (i.e. the rapid \imp{nogtail}) become susceptible to all attacks in the presence of certain creatures or objects. 

Your \imp{GM} may also rule that a certain \imp{called} shot would hit a \imp{susceptible} area, even if it is not explicitly listed - kicking a human between the legs (particularly the male of the species) is likely to confer a significant advantage to the attacker, even if this is not explicitly mentioned in the stat block of each humanoid!


\chapter{Status Effects}

In addition to dealing damage, you can also inflict negative statuses on your foes, or conversely you may gain a positive status from an item or spell. Statuses are (often temporary) effects and conditions which alter a being's capability for the duration of their effect. Statuses can arise as a result of an enemies attack, a magic spell, or from an interaction with the environment. 


\section{Removing Statuses}
Like \imp{attacks}, the imposition of status effects come with an associated \imp{power}, which determines the duration of the status effect. The effect is only lifted once the \imp{power} of the status effect has been reduced to zero. 

Some status effects allow the victim to perform an additional check, either as an action, or as an instantaneous effect at the end of each round. Each success gained on one of these actions reduces the remaining power of the \imp{status} by one, until it is removed. 

Some status effects (such as \imp{poisoned}) have this explicitly mentioned - whilst other status effects are intentionally left vague as to if these additional checks are allowed - your \imp{GM} should make the call based on the specifics of the situation at hand. 

Status effects which allow `free' rolls at the end of each turn are usually performed with only a single \imp{aspect} or \imp{defence statistic}. These usually represent statuses that wear of gradually with time - a \imp{poison} being fought off within the body, or a magical \imp{charm} fading from your mind. 

If a status can be reduced by taking an whole action, then you generally perform a full \imp{ability} check - these represent active efforts by the character to end the status effect - a \imp{Fitness (Strength)} check to break free of the chains causing a \imp{trapped} status, for example, or a being taking a moment to clear their head and reorient themselves after a \imp{Confusion} effect was imposed upon them.

All \imp{status effects} can have their remaining power reduced by external intervention - a magical \imp{restore} spell, for instance, would purge a \imp{poison} from the body, removing remaining \imp{poison power}  equal to the \imp{power} of the spell, and a suitably powerful \imp{disintegrate} hex would certainly help release a \imp{trapped} ally. 


\section{Status List}

\newcommand\status[3]
{
	\subsubsection{#1}
	
	{\it #2}
	
	\begin{itemize}
	\renewcommand\labelitemi{\minus{}}
		#3
	\end{itemize}
}

\status{Asleep}{Whether by choice\comma{} or by magical influence\comma{} a being is generally completely helpless whilst asleep\comma{} and unable to take any form of action. 

If imposed by an external source on an unwilling target\comma{} the victim remains asleep for approximately 5 minutes for every \imp{power} used to subdue them.}{\item No actions or movement can be taken\item After 7 hours\comma{} a \imp{Long Rest} has been completed.\item Status is terminated upon taking \imp{harm}\comma{} or if a suitable stimulus is present.}

\status{Blinded}{Physical trauma to the eyeballs\comma{} as well as overloading them with a bright light leads to the optic centres shutting down. 

Status typically lasts for one round per \imp{Power} used\comma{} though in exceptional circumstances\comma{} may be permanent.}{\item All \imp{Attack} checks by the afflicted are considered \imp{Fighting Blind}\item \imp{Resist} checks are performed at a 2d penalty\item Most checks which rely on vision (i.e. \imp{perception} and \imp{insight} etc.) will automatically fail.}

\status{Burned}{Prolonged contact with a heat source can leave one with severe tissue damage\comma{} and leaves the victim particularly susceptible to changes in temperature.}{\item Target is considered \imp{Susceptible} to \imp{Fire} and \imp{Cold} damage.}

\status{Charmed}{Almost always imposed by magical or hypnotic means\comma{} the \imp{charmed} status means that the target percieves their charmer as their dearest\comma{} friend\comma{} and an ally to be protected at all costs.}{\item A charmed being cannot attack or otherwise target their charmer with negative effects.\item Charmer gains +3d on all \imp{social} checks made against the charmed being}

\status{Confused}{A confused target cannot speak coherently and cannot move.}{\item Confused entities are considered {\it Distracted}\item Take check\minus{}disadvantage on all rolls.\item Can attempt to snap out of confusion once per turn by repeforming the original Resist check.}

\status{Critical (But Stable)}{Take this status after being cured of the {\it Critical Condition} status\comma{} but still below 0HP.}{\item Character falls unconscious (see below)\comma{} and can take no action.}

\status{Critical Condition}{A character takes this status after falling to 0HP}{\item Character falls unconscious (see below)\comma{} and can take no action.\item Lose 1 HP per combat cycle.\item At \minus{}10 HP\comma{} the being dies.}

\status{Deaf}{A deafened being cannot hear\comma{} and so fails on all ability checks relating to sound.}{\item Perception attribute takes a 4 point penalty\item Can only communicate through vague gestures or written word\comma{} unless both parties know sign language.}

\status{Distracted}{The next attack on you is considered a {\it Critical Strike}.}{\item When taking damage\comma{} you must succeed a DV10 Spirit (Willpower) check\comma{} or halt all actions this turn.}

\status{Encumbered}{Being is overloaded by too many heavy objects}{\item All movement speeds reduced to 25\% of their normal value\comma{} and Dodge stat reduced to half its normal value.\item Gain one exhaustion level for every kilometre moved whilst encumbered.}

\status{Enraged}{Become mindlessly furious\comma{} and perceive all beings as hostile to you.}{\item All actions must be spent performing attacks on the nearest living (or unliving) being to you\comma{} or moving into a position where you can attack them.\item The GM reserves the right to take control of your character for the duration of the effect}

\status{Exhaustion}{Exhaustion is a measure of how tired a being is\comma{} and comes in multiple degrees of severity. A being gains levels in Exhaustion through magical means\comma{} or through failing to look after themselves\comma{} as per page \pageref{S:Survival}. They may lose levels through healing\comma{} or by finding a place to rest and recover. 

\begin{tabular}{l l}
\bf Level  &  \bf Effect
\\
0: Fine    & No effect
\\
1: Distracted   & Disadvantage on Finesse and Perception checks
\\
2: Tired:  & Disadvantage on all ability and accuracy checks
\\
3:  Lethargic:  & Speed halved
\\
4: Drained:  & HP and FP maximum halved
\\
5: Catatonic:  & Speed set to 0
\\
6: Dead   & Character Death
\end{tabular}

These effects are compounding\comma{} so a Lethargic character has disadvantage on checks\comma{} as well as having their speed halved.}{}

\status{Frostbite: Mild}{A creature with mild frostbite finds that their natural regeneration abilities are halted.}{\item Finesse attribute takes a 4 point penalty\comma{} as your fingers get clumsy and lose feeling.}

\status{Frostbite: Severe}{A severe case of frostbite is a {\it Serious Injury}}{\item Lose FP at a rate of 2 per minute. When FP is zero\comma{} lose HP at the same rate.}

\status{Hypoxia}{A being becomes hypoxic if oxygen cannot reach the brain.}{\item Intelligence attribute takes a 4 point penalty.\item FP is set to zero.\item If not cured within 2 minutes\comma{} the being dies.}

\status{Incapacitated}{An incapacitated being can take no actions.}{\item All Athletics and Finesse resist checks fail.}

\status{Invisible}{An invisible creature cannot be detected through sight. For the purposes of Stealth\comma{} the creature is considered {\it Severely Obscured}.}{\item In adverse conditions (i.e. rain and snow)\comma{} can still be visually detected. Does not stop noise.\item Attacks on the creature must be considered {\it Blindfighting}}

\status{Paralyzed}{A paralyzed creature is totally incapacitated\comma{} but is aware of their surroundings.}{\item For the purposes of accuracy\comma{} they are considered inanimate objects.}

\status{Poisoned: Mild}{A mild poison causes you to vomit if you overexert yourself: beings cannot take full\minus{}turn movements without passing a DV15 Vitality check.}{\item Athletics attribute takes a 4 point penalty.\item Accuracy checks take check disadvantage}

\status{Poisoned: Severe}{A badly poisoned being is suffering from a {\it Serious injury}\comma{} and will surely perish soon.}{\item Being experiences visual and auditory hallucinations\item Lose HP at a rate of 3 HP per minute.\item Athletics attribute takes an 8 point penalty (min 0).\item Accuracy checks take check disadvantage}

\status{Prone Position}{A prone creature can only move via crawling\comma{} at half speed.}{\item Take disadvantage on all accuracy checks\item All close\minus{}range attacks on the prone creature are considered Critical Strikes.\item Condition can be ended by taking a major action to stand up.}

\status{Serious Injury}{A serious injury is one which cannot be expected to heal naturally\comma{} without major medical intervention.}{\item All HP regeneration is capped at 50\% the maximum health\comma{} until the injury is healed.}

\status{Silenced}{A silenced being cannot speak.}{\item Can only communicate through vague gestures or written word\comma{} unless both parties know sign language.\item Spellcasting is forbidden\comma{} unless they have he {\it Silent Casting} ability.}

\status{Terrified}{A terrified creature has check\minus{}disadvantage whilst they can see the source of their fear.}{\item Cannot willingly move closer to the source of their fear.}

\status{Trapped}{You are fixed in one place\comma{} and cannot move.}{\item Your speed is set to zero.\item Must use the {\it Block} instinct. Dodge value is set to zero.}

\status{Unconscious}{An unconscious creature is totally incapacitated\comma{} and can take no actions. They are totally unaware of their surroundings.}{\item For the purposes of accuracy\comma{} they are considered inanimate objects.\item The creature drops whatever they were holding and takes the prone position.\item All resist checks fail.\item All attacks on the being are considered Critical Strikes.}


