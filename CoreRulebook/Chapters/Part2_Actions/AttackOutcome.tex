\chapter{Calculating Effects}


After the \imp{Combat Cycle} has completed, all the attacks and responses have been declared, the effects are then applied to the respective characters. 



\section{Dealing Harm}

Each character and being possesses a \key{Health Rating}, a measure of their physical wellbeing, which is discussed in more detail on page \pageref{S:Health}. This \imp{Health} rating encodes a being's ability to absorb \imp{harm}.

Each level of \imp{Health} corresponds to a \imp{health-diamond} on the character sheet, and are numbered from \imp{level one}, up to the maximum number. For most normal humans this is 7, though more advanced and powerful may have a higher maximum health. Non-human beings may have a higher or lower maximum \imp{Health} depending on their hardiness. 

Whenever an attack or a spell would do damage to a character, the \imp{Power} of the attack is converted into \imp{harm}, with each point of \imp{power} increasing the \imp{harm level} by one.  

\subsection{Basic Harm}

When a character takes an amount of \imp{harm}, a character would immediately fill in the number of diamonds up to the level of harm that was dealt. 

If a character takes \imp{level three} harm, they would therefore fill in the first 3 diamonds on their \imp{health track}

\newcommand\df[1]
{
	\begin{tikzpicture}
		\def\r{0.1}
		\def\theta{45}
		\def\dfill{white}
		\if#15
			\def\dfill{black}
		\fi
		\draw[rotate =\theta,fill = \dfill] ({-\r},{-\r}) rectangle ({\r},{\r});
		
		\ifnum#1>0
			\draw[rotate =\theta,fill = \dfill] ({-\r},{-\r}) -- ({\r},{\r});
			
			\ifnum#1>1
				
				\draw[rotate =\theta,fill = \dfill] ({-\r},{\r}) -- ({\r},{-\r});
				\ifnum#1>2
					
					\draw[rotate =\theta,fill = \dfill] ({-\r},{0}) -- ({\r},{0});
					
					\ifnum#1>3
						
						\draw[rotate =\theta] ({0},{-\r}) -- ({0},{\r});
					\fi
					
				\fi
			\fi
		\fi
		
	\end{tikzpicture}
}

\begin{center}
	\df{0}\df{0}\df{0}\df{0}\df{0}\df{0}\df{0} + \imp{3 harm}$~~\Longrightarrow~~$\df{5}\df{5}\df{5}\df{0}\df{0}\df{0}\df{0}
\end{center}

Harm dealt in this fashion is not additive, so if the 3-\imp{harm} above had been dealt to a character with one diamond filled in, they would still end up on \imp{3-harm}:

\begin{center}
	\df{5}\df{0}\df{0}\df{0}\df{0}\df{0}\df{0} + \imp{3 harm}$~~\Longrightarrow~~$\df{5}\df{5}\df{5}\df{0}\df{0}\df{0}\df{0}
\end{center}


\subsection{Cumulative Harm}

If an attack would deal \imp{harm} below the amount already taken by a creature, you evidently cannot fill in the required diamonds:

\begin{center}
	\df{5}\df{5}\df{5}\df{0}\df{0}\df{0}\df{0} + \imp{2 harm}$~~\Longrightarrow~~$ ??
\end{center}

When this occurs, instead convert the \imp{harm} into \key{cumulative tallies}. These take the form of a number of marks placed into the next highest \imp{health diamond} - one mark for each level of \imp{harm}. 

\begin{center}
	\df{5}\df{5}\df{5}\df{0}\df{0}\df{0}\df{0} + \imp{1 harm}$~~\Longrightarrow~~$\df{5}\df{5}\df{5}\df{1}\df{0}\df{0}\df{0}
	
	\df{5}\df{5}\df{5}\df{0}\df{0}\df{0}\df{0} + \imp{2 harm}$~~\Longrightarrow~~$\df{5}\df{5}\df{5}\df{2}\df{0}\df{0}\df{0}
	
	\df{5}\df{5}\df{5}\df{1}\df{0}\df{0}\df{0} + \imp{2 harm}$~~\Longrightarrow~~$\df{5}\df{5}\df{5}\df{3}\df{0}\df{0}\df{0}
	
	\df{5}\df{5}\df{5}\df{3}\df{0}\df{0}\df{0} + \imp{1 harm}$~~\Longrightarrow~~$\df{5}\df{5}\df{5}\df{4}\df{0}\df{0}\df{0}
\end{center}

A diamond is considered filled in when it contains five tallies:
\begin{center}
	\df{5}\df{5}\df{5}\df{4}\df{0}\df{0}\df{0} + \imp{1 harm}$~~\Longrightarrow~~$\df{5}\df{5}\df{5}\df{5}\df{0}\df{0}\df{0}
\end{center}

If you deal enough damage to a character that it would fill in a new diamond {\it but} a character would end up with a higher level of harm if considered cumulatively, use the cumulative rules instead. 

For example, the character below has taken \imp{level 4} harm in total, and then gained an additional 4 tallies in the next diamond, such that taking anything above \imp{level one harm} would push them up to 5 filled in diamonds, plus some additional cumulative marks. However, taking \imp{level 5} harm would {\it also} place them onto five filled diamonds, without the `spare change'. This is obviously silly, so the harm is treated as five \imp{cumulative tallies}:


\begin{center}
	\df{5}\df{5}\df{5}\df{5}\df{4}\df{0}\df{0} + \imp{5 harm}$~~\Longrightarrow~~$\df{5}\df{5}\df{5}\df{5}\df{5}\df{4}\df{0}
\end{center}



\section{Healing}

The restoration of \imp{health} through magical, mundane or alchemical means follows a similar pattern - albeit in reverse. 

A \imp{healing} spell with a \imp{power} of three would clear the \imp{health} track of any character who had taken less than 3 \imp{harm}:

\begin{center}
	\df{5}\df{5}\df{3}\df{0}\df{0}\df{0}\df{0} + \imp{3 healing}$~~\Longrightarrow~~$\df{0}\df{0}\df{0}\df{0}\df{0}\df{0}\df{0}
\end{center}

However, if you have more \imp{harm} than \imp{healing}, the effect is converted into (negative) tallies, and removed one at a time:

\begin{center}
	\df{5}\df{5}\df{5}\df{5}\df{0}\df{0}\df{0} + \imp{3 healing}$~~\Longrightarrow~~$\df{5}\df{5}\df{5}\df{2}\df{0}\df{0}\df{0}
\end{center}


\section{Damage Types}



Many effects specify what kind of damage they do: for instance, a punch deals \imp{Bashing} damage, whilst summoning a bolt of lightning onto a foe deal \imp{Electric} damage. 

Whilst this is often merely just for flavour, enabling your group to differentiate between the burns you gained fighting the fire demons, vs the cuts and slashes gained from annoying a hippogriff, some creatures have a natural ability to resist certain kinds of attacks, and others are much more vulnerable to them. 

Magical effects, potions and enduring status effects can also provide a vulnerability to, or protection from, certain kinds of attack. If you know you are going up against a powerful foe, you may therefore tailor your attacks towards those that they are weak against, whilst bolstering your defences against their preferred form of attack. 

Each damage type falls into one of three categories: \key{Physical}, \key{Energetic} and \key{Vitriolic}.


\newcommand\damCat[3]
{
	\subsection{#1}
	
	#2
	
	\begin{itemize}
	#3
	\end{itemize}
}
\newcommand\damage[2]
{

\keyItem{#1}{#2}
}


\damCat{Physical}
{
	\key{Physical damage} causes cuts and bruises, breaks bones and pierces vital organs. Almost all normal weaponry deals \imp{Physical} damage, and many magical effects mimic the effects by summoning walls of force to crush a foe. 
}
{
	\damage{Bashing}{This kind of damage arises when someone is struck forcefully with a blunt object, such as a fist, a club, or (in extreme cases) a warhammer. Commonly leaves bruising and broken bones.}
	\damage{Crushing}{Whilst similar to \imp{Bashing}, \imp{crushing} damage is dealt when something falls from a great height, or is trapped between two heavy objects. Normally very painful, and very hard to evade or negate.}
	\damage{Cutting}{When attacked with a sharp blade, expect to suffer \imp{Cutting} damage. \imp{Cutting} damage causes targets to bleed profusely.}
	\damage{Stabbing}{\imp{Stabbing} damage is caused by attacks that pierce the skin, and cause trouble on the inside of your body. Almost all \imp{ranged} weaponry relies on stabbing damage from their projectiles. }
}

\damCat{Energetic}
{
	\key{Energetic} damage is caused by an interaction with some kind of elemental or energetic force - sticking ones hand into a burning flame, or subjecting yourself to freezing cold temperatures. Whilst \imp{Muggles} are certainly familiar with most types of \imp{Energetic} damage, it is normally only wizards who can manipulate these forces on such an individual and powerful level. 
}
{

	\damage{Cold}{Freezing temperatures seep into your flesh, causing frostbite and even freezing your limbs solid. }
	\damage{Fire}{\imp{Fire damage} burns of a the victim, and can often lead to long-lasting burns. In extreme cases, the target may catch on fire and continue to take fire damage until the fire is extinguished.}	
	\damage{Electric}{Bolts of lightning, or simply touching a high-voltage wire, can lead to \imp{Electic damage}. Known for conducting itself through all manner of materials such as metal and water, \imp{electric} damage can often be used to control large crowds of foes.}
	\damage{Incandesence}{A rare and unusual form of damage, imposed by pure radiant light. Whilst many living beings have some form of resistance to it, beings of darkness and shadow find it repulsive. As such, it is usually associated with life and goodness.}
}


\damCat{Vitriolic}
{
	\key{Vitriolic} damage is an unusual form of damage, and is usually classified as that which saps, degrades and destroys its target. 
}
{
	\damage{Acid}{When coming into contact with a corrosive or caustic substance, you are likely to suffer \imp{Acid} damage....physical items are also very likely to get harmed as well!}
	\damage{Necrotic}{An evil, soul-sapping, rotting force degrades the body of living beings, causing it to blacken and die. If \imp{Incandesence} is associated with light and life, \imp{Necrotic} is the exact opposite.}
	\damage{Poison}{When injected with a toxic substance, the venom courses around the victim's system, dealing awful \imp{Poison} damage to their internal organs. }
	\damage{Psychic}{A rare form of damage, where a psychic force infiltrates a target's brain and causes them to suffer such mental anguish that they near death.}
}


\section{Immunity, Resistance \& Susceptibility}

Some creatures inherently possess a particular aversion to a particular kind of \imp{harm}, whilst others are particularly resistant to it. A tree-dwelling \imp{Bowtruckle}, for instance, is very vulnerable to \imp{fire Damage}, whilst Hagrid's \imp{Blast-Ended Skrewts} could brush off a \imp{Dragon}'s attack with nary a scratch. Equally, a witch or wizard might find themselves under the influence of a magical artefact or a spell which confers a similar affinity or aversion. 

These 


%~ \section{Statuses}

%~ In addition to dealing damage, you can also inflict negative statuses on your foes, or conversely you may gain a positive status from an item or spell. Statuses are (often temporary) effects and conditions which alter a being's capability for the duration of their effect. Statuses can arise as a result of an enemies attack, a magic spell, or from an interaction with the environment. 

%~ The majority of statuses are negative - they impair the character. However, a few statuses such as {\it Invisible} and {\it Calm Mind} are beneficial. 

%~ Most conditions are only temporary, and will wear off after a certain amount of time - or can be ended by a simple character action. Some Statuses, however, are more serious and can only be removed by magical or medical intervention. The effect which causes a status should specify the termination condition, if any. 

%~ A being can be afflicted by multiple statuses at once, and the effects do stack. However, if you have multiple effects which knock you `unconscious', for example, you are not {\it more} unconscious than if you only had the one effect. 

%~ Some Statuses, such as {\it Burned} and {\it Frostbite} have multiple levels of severity, which are listed as separate statuses.

%~ The full list of Statuses, and the effects they have on a being can be found on page \pageref{S:StatusList}.


%~ \subsection{Critical Strikes}\label{S:Sneak}

%~ A {\it Critical Strike} is an attack which is especially devastating. 

%~ A critical strike can be triggered in a number of ways. Common triggers are: attacking a target you are Hidden from, rolling a `natural 20' on an accuracy check, attacking an entity with the {\it Distracted} status effect. 

%~ When a critical strike happens, you double the number of dice used in the damage roll. For instance, a critical strike with a shortsword normally does 1d6 damage + modifiers. On a critical strike, however, you would do 2d6 + modifiers. 

%~ Alternatively, the attacker may choose to forgo doing damage to the target and damage their armour, using the rules discussed on page \pageref{S:DestroyArmour}.

%~ \section{Immunities \& Weaknesses}

%~ Some beings are more or less effected by certain damage types. This is quantified through one of three descriptors: {\it Immune}, {\it Resistant} and {\it Susceptible}. 

%~ A being which is {\it Immune} to a particular damage type takes no damage when it is inflicted upon them. Most dragons, for instance, are totally immune to Fire damage and the fearsome Basilisk is immune to all forms of Poison damage. Some beings may also be stated to be immune to given status effects (the Basilisk would be immune to the {\it Poisoned} status effect). This means that effect cannot be applied to them. 

%~ A being which is {\it Resistant} is not quite immune, but requires significantly more {\it oomph} to get the same effect. When taking damage of the specified type, the {\it damage check} is performed with disadvantage. 

%~ {\it Susceptible} is the inverse of {\it Resistant}: a being which is susceptible can easily be damaged by a certain damage type. The wood-based dugbog and bowtruckle would be particularly susceptible to taking fire damage, for instance. Damage checks associated with this type are performed with check-advantage. 

%~ \section{Resisting}

%~ Not all effects of actions are cut and dried -- some effects can be {\bf Resisted}. 

%~ Many spells, for example, can be resisted by the target. This occurs if they have a strong enough willpower to overpower the caster; spells such as {\it confundus}, and {\it stupefy}, as well as most illusion spells. Alternatively, somebody might try to restrain you, and your character can perform a physical Resist to break free, if they are strong enough. 

%~ Resist actions, like normal checks, are assigned an attribute (and possibly Proficiencies) that may boost the Resist check. Unless otherwise specified, the Resist check is performed using the standard d20 dice. 

%~ This Resist check is then compared with the assigned or contested DV. If the Resist check is greater than the CV, then the action is either denied, or has a lesser effect. 

%~ Successfully Resisting costs 2 FP. If you have fewer than 2 FP, then you cannot Resist.

%~ You can perform multiple Resists over the course of a Turn Cycle, if multiple combatants attack you with spells that require one, for example. The only limit is when your FP runs out. However, each subsequent resist gets harder and harder: you suffer a 1 point penalty to your check for each Resist you have already performed this cycle. This counter resets at the end of the cycle.

%~ \section{Stealth} \label{S:Stealth}

%~ Being noticed by the enemy is generally regarded as a bad thing. It therefore often pays to be sneaky, to stay hidden from the enemy. Stealth is governed by the FIN attribute, via the Stealth proficiency. 

%~ \subsection{Hiding}

%~ If you are not currently being observed by a being, you may take a major action to {\it Hide}, by performing a d20 Finesse (Stealth) check. This stealth value will then be contested by any hostile beings around you. 

%~ Whilst you are hidden you are considered an `unseen' foe, with the bonuses that come with that (see \pageref{S:Unseen}), and you are not a valid target for an attack. However, you may still take damage from area of effects that include you in their area. 

%~ The GM may ask you to re-perform the sneak check if there is a material change in circumstance. For instance, if you performed the check in a dingy room, and suddenly the lights are turned up, then you may need to re-perform the check, in line with your character altering their strategy for the new environment. Equally, if you take damage whilst hidden, you must perform a DV 15 Spirit (Endurance) check to grit your teeth and avoid shouting out and revealing yourself. 

%~ You remain hidden until you to do something to give away your position: shouting to your allies, or jumping from the shadows, sword in hand. 

%~ If an individual enemy does manage to spot you, but their allies fail to, they can use a {\it communication} action to alert everyone else to your presence. 


%~ \subsection{Being Discovered}

%~ Every character and beast has a baseline level of awareness, even when not actively searching for hidden creatures or traps. This is your {\it passive perception}, discussed on page \pageref{S:PassivePerception}.
%~ Alternatively, the beings might decide to take a major action to survey their surroundings, in which case they may perform an active Perception check, which may increase their perception value for this turn. 

%~ If a being's perception value exceeds your sneak value (and it is reasonable for them to be able to percieve you), then they have spotted you, and you are no longer hidden from that creature.  



