\chapter{Performing Checks}\label{C:Checks}


In general, when you want to perform an action, simply tell the GM what you wish to do. 

If it is a simple action – for example, “I walk to the shop”, then the action is completed with no further involvement. More complex actions may require a ‘check’ to be performed, to determine their success: inform the GM of what you want to do, and how you want to achieve it. Working with the GM, you then decide what check to perform, and the GM will decide the condition of the success.


\section{Forming a Dice Pool}

An action's success or failure is determined by rolling a \key{Dice Pool}, comparing each dice with a set value, and counting the number of `successes'.

\subsection{Ratings}

The number of dice which go into a Pool is determined by a character's \key{Rating} in the required area. These ratings are associated with each of the \key{Aspects}, \key{Abilities} and \key{Affinities} discussed on page \pageref{C:Aspects}, and are (generally) numbers between 0 and 7. These numbers indicate the number of dice that are allocated to the dice pool when that skill is used, and may be interpreted as a general measure of a character's ability in that field:

\ratingTable{Usless, totally untrained}
{Beginner, marginal training}
{Novice, some training}
{Average, fair training}
{Adept, some high-level training, low level professional}
{Expert, high-level training, trained professional}
{Master, ultimate training, famous professional}
{Ascendant, no known rivals}
\normalsize
A character with 3 dots in Intelligence therefore rolls 3d12 when an Intelligence check is called for. 

\subsubsection{Combining Skills}

More often than not, you will not be using one of your base \key{Aspects} to complete a skill - you will also be utilising an \key{Ability} to complete the action. For spellcasting, this would instead be replaced by an \key{Affinity}. 

For example, leaping over a cliff clearly uses the \imp{Fitness} Aspect, but if you take a run-up to increase your range, you may utilise your \imp{Speed} ability to increase your chances of success. This would therefore be termed a \imp{Fitness (Speed)} check, and the dice pool would be formed from the sum of your dots in both the Fitness and Speed areas.

Hence, if performing a flying leap over a cavern, a character would perform the \imp{Fitness (Speed)} check, and sum their dots in the \imp{Fitness} aspect (say, 3) and the \imp{Speed} ability (2), and hence have a pool of 5 12-sided dice to use to complete the action. 

\subsubsection{Bonuses and Penalties}

You might sometimes have magical or mundane effects present which modify the number of dice you are allowed to allocate to a given pool. These modifiers cannot push the number below zero, but they may push them above 7, given you truly superhuman abilities.

The GM may also decree that certain situations give you more or less dice than you would normally expect - if you are caught totally unawares, for example, they may dock one or two dice from your Pool, to represent your surprise and lack of care. Equally, if you have spent a few hours preparing an ambush, you may gain an additional die. These effects could also be represented through a modification of the DV - it is up to the GM to decide which approach is most appropriate.

\subsubsection{Zero-Pools}

Sometimes, either as a result of your own lack of ability, or because of some negative magical or mundane effect, a character may find themselves in a situation where there they have no dots to allocate to a dice pool. 

Sometimes this precludes your ability to undertake the action entirely (a \key{Knowledge} ability, for example, often requires at least one dot for the information to be retained), but often does not. In such a situation, you roll two dice, and use the lowest value. 


\section{Determing the Checktype}



The most important ingredient in performing a check is deciding how exactly you are achieving this goal. This is an exercise in roleplaying, more than a hard-and-fast rule: you must evaluate the situation and your character's capabilities and decide how they would utilise their skills to best complete the action. 

In certain situations, the GM may rule that all but a single attribute check would render an action impossible: when sitting a physics exam, nothing but an \imp{Intelligence (Science)} check is going to help you. However, the GM is encouraged to generally avoid this hardline stance, and instead state an {\it ideal} check which would result in the most success for the stated action, and then allow the players to roleplay a potentially different route to completing the action. 

For the players, the temptation is to use this as an opportunity to default to what a character is best at, in order to get the most amount of dice. However, trying to leap over a cavern using your knowledge of Biology from an \imp{Intelligence (Science)} check is not going to be very successful, and will probably lead to you plummeting to your death no matter how many dice you roll! 

However, a player {\it could} attempt to argue that such a check could still be relevant: perhaps your character has studied Newtonian Physics and so can calculate the correct angles and distances required. This might hold some sway with your GM, though you will probably have to defeat a much higher DV, or require more successes for the action to fully complete. 

Developing an interesting narrative about a character and their backstory is more important than a strict adherence to logic and rules, so as long as you can weave and roleplay a convincing narrative as to how you are using a certain skill to complete the action at hand, the GM is encouraged to let you try. The tradeoff is that outlandish skill choices are often poorly suited and therefore make a task significantly more difficult and, even if they succeed, will bring you less success than if you had used a more straitforward approach. 

The players and the GM should work together to decide upon the relevant mix of Aspects and Abilities which form the dice pool, using the combination which makes the most sense, or which provides the richest source of roleplaying material. 


\section{Difficulty}

After deciding upon the ingredients which will go into making the dice pool, the next thing is for the GM to determine the \key{Difficulty} of the task. This is a single number, between 2 and 12 which represents the liklihood of failure. A GM may also decide that an action is utterly impossible under the current circumstances (no matter the dice rolls, you cannot jump to the moon!). 

An example of some common difficulties is shown below:
\def\w{5}
\begin{center}
\begin{rndtable}{|c p{\w cm} c|}
\hline
Task Difficulty & 	Description & DV	
\\ 
\key{Very Easy} & \parbox[t]{\w cm}{\raggedright An everyday task that most people could be expected to carry out first time.}	&	4
\\
\key{Easy} & \parbox[t]{\w cm}{\raggedright A simple task that has only a small chance of failure.}& 6
\\
\key{Standard} & \parbox[t]{\w cm}{\raggedright A task that a normal person might require a few tries to get right} & 8
\\
\key{Hard} & \parbox[t]{\w cm}{\raggedright A task that a normal person could not reliably carry out without specialist training} & 10 
\\
\key{Very Hard} & \parbox[t]{\w cm}{\raggedright A task that even a trained expert might struggle to complete. } & 12
\\ 
\end{rndtable}
\end{center}

\subsubsection{Intractable Difficulties}

Of course, a DV of 12 is the maximum value that an action can require, as no dice roll can exceed 12. However, this cannot always represent the difficulty of some exquisitley difficult tasks. 

For example, a DV 12 action has a 1-in-12 (8\%) chance of succeeding for a character with a one-dot rating, and even 1-in-144 (0.7\%) for a zero-dot rating. A character with 10 dots allocated to a DV 12 action has a 33\% chance of success. However, there are clearly actions for which the chances need to be below that point. 

For example, even Albus Dumbledore would be hard-pressed to even hit a Snitch moving at 3 times the speed of sound with a simple hex. Such an action is possible, but it's rarity is far below a simple DV12 action. Therefore the GM can invoke an {\it Intractable Difficulty}. 

In this case, they simply state a minimum number of successes that must be achieved in order for the action to succeed. These successes are automatically absorbed into the action to even make it into a possibility - this is done before cancelling the successes with the catastrophes, which can lead to an extraordinary amount of bad luck. 

For example, if Dumbledore were to attempt to hit the aforementioned Snitch, the GM may rule that this extraordinary action requires a DV 12 action, with an intractability of 2. Albus goes ahead and rolls 14 dice (he was an extraordinary wizard!), gaining \imp{1-1-4-5-5-6-7-7-8-10-11-12-12-12}. Two of these 12s are absorbed by the intractability, leaving Albus with \imp{1-1-12}, which results in a Catastrophe - perhaps he accidentally hexes a student as collatoral damage. 

If the action had been DV 10, intractability of 2, then Albus would have scored a single success, and so would have skimmed the Snitch, but probably not done any serious damage. 

\section{Success \& Failure}

Every dice rolled from the pool which meets or exceeds the DV counts towards a success. You then report back to the GM the number of success you achieved, which determines how effective the action was:


\begin{center}
	\begin{rndtable}{c p{6cm}}
		\bf \# Successes & \bf Degree \\
		\dualRow{1}{\key{Marginal}: You just scrape by completing the action, perhaps incurring some side effects}
		\dualRow{2}{\key{Okay}: you complete the action with a small amount of leeway. You did it, but not elegantly and there may be side effects}
		\dualRow{3}{\key{Good}: you did the action, and you did it well}
		\dualRow{4}{\key{Complete}: you did the action, and got more than you expected.}
		\dualRow{5}{\key{Excellent}: you did the action, and achieved significantly more than you set out to do}
		\dualRow{6}{\key{Flawless}: You did the action perfectly, and got lots more besides}
		\dualRow{7+}{\key{Legendary}: You completed the action so well people will be telling stories of it for years to come.}
	\end{rndtable}
\end{center}

\subsubsection{Automatic Success}

Rolling dozens of dice may be fun, but when Barry, the most prolific athlete of his generation, wants to hop across a small stream, rolling $>10$d12 against a DV of 3 seems a little overkill. 

As a general rule, if the number of dice in the pool exceeds the DV of an action your GM may simply decide that the action completes automatically. This rule does not generaly apply in combat, or in situations where the consequences of failure are particularly dire. 

\subsubsection{Failure}

If you fail to gain any successes, the action is a failure. You fall short of your leap, you bungle your attempt to charm a guard, or you simply cannot remember the information you seek. 

The exact outcome of a failed check are up to the GM to narrate, based on the current situation and the degree of failure. Generally speaking, the outcome of a failure is not overly severe, however: a failed \imp{Charm (Eloquence)} to charm your way past a guard won't result in the guard arresting or impaling you, they would simply not allow you to pass. Of course, you may have roleplayed yourself into an all-or-nothing situation, in which case a failure can have some very serious negative consequences - you should be careful to try and mitigate situations like that!

\subsubsection{Catastrophes}

Some failures are much worse than others: \key{Catastrophes}. 

A catastrophe is triggered when a dice rolls comes up as a \key{1}, and results in a fumble so serious that it absorbs one of your successes in order to mitigate the failure. Every Catastrophe reduces the number of successes by 1, starting with the highest rolled number. For example, a roll of \imp{1-5-7-8-8-10} against a DV of 8 would normally have 3 successes, however the single catastrophe reduces this down to 2 successes. 

If a check ever results in more Catastrophes than successes, then you suffer a \key{Catastrophic Failure}. These are the worst possible failures, and can often really ruin your day. Rather than hijacking a broomstick, you find yourself plummeting towards the ground, or an attempt to hex your foe leads you to vomiting slugs all over the school field. 

Some effects can also increase the \key{Catastrophe Range} for certain actions. This means that a Catastrophe is triggered even on higher numbers. For example, a {\it Broken Wand} raises the Catastrophe Range to 3 for spellcasting efforts, meaning that the dice rolls 1, 2 \& 3 all trigger catastrophes. If the Catastrophe Range ever exceeds the nominal DV of an action, the DV is reset to be one more than the maximum catastrophe trigger - with a broken wand, no spellcasting DV can ever be below 4, for example.

\subsubsection{Miracles}

The opposite of a \imp{Catastrophe} is, of course, a \key{Miracle}. 

A miracular success occurs whenever you roll a \key{12}. If a \imp{Catastrophe} occurs and attempts to remove a \imp{Miracle}, you instead remove the \imp{Catastrophe}. Each \imp{Miracle} can only ward off a single rolled \imp{Catastrophe}, but this ensures that the action succeeds - even if only with a single success - and therefore prevents a \imp{Catastrophic Failure}. 

For example, against a DV of 8, the roll \imp{1-1-4-7-9-10-10-12} has 1 \imp{miracle}, 3 \imp{successes} and two \imp{catastrophes}. The \imp{Miracle} reduces the number of \imp{catastrophes} to 1, and so the final catastrophe eliminates the rolled \imp{10}, leaving a three \imp{successes}. 

However, if you had rolled \imp{1-1-1-8-12}, two \imp{successes} and three \imp{catastrophes}, the total remains at {\bf a single success}, despite the \imp{catastrophes} seemingly outnumbering the successes. A \imp{Miracle} can therefore really save your bacon when the chips are down! 

Note that \imp{Miracles} are only relevant when the DV is below 12 - if the DV is equal to 12, even being able to attempt the action is miracular. 


\subsection{Contests}

Many actions are not just one character doing something whilst the world holds its breath. Quite often the efforts of one character are being opposed by the other. For example, if Bruce is trying to shove a Death Eater off a ledge, they will contest their strengths to see who ends up the victor. 

Each character performs an action with the DV set by the difficulty of the {\it instigator's} action. Hence if Bruce attempts a simple shove, with a DV 4 or 5, the standard DV for the Death Eater to Resist would be 4 or 5. Of course, the attacker might choose to make things harder for themselves - maybe Bruce attempts a flying leap (DV 7), or a roundhouse kick (DV 9). The opponent would then have to match the DV of the `attack' in their attempt to Resist it. 

The GM may also rule that the situations are different enough that the characters have different DVs: if Bruce is on firm, safe ground whilst the Death Eater is on loose terrain at the very edge of the cliff - pushing is incredibly easy, whilst Resisting is hard. Hence Bruce may be able to perform a simple DV 4 push, whilst the Death Eater must resist with a DV of 8 or higher. 

When the DV is chosen, both characters perform the check and compare their number of successes. The `success' of the contest is determined by how many more successful Instigations there were than Resists: if Bruce rolled 5 successes and the Death Eater rolled 3, then the push action has 2 successes, and the situation is resolved using this value - in this case an \imp{Okay} success: the Death Eater probably plummets off the ledge, but has time to shout out a warning to their allies as they do so.

If the opponent gets more successes than the instigator, the action fails. The Death Eater manages to hold their ground, and the action then proceeds. 

In the case where the number of Instigations equals the number of Resists, the status quo is preserved. In the example above, the Death Eater was not plummeting off the ledge, so on an equal-contest, they remain that way. However, if Bruce was trying to save his ally from being dropped off a ledge by a Death Eater, the status quo is that the ally is going off the ledge. 

\section{Working Together}

Occasionally two or more characters might decide that, together, they have a better chance of succeeding in a given task, and can work together. A character may only help if they could perform the action themselves (so you could only help pick a lock if you also had experience with lockpicking tools), or if you can provide a reasonable justification for how you are helping the action succeed (an untrained individual could help an engineer fix an engine by passing them tools, and holding a flashlight, for example). 

If multiple parties are actively partaking in the action, they each roll a check, and sum their successes and catastrophes together. For example Bruce and Jane are both searching the dungeons for clues, a DV 6 action. Bruce rolls \imp{1,1,3,10} and Jane rolls \imp{3,7,10,10,11}. Jane uses one of her successes to nullify Bruce's catastrophe - perhaps she spots a trap just moments before he steps in it - giving the group a total of 2 successes for the search - an adequate but not great sweep of the room. 

Alternatively, if a character is providing only incidental help - i.e. the engineer/assistant example given above, the GM may rule that this simply grants an additional dice to the person performing the bulk of the action.


\section{Multiple Attempts}

Sometimes, after an action fails, a character may want to try again immediately. This is generally to be discouraged - it makes the game less fun if everyone is just waiting for Mike to (finally) roll a 3 successes. 

A general rule is that you can't repeat an action until there is a material change in circumstance that might alter the outcome. This doesn't usually apply in combat as you are sacrificing your other combat actions each turn cycle to try anew. 

Outside of combat, however, the GM may make allowances for multiple attempts. This will most commonly occur if you have some finite resource that you are burning through. If you only have 3 fragile lockpicks, there's no particular harm in giving you 3 attempts at opening the door. 

