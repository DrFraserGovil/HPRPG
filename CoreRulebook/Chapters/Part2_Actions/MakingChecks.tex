\chapter{Performing Checks}


In general, when you want to perform an action, simply tell the GM what you wish to do. 

If it is a simple action – for example, “I walk to the shop”, then the action is completed with no further involvement. More complex actions may require a ‘check’ to be performed, to determine their success: inform the GM of what you want to do, and how you want to achieve it. Working with the GM, you then decide what check to perform, and the GM will decide the condition of the success.


\section{Forming a Dice Pool}

An action's success or failure is determined by rolling a \key{Dice Pool}, comparing each dice with a set value, and counting the number of `successes'.

\subsection{Ratings}

The number of dice which go into a Pool is determined by a character's \key{Rating} in the required area. These ratings are associated with each of the \key{Aspects}, \key{Abilities} and \key{Affinities} discussed on page \pageref{C:Aspects}, and are (generally) numbers between 0 and 7. These numbers indicate the number of dice that are allocated to the dice pool when that skill is used, and may be interpreted as a general measure of a character's ability in that field:

\newcommand\dualRow[2]{#1	&	#2	\\}
\begin{center}
\begin{rndtable}{c p{5 cm}}
\bf Ratings&\bf Summary \\
\dualRow{\emptyCape}{Usless, totally untrained}
\dualRow{\oneCape}{Beginner, marginal training}
\dualRow{\twoCape}{Novice, some training}
\dualRow{\threeCape}{Average, fair training}
\dualRow{\fourCape}{Adept, some high-level training, low level professional}
\dualRow{\fiveCape}{Expert, high-level training, trained professional}
\dualRow{\sixCape}{Master, ultimate training, famous professional}
\dualRow{\sevenCape}{Ascendant, no known rivals}
\end{rndtable}
\end{center} 

A character with 3 dots in Intelligence therefore rolls 3d12 when an Intelligence check is called for. 

\subsubsection{Combining Skills}

More often than not, you will not be using one of your base \key{Aspects} to complete a skill - you will also be utilising an \key{Ability} to complete the action. For spellcasting, this would instead be replaced by an \key{Affinity}. 

For example, leaping over a cliff clearly uses the \imp{Fitness} Aspect, but if you take a run-up to increase your range, you may utilise your \imp{Speed} ability to increase your chances of success. This would therefore be termed a \imp{Fitness (Speed)} check, and the dice pool would be formed from the sum of your dots in both the Fitness and Speed areas.

Hence, if performing a flying leap over a cavern, a character would perform the \imp{Fitness (Speed)} check, and sum their dots in the \imp{Fitness} aspect (say, 3) and the \imp{Speed} ability (2), and hence have a pool of 5 12-sided dice to use to complete the action. 

\subsubsection{Bonuses and Penalties}

You might sometimes have magical or mundane effects present which modify the number of dice you are allowed to allocate to a given pool. These modifiers cannot push the number below zero, but they may push them above 7, given you truly superhuman abilities.


\subsubsection{Zero-Pools}

Sometimes, either as a result of your own lack of ability, or because of some negative magical or mundane effect, a character may find themselves in a situation where there they have no dots to allocate to a dice pool. 

Sometimes this precludes your ability to undertake the action entirely (a \key{Knowledge} ability, for example, often requires at least one dot for the information to be retained), but often does not. In such a situation, you roll two dice, and use the lowest value. 


\section{Determing the Checktype}



The most important ingredient in performing a check is deciding how exactly you are achieving this goal. This is an exercise in roleplaying, more than a hard-and-fast rule: you must evaluate the situation and your character's capabilities and decide how they would utilise their skills to best complete the action. 

In certain situations, the GM may rule that all but a single attribute check would render an action impossible: when sitting a physics exam, nothing but an \imp{Intelligence (Science)} check is going to help you. However, the GM is encouraged to generally avoid this hardline stance, and instead state an {\it ideal} check which would result in the most success for the stated action, and then allow the players to roleplay a potentially different route to completing the action. 

For the players, the temptation is to use this as an opportunity to default to what a character is best at, in order to get the most amount of dice. However, trying to leap over a cavern using your knowledge of Biology from an \imp{Intelligence (Science)} check is not going to be very successful, and will probably lead to you plummeting to your death no matter how many dice you roll! 

However, a player {\it could} attempt to argue that such a check could still be relevant: perhaps your character has studied Newtonian Physics and so can calculate the correct angles and distances required. This might hold some sway with your GM, though you will probably have to defeat a much higher DV, or require more successes for the action to fully complete. 

Developing an interesting narrative about a character and their backstory is more important than a strict adherence to logic and rules, so as long as you can weave and roleplay a convincing narrative as to how you are using a certain skill to complete the action at hand, the GM is encouraged to let you try. The tradeoff is that outlandish skill choices are often poorly suited and therefore make a task significantly more difficult and, even if they succeed, will bring you less success than if you had used a more straitforward approach. 

The players and the GM should work together to decide upon the relevant mix of Aspects and Abilities which form the dice pool, using the combination which makes the most sense, or which provides the richest source of roleplaying material. 


\section{Success \& Failure}

After deciding upon the ingredients which will go into making the dice pool, the next thing is for the GM to determine the \key{Difficulty} of the task. This is a single number, between 2 and 12 which represents the liklihood of failure. A GM may also decide that an action is utterly impossible under the current circumstances (no matter the dice rolls, you cannot jump to the moon!). 

An example of some common difficulties is shown below:
\def\w{5}
\begin{center}
\begin{rndtable}{|c p{\w cm} c|}
\hline
Task Difficulty & 	Description & DV	
\\ 
\key{Very Easy} & \parbox[t]{\w cm}{\raggedright An everyday task that most people could be expected to carry out first time.}	&	4
\\
\key{Easy} & \parbox[t]{\w cm}{\raggedright A simple task that has only a small chance of failure.}& 6
\\
\key{Standard} & \parbox[t]{\w cm}{\raggedright A task that a normal person might require a few tries to get right} & 8
\\
\key{Hard} & \parbox[t]{\w cm}{\raggedright A task that a normal person could not carry out without specialist training} & 10 
\\
\key{Very Hard} & \parbox[t]{\w cm}{\raggedright A task that even a trained expert might struggle to complete. } & 12
\\ 
\end{rndtable}
\end{center}

Every dice rolled from the pool which meets or exceeds the DV counts towards a success. You then report back to the GM the number of success you achieved, which determines how effective the action was:


\begin{center}
	\begin{rndtable}{c p{6cm}}
		\bf \# Successes & \bf Degree \\
		\dualRow{1}{\key{Marginal}: You just scrape by completing the action, perhaps incurring some mild side effects}
		\dualRow{2}{\key{Okay}: you complete the action with a small amount of leeway. You did it, but not elegantly}
		\dualRow{3}{\key{Good}: you did the action, and you did it well}
		\dualRow{4}{\key{Complete}: you did the action, and got more than you expected.}
		\dualRow{5}{\key{Excellent}: you did the action, and achieved significantly more than you set out to do}
		\dualRow{6}{\key{Flawless}: You did the action perfectly, and got lots more besides}
		\dualRow{7+}{\key{Legendary}: You completed the action so well people will be telling stories of it for years to come.}
	\end{rndtable}
\end{center}

\subsubsection{Automatic Success}

Rolling dozens of dice may be fun, but when Barry, the most prolific athlete of his generation, wants to hop across a small stream, rolling $>10$d12 against a DV of 3 seems a little overkill. 

As a general rule, if the number of dice in the pool exceeds the DV of an action your GM may simply decide that the action completes automatically. This rule does not generaly apply in combat, or in situations where the consequences of failure are particularly dire. 

\subsubsection{Failure}

If you fail to gain any successes, the action is a failure

\subsubsection{Botches}

\subsection{Contests}

Many actions are not just one character doing something whilst the world holds its breath. Quite often the 

\section{Check Advantage}

If you have the status effect {\it Check Advantage}, or are otherwise granted this ability on certain checks, then you may perform checks twice -- and take the largest value. This decreases the likelihood of a negative outcome, and increases the likelihood of a positive one. 

Conversely, a {\it Check Disadvantage} requires you to perform a check twice and take the lower of the two values. 

Check-Advantage and Check-Disadvantage compound each other, to a limited extent. If a character already possesses check-advantage, and gets a second separate effect which also gives them check-advantage, then they are in a state of `super-advantage', in which case you roll three dice, and take the highest. Equally, two disadvantages compound into super-disadvantage. 

A disadvantage layered on an advantage cancel each other out, and a disadvantage on a super-disadvantage reduces it to normal. 

{\bf However, more than two buffs in either direction have no additional effect}. 10 disadvantages and 11 advantages are treated as 2-against-2 (i.e. a normal roll), as are 3 advantages against 10 disadvantages. 

Use the following table for reference:


\def\cc{\cellcolor{\tablecolorhead}\bf }
\begin{center}
{
\small
\renewcommand{\arraystretch}{1.4}
	\begin{rndtable}{c c c c c}
	~ & ~	&	\multicolumn{3}{c}{\bf \# Advantages}
	\\
	\cc ~	&	\cc~ & \cc 0	&\cc	1	& \cc 2+ 
	\\
	\cc~& \cc 0	&	Normal	&	Advantage	&	Super Advantage
	\\
	\cc~& \cc 1	&	Disadvantage	&	Normal	&	Advantage
	\\
	\multirow{-4}{*}{\rotatebox[origin=c]{90}{\cc \bf \# Disadvantages}} & \cc 2+	&	Super Disadvantage	&	Disadvantage	&	Normal 
	\end{rndtable}
}
\end{center}


For (dis)advantages to compound, they must arise from totally different sources - drinking two potions which both provide Advantage will not give super advantage, but being invisible {\it and} drinking a potion would. 

\section{Working Together}

Occasionally two or more characters might decide that, together, they have a better chance of succeeding in a given task, and can work together. A character may only help if they could perform the action themselves (so you could only help pick a lock if you also had proficiency in lockpicking tools), or if you can provide a reasonable justification for how you are helping the action succeed (an untrained individual could help an engineer fix an engine by passing them tools, and holding a flashlight, for example). 

When working together like this, the character with the highest relevant modifier performs the check with check-advantage. 

Sometimes, you might need to complete a task where the entire group needs to succeed, but the group may help each other -- for example, if the entire group needs to jump across a ravine, or if the entire group is searching for a single hidden item. The GM may decided on the most appropriate course of action, but a general first-start is to ask all members of the group to perform the check -- if at least half of the group succeed, the entire group succeeds. 


\section{Multiple Attempts}

Sometimes, after an action fails, a character may want to try again immediately. This is generally to be discouraged - it makes the game less fun if everyone is just waiting for Mike to (finally) roll a 20. 

A general rule is that you can't repeat an action until there is a material change in circumstance that might alter the outcome. This doesn't usually apply in combat as you are sacrificing your other combat actions each turn cycle to try anew. 

Outside of combat, however, the GM may make allowances for multiple attempts. This will most commonly occur if you have some finite resource that you are burning through. If you only have 3 fragile lockpicks, there's no particular harm in giving you 3 attempts at opening the door. 

If, however, a character is attempting to `spam' a check -- i.e. just keep rolling the dice until they succeed, and it makes enough narrative sense that the GM doesn't overrule it, then they instead ask you to roll a d100 on the table found on page \pageref{S:Multi}, which will determine the number of failed attempts. 

