

\section{Downtime}\index{Downtime}

In addition to performing non-combat actions in between individual conflicts, you may occasionally find yourself with a considerable amount of time to spare -- in which you can devote entire days to activities that further your character, heal them from egregious injuries, or earn some spare cash. 

Given that extended downtime will probably be taking place in population centres, you will need to find enough resources to live a normal life -- particularly food and shelter. See page \pageref{S:Shelter} for more details. 


\subsection{Working}\index{Employment|see {Downtime (Working)}}\index{Downtime!Working}

Perhaps one of the mos useful things you can do is try to bolster your finances with some hard work. You may find the kind of jobs available limited by the area you are in -- a tiny village isn't going to have much call for a librarian, and a a bustling city won't have much need for a thatcher. You will need to search out clients or an employer to practice your skills. 

In general, the payment one can expect to recieve varies depending on how skilled the job is you perform, though again, the region you are in might have an economic boom in one area, or a financial collapse, which alters these wages:
\begin{center}
	\begin{rndtable}{c p{3cm} c}
	Skill 	&	Examples	&	Wage 
\\
	Unskilled	&	Manual labour, farmwork &	\galleon{1} per week
	\\
	Moderate	&	Shopwork, guard	&	\galleon{0.5} per day
	\\
	Skilled	&	Teacher, performer, nurse	&	\galleon{1} per day
	\\
	Highly skilled	&	Artificier, surgeon	&	\galleon{2} per day
	\end{rndtable}
\end{center}

\subsection{Crafting}\index{Downtime!Crafting} \index{Artificing}

Witches, wizards and many other sentient species in the world rely on the production of magic potions and enchanted items for their day-to-day life. Downtime is a perfect time to attempt to get in on this. See the rules for artificing on page \pageref{S:Artificing} for more details on magical item creation.

In addition, you may also manufacture or assemble non-magical items during your downtime, if you have access to the necessary raw material, tools and machinery required. A general rule is that you can only manufacture goods up to a value of \galleon{1} per day. If you wish to exceed this value, you need to spend multiple days performing the task, following the extended project rules on page \pageref{S:Extended}.

\subsection{Recuperating}\index{Downtime!Recuperating}\index{Health!Long-Term Healing}

Although not a substitute for seeking genuine medical attention, a long period of rest may allow you to recover from even the most serious of injuries. 

After at least 7 days of rest, you may perform a pure \imp{Vitality} check (DV 10), each success allows you to remove that amount of harm. If a character with \imp{Level 5 Harm} succeeds with 2 successes, they would reduce this to only \imp{Level 3 Harm}.


\subsection{Researching}\index{Downtime!Researching}

Downtime is also the perfect time to go searching for new knowledge, whether it is to find new information about mysteries that have been partially revealed to you, to find new and interesting types of magic, or to learn about weaknesses and habits of the magical and dangerous beasts that roam nearby. You may find libraries to comb through for fusty old tomes, or go out and speak to people and try to extract local knowledge from them. 

Tell the GM what information you are looking for, and the route you will take to finding it. They will determine if the information is available, and then how long you have to spend before you hit the jackpot. 

This might also include Persuasion checks, or Research checks, to determine how well your character performs their research. 
 

%~ \subsection{Training}

%~ You might also dedicate your time to training in a new skill: learning to use new weapons, new languages, new magic, or new tools. 

%~ Though not nearly as useful an experience as real-life experience, this can be an important aspect of preparing yourself for the trials and tribulations you will face. 

%~ In order to train, you will need to find an experienced person, willing to teach you. The classes cost around \galleon{3} per day, though if the skill you are attempting to learn is particularly rare, or the teacher particularly noteworthy, the classes may cost more.  

%~ 5 weeks worth of dedicated practice (\galleon{75}) is enough to call yourself proficient in the field, and you may take up a proficiency in a tool, weapon, or language of your choice. Note that training with a weapon gives you proficiency {\it only} in that weapon, not in the entire class of weapons associated with that weapon, to learn an entire class of weapons would take 10 weeks worth of dedicated practice. 

%~ If you find a magic teacher, they may help you memorise new spells without risking yourself. Spending two days is enough to memorise a new spell, though a teacher can only help you with spells they themselves have memorised. 
