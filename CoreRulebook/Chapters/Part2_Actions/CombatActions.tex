
\chapter{Combat}

\section{The Combat Cycle}
Unlike most RPGs, which tend to use a turn-based system for combat, this game uses a simultaneous combat system. The reason for this is that whilst the turn-based combat fits in with how we play games (I have my turn, you have yours, etc.), it is not entirely realistic: in a fight, you don't wait patiently for everyone else to complete attacking you before finally returning fire: everybody is completing actions at once. 

After combat is initiated, a series of turn cycles occur. Each turn cycle allows every character in combat one major action, such as: a movement, casting a spell, or using an item. 

At the start of each turn cycle there is a period of time (to be decided by your GM), during which you must decide on what you will do. Players may talk to each other during this time, but do be aware that discussing your tactics in front of the GM may give the game away, you wouldn't start shouting your plan out whilst fighting the enemy now, would you? 

After this time is up, each player writes down their action on a scrap of paper (to prevent last minute changes of heart), and then all players (including the GM) reveal their action simultaneously. 

The GM then resolves the effects of all these actions - directing characters to perform accuracy and damage checks where appropriate - and then narrating the outcome, and the response (if any) of the remaining aggressors. 

The combat cycle then begins anew until the conflict is resolved. 

\subsection{Time}

Each combat cycle is assumed to have a duration of around 3 seconds. 

Attempting to perform actions that last significantly longer than this requires spreading the action across multiple turns -- though may choose to abort such an action if you feel your talents are better placed elsewhere. 

\subsection{Resolving Conflicts}

Since all actions are considered to be simultaneous, the order in which the actions are resolved does not usually matter. Recall that spells, arrows, and sword swings have a finite travel time, so it is entirely feasible for two players to attack each other simultaneously and it does not matter who initiated first.

It might, of course, still be possible for actions to come into conflict with each other: if two characters attempt to occupy the same space, for example. It is up to the GM's discretion how to deal with edge cases like this - for the example given, it is recommended that this be treated as a `body slam', and both characters should recoil and take some damage. 

There might also be cases where two spells are cast simultaneously where the ordering does actually matter: for example, if you heal someone at the same time that someone casts a damaging spell that would take them below 50\% health, incurring the ``major injury{\apos \apos} status. If the healing action occurs first, then they are not taken below 50\% health, but if the damage action occurs first, then they do fall below 50\%, even if they are then brought back up over that threshold. The final health that the character ends up on might be the same, but the ordering of actions effects whether they have the {\it major injury} status at the end of the turn. 

In cases such as this it is useful to remember that it is the {\it casting} of the spell that is simultaneous: so the ordering in which the spell effects should take place can be inferred from the distance between the caster and the target. The issue above is resolved simply by looking at whoever is closest to the target. 
 

 \section{Taking Actions}
 
 During each combat cycle, each character may take {\bf one} major action, or {\bf two} minor actions. In addition, your character has a number of {\it instincts} which they execute to avoid damage and brace against incoming attacks. 
 
 The list below gives some common mechanics for both major and minor actions. As usual, however, characters are free to be as inventive as they like. It is up to the GM to determine if an action is major or minor in nature, and how to resolve it. 
 
 \subsection{Major Actions}
 
 Major actions take an entire turn to complete, and as such are considered the main way to engage in combat. Some skills and archetype abilities allow you to perform multiple iterations of a single major action per turn, or may grant you multiple major actions to take. 
 
 \subsubsection{Attacking}
 
 Casting a spell, swinging a sword, or loosing an arrow takes (usually) a full turn to complete, and so you may decide to use your entire turn to make an attack.
 
The rules for performing attacks are elaborated on page \pageref{S:Attacks}
 \subsubsection{Movement}
 
 When used as a major action, movement allows you to move on foot up to a distance given by your {\it running speed} statistic, which is calculated from your base speed (derived from your race) and your fitness attribute:
\small
$$ \text{running speed} = \big(\text{Base Speed } + \text{\attPhys{} modifier}\big) \text{ per round} $$  
\normalsize

The rules discussed on page \pageref{S:SpecialMovement} concerning special movement, such as climbing, swimming or crawling, also apply in combat. 

{\bf Sprinting:} If you possess the {\it Speed} proficiency and you made a full-turn movement last cycle, you may convert your movement into a {\it sprint}, and add your expertise bonus to your speed. You may then maintain this until you need to stop or change direction. 

Whilst moving, you need to be careful that you do not collide with other beings - either your allies or your enemies. You cannot enter space that is currently being occupied by another solid being (ghosts, however, are fair game). 

 \subsubsection{Using Items (sometimes)}
 
Some `uses' of items include using swords, wands and ranged weapons, which have already been covered by `attacking'. 

However, sometimes you might want to use an action to get something big done, outside of hitting somebody. Using a crowbar to pry open a door, changing your weapon, finding the right page of a book -- all of these take enough time to be considered major actions. 

Some uses might take multiple turns -- for instance, climbing into a full suit of armour takes more than 3 seconds to complete, and will therefore require multiple, consecutive major actions. 

In contrast, some actions (see below) are small enough to be considered minor actions. The GM has veto on which actions are major or minor. 

\subsubsection{Trading Items}

If two characters are standing within touching distance, they may trade items between them. 

Alternatively, you may attempt to throw an item to your ally, treating the item as an `improvised weapon'. If the throwing check is successful, the catcher adds the item to their inventory. 

Whichever method is chosen, giving items to other people takes the major actions of both the giver and the receiver. 


\subsection{Minor Actions}
You may perform two minor actions in place of a single major action. Generally, these two actions happen simultaneously: if you drink a potion and make a minor movement, then you are drinking the potion whilst moving. This places a good guide on what can be considered a minor action: is it possible to do this at the same time as I'm walking/talking/dodging? 

\subsubsection{Minor Movements}

Actions such as taking a single step, or peeking out from behind cover, do not take any time, and can be performed in the same turn as a major action. 

However, there is a middle ground between the sprint of a full-turn movement, and the zero-time of a single step. This is called a {\it minor movement}. 

During a minor movement, one moves only {\bf half as far} as during a full-turn movement, but since you are not focussed solely on moving as far as possible, you can perform other minor actions. 

\subsubsection{Quick Attack}

Just as there is a difference between a full-on sprint (a major action) and a quick jog (a minor action), so to is there a difference between a zeroed in shot on your enemy (a major action), and releasing a spray of covering fire to keep your enemies on their toes (a minor action). 

A quick attack takes only a minor action to complete. The penalty for this, however, is that you must take check-disadvantage on the associated accuracy checks (or for spells which only require a Resist check, they get advantage on the Resist check). 


\subsubsection{Communication}

Communicating vital information - such as the location of a hidden enemy or trap - to your comrades takes a minor action. Note that it is assumed that the enemy can hear you, unless you make an effort to not be understood. 

\subsubsection{Using Items (sometimes)}

Item use has already been discussed as a major action, but there are conceivably such actions that would fall into the minor action category. Consuming a potion, checking a rememberall, removing an item from your bag and so on would be considered `minor actions'. 

Any item use that can be completed in around 1 second, or which can be easily `multitasked', is considered a minor action. 

\subsubsection{Bolstering Defenses}
 
 You may also choose to ready yourself against incoming attacks, by bolstering your ability to either {\it Dodge} or {\it Block}. This gives you a better chance of negating incoming effects.  

See page \pageref{S:Accuracy} for more details on this mechanic. 



\subsection{Conditional Actions}

The use of the simultaneous combat system raises some interesting opportunities with conditional actions, which are actions that depend on the actions that another character takes.

The actual action, as well as the trigger condition, needs to be declared during the normal turn cycle -- but the action itself is not triggered until all other actions had been triggered. 

For example, it could be that you declare as your action \textit{if the troll attacks player A, then I cast a healing spell on player A}. You could also attempt to prevent the damage from being taken in the first place, by declaring \textit{if the troll attacks player A, then I cast the knockback charm on the troll}. The GM may ask for a check to determine if you are close enough and have fast enough reactions for your spell to interrupt the action, but if you pass this, then you may be able to save your friend.

You are only allowed a single conditional clause in your declaration, and if that conditional does not come to pass, then your character does not do anything: there is no \verb|if-then-else| in this game!

If a seemingly unbreakable condition-chain arises (i.e. player A says he will perform X if player B does Y, but player B says he will only perform Y if player A does X), it is up to the GM to resolve the conditionals -- in such cases the answer is usually \textit{nothing happens}, but there may be examples where the GM feels it is more appropriate that the action-chain is triggered. 


