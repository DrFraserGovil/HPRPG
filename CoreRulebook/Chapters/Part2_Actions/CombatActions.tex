
\chapter{Combat}

\section{The Combat Cycle}

In real life, combat is a dangerous, messy and chaotic affair. This does not lend itself overly well to a game, as it will inevitably descend into a rush of people talking over each other as everyone attempts to defend and attack at the same time 

Therefore, when entering combat play enters into a more structured format. A number of {\it Turn Cycles} occur one after another, during which each player reveals and resolves their actions in turn. However, note that this is merely an abstraction from the chaos of several actions going on at once: every action in a turn cycle is considered to be simultaneous with every other action. The order in which events occur is as follows:

\begin{itemize}
	\item All attacks and movements occur simultaneously
	\item All reactions to the main actions
	\item Any Conditional actions 
	\item Any reactions to the conditional actions
\end{itemize}

Because of fixed ordering of events, the order in which players take their turns does not matter. The GM may elect to go around the table in a certain direction or use any other method to decide who announces their action first, usually leaving characters under the GM's control until last (to prevent too much metagaming). During your alloted slot, you announce any actions you will be taking, and then perform the necessary rolls to resolve them.

If at any point an action executed before you in the annoucement order would conflict with an action you are planning on taking, you may alert the GM and jump up in the order in order to execute your action\footnote{This is similar to the system used in the popular boardgame {\it Diplomacy} } before your chosen action becomes irrelevant or impossible. 

For instance, consider a fight where Jonathan and Susan are fighting an escaped Runespoor. The GM tells Jonathan's player that he is to go first, Susan second and then the Runespoor last. Jonathan is running low on HP, and so elects to flee. Unfortunately, the GM then announces that the Runespoor was using its action to attack Jonathan this turn - so the Runespoor jumps up in the turn order, quickly executes its attack, before Jonathan continues with his movement. 

Equally if, going first in the cycle, the Runespoor elects to run and take cover, both Jonathan and Susan may announce that they were intending to attack the Runespoor this turn, and so they may jump ahead in the order and resolve their attacks on the fleeing creature before it slips out of sight. When the turn cycle returns to them, they may use any remaining actions, reactions or movement as appropriate.

Any damage or status effects applied due to attacks are applied in a special `results' phase which occurs at the end of the turn cycle. It is during this phase that all attacks are assumed to land (or miss), therefore spell effects do not take hold until the beginning of the next round, even if they were announced at the very beginning of the Turn Cycle, unless the effect explicitly states that its effect applies this turn.  



\subsubsection{Using Minis}

If you are using miniatures or tokens to represent the positions of your characters and their enemies during combat, it can be helpful to have two tokens to represent a character: one to represent where a character is whilst all players are deciding upon and resolving their actions, and a second to denote the position they have elected to move to. 

At the end of the turn cycle, all the simultaneous actions `execute', and the characters move to their chosen locations. 

This allows you to keep track of both where characters are moving to, and also what characters can still see as any remaining actions this turn cycle are resolved. 

\subsection{Time}

Each combat cycle is assumed to have a duration of around 3 seconds. 

Attempting to perform actions that last significantly longer than this requires spreading the action across multiple turns -- though may choose to abort such an action if you feel your talents are better placed elsewhere. 

If an effect or action has a specified duration, such as a spell which lasts for 10 seconds, this is measured from one `Result' phase to another. 

If any part of the duration of an effect overlaps with a cycle, it is assumed to apply to all of it. The `10 second' effect, therefore, applies over the next 4 combat cycles after the effect is applied. 


 \section{Taking Actions} \label{S:CombatActions}
 
 During each Turn Cycle, you may decide how to allocate your character\apos{}s time during this combat cycle. Every character may take the following actions: 
 
 \begin{itemize}
	\item One minor-action Movement
	\item One major action, or two minor actions
	\item (Potentially) a reaction
 \end{itemize}
 
 In addition, a character has a number of {\it instincts} which they automatically execute to avoid damage and brace against incoming attacks. 
 
 The dedicated ``movement'' action can be allocated as any number of smaller sub-movements (within reason). You may therefore move half your minor-movement speed, stop to use a major action, and then use your remaining movement. Alternatively you may use your movement entirely before, after and even during your other actions.  
 
 The list below gives some common mechanics for both major and minor actions. As usual, however, characters are free to be as inventive as they like. If it is not counted in the actions below, is up to the GM to determine if an action is major or minor in nature, and how to resolve it. 
 
 \subsection{Major Actions}
 
 Major actions take virtually the entire turn to complete, and as such are considered the main way to engage in combat. Some skills and archetype abilities allow you to perform multiple iterations of a single major action per turn (i.e. 3 attacks as one major action), or may grant you multiple major actions to take, overriding the normal allocation. 
 
 \subsubsection{Attacking}
 
 Casting a spell, swinging a sword, or loosing an arrow takes (usually) a full turn to complete, and so you may decide to use your entire turn to make an attack.
 
The rules for performing attacks are elaborated on page \pageref{S:Attacks}.

 \subsubsection{Movement}
 
 When used as a major action, movement allows you to move on foot up to a distance given by your {\it running speed} statistic, which is calculated from your base speed (derived from your race) and your fitness attribute:
\small
$$ \text{running speed} = \big(\text{Base Speed } + \text{\attPhys{} modifier}\big) \text{ metres per round} $$  
\normalsize

The rules discussed on page \pageref{S:SpecialMovement} concerning special movement, such as climbing, swimming or crawling, also apply in combat. 

{\bf Sprinting:} If you possess the {\it Speed} proficiency and you made a full-turn movement last cycle, you may convert your movement into a {\it sprint}, and add your expertise bonus to your speed. You may then maintain this until you need to stop or change direction. 

Whilst moving, you need to be careful that you do not collide with other beings - either your allies or your enemies. You cannot enter space that is currently being occupied by another solid being (ghosts, however, are fair game). 

 \subsubsection{Using Items (sometimes)}
 
Some `uses' of items include using swords, wands and ranged weapons, which have already been covered by `attacking'. 

However, sometimes you might want to use an action to get something big done, outside of hitting somebody. Using a crowbar to pry open a door, changing your weapon, finding the right page of a book -- all of these take enough time to be considered major actions. 

Some uses might take multiple turns -- for instance, climbing into a full suit of armour takes more than 3 seconds to complete, and will therefore require multiple, consecutive major actions. 

In contrast, some actions (see below) are small enough to be considered minor actions. The GM has veto on which actions are major or minor. 

\subsubsection{Trading Items}

If two characters are standing within touching distance, they may trade items between them. 

Alternatively, you may attempt to throw an item to your ally, treating the item as an `improvised weapon'. If the throwing check is successful, the catcher adds the item to their inventory. 

Whichever method is chosen, giving items to other people takes the major actions of both the giver and the receiver. 


\subsection{Minor Actions}
You may perform two minor actions in place of a single major action. Generally, these two actions happen simultaneously: if you drink a potion and make a minor movement, then you are drinking the potion whilst moving. This places a good guide on what can be considered a minor action: is it possible to do this at the same time as I'm walking/talking/dodging? 

\subsubsection{Minor Movements}

Actions such as taking a single step, or peeking out from behind cover, do not take any time, and can be performed in the same turn as a major action. 

However, there is a middle ground between the sprint of a full-turn movement, and the zero-time of a single step. This is called a {\it minor movement}. 

During a minor movement, one moves only {\bf half as far} as during a full-turn movement, but since you are not focussed solely on moving as far as possible, you can perform other minor actions. 

\subsubsection{Quick Attack}

Just as there is a difference between a full-on sprint (a major action) and a quick jog (a minor action), so to is there a difference between a zeroed in shot on your enemy (a major action), and releasing a spray of covering fire to keep your enemies on their toes (a minor action). 

A quick attack takes only a minor action to complete, however their rapid and slightly careless nature imposes the following effects:
\begin{itemize}
	\item All accuracy checks are performed with disadvantage
	\item All Resist attempts performed by the target of your attack are performed with advantage
\end{itemize}

You can only cast `offensive' spells as a Quick-Attack action, i.e. those which inflict harm or negative effects directly on your opponents. ``Utility'' spells, even if they would inflict harm eventually, require more attention and focus than a careless Quick-Attack can provide. 
 
\subsubsection{Communication}

Communicating vital information - such as the location of a hidden enemy or trap - to your comrades takes a minor action. Note that it is assumed that the enemy can hear you communicating, unless you make an effort to not be understood. 

\subsubsection{Using Items (sometimes)}

Item use has already been discussed as a major action, but there are conceivably such actions that would fall into the minor action category. Consuming a potion, checking a rememberall, removing an item from your bag and so on would be considered `minor actions'. 

Any item use that can be completed in around 1 second, or which can be easily `multitasked', is considered a minor action. 

\subsubsection{Bolstering Defenses}
 
 You may also choose to ready yourself against incoming attacks, by bolstering your ability to either {\it Dodge} or {\it Block}. This gives you a better chance of negating incoming effects.  

See page \pageref{S:Accuracy} for more details on this mechanic. 

\section{Reactions} \label{S:Reactions}

A Reaction occurs when a being responds to some action being imposed upon them. Reactions are declared when another being declares an action such as an attack, and as normal, allow you to jump up in the turn order, quickly attempt to modify or nullify the action, before the turn cycle proceeds as normal. 

In general, the most common uses of a reaction are to use one of the following:

\begin{itemize}
	\item Cast a {\it Shield} spell, or a {\it Counterspell} to negate a magical attack
	\item Use an {\it Evade} or {\it Brace} action to bolster your defences. 
\end{itemize}

Using a Reaction is not something you can do reliably, as it requires a clarity of mind and purpose. {\bf After using a Reaction, you temporarily lose the ability to use one} until you can regain your composure. 

During the {\it Results} phase of every turn, roll a d6. On a 5-6, you regain your ability to use a reaction.

\subsubsection{Reaction Spells}

You may only cast a spell as a reaction if you have memorised it. Bookcasting takes too long to work effectively as a Reaction. However, all spells which are cast as a reaction {\it must} perform a spellcasting check, as the casting is occuring under immense time pressure. 

In addition to the two spells mentioned, you may also cast more complex spells, with the limitation that they must solely be focussed on denying the effect you are reacting to, and should never include attacks without an ability saying otherwise. It would, for example, be acceptable to cast {\it Compel Flames} to redirect a fire-based attack, though you would need to upcast it to a suitable level (usually equal to the attacking spell\apos{}s level) in order for it to be effective, but you would not then be able to direct the flames to attack an enemy as part of your reaction. 

If a spell is cast silently then it is impossible to identify what spell is being cast before it is too late, and so reactions to silent-cast spells must rely on more basic protection, or simply guess at what the spell {\it might} be. 



\subsection{Conditional Actions}

The use of the simultaneous combat system raises some interesting opportunities with conditional actions: those which depend on the actions that another character takes.

The actual action, as well as the trigger condition, needs to be declared during the normal Reveal Phase -- but the action itself is treated as a `Reaction'. You may use a conditional action even if you have not yet recovered your reaction, however you may not expend a reaction on the same turn as using a conditional action. 

For example, it could be that you declare as your action \textit{if the troll attacks player A, then I cast a healing spell on player A}. You could also attempt to prevent the damage from being taken in the first place, by declaring \textit{if the troll attacks player A, then I cast the knockback charm on the troll}. The GM may ask for a check to determine if you are close enough and have fast enough reactions for your spell to interrupt the action, but if you pass this, then you may be able to save your friend.

You are only allowed a single conditional clause in your declaration, and if that conditional does not come to pass, then your character does not do anything: there is no \verb|if-then-else| in this game!

If a seemingly unbreakable condition-chain arises (i.e. player A says he will perform X if player B does Y, but player B says he will only perform Y if player A does X), it is up to the GM to resolve the conditionals -- in such cases the answer is usually \textit{nothing happens}, but there may be examples where the GM feels it is more appropriate that the action-chain is triggered. 


