
\chapter{Combat Basics}\index{Combat}

\section{The Combat Cycle}\index{Combat!Combat Cycle}\index{Initiative Order|see{Combat Cycle}}\index{Reflexes|see{Combat Cycle}}

In real life, combat is a dangerous, messy and chaotic affair. This does not lend itself overly well to a game, as it will inevitably descend into a rush of people talking over each other as everyone attempts to defend and attack at the same time 

Therefore, when entering combat play enters into a more structured format, known as the \key{Reaction Cycle}, or the \key{Combat Cycle}. It is critical to note that, although this breaks the action down into a nice cycle of resolving actions one at a time, this is an abstraction from the chaos of combat: every action declared in a single \imp{Reaction Cycle} is occuring near-simultaneously.

Since every action occurs all at once, what matters is not necessarily who acts first, but what information you have before you are forced to act, and therefore what you are able to reasonably react to.

Upon entering combat, every character and being involved in the conflict rolls a single d12, and adds the result of this dice to the number of dice in their \imp{Perception (Alertness)} pool. This is known as a \key{Reflex Roll}:
$$ \text{\imp{Reflex Roll}} = \text{1d12 + \imp{Perception} value} $$

The GM then orders the players and the antagonists in order of their \imp{Reflex Roll}. Any ties are resolved by mutual agreemet (if between players), by comparing \imp{Perception} values (highest wins), and if necessary, a simple roll-off until a winner is decided.

The character at the top of the resulting \key{Reflex Order} order may choose to be the initiator and voluntarily choose to act first -- this in effect places them at the bottom of the \imp{Reflex Order}, but potentially allows them to put everyone else on the back foot by taking control of the situation. Alternatively, they may choose to use their magnificent reflexes to wait and see what happens.

Actions are then announced in {\bf Reverse Order} - the characters who rolled a low \imp{Reflex Roll} must announce their actions first, without knowing what those higher up in the order are up to. This allows characters with a better \imp{Reflexes} to react to actions going on around them, and potentially counter and negate actions made by those with worse reflexes than them.  

\subsection{Applying Effects}

After all actions have been announced, the effects of all actions this round are applied. 

Because the effects are explicitly not applied until after all actions are announced, it is perfectly possible for two characters to attack each other (and indeed, stun or even kill each other) on the same turn. The character's utter the spells simultaneously, with the spells crossing in midair.

\subsection{Jumping Initiative} \index{Combat!Jumping Initiative}

Of course, it could get a little tricky to keep track of all the `pending' effects that have been announced, remembering the effects of the seven spells cast this round, and how they have been modified by attempts to dodge or shield against them is obviously beyond even the most dedicated GM. 

It is therefore encouraged that players should `jump initiative' when they wish to modify the effects of a spell that is being cast by someone lower down the initiative than them. Consider for example a duel between John, his arch-rival Samantha and 3 others. Samantha is at the bottom of the reflex cycle, whilst John is at the top. 

Samantha goes first and announces that she is casting a \levelThree{} blast of fire, directed towards John. John is pretty hurt already, and knows that, no matter what, he needs to deflect this fire blast, so he tells the GM that he would like to `jump initiative' and expend his action out of the usual reflex cycle to immediately cast a \levelTwo{} shield spell. 

John would, of course, be allowed to wait to see what the remaining three characters did before doing this -- but for ease of keeping track of what is happening, it is encouraged that players jump up in initiative in order to allow all effects related to an individual to be resolved at once. 

A benign GM may allow a player to `undo' this action if it turns out to have been a catastrophic error (i.e. if one of Samantha's allies releases a much more powerful attack which would have obviously been the one they would have shielded against the most strongly), but this should not be used to escape from a bad dice roll - this should be used only whe the additional information would genuinely have altered the course of action. The GM may choose to simply transfer the previous dice roll to the new attempt, or ask for an entirely new dice roll. 


\subsection{Time}

Each \imp{combat cycle} is assumed to have a duration of around 3 seconds. 

Attempting to perform actions that last significantly longer than this requires spreading the action across multiple turns -- though may choose to abort such an action if you feel your talents are better placed elsewhere. 

If an effect or action has a specified duration, such as a spell which lasts for 10 seconds, this is measured from the end of a \imp{turn cycle}.

If any part of the duration of an effect overlaps with a cycle, it is assumed to apply to all of it. The `10 second' effect, therefore, applies over the next 4 \imp{combat cycles} after the effect is applied. 


 \section{Major and Minor Actions} \label{S:CombatActions}
 
 During each \imp{Turn Cycle}, you may decide how to allocate your character\apos{}s time during this combat cycle. Every character may take the following actions: 
 
 \begin{itemize}
	\item One normal \key{Movement}
	\item One \key{major action}, or two \key{minor actions}
 \end{itemize}
 
 The dedicated \imp{movement} action can be allocated as any number of smaller sub-movements (within reason). You may therefore move half your movement speed, stop to use a \imp{major action}, and then use your remaining movement. Alternatively you may use your movement entirely before, after and even during your other actions.  
 
 The list below gives some common mechanics for both \imp{major} and \imp{minor} actions. As usual, however, characters are free to be as inventive as they like. If it is not counted in the actions below, is up to the GM to determine if an action is major or minor in nature, and how to resolve it. 
 
 \subsection{Major Actions}\index{Combat!Major Actions}
 
 \key{Major actions} take virtually the entire turn to complete, and as such are considered the main way to engage in combat. Some skills and archetype abilities allow you to perform multiple iterations of a single major action per turn (i.e. 3 attacks as one major action), or may grant you multiple major actions to take, overriding the normal allocation. 
 
 \subsubsection{Attacking}\index{Combat!Major Actions!Attacking}\index{Attacks|see{Combat}}
 
 Casting a spell, swinging a sword, or loosing an arrow takes (usually) a full turn to complete, and so you may decide to use your entire turn to make an attack.
 
The rules for performing attacks are elaborated on page \pageref{S:Attacks}.


	\subsubsection{Defending} \index{Combat!Major Actions!Negating} \index{Blocking|see{Combat, Negation}} \index{Dodging|see{Combat, Negation}} \index{Enduring|see{Combat, Negation}}
	
	If an attack is on its way towards you (or you suspect it will be soon), you may attempt to negate the effects of it. 
	
	You may choose to \key{Block}, \key{Dodge} or \key{Endure} an attack, rolling dice pools as appropriate. Alternatively, you may attempt to cast a defensive spell to divert or otherwise dilute the power of incoming attacks. 
	
	Every success reduces the \imp{Power} of the incoming attack by 1 point. Successes may be divided up between any attacks which are directed towards you this turn cycle. 
	
	See page \pageref{S:Defence} for more detail.
	
 \subsubsection{Movement}\index{Combat!Major Actions!Movement}\index{Movement!Combat Movement}
 
 When used as a major action, movement allows you to move on foot up to a distance given by your {\it running speed} statistic, which is calculated from your \imp{Speed} attribute:
\small
$$ \text{running speed} = \big(\text{3 } + \text{\imp{speed} rating }\big) \text{ metres per round} $$  
\normalsize

The rules discussed on page \pageref{S:SpecialMovement} concerning special movement, such as climbing, swimming or crawling, also apply in combat. 

Whilst moving, you need to be careful that you do not collide with other beings - either your allies or your enemies. You cannot enter space that is currently being occupied by another solid being (ghosts, however, are fair game). 

 \subsubsection{Using Items (sometimes)}\index{Items!Usage}
 
Some `uses' of items include using swords, wands and ranged weapons, which have already been covered by `attacking'. 

However, sometimes you might want to use an action to get something big done, outside of hitting somebody. Using a crowbar to pry open a door, changing your weapon, finding the right page of a book -- all of these take enough time to be considered major actions. 

Some uses might take multiple turns -- for instance, climbing into a full suit of armour takes more than 3 seconds to complete, and will therefore require multiple, consecutive major actions. 

In contrast, some actions (see below) are small enough to be considered minor actions. The GM has veto on which actions are major or minor. 

\subsubsection{Trading Items}

If two characters are standing within touching distance, they may trade items between them. 

Alternatively, you may attempt to throw an item to your ally, treating the item as an `improvised weapon'. If the throwing check is successful, the catcher adds the item to their inventory. 

Whichever method is chosen, giving items to other people takes the major actions of both the giver and the receiver. 


\subsection{Minor Actions}\index{Combat!Minor Actions}
You may perform two minor actions in place of a single major action. Generally, these two actions happen simultaneously: if you drink a potion and make a minor movement, then you are drinking the potion whilst moving. This places a good guide on what can be considered a minor action: is it possible to do this at the same time as I'm walking/talking/dodging? 

\subsubsection{Take Stock}

When a character \key{Takes Stock}, they take a moment to step back and evaluate the situation, and allow themselves to gain a bit more initiative. At the end of the current \imp{Cycle}, if the character is not incapacitated, they may re-roll their \imp{Reflex Roll}, and may choose to use either the new or the result.  

If more than one individual \imp{Takes Stock} on a single round of combat, then instead, all characters must re-roll their \imp{reflex rolls}, and be forced to use the new result.

\subsubsection{Minor Movements} \index{Combat!Minor Actions!Minor Movement}

Actions such as taking a single step, or peeking out from behind cover, do not take any time, and can be performed in the same turn as a major action. 

However, there is a middle ground between the sprint of a full-turn movement, and the zero-time of a single step. This is called a {\it minor movement}. 

During a minor movement, one moves only {\bf half as far} as during a full-turn movement, but since you are not focussed solely on moving as far as possible, you can perform other minor actions. 

\subsubsection{Communication} \index{Combat!Minor Actions!Communication}

Communicating vital information - such as the location of a hidden enemy or trap - to your comrades takes a minor action. Note that it is assumed that the enemy can hear you communicating, unless you make an effort to not be understood. 

\subsubsection{Using Items (sometimes)}\index{Items!Usage}

Item use has already been discussed as a major action, but there are conceivably such actions that would fall into the minor action category. Consuming a potion, checking a rememberall, removing an item from your bag and so on would be considered `minor actions'. 

Any item use that can be completed in around 1 second, or which can be easily `multitasked', is considered a minor action. 

\subsubsection{Bolstering Defenses}
 
 You may also choose to ready yourself against incoming attacks, by bolstering your ability to resist incoming attacks. Whilst not as powerful as dedicating a whole \imp{Major Action} to the effort, it is nonetheless better than nothing. 

