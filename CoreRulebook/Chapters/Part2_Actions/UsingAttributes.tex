\section{Using Each Attribute} \label{S:Proficiencies}

Almost every task a character attempts falls into one of the 8 abilities. In this section, the kinds of actions associated with each Attribute, and the encapsulated proficiencies is elaborated on in more detail. 


\def\itdef{\renewcommand\labelitemi{-}
\itemsep-0.5em}
\newcommand\proficiency[2]
{
	\textbf{\textit{#1}}: {\raggedright #2} 
}

\subsection{\attPhys{}}

\attPhys{} measures your ability to exert yourself physically. 


A \attPhys{} check will be required almost every time a being attempts to do something more strenuous that break into a light jog, or lift a heavy backpack. It is used to run, jump, swim and climb, as well as wielding heavy weapons and beating down doors. 

{\it Speed}, {\it Strength} and {\it Vitality} checks generally fall under the \attPhys{} umbrella:

\proficiency{Speed}{A \attPhys{} (Speed) check is used in situations where you need to act and move quickly, or to exert an explosive burst of speed, such as fleeing from a ravenous beast or running down an escaping prisoner. }

\proficiency{Strength}{A \attPhys{} (Strength) check is needed whenever you utilise the raw power of your muscles. For example: 
\begin{itemize}
\itdef
\item Attempting to break down a locked or jammed door
\item Wrestling a beast's jaws shut to prevent it from biting others
\item Move an extremely heavy object 
\item Break free of restraints
\end{itemize}
}

\proficiency{Vitality}{Your \attPhys{} (Vitality) check measures the physical well-being and fortitude of a character. A higher value means you can stave off the effects of starvation, exhaustion and resist the effects of diseases and poisons. Vitality is mostly a passive ability, and hence will most commonly be used in the form of Resist checks to evade the harmful effects of the environment of malicious acts. }

\subsubsection{Melee Weapons}

In addition, \attPhys{} is used as the primary attribute for most melee weapons and hand-to-hand combat. The \attPhys{} modifier is therefore added to the Accuracy and Damage rolls for weapons such as clubs, swords and battleaxes.
\raggedbottom

\subsection{\attFin{}}

\attFin{} is the measure of a beings ability to perform acts with precision and care, and to maintain balance and poise. It also measures your ability to work with your hands - to craft intricate items, tie secure knots or steer an out of control vehicle. 

The {\it Acrobatics}, {\it Chicanery} and {\it Stealth} proficiencies measure a being's aptitude in certain types of \attFin{} checks. 

\proficiency{Acrobatics}{A \attFin{} (Acrobatics) check is used whenever a being's balance is called into question, such as maintaining balance on a rocking boat or slipping on an icy floor, as well as for more extravagant feats such as rolling, flipping, diving and somersaulting. }

\proficiency{Chicanery}{Chicanery is the trickster's and the thief's domain: a \attFin{} (Chicanery) check will be called for whenever you try to use duplicity, trickery, distraction or slight of hand to achieve your goal. }

\proficiency{Stealth}{A stealth check is used whenever you wish to remain hidden, and is the primary check used for the Stealth mechanic discussed on page \pageref{S:Stealth}. In addition, you may be asked for a \attFin{} (Stealth) check to hide an object away from prying eyes.}

\subsubsection{Ranged Weapons}

Most ranged weapons use the \attFin{} modifier to reflect the accuracy of the wielder. Some melee weapons which are classed as `elegant', such as rapiers, also use \attFin{} for their accuracy check.  In both cases, the \attFin{} modifier is added to the associated accuracy and damage checks.

\subsubsection{Spells}

Some spells rely on careful manipulation and high levels of precision and control: these spells belong to the {\it Kinesis} and {\it Alteration} disciplines. Spells belonging to this school use the \attFin{} modifier to perform Casting and Accuracy checks. 

\subsection{\attSpr{}}

The \attSpr{} of a character is a measure of their internal strength. 

\attSpr{} checks are used to maintain order in your own mind, or to project that inner strength outward to dominate others. 

The {\it Conviction} and {\it Willpower} proficiencies measure your ability at certain types of \attSpr{} checks. 

\proficiency{Conviction}{A \attSpr{} (Conviction) check is used whenever something attempts to sway a tenet of your character - whether someone is trying to tell you that a deeply held belief is false, to persuade you that your idea is bad, or to magically influence your thoughts. Conviction measures how strongly you hold to your fundamental principles. }

\proficiency{Willpower}{A \attSpr{} (Willpower) check is used whenever a being needs to have control over their own mind:
\begin{itemize}
\itdef
\item Enforce defences around their mind to repel intruders
\item Withstand the effects of mind-altering spell
\item Use magic which dominates the minds of others
\item Withstand terror and stand brave in the face of danger
\end{itemize}
}

\subsubsection{Spells}

Spells which rely on projecting your force of will, and an iron control of your mind use the \attSpr{} modifier for their casting and accuracy checks. Such spells include those in the {\it Psionics}, {\it Conjuration} and the {\it Hexes} discipline.  

\subsubsection{Passive Endurance}

Your {\it Passive Endurance} is a base level of endurance that every being has when they are not even aware they are actively resisting anything. 

If an effect is inflicted on you when you are not specifically expecting it, or searching for it, then the {\it passive} score is used. This can also be used by the GM to keep the fact that an enemy is influencing your mind, for example. The passive Endurance score is calculated from the `average' dice roll, plus the usual bonuses for a \attSpr{} (Willpower) check. 

Therefore it is calculated from a score of 10, plus the usual bonuses. If a being has advantage or disadvantage, you add or subtract 4 from the score. If you have super-advantage or disadvantage, you add or subtract a further 2. 


\subsection{\attChr{}}

\attChr{} is the social attribute - it measures a being's ability to interact with others with confidence, eloquence and panache. A high-\attChr{} being is percieved by others as charming and friendly. 

A \attChr{} check will be called for on almost all social interactions beyond basic introductions, services and general `how-do-you-do's. For a forthcoming individual, you may only have to ask the right questions to get the information or services you desire with no check needed, but for the more recalcitrant, you must succeed on a \attChr{} check to get what you want. 

The \attChr{} domain is divided into three proficiencies: {\it Deception}, {\it Performance}, and {\it Persuasion}. 

\proficiency{Deception}{A \attChr{} (Deception) check will, as the name suggests, be called for whenever you attempt to tell a convincing lie, or otherwise mislead an individual. Manipulate both your voice and your body language to give a false sense of honesty and truth to waylay the authorities, cheat an opponent out of some money, or bluff your way past a guard.}

\proficiency{Performance}{A \attChr{} (Performance) check is used whenever a being puts on an act to delight and impress an audience with their skills or stage presence. Perfomance is a form of {\it Deception}, with the difference usually being that the purpose is to inspire, delight or entertain, rather than mislead. } 

\proficiency{Persuasion}{A \attChr{} (Persuasion) check measures the ability of a being to sway others with convincing arguments, charm, and social know-how. Generally used in good faith to convince a neutral party to take a side, to persuade a guard to let you past, or to negotiate a better price for an item.}

\subsubsection{Spells}

Spells which belong to the {\it Bewitchment} discipline rely heavily on subtly altering and influencing a being's \attPer{} of reality. These spells use the \attChr{} modifier for their casting and accuracy checks. In addition, the {\it Occultism} discipline relies on persuading extradimensional powers to listen to you, and so also use the \attChr{} modifier. 

\subsection{\attInt{}}

\attInt{} is a being's innate mental capacity, their memory, their ability to reason and logically deduct as well as encompassing their prior education and learning.   

An \attInt{} check will be called for whenever a character attempts to assimilate new information, or recall information they have previously used. It may also be used to solve riddles, use logic to deduce where an item might be hidden, and so on. 

As \attInt{} is a wide and somewhat nebulous field, there are a number of proficiencies under this umbrella, particularly: {\it Arcane Knowledge}, {\it History}, {\it Logic}, {\it Nature}, {\it Research}, {\it Un-nature} 

\proficiency{Arcane Knowledge}{An \attInt{} (Arcane Knowledge) check - often shortened to simply `Arcane' - is a measure of a being's understanding of the nature and use of magic. Used to recall or infer knowledge about spells, magical items, mystic runes and other intrinsically magical objects.}

\proficiency{History}{An \attInt{} (History) check measures your ability to recall information about historical events, places and people}

\proficiency{Logic}{An \attInt{} (Logic) check is used to connect the dots between disparate and incomplete information, to gain an understanding of the larger picture. When faced with riddles, mysteries and utterly unknowable forces, a high logic can be used to discern the fundamentals of the problem at hand.}

\proficiency{Nature}{\attInt{} (Nature) checks are used to remember information about naturally occuring plants and beasts (both magical and mundane), the terrain or the weather. }

\proficiency{Research}{Attempting to learn new information about a known target subject falls under the domain of an \attInt{} (Research) check. When faced with a library full of books and information to assimilate, Research is your friend. {\it Research} differs from {\it Investigation} in that whilst {\it Investigation} helps you find a book, only {\it Research} can help you glean knowledge from it.  }   

\proficiency{Un-nature}{The partner to the {\it Nature} proficiency, an \attInt{} (Unnature) check is used to recall information and lore about unnatural, otherwordly, un-living or otherwise artifical items, creatures and constructs.}


\subsubsection{Spells}

Some spells rely on nothing more than a razor sharp mind and a deep understanding of the task at hand, and hence use the \attInt{} modifier for their casting and accuracy checks. Such spells include those from the {\it Temporal} and {\it Warding} disciplines.


\subsection{\attPer{}}

The \attPer{} attribute is your awareness and openness to the world around you - both in a material sense, and on an emotional level. 

A \attPer{} check will be used any time you wish to take in information around you, be it to spot hidden enemies, traps or paths, search through a vault of treasures, or discern the true intentions of a being. 

To that end, the \attPer{} attribute is split into three proficiencies: {\it Empathy}, {\it Investigation}, and {\it Observation}. 

\proficiency{Empathy}{A \attPer{} (Empathy) check is used whenever a being needs to put themselves in another's shoes - to understand their current state of mind, understand motive and intent, and possibly glean any hint that they are lying or omitting the truth. A high Empathy check might mean that you understand an individual better than they understand themselves.}

\proficiency{Investigation}{A \attPer{} (Investigation) check is used for in-depth scrutiny of an object, container or region. Unlike an {\it Observation} check, an Investigation is always used consciously. A high Investigation check would allow you to:
\begin{itemize}
\itdef
\item Spot a tiny inscription on the inside of a ring
\item Rifle through a chest full of nicknacks, to find a priceless object
\item Find a given book in a packed and disorganised library
\item Notice a hidden chamber hidden inside a wall, or spot the secret mechanism to trigger the door
\item Search the body of a slain enemy (or hapless victim) for useful items or clues
\end{itemize}
}

\proficiency{Observation}{A \attPer{} (Observation) check will be called for whenever you survey your surroundings, either with sight, sound or smell - to spot an ambush waiting for you to pass, or to notice a whispered conversation. Your Observation skill denotes both your spatial awareness, and your awareness of actions occurring within that space.  }

\subsubsection{Passive \attPer{}}\label{S:PassivePerception}

As with the \attSpr{} attribute, \attPer{} checks will often occur without conscious effort from the part of the individual - sneaking past bored guards is different from sneaking past guards who are actively searching from you! Your own passive \attPer{} may be used by the GM to decide whether to alert you or not to a hidden creature stalking you. In such cases you use the {\it Passive \attPer{}} score, which is calculated from the average dice roll the being would be expected to make.

Therefore it is calculated from a score of 10, plus the usual \attPer{} (Observation) bonuses. If a being has advantage or disadvantage, you add or subtract 4 from the score. If you have super-advantage or disadvantage, you add or subtract a further 2. 

\subsubsection{Spells}

Some spells require a deep attunement to the world around you, and the ability to notice and react to very fine details. Such spells use the \attPer{} modifier in for both the Spellcasting and Accuracy check. This spells generally fall into the {\it Telepathy} and {\it Healing} disciplines. 



\subsection{\attPow{}}

The \attPow{} attribute is a measure of the \attPow{} that a being has at their disposal - usually in the form of magical \attPow{}, though it may also be used as a proxy for political \attPow{}, or the simply the aura of \attPow{} that one projects. 

A \attPow{} check will rarely be called for outside of the context of a spellcasting context, or when resisting the effects of a spell, however you may be called on to perform a \attPow{} check when performing an extraordinary feat of magic that goes beyond the normal remit of a spell's abilities. 

A \attPow{}ful being may be able to use their formidable aura through the {\it Intimidation} proficiency.

\proficiency{Intimidation}{A \attPow{} (Intimidation) check will be called for whenever you attempt to leverage your superior abilities to threaten an individual into doing what you wish. }

\subsubsection{Spells}

Spells which simply require raw magical \attPow{} use the \attPow{} attribute in both spellcasting and Accuracy checks. Spells which fall into this category belong to the {\it Curses} and {\it Elemental} disciplines. 

In addition, raw magical \attPow{} may be leveraged into making spells more potent. Spells which require a Resist check to be performed (both damage causing and otherwise), the DV of the Resist is set by your {\it Subjugate} value, which is calculated from:
$$ \text{Subjugate} = 8 + \text{Expertise bonus + \attPow{} modifier} $$


\subsection{\attEvl{}}

The \attEvl{} attribute is a measure of the darkness and corruption which lies in the heart of an individual. 

In a perhaps na{\"i}ve view of the world, this game system presumes people are, by default, inherently good. Comitting \attEvl{} acts therefore requires conquering your inner, better nature. Slitting the throat of a incapcitated prisoner might be physically easy to do, but to actually go through with such a foul deed you must overcome this inner good - which requires passing an \attEvl{} check. 

Each time you commit such a deed, you will likely find your \attEvl{} rising in tandem with the blackening of your soul. 

\attEvl{} has no proficiencies associated with it. 

\subsubsection{Spells}

The most evil spells in existence can only be cast by those with a corrupted and wicked soul - the unforgivable curses, the animation of the dead as gruesome puppets and so on - and hence use the \attEvl{} attribute for casting and accuracy checks. This spells form the discipline known as {\it Necromancy}. 
