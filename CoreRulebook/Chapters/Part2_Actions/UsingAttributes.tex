\breaklesschapter{Using \key{Aspects} and \key{Abilities}} \label{S:Proficiencies}

Unless it relies on pure chance, almost every task a character attempts will utilise one of their 9 \key{Aspects} in some way, forming the basis of their dice pool. 

On top of this base level of competence, you may then add on your ability in a certain field, you \key{Abilities}, which are split into \key{innate}, \key{Practical} and \key{Knowledge}. The section below discusses the common circumstances and actions which would require each action, as well as common \key{aspect}-\key{ability} pairings. 

\def\itdef{\renewcommand\labelitemi{-}
\itemsep-0.5em}
\newcommand\proficiency[2]
{
	\textbf{\textit{#1}}: {\raggedright #2} 
}
\subsection{Aspect Descriptions}

\newcommand\indexLinker[1]{\index{Aspects!#1 Aspect|see{#1}}}
\indexLinker{Fitness}
\indexLinker{Precision}
\indexLinker{Vitality}
\indexLinker{Charm}
\indexLinker{Deception}
\indexLinker{Insight}
\indexLinker{Intelligence}
\indexLinker{Willpower}
\indexLinker{Perception}

\newcommand\aspectHead[1]{\subsubsection{#1} \index{#1}}
\newcommand\abilityHead[2]{\subsubsection{#1}\index{Abilities!#2!#1}}
\aspectHead{Fitness}

\key{Fitness} is the base aspect for any action which requires a character to exert themselves physically: to run and jump, or to lift heavy objects. 

It is also used in combat whenever using a weapon which relies on speed or strength in order to do damage, paired with either \imp{Brawl} or \imp{Skirmish} abilities. \imp{Fitness} also forms the basis of the \imp{Block} defense statistic, representing your ability to physically stop an attack doing damage.

Some more inventive uses of \imp{Fitness} could include pairing it with \imp{Intimidation}, if you are trying to physically intimidate somebody, or perhaps a \imp{Fitness (Survival)} or \imp{Fitness (World)} could also be used to flee from a foe, using your knowledge of the local area to escape from their sight. An acrobat or a trickster could make great use of a \imp{Fitness (Performance)} or \imp{Fitness (Acrobatics)}, either to please an audience or to create a distraction. 


\aspectHead{Precision}

\key{Precision} is the base aspect for any action in which accuracy and a steady hand is key: picking locks, painting a glorious fresco or assembling a delicate machine. 

\imp{Precision} is also important for its use in aiming ranged weapons such as firearms, when paired with \imp{Marksmanship}. It could also be paired with a number of \imp{Affinites} in order to carefully cast an intricate enchantment, and those who prefer not to be seen would use a \imp{Precision (Covert)} action to sneak around unseen, or to delicately pick a lock. 

\imp{Precision} also forms the basis of the \imp{Dodge} defense statistics, representing a character's ability to carefully maneouvre themselves out of harm's way.


\aspectHead{Vitality}

\key{Vitality} is the base aspect for any action taken whilst under physical stress, or in which the health of a character is called into question. As a measure of the character's general health levels, Vitality therefore represents their ability to stave off physical and medical impediments. 

When under extreme physical duress, at the limits of your stamina, you can often substitute \imp{Vitality} for a number of other \imp{Aspects}. This would probably have a lesser effect than usual, but when the chips are down, you don't have much of a choice.  


\aspectHead{Charm} 

Any social interaction which requires persasion, charisma and vivacity to try and convince another to part with something will use \key{Charm} as its base aspect. 

For honest folk, it therefore forms the basis of most social interactions, frequently favouring a pairing with \imp{Eloquence}. Some situations may also call for you to pair it with a wide variety of \imp{knowledge} abilities - if you are trying to sweet talk a physicist a \imp{Charm (Science)} check would probably be very effective, for example.

Casting magic spells which require an appeal to some other, higher power will also find \imp{Charm} to be a valuable source of magical strength. 

\index{Abilities!Innate!Eloquence}
\index{Abilities!Learned!Science}

\aspectHead{Deception}

When subterfuge and trickery are called for, look no further than \key{Deception}. Lies, half-truths, as well as the ability to convincingly embody another character fall into the domain of \imp{Deception}. 

Like \imp{Charm} deception is often paired with \imp{Eloquence} as well as various fields of \imp{Knowledge}, though \imp{Performance} also naturally falls into \imp{Deception}'s remit. 

Some magic, such as illusions, require the caster to mislead their foes and so often use \imp{Deception} as their basis. 

\index{Abilities!Innate!Eloquence}
\index{Abilities!Practical!Performance}

\aspectHead{Insight}

\key{Insight} is called for as a base aspect whenever a character is trying to ascertain the true meaning behind another character's words or actions. It represents the emotional intelligence of a character, and their ability to peer beyond the facade being presented to the world. 

A benevolent person would pair \imp{Insight} with \imp{Kindness} or \imp{Eloquence}, allowing them to empathise and care for other creatures, though those of a more malevolent disposition could equally pair it with \imp{Intimidation}, using a being's own fears and weaknesses against them. Those on the lookout for clues, or interrogating a suspect could use an \imp{Insight (Alertness)} check to spot a facial twitch or a statement which doesn't quite ring true.

Some healing magics rely heavily on the caster being able to understand what ails their patient, and \imp{Insight} can be used to great effect.  

\aspectHead{Intelligence}

Whenever sheer mental processing power is needed, \key{Intelligence} is appropriate. Particularly complex tasks - including certain spells - are well suited to an Intelligence check.

\imp{Intelligence} will almost always be used in conjunction with one of the \imp{Knowledge} abilities or simply the innate \imp{Logic}, allowing a character to recall and process information related to the task at hand. 

In a pinch, a character can also use their \imp{Intelligence} as a substitute for many other activities, but in doing so they are probably relying on theoretical knowledge, rather than practical experience, so the risk of failure can increase significantly. 

\index{Abilities!Innate!Logic}

\aspectHead{Willpower}

\key{Willpower} is the ability to manipulate your own mind, as well as the ability to project change onto others. 

Willpower works well with the \imp{Conviction} ability, in order to resist and defy those who would alter your understanding and perception of reality, as well as with \imp{Bravery}. It can also work as a substitute for many other abilities when suffering from mental distress - attempting to shut a door whilst a Banshee wails in your ear could use a \imp{Willpower (Strength)} check to represent the combined mental and physical struggle. 

Casting hexes and spells which have the intent to cause harm to others also requires you to have control over your reflexive tendency to hold back, and to project your will through a magical strike. Such spells often use \imp{Willpower} as a basis of their casting.  

Whenever sheer force of will is needed, \imp{Willpower} will work as a great aspect. 

\index{Abilities!Innate!Conviction}
\index{Abilities!Innate!Bravery}
\index{Abilities!Innate!Strength}


\aspectHead{Perception}

The ability to absorb information, through all 5 senses is governed by \key{Perception}. Perception is vital in the race to defeat foes, else a character risks being ambushed. Perception can also be used whenever great attention to detail is needed, allowing the detection of even the tiniest flaw.  

\imp{Perception}'s greatest ally is \imp{Alertness}, the combination of the two allowing a character to recognise threats from a distance. \imp{Perception (Investigation)} checks are also a good combination when combing through a large pile of books, or searching for a hidden groove in the floor. A careful flier might also prefer a \imp{Perception (Pilot)} check in order to fly their broomstick safely, avoiding potential dangers.


\section{Ability Descriptions}\index{Abilities}

\subsection{Innate Abilities}

An \key{Innate} ability is one which represents some aspect of a character's intrinsic social, mental or physical abilities, differing from the fundamental Aspects by their specificity to a single task. Though many people are born being particularly good in one or more of these areas (hence `innate'), they are still areas that can be worked on and improved. 

As these mostly represent extensions of you fundamental aspects to individual fields, having 0 dots in an Innate ability is not a barrier to attempting the activity, though you must rely wholly on your base Aspect to complete the task. 

\abilityHead{Alertness}{Innate}

\key{Alertness} is your ability to detect, process and notice external threats. It is their continual, total awareness of their external surroundings. A person with a high Alertness is very difficult to surprise or ambush, whilst conversly, those with a low \imp{Alertness} often find themselves falling into traps.

\imp{Alterness} is often paired with \imp{Perception} in order to spot foes, but it can also commonly be paired with \imp{Precision} - in order to spot the flaws you are straining to prevent. Whenever an action would be improved by a heightened awareness of your surroundings, you can consider \imp{Alterness} as a viable ability. 


\key{Passive Perception}

Often the GM will want to know if your character can spot a hidden threat such as a foe sneaking up behind you. They may often wish to do so without alerting the players that something is afoot, as this would necessarily change how the characters were being played. 

Therefore, if a character is attempting to remain hidden, the GM may use your \imp{Passive Perception} value: the number of dice in your \imp{Perception (Altertness)} pool. This is a base-level of awareness that a character has of their surroundings, and sets the DV for any sneak actions against you. The DV of the sneak action is equal to the passive perception, and a character is revealed if they fail or suffer a catastrophe. 

For example, Bruce, Jane and Simon are searching through the Forbidden Forest. Jane is the most alert, with 3 dots in Perception and 4 in Alertness, giving her a passive perception of 7. The GM then rolls for the Acromantula sneaking around in the canopy above them, getting \imp{2-3-5-6-8-10}, which scores two successes, an \imp{Okay} success. Jane would perhaps hear a small rustling, or get a feeling she is being watched but nothing else. 


\abilityHead{Bravery}{Innate}

The wizarding world is full of terrifying monsters and evil mages who would do you harm. \key{Bravery} is the ability to resist the urge to flee in terror, and instead stare down beings far more terrifying and powerful than yourself, without batting an eye. 

\imp{Bravery} is often paired with \imp{Willpower} to provide the raw force of will to look a monstrosity in the eyes, but also works well with \imp{Vitality}, when you have been beaten and bloodied, this combination can allow you to get back up and try all over again. If you are simply trying to bluff your way through a terrifying encounter, \imp{Deception (Bravery)} might also be of use. 


\abilityHead{Conviction}{Innate}

Magic can often make you doubt your own reality, conjuring impossible images in your mind, or compelling you to take actions. Politicians and leaders throughout history have also used more mundane methods to achieve the same goals, using cunning words and rhetorical tricks to convince you to take immoral and illegal actions. \key{Conviction} allows you to resist all of these, by grounding and cemeting your understanding of reality, and providing you with the moral strength to understand right from wrong. 

Often paired with \imp{Willpower}, allowing you to resist the effects of mind-altering effects, it is also often paired with \imp{Intelligence}, allowing you to deduce your way out of an illusory maze, or see through a logic-defying illusion.  

\abilityHead{Eloquence}{Innate}
\key{Eloquence} allows you to choose the perfect choice of words for a situation, whether it is to \imp{Charm} or \imp{Decieve} someone using their lingo, or to choose the right words to navigate a delicate emotional situation, using \imp{Insight} as your guide. You might also use \imp{Eloquence} when trying to find the correct words to describe to others a complex idea you have figured out using your phenomenal \imp{intelligence}.

\abilityHead{Intimidation}{Innate}

\key{Intimidation} is called for whenever you want to exude authority, give commands, compel swift obediance and even imbue your foes with terror. 

The skills combined with intimidation depend on how you are going about exerting your authority: \imp{Fitness} is often used if you are attempting to be physically imposing, whilst \imp{Willpower} can be used if you just want to seem officious through sheer force of will. \imp{Intelligence} could be used if you are attempting to intimidate with your vast knowledge of a certain area, or even \imp{Deception} if you are merely pretending to be powerful. 


\abilityHead{Kindness}{Innate}

With \key{Kindness} you exude a calming aura, and have the ability to show affection. You use \imp{kindness} to interact with those you truly care about, using \imp{Charm} or \imp{Insight} to get them on your side. If you're merely pretending to be kind to get something, perhaps \imp{Deception (Kindness)} would be more appropriate. 

Whilst using \imp{Eloquence} might help you make stunning speeches, or sway voters to your side - \imp{kindness} allows you to genuinely connect with a person, earning their trust and friendship.

\abilityHead{Kinship}{Innate}

A character with a high \key{Kindship} feels a close connection to animals and other living beings, able to gain their trust, train them, and with sufficient kindness and patience, get them to listen to you. 

Most commonly paired with \imp{Charm}, though someone who has extensively studied animals may find that their \imp{Intelligence} can come into play. 


\abilityHead{Logic}{Innate}

\key{Logic} is a character's ability to exhibit inference and deduction - working out how ideas are linked and follow on from one another. It allows a character to solve puzzles and riddles or deduce motives based on a disparate set of clues. It is very often paired with raw \imp{Intelligence}, though a \imp{Charm (Logic)} check could be used to convince somebody that a given course of action is the only logical choice. 

Within the game, a \imp{Intelligence (Logic)} check can be used as a gateway to the \imp{GM}, allowing them to give hints and tips based on things that the players might have not picked up on.

\abilityHead{Speed}{Innate}

\key{Speed} allows you to move rapidly, when coupled with \imp{Fitness} it allows you to outrun your foes and performing running leaps, and with the \imp{Dodge} statistics it allows you to leap out of the way of attacks. 

Your \imp{speed} rating also determines your movement speed in combat, the formula for calculating movement speed in a round is:
$$ \text{movement speed} = 3 + \text{speed rating} $$


\abilityHead{Strength}{Innate}

\key{Strength} is needed whenever you must exert immense physical force: moving or lifting heavy objects, or performing physical acts which require explosive bursts of power, such as a standing leap. 

A character with a high \imp{Strength} score should be given much more latitude on their amount of equipment they are carrying than a character with a low \imp{Strength}. They may also \imp{grapple} and pull around foes on the battlefield.


\subsection{Practical Abilities}


A \key{Practical} ability is one which you have learned through hands-on experience, laborious training and practice. Though they rely on an Aspect to direct the task, they are separate from your intrinsic abbilities and often requires some special tool or equipment to complete. 

\abilityHead{Acrobatics}{Practical} 

\key{Acrobatics} is your go-to skill whenever you need to perform feats of tumbling, flipping and other such marvels. Mostly reliant on either \imp{Fitness} or \imp{Precision} depending on the feat being attempted, it is often useful for impressing people - but a cat burglar might find the ability to silently tumble through a window a useful skill, and a cinematic flip over an enemy attacker in a \imp{Dodge} action would certainly be a sight to behold. 

If you are looking to leap over a chasm, a \imp{Fitness (Speed)} or \imp{Strength} might be more appropriate - but for anything with a bit more panache, \imp{Acrobatics} is your friend.

\abilityHead{Brawl}{Practical} 

\key{Brawl} covers all manner of fist-fighting, kicking and other acts of fighting which rely solely on your martial abilities, as well as \imp{improvised weapons}. The name might conjure up images of a brutal, ugly bar fight - but in the hands of a skilled martial artist, a \imp{brawl} action can be elegant and precise. 

Attacks with brawl (typically a \imp{Fitness (Brawl)} check) have a \imp{base damage} of 0, meaning that if you roll only a single success, you deal no \imp{harm} to your foe. This can be modified with a \imp{martial arts} feat, or by using a nearby chair, or other environmental feature, to give you a bit of a boost.

\abilityHead{Covert}{Practical} 

The \key{Covert} skill is used whenever you wish to do something that you would rather not be seen by anybody else - be it sneaking around unseen in an \imp{Acromantula}'s lair, subtly placing something in a drink at a party, or performing a discrete handoff with a buyer. 

\imp{Covert} is usually paired with \imp{precision}, representing the need for such actions to be careful and neat - but you could as easily use \imp{insight} to use your understanding of onlookers against them, or \imp{perception} to keep an eye out for anyone who might be watching. 

\imp{Covert} actions will almost always become \imp{contested} by onlookers, using their \imp{perception} checks. If they are actively searching for something out of place, then you must gain more successes than they do on a \imp{perception (alertness)} check. If they are only passively aware, then the DV of your \imp{covert} check is set by the highest \imp{passive perception} of your onlookers.

\abilityHead{Craft}{Practical} 

\key{Craft} is the ability associated with less magical acts of creation: \imp{artistry} and feats of \imp{engineering}. It represents your ability to create something with nothing but your bare hands. 

Some acts of \imp{crafting} are quick and easy - a simple whittling of a spoon, or painting a banner welcoming home a friend - others require more care and thought, following the \imp{Artificing} rules found on page \pageref{S:ArtificingBasic}. 

\abilityHead{Imbue}{Practical} 

The \key{Imbue} ability is used by artisans focussed on the creation of magical \imp{potions} and \imp{enchanted items}. It represents the ability to harness and transfer magical energies, to bend them to your own design, and then impregnate them into a vessel of your choosing.

Almost all such actions are a large investment of time and resources, and so will be \imp{extended actions} under the rules discussed on pages \pageref{S:ArtificingBasic}-\pageref{S:ArtificingEnd}.

\abilityHead{Marksmanship}{Practical} 

The ability to reliably hit a target from a distance, whether it be with a \imp{basketball} or a \imp{sniper rifle}, is covered by the \key{Marksmanship} skill. 

Primarily a combat-oriented skill, used for ranged weaponry, it can also be used when you really need to hit the target with something - like the aforementions basketball. Due to the need to hit a small target from a long way away, \imp{Precision} will usually be the base statistics here - though \imp{Perception} might be useful if you are lying in wait for your prey. If you are showing off, it is conceivable that you might attempt a \imp{Charm (Marksmanship)} check, though the consequence for failing might be a bit humiliating.

\abilityHead{Performance}{Practical} 

To be, or not to be, that is the question facing someone who relies on \key{Performance}. When one must embody another character - either for the purposes of entertainment, or deception and infiltration, \imp{Performance} will be the best ability for the job. You might also use this skill to put on a beautiful show of musical talent, or simply stand on stage and tell jokes. 

\imp{Deception} might be the tool to use if you want someone to believe that you are someone you are not, but  \imp{Charm} and \imp{Insight} are also great choices. 

\abilityHead{Pilot}{Practical} 

The ability to drive and navigate - whether it be a \imp{muggle} car, or a \imp{flying broomstick}, falls under the domain of the \key{Pilot} ability. 

A higher rating in this ability allows you to successfully pull of more elaborate and potentially lethal maneouvers, as well as to intuitively find the controls on a vehicle you have yet to encounter. 

\abilityHead{Skirmish}{Practical} 

Whilst \imp{Brawl} covers fighting with your fists and knees and head, and \imp{marksmanship} covers all manner of ranged weaponry, the \key{skirmish} ability covers the final gap in fighting skills: the ability to fight with close-range weapons. 

In a \imp{skirmish} situation, you might wish to swash some buckles with a rapier, or bring the hammer down with a might two-handed club. A high \imp{skirmish} ability allows you to wield these weapons to their fullest extent, doing more damage, and hitting your targets more reliably.

\abilityHead{Survival}{Practical} 

\key{Survival} covers the ability to stay alive whilst out of the comfort of your own home. 

Be it constructing a makeshift shelter, trapping rabbits for food, or foraging for berries - and even tracking larger prey through the marks they leave on their environment. \imp{Survival} is also instrumental in helping you navigate in an unfamiliar environment - using the stars, or the position of the sun to keep track of where you have been, and where you are going. 

A character with a high \imp{survival} will usually find it very hard to get lost, and if they do, they will be able to survive until they find their way again. 

\subsection{Knowledge Abilities}


A \key{Knowledge} ability is one which has been learned through intensive study, attending classes and days spent in the library. A knowledge ability can be used either to recall information, or to weave that information into another action. 
	

\abilityHead{Arcane}{Learned}

Knowledge which is classified as \key{Arcane} is that information which relates to the nature of magic itself - what it can do and create, as well as its limitations. It can help you identify and study magical spells and effects, items of an arcane nature, as well as to study the world beyond the one we see before us.

\abilityHead{First Aid}{Learned}

When you or your allies are wounded, or suffering from an unknown disease, the knowledge of \key{First Aid} is vitally important. 

This knowledge can direct you on the best course of action to help mitigate the problems at hand, patching them up and getting them back on their feet. It is also vitally important in helping a patient who is suffering from the \imp{Critical Condition} status. 

You might be able to use this ability to identify the cause of injury (or death) if you encounter someone, but even the highest rating of \imp{First Aid} is not a medical degree - the ability to diagnose more complex issues, and to treat the most serious of maladies falls under the domain of the \imp{Responder}'s special \imp{Pathology} ability.

\abilityHead{History}{Learned}

A high \key{History} rating grants you the ability to recall information, stories, facts and dates regarding historical events, famous people, ancient kingdoms, wars won and lost - as well as the many myths and legends which surrounds such important events in the historical record.

\abilityHead{Investigation}{Learned}

\key{Investigation} is meta-knowledge - that is, it grants you the ability to learn new information rapidly, and allows you the insight into where and how to find the knowledge that you seek. 

When faced with a formidable library, \imp{Investigation} can help you rapidly search through for the book you seek, and once you have found the tome, it can also help you identify and summarize the pertinent information for you. 

An unusual \imp{knowledge} skill, in that is is paired as often with \imp{Perception} as it is with \imp{Intelligence}.

\abilityHead{Linguistics}{Learned}

The ability to understand and comprehend different languages is covered by the \key{Linguistics} skill. A \imp{Linguistics} check might be called for to recognise a spoken or written language that you have encountered - or to analyse its structural components, and hence deduce who or what you might have just encountered.

Perhaps more importantly, however, for each rank gained in \imp{Linguistics}, you are able to learn one additional language, from the selections discussed on page \pageref{S:Languages}, or from one agreed with your \imp{GM}. The choice should make sense within the narrative, and you may not learn it {\it immediately} upon gaining the rank - but you are able to eventually learn this language. 

\abilityHead{Muggle}{Learned}

The non-magical world baffles most wizards, even simple things such as a rubber duck can cause great confusion - interactions between wizards and muggles are therefore often fraught with danger, as the wizard bumbles their way through even the simplest interaction. 

The knowledge of \key{Muggle}s therefore allows you to understand this vastly different world, and hence sidestep these issues - a high \imp{Muggle} knowledge allows you to know what is going on, keeping on top of the latest trends and news in their world, allowing a normal interaction.

\abilityHead{Nature}{Learned}

The study of \key{Nature} covers knowledge relating to all forms of naturally occuring beings, beasts and creatures and plants - understanding what they are and where they are from - as well as natural phenomena such as the weather and surrounding terrain.

You might not be able to know their exact weaknesses, but a high \imp{Nature} rating can allow you to recall information about the plant or animal you are searching for, where they might be found, and abilities and dangers you might be about to face. 

\abilityHead{Science}{Learned}

\key{Science} is, poetically put, the `art of knowing'. In modern times, it means the study of the rules and laws which govern the behaviour of the universe. From physics, chemistry and biology - to the `softer' sciences such as sociology and psychology. 

A high \imp{Science} rating means that you have a great understanding of the natural order of things, able to perform calculations and predictions, observations and experiments - as well as notice when something is not quite right, or when science appears to be being toyed with....

\abilityHead{Un-Nature}{Learned}

Whilst \imp{Nature} covered knowledge relating to living creatures which exist within nature, \key{Un-nature} covers those entities which sit outside this categorisation: the awful \imp{undead} creatures, as well as creatures which sit outside such paltry concerns as life and death, both benevolent and benign: \imp{sprites} and \imp{abominations}. 

You might also gain some insight into entities which were created through magical means, such as golems, though an \imp{Arcane} check might be more suitable.



\abilityHead{World}{Learned}

The \key{World} domain of knowledge encompasses your knowledge of the world around you - both on a geographic level, and in the sense of a broad base of \imp{general knowledge}. 

The location and behaviour of rivers, mountains, lakes and glaciers, as well as the location and nature of nations, and bits of information about the people who live in them. 

It is impossible to learn about the nature of people, without it crossing over into some other fields, so you may often use \key{World} knowledge in place of a more specific skill - though the results are likely to be more vague, and less concrete. 
