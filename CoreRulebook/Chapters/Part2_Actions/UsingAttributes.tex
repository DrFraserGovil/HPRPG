\section{Using \key{Aspects} and \key{Abilities}} \label{S:Proficiencies}

Unless it relies on pure chance, almost every task a character attempts will utilise one of their 9 \key{Aspects} in some way, forming the basis of their dice pool. 

On top of this base level of competence, you may then add on your ability in a certain field, you \key{Abilities}, which are split into \key{innate}, \key{Practical} and \key{Knowledge}. The section below discusses the common circumstances and actions which would require each action, as well as common \key{aspect}-\key{ability} pairings. 

\def\itdef{\renewcommand\labelitemi{-}
\itemsep-0.5em}
\newcommand\proficiency[2]
{
	\textbf{\textit{#1}}: {\raggedright #2} 
}
\subsection{Aspect Descriptions}

\newcommand\indexLinker[1]{\index{Aspects!#1 Aspect|see{#1}}}
\indexLinker{Fitness}
\indexLinker{Precision}
\indexLinker{Vitality}
\indexLinker{Charm}
\indexLinker{Deception}
\indexLinker{Insight}
\indexLinker{Intelligence}
\indexLinker{Willpower}
\indexLinker{Perception}

\newcommand\aspectHead[1]{\subsubsection{#1} \index{#1}}
\newcommand\abilityHead[2]{\subsubsection{#1}\index{Abilities!#2!#1}}
\aspectHead{Fitness}

\key{Fitness} is the base aspect for any action which requires a character to exert themselves physically: to run and jump, or to lift heavy objects. 

It is also used in combat whenever using a weapon which relies on speed or strength in order to do damage, paired with either \imp{Brawl} or \imp{Skirmish} abilities.

Some more inventive uses of \imp{Fitness} could include pairing it with \imp{Intimidation}, if you are trying to physically intimidate somebody, or perhaps a \imp{Fitness (Survival)} or \imp{Fitness (World)} could also be used to flee from a foe, using your knowledge of the local area to escape from their sight. An acrobat or a trickster could make great use of a \imp{Fitness (Performance)} or \imp{Fitness (Acrobatics)}, either to please an audience or to create a distraction. 

\index{Abilities!Practical!Brawl}
\index{Abilities!Practical!Skirmish}
\index{Abilities!Innate!Intimidation}
\index{Abilities!Practical!Survival}
\index{Abilities!Learned!World}
\index{Abilities!Practical!Acrobatics}
\index{Abilities!Practical!Performance}

\aspectHead{Precision}

\key{Precision} is the base aspect for any action in which accuracy and a steady hand is key: picking locks, painting a glorious fresco or assembling a delicate machine. 

\imp{Precision} is also important for its use in aiming ranged weapons such as firearms, when paired with \imp{Marksmanship}. It could also be paired with a number of \imp{Affinites} in order to carefully cast an intricate enchantment, and \imp{Precision} would be absolutely vital in an \imp{Imbuing} attempt such as enchanting an item with magical effects, or \imp{Crafting} a mundane item. 

Those who prefer not to be seen would use a \imp{Precision (Covert)} action to sneak around unseen, or to delicately pick a lock. 

\index{Abilities!Practical!Marksmanship}
\index{Abilities!Practical!Imbuing}
\index{Abilities!Practical!Crafting}
\index{Abilities!Practical!Covert}

\aspectHead{Vitality}

\key{Vitality} is the base aspect for any action taken whilst under physical stress, or in which the health of a character is called into question. As a measure of the character's general health levels, Vitality therefore represents their ability to stave off physical and medical impediments. 

When under extreme physical duress, at the limits of your stamina, you can often substitute \imp{Vitality} for a number of other \imp{Aspects}. This would probably have a lesser effect than usual, but when the chips are down, you don't have much of a choice.  


\aspectHead{Charm} 

Any social interaction which requires persasion, charisma and vivacity to try and convince another to part with something will use \key{Charm} as its base aspect. 

For honest folk, it therefore forms the basis of most social interactions, frequently favouring a pairing with \imp{Eloquence}. Some situations may also call for you to pair it with a wide variety of \imp{knowledge} abilities - if you are trying to sweet talk a physicist a \imp{Charm (Science)} check would probably be very effective, for example.

Casting magic spells which require an appeal to some other, higher power will also find \imp{Charm} to be a valuable source of magical strength. 

\index{Abilities!Innate!Eloquence}
\index{Abilities!Learned!Science}

\aspectHead{Deception}

When subterfuge and trickery are called for, look no further than \key{Deception}. Lies, half-truths, as well as the ability to convincingly embody another character fall into the domain of \imp{Deception}. 

Like \imp{Charm} deception is often paired with \imp{Eloquence} as well as various fields of \imp{Knowledge}, though \imp{Performance} also naturally falls into \imp{Deception}'s remit. 

Some magic, such as illusions, require the caster to mislead their foes and so often use \imp{Deception} as their basis. 

\index{Abilities!Innate!Eloquence}
\index{Abilities!Practical!Performance}

\aspectHead{Insight}

\key{Insight} is called for as a base aspect whenever a character is trying to ascertain the true meaning behind another character's words or actions. It represents the emotional intelligence of a character, and their ability to peer beyond the facade being presented to the world. 

A benevolent person would pair \imp{Insight} with \imp{Kindness} or \imp{Eloquence}, allowing them to empathise and care for other creatures, though those of a more malevolent disposition could equally pair it with \imp{Intimidation}, using a being's own fears and weaknesses against them. Those on the lookout for clues, or interrogating a suspect could use an \imp{Insight (Alertness)} check to spot a facial twitch or a statement which doesn't quite ring true.

Some healing magics rely heavily on the caster being able to understand what ails their patient, and \imp{Insight} can be used to great effect.  

\index{Abilities!Innate!Kindness}
\index{Abilities!Innate!Eloquence}
\index{Abilities!Innate!Intimidation}
\index{Abilities!Innate!Alertness}

\aspectHead{Intelligence}

Whenever sheer mental processing power is needed, \key{Intelligence} is appropriate. Particularly complex tasks - including certain spells - are well suited to an Intelligence check.

\imp{Intelligence} will almost always be used in conjunction with one of the \imp{Knowledge} abilities or simply the innate \imp{Logic}, allowing a character to recall and process information related to the task at hand. 

In a pinch, a character can also use their \imp{Intelligence} as a substitute for many other activities, but in doing so they are probably relying on theoretical knowledge, rather than practical experience, so the risk of failure can increase significantly. 

\index{Abilities!Innate!Logic}

\aspectHead{Willpower}

\key{Willpower} is the ability to manipulate your own mind, as well as the ability to project change onto others. 

Willpower works well with the \imp{Conviction} ability, in order to resist and defy those who would alter your understanding and perception of reality, as well as with \imp{Bravery}. It can also work as a substitute for many other abilities when suffering from mental distress - attempting to shut a door whilst a Banshee wails in your ear could use a \imp{Willpower (Strength)} check to represent the combined mental and physical struggle. 

Casting hexes and spells which have the intent to cause harm to others also requires you to have control over your reflexive tendency to hold back, and to project your will through a magical strike. Such spells often use \imp{Willpower} as a basis of their casting.  

Whenever sheer force of will is needed, \imp{Willpower} will work as a great aspect. 

\index{Abilities!Innate!Conviction}
\index{Abilities!Innate!Bravery}
\index{Abilities!Innate!Strength}


\aspectHead{Perception}

The ability to absorb information, through all 5 senses is governed by \key{Perception}. Perception is vital in the race to defeat foes, else a character risks being ambushed. Perception can also be used whenever great attention to detail is needed, allowing the detection of even the tiniest flaw.  

\imp{Perception}'s greatest ally is \imp{Alertness}, the combination of the two allowing a character to recognise threats from a distance. \imp{Perception (Investigation)} checks are also a good combination when combing through a large pile of books, or searching for a hidden groove in the floor. A careful flier might also prefer a \imp{Perception (Pilot)} check in order to fly their broomstick safely, avoiding potential dangers. . 

\index{Abilities!Innate!Alertness}
\index{Abilities!Learned!Investigation}
\index{Abilities!Practical!Pilot}

\section{Ability Descriptions}\index{Abilities}

\subsection{Innate Abilities}

An \key{Innate} ability is one which represents some aspect of a character's intrinsic social, mental or physical abilities, differing from the fundamental Aspects by their specificity to a single task. Though many people are born being particularly good in one or more of these areas (hence `innate'), they are still areas that can be worked on and improved. 

As these mostly represent extensions of you fundamental aspects to individual fields, having 0 dots in an Innate ability is not a barrier to attempting the activity, though you must rely wholly on your base Aspect to complete the task. 

\abilityHead{Alertness}{Innate}

\key{Alertness} is your ability to detect, process and notice external threats. It is their continual, total awareness of their external surroundings. A person with a high Alertness is very difficult to surprise or ambush, whilst conversly, those with a low \imp{Alertness} often find themselves falling into traps.

\imp{Alterness} is often paired with \imp{Perception} in order to spot foes, but it can also commonly be paired with \imp{Precision} - in order to spot the flaws you are straining to prevent. Whenever an action would be improved by a heightened awareness of your surroundings, you can consider \imp{Alterness} as a viable ability. 

\ratingTable
{Utterly oblivious and easily duped}
{An easy target}
{A little slow, but you can eventually spot what you are looking for}
{Competent - you can keep an eye out, but not notably so}
{You can react quickly as soon as you spot something wrong}
{Your senses are trained, able to hear an irregular or unexected footstep from a large distance, or spot an off-kilter button tucked into a dress}
{Your eyes pierce darkness, your ears can hear an argument through thick walls, and your nose has an excellent bouquet}
{You are aware of everything around you: sensing the beating of a gnat's wing and the barest fluctuations in temperature and windspeed.}


{\bf Passive Perception}

Often the GM will want to know if your character can spot a hidden threat such as a foe sneaking up behind you. They may often wish to do so without alerting the players that something is afoot, as this would necessarily change how the characters were being played. 

Therefore, if a character is attempting to remain hidden, the GM may use your {\it Passive Perception} value: the number of dice in your \imp{Perception (Altertness)} pool. This is a base-level of awareness that a character has of their surroundings, and sets the DV for any sneak actions against you. The DV of the sneak action is equal to the passive perception, and a character is revealed if they fail or suffer a catastrophe. 

For example, Bruce, Jane and Simon are searching through the Forbidden Forest. Jane is the most alert, with 3 dots in Perception and 4 in Alertness, giving her a passive perception of 7. The GM then rolls for the Acromantula sneaking around in the canopy above them, getting \imp{2-3-5-6-8-10}, which scores two successes, an \imp{Okay} success. Jane would perhaps hear a small rustling, or get a feeling she is being watched but nothing else. 


\abilityHead{Bravery}{Innate}

The wizarding world is full of terrifying monsters and evil mages who would do you harm. \key{Bravery} is the ability to resist the urge to flee in terror, and instead stare down beings far more terrifying and powerful than yourself, without batting an eye. 

\imp{Bravery} is often paired with \imp{Willpower} to provide the raw force of will to look a monstrosity in the eyes, but also works well with \imp{Vitality}, when you have been beaten and bloodied, this combination can allow you to get back up and try all over again. If you are simply trying to bluff your way through a terrifying encounter, \imp{Deception (Bravery)} might also be of use. 

\ratingTable
{Total wuss, you jump at your own shadow}
{Prone to bouts of terror}
{You can stand your ground, but you really, really, really want to run away}
{You can usually manage your fears}
{You excel at managing even your most primal fears, it takes something quite powerful to make you turn and run}
{Though very little makes you want to flee, your teeth can still be set on edge when faced with powerful and terrifying foes}
{You are close to completely mastering your fear - you can stare death in the face and barely flinch}
{You fear nothing. Monsters are the ones who fear {\it you} }

\abilityHead{Conviction}{Innate}

Magic can often make you doubt your own reality, conjuring impossible images in your mind, or compelling you to take actions. Politicians and leaders throughout history have also used more mundane methods to achieve the same goals, using cunning words and rhetorical tricks to convince you to take immoral and illegal actions. \key{Conviction} allows you to resist all of these, by grounding and cemeting your understanding of reality, and providing you with the moral strength to understand right from wrong. 

Often paired with \imp{Willpower}, allowing you to resist the effects of mind-altering effects, it is also often paired with \imp{Intelligence}, allowing you to deduce your way out of an illusory maze, or see through a logic-defying illusion.  

\ratingTable
{A total pushover, you mind is putty and can be molded and shaped. You don't really have any coherent ethical positions}
{You can't really express your beliefs, and your reality is easily shaped}
{You have some understanding of your own internal moral compass, and how you interact with the outside world}
{You have a decent understanding of your own reality, and can express your moral and philosophical beliefs with a degree of eloquence}
{You are rather strong willed, and hold to your fundamental tenets with great strength. You easily notice more clumsy efforts to sway you from those core tenets.}
{You are feverent in your beliefs and understanding of your place in the world. It would require great power to sway you in this fashion.}
{A masterful level of understanding of your own moral positions and contradictions, you can sense when someone is subtly trying to undermine them.}
{Your conviction is iron, and your reality is concrete. You know your world, and your place within it - nothing can sway you from your beliefs}


\abilityHead{Eloquence}{Innate}
\key{Eloquence} allows you to choose the perfect choice of words for a situation, whether it is to \imp{Charm} or \imp{Decieve} someone using their lingo, or to choose the right words to navigate a delicate emotional situation, using \imp{Insight} as your guide. You might also use \imp{Eloquence} when trying to find the correct words to describe to others a complex idea you have figured out using your phenomenal \imp{intelligence}.


\ratingTable
{Your foot practically lives in your mouth}
{You are a clumsy speaker, and rarely know what to say. Your explanations are weird and disjointed, making them hard to follow.}
{You have a basic understanding of social mores, but you execute them without finesse. }
{You are an average speaker, able to use an appropriate voice ot talk to those in authority, for example, but you are not a great orator. You can explain relatively simple ideas to other people, but your analogies often leave much to be desired. }
{Whilst certainly above average, your eloquence is not particularly noteworthy, though it does get results. You can get even the most stubborn student to understand moderately complex ideas.}
{You are a skilled speaker, able to navigate difficult social situations and express complex ideas in simple language.}
{You are a great orator, able to move people to tears with your delicate choice of words, or show just the right amount of deference in your tone.}
{You always have the perfect words, and speak to everyone in their own language, giving them exactly what they need in order to understand your intent. Your words can alter the course of entire nations for decades to come.}

\abilityHead{Intimidation}{Innate}

\key{Intimidation} is called for whenever you want to exude authority, give commands, compel swift obediance and even imbue your foes with terror. 

The skills combined with intimidation depend on how you are going about exerting your authority: \imp{Fitness} is often used if you are attempting to be physically imposing, whilst \imp{Willpower} can be used if you just want to seem officious through sheer force of will. \imp{Intelligence} could be used if you are attempting to intimidate with your vast knowledge of a certain area, or even \imp{Deception} if you are merely pretending to be powerful. 

\ratingTable
{Fluffy little bunnies command more respect than you}
{You can form the right words, but you're just not very scary}
{Those weaker than you become mildly worried in your presence}
{You can command the weak willed, and make people think twice about crossing you}
{You have a commanding air, and your orders carry more weight than average}
{People go out of their way to obey you and avoid annoying you}
{Even the strongest willed people struggle to resist your terrible wrath}
{You exude authority from every pore, and those around you snap into line as your very aura instils terror into their hearts}


\abilityHead{Kindness}{Innate}

With \key{Kindness} you exude a calming aura, and have the ability to show affection. You use \imp{kindness} to interact with those you truly care about, using \imp{Charm} or \imp{Insight} to get them on your side. If you're merely pretending to be kind to get something, perhaps \imp{Deception (Kindness)} would be more appropriate. 

\ratingTable
{You are rough and coarse. You have very few friends because of how rude you are.}
{You know what kindness looks like, but you struggle to put it into practice}
{Coarse around the edges, but you can exhibit kindness to your friends.}
{You are a pretty kind person, though you struggle to be kind to those you are not already familiar with}
{You have a kind face, and people are automatically inclined to like you}
{You are a truly great friend, a kind heart and a gentle soul}
{You extend your true kindness to every living soul, and likewise, otherwise neutral people will go out of their way to please you, simply for being so kind}
{The ultimate benevolent soul, you exude kindness and decency in an aura. People become happier just being in your presence.}


\abilityHead{Kinship}{Innate}

A character with a high \key{Kindship} feels a close connection to animals and other living beings, able to gain their trust, train them, and with sufficient kindness and patience, get them to listen to you. 

Most commonly paired with \imp{Charm}, though someone who has extensively studied animals may find that their \imp{Intelligence} can come into play. 

\ratingTable
{Animals hate you, and will flee or attack you on sight. The best you can do is convince them to retract their claws before they take a swipe.}
{Creatures remain wary of you, seeing you as different and alien. You find it difficult to convince them to hang around}
{You can handle domesticated animals with relative ease, though they remain reluctant to follow your commands.}
{Domesticated animals such as cats and dogs will enjoy your presence, and will even allow you to train and command them}
{Domesticated animals will flock to you, prefering your presence to others around you. They will allow you to command and handle them in ways they would not let anyone else to}
{Even wild animals begin to sense your kinship with them, with sufficient luck you can convince them to pass you by}
{Ferocious beasts have to take a moment before they attack you, though you probably can't stop them if they're hungry.}
{Newt Scamander would be proud: animals love you, even those normally considered utterly untrainable.}

\abilityHead{Logic}{Innate}

\key{Logic} is a character's ability to exhibit inference and deduction - working out how ideas are linked and follow on from one another. It allows a character to solve puzzles or deduce motives based on a disparate set of clues. It is very often paired with raw \imp{Intelligence}, though a \imp{Charm (Logic)} check could be used to convince somebody that a given course of action is the only logical choice. 

\ratingTable
{Dunce. You have no understanding of how things connect together.}
{A bit stupid - you get easily stumped by sudokus}
{Maths makes your head hurt, but you can get there in the end}
{Average: you can follow simple trains of thoughts to their conclusion, but complicated paths hurt your brain}
{Trained: you have practiced solving puzzles, and can do so in record time.}
{Sophisticated: you can follow increasingly complex lines of thought, and solve complex riddles and puzzles.}
{Logician - you can see the relevance of seemingly insignificant clues, even if you can't quite fit them into the big picture yet}
{Sherlockian: you can assemble the bigger picture from the most fragmented and disparate of clues. Your powers of deduction are unrivalled.}

\abilityHead{Speed}{Innate}

\key{Speed} allows you to move rapidly, when coupled with \imp{Fitness} it allows you to outrun your foes and performing running leaps, and with \imp{Precision} it allows you to dodge out of the way of attacks. 

Each level of \imp{Speed} also allows you to move 1m further in each round of combat.
 
 \ratingTable
 {Slowpoke: you can barely move above a walk.}
 {Lethargic: you can jog when needs be, but sprinting is out of the question}
 {Dawdler: you can jog for extended periods of time, and exhibit short bursts of speed}
 {Average: you can jog pretty well, and break into a sprint when needed, but you're winning no awards}
 {Brisk: you can outrun most average people, and your sprinting is pretty rapid}
 {Rapid: you can complete 10ks with ease, and over a short distance you're difficult to catch}
 {Breakneck: you can jog for long periods of time, and your sprinting is enough to turn heads}
 {Usain Bolt would be proud, you can sprint with the best of the best}


\abilityHead{Strength}{Innate}

\key{Strength} is needed whenever you must exert immense physical force: moving or lifting heavy objects, or performing physical acts which require explosive bursts of power, such as a standing leap. 


 \ratingTable
 {Pathetic: you can barely lift your own arm, let alone anything else.}
 {Limp: you struggle to carry anything particularly heavy}
 {Weak: can carry a reasonably heavy object for a short amount of time}
 {Average: you can support your own weight and carry heavy items for some time}
 {Tough: you work out, and can carry a large amount of weight, but not for particularly long}
 {Strong: You're significantly above average and can carry large amounts of weight for an extended period of time. }
 {Jacked: your muscles ripple beneath your shirt: you're strong, and you know how to utilise it.}
 {Powerlifter: you can shift entire trees and hunks of rock if you put your mind to it. People stop in awe as you exert your strength.}


\subsection{Practical Abilities}


A \key{Practical} ability is one which you have learned through hands-on experience, laborious training and practice. Though they rely on an Aspect to direct the task, they are separate from your intrinsic abbilities and often requires some special tool or equipment to complete. 

\abilityHead{Acrobatics}{Practical} 

\abilityHead{Brawl}{Practical} 

\abilityHead{Covert}{Practical} 

\abilityHead{Craft}{Practical} 

\abilityHead{Imbue}{Practical} 

\abilityHead{Marksmanship}{Practical} 

\abilityHead{Performance}{Practical} 

\abilityHead{Pilot}{Practical} 

\abilityHead{Skirmish}{Practical} 

\abilityHead{Survival}{Practical} 

\subsection{Knowledge Abilities}{Practical} 


A \key{Knowledge} ability is one which has been learned through intensive study, attending classes and days spent in the library. A knowledge ability can be used either to recall information, or to weave that information into another action. 
	

\abilityHead{Arcane}{Learned}

\abilityHead{General}{Learned}

\abilityHead{History}{Learned}

\abilityHead{Investigation}{Learned}

\abilityHead{Medicine}{Learned}

\abilityHead{Muggle}{Learned}

\abilityHead{Nature}{Learned}

\abilityHead{Science}{Learned}

\abilityHead{Technology}{Learned}

\abilityHead{World}{Learned}
