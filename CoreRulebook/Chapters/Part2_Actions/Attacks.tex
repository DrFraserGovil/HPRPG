
\section{Making Attacks}\label{S:Attacks} \index{Combat!Attacking}

When making an attack, either with spells, arrows, or with a blade, there are 4 key steps:
\begin{itemize}
	\item Select a target 
	\item Perform a check to resolve the action 
	\item See if the target defends themselves
	\item Calculate the damage and other effects inflicted
\end{itemize}

There are also some special rules regarding melee and ranged attacks.
\subsection{Target Acquisition}

You may only attack targets that are within the range of the attack you are making. For melee weapons, this is usually 1 metre, though some long weapons such as lances have additional reach. For ranged weapons, the maximum range is specified in the weapon description. 

In addition, to determining if the target is in range, you must determine if it is a valid target - you cannot shoot arrows around walls, after all. You must be able to see a target in order to attack it (or see below for rules on fighting invisible enemies), and you may need to consider the fact that a target has cover. 


\subsection{Melee Attacks}

A melee attack encompasses all close-range fighting, including fist-fighting, sword-swinging and whip-wrangling. 

Typically, a melee attack can only be made against a target if they are within 1 metre of the attacker, with a clear line-of-reach between the two. Some weapons, as well as larger creatures, are able to perform melee attacks at a larger range.  

\subsubsection{Grappling}

If you wish to grab your opponent- either to immobilise them, or to pick them up and throw them off a cliff - you may attempt to initiate a grapple in place of a regular attack. 

To perform a grapple you need two free hands and perform (usually) a \imp{Fitness (Strength)} check, which is contested by the target performing another check to escape (usually either a \imp{Speed} or \imp{Acrobatics} check, or simply going head to head with \imp{Strength}). If the grappling succeeds, the target acquires the \key{trapped} status. 

If the grappler is strong enough (determined by an additional \imp{Fitness (Strength)} check with the DV determined by the size of the target), they may drag the grappled opponent along with them. Unless the strength check was spectacularly successful, this will usually significantly reduce the movement speed. 

A grappled target may attempt to use their action to escape, in which case, repeat the contest. 

\subsubsection{Two-Weapon Fighting}

It is possible to have multiple one-handed weapons equipped at once -- for example, a dagger in each hand. 

If you have the requisite skill level to be considered \key{proficient} in both weapons, this allows you to perform a second attack when you use a \imp{major action} to use a melle attack. The second, attack, however, must halve the dice pool for the check.


\subsection{Ranged Attacks}

A ranged attack occurs over a longer distance by firing a projectile or magical effect over hundreds of metres in some cases. 

\subsubsection{Ranged Weapons}

The description of every ranged weapon gives a maximum range at which the weapon may be fired. Some weapons have multiple ranges depending on the way in which they are used. 

Generally speaking, you cannot fire a projectile further than this range, as it represents the maximum distance that the projectile can reach. Some weapons (particularly the {\it firearms} class, though your GM will rule which weapons this is true for), however, the stated range is merely the range at which you can fire accurately. These weapons {\it can} be fired up to twice their stated range, but the DV increases by 3 when you go over this range.

In addition, you will need to ensure that you have enough ammunition to properly use your ranged weapon.


\subsubsection{Close-Combat Firing}

Ranged weapons and spells are significantly less effective when used on targets which are in close-quarters: aiming requires a clarity of thought that a monster trying to bite your face off denies. 

When attempting to use a ranged attack on a non-incapacitated target within melee range, take a 2d penalty on the accuracy check.

\section{Accuracy}

The attacker quantifies their ability to successfully hit their target through an \key{accuracy check}. 

The \imp{Dice Pool} and \imp{DV} of this check is determined by the type of attack and the weapon 



%~ \subsubsection{Additional Difficulty} \label{S:HardToHit}

%~ Targeting objects which are particularly small, or (for ranged attacks) far away is more difficult.  The additional penalty for hitting such away targets is, with everything measured in metres:

%~ $$ P = \frac{\text{distance}}{10 \times \text{size}} ~~~~ \text{(rounded down)}$$

%~ Therefore, hitting a 1m target at a distance of up to 10m has a DV of 5, whilst the same target 30m away has a DV of 8, and hitting a 1cm target at a distance of of 1m has a DV of 15. 

%~ \subsubsection{Blindfight}\label{S:Unseen}

%~ If you cannot see your enemy, then you cannot select them as a target. You may, however, choose to simply start swinging your sword, or firing spells off in a random direction. You must tell the GM which direction you are attacking in, and then perform an accuracy check with check disadvantage.

%~ If the target is not in the region  you are attacking, you automatically miss (though the GM will still ask for the accuracy roll, to avoid giving away where they actually are!). 

%~ After you successfully hit an unseen attacker, you avoid the disadvantage penalty until your next attack misses or the target moves. You must then retake the penalty until you next land a successful hit, or you detect them through other means. 
