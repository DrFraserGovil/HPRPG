
\chapter{Making Attacks}\label{S:Attacks} \index{Combat!Attacking}

Attacking your foe, dealing damage to them in an attempting to subdue (or even destroy) them is the key to winning your more violent encounters. 

The basic mechanism for performing an attack, be it with a magical \imp{spell}, a swinging blade,a mighty headbutt, or a sniper shot from way in the distance, is to follow this 4-step checklist:

\begin{itemize}
	\item Select a target and the attack you wish to perform against them, declaring this to the \imp{GM}
	\item Perform a check to resolve the action 
	\item See if the target defends themselves
	\item Calculate the damage and other effects inflicted
\end{itemize}

\subsection{Target Acquisition}

You may only attack targets that are within the range of the attack you are making. For melee weapons, this is usually 1 metre, though some long weapons such as lances have additional reach. For ranged weapons, the maximum range is specified in the weapon description. 

In addition, to determining if the target is in range, you must determine if it is a valid target - you cannot shoot arrows around walls, after all. You must be able to see a target in order to attack it (or see below for rules on fighting invisible enemies), and you may need to consider the fact that a target has cover. 

\section{Performing Attacks}

If you are performing an attack using a magical spell, you simply cast the spell using the rules discussed on page \pageref{S:CastingChecks}: using your \imp{spellcasting pool} to manifest the desired effect. 

If, however, you are performing an attack using a weapon, or even simply with your bare fists, then you must perform an \key{Attack Roll}. To do this, you must, as usual, determine the \imp{skills} that are being used to perform the attack. As a general guide:
\begin{itemize}
	\keyItem{Melee (Brawl)}{Punching someone in the face, or using an improvised weapon in a bar fight would use \key{Fitness (Brawl)}}
	\keyItem{Melee (Refined)}{Using a melee weapon in a more controlled fashion would use \key{Fitness (Skirmish)}, though you may argue that a refined and elegant attacker could use \key{Precision (Skirmish)}} 
	\keyItem{Ranged}{Strinking a target from a distance will almost always use \key{Precision (Marksmanship)}}
\end{itemize} 

Each weapon provides a \key{Base Damage} statistic and a DV associated with them (note that if you do not meet the requirements to be considered proficient with the weapon, the DV is increased by 2). See the weapon list on page

\section{Melee Attacks}

A melee attack encompasses all close-range fighting, including fist-fighting, sword-swinging and whip-wrangling. 

Typically, a melee attack can only be made against a target if they are within 1 metre of the attacker, with a clear line-of-reach between the two. Some weapons, as well as larger creatures, are able to perform melee attacks at an extended range. 


Whilst at close range, there are a number of additional manoeuvres that you might wish to use, as well as a simple  

\subsection{Melee Manouvers}
\subsubsection{Grappling}

If you wish to grab your opponent- either to immobilise them, or to pick them up and throw them off a cliff - you may attempt to initiate a grapple in place of a regular attack. 

To perform a grapple you need two free hands and perform (usually) a \imp{Fitness (Strength)} check, which is contested by the target performing another check to escape (usually either a \imp{Speed} or \imp{Acrobatics} check, or simply going head to head with \imp{Strength}). If the grappling succeeds, the target acquires the \key{trapped} status. 

If the grappler is strong enough (determined by an additional \imp{Fitness (Strength)} check with the DV determined by the size of the target), they may drag the grappled opponent along with them. Unless the strength check was spectacularly successful, this will usually significantly reduce the movement speed. 

A grappled target may attempt to use their action to escape, in which case, repeat the contest. 

\subsubsection{Two-Weapon Fighting}

It is possible to have multiple one-handed weapons equipped at once -- for example, a dagger in each hand. 

If you have the requisite skill level to be considered \key{proficient} in both weapons, this allows you to perform a second attack when you use a \imp{major action} to use a melee attack. The second, attack, however, must halve the dice pool for the check.


\section{Ranged Attacks}

A ranged attack occurs over a longer distance by firing a projectile or magical effect over hundreds of metres in some cases. 

\subsubsection{Ranged Weapons}

The description of every ranged weapon gives a maximum range at which the weapon may be fired. Some weapons have multiple ranges depending on the way in which they are used. 

Generally speaking, you cannot fire a projectile further than this range, as it represents the maximum distance that the projectile can reach. Some weapons (particularly the {\it firearms} class, though your GM will rule which weapons this is true for), however, the stated range is merely the range at which you can fire accurately. These weapons {\it can} be fired up to twice their stated range, but the DV increases by 3 when you go over this range.

In addition, you will need to ensure that you have enough ammunition to properly use your ranged weapon.


\subsubsection{Close-Combat Firing}

Ranged weapons and spells are significantly less effective when used on targets which are in close-quarters: aiming requires a clarity of thought that a monster trying to bite your face off denies. 

When attempting to use a ranged attack on a non-incapacitated target within melee range, take a 2d penalty on the accuracy check.

\section{Accuracy}

The attacker quantifies their ability to successfully hit their target through an \key{accuracy check}. 

The \imp{Dice Pool} and \imp{DV} of this check is determined by the type of attack and the weapon 



%~ \subsubsection{Additional Difficulty} \label{S:HardToHit}

%~ Targeting objects which are particularly small, or (for ranged attacks) far away is more difficult.  The additional penalty for hitting such away targets is, with everything measured in metres:

%~ $$ P = \frac{\text{distance}}{10 \times \text{size}} ~~~~ \text{(rounded down)}$$

%~ Therefore, hitting a 1m target at a distance of up to 10m has a DV of 5, whilst the same target 30m away has a DV of 8, and hitting a 1cm target at a distance of of 1m has a DV of 15. 

%~ \subsubsection{Blindfight}\label{S:Unseen}

%~ If you cannot see your enemy, then you cannot select them as a target. You may, however, choose to simply start swinging your sword, or firing spells off in a random direction. You must tell the GM which direction you are attacking in, and then perform an accuracy check with check disadvantage.

%~ If the target is not in the region  you are attacking, you automatically miss (though the GM will still ask for the accuracy roll, to avoid giving away where they actually are!). 

%~ After you successfully hit an unseen attacker, you avoid the disadvantage penalty until your next attack misses or the target moves. You must then retake the penalty until you next land a successful hit, or you detect them through other means. 
