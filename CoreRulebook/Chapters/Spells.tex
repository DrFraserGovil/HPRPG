\documentclass[../CoreRulebook.tex]{subfile}


\chapter{Types of Magic}

\newcommand\schoolDescribe[6]
{
	\subsection{#1}
	
	#2
	
	\subsubsection{{\it #3} Magic}
	
	#4
	
	\subsubsection{{\it #5} Magic}
	
	#6
}

Magic is an all-encompassing supernatural force within the universe, with the ability to alter reality at a most basic level. Incredibly powerful and difficult to control, magic is - at its heart - formless, chaotic and without boundaries. 

However, over the centuries, some humans have been born with the ability to touch this immense reservoir of power: witches and wizards. These magic-users have attempted to tame and define magic, and to shepherd into easily understood forms. 

The greatest discovery in wizarding history - comparable to the discovery of fire, or the wheel in the muggle world - was the discovery of the magical spell, followed by the discovery of the magical wand. 

Prior to this discovery, witches and wizards had attempted to harness the infinite force of magic through sheer mental effort. Of course, this meant that a single stray thought at an inopportune moment could lead to blowing up a city, rather than lighting a candle. 

Magical spells however, combine a ritualistic element (usually in the form of an incantation and a physical movement) to condition and focus the mind into the correct shape. The discovery of wands to focus and channel magical energies helped popularise this new way of casting magic - and it is now the utterly dominant way for magical folk to use their skills.  

\section{Spell \& Caster Levels}

Not all spells are created equal - some spells can only be cast by those who are exceptionally skilled in the arcane arts. There are 6 `levels' of spells, which denote how powerful they are:
\begin{center}
\begin{rndtable}{c c p {5.2 cm}}
	\bf Spell Level	&	\bf Name	&	\bf Description
	\\
	1	&	Beginner	&	Basic spells that everyone can cast
	\\
	2	&	Novice		&	More powerful, yet still basic magic
	\\
	3	&	Adept		&	The most powerful magic that a `normal' person uses in their day-to-day life.
	\\
	4	&	Expert		&	Spells which go beyond those learned in a normal education
	\\
	5	&	Master		&	Powerful spells cast by those at the top of their field
	\\
	6	&	Ascendant	&	Nearing god-like abilities
\end{rndtable}
\end{center}

Every witch or wizard has an associated `magic level' as well, which denotes the power of spells that they can cast. A {\it Novice} wizard is able to cast Beginner and Novice-level spells, but no higher. The types of spell a character has access to is determined by the total character level:

\begin{center}
\begin{rndtable}{c c}
	\bf Character Level	&	\bf Spell Level
	\\
	1-3		&	Beginner
	\\
	4-7		&	Novice
	\\
	8-10		&	Adept
	\\
	11-13	&	Expert
	\\
	14-17	&	Master
	\\
	18 + 	&	Ascendant	
\end{rndtable}
\end{center}

\section{Magical Schools}\label{S:Schools}

The study of magic is a far-reaching field, which encompasses many different areas and skills -- some of which require vastly different skillsets to use. For this reason, a magical taxonomy was introduced by the Wizangemot in 1755, which divides the study of magic up into 7 `Schools', each of which contains a number of `Disciplines'. 



	\schoolDescribe{Charms}{The Charms school of magic fundamentally relies on magically manipulating the position and speed of matter, whether on a large scale, to cause objects to levitate and fly - or on a microscopic level, to excite and energise the inside of an object, causing it to burst into flame. \\Those who are proficient in Charms are known as {\it Sorcerers}.}{Elemental}{Elemental magic studies the manipulation and invocation of very primal forces -- heat, light, energy, matter, and the classical elements.}{Kinetic}{Kinetics is a discipline which relies on moving and manipulating physical objects, and often forms the basis of `everyday' magic.}

	\schoolDescribe{Divination}{The Divination school encompasses magic which taps into forces which exist beyond the physical world to discern knowledge that would have previously remained hidden - entering the domain of the senses, memory, and the spiritual realms. }{Cerebral}{Cerebral magic is the study of peering into the human mind, extending the senses beyond their normal range and detecting the undetectable.\\Those who are proficient in the field of Divination are known as {\it Clairvoyants}.}{Temporal}{One of the most mysterious disciplines, temporal magic allows one to see beyond concerns such as time and space, and observe (and perhaps manipulate) the universe at an extraplanar level}
	
	\schoolDescribe{Illusion}{The Illusion school of magic is, as the name might suggest, focussed on magic which produces false images and tweaks the mind into seeing things which are not really there. \\Witches and Wizards who excel in Illusion magics are known as {\it Magicians}. }{Bewitching}{This discipline focusses on the gentle persuasion of the mind and the manipulation and conjuring of images to convince the target of something which is not true.}{Psionic}{A darker side of illusion magics, psionics is the art of imposing your will over that of your target -- forcing your way into their mind and altering it as you see fit.}
	
	\schoolDescribe{Malediction}{The Malediction school of magic contains those spells which have the primary intent to hurt, inflict harm on and otherwise incapacitate others. \\ Those who are experts in the field of Malediction are known as {\it Battlemages}. }{Hexes}{Hexes are a field which focusses on magic that directly harms the targeted person or object.}{Curses}{Unlike hexes, curses do not directly harm the target but instead incapacitates them, inhibits their capabilities, or otherwise reduces the threat they pose.}
   
   \schoolDescribe{Recuperation}{The Recuperation school of magic is often considered unglamourous, but those who can look past that can see that the ability to heal and protect yourself and others from harm is utterly invaluable. \\ Those who are proficient in the use of Recuperation magic are known as {\it Aegistes}. }{Healing}{Healing is, unsurprisingly, the study of magic used to heal the sick and wounded, break curses and project powerful positive energies.}{Warding}{Warding magic is almost entirely defensive in nature, allowing the caster to protect themselves and others from harm by casting powerful and long lasting shields and force-fields.}
	
	\schoolDescribe{Transfiguration}{The Transfiguration school of magic is focused on the transformation of the natural order - either by altering and reshaping the form of existing objects, or by summoning entirely new matter from thin air. \\Those who excel in Transfiguration are known as {\it Thaumaturges}.}{Alteration}{The alteration discipline studies the ability to change things from one form into another.}{Conjuration}{Conjuration magic is concerned with the ability to summon new objects and beings out of thin air, or to banish objects from existence.}
	
	\schoolDescribe{Dark Arts}{The Dark Arts school of magic encompasses magic which is frowned on in polite society, either because it involves truly evil spells - those which cannot be used without leaving scars on the soul, or those which tap into the dangerous and unfathomable energies of the dark and unspeakable things which lie just out of sight - under your bed and in the corner of your eye...\\Those who wield this forbidden magic are known as {\it Warlocks}.}{Necromancy}{A taboo discipline which contains deeply unpleasant spells which can only be cast by beings corrupted by evil - torture, death and worse lie in the domain of necromancy.}{Occultism}{Occultism is a rarely studied discipline that accesses and manipulates otherworldly energies originating from the Eldritch domain -- powerful, yet highly unpredictable.}

\normalsize

\section{Spell Types} 

In addition to falling into one of the seven Schools (a taxonomy based on the spell effect), every spell can also be categorised as a {\it type}, which is based on how the spell is cast. 

\subsection{Instant}

An instant spell is `cast and forget': as soon as you complete the requisite casting checks, the spell is `launched' (usually in the form of a magical bolt of light) towards the target. These bolts travel at speeds of 100m per cycle, which means in most cases, the effect is applied between the successful casting and the beginning of the next turn cycle.

Instant spells are denoted by the symbol \instSymb.

\subsection{Focus}

A focus spell is cast like an Instant spell, but may then be continued indefinitely, repeating the initial effects once per turn as long as you keep the spell active. No further checks are needed to continue the spell, but you must keep your mind focussed on the task at hand. Unless stated otherwise, Focus spells {\bf do not} cost additional FP after the first round in which they are cost. 

Because you must remain focussed, no further spells can be cast for the duration of this spell. In combat, maintaining a Focus spell takes your entire major action.    

Whilst maintaining a {\it Focus} spell you are considered {\it Distracted} and take the associated status effect. This renders you vulnerable to Critical Strikes, and upon taking damage you must pass a Willpower Resist check to maintain your concentration. 

You may end the spell effect at any time without it counting as an action. 

Focus spells are denoted by the symbol \concSymb.

\subsection{Ward}
A ward is a spell that affects a large area, and typically lasts for a long time after being cast. Most wards are centred on a single `focal point', which is selected at the time of casting. Some wards limit the kind of target that a valid focal point can be attached to. 

Unless stated otherwise, a ward spell is assumed to move as the object the focal point is attached to moves - a warded individual, therefore, does not need to stay still to remain protected. 

Ward spells are denoted by the symbol \wardSymb.

\subsection{Ritual}

A Ritual spell is a spell that requires a large amount of preparation -- be it meditation, drawing a summoning circle upon the ground, or performing a special dance. Each Ritual spell has a designated time that the ritual takes to complete, to cast a ritual spell you must spend this length of time preparing for the spell, and after the requisite time has passed, {\it then} you perform the check, and the spell effect is activated. If you fail the check, or choose to stop the ritual, i.e. to take another action, you must restart the ritual spell from the beginning. 

As with a focus spell, concentration is key to completing a ritual, and whilst performing a ritual, you are considered {\it Distracted}. 

Ritual spells are denoted by the symbol \ritSymb.


\subsection{Other Spell Types}

\subsubsection{Runic}



\subsubsection{Beast}

Beast spells are denoted by the symbol \beastSymb.


\section{Spell Shapes}

Some spells produce bolts of energy which fly towards a target, whilst others project their energy into a given region, which are often classified via geometrical shapes: a {\it line}, a {\it cube}, a {\it sphere}, a {\it circle} a {\it cone} or a {\it cylinder}. These shapes may either originate around the caster, or from a point designated by the spell. 

\subsection{Circle}

A circular spell extends outwards from the point of origin in a 2D circular shockwave that lies parallel to the ground. The height of the shockwave above the ground is set by the point of origin, which is not included in the shockwave region (unless the caster chooses it to be). Because of its 2D nature, a circular spell can be avoided by ducking beneath it, or jumping over it -- it is only if the shockwave impacts you that the spell effect is applied. 

\subsection{Cone}

The point of origin of a cone is typically the caster's wand, and a cone extends outwards from the wand, in the direction that the wand is pointing. A cone extends forwards to the specified distance, and has a circular cross section, the radius of which is equal to the distance away from the point of origin (so it is a 45$^\circ$ cone).

 The point of origin of the cone is not considered part of the spell area. 

\subsection{Cube}

The point of origin for a cubic spell may be selected to be either the centre of the cube, or the centre of one of its 6 sides. The cube's side-length is specified by the spell effect. The cube point of origin is only affected by the spell if you choose the centre-origin.

\subsection{Cylinder}

A cylinder point of origin is specified to be a point on the ground, around which a circular cross section is drawn, and then a cylinder of energy rises up vertically to a specified height. Generally, a cylinder spell adjusts its size to an individual, and if not otherwise specified, the cylinder is 5cm wider than the target individual is wide, and 5cm taller than the target. The point of origin is affected by the spell. 

\subsection{Line}

A line extends in a straight path from the origin (a caster's wand) towards the target for a specified distance. Unless otherwise specified, the beam is considered to have the cross section equivalent to a pencil. The point of origin is not affected by the spell. 

\subsection{Sphere}

A sphere's point of origin lies at the centre, and the spell effect expands equally out in all directions from that point. Generally, the spell effect cannot penetrate into the ground or through solid objects (unless, for example, it is an explosion). The point of origin is affected by the spell. 



\chapter{Casting Spells}

Spellcasting is the process by which a witch or wizard harness the infinite, chaotic and formless power of {\it magic}, shape it through their intellect or force of will, and project it into the world around them. 

For most wizards, this is achieved through the use of an incantation, a movement of the wand, and deep concentration, though some magic spells require a ritual be conducted before the magic can be executed. 

Some powerful wizards understand that these are simply crutches, guiding tools for the weaker mind - and can cast magic both silently, and without their wand to focus the magical energies. This, however, is an advanced feat and is not to be taken lightly. 

\section{Learning Spells}

In order to cast a spell, you must be guided in how this is achieved - to learn the incantation, the wandwork and the correct patterns of thought which will channel the magical energy correctly. 

\subsection{Spellbooks and Book-Casting}

The most common source of such information is in spellbooks, such as those listed in the Items chapter. If you have a spellbook in your possession, you may be able to flip through and find a spell you would like to cast. By carefully studying the text, you may attempt to cast the spell, whilst using the book for reference. 

This is known as {\it Book-Casting}. 

Book casting is a fairly slow process - even the slightest misreading of the text could result in drastic consequences! When used in combat, book casting always takes up the entirety of your turn. 

After choosing the spell you would wish to cast, you must perform a {\it Casting Check} (see below). If the check succeeds, you must then perform an accuracy check (if relevant), and then the magic effect takes hold. 

Congratulations - you just cast your first spell!

\subsection{Memorising Spells}

After you have book-cast a spell a couple of times - you will begin to get the hang of it. Eventually, you will have comitted the spell to memory. This occurs after you have book-cast a spell a number of times equal to:
$$ N = 5 - \text{\attInt{} Modifier} ~~~\text{(min 1)} $$
These book-casts have to be in an appropriate use of the spell - you can't sit and hex a tree 5 times in a row, and expect to learn the spell. You must successfully use the spell for its intended purpose for it to be a valid learning experience. 

Alternatively, you may spend your downtime studying the {\it theory} of the spell, over the practice. Studying a spellbook, or working with a proficient teacher for 1 hours is equivalent to casting the spell once in a real-life scenario. However, knowing something is theory is not always quite enough: you can never {\it completely} learn a spell this way. After completing your research, you must book-cast the spell at least once more, before it is truly memorised. 

\subsection{Memory-Casting}

After a spell is memorised, you no longer need the spellbook to hand in order to cast the spell - instead you can {\it Memory-Cast} it. 

When you are comforable enough with the spell to memory-cast it, the casting check is assumed to succeed, unless you are trying to do something particularly out of the ordinary - such as silent casting. 

A memory cast spell therefore skips the casting check stage, and jumps straight to the accuracy check (if applicable), and then applies the specified spell effect. 





\section{Casting Checks}

When casting an unfamiliar spell (or casting a familiar spell at a higher level) there is a non-trivial chance for a spellcaster to flub some important aspect of the spellcasting - which causes the spell to fail to materialise. 

This is quantified through the {\it Casting Check}. A casting check is a normal ability check, performed with a d20 dice. The relevant ability modifier is determined by the kind of spell you are attempting to cast. Spells from different disciplines require different mental abilities in order to manifest, as shown in the table below: 

\def\xS{2}
\def\wS{2}
\begin{center}
	\begin{rndtable}{c m{\xS cm} p{\wS cm}}
	\bf School	&	\bf Discipline	&	\bf Attribute
	\\
	\school{Charms}{Elemental}{\ElCheck}{Kinesis}{\KinCheck}
	\\
	\school{Divination}{Telepathy}{\TelCheck}{Temporal}{\TemCheck}
	\\
	\school{Illusion}{Bewitchment}{\BewCheck}{Psionics}{\PsiCheck}
	\\
	\school{Malediction}{Hexes}{\HexCheck}{Curses}{\CurCheck}
   \\ 
   \school{Recuperation}{Healing}{\HeaCheck}{Warding}{\WarCheck}
	\\
	\school{Transfiguration}{Alteration}{\AltCheck}{Conjuration}{\ConCheck}
	\\
	\school{Dark Arts}{Necromancy}{\NecCheck}{Occultism}{\OccCheck}
	\end{rndtable}
\end{center}

In addition, as well as an affinity based on their attribute scores, some beings possess proficiencies in various disciplines. If a being is considered proficient in the spell-school they are attempting to cast, then they add their Expertise Bonus to the casting check. 


The difficulty of a spell is determined by the caster's own level, and the difficulty of the spell they are trying to cast. Use the table below to determine the casting DV:
\def\cc{\cellcolor{\tablecolorhead}\bf}

\begin{center}
\begin{rndtable}{c c c c c c c c}
~	& ~ &	\multicolumn{6}{c}{\bf Spell Level}
\\
\cc	&	\cc	&	\cc 1 &\cc 2&\cc 3&\cc 4&\cc 5&\cc 6	
\\
\cc~	&	\cc1	&	15~&~&~&~&~&
\\
\cc&\cc	2	&	10	&	15~&~&~&~&
\\
\cc&	\cc3	&	5	&	10	&	20~&~&~&~
\\
\cc&	\cc4	&	5	&	10	&	15	&	20~&~&~
\\
\cc&	\cc5	&	5	&	10	&	15	&	20	&	25 & 
\\
\multirow{-6}{*}{\rotatebox[origin=c]{90}{\cc \bf Caster Level}}&	\cc 6 &	5	&	10	&	15	&20	&25	&30
\end{rndtable}
\end{center}
\subsection{Spell Accuracy}

Spells require an accuracy check in one of two circumstances:

\begin{itemize}
	\item The spell is classified as either {\it Blockable} or {\it Dodgeable}. 
	\item The target of the spell is far enough away, or small enough to trigger the `hard-to-hit' rules discussed on page \pageref{S:HardToHit}.
\end{itemize} 

Perform an accuracy check using the normal d20 dice. The modifier used is the same as the one that is used in the casting check - determined by the spell's discipline, plus the Expertise Bonus if applicable. 

\section{Fortitude Cost}


Casting spells is not as simple as waving your wands and saying the magic words -- it takes great mental clarity to cast, and you can become exhausted from casting difficult spells. This mental burden is enumerated through the Fortitude Points attribute. 

You cannot cast a spell if it would send you into negative FP -- you must wait for your head to clear before attempting that spell.  

The FP cost of casting a spell is determined by the difficulty of the spell - i.e. the spell level -- as shown in the table below:

{
\small
\def\wFP{1}
\begin{center}
	\begin{rndtable}{c c    c  c c  c}
		{\bf Beginner}	&	{\bf Novice}	&	{\bf Adept}	&	{\bf Expert}	&	{\bf Master}	& {\bf Ascendant}
		\\
	2	&	4	&	8	&	16	&	32	&	64
	\end{rndtable}
\end{center}
}
\subsection{Casting at Higher Level}

When memory-casting, some spells can be cast at varying levels of power - injecting more magical energy into the spell effect, and thereby increasing the effectiveness of these spells. 

If the spell description states that such additional effects are available, then you may choose to cast it as a higher level spell. You cannot cast it as a higher level spell than your current casting level, but you may choose any level between the spell's base level and your spellcasting level. 

Despite the fact that you are memory-casting the chosen spell, because you are casting the spell in an unfamiliar way, you do need to perform a casting check when casting a spell at a higher level. The DV and FP of the spell are equal to that of a normal spell of the chosen level. 

You may `memorise' the higher level spell in the same fashion as you may memorise a book-cast spell - by successfully casting it. At this point, you may forgo the casting check when casting the spell at any level below the one you have just re-memorised. 

{\bf Example:} Sarah is an 8th level witch, trying to levitate a boulder around 80kg in weight. Sarah has memorised the Beginner {\it Levitate} spell, which states it can lift only 1kg of matter. 

However, this mass is multiplied by 10 for every additional spell level. Since Sarah has access to Adept-level spells, she chooses to try to cast {\it Levitate} as an Adept spell, which allows her to lift up to 100kg. 

She successfully passes her DV 20 casting check, and the boulder is lifted. She still needs to pass the casting check to continue lifting boulders - but after 5 or 6, she has `memorised' the adept version of the spell, and does not need to pass the check any more. In the future, she may cast {\it Levitate} as either a Beginner, Novice or Adept level spell, without needing the check - she learned the Novice level for `free', by learning the more powerful Adept version.  



\section{Resisting Spells}

Even after a spell has successfully hit a target, it is possible for them to fight against the magic, reducing the effects and sometimes negating it entirely. 

This is normaly done by performing a {\it Resist} check before the spell effect is applied, and comparing it to the spellcaster\apos{} Resist DV. If the Resist is greater than or equal to the Resist DV of the spellcaster, the spell effect is modified as the spell description states. 

The Resist DV of a cast spell is enumerated through the {\it Subjugation} statistic:

$$\text{Subjugate} = 8 + \text{Expertise bonus}  + \text{POW modifier}$$



\section{Spell Range} \label{S:Range}

Some spells have effects which can apply over immense distances, whilst others infuse only the caster with magical energies, and some are only effective up to a certain distance. 

The maximum distance a spell can effect a person is known as the {\it range} of the spell. There are 4 classes of ranges for spells: {\it Self}, {\it Wandtip}, {\it Close} and {\it Sight}.

\subsection{Self}

Spells which have a range of `self' effect only the caster, or in this case of ritual spells involving multiple people, those involved with casting the spell.   

Some spells which fall into this category also extend to cover a given radius - in which case the `self' indicates that the focal point of the spell is the caster. 

\subsection{Wandtip}

A `Wandtip' spell has an extremely limited range. You have to hold your wand directly over the region or being you wish to target, or (in some cases) make physical contact between your wand and the target. 

\subsection{Close}

Most spells are considerd `close range' spells. This means that you can project the magic out of your wand a certain distance - but over extreme ranges, the magic becomes diluted and fizzles out. 

For most individuals, `close range' means the spell can be cast at a target up to 25m away. 

Some individuals have trained themselves to be particularly good at targeting spells at beings a long distance away, and have picked up the {\it Extended Range} skill, which allows them to cast spells further than they normally would. 

\subsection{Sight}

Sight spells are those which have practically no limitation on their range - the only limitation is your ability to detect and select a target.  

