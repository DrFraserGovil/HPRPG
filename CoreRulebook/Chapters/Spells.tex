\documentclass[../CoreRulebook.tex]{subfile}


\chapter{Spells}

Spells are how Wizards and Witches get by in the world, it is what makes them special. 



\section{Casting Spells}

Spells are broken down into 7 categories: Hexes \& Curses, Transfiguration, Charms, Recuperation, Illusion, Divination, and Dark Arts. Different types of spells require different abilities to cast: illusion spells often require the charisma and deception to overpower the target{\apos}s willpower, whilst hexes and curses often rely on sheer magical power for their effectiveness.

All spells are cast by performing the relevant checks -- rolling a dice, and then adding on the associated skill modifiers and bonuses for that check type -- and then deducting the associated fortitude cost. Each spell has its own check type, which is specified in the spell list below. The dice that you get to use to perform the casting check depends on the level of the skill that you have in the associated school of magic (i.e. the 7 skills associated with each of the 7 schools of magic). 

If you possess enough FP to make the cast, and the casting check is greater than or equal to the difficulty of the spell (also listed in the spell tables), then the spell is successfully cast. 

The dice that you get are enumerated below:

\begin{center}
	\begin{rndtable}{|c c c|}

	\bf Level & \bf  Name & \bf Die
	\\ 
	1 & Beginner & 1d6
	\\
	2 & Novice & 1d8
	\\
	3 & Adept & 1d10 (with 0 = 10)
	\\
	4 & Expert & 1d12
	\\
	5 & Master & 1d20
	\\ \hline
	\end{rndtable}
\end{center}

As you get more and more powerful in each of the 7 schools of magic, you therefore unlock larger and larger dice, which grants you access to more powerful spells, and also makes casting spells of a lower level easier and easier (as there is less chance of failure). 

There are two kinds of spell: an effect-only spell, which either succeeds or does not; and a power-dependent spell, where the outcome of the spell can vary depending on the success of the casting. 

Effect-Only spells are the most basic kind of spell, for example, the Fire-Starting charm (\textit{incendio}), either starts a fire, or it does not, and the Compass Spell (\textit{point me}) either tells you which way North is, or it does not. These spells are therefore in the first class: effect-only. Contrast this with the Torturing Curse (\textit{crucio}). This can not only succeed or fail, but can cause varying amounts of pain. It is therefore a power-dependent spell, as the success of the spell can vary.

Effect-only (E-Class) spells are cast by performing the check type specified in the tables below.

Power-dependent (P-class) spells have the same casting mechanism except you may choose to donate PWR points towards the spell before the check is initialised, up to the total PWR of your character. Each additional PWR point dedicated towards the spell increases the effectiveness of the spell, doing more damage, or adding more side effects. However, powerful spells are more difficult to cast: for each power point you add on, you increase the difficulty value and the fortitude cost of the spell by 1.

Hence, a spell which does 1d8 of damage to target, with a difficulty of 10 and costing 5 Fortitude can be boosted with 5 PWR points to do 1d8 + 5 damage, but the boosted spell has a difficulty of 15, and a Fortitude cost of 10. 
Hence, there is a higher chance of failure, but the rewards for success are much higher! The maximum number of PWR points you can assign a spell is limited by your PWR attribute itself (you can{\apos}t assign more PWR than you have).

The GM may decide that a particular usage of an E-Class spell warrants an extra effort, despite a nominal effect-only status, i.e. setting light to an entire building, rather than starting a campfire would still use incendio, but clearly is a far more powerful use of the firelighting charm! The GM has the authority to override the difficulty and checks required for a spell if the situation calls for it. 

The Fortitude cost for a spell is deducted only after it is successfully cast. If the casting fails, then only half of the fortitude cost is deducted (rounded up), plus whatever negative effects the failed cast might have. 

Oh, one final rule. If your character is saying the incantation, so must you. Bonus points for good acting.



\section{Learning New Spells}

Spells are learned by studying, either from books, or from a teacher.
 
To learn a spell, you must purchase a spell book from a vendor (or find a Professor who already knows the spell), and dedicate an entire day to learning that spell. When that day is up, you get three attempts to cast the spell using the usual casting check. 

If at least one of these casting checks succeeds, you have successfully learned the spell, and you may add it to your arsenal. If all three checks fail, you go have to go back to the start, and begin the learning process again, losing another day in the process. 

Each spellbook contains three spells that you can learn, before you must purchase another, and you are originally limited to learning only 3 spells per level. When you level up, this counter resets, and you may learn another 3 spells.

A professor who already knows the spell you want to learn is an adequate replacement for a spellbook, and may teach an unlimited number of students an unlimited number of spells, but only if they already know the spell. A professor must learn a spell in the usual fashion (unless they can find another professor to teach them!)

\section{Wards}

A ward is (usually) a Recuperation spell that affects a large area. A ward may be centred on a fixed point or object, or may be centred on a moving location or even a sentient being. 

Wards, however, have an unfortunate habit of interfering with each other when used in unison. If two wards have a significant overlapping region of effect and the caster does not have the {\it Multiward} skill or an equivalent feat, there is a significant chance (determined by the GM) that both wards will collapse. 

The interference only applies if the wards are similar in magnitude and intent. For example, Hogwarts castle is a heavily warded region, but a small ward could be placed in a room without problem. Interference would only become a problem when a new castle-wide ward was attempted. 

Equally, intereference only applies if the effects of the ward compound each other -- if they lie in opposition, then the usual spell mechanics are applied. For example, a character with a personal shield ward touches a beartrap ward -- neither ward collapses, but the beartrap ward is triggered, and the shield will attempt to protect the character. 
\cleartoleftpage
\onecolumn
\chapter{Spell List}

This section contains a list of all the spells available in the game. First, the spells are presented broken down into the school and level to which they belong. The next section then contains a full description of the spell, including its casting check, casting difficulty, and spell effects.

Spells marked with a (*) gain more effects, or increase in power, at higher levels. 

\define@key{spell}{name}{\def\name{#1}}
\define@key{spell}{incant}{\def\incant{#1}}
\define@key{spell}{school}{\def\school{#1}}
\define@key{spell}{type}{\def\type{#1}}
\define@key{spell}{level}{\def\level{#1}}
\define@key{spell}{fp}{\def\fp{#1}}
\define@key{spell}{attribute}{\def\att{#1}}
\define@key{spell}{proficiency}{\def\prof{#1}}
\define@key{spell}{dv}{\def\dv{#1}}
\define@key{spell}{effect}{\def\effect{#1}}
\define@key{spell}{duration}{\def\duration{#1}}
\define@key{spell}{noIncant}{\def\incantMode{#1}}
\define@key{spell}{noProf}{\def\profMode{#1}}
\define@key{spell}{noDur}{\def\durMode{#1}}
\define@key{spell}{higher}{\def\higher{#1}}
\define@key{spell}{noHigh}{\def\highMode{#1}}
\newcommand{\spell}[1]
{
	\setkeys{spell}{name=None,incant = -, school = None, type = None, level = 0, fp = 0, attribute = None, proficiency = None, dv = 0, effect = None,noIncant = 0,noProf=0,noDur=0,duration=None,higher=None,noHigh=0}
	\setkeys{spell}{#1}
	\vbox{
	{\normalsize \color{rulered}\name }
	\footnotesize
	
		
	{\it \level{}-level \school{}} 
	\vspace{1ex}
	
	
	\if\incantMode0 
		{\bf Incantation:} {\it \incant}
	\fi
	
	{\bf Spell Type:}~~~\type
	
	\if\durMode0
		{\bf Duration:}~~~~~~\duration
	\fi
	
	{\bf Fortitude:}~~~~~\fp
	
	{\bf Check:}~~~~~~\quad~\att
	\if\profMode0 
		({\it \prof})
	\fi
	, ~~DV  \dv
	%~{\bf DV:}~~~~~~~~~\quad~~~~\dv
	
	\vspace{1ex}
	
	\effect
	
	\vspace{1 ex}
	
	\if\highMode0
		{\bf Higher Level Casting: } 
		
		\higher
	\fi
	}
	\vspace{3ex}
	
	\small
}
\newcommand{\cvdv}{for each point that the casting check exceeds the difficulty value}

\small
\setlength{\parskip}{0em}
\if \coreMode1
	\begin{multicols}{4} \raggedbottom\subsubsection{Charms}
\textbf{Level 1 Spells}
\begin{itemize}[itemsep=0em]
\renewcommand\labelitemi{-}
\item Banshee Wail

\item Create Fire

\item Create Trap

\item Create Water

\item Fresh Air

\item Halt

\item Illuminate Wand

\item Levitation

\item Mark Surface

\item Preserve Object


\end{itemize}
\textbf{Level 2 Spells}
\begin{itemize}[itemsep=0em]
\renewcommand\labelitemi{-}
\item Cut Object

\item Fix Object

\item Haste

\item Lightning Bolt

\item Lock

\item Mage Hands

\item Silence

\item Smokescreen

\item Stick

\item Summon Object

\item Unlock


\end{itemize}
\subsubsection{Dark Arts}
\textbf{Level 1 Spells}
\begin{itemize}[itemsep=0em]
\renewcommand\labelitemi{-}
\item Blight

\item Eldritch Knowledge

\item Shadow Blast

\item Shroud of Darkness

\item Vicious Slash


\end{itemize}
\textbf{Level 2 Spells}
\begin{itemize}[itemsep=0em]
\renewcommand\labelitemi{-}
\item Contagion

\item Dark Healing

\item Incomprehensible Torture

\item Instill Terror

\item Necrosis

\item Plague of Insects

\item Summon Void


\end{itemize}
\vfill\null
\columnbreak\subsubsection{Divination}
\textbf{Level 1 Spells}
\begin{itemize}[itemsep=0em]
\renewcommand\labelitemi{-}
\item Astral Assistance

\item Hunter\apos{}s Mark

\item Identify

\item Locate

\item Receive Omen

\item Sense Traps

\item Speak in Tongues

\item Telepathic Bond


\end{itemize}
\textbf{Level 2 Spells}
\begin{itemize}[itemsep=0em]
\renewcommand\labelitemi{-}
\item All\minus{}seeing Eye

\item Astral Caltrops

\item Crystal Gazing

\item Detect Magic

\item Detect Thoughts

\item Eavesdrop

\item Obfuscation


\end{itemize}
\subsubsection{Hexes \& Curses}
\textbf{Level 1 Spells}
\begin{itemize}[itemsep=0em]
\renewcommand\labelitemi{-}
\item Acidic Burst

\item Cause Confusion

\item Green Sparks

\item Knockback

\item Sting

\item Trip


\end{itemize}
\textbf{Level 2 Spells}
\begin{itemize}[itemsep=0em]
\renewcommand\labelitemi{-}
\item Arctic Chill

\item Cascading Missiles

\item Disarm

\item Hoist Enemy

\item Mental Burden

\item Perpetual Hunger

\item Prevent Movement

\item Strangle

\item Stunning Blast

\item Summon Bat Bogeys


\end{itemize}
\vfill\null
\columnbreak\subsubsection{Illusion}
\textbf{Level 1 Spells}
\begin{itemize}[itemsep=0em]
\renewcommand\labelitemi{-}
\item Blur

\item Chaotic Whispers

\item Charm Entity

\item Glamour

\item Imbue Bravery

\item Night Vision

\item Throw Voice


\end{itemize}
\textbf{Level 2 Spells}
\begin{itemize}[itemsep=0em]
\renewcommand\labelitemi{-}
\item Blinding Light

\item Calming Aura

\item Conceal Inscription

\item Enchant Animal

\item Piper{\apos}s Illusion

\item Sleep

\item Violent Phantasms


\end{itemize}
\subsubsection{Recuperation}
\textbf{Level 1 Spells}
\begin{itemize}[itemsep=0em]
\renewcommand\labelitemi{-}
\item Aid Charm

\item Caterwauling Ward

\item Magical Shield

\item Minor Healing

\item Privacy Ward

\item Reinforce Shield

\item Stoneskin

\item Sunburst


\end{itemize}
\textbf{Level 2 Spells}
\begin{itemize}[itemsep=0em]
\renewcommand\labelitemi{-}
\item Anti\minus{}Muggle Ward

\item Checkup

\item Countercurse

\item Heal Being

\item Lesser Ward

\item Release Trapped Being

\item Runic Shield

\item Stabilise Patient


\end{itemize}
\vfill\null
\columnbreak\subsubsection{Transfiguration}
\textbf{Level 1 Spells}
\begin{itemize}[itemsep=0em]
\renewcommand\labelitemi{-}
\item Alter Hair

\item Basic Transmutation

\item Change Colour

\item Conjure Flowers

\item Launch Spike

\item Potion Mixing Spell

\item Silver Shield

\item Steelclaw


\end{itemize}
\textbf{Level 2 Spells}
\begin{itemize}[itemsep=0em]
\renewcommand\labelitemi{-}
\item Conjure Bubble

\item Fabricate Object

\item Harden Object

\item Make Trecherous

\item Pumpkin Head

\item Summon Snake

\item Summon Weak Avatar

\item Thick Air


\end{itemize}
\end{multicols}\clearpage\begin{multicols}{3}\spell{name = Acidic Burst, incant = ambustum, school = Hexes \& Curses, type = Instant, level =Beginner, fp = 3, attribute =POW, proficiency = , noProf = 1, dv = 4, effect =Fills a 5m target area with an acidic cloud that does (1+ CV � DV) acid damage per turn. Cloud lasts for 10 cycles\comma{} unless in a confined space\comma{} where it lasts until removed by other means.}
\spell{name = Aid Charm, incant = subsidium, school = Recuperation, type = Instant, level =Beginner, fp = 2, attribute =EMP, proficiency = , noProf = 1, dv = 4, effect =Target has HP ceiling raised by 5 points for 1 hour}
\spell{name = All\minus{}seeing Eye, incant = orbis, school = Divination, type = Instant, level =Novice, fp = 4, attribute =INT, proficiency = Perception, dv = 6, effect =You may create an invisible\comma{} magic eye in front of you\comma{} that hovers. You mentally see everything that the eye sees\comma{} and may use a major action to instruct the eye to move up to 10m in any direction (including vertical). Eye cannot pass through solid walls\comma{} but may squeeze through gaps as small as 4cm in diameter.}
\spell{name = Alter Hair, incant = crinus muto, school = Transfiguration, type = Instant, level =Beginner, fp = 2, attribute =CHR, proficiency = Deception, dv = 4, effect =Alters the colour and style of the casters hair. Useful for disguises. Degrades after 5 hours.}
\spell{name = Anti\minus{}Muggle Ward, incant = repello mugletum, school = Recuperation, type = Ward, level =Novice, fp = 6, attribute =INT, proficiency = , noProf = 1, dv = 7, effect =Forms a warded area that muggles can neither see\comma{} nor enter. 
The warded area is a circle 5m in radius + 5m for every power point dedicated to the spell.}
\spell{name = Arctic Chill, incant = gelidus, school = Hexes \& Curses, type = Concentration, level =Novice, fp = 7, attribute =POW, proficiency = , noProf = 1, dv = 6, effect =An area of (2 + PP) metres around the target is decreased in temperature by 50 degrees celsius. Those caught in the region take (1+PP)d4 of cold damage\comma{} and apply the mild Frostbite status effect.}
\spell{name = Astral Assistance, incant = auxilio, school = Divination, type = Ritual (2 turns), level =Beginner, fp = 5, attribute =EMP, proficiency = Arcane, dv = 5, effect =By laying your hand upon a sapient being\comma{} you may channel magical energy into them. On the next check the target performs\comma{} roll 1d4\comma{} and add it to the check (+1 per PP\comma{} max 3). If the check fails\comma{} both the target and the caster take 1d6 psychic damage.}
\spell{name = Astral Caltrops, incant = Caltrops, school = Divination, type = Instant, level =Novice, fp = 5, attribute =SPR, proficiency = , noProf = 1, dv = 7, effect =The target acts as if the terrain posseses caltrops for 10 turns. Caltrops do psychic damage}
\spell{name = Banshee Wail, incant = magnus surgerus, school = Charms, type = Instant, level =Beginner, fp = 3, attribute =SPR, proficiency = Performance, dv = 5, effect =All targets in hearing range take 2 points of psychic damage (+3 per PP)\comma{} and awaken if they are sleeping.}
\spell{name = Basic Transmutation, incant = formum mutatio, school = Transfiguration, type = Instant, level =Beginner, fp = 4, attribute =FIN, proficiency = , noProf = 1, dv = 4, effect =Transform a 200g non\minus{}sapient animal or object into a different animal or solid object. 
Each power point doubles the mass of objects that can be transformed.  Lasts for 1 hour. Objects must be simple in nature.}
\spell{name = Blight, incant = thanatos, school = Dark Arts, type = Instant, level =Beginner, fp = 4, attribute =EVL, proficiency = , noProf = 1, dv = 5, effect =A wave of necrotic energy extends outwards from you in a radius of 10m (doubled with every PP\comma{} max 1km). All plants within range die instantly\comma{} and all other living beings take 1d4 necrotic damage (+1 per PP)}
\spell{name = Blinding Light, incant = caecus, school = Illusion, type = Instant, level =Novice, fp = 5, attribute =POW, proficiency = , noProf = 1, dv = 5, effect =Direct a brilliant beam of light at the target. If target fails an INT(Perception) Resist check against the casting check\comma{} they are blinded for 4 turns.}
\spell{name = Blur, incant = celeritate, school = Illusion, type = Instant, level =Beginner, fp = 4, attribute =CHR, proficiency = , noProf = 1, dv = 3, effect =The target seems to become blurry around the edges\comma{} it is difficult to tell exactly where they are\comma{} and where they aren{\apos}t.
Gain check advantage on evasion checks for 3 turns.}
\spell{name = Calming Aura, incant = paxus, school = Illusion, type = Instant, level =Novice, fp = 8, attribute =CHR, proficiency = Willpower, dv = 6, effect =Calms the target down. Remove terrified status from target.}
\spell{name = Cascading Missiles, incant = unda delor, school = Hexes \& Curses, type = Instant, level =Novice, fp = 5, attribute =FIN, proficiency = , noProf = 1, dv = 7, effect =Fires multiple bursts of energy that do 3+1d4 force damage to up to (3+PP) targets. 
Each power point added adds +1 damage\comma{} and +1 target.}
\spell{name = Caterwauling Ward, incant = caterwaul, school = Recuperation, type = Ward, level =Beginner, fp = 4, attribute =INT, proficiency = , noProf = 1, dv = 3, effect =Casts a ward on the area which emits a high\minus{}pitched scream when an unknown being crosses the threshold. 
Radius is (10 + $2\times$PP) metres. Ward decays after 2 weeks.}
\spell{name = Cause Confusion, incant = confundo, school = Hexes \& Curses, type = Instant, level =Beginner, fp = 5, attribute =CHR, proficiency = Deception, dv = 3, effect =Do 2 Fatigue damage.
Target performs a Resist Magic check against casting check\comma{} if it fails\comma{} then target acquires the Confused status.  If it succeeds\comma{} do 5 fatigue damage.}
\spell{name = Change Colour, incant = pigmentus, school = Transfiguration, type = Instant, level =Beginner, fp = 4, attribute =INT, proficiency = , noProf = 1, dv = 4, effect =Causes the colour of an object to change. Lasts for 2 days.}
\spell{name = Chaotic Whispers, incant = rastarum, school = Illusion, type = Concentration, level =Beginner, fp = 4, attribute =SPR, proficiency = Deception, dv = 4, effect =Target hears a voice in their ear whispering maddening words\comma{} that slowly drive them insane. Target takes (1 + PP)d4  psychic damage per turn\comma{} until they pass a SPR(endurance) Resist check with DV = casting check.}
\spell{name = Charm Entity, incant = sismeus amici, school = Illusion, type = Instant, level =Beginner, fp = 5, attribute =CHR, proficiency = Persuasion, dv = 4, effect =Causes the target to like you\comma{} persuasion checks get a (2+PP) bonus\comma{} max 5.}
\spell{name = Checkup, incant = dispungo, school = Recuperation, type = Instant, level =Novice, fp = 6, attribute =EMP, proficiency = Understand Other, dv = 6, effect =Enquire as to the health status of the target\comma{} find out their remaining HP\comma{} as well as any status effects they currently posses.}
\spell{name = Conceal Inscription, incant = occulto, school = Illusion, type = Instant, level =Novice, fp = 6, attribute =INT, proficiency = Deception, dv = 4, effect =Makes a message\comma{} drawing or marking on a surface invisible to the naked eye.}
\spell{name = Conjure Bubble, incant = ebublio, school = Transfiguration, type = Instant, level =Novice, fp = 6, attribute =FIN, proficiency = , noProf = 1, dv = 6, effect =Conjures a large\comma{} hard\minus{}to\minus{}pop airtight bubble (strength 8) which the caster can use to encase enemies\comma{} or to protect themselves.}
\spell{name = Conjure Flowers, incant = orchideous, school = Transfiguration, type = Instant, level =Beginner, fp = 3, attribute =EMP, proficiency = , noProf = 1, dv = 5, effect =Conjures flowers from thin air. Lasts for 3 days.}
\spell{name = Contagion, incant = vastantes, school = Dark Arts, type = Instant, level =Novice, fp = 6, attribute =EVL, proficiency = , noProf = 1, dv = 6, effect =If target fails a Resist (health) check against the casting check\comma{} contracts a disease. All positive modifiers and proficiency bonuses are set to zero until cured. Disease is contagious and each time they touch an afflicted individual\comma{} being must Resist\comma{} or contract the disease also.}
\spell{name = Countercurse, incant = finite maledictum, school = Recuperation, type = Instant, level =Novice, fp = 5, attribute =INT, proficiency = Arcane, dv = 6, effect =Remove the effects of an active curse or hex. The caster of the curse performs the casting check again: If the casting check of the counterpsell (+1 for each power point dedicated) is larger than this second check\comma{} the curse is removed.}
\spell{name = Create Fire, incant = incendio, school = Charms, type = Concentration, level =Beginner, fp = 3, attribute =SPR, proficiency = , noProf = 1, dv = 3, effect =A small jet of fire is emitted from the tip of your wand. 
Coming into contact with fire does 1d6 fire damage\comma{} and applies a minor Burned status effect.
(Larger jets of fire have a difficulty of 9\comma{}  do 4d6  fire damage and apply a Moderate burn)}
\spell{name = Create Trap, incant = dolus, school = Charms, type = Ritual (2 turns), level =Beginner, fp = 4, attribute =FIN, proficiency = Stealth, dv = 5, effect =Combine a magical ward with one of your existing spells. Cast the other spell first\comma{} then perform the trapping check.
If successful\comma{} creates a hidden magical trap of radius 50cm on any solid surface\comma{} with the effect of the original spell when triggered by an entity touching the trap. The effects of the trap are less than the original spell\comma{} but more power points make the trap more powerful. 
If you wish to keep a trap hidden from the GM\comma{} write down the location\comma{} spell and associated check values on a piece of paper\comma{} to be revealed when the trap is triggered.}
\spell{name = Create Water, incant = aguamente, school = Charms, type = Concentration, level =Beginner, fp = 4, attribute =INT, proficiency = , noProf = 1, dv = 4, effect =A jet of water is emitted from the tip of your wand\comma{} useful for extinguishing fires\comma{} or cleaning surfaces. 
(Larger jets of water have a difficulty of 16. Conjured water cannot be drunk)}
\spell{name = Crystal Gazing, incant = Gazing, school = Divination, type = Ritual (4 turns), level =Novice, fp = 3, attribute =EMP, proficiency = , noProf = 1, dv = 7, effect =Gaze into your crystal ball\comma{} and ask a question of the cosmos. You will receive a yes or a no answer to any question you ask.}
\spell{name = Cut Object, incant = diffindo, school = Charms, type = Instant, level =Novice, fp = 6, attribute =FIN, proficiency = Precision, dv = 5, effect =Cut two objects apart. 
If used on a living being\comma{} causes a deep cut\comma{} for 1d6 + 3 slashing damage.}
\spell{name = Dark Healing, incant = tenebrosa sudarium, school = Dark Arts, type = Instant, level =Novice, fp = 6, attribute =POW, proficiency = , noProf = 1, dv = 5, effect =Heal for one HP for each casting point over the difficulty. Remove half of this HP from an ally.}
\spell{name = Detect Magic, incant = revelio, school = Divination, type = Instant, level =Novice, fp = 5, attribute =EMP, proficiency = , noProf = 1, dv = 5, effect =Reveals to the caster any active spells in the in 15m range. Will deactivate charms whose sole purpose is to remain hidden.}
\spell{name = Detect Thoughts, incant = psychopractum, school = Divination, type = Concentration, level =Novice, fp = 15, attribute =EMP, proficiency = Understand Other, dv = 7, effect =You may observe the mind of a target individual. Unlike legilimency\comma{} thought\minus{}detection is not an exact science\comma{} and you will only get a vague shape of their thoughts. If casting check is below the target{\apos}s passive perception\comma{} they become aware of the process.}
\spell{name = Disarm, incant = expelliarmus, school = Hexes \& Curses, type = Instant, level =Novice, fp = 6, attribute =POW, proficiency = , noProf = 1, dv = 6, effect =Target performs a Resist Magic check\comma{} if casting check exceeds Resist check\comma{} then the object in the target{\apos}s hand is hurled in a random direction.}
\spell{name = Eavesdrop, incant = dumauris, school = Divination, type = Concentration, level =Novice, fp = 5, attrurns.}
\spell{name = Blight, incant = thanatos, school = Dark Arts, type = Cast, level =Beginner, fp = 4, attribute =EVL, proficiency = , dv = 5, effect =A wave of necrotic energy extends outwards from you in a radius of 10m (doubled with every PP\comma{} max 1km). All plants within range die instantly\comma{} and all other living beings take 1d4 necrotic damage (+1 per PP)}
\spell{name = Blingind Light, incant = caecus, school = Illusion, type = Effect Only, level =Novice, fp = 5, attribute =POW, proficiency = , dv = 5, effect =Direct a brilliant beam of light at the target. If target fails an INT(Perception) Resist check\comma{} they are blinded for 4 turns.}
\spell{name = Blood Barrier, incant = confusangui, school = Dark Arts, type = Ritual (8 turns), level =Expert, fp = 5, attribute =EVL, proficiency = , dv = 8, effect =Use blood to draw warding runes onto an object or person. Erects a swirling red magical barrier with AC 10\comma{} plus 5 for every casting point over the difficulty. Barrier blocks all physical and magical damage and is immune to acid erosion\comma{} but is eroded by holy damage. 
Each individual{\apos}s blood can only be used once for blood magic.}
\spell{name = Blur, incant = celeritate, school = Illusion, type = Effect Only, level =Beginner, fp = 4, attribute =CHR, proficiency = , dv = 3, effect =The target seems to become blurry around the edges\comma{} it is difficult to tell exactly where they are\comma{} and where they aren{\apos}t.
Gain check advantage on evasion checks for 3 turns.}
\spell{name = Boost Health, incant = levo, school = Recuperation, type = Cast, level =Expert, fp = 11, attribute =EMP, proficiency = Healing, dv = 9, effect =Give the target a temporary +150\caster\comma{} removing any gaseous effects and smelling faintly of lavender.}
\spell{name = Glamour, incant = lux stultium, school = Illusion, type = Instant, level =Beginner, fp = 3, attribute =INT, proficiency = Deception, dv = 3, effect =Create a superficial glamour around a person\comma{} a simple trick of the light. The glamour disintegrates upon physical or magical contact.}
\spell{name = Green Sparks, incant = verdimillious, school = Hexes \& Curses, type = Instant, level =Beginner, fp = 4, attribute =FIN, proficiency = , noProf = 1, dv = 4, effect =Emits (5+PP) green sparks from your wand\comma{} which can be made to strike at the enemy. 
Each spark does (1 + CV � DV) force damage.}
\spell{name = Halt, incant = stabit, school = Charms, type = Instant, level =Beginner, fp = 2, attribute =SPR, proficiency = , noProf = 1, dv = 4, effect =Stop 1 inanimate object (+1 for every power point dedicated) in its tracks\comma{} if mid\minus{}air\comma{} it drops to the ground.}
\spell{name = Harden Object, incant = duro, school = Transfiguration, type = Instant, level =Novice, fp = 4, attribute =INT, proficiency = , noProf = 1, dv = 7, effect =Freezes a non\minus{}living object into its current form. Object gains an effective AC of 25. Lasts for 2 days.}
\spell{name = Haste, incant = silvam currere, school = Charms, type = Instant, level =Novice, fp = 5, attribute =INT, proficiency = , noProf = 1, dv = 5, effect =The target has their Speed proficiency increased by 1 point (+1 per PP) for 5 minutes. At the end of the effect\comma{} target must take 1 turn to rest.}
\spell{name = Heal Being, incant = episkey, school = Recuperation, type = Instant, level =Novice, fp = 5, attribute =EMP, proficiency = Healing, dv = 5, effect =Heal minor status effects like burns\comma{} bruises\comma{} broken noses and so on. If no status effect present\comma{} heal for 2HP + two for each CV over DV}
\spell{name = Hoist Enemy, incant = levicorpus, school = Hexes \& Curses, type = Instant, level =Novice, fp = 8, attribute =FIN, proficiency = , noProf = 1, dv = 8, effect =Target is hoisted into the air. Whilst airborne\comma{} all checks by the target suffer a \minus{}2 penalty. 
Caster can then throw target up to 2 metres in any direction\comma{} with the target taking 1d6 damage.}
\spell{name = Hunter\apos{}s Mark, incant = venari, school = Divination, type = Instant, level =Beginner, fp = 3, attribute =INT, proficiency = , noProf = 1, dv = 4, effect =If casting check exceeds passive resist value\comma{} caster is aware of the location of the target for the next 3 days\comma{} or until the mark is removed by magical means.}
\spell{name = Identify, incant = dicemi, school = Divination, type = Instant, level =Beginner, fp = 6, attribute =INT, proficiency = Research, dv = 3, effect =Learn the properties of the target: be it learning about the nature of the target\comma{} or the ingredients of a potion.
The more power points dedicated to the spell\comma{} the more information that is revealed.}
\spell{name = Illuminate Wand, incant = lumos, school = Charms, type = Concentration, level =Beginner, fp = 1, attribute =INT, proficiency = , noProf = 1, dv = 2, effect =Causes the tip of your wand to glow\comma{} like a torch. Casts bright light for 2m radius\comma{} and dim light for 10m. Spell last indefinitely\comma{} until the counterspell (knox) is used. No other spells can be used whilst lumos is active.}
\spell{name = Imbue Bravery, incant = fortudus, school = Illusion, type = Instant, level =Beginner, fp = 2, attribute =SPR, proficiency = Persuasion, dv = 3, effect =Imbue your target with fortitude and vigour. They gain check\minus{}advantage on all Fear\minus{}Resist checks for 1 hour.}
\spell{name = Incomprehensible Torture, incant = Crucio, school = Dark Arts, type = Concentration, level =Novice, fp = 8, attribute =EVL, proficiency = Chaos, dv = 6, effect =Causes immense pain to the target\comma{} paralysing them whilst the spell is cast. 
Does (PP)d4 psychic damage per turn.}
\spell{name = Instill Terror, incant = timeant, school = Dark Arts, type = Instant, level =Novice, fp = 5, attribute =CHR, proficiency = Intimidation, dv = 6, effect =Target performs a SPR (endurance) Resist check\comma{} if the check is less than the casting check\comma{} the target acquires the {\it Terrified} status.}
\spell{name = Knockback, incant = flipendo, school = Hexes \& Curses, type = Instant, level =Beginner, fp = 3, attribute =POW, proficiency = , noProf = 1, dv = 3, effect =Causes 2 points of force damage\comma{} and knocks the target back 1 metre. Each power point adds one metre to the knockback distance and 1 damage point. May need to consider impact (see `falling�)}
\spell{name = Launch Spike, incant = voco dens, school = Transfiguration, type = Instant, level =Beginner, fp = 4, attribute =POW, proficiency = , noProf = 1, dv = 4, effect =Conjure 1 enormous spike (+ 1 for each power point) to transfigure itself from the surrounding walls/floor\comma{} impaling the target. Each spike does 1d6 piercing damage.}
\spell{name = Lesser Ward, incant = tueor, school = Recuperation, type = Ward, level =Novice, fp = 5, attribute =POW, proficiency = , noProf = 1, dv = 6, effect =Erects a ward in a cylinder around an individual. Ward is 20cm larger in radius than the individual is wide\comma{} and 20cm taller. This ward protects you from up to 15 damage of all types\comma{} before it fails. Ward may move with the target\comma{} and can be cast on self. 
Ward disintegrates after 1 day.}
\spell{name = Levitation, incant = wingardium leviosa, school = Charms, type = Concentration, level =Beginner, fp = 5, attribute =FIN, proficiency = , noProf = 1, dv = 4, effect =Cause an object of 500g or less to levitate\comma{} controlling the vertical distance at will. 
Each power point dedicated doubles the mass of the object that can be lifted.}
\spell{name = Lightning Bolt, incant = baubilious, school = Charms, type = Instant, level =Novice, fp = 8, attribute =POW, proficiency = , noProf = 1, dv = 5, effect =Releases a bolt of lightning from the end of your wand. 
Lightning can initiate fires\comma{} blind your foes\comma{} or can be used directly in combat\comma{} where it deals 1 damage for every check point over the difficulty\comma{} + 3 for every power point dedicated.}
\spell{name = Locate, incant = locus, school = Divination, type = Instant, level =Beginner, fp = 3, attribute =EMP, proficiencout of a vial\comma{} but cannot use them to attack.}
\spell{name = Magical Shield, incant = protego, school = Recuperation, type = Concentration, level =Beginner, fp = 5, attribute =POW, proficiency = , dv = 5, effect =Erects an ethereal shield in front of you that absorbs incoming magical attacks.
Shielding charm increases AC by 15+PP against all incoming spells\comma{} but does not protect against physical damage\comma{} or the aftereffects of magic (i.e. a nearby explosion)}
\spell{name = Make Trecherous, incant = transgresso, school = Transfiguration, type = Cast, level =Novice, fp = 8, attribute =INT, proficiency = , dv = 6, effect =Transform the ground in a 5m radius around target into a deep bog\comma{} a bed of sharpened blades\comma{} or into a sticky mess\comma{} with the associated terrain costs.}
\spell{name = Mark Surface, incant = stylum, school = Charms, type = Concentration, level =Beginner, fp = 2, attribute =FIN, proficiency = Dxterity, dv = 2, effect =Use your wand as anything from a thin marker to a thick paintbrush\comma{} the {\it paint} is a magical adhesive that sticks to any surface\comma{} and may be of any colour you choose.}
\spell{name = Mental Burden, incant = onus, school = Hexes \& Curses, type = Cast, level =Novice, fp = 5, attribute =SPR, proficiency = Willpower, dv = 8, effect =If the target fails a resist magic check\comma{} all spells cost 2FP more than their stated value for 6 turns.}
\spell{name = Minor Healing, incant = enervate, school = Recuperation, type = Concentration, level =Beginner, fp = 3, attribute =EMP, proficiency = Healing, dv = 3, effect =Heal for 2 points per turn. 
If the target has a serious wound\comma{} i.e. a broken bone\comma{} cannot heal beyond 50\te =FIN, proficiency = Dxterity, dv = 2, effect =Use your wand as anything from a thin marker to a thick paintbrush\comma{} the {\it paint} is a magical adhesive that sticks to any surface\comma{} and may be of any colour you choose.}
\spell{name = Mental Burden, incant = onus, school = Hexes \& Curses, type = Instant, level =Novice, fp = 5, attribute =SPR, proficiency = Willpower, dv = 8, effect =If the target fails a resist magic check\comma{} all spells cost 2FP more than their stated value for 6 turns.}
\spell{name = Minor Healing, incant = enervate, school = Recuperation, type = Concentration, level =Beginner, fp = 3, attribute =EMP, proficiency = Healing, dv = 3, effect =Heal for 2 points per turn. 
If the target has a serious wound\comma{} i.e. a broken bone\comma{} cannot heal beyond 50\% health. Only works on living creatures.}
\spell{name = Necrosis, incant = carnes mortis, school = Dark Arts, type = Instant, level =Novice, fp = 6, attribute =POW, proficiency = Chaos, dv = 7, effect =Do 1d4 necrotic damage and 1d4 poison damage for every point over the casting check.}
\spell{name = Night Vision, incant = aspectu, school = Illusion, type = Instant, level =Beginner, fp = 3, attribute =EMP, proficiency = Perception, dv = 3, effect =Give the target nightvision for one hour: dim light is as bright as daylight\comma{} and darkness is consdiered dim.}
\spell{name = Obfuscation, incant = obscuras, school = Divination, type = Ritual (1 hour), level =Novice, fp = 7, attribute =POW, proficiency = Willpower, dv = 6, effect =Those attempting to use divination to spy on you must pass a SPR (perception) Resist check  (difficulty 10 + PP) for the spell to work.
Lasts for one week.}
\spell{name = Perpetual Hunger, incant = inedia, school = Hexes \& Curses, type = Instant, level =Novice, fp = 6, attribute =SPR, proficiency = , noProf = 1, dv = 7, effect =The afflicted feels perpetual\comma{} soul\minus{}sapping hunger. Every turn where food is not consumed\comma{} suffer 2 necrotic damage. Lasts for 10 turns.}
\spell{name = Piper{\apos}s Illusion, incant = , noIncant = 1, school = Illusion, type = Ritual (2 turns), level =Novice, fp = 6, attribute =CHR, proficiency = Performance, dv = 5, effect =If one has an instrument\comma{} this spell hypnotises all those who hear it and fail SPR (willpower) Resist check. When the spelncy = \comma{} dv = 5\comma{} effect =Erects an ethereal shield in front of you that absorbs incoming magical attacks.
Shielding charm increases AC by 15+PP against all incoming spells\comma{} but does not protect against physical damage\comma{} or the aftereffects of magic (i.e. a nearby explosion)}
\spell{name = Make Trecherous\comma{} incant = transgresso\comma{} school = Transfiguration\comma{} type = Cast\comma{} level =Novice\comma{} fp = 8\comma{} attribute =INT\comma{} proficiency = \comma{} dv = 6\comma{} effect =Transform the ground in a 5m radius around target into a deep bog\comma{} a bed of sharpened blades\comma{} or into a sticky mess\comma{} with the associated terrain costs.}
\spell{name = Mark Surface\comma{} incant = stylum\comma{} school = Charms\comma{} type = Concentration\comma{} level =Beginner\comma{} fp = 2\comma{} attribute =FIN\comma{} proficiency = Dxterity\comma{} dv = 2\comma{} effect =Use your wand as anything from a thin marker to a thick paintbrush\comma{} the {\it paint} is a magical adhesive that sticks to any surface\comma{} and may be of any colour you choose.}
\spell{name = Mental Burden\comma{} incant = onus\comma{} school = Hexes \\& Curses\comma{} type = Cast\comma{} level =Novice\comma{} fp = 5\comma{} attribute =SPR\comma{} proficiency = Willpower\comma{} dv = 8\comma{} effect =If the target fails a resist magic check\comma{} all spells cost 2FP more than their stated value for 6 turns.}
\spell{name = Minor Healing\comma{} incant = enervate\comma{} school = Recuperation\comma{} type = Concentration\comma{} level =Beginner\comma{} fp = 3\comma{} attribute =EMP\comma{} proficiency = Healing\comma{} dv = 3\comma{} effect =Heal for 2 points per turn. 
If the target has a serious wound\comma{} i.e. a broken bone\comma{} cannot heal beyond 50\ead, incant = melofors, school = Transfiguration, type = Instant, level =Novice, fp = 7, attribute =SPR, proficiency = , noProf = 1, dv = 6, effect =Target performs a Resist Magic check\comma{} if the casting check exceeds the Resist check\comma{} the enemy{\apos}s head is encased in a pumpkin. Apply the Blinded effect until it is removed.}
\spell{name = Receive Omen, incant = , noIncant = 1, school = Divination, type = Ritual (3 turns), level =Beginner, fp = 2, attribute =INT, proficiency = , noProf = 1, dv = 3, effect =Use your tea leaves to receive an omen about the future. Ask a question about the outcome of an event. The tea leaves will tell you if the outcome is positive\comma{} negative\comma{} or neutral. Takes 4 minutes to cast.}
\spell{name = Reinforce Shield, incant = praesidium, school = Recuperation, type = Concentration, level =Beginner, fp = 2, attribute =INT, proficiency = Arcane, dv = 4, effect =Restore the strength of a target shield or magical ward by (2+PP) points per turn that this spell is maintained.}
\spell{name = Release Trapped Being, incant = relashio, school = Recuperation, type = Instant, level =Novice, fp = 6, attribute =SPR, proficiency = Willpower, dv = 5, effect =Force objects and beings to release the target from their grip if they fail an ATH(strength) Resist check.}
\spell{name = Runic Shield, incant = scutum, school = Recuperation, type = Instant, level =Novice, fp = 5, attribute =INT, proficiency = Arcane, dv = 6, effect =Choose a Damage Type. Target is 10\% resistant to that damage type (+10\% for each PP) for 1 hour.}
\spell{name = Sense Traps, incant = antidolus, school = Divination, type = Instant, level =Beginner, fp = 4, attribute =INT, proficiency = Understand Other, dv = 5, effect =Attempt to discover any traps in your immediate vicinity. If successful\comma{} you may learn the location of the trap\comma{} and the trigger (but not the effect). Success conditions are set by the GM.}
\spell{name = Shadow Blast, incant = malusangui, school = Dark Arts, type = Instant, level =Beginner, fp = 3, attribute =POW, proficiency = , noProf = 1, dv = 2, effect =Hurl shadows at you enemy\comma{} dealing 1 necrotic damage for every casting point over the difficulty level.}
\spell{name = Shroud of Darkness, incant = tenebrosa, school = Dark Arts, type = Instant, level =Beginner, fp = 4, attribute =EVL, proficiency = , noProf = 1, dv = 4, effect =Extinguish all light within a 10m radius (+2 for every PP to the spell)}
\spell{name = Silence, incant = silencio, school = Charms, type = Instant, level =Novice, fp = 6, attribute =CHR, proficiency = Persuasion, dv = 5, effect =Target performs a resist magic check. If the check fails\comma{} they cannot speak for 2 turns +1 for each power point dedicated.}
\spell{name = Silver Shield, incant = argentipus, school = Transfiguration, type = Instant, level =Beginner, fp = 6, attribute =INT, proficiency = , noProf = 1, dv = 5, effect =Conjures a silver shield from thin air\comma{} to defend you. Shield absorbs both physical and magical attacks for up to 15 damage points\comma{} before breaking.}
\spell{name = Sleep, incant = somnus, school = Illusion, type = Instant, level =Novice, fp = 5, attribute =CHR, proficiency = , noProf = 1, dv = 6, effect =If target fails a SPR(Endurance) resist magic check\comma{} they enter into a deep slumber for (5 + 2 $\times$ PP) turns}
\spell{name = Smokescreen, incant = fumus insterio, school = Charms, type = Instant, level =Novice, fp = 5, attribute =FIN, proficiency = Deception, dv = 3, effect =Thick white smoke issues from the end of your wand\comma{} giving a Severe obscuration for all targets in a 10m radius.}
\spell{name = Speak in Tongues, incant = lingua maxima, school = Divination, type = Instant, level =Beginner, fp = 8, attribute =EMP, proficiency = Understand Other, dv = 4, effect =By meditating for 5 minutes\comma{} you may understand and speak the language of a willing target individual. Effect lasts until concentration is broken.}
\spell{name = Stabilise Patient, incant = firmum, school = Recuperation, type = Instant, level =Novice, fp = 4, attribute =EMP, proficiency = Healing, dv = 7, effect =Stabilises the patient and removes the \textit{Critical Condition} status.}
\spell{name = Steelclaw, incant = ferscabere, school = Transfiguration, type = Instant, level =Beginner, fp = 4, attribute =POW, proficiency = , noProf = 1, dv = 4, effect =Transfigures an animal{\apos}s claws into large steel talons\comma{} increasing their physical damage by +5 . Each power point dedicated gives these talons + 2 damage.  Lasts for 1 day.}
\spell{name = Stick, incant = obharesco, school = Charms, type = Instant, level =Novice, fp = 4, attribute =INT, proficiency = , noProf = 1, dv = 6, effect =Stick two objects together.}
\spell{name = Sting, incant = ictus, school = Hexes \& Curses, type = Instant, level =Beginner, fp = 5, attribute =SPR, proficiency = , noProf = 1, dv = 2, effect =Stings the target for (2 + CV � DV) poison damage.}
\spell{name = Stoneskin, incant = lapis pellium, school = Recuperation, type = Instant, level =Beginner, fp = 4, attribute =SPR, proficiency = Endurance, dv = 4, effect =Increase the target{\apos} AC by 10 for 5 minutes (25 combat rounds). Does not stack.}
\spell{name = Strangle, incant = offoco, school = Hexes \& Curses, type = Instant, level =Novice, fp = 4, attribute =SPR, proficiency = , noProf = 1, dv = 7, effect =Target must resist magic every turn until they succeed\comma{} during this time they are deprived of oxygen\comma{} and eventually succumb to hypoxia under the usual rules. A successful resist check breaks the spell.}
\spell{name = Stunning Blast, incant = stupefy, school = Hexes \& Curses, type = Instant, level =Novice, fp = 8, attribute =POW, proficiency = Willpower, dv = 6, effect =Target performs a Resist Magic check\comma{} if casting check exceeds Resist check\comma{} then target is Stunned for 5 turns.}
\spell{name = Summon Bat Bogeys, incant = vespernasum, school = Hexes \& Curses, type = Instant, level =Novice, fp = 7, attribute =POW, proficiency = , noProf = 1, dv = 6, effect =Causes the mucus in the target{\apos}s nose to gain sentience\comma{} take the form of a (1+PP) small bats\comma{} and attack the target. 
Each bat\minus{}bogey does 1d4 +2 points of acid damage per turn for 3 turns (unless removed).}
\spell{name = Summon Object, incant = accio, school = Charms, type = Concentration, level =Novice, fp = 6, attribute =SPR, proficiency = , noProf = 1, dv = 6, effect =Summon non\minus{}shielded objects within a 500m radius. They will fly to your current position as long as concentration is maintained.}
\spell{name = Summon Snake, incant = serpensortia, school = Transfiguration, type = Instant, level =Novice, fp = 4, attribute =POW, proficiency = , noProf = 1, dv = 6, effect =Summons a venomous snake out of the tip of the caster{\apos}s wand. The snake has 8HP and does 1d6 poison damage upon biting. Every extra power point gives the snake +1 HP and +1 attack.  Lasts for 10 minutes.}
\spell{name = Summon Void, incant = inanis, school = Dark Arts, type = Concentration, level =Novice, fp = 8, attribute =EVL, proficiency = Chaos, dv = 7, effect =Summon a true Void\comma{} a gap in the fabric of reality that attracts all objects within a 5m radius. Everything in radius must perform an ATH(Strength) Resist against the casting check to grab onto something.}
\spell{name = Summon Weak Avatar, incant = elementos, school = Transfiguration, type = Ritual (5 minutes), level =Novice, fp = 7, attribute =INT, proficiency = Arcane, dv = 8, effect =Summon a Weak Avatar of your choice (Storm\comma{} Ice or Fire) to be under your command for 10 turns\comma{} after which it dissolves.}
\spell{name = Sunburst, incant = sol maxima, school = Recuperation, type = Instant, level =Beginner, fp = 4, attribute =SPR, proficiency = , noProf = 1, dv = 4, effect =A burst of bright light does 1d6 holy damage to all targets in a 5m radius.}
\spell{name = Telepathic Bond, incant = conanimus, school = Divination, type = Ritual (2 turns), level =Beginner, fp = 5, attribute =EMP, proficiency = Understand Other, dv = 5, effect =Form a mental connection between your mind and the mind of a willing target. You may then use this connection to communicate silently. 
Target must be within touching distance when the spell is cast\comma{} but the bond has no distance limit after that.
Lasts for 2 days.}
\spell{name = Thick Air, incant = temporio, school = Transfiguration, type = Concentration, level =Novice, fp = 7, attribute =POW, proficiency = , noProf = 1, dv = 6, effect =Transforms the air around the target into a thick soup\comma{} slowing their movement by 20\% (each power point makes the target move slower). Lasts for 1 minute.}
\spell{name = Throw Voice, incant = ventrilofors, school = Illusion, type = Concentration, level =Beginner, fp = 4, attribute =INT, proficiency = Deception, dv = 2, effect =Cast your voice such that it appears to be coming from somewhere up to 5+$\times PP$ metres away.}
\spell{name = Trip, incant = lubricor, school = Hexes \& Curses, type = Instant, level =Beginner, fp = 4, attribute =FIN, proficiency = , noProf = 1, dv = 4, effect =If the target is moving this turn cycle and fails an ATH Resist check\comma{} they go sprawling onto the ground taking 1d4 bludgeoning damage\comma{} and take the `Prone Position� status.}
\spell{name = Unlock, incant = alohomora, school = Charms, type = Instant, level =Novice, fp = 3, attribute =FIN, proficiency = Dexterity, dv = 5, effect =Unlock objects. Mundane locks will fall open for you\comma{} whilst to open magically locked objects\comma{} the unlocking must exceed the locking casting check.}
\spell{name = Vicious Slash, incant = sectumsempra, school = Dark Arts, type = Instant, level =Beginner, fp = 4, attribute =POW, proficiency = , noProf = 1, dv = 4, effect =Gouges at the target\comma{} leaving deep\comma{} cursed wounds\comma{} for 1d6 points of slashing damage\comma{} plus two for every PP.}
\spell{name = Violent Phantasms, incant = umbra impetia, school = Illusion, type = Instant, level =Novice, fp = 7, attribute =SPR, proficiency = , noProf = 1, dv = 6, effect =Multiple phantasms attack the target\comma{} doing (1+PP)d4 psychic damage for every turn that the phantasms are active. 
Once the original spell hits the targets\comma{} phantasms exist only within the target{\apos}s mind\comma{} and may pass through all shields and defences. 
Phantasms are active for (3+PP) turns}
\end{multicols}
\fi
\if \coreMode0
	\begin{multicols}{4} \raggedbottom\subsubsection{Charms}
\vbox{
\textbf{Level 1 Spells}
\begin{itemize}[itemsep=0em]
\renewcommand\labelitemi{-}
\item Create Fire (*) 

\item Create Trap (*) 

\item Create Water (*) 

\item Fresh Air

\item Halt

\item Illuminate Wand

\item Levitation (*) 

\item Mark Surface

\item Piercing Wail

\item Preserve Object


\end{itemize}
} \vfill\subsubsection{Dark Arts}
\vbox{
\textbf{Level 1 Spells}
\begin{itemize}[itemsep=0em]
\renewcommand\labelitemi{-}
\item Blight (*) 

\item Eldritch Knowledge

\item Shadow Blast (*) 

\item Shroud of Darkness

\item Vicious Slash (*) 


\end{itemize}
} \vfill\subsubsection{Divination}
\vbox{
\textbf{Level 1 Spells}
\begin{itemize}[itemsep=0em]
\renewcommand\labelitemi{-}
\item Astral Assistance (*) 

\item Hunter\apos{}s Mark

\item Identify

\item Locate (*) 

\item Receive Omen

\item Sense Traps

\item Speak in Tongues

\item Telepathic Bond


\end{itemize}
} \vfill\subsubsection{Hexes}
\vbox{
\textbf{Level 1 Spells}
\begin{itemize}[itemsep=0em]
\renewcommand\labelitemi{-}
\item Acidic Burst (*) 

\item Cause Confusion

\item Green Sparks (*) 

\item Knockback

\item Sting (*) 

\item Trip


\end{itemize}
} \vfill\subsubsection{Illusion}
\vbox{
\textbf{Level 1 Spells}
\begin{itemize}[itemsep=0em]
\renewcommand\labelitemi{-}
\item Blur (*) 

\item Chaotic Whispers (*) 

\item Charm Entity

\item Glamour (*) 

\item Imbue Bravery

\item Night Vision

\item Throw Voice


\end{itemize}
} \vfill\subsubsection{Recuperation}
\vbox{
\textbf{Level 1 Spells}
\begin{itemize}[itemsep=0em]
\renewcommand\labelitemi{-}
\item Aid Charm (*) 

\item Caterwauling Ward

\item Magical Shield (*) 

\item Minor Healing

\item Privacy Ward

\item Reinforce Shield (*) 

\item Stoneskin

\item Sunburst (*) 


\end{itemize}
} \vfill\subsubsection{Transfiguration}
\vbox{
\textbf{Level 1 Spells}
\begin{itemize}[itemsep=0em]
\renewcommand\labelitemi{-}
\item Alter Hair

\item Basic Transmutation (*) 

\item Change Colour

\item Conjure Flowers

\item Launch Spike (*) 

\item Potion Mixing Spell

\item Silver Shield (*) 

\item Steelclaw


\end{itemize}
} \vfill~\vfill~\end{multicols}\clearpage\begin{multicols}{3}\spell{name = Acidic Burst, school = Hex, type = Instant, level =Beginner, fp = 3, attribute =POW, dv = 4, incant = ambustum, noProf = 1, duration = 2 minutes,higher = An Adept level caster may add 1d6 damage for every 3 character levels over 5th level.,travel = Green gas,noResist =1, effect =Fills a cube of size 4m with an acidic cloud that does (5+PP) acid damage per turn. In a confined space\comma{} the cloud lasts indefinitely.}
\spell{name = Aid Charm, school = Recuperant, type = Instant, level =Beginner, fp = 2, attribute =EMP, dv = 3, incant = subsidium, noProf = 1, duration = 1 hour,higher = At 4th\comma{} 8th\comma{} 12th and 16th levels\comma{} the HP ceiling is raised by 5\comma{} 8\comma{} 10\comma{} and 15 respectively.,travel = Red\minus{}orange rays,noResist =1, effect =Raise the HP ceiling of a target by 3. If target has HP$>0$\comma{} also increase HP by this amount.}
\spell{name = Alter Hair, school = Transfiguration, type = Instant, level =Beginner, fp = 2, attribute =CHR, dv = 4, incant = crinus muto, proficiency = Deception, duration = 2 hours,noHigh = 1, noTravel = 1, noResist =1, effect =Alters the colour and style of the casters hair. Useful for disguises.}
\spell{name = Astral Assistance, school = Divination, type = Ritual (2 turns), level =Beginner, fp = 5, attribute =EMP, dv = 5, incant = auxilio, proficiency = Arcane, noDur = 1, higher = An expert\minus{}level caster may roll 2d4 when performing this spell.,travel = Golden glow,noResist =1, effect =By laying your hand upon a sapient being\comma{} you may channel magical energy into them. On the next check the target performs\comma{} roll 1d4\comma{} and add it to the check (+1 per PP\comma{} max 3). If the check fails\comma{} both the target and the caster take 1d6 psychic damage.}
\spell{name = Basic Transmutation, school = Transfiguration, type = Instant, level =Beginner, fp = 4, attribute =FIN, dv = 4, incant = formum mutatio, noProf = 1, duration = 1 hour,higher = A character above 6th level may add 1 free PP for every 3 character levels above 3rd.,noTravel = 1, noResist =1, effect =Transform a 200g non\minus{}sapient animal or object into a different animal or solid object. 
Each power point doubles the mass of objects that can be transformed.  Objects must be simple in nature.}
\spell{name = Blight, school = Abomination, type = Instant, level =Beginner, fp = 4, attribute =EVL, dv = 5, incant = thanatos, noProf = 1, noDur = 1, higher = An adept level caster may add an extra d4 of damage for every 4 character levels above 2nd.,travel = Sickly\minus{}green shockwave,resist = s, resistDV = s, effect =A cylinder of necrotic energy extends outwards from you in a radius of 10m (doubled with every PP\comma{} max 1km). All simple plants within range die instantly\comma{} and all other living beings take 1d4 necrotic damage (+1 per PP)}
\spell{name = Blur, school = Illusion, type = Instant, level =Beginner, fp = 4, attribute =CHR, dv = 3, incant = celeritate, noProf = 1, duration = 3 turns,higher = When cast by an adept\minus{}level caster\comma{} the first attack directed at the target also automatically misses.,noTravel = 1, noResist =1, effect =The target seems to become blurry around the edges\comma{} it is difficult to tell exactly where they are\comma{} and where they aren{\apos}t.
Gain check advantage on evasion checks for 3 turns.}
\spell{name = Caterwauling Ward, school = Recuperant, type = Ward, level =Beginner, fp = 4, attribute =INT, dv = 3, incant = caterwaul, noProf = 1, duration = 2 weeks,noHigh = 1, noTravel = 1, resist = FIN(Stealth), resistDV = 15, effect =Casts a ward on the area which emits a high\minus{}pitched scream when an unknown being crosses the threshold. 
Radius is (10 + $2\times$PP) metres. Ward decays after 2 weeks.}
\spell{name = Cause Confusion, school = Hex, type = Instant, level =Beginner, fp = 5, attribute =CHR, dv = 3, incant = confundo, proficiency = Deception, duration = 3 turns,noHigh = 1, travel = Pink bolt,resist = SPR (Endurance), resistDV = CC, effect =A target individual acquires the Confused status and takes 5 fatigue damage. On a successful Resist\comma{} no status is applied.}
\spell{name = Change Colour, school = Transfiguration, type = Instant, level =Beginner, fp = 4, attribute =INT, dv = 4, incant = pigmentus, noProf = 1, duration = 2 days,noHigh = 1, travel = Bolt of specified colour,noResist =1, effect =Causes the colour of an object to change into the colour specified by the caster.}
\spell{name = Chaotic Whispers, school = Illusion, type = Concentration, level =Beginner, fp = 4, attribute =SPR, dv = 4, incant = rastarum, proficiency = Deception, duration = 2 minutes,higher = At 7th\comma{} 15th and 18th level\comma{} use a d6\comma{} d10 and d12 respectively for the damage check.,travel = Wand\minus{}tip glows purple,resist = SPR (endurance), resistDV = CC, effect =Whilst the caster maintains concentration\comma{} the target hears a voice in their ear whispering maddening words\comma{} that slowly drive them insane. Target may perform a resit check once per turn\comma{} when one succeeds\comma{} the spell is broken. Whispers do (1+PP)d4 psychic damage per turn that the spell is active.}
\spell{name = Charm Entity, school = Illusion, type = Instant, level =Beginner, fp = 5, attribute =CHR, dv = 4, incant = sismeus amici, proficiency = Persuasion, duration = 1 hour,noHigh = 1, travel = Green rays,noResist =1, effect =If target is not overtly hostile\comma{} this spell causes then to like you: persuasion checks by the caster on the individual get a (2+PP) bonus (max 5).}
\spell{name = Conjure Flowers, school = Transfiguration, type = Instant, level =Beginner, fp = 3, attribute =EMP, dv = 5, incant = orchideous, noProf = 1, duration = 3 days,noHigh = 1, noTravel = 1, noResist =1, effect =Conjures flowers from thin air.}
\spell{name = Create Fire, school = Charm, type = Concentration, level =Beginner, fp = 3, attribute =SPR, dv = 3, incant = incendio, noProf = 1, noDur = 1, higher = An Adept\minus{}level caster may summon a larger gout of flame\comma{} which does an extra 1d6 fire damage for every 4 character levels above 2nd.,noTravel = 1, noResist =1, effect =A small jet of fire is emitted from the tip of your wand. 
Coming into contact with fire does 1d6 fire damage\comma{} and applies a minor Burned status effect.}
\spell{name = Create Trap, school = Charm, type = Ritual (3 turns), level =Beginner, fp = 4, attribute =FIN, dv = 5, incant = dolus, proficiency = Stealth, noDur = 1, higher = A character above 10th level may add free PP to the effect\minus{}spell equal to one\minus{}third their character level.,noTravel = 1, noResist =1, effect =Combine a magical ward with one of your existing spells. After casting the trap spell\comma{} cast the effect\minus{}spell to imbue the trap with that effect. 
If successful\comma{} creates a hidden magical trap of radius 50cm on any solid surface\comma{} with the effect of the original spell when triggered by an entity touching the trap. If you wish to keep a trap hidden from the GM\comma{} write down the location\comma{} spell and associated check values on a piece of paper\comma{} to be revealed when the trap is triggered.}
\spell{name = Create Water, school = Charm, type = Concentration, level =Beginner, fp = 4, attribute =INT, dv = 4, incant = aguamente, noProf = 1, noDur = 1, higher = An adept\minus{}level caster may summon a torrent of water\comma{} which does 1d4 bludgeoning damage for every 3 character levels above 3rd.,noTravel = 1, noResist =1, effect =A jet of water is emitted from the tip of your wand\comma{} useful for extinguishing fires\comma{} or cleaning surfaces\comma{} however conjured water cannot be drunk.}
\spell{name = Eldritch Knowledge, school = Abomination, type = Ritual (3 turns), level =Beginner, fp = 6, attribute =EVL, dv = 3, incant = vetitum scenticus, proficiency = Arcane, noDur = 1, noHigh = 1, travel = Yellow\minus{}black aura,noResist =1, effect =Attune your mind to the Eldritch Domains. The Demons of the Deep will answer one of your questions\comma{} but the answers might drive you mad.
The question must be said out loud for all to hear\comma{} but the answer may be written down and passed to your privately.}
\spell{name = Fresh Air, school = Charm, type = Instant, level =Beginner, fp = 3, attribute =POW, dv = 3, incant = klinneract, noProf = 1, noDur = 1, noHigh = 1, noTravel = 1, noResist =1, effect =A gust of air refreshes the air in a sphere of radius (2 + PP) metres around the caster\comma{} removing any gaseous effects and smelling faintly of lavender.}
\spell{name = Glamour, school = Illusion, type = Instant, level =Beginner, fp = 3, attribute =INT, dv = 3, incant = lux stultium, proficiency = Deception, duration = 1 hour,higher = When cast by a character greater than 8th level\comma{} the DV of the Resist check is equal to the caster level.,noTravel = 1, resist = INT (perception), resistDV = 5, effect =Create a superficial glamour around a person or object\comma{} a simple trick of the light. The glamour disintegrates upon physical or magical contact\comma{} and can be seen to be fake if observer succeeds on a Resist check.}
\spell{name = Green Sparks, school = Hex, type = Instant, level =Beginner, fp = 4, attribute =FIN, dv = 4, incant = verdimillious, proficiency = Arcane, noDur = 1, higher = Every 3 character levels above 2nd level\comma{} add another spark.,travel = Green bolts,noResist =1, effect =Emits 3 green sparks from your wand\comma{} which can be made to strike at the enemy. 
Each spark does 1 force damage \cvdv.}
\spell{name = Halt, school = Charm, type = Instant, level =Beginner, fp = 2, attribute =SPR, dv = 4, incant = stabit, noProf = 1, noDur = 1, noHigh = 1, travel = Pale blue bolt,noResist =1, effect =Stop 1 inanimate object in its tracks\comma{} if mid\minus{}air\comma{} it drops to the ground. If the target is particularly small or fast (i.e. an arrow in mid\minus{}flight) the caster must pass a FIN(precision) check (DV 12) in order to hit the target.}
\spell{name = Hunter\apos{}s Mark, school = Divination, type = Instant, level =Beginner, fp = 3, attribute =INT, dv = 4, incant = venari, noProf = 1, duration = 3 days,noHigh = 1, travel = Semi\minus{}transparent arrow,resist = INT (Perception\comma{} passive), resistDV = CC, effect =If casting check exceeds passive resist value\comma{} caster is aware of the location of the target for the next 3 days\comma{} or until the mark is removed by magical means.}
\spell{name = Identify, school = Divination, type = Instant, level =Beginner, fp = 6, attribute =INT, dv = 3, incant = dicemi, proficiency = Research, noDur = 1, noHigh = 1, travel = Blue rays,noResist =1, effect =Learn the properties of the target: be it learning about the nature of the target\comma{} or the ingredients of a potion.
The more power points dedicated to the spell\comma{} the more information that is revealed.}
\spell{name = Illuminate Wand, school = Charm, type = Concentration, level =Beginner, fp = 1, attribute =INT, dv = 2, incant = lumos, noProf = 1, noDur = 1, noHigh = 1, noTravel = 1, noResist =1, effect =Causes the tip of your wand to glow\comma{} like a torch. Casts bright light for 2m radius\comma{} and dim light for 10m. Spell last indefinitely\comma{} until concentration is broken\comma{} and does not require extra FP per turn.}
\spell{name = Imbue Bravery, school = Illusion, type = Instant, level =Beginner, fp = 2, attribute =SPR, dv = 3, incant = fortudus, proficiency = Persuasion, duration = 1 hour,noHigh = 1, travel = Golden rays,noResist =1, effect =Imbue your target with fortitude and vigour. They gain check\minus{}advantage on all Fear\minus{}Resist checks for 1 hour.}
\spell{name = Knockback, school = Hex, type = Instant, level =Beginner, fp = 3, attribute =POW, dv = 3, incant = flipendo, proficiency = Intimidation, noDur = 1, noHigh = 1, travel = Bue pulse,resist = ATH (Speed), resistDV = , effect =A wave of energy strikes into the target\comma{} causing (1+PP)d4 force damage\comma{} and pushing the target backwards up to (1+PP) metres. Resist for half damage.}
\spell{name = Launch Spike, school = Transfiguration, type = Instant, level =Beginner, fp = 4, attribute =POW, dv = 4, incant = voco dens, noProf = 1, noDur = 1, higher = An expert\minus{}level caster may do 1d12 piercing damage per spike,noTravel = 1, resist = ATH (Speed), resistDV = 10, effect =Conjure (1+PP) enormous spikes to transfigure itself from the surrounding walls/floor\comma{} impaling the target. Each spike does 1d6 piercing damage. Resist for half damage.}
\spell{name = Levitation, school = Charm, type = Concentration, level =Beginner, fp = 5, attribute =FIN, dv = 4, incant = wingardium leviosa, noProf = 1, noDur = 1, higher = A character above 6th level may add 1 free PP for every 3 character levels above 3rd.,noTravel = 1, noResist =1, effect =Cause an object of 500g or less to levitate\comma{} controlling the vertical distance at will. 
Each power point dedicated doubles the mass of the object that can be lifted.}
\spell{name = Locate, school = Divination, type = Instant, level =Beginner, fp = 3, attribute =EMP, dv = 4, incant = locus, proficiency = Research, noDur = 1, higher = An master\minus{}level clairvoyant may perform a SPR(willpower) check to overcome magical shields blocking this spell\apos{}s effect.,noTravel = 1, resist = INT (Stealth), resistDV = CC, effect =Learn the location of non\minus{}magical objects or an unshielded living being. A being may hide from this spell by Resisting.}
\spell{name = Magical Shield, school = Recuperant, type = Concentration, level =Beginner, fp = 5, attribute =POW, dv = 5, incant = protego, noProf = 1, noDur = 1, higher = When cast by a character greater than 10th level\comma{} the AC provided is equal to the character level + $2 \times PP$.,travel = Etheral Shield,noResist =1, effect =Erects an ethereal shield in front of you that absorbs incoming magical attacks.
Shielding charm provides a magical AC by 10+PP against all incoming spells\comma{} but does not protect against physical damage\comma{} or the aftereffects of magic (i.e. a nearby explosion). This AC is eroded by all damage\minus{}causing effects.}
\spell{name = Mark Surface, school = Charm, type = Concentration, level =Beginner, fp = 2, attribute =FIN, dv = 2, incant = stylum, proficiency = Dxterity, noDur = 1, noHigh = 1, noTravel = 1, noResist =1, effect =Use your wand as anything from a thin marker to a thick paintbrush\comma{} the {\it paint} is a magical adhesive that sticks to any surface\comma{} and may be of any colour you choose.}
\spell{name = Minor Healing, school = Recuperant, type = Concentration, level =Beginner, fp = 3, attribute =EMP, dv = 3, incant = enervate, proficiency = Healing, noDur = 1, noHigh = 1, travel = Yellow\minus{}white rays,noResist =1, effect =Heal for 2 points per turn. 
If the target has a serious wound\comma{} i.e. a broken bone\comma{} cannot heal beyond 50\% health. Only works on living creatures.}
\spell{name = Night Vision, school = Illusion, type = Instant, level =Beginner, fp = 3, attribute =EMP, dv = 3, incant = aspectu, proficiency = Perception, duration = 2 hours,noHigh = 1, noTravel = 1, noResist =1, effect =Give the target nightvision for one hour: dim light is as bright as daylight\comma{} and darkness is consdiered dim.}
\spell{name = Piercing Wail, school = Charm, type = Instant, level =Beginner, fp = 3, attribute =SPR, dv = 5, incant = magnus surgerus, proficiency = Performance, noDur = 1, noHigh = 1, noTravel = 1, noResist =1, effect =All targets in a 10m spherical radius of the caster take 2 points of psychic damage (+3 per PP)\comma{} and awaken if they are sleeping.}
\spell{name = Potion Mixing Spell, school = Transfiguration, type = Ritual(5 turns), level =Beginner, fp = 2, attribute =INT, dv = 0, noIncant = 1, proficiency = Arcane, noDur = 1, noHigh = 1, noTravel = 1, noResist =1, effect =Used to mix a potion. See page \pageref{S:Enchanting} for details.}
\spell{name = Preserve Object, school = Charm, type = Instant, level =Beginner, fp = 2, attribute =FIN, dv = 3, incant = preseritas, proficiency = dexterity, duration = 10 days,noHigh = 1, travel = Silver rays,noResist =1, effect =The target is unaffected by the flow of time for the duration of the spell\comma{} and does not rot or otherwise decay.}
\spell{name = Privacy Ward, school = Recuperant, type = Ward, level =Beginner, fp = 6, attribute =SPR, dv = 4, incant = muffliato, noProf = 1, duration = 1 hour,noHigh = 1, noTravel = 1, noResist =1, effect =A buzzing sound fills the ears of anyone trying to listen in on your conversations whilst you are in the warded area. Lasts for one hour\comma{} and has a radius of 2m.}
\spell{name = Receive Omen, school = Divination, type = Ritual (3 turns), level =Beginner, fp = 2, attribute =INT, dv = 3, noIncant = 1, noProf = 1, noDur = 1, noHigh = 1, noTravel = 1, noResist =1, effect =Use your tea leaves to receive an omen about the future. Ask a question about the outcome of an event. The tea leaves will tell you if the outcome is positive\comma{} negative\comma{} or neutral. Takes 4 minutes to cast.}
\spell{name = Reinforce Shield, school = Recuperant, type = Concentration, level =Beginner, fp = 2, attribute =INT, dv = 4, incant = praesidium, proficiency = Arcane, noDur = 1, higher = When cast by an expert\minus{}level caster\comma{} you may restore a shield to 150\% of its original strength.,travel = Brick\minus{}red rays,noResist =1, effect =Restore the strength of a target shield or magical ward by (2+PP) points per turn that this spell is maintained. Cannot restore the strength to more than the original level.}
\spell{name = Sense Traps, school = Divination, type = Instant, level =Beginner, fp = 4, attribute =INT, dv = 5, incant = antidolus, proficiency = Understand Other, noDur = 1, noHigh = 1, noTravel = 1, noResist =1, effect =Attempt to discover any traps in your immediate vicinity. If successful\comma{} you may learn the location of the trap\comma{} and the trigger (but not the effect). Success conditions are set by the GM.}
\spell{name = Shadow Blast, school = Abomination, type = Instant, level =Beginner, fp = 3, attribute =POW, dv = 2, incant = malusangui, noProf = 1, noDur = 1, higher = An novice\minus{}level caster does 2 extra points for each point that the CV exceeds the DV\comma{} and an Expert\minus{}level caster does 4 extra.,travel = Black bolt,noResist =1, effect =Hurl shadows at you enemy\comma{} dealing 1 necrotic damage for every casting point over the difficulty level.}
\spell{name = Shroud of Darkness, school = Abomination, type = Instant, level =Beginner, fp = 4, attribute =EVL, dv = 4, incant = tenebrosa, noProf = 1, duration = 2 minutes,noHigh = 1, noTravel = 1, noResist =1, effect =Extinguish all light within a (10 + 2$\times$PP) metre radius\comma{} and all attempts to create new light fail\comma{} unless caster\apos{} passive POW check exceds the casting check.}
\spell{name = Silver Shield, school = Transfiguration, type = Instant, level =Beginner, fp = 6, attribute =INT, dv = 5, incant = argentipus, noProf = 1, duration = 1 hour,higher = When cast by a character above 10th level\comma{} the shield no longer degrades with each strike\comma{} and instead acts as a normal shield with an AC equal to 15 + 2$\times$PP.,travel = Silver Mist,noResist =1, effect =Conjures a floating silver shield from thin air\comma{} to defend you. Shield absorbs both physical and magical attacks for up to (15+2$\times$PP) damage points\comma{} before breaking. The caster has limited control over the shield whilst it is active\comma{} using a major action to move it up to 3m in any direction.}
\spell{name = Speak in Tongues, school = Divination, type = Ritual (5 minutes), level =Beginner, fp = 8, attribute =EMP, dv = 4, incant = lingua maxima, proficiency = Understand Other, duration = 4 minutes,noHigh = 1, noTravel = 1, noResist =1, effect =By meditating for 5 minutes\comma{} you may understand and speak the language of a willing target individual. Target must be a sapient being\comma{} or otherwise able to speak at least one language.}
\spell{name = Steelclaw, school = Transfiguration, type = Instant, level =Beginner, fp = 4, attribute =POW, dv = 4, incant = ferscabere, noProf = 1, duration = 1 day,noHigh = 1, noTravel = 1, noResist =1, effect =Transfigures an animal{\apos}s claws into large steel talons\comma{} increasing their physical damage by (5 + 2$\times$PP)}
\spell{name = Sting, school = Hex, type = Instant, level =Beginner, fp = 5, attribute =SPR, dv = 2, incant = ictus, proficiency = Precision, noDur = 1, higher = An expert level caster may add 2 damage \cvdv\comma{} rather than the usual 1.,travel = Green dart,noResist =1, effect =Stings the target for 2 poison damage\comma{} plus one \cvdv.}
\spell{name = Stoneskin, school = Recuperant, type = Instant, level =Beginner, fp = 4, attribute =SPR, dv = 4, incant = lapis pellium, proficiency = Endurance, duration = 5 minutes,noHigh = 1, travel = Dark green rays,noResist =1, effect =Increase the target{\apos} AC by 10+PP.}
\spell{name = Sunburst, school = Recuperant, type = Instant, level =Beginner, fp = 4, attribute =SPR, dv = 4, incant = sol maxima, noProf = 1, noDur = 1, higher = When cast by a character greater than 6th level\comma{} do 1d6 extra damage for every 3 levels above 3rd.,travel = Searing\minus{}white bolt,noResist =1, effect =A bolt of magic explodes on contact with a solid {\it or} astral object\comma{} releasing a searing white light that does 2d6 Holy Damage.}
\spell{name = Telepathic Bond, school = Divination, type = Ritual (2 turns), level =Beginner, fp = 5, attribute =EMP, dv = 5, incant = conanimus, proficiency = Understand Other, duration = 2 days,noHigh = 1, noTravel = 1, noResist =1, effect =Form a mental connection between your mind and the mind of a willing target. You may then use this connection to communicate silently. Target must be within touching distance when the spell is cast\comma{} but the bond has no distance limit after that.}
\spell{name = Throw Voice, school = Illusion, type = Concentration, level =Beginner, fp = 4, attribute =INT, dv = 2, incant = ventrilofors, proficiency = Deception, noDur = 1, noHigh = 1, noTravel = 1, noResist =1, effect =Cast your voice such that it appears to be coming from somewhere up to 5+$\times PP$ metres away.}
\spell{name = Trip, school = Hex, type = Instant, level =Beginner, fp = 4, attribute =POW, dv = 4, incant = lubricor, proficiency = Stealth, noDur = 1, noHigh = 1, noTravel = 1, resist = ATH, resistDV = CC, effect =If the target is moving this turn cycle and fails to Resist\comma{} they go sprawling onto the ground taking 1d4 bludgeoning damage\comma{} and take the `Prone Position’ status.}
\spell{name = Vicious Slash, school = Abomination, type = Instant, level =Beginner, fp = 4, attribute =POW, dv = 4, incant = sectumsempra, noProf = 1, noDur = 1, higher = At 8th level\comma{} do 2d6 damage. At 14th\comma{} use 4d6\comma{} and at 20th level\comma{} use 10d6.,travel = Red slash,noResist =1, effect =Gouges at the target\comma{} leaving deep\comma{} cursed wounds\comma{} for 1d6 points of slashing damage.}
\end{multicols}
\fi
