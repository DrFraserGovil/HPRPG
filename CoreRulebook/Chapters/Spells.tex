\documentclass[../CoreRulebook.tex]{subfile}


\chapter{Spells}


\section{Magical Schools}

The study of magic is a far-reaching field, which encompasses many different areas and skills -- some of which require vastly different skillsets to use. For this reason, a magical taxonomy was introduced by the Wizangemot in 1755, which divides the study of magic up into 7 `Schools', each of which contains a number of `Disciplines'. 


{
\footnotesize
\begin{center}
	\begin{rndtable}{c m{\xS cm} p{\wS cm}}
	\bf School	&	\bf Discipline	&	\bf Description
	\\
	\school{Charms}{Elemental}{Elemental magic studies the manipulation and invocation of very primal forces -- heat, light, energy, matter, and the classical elements.}{Kinesis}{Kinesis is a discipline which relies on moving and manipulating physical objects, and often forms the basis of `everyday' magic.}
	\\
	\school{Divination}{Telepathy}{Telepathic magic is the study of the human mind, and that which extends the senses.}{Temporal}{One of the most mysterious disciplines, temporal magic allows one to see beyond concerns such as time and space, and observe (and perhaps manipulate) the universe at an extraplanar level}
	\\
	\school{Illusion}{Bewitchment}{This discipline focusses on the gentle persuasion of the mind and the manipulation of images to convince the target of something which is not true.}{Psionics}{A darker side of illusion magics, psionics is the art of imposing your will over that of your target -- an act which usually shatters their sanity.}
	\\
	\school{Malediction}{Hexes}{Hexes are a field which focusses on magic that directly harms the targeted person or object.}{Curses}{Unlike hexes, curses do not directly harm the target but instead incapacitates them, inhibits their capabilities, or otherwise reduces the threat they pose.}
   \\ 
   \school{Recuperation}{Healing}{Healing is, unsurprisingly, the study of magic used to heal the sick and wounded.}{Warding}{Warding magic is almost entirely defensive in nature, allowing the caster to protect themselves and others from harm.}
	\\
	\school{Transfiguration}{Alteration}{The alteration discipline studies the ability to change things from one form into another.}{Conjuration}{Conjuration magic is concerned with the ability to summon new objects and beings out of thin air.}
	\\
	\school{Dark Arts}{Necromancy}{A taboo discipline that attempts to bend the very forces of life and death to the will of the caster}{Occultism}{Occultism is a rarely studied discipline that accesses and manipulates otherworldly energies originating from the Eldritch domain -- powerful, yet highly unpredictable.}
	\end{rndtable}
\end{center}
}
\normalsize

Every spell is assigned to be a member of one of these disciplines, which determines the skills that are necessary to cast it. 

\newpage
\section{Spell Types} 

In addition to falling into one of the seven Schools (a taxonomy based on the spell effect), every spell can also be categorised as a {\it type}, which is based on how the spell is cast. These categories are {\it instant}, {\it focus}, {\it ritual} and {\it ward} spells. 

\subsection{Instant}

An instant spell is cast as a single major action, and is `cast and forget': as soon as you complete the requisite casting check, the spell is `launched' (usually in the form of a magical bolt of light) towards the target. These bolts travel at speeds of 100m per cycle, corresponding to about 40mph. This means that, unless the target is at ane extreme range, the effect is applied between the major action phase and the beginning of the next turn.

\subsection{Focus}

A focus spell is cast like an Instant spell, but may then be continued indefinitely, repeating the initial effects once per turn as long as you keep the spell active. No further checks are needed to continue the spell, but you must remain focussed and unless stated otherwise, the FP cost is deducted once per turn as well.

Because you must remain focussed, no further spells can be cast for the duration of this spell, and all subsequent movement checks must be `considered', and no other actions (such as evasion) may be taken.   

If you take damage whilst casting a focus spell, you must pass a d20 SPR(willpower) check (DV 10) in order to remain casting. In addition, all attacks on you are considered {\it Attacks of Opportunity} (see page \pageref{S:Sneak}). You may, however, end the spell effect at any time without it counting as an action. 


\subsection{Ward}
A ward is (usually) a Recuperation spell that affects a large area. A ward may be centred on a fixed point or object, or may be centred on a moving location or even a sentient being. 

Wards, however, have an unfortunate habit of interfering with each other when used in unison. If two wards have a significant overlapping region of effect and the caster does not have the {\it Multiward} skill or an equivalent feat, there is a significant chance (determined by the GM) that both wards will collapse. 

The interference only applies if the wards are similar in magnitude and intent. For example, Hogwarts castle is a heavily warded region, but a small ward could be placed in a room without problem. Interference would only become a problem when a new castle-wide ward was attempted. 

Equally, intereference only applies if the effects of the ward compound each other -- if they lie in opposition, then the usual spell mechanics are applied. For example, a character with a personal shield ward touches a beartrap ward -- neither ward collapses, but the beartrap ward is triggered, and the shield will attempt to protect the character. 

\subsection{Ritual}

A Ritual spell is a spell that requires a large amount of preparation -- be it meditation, drawing a summoning circle upon the ground, or performing a special dance. Each Ritual spell has a deisgnated time that the ritual takes to complete, to cast a ritual spell you must spend this length of time preparing for the spell, and after the requisite time has passed, {\it then} you perform the check, and the spell effect is activated. If you fail the check, or choose to stop the ritual, i.e. to take another action, you must restart the ritual spell from the beginning. 

As with a focus spell, concentration is key to completing a ritual. If you are interrupted during the preparation phase, it is considered an {\it Attack of Opportunity} and you must pass a DV10 SPR(willpower) check in order to continue.

~


\section{Casting Spells}


All spells are cast by performing a `check' -- rolling a dice, and then adding on the associated skill modifiers and bonuses that apply for that spell, and comparing it to the Difficulty Value (DV) for the spell. If the Casting Check (CC) is greater than or equal to the DV and you have enough FP, then the spell is considered to be cast, and the effects are applied.  

When performing the check, you use a die of a size commensurate to your ability in that school of magic. As you become a more proficient magic-user, you get access to bigger dice, which enables you to cast more powerful spells, and increases the success rate and power of lower-level spells. 

\begin{center}
	\begin{rndtable}{|c c c|}

	\bf Level & \bf  Name & \bf Die
	\\ 
	1 & Beginner & 1d6
	\\
	2 & Novice & 1d8
	\\
	3 & Adept & 1d10 (with 0 = 10)
	\\
	4 & Expert & 1d12
	\\
	5 & Master & 1d20
	\\ \hline
	\end{rndtable}
\end{center}

The size of dice you are allowed to use is determined on a school-by-school basis via the relevant skills discussed on page \pageref{S:Skills}.

\subsection{Spellbooks and Memory} \label{S:Memory}

There are two ways to cast a spell -- either by reading it from the pages of a book, or by being familiar enough with the spell that you can cast it from memory. 

For each of the 7 schools of magic, there are 5 textbooks. Each of these 35 textbooks is associated with a spell-level and a school, and contains all the spells in that school for that level. For example, the book {\it Dark Forces: A Guide to Self Protection} is a level 4 Hexes \& Curses book, and so contains all level level 4 Hexes \& Curses, but {\bf not} the 3rd level spells, for example. 

To cast a spell from a book, you must be holding a book which contains the specified spell in one hand, and your wand in another. You must then perform the checks, and the spell will be cast. Casting like this takes twice as long as normal, often has a higher casting check and fortitude cost associated with it, and you are open to {\it Attacks of Opportunity} when doing this in combat. Swapping books takes a minor action. 

If, however, you become familiar with a spell, then it is no longer necessary to have the book in your possession -- you can cast from memory. Spells cast from memory are almost always superior, and may be used as quickcast actions etc. Memory-casting is considered the `normal' way to cast, and all spellcasting rules discussed are assumed to apply to memory-casting. 

A spell is considered memorised when it has been cast successfully a number of times from a book in a `real life' scenario (i.e. you have to actually use the spell for its intended purpose, not just cast it wildly into thin air). The number of successful book-casts, $N$, is calculated from:
$$N = 10 - \left( \text{INT modifier + Arcane Proficiency} \right)$$


\subsection{Casting Checks}

A check has two ingedients: the check type, so that you may know which bonuses to apply, and the check-difficulty, so that you may know the target value. 

The check-type is determined by the Discipline that the spell originates from, according to the following prescription:
\def\xS{2}
\def\wS{2}
\begin{center}
	\begin{rndtable}{c m{\xS cm} p{\wS cm}}
	\bf School	&	\bf Discipline	&	\bf Attribute
	\\
	\school{Charms}{Elemental}{\ElCheck}{Kinesis}{\KinCheck}
	\\
	\school{Divination}{Telepathy}{\TelCheck}{Temporal}{\TemCheck}
	\\
	\school{Illusion}{Bewitchment}{\BewCheck}{Psionics}{\PsiCheck}
	\\
	\school{Malediction}{Hexes}{\HexCheck}{Curses}{\CurCheck}
   \\ 
   \school{Recuperation}{Healing}{\HeaCheck}{Warding}{\WarCheck}
	\\
	\school{Transfiguration}{Alteration}{\AltCheck}{Conjuration}{\ConCheck}
	\\
	\school{Dark Arts}{Necromancy}{\NecCheck}{Occultism}{\OccCheck}
	\end{rndtable}
\end{center}

You may, therefore, apply your INT modifier when casting a spell belonging to the Elemental Discipline. You may also ask your GM if it is appropriate to add on a Proficiency modifier to the check, i.e. if you are using a spell from the Bewitchment discipline, the Persuasion or Deception proficiencies may be appropiate, depending on the spell and the specific circumstances. 

The target roll of a check (the DV) is the minimum value of the casting check (CC) which is required in order for the spell effect to be successfully initiated. It is determined by the level and type of the spell, as given by the following table:

\begin{center}
	\begin{rndtable}{c c c c c}
		~	&	Instant	&	Focus	&	Ward	&	Ritual
		\\
		\cellcolor{\tablecolorhead} Beginner & 	3	&	2	&	4	&	3
		\\
		\cellcolor{\tablecolorhead} Novice	& 	5	&	3	&	6	&	5
		\\
		\cellcolor{\tablecolorhead} Adept	&	6	&	4	&	8	&	8	
		\\
		\cellcolor{\tablecolorhead} Expert	&	8	&	6	&	10	&	10
		\\
		\cellcolor{\tablecolorhead} Master	&	10	&	8	&	12	&	12
		
	\end{rndtable}
\end{center}

\subsection{Fortitude}


Casting spells is not as simple as waving your wands and saying the magic words -- it takes great mental clarity to cast, and you can become exhausted from casting difficult spells. This mental burden is enumerated through the Fortitude Points attribute. 

Each spell has an associated FP cost, which is deducted only after it is successfully cast. If the casting fails, then only half of the fortitude cost is deducted (rounded up).
 
You cannot cast a spell if it would send you into negative FP -- you must wait for your head to clear before attempting that spell.  

The fortitude cost of a given spell is determined by the spell type (Instant, Focus, Ritual etc.) and the difficulty of the spell, and if the spell is cast from memory or not. A book-cast spell has a 50\% higher FP cost than if the caster is familiar with the spell. 

The table below gives the memory-cast FP cost first, with the book-cast cost in brackets.


\small
\begin{center}
	\begin{rndtable}{p {1.2cm} |c m{\wFP cm} | c m{\wFP cm} |c m{\wFP cm}| c m{\wFP cm}}
		~ & \multicolumn{2}{c}{\bf Instant} & \multicolumn{2}{c}{\bf Focus} & \multicolumn{2}{c}{\bf Ward} & \multicolumn{2}{c}{\bf Ritual}
		\\
		\cellcolor{\tablecolorhead} Beginner & 	\FPEntry{\FPBegI}  & 	\FPEntry{\FPBegC}	& 	\FPEntry{\FPBegW} 	& 	\FPEntry{\FPBegR}
		\\
		\cellcolor{\tablecolorhead} Novice	& 	\FPEntry{\FPNovI}  & 	\FPEntry{\FPNovC}	& 	\FPEntry{\FPNovW} 	& 	\FPEntry{\FPNovR}
		\\
		\cellcolor{\tablecolorhead} Adept	&	\FPEntry{\FPAdpI}  & 	\FPEntry{\FPAdpC}	& 	\FPEntry{\FPAdpW} 	& 	\FPEntry{\FPAdpR}
		\\
		\cellcolor{\tablecolorhead} Expert	&	\FPEntry{\FPExpI}  & 	\FPEntry{\FPExpC}	& 	\FPEntry{\FPExpW} 	& 	\FPEntry{\FPExpR}
		\\
		\cellcolor{\tablecolorhead} Master	&	\FPEntry{\FPMasI}  & 	\FPEntry{\FPMasC}	& 	\FPEntry{\FPMasW} 	& 	\FPEntry{\FPMasR}
	\end{rndtable}
\end{center}
\normalsize
Some rare spells are noted as having a different FP cost than this stated value. If so, that value should be used instead. 


\subsection{Power Points}

Some spells have the option to dedicate {\it Power Points} (PP) to their casting when cast from memory. Adding Power Points to a spell amplifies that spells effects, it might make it do more damage, last longer or have a wider area of effect. 

You must declare the number of power points you are dedicating to a spell before performing the check. Each power point dedicated increases the DV of the casting by one, and the FP cost by two. The maximum number of PP that can be dedicated is equal to one third of your POW attribute. 

Some spells specify that, when cast by a character of a certain level, they acquire `free' power points. These power points do not increase the DV or FP cost of the spell, and only apply when calculating the effect. You may still add your own PP to these spells following the normal rules. 


 

\newpage
\section{Spell Shapes}

Some spells produce bolts of energy which fly towards a target, whilst others project their energy into a given region, which are often classified via geometrical shapes: a {\it line}, a {\it cube}, a {\it sphere}, a {\it circle} a {\it cone} or a {\it cylinder}. These shapes may either originate around the caster, or from a point designated by the spell. 

\subsection{Circle}

A circular spell extends outwards from the point of origin in a 2D circular shockwave that lies parallel to the ground. The height of the shockwave above the ground is set by the point of origin, which is not included in the shockwave region (unless the caster chooses it to be). Because of its 2D nature, a circular spell can be avoided by ducking beneath it, or jumping over it -- it is only if the shockwave impacts you that the spell effect is applied. 

\subsection{Cone}

The point of origin of a cone is typically the caster's wand, and a cone extends outwards from the wand, in the direction that the wand is pointing. A cone extends forwards to the specified distance, and has a circular cross section, the radius of which is equal to the distance away from the point of origin (so it is a 45$^\circ$ cone).

 The point of origin of the cone is not considered part of the spell area. 

\subsection{Cube}

The point of origin for a cubic spell may be selected to be either the centre of the cube, or the centre of one of its 6 sides. The cube's side-length is specified by the spell effect. The cube point of origin is only affected by the spell if you choose the centre-origin.

\subsection{Cylinder}

A cylinder point of origin is specified to be a point on the ground, around which a circular cross section is drawn, and then a cylinder of energy rises up vertically to a specified height. Generally, a cylinder spell adjusts its size to an individual, and if not otherwise specified, the cylinder is 5cm wider than the target individual is wide, and 5cm taller than the target. The point of origin is affected by the spell. 

\subsection{Line}

A line extends in a straight path from the origin (a caster's wand) towards the target for a specified distance. Unless otherwise specified, the beam is considered to have the cross section equivalent to a pencil. The point of origin is not affected by the spell. 

\subsection{Sphere}

A sphere's point of origin lies at the centre, and the spell effect expands equally out in all directions from that point. Generally, the spell effect cannot penetrate into the ground or through solid objects (unless, for example, it is an explosion). The point of origin is affected by the spell. 

\cleartoleftpage
\onecolumn
\chapter{Spell List} \label{S:SpellList}

This section contains a list of all the spells available in the game. First, the spells are presented broken down into the school and level to which they belong. The next section then contains a full description of the spell, including its casting check, casting difficulty, and spell effects. Spells marked with a (*) gain more effects, or increase in power, at higher levels. 
\def\s{0.3}
\def\w{2.6}
\def\instSymb{\faMagic}
\def\concSymb{\faEye}
\def\wardSymb{\faShield}
\def\musicSymb{$\twonotes$}
\def\ritSymb{$\largepentagram$}
\def\beastSymb{\faPaw}
\define@key{spell}{name}{\def\name{#1}}
\define@key{spell}{incant}{\def\incant{#1}}
\define@key{spell}{school}{\def\school{#1}}
\define@key{spell}{discipline}{\def\discipline{#1}}
\define@key{spell}{type}{\def\type{#1}}
\define@key{spell}{level}{\def\level{#1}}
\define@key{spell}{fp}{\def\fp{#1}}
\define@key{spell}{attribute}{\def\att{#1}}
\define@key{spell}{proficiency}{\def\prof{#1}}
\define@key{spell}{dv}{\def\dv{#1}}
\define@key{spell}{effect}{\def\effect{#1}}
\define@key{spell}{duration}{\def\duration{#1}}
\define@key{spell}{noIncant}{\def\incantMode{#1}}
\define@key{spell}{noProf}{\def\profMode{#1}}
\define@key{spell}{noDur}{\def\durMode{#1}}
\define@key{spell}{higher}{\def\higher{#1}}
\define@key{spell}{noHigh}{\def\highMode{#1}}
\define@key{spell}{travel}{\def\travel{#1}}
\define@key{spell}{resist}{\def\resist{#1}}
\define@key{spell}{resistDV}{\def\resistDV{#1}}
\define@key{spell}{noTravel}{\def\travelMode{#1}}
\define@key{spell}{noResist}{\def\resistMode{#1}}
\newcommand{\spell}[1]
{
	\setkeys{spell}{name=None,incant = -, school = None, discipline=None,type = None, level = 0, fp = 0, attribute = None, proficiency = None, dv = 0, effect = None,noIncant = 0,noProf=0,noDur=0,duration=None,higher=None,noHigh=0,travel = None,noTravel = 0, resist = None, resistDV = none, noResist = 0}
	\setkeys{spell}{#1}
	\vbox{
	{\normalsize \color{rulered}\name }
	\footnotesize
	
		
	{\it \level{}-level \school{} (\discipline)} 
	\vspace{1ex}
	
	\begin{tabular}{l  l }
		\if\incantMode0 
			{\bf Incantation:} & {\it \incant} \\
		\fi
	
	
		
		{\bf Spell Type:} & \type \\ 
	
		\if\travelMode0
			{\bf Visual:} & \travel \\
		\fi
	
	\if\durMode0
		{\bf Duration}  & \duration \\
	\fi
	
	\if\resistMode0
		{\bf Resist: } & \resist,  DV \resistDV \\
	\fi
	%~{\bf DV:}~~~~~~~~~\quad~~~~\dv
	
	\end{tabular}
	\vspace{1ex}
	
	\effect
	
	\vspace{1 ex}
	

	\if\highMode0
		{\bf Higher Level Casting: } 
		
		\higher
	\fi
	}
	\vspace{3ex}
	
	\small
}
\newcommand{\cvdv}{for each point that the casting check exceeds the difficulty value}

\small
\setlength{\parskip}{0em}
\if \coreMode1
	\scriptsize\vbox{
\subsection{Charms}

\vbox{
\begin{rndtable}{>{\centering\arraybackslash}m{\w cm} >{\centering\arraybackslash}m{\s cm}>{\centering\arraybackslash}m{\w cm} >{\centering\arraybackslash}m{\s cm}>{\centering\arraybackslash}m{\w cm} >{\centering\arraybackslash}m{\s cm}>{\centering\arraybackslash}m{\w cm} >{\centering\arraybackslash}m{\s cm}>{\centering\arraybackslash}m{\w cm} >{\centering\arraybackslash}m{\s cm}}
\multicolumn{10}{c}{\bf \normalsize Elemental} 
\\
 \multicolumn{2}{c}{\cellcolor{\tablecolorhead} \bf Beginner}&\multicolumn{2}{c}{\cellcolor{\tablecolorhead} \bf Novice}&\multicolumn{2}{c}{\cellcolor{\tablecolorhead} \bf Adept}&\multicolumn{2}{c}{\cellcolor{\tablecolorhead} \bf Expert}&\multicolumn{2}{c}{\cellcolor{\tablecolorhead} \bf Master}
 \\ 
Control Air 1: Coax & \concSymb & Control Air 2: Handle & \concSymb & Charge Region & \wardSymb & Control Air 4: Wield & \concSymb & Control Air 5: Master & \concSymb
 \\ 
Control Earth 1: Coax & \concSymb & Control Earth 2: Handle & \concSymb & Control Air 3: Exert & \concSymb & Control Earth 4: Wield & \concSymb & Control Earth 5: Master & \concSymb
 \\ 
Control Fire 1: Coax & \concSymb & Control Fire 2: Handle & \concSymb & Control Earth 3: Exert & \concSymb & Control Fire 4: Wield & \concSymb & Control Fire 5: Master & \concSymb
 \\ 
Control Water 1: Coax & \concSymb & Control Water 2: Handle & \concSymb & Control Fire 3: Exert & \concSymb & Control Water 4:  Wield & \concSymb & Control Water5: Master & \concSymb
 \\ 
Create Fire & \concSymb & Elemental Weapon & \instSymb & Control Water 3: Exert & \concSymb & ~	 & ~	 & ~	 & ~	
 \\ 
Create Water & \concSymb & Extinguish Flame & \instSymb & Floodlight & \instSymb & ~	 & ~	 & ~	 & ~	
 \\ 
Fresh Air & \instSymb & Hovering Light & \instSymb & Freeze & \concSymb & ~	 & ~	 & ~	 & ~	
 \\ 
Illuminate Wand & \concSymb & Lightning Bolt & \instSymb & Smokescreen & \instSymb & ~	 & ~	 & ~	 & ~	
 \\ 
Spark & \instSymb & ~	 & ~	 & ~	 & ~	 & ~	 & ~	 & ~	 & ~	
\end{rndtable}
\vspace{3ex}
}
\vbox{
\begin{rndtable}{>{\centering\arraybackslash}m{\w cm} >{\centering\arraybackslash}m{\s cm}>{\centering\arraybackslash}m{\w cm} >{\centering\arraybackslash}m{\s cm}>{\centering\arraybackslash}m{\w cm} >{\centering\arraybackslash}m{\s cm}>{\centering\arraybackslash}m{\w cm} >{\centering\arraybackslash}m{\s cm}>{\centering\arraybackslash}m{\w cm} >{\centering\arraybackslash}m{\s cm}}
\multicolumn{10}{c}{\bf \normalsize Kinesis} 
\\
 \multicolumn{2}{c}{\cellcolor{\tablecolorhead} \bf Beginner}&\multicolumn{2}{c}{\cellcolor{\tablecolorhead} \bf Novice}&\multicolumn{2}{c}{\cellcolor{\tablecolorhead} \bf Adept}&\multicolumn{2}{c}{\cellcolor{\tablecolorhead} \bf Expert}&\multicolumn{2}{c}{\cellcolor{\tablecolorhead} \bf Master}
 \\ 
Create Trap & \ritSymb & Clean Surface & \concSymb & Apparate & \instSymb & Cushion Fall & \instSymb & Cancel Gravity & \ritSymb
 \\ 
Halt & \instSymb & Cut Object & \instSymb & Haste & \instSymb & General Counterspell & \ritSymb & Mass Kinesis & \concSymb
 \\ 
Launder Clothes & \instSymb & Fix Object & \concSymb & Leapfrog & \instSymb & Teleport & \instSymb & ~	 & ~	
 \\ 
Levitation & \concSymb & Lock & \instSymb & Shatter & \concSymb & ~	 & ~	 & ~	 & ~	
 \\ 
Mark Surface & \concSymb & Mage Hands & \concSymb & Shatterblast & \instSymb & ~	 & ~	 & ~	 & ~	
 \\ 
~	 & ~	 & Stick & \instSymb & Spider Hands & \instSymb & ~	 & ~	 & ~	 & ~	
 \\ 
~	 & ~	 & Summon Object & \concSymb & ~	 & ~	 & ~	 & ~	 & ~	 & ~	
 \\ 
~	 & ~	 & Unlock & \instSymb & ~	 & ~	 & ~	 & ~	 & ~	 & ~	
\end{rndtable}
\vspace{3ex}
}
}

\vbox{
\subsection{Divination}

\vbox{
\begin{rndtable}{>{\centering\arraybackslash}m{\w cm} >{\centering\arraybackslash}m{\s cm}>{\centering\arraybackslash}m{\w cm} >{\centering\arraybackslash}m{\s cm}>{\centering\arraybackslash}m{\w cm} >{\centering\arraybackslash}m{\s cm}>{\centering\arraybackslash}m{\w cm} >{\centering\arraybackslash}m{\s cm}>{\centering\arraybackslash}m{\w cm} >{\centering\arraybackslash}m{\s cm}}
\multicolumn{10}{c}{\bf \normalsize Telepathy} 
\\
 \multicolumn{2}{c}{\cellcolor{\tablecolorhead} \bf Beginner}&\multicolumn{2}{c}{\cellcolor{\tablecolorhead} \bf Novice}&\multicolumn{2}{c}{\cellcolor{\tablecolorhead} \bf Adept}&\multicolumn{2}{c}{\cellcolor{\tablecolorhead} \bf Expert}&\multicolumn{2}{c}{\cellcolor{\tablecolorhead} \bf Master}
 \\ 
Astral Assistance & \ritSymb & Detect Magic & \instSymb & Commune with Nature & \ritSymb & Ethereal Tag & \instSymb & Invert Connection & \instSymb
 \\ 
Sense Traps & \instSymb & Detect Thoughts & \concSymb & Disrupt Connection & \instSymb & ~	 & ~	 & True Sight & \ritSymb
 \\ 
Speak in Tongues & \ritSymb & Eavesdrop & \concSymb & Occlumency & \ritSymb & ~	 & ~	 & ~	 & ~	
 \\ 
Telepathic Bond & \ritSymb & Obfuscation & \ritSymb & Sense Humans & \instSymb & ~	 & ~	 & ~	 & ~	
 \\ 
Thought Extractor & \concSymb & ~	 & ~	 & ~	 & ~	 & ~	 & ~	 & ~	 & ~	
\end{rndtable}
\vspace{3ex}
}
\vbox{
\begin{rndtable}{>{\centering\arraybackslash}m{\w cm} >{\centering\arraybackslash}m{\s cm}>{\centering\arraybackslash}m{\w cm} >{\centering\arraybackslash}m{\s cm}>{\centering\arraybackslash}m{\w cm} >{\centering\arraybackslash}m{\s cm}>{\centering\arraybackslash}m{\w cm} >{\centering\arraybackslash}m{\s cm}>{\centering\arraybackslash}m{\w cm} >{\centering\arraybackslash}m{\s cm}}
\multicolumn{10}{c}{\bf \normalsize Temporal} 
\\
 \multicolumn{2}{c}{\cellcolor{\tablecolorhead} \bf Beginner}&\multicolumn{2}{c}{\cellcolor{\tablecolorhead} \bf Novice}&\multicolumn{2}{c}{\cellcolor{\tablecolorhead} \bf Adept}&\multicolumn{2}{c}{\cellcolor{\tablecolorhead} \bf Expert}&\multicolumn{2}{c}{\cellcolor{\tablecolorhead} \bf Master}
 \\ 
Hunter\apos{}s Mark & \instSymb & All\minus{}seeing Eye & \instSymb & Astral Attack & \instSymb & Astral Projection & \ritSymb & Planemeld & \ritSymb
 \\ 
Identify & \instSymb & Astral Caltrops & \instSymb & Foresight & \instSymb & Commune with the Dead & \ritSymb & Planewalk & \ritSymb
 \\ 
Locate & \instSymb & Crystal Gazing & \ritSymb & Glimpse Future & \instSymb & Contingency & \instSymb & ~	 & ~	
 \\ 
Receive Omen & \ritSymb & ~	 & ~	 & ~	 & ~	 & Mists of Time & \ritSymb & ~	 & ~	
 \\ 
Replay Spell & \instSymb & ~	 & ~	 & ~	 & ~	 & Timeslip & \instSymb & ~	 & ~	
\end{rndtable}
\vspace{3ex}
}
}

\vbox{
\subsection{Illusion}

\vbox{
\begin{rndtable}{>{\centering\arraybackslash}m{\w cm} >{\centering\arraybackslash}m{\s cm}>{\centering\arraybackslash}m{\w cm} >{\centering\arraybackslash}m{\s cm}>{\centering\arraybackslash}m{\w cm} >{\centering\arraybackslash}m{\s cm}>{\centering\arraybackslash}m{\w cm} >{\centering\arraybackslash}m{\s cm}>{\centering\arraybackslash}m{\w cm} >{\centering\arraybackslash}m{\s cm}}
\multicolumn{10}{c}{\bf \normalsize Bewitchment} 
\\
 \multicolumn{2}{c}{\cellcolor{\tablecolorhead} \bf Beginner}&\multicolumn{2}{c}{\cellcolor{\tablecolorhead} \bf Novice}&\multicolumn{2}{c}{\cellcolor{\tablecolorhead} \bf Adept}&\multicolumn{2}{c}{\cellcolor{\tablecolorhead} \bf Expert}&\multicolumn{2}{c}{\cellcolor{\tablecolorhead} \bf Master}
 \\ 
Blur & \instSymb & Calm Being & \instSymb & Entrance Other & \instSymb & Beguiling Totem & \instSymb & Mass Suggestion & \instSymb
 \\ 
Charm Entity & \instSymb & Conceal Inscription & \instSymb & Illusory Construction & \concSymb & ~	 & ~	 & True Illusion & \ritSymb
 \\ 
Glamour & \instSymb & Enchant Animal & \instSymb & Illusory Disguise & \concSymb & ~	 & ~	 & ~	 & ~	
 \\ 
Hypnotic Lights & \instSymb & ~	 & ~	 & Sleep & \instSymb & ~	 & ~	 & ~	 & ~	
 \\ 
Imbue Bravery & \instSymb & ~	 & ~	 & Suggestion & \instSymb & ~	 & ~	 & ~	 & ~	
 \\ 
Night Vision & \instSymb & ~	 & ~	 & ~	 & ~	 & ~	 & ~	 & ~	 & ~	
 \\ 
Throw Voice & \concSymb & ~	 & ~	 & ~	 & ~	 & ~	 & ~	 & ~	 & ~	
\end{rndtable}
\vspace{3ex}
}
\vbox{
\begin{rndtable}{>{\centering\arraybackslash}m{\w cm} >{\centering\arraybackslash}m{\s cm}>{\centering\arraybackslash}m{\w cm} >{\centering\arraybackslash}m{\s cm}>{\centering\arraybackslash}m{\w cm} >{\centering\arraybackslash}m{\s cm}>{\centering\arraybackslash}m{\w cm} >{\centering\arraybackslash}m{\s cm}>{\centering\arraybackslash}m{\w cm} >{\centering\arraybackslash}m{\s cm}}
\multicolumn{10}{c}{\bf \normalsize Psionics} 
\\
 \multicolumn{2}{c}{\cellcolor{\tablecolorhead} \bf Beginner}&\multicolumn{2}{c}{\cellcolor{\tablecolorhead} \bf Novice}&\multicolumn{2}{c}{\cellcolor{\tablecolorhead} \bf Adept}&\multicolumn{2}{c}{\cellcolor{\tablecolorhead} \bf Expert}&\multicolumn{2}{c}{\cellcolor{\tablecolorhead} \bf Master}
 \\ 
Chaotic Whispers & \concSymb & Silence & \instSymb & Drain Fortitude & \concSymb & Delusion & \instSymb & Mass Delusion & \instSymb
 \\ 
Piercing Wail & \instSymb & Violent Phantasms & \instSymb & False Friend & \instSymb & Psychosomatism & \concSymb & Modify Memory & \instSymb
 \\ 
Piper{\apos}s Illusion & \musicSymb & ~	 & ~	 & Fury & \instSymb & Relive Memory & \instSymb & ~	 & ~	
 \\ 
~	 & ~	 & ~	 & ~	 & Shatter Illusions & \instSymb & ~	 & ~	 & ~	 & ~	
 \\ 
~	 & ~	 & ~	 & ~	 & Suppress Intelligence & \instSymb & ~	 & ~	 & ~	 & ~	
\end{rndtable}
\vspace{3ex}
}
}

\vbox{
\subsection{Maledictions}

\vbox{
\begin{rndtable}{>{\centering\arraybackslash}m{\w cm} >{\centering\arraybackslash}m{\s cm}>{\centering\arraybackslash}m{\w cm} >{\centering\arraybackslash}m{\s cm}>{\centering\arraybackslash}m{\w cm} >{\centering\arraybackslash}m{\s cm}>{\centering\arraybackslash}m{\w cm} >{\centering\arraybackslash}m{\s cm}>{\centering\arraybackslash}m{\w cm} >{\centering\arraybackslash}m{\s cm}}
\multicolumn{10}{c}{\bf \normalsize Curse} 
\\
 \multicolumn{2}{c}{\cellcolor{\tablecolorhead} \bf Beginner}&\multicolumn{2}{c}{\cellcolor{\tablecolorhead} \bf Novice}&\multicolumn{2}{c}{\cellcolor{\tablecolorhead} \bf Adept}&\multicolumn{2}{c}{\cellcolor{\tablecolorhead} \bf Expert}&\multicolumn{2}{c}{\cellcolor{\tablecolorhead} \bf Master}
 \\ 
Confound & \instSymb & Disarm & \instSymb & Bind Target & \instSymb & Break Focus & \instSymb & Bestow Curse & \ritSymb
 \\ 
Howl & \beastSymb & Hoist Enemy & \concSymb & Cause Confusion & \instSymb & Shield Breaker & \instSymb & ~	 & ~	
 \\ 
Trip & \instSymb & Mental Burden & \instSymb & Delayed Effect & \instSymb & ~	 & ~	 & ~	 & ~	
 \\ 
~	 & ~	 & Prevent Movement & \concSymb & Perpetual Hunger & \instSymb & ~	 & ~	 & ~	 & ~	
 \\ 
~	 & ~	 & Strangle & \instSymb & Scramble Abilities & \concSymb & ~	 & ~	 & ~	 & ~	
 \\ 
~	 & ~	 & Stunning Blast & \instSymb & ~	 & ~	 & ~	 & ~	 & ~	 & ~	
\end{rndtable}
\vspace{3ex}
}
\vbox{
\begin{rndtable}{>{\centering\arraybackslash}m{\w cm} >{\centering\arraybackslash}m{\s cm}>{\centering\arraybackslash}m{\w cm} >{\centering\arraybackslash}m{\s cm}>{\centering\arraybackslash}m{\w cm} >{\centering\arraybackslash}m{\s cm}>{\centering\arraybackslash}m{\w cm} >{\centering\arraybackslash}m{\s cm}>{\centering\arraybackslash}m{\w cm} >{\centering\arraybackslash}m{\s cm}}
\multicolumn{10}{c}{\bf \normalsize Hex} 
\\
 \multicolumn{2}{c}{\cellcolor{\tablecolorhead} \bf Beginner}&\multicolumn{2}{c}{\cellcolor{\tablecolorhead} \bf Novice}&\multicolumn{2}{c}{\cellcolor{\tablecolorhead} \bf Adept}&\multicolumn{2}{c}{\cellcolor{\tablecolorhead} \bf Expert}&\multicolumn{2}{c}{\cellcolor{\tablecolorhead} \bf Master}
 \\ 
Acidic Burst & \instSymb & Cascading Missiles & \instSymb & Acid Stream & \concSymb & Dragon{\apos}s Breath & \concSymb & Crush Bones & \instSymb
 \\ 
Green Sparks & \instSymb & Summon Bat Bogeys & \instSymb & Fireball & \instSymb & Electrical Arc & \concSymb & Disintegrate & \instSymb
 \\ 
Ignite Being & \instSymb & ~	 & ~	 & Heat Object & \instSymb & Glacial Chill & \instSymb & ~	 & ~	
 \\ 
Knockback & \instSymb & ~	 & ~	 & Object Swarm & \concSymb & Magical Detonation & \instSymb & ~	 & ~	
 \\ 
Sting & \instSymb & ~	 & ~	 & Recurring Light & \concSymb & Meteor Strike & \instSymb & ~	 & ~	
 \\ 
~	 & ~	 & ~	 & ~	 & ~	 & ~	 & Shockwave & \instSymb & ~	 & ~	
\end{rndtable}
\vspace{3ex}
}
}

\vbox{
\subsection{Recuperation}

\vbox{
\begin{rndtable}{>{\centering\arraybackslash}m{\w cm} >{\centering\arraybackslash}m{\s cm}>{\centering\arraybackslash}m{\w cm} >{\centering\arraybackslash}m{\s cm}>{\centering\arraybackslash}m{\w cm} >{\centering\arraybackslash}m{\s cm}>{\centering\arraybackslash}m{\w cm} >{\centering\arraybackslash}m{\s cm}>{\centering\arraybackslash}m{\w cm} >{\centering\arraybackslash}m{\s cm}}
\multicolumn{10}{c}{\bf \normalsize Healing} 
\\
 \multicolumn{2}{c}{\cellcolor{\tablecolorhead} \bf Beginner}&\multicolumn{2}{c}{\cellcolor{\tablecolorhead} \bf Novice}&\multicolumn{2}{c}{\cellcolor{\tablecolorhead} \bf Adept}&\multicolumn{2}{c}{\cellcolor{\tablecolorhead} \bf Expert}&\multicolumn{2}{c}{\cellcolor{\tablecolorhead} \bf Master}
 \\ 
Aid Charm & \instSymb & Blessing & \instSymb & Feign Death & \ritSymb & Boost Health & \instSymb & Ultimate Healing & \ritSymb
 \\ 
Minor Healing & \concSymb & Checkup & \instSymb & Mend Bones & \instSymb & Major Healing & \instSymb & ~	 & ~	
 \\ 
Sunburst & \instSymb & Countercurse & \instSymb & Spare the Wounded & \instSymb & Patronus Charm & \concSymb & ~	 & ~	
 \\ 
~	 & ~	 & Endure Environment & \instSymb & ~	 & ~	 & ~	 & ~	 & ~	 & ~	
 \\ 
~	 & ~	 & Heal Being & \instSymb & ~	 & ~	 & ~	 & ~	 & ~	 & ~	
 \\ 
~	 & ~	 & Release Trapped Being & \instSymb & ~	 & ~	 & ~	 & ~	 & ~	 & ~	
 \\ 
~	 & ~	 & Stabilise Patient & \instSymb & ~	 & ~	 & ~	 & ~	 & ~	 & ~	
\end{rndtable}
\vspace{3ex}
}
\vbox{
\begin{rndtable}{>{\centering\arraybackslash}m{\w cm} >{\centering\arraybackslash}m{\s cm}>{\centering\arraybackslash}m{\w cm} >{\centering\arraybackslash}m{\s cm}>{\centering\arraybackslash}m{\w cm} >{\centering\arraybackslash}m{\s cm}>{\centering\arraybackslash}m{\w cm} >{\centering\arraybackslash}m{\s cm}>{\centering\arraybackslash}m{\w cm} >{\centering\arraybackslash}m{\s cm}}
\multicolumn{10}{c}{\bf \normalsize Warding} 
\\
 \multicolumn{2}{c}{\cellcolor{\tablecolorhead} \bf Beginner}&\multicolumn{2}{c}{\cellcolor{\tablecolorhead} \bf Novice}&\multicolumn{2}{c}{\cellcolor{\tablecolorhead} \bf Adept}&\multicolumn{2}{c}{\cellcolor{\tablecolorhead} \bf Expert}&\multicolumn{2}{c}{\cellcolor{\tablecolorhead} \bf Master}
 \\ 
Caterwauling Ward & \wardSymb & Anti\minus{}Muggle Ward & \wardSymb & Anti\minus{}Apparition Ward & \wardSymb & Anti\minus{}Magic Ward & \wardSymb & Fidelius Ward & \ritSymb
 \\ 
Magical Shield & \concSymb & Beartrap Ward & \wardSymb & Bladed Wall & \wardSymb & Holy Ward & \wardSymb & Magical Stability Ward & \wardSymb
 \\ 
Privacy Ward & \wardSymb & Lesser Ward & \wardSymb & Ironwall Ward & \wardSymb & Inversion Zone & \wardSymb & ~	 & ~	
 \\ 
Reinforce Shield & \concSymb & Runic Shield & \instSymb & Minefield Ward & \wardSymb & Major Ward & \wardSymb & ~	 & ~	
 \\ 
~	 & ~	 & Stopping Shield & \concSymb & Mirror Shield & \concSymb & ~	 & ~	 & ~	 & ~	
 \\ 
~	 & ~	 & ~	 & ~	 & Threshold Ward & \wardSymb & ~	 & ~	 & ~	 & ~	
\end{rndtable}
\vspace{3ex}
}
}

\vbox{
\subsection{Transfiguration}

\vbox{
\begin{rndtable}{>{\centering\arraybackslash}m{\w cm} >{\centering\arraybackslash}m{\s cm}>{\centering\arraybackslash}m{\w cm} >{\centering\arraybackslash}m{\s cm}>{\centering\arraybackslash}m{\w cm} >{\centering\arraybackslash}m{\s cm}>{\centering\arraybackslash}m{\w cm} >{\centering\arraybackslash}m{\s cm}>{\centering\arraybackslash}m{\w cm} >{\centering\arraybackslash}m{\s cm}}
\multicolumn{10}{c}{\bf \normalsize Alteration} 
\\
 \multicolumn{2}{c}{\cellcolor{\tablecolorhead} \bf Beginner}&\multicolumn{2}{c}{\cellcolor{\tablecolorhead} \bf Novice}&\multicolumn{2}{c}{\cellcolor{\tablecolorhead} \bf Adept}&\multicolumn{2}{c}{\cellcolor{\tablecolorhead} \bf Expert}&\multicolumn{2}{c}{\cellcolor{\tablecolorhead} \bf Master}
 \\ 
Alter Hair & \instSymb & Alter Aura & \instSymb & Alter Size & \instSymb & Draconic Guardians & \instSymb & Fearsome Guardians & \instSymb
 \\ 
Basic Transmutation & \instSymb & Fabricate Object & \instSymb & Enchantment Ritual & \ritSymb & Fix Transformation & \ritSymb & True Shapeshift & \instSymb
 \\ 
Change Colour & \instSymb & Harden Object & \instSymb & Featherweight & \instSymb & Internal Extension & \instSymb & ~	 & ~	
 \\ 
Potion Mixing Spell & \ritSymb & Stoneskin & \instSymb & Ironmass & \instSymb & ~	 & ~	 & ~	 & ~	
 \\ 
Preserve Object & \instSymb & Thick Air & \concSymb & Sculpt Matter & \concSymb & ~	 & ~	 & ~	 & ~	
 \\ 
Slip & \concSymb & Trecherous Terrain & \instSymb & Undo Transformation & \instSymb & ~	 & ~	 & ~	 & ~	
 \\ 
Steelclaw & \instSymb & ~	 & ~	 & ~	 & ~	 & ~	 & ~	 & ~	 & ~	
\end{rndtable}
\vspace{3ex}
}
\vbox{
\begin{rndtable}{>{\centering\arraybackslash}m{\w cm} >{\centering\arraybackslash}m{\s cm}>{\centering\arraybackslash}m{\w cm} >{\centering\arraybackslash}m{\s cm}>{\centering\arraybackslash}m{\w cm} >{\centering\arraybackslash}m{\s cm}>{\centering\arraybackslash}m{\w cm} >{\centering\arraybackslash}m{\s cm}>{\centering\arraybackslash}m{\w cm} >{\centering\arraybackslash}m{\s cm}}
\multicolumn{10}{c}{\bf \normalsize Conjuration} 
\\
 \multicolumn{2}{c}{\cellcolor{\tablecolorhead} \bf Beginner}&\multicolumn{2}{c}{\cellcolor{\tablecolorhead} \bf Novice}&\multicolumn{2}{c}{\cellcolor{\tablecolorhead} \bf Adept}&\multicolumn{2}{c}{\cellcolor{\tablecolorhead} \bf Expert}&\multicolumn{2}{c}{\cellcolor{\tablecolorhead} \bf Master}
 \\ 
Conjure Flowers & \instSymb & Conjure Bubble & \instSymb & Binding Ropes & \instSymb & Banish & \instSymb & Bind Being & \ritSymb
 \\ 
Launch Spike & \instSymb & Eternal Flame & \instSymb & Conjure Object & \instSymb & Duplicate Object & \instSymb & ~	 & ~	
 \\ 
Shimmering Confetti & \instSymb & Summon Snake & \instSymb & Create Golem & \ritSymb & Summon Avatar & \ritSymb & ~	 & ~	
 \\ 
Silver Shield & \instSymb & ~	 & ~	 & Summon Birds & \concSymb & Summon Daggers & \instSymb & ~	 & ~	
 \\ 
~	 & ~	 & ~	 & ~	 & Vanish Object & \instSymb & ~	 & ~	 & ~	 & ~	
\end{rndtable}
\vspace{3ex}
}
}

\vbox{
\subsection{Dark Arts}

\vbox{
\begin{rndtable}{>{\centering\arraybackslash}m{\w cm} >{\centering\arraybackslash}m{\s cm}>{\centering\arraybackslash}m{\w cm} >{\centering\arraybackslash}m{\s cm}>{\centering\arraybackslash}m{\w cm} >{\centering\arraybackslash}m{\s cm}>{\centering\arraybackslash}m{\w cm} >{\centering\arraybackslash}m{\s cm}>{\centering\arraybackslash}m{\w cm} >{\centering\arraybackslash}m{\s cm}}
\multicolumn{10}{c}{\bf \normalsize Necromancy} 
\\
 \multicolumn{2}{c}{\cellcolor{\tablecolorhead} \bf Beginner}&\multicolumn{2}{c}{\cellcolor{\tablecolorhead} \bf Novice}&\multicolumn{2}{c}{\cellcolor{\tablecolorhead} \bf Adept}&\multicolumn{2}{c}{\cellcolor{\tablecolorhead} \bf Expert}&\multicolumn{2}{c}{\cellcolor{\tablecolorhead} \bf Master}
 \\ 
Blight & \instSymb & Crippling Fatigue & \instSymb & Blood Moon & \ritSymb & Blood Barrier & \wardSymb & Create Horcrux & \ritSymb
 \\ 
Instil Terror & \instSymb & Dark Healing & \instSymb & Contagion & \instSymb & Create Thrall & \concSymb & Kill Target & \instSymb
 \\ 
Shadow Blast & \instSymb & Necrosis & \instSymb & Fiendfyre & \instSymb & Create Zombie & \ritSymb & Soul Snare & \instSymb
 \\ 
Vicious Slash & \instSymb & ~	 & ~	 & Plague of Insects & \instSymb & ~	 & ~	 & ~	 & ~	
 \\ 
~	 & ~	 & ~	 & ~	 & Torture & \concSymb & ~	 & ~	 & ~	 & ~	
\end{rndtable}
\vspace{3ex}
}
\vbox{
\begin{rndtable}{>{\centering\arraybackslash}m{\w cm} >{\centering\arraybackslash}m{\s cm}>{\centering\arraybackslash}m{\w cm} >{\centering\arraybackslash}m{\s cm}>{\centering\arraybackslash}m{\w cm} >{\centering\arraybackslash}m{\s cm}>{\centering\arraybackslash}m{\w cm} >{\centering\arraybackslash}m{\s cm}>{\centering\arraybackslash}m{\w cm} >{\centering\arraybackslash}m{\s cm}}
\multicolumn{10}{c}{\bf \normalsize Occultism} 
\\
 \multicolumn{2}{c}{\cellcolor{\tablecolorhead} \bf Beginner}&\multicolumn{2}{c}{\cellcolor{\tablecolorhead} \bf Novice}&\multicolumn{2}{c}{\cellcolor{\tablecolorhead} \bf Adept}&\multicolumn{2}{c}{\cellcolor{\tablecolorhead} \bf Expert}&\multicolumn{2}{c}{\cellcolor{\tablecolorhead} \bf Master}
 \\ 
Eldritch Knowledge & \ritSymb & Abyssal Fluid & \concSymb & False Moon & \concSymb & Chaos Magic & \instSymb & Universal Tear & \ritSymb
 \\ 
Fury\apos{}s Fire & \beastSymb & Shadowsight & \concSymb & Shadow Demon & \instSymb & Coven\apos{}s Protection & \ritSymb & ~	 & ~	
 \\ 
Shroud of Darkness & \instSymb & Unfathomable Visage & \instSymb & Summon Void & \concSymb & Summoning Circle & \wardSymb & ~	 & ~	
 \\ 
Use Ancient Powers & \ritSymb & ~	 & ~	 & ~	 & ~	 & ~	 & ~	 & ~	 & ~	
\end{rndtable}
\vspace{3ex}
}
}


 \clearpage 
\begin{multicols}{3}
\spell{name = Abyssal Fluid, school = Dark Arts, discipline = Occultism, type = Focus, level =Novice, incant = sucus infernum, duration = 2 turns,higher = An expert\minus{}level caster may expand the jet into a cone.,travel = Black jet,resist = ATH, resistDV = 10, effect =A pencil\minus{}thin jet of inky black fluid emerges from the end of your wand for as long as Focus is maintained\comma{} reaching up to 2m away. All targets touched by the fluid take (2+PP)d4 acid damage for 2 turns. Resist for half damage.}
\spell{name = Acid Stream, school = Maledictions, discipline = Hex, type = Focus, level =Adept, incant = saeclifors, noDur = 1, noHigh = 1, travel = Green jet,resist = ATH (Speed), resistDV = 14, effect =Conjures a pencil\minus{}thin stream of corrosive\comma{} poisonous acid from the tip of your wand up to a distance of 3m. Dissolves objects\comma{} clothes and skin alike\comma{} doing 4 + (2+PP)d6 acid damage. A successful resist dodges for half damage.}
\spell{name = Acidic Burst, school = Maledictions, discipline = Hex, type = Instant, level =Beginner, incant = ambustum, duration = 2 minutes,noHigh = 1, travel = Green gas,noResist =1, effect =Fills a cube of size 4m with an acidic cloud that does (2 + PPd6) acid damage per turn. In a confined space\comma{} the cloud lasts indefinitely.}
\spell{name = Aid Charm, school = Recuperation, discipline = Healing, type = Instant, level =Beginner, incant = subsidium, duration = 1 hour,higher = At 4th\comma{} 8th\comma{} 12th and 16th levels\comma{} the HP ceiling is raised by 5\comma{} 8\comma{} 10\comma{} and 15 respectively.,travel = Red\minus{}orange rays,noResist =1, effect =Raise the HP ceiling of a target by 3. If target has HP$>0$\comma{} also increase HP by this amount.}
\spell{name = All\minus{}seeing Eye, school = Divination, discipline = Temporal, type = Instant, level =Novice, incant = orbis, noDur = 1, noHigh = 1, noTravel = 1, noResist =1, effect =You may create an invisible\comma{} magic eye in front of you\comma{} that hovers. You mentally see everything that the eye sees\comma{} and may use a major action to instruct the eye to move up to 10m in any direction (including vertical). Eye cannot pass through solid walls\comma{} but may squeeze through gaps as small as 4cm in diameter. Eye can detect astral projections.}
\spell{name = Alter Aura, school = Transfiguration, discipline = Alteration, type = Instant, level =Novice, incant = madas, duration = 1 hour,noHigh = 1, noTravel = 1, noResist =1, effect =Change how the object is registers when viewed by magical means (I.e {\it Identify}) – make a mundane object appear magical\comma{} or make a wizard appear as a non\minus{}sapient creature. 

Because this spell truly alters the object\apos{} astral nature\comma{} and so spells such as {\it True Sight} can be fooled by this effect\comma{} though they may be able to detect this spell in addition to the altered aura.}
\spell{name = Alter Hair, school = Transfiguration, discipline = Alteration, type = Instant, level =Beginner, incant = crinus muto, duration = 2 hours,noHigh = 1, noTravel = 1, noResist =1, effect =Alters the colour and style of the casters hair. Useful for disguises.}
\spell{name = Alter Size, school = Transfiguration, discipline = Alteration, type = Instant, level =Adept, incant = engorgio/reducio, duration = 5 minutes,noHigh = 1, travel = White bolt,noResist =1, effect =Multiply or divide the size of a target by (2 + PP)\comma{} target may resist by performing a SPR(endurance) Resist check against the casting check.}
\spell{name = Anti\minus{}Apparition Ward, school = Recuperation, discipline = Warding, type = Ward, level =Adept, incant = nonvidetus, duration = 1 week,noHigh = 1, noTravel = 1, noResist =1, effect =Prevents apparition inside the designated area: no human can apparate in our out for the duration of the ward. The ward covers an area up to 20m in radius.}
\spell{name = Anti\minus{}Magic Ward, school = Recuperation, discipline = Warding, type = Ward, level =Expert, incant = prohibere incatatum, duration = (2+PP) days,noHigh = 1, noTravel = 1, noResist =1, effect =No magic can be cast inside the warded area\comma{} and all magic effects passing over the boundary vanish. Range is a sphere (10 + 2$\times$PP) metres in radius.}
\spell{name = Anti\minus{}Muggle Ward, school = Recuperation, discipline = Warding, type = Ward, level =Novice, incant = repello mugletum, duration = 1 month,noHigh = 1, noTravel = 1, noResist =1, effect =Forms a warded area that muggles can neither see\comma{} nor enter. The warded area is a circle (5 + 5$\times$PP) metres in radius.}
\spell{name = Apparate, school = Charms, discipline = Kinesis, type = Instant, level =Adept, noIncant = 1, noDur = 1, noHigh = 1, noTravel = 1, noResist =1, effect =You may teleport yourself and up to PP additional passengers to a place you are intimitaely familiar with. Passengers must be in physical contact with you the moment this spell is cast. 

This spell may be cast without the use of a wand. 

If anything happens to the caster in the turn that this spell is cast which would disrupt a Focus spell\comma{} all passengers become splinched and take 2d12 force damage.}
\spell{name = Astral Assistance, school = Divination, discipline = Telepathy, type = Ritual (2 turns), level =Beginner, incant = auxilio, noDur = 1, higher = An expert\minus{}level caster may roll 2d4 when performing this spell.,travel = Golden glow,noResist =1, effect =By laying your hand upon a sapient being\comma{} you may channel magical energy into them. On the next check the target performs\comma{} roll 1d4\comma{} and add it to the check (+1 per PP\comma{} max 3). If the check fails\comma{} both the target and the caster take 1d6 psychic damage.}
\spell{name = Astral Attack, school = Divination, discipline = Temporal, type = Instant, level =Adept, incant = devonus, noDur = 1, noHigh = 1, noTravel = 1, noResist =1, effect =By focussing your inner energies\comma{} you are able to summon an ethereal weapon to strike at enemies with a presence on other planes of existence. Do (2+PP)d8 Celestial damage to targets in both the material world\comma{} and the astral realm.}
\spell{name = Astral Caltrops, school = Divination, discipline = Temporal, type = Instant, level =Novice, incant = Caltrops, duration = 1 turns,higher = When cast by an adept\minus{}level caster\comma{} this spell can effect all beings in a 1d4 metre radius.,travel = Purple bolt,resist = SPR (Endurance), resistDV = CC+3$\times$PP, effect =The target acts as if any terrain they touch has caltrops\comma{} for the duration of the spell. Caltrops do (1+PP)d6 psychic damage for every metre moved by the target. Resist for half damage.}
\spell{name = Astral Projection, school = Divination, discipline = Temporal, type = Ritual (2 turns), level =Expert, incant = ambilofors, noDur = 1, noHigh = 1, travel = Invisible ripple,noResist =1, effect =Leave your physical form behind\comma{} and project your spirit into the Astral Realm. 
Your astral self is undetectable to most living beings\comma{} and has (3+PP) HP\comma{} but can only interact with other entities on the Astral Realm. 
If your astral self is killed\comma{} your physical body{\apos}s HP is reduced to zero\comma{} and your enter into the Critical Condition status.}
\spell{name = Banish, school = Transfiguration, discipline = Conjuration, type = Instant, level =Expert, incant = valeo fendus, noDur = 1, noHigh = 1, travel = White bolt,resist = INT (Endurance), resistDV = CC, effect =Target a summoned creature\comma{} if it fails to Resist\comma{} it is banished from this plane of existence.}
\spell{name = Basic Transmutation, school = Transfiguration, discipline = Alteration, type = Instant, level =Beginner, incant = formum mutatio, duration = 1 hour,higher = A character above 6th level may add 1 free PP for every 3 character levels above 3rd.,noTravel = 1, noResist =1, effect =Transform a 200g non\minus{}sapient animal or object into a different animal or solid object. 
Each power point doubles the mass of objects that can be transformed.  Objects must be simple in nature.}
\spell{name = Beartrap Ward, school = Recuperation, discipline = Warding, type = Ward, level =Novice, incant = ursa dentes, duration = 5 days,noHigh = 1, noTravel = 1, resist = INT, resistDV = 12, effect =A ward that creates an invisible trap of 2m in radius. When a being crosses over the threshold\comma{} the ward slams shut\comma{} doing (2+PP)d8 worth of piercing damage and applying the Trapped status effect. A successful resist takes half damage and nullifies the Trapping effect.}
\spell{name = Beguiling Totem, school = Illusion, discipline = Bewitchment, type = Instant, level =Expert, incant = fascinare, duration = (1+PP) days,noHigh = 1, noTravel = 1, resist = EMP (perception), resistDV = CC, effect =Target an object between 1m and 20m in size. Caster decides upon a single species\comma{} and imbues the target with an aura that either attracts or repels (caster\apos{}s choice) that species in a radius of (10$\times$(1+PP)) metres. Members of the species that fail to resist feel an irresistible urge to either approach or flee the object. Effect lasts for (1+PP) days.}
\spell{name = Bestow Curse, school = Maledictions, discipline = Curse, type = Ritual (10 minutes), level =Master, incant = maledicto, noDur = 1, noHigh = 1, noTravel = 1, resist = SPR (Endurance), resistDV = CC, effect =Casts a permanent curse on the target. You may choose the effects of this curse\comma{} though the GM has a veto. Be inventive!}
\spell{name = Bind Being, school = Transfiguration, discipline = Conjuration, type = Ritual (5 turns), level =Master, incant = subjungus, duration = 1 day,noHigh = 1, noTravel = 1, resist = POW, resistDV = 15, effect =By inscribing a magic circle on the floor\comma{} you create a region where celestial beings from other planes can be trapped and bent to your will. For the next hour\comma{} if an Unlife\comma{} or other being originating from any plane other than the Mortal Realm\comma{} enters into the region\comma{} you may cast a pinch of salt into the circle to complete the ritual and attempt to impose your will over it. 

On a failed resist\comma{} the being is bound to serve you for the duration of the spell. If the being was summoned or created by another spell\comma{} that spell is extended to match the duration of this spell. The being will obey your commands to the letter for the duration of the spell\comma{}and if they are hostile to you\comma{} they may do so in a deliberately obtuse fashion.}
\spell{name = Bind Target, school = Maledictions, discipline = Curse, type = Instant, level =Adept, incant = petrificus totalus, noDur = 1, noHigh = 1, noTravel = 1, resist = SPR (Endurance), resistDV = CC, effect =If the target fails to Resist\comma{} they are Paralyzed for (3+PP) turns. The target cannot take major actions\comma{} move\comma{} or communicate verbally until the spell ends.}
\spell{name = Binding Ropes, school = Transfiguration, discipline = Conjuration, type = Instant, level =Adept, incant = incarcerous, duration = 5 minutes,noHigh = 1, noTravel = 1, resist = ATH (Strength), resistDV = 15, effect =Conjures thick ropes from thin air\comma{} to wrap around the target\comma{} immobilising them. Target may Resist once per turn to break free.}
\spell{name = Bladed Wall, school = Recuperation, discipline = Warding, type = Ward, level =Adept, incant = heus nocivious, duration = (3 + PP) minutes,noHigh = 1, noTravel = 1, resist = INT (Perception), resistDV = 10, effect =Create a warded region up to 10 m long and 3m tall. This wall is composed of swirling magical blades that do (3+PP)d8 slashing damage to any creature that touches it (targets may Resist for half damage). Wall has an AC of 10+$3 \times $PP.}
\spell{name = Blessing, school = Recuperation, discipline = Healing, type = Instant, level =Novice, incant = benedicte, duration = 2 minutes,noHigh = 1, travel = Pink flash,noResist =1, effect =The target gets check advantage on all checks for the duration of the blessing. If they already had check advantage due to another effect other than this spell\comma{} take check double\minus{}advantage. If they had check\minus{}(double)disadvantage\comma{} remove it for the duration.}
\spell{name = Blight, school = Dark Arts, discipline = Necromancy, type = Instant, level =Beginner, incant = thanatos, noDur = 1, noHigh = 1, travel = Sickly\minus{}green shockwave,noResist =1, effect =A cylinder of necrotic energy extends outwards from you in a radius of 10m (doubled with every PP\comma{} max 1km). All simple plants within range die instantly\comma{} and all other living beings take 1 + (1+PP)d4 necrotic damage.}
\spell{name = Blood Barrier, school = Dark Arts, discipline = Necromancy, type = Ward, level =Expert, incant = confusangui, noDur = 1, noHigh = 1, noTravel = 1, noResist =1, effect =Use blood to draw warding runes onto an object or person. Erects a swirling red magical barrier with AC 10\comma{} plus 5 for every casting point over the difficulty. Barrier blocks all physical and magical damage and is immune to acid erosion\comma{} but is eroded by celestial damage. 
Each individual{\apos}s blood can only be used once for blood magic.}
\spell{name = Blood Moon, school = Dark Arts, discipline = Necromancy, type = Ritual (10 minutes), level =Adept, noIncant = 1, duration = 2 hours,higher = When cast by a Master\minus{}level cast\comma{} this spell lasts for one week.,travel = Sky turns red,noResist =1, effect =By sacrificing an animal larger than a cat\comma{} you may use the inherent power of its blood to manipulate the power of the Sun and the Moon: the sky becomes overcast and takes on an unhealthy red glow. This blocks out the effects of the sun and the moon on Vampires\comma{} Werewolves\comma{} and other such creatures. Werewolves may still choose to undergo their transformation\comma{} but retain humanoid intelligence when doing so.}
\spell{name = Blur, school = Illusion, discipline = Bewitchment, type = Instant, level =Beginner, incant = celeritate, duration = 3 turns,higher = When cast by an adept\minus{}level caster\comma{} the first attack directed at the target also automatically misses.,noTravel = 1, noResist =1, effect =The target seems to become blurry around the edges\comma{} it is difficult to tell exactly where they are\comma{} and where they aren{\apos}t. May be cast on self. 
Gain check advantage on evasion checks for 3 turns.}
\spell{name = Boost Health, school = Recuperation, discipline = Healing, type = Instant, level =Expert, incant = levo, duration = (3 + PP) turns,noHigh = 1, travel = Yellow\minus{}white rays,noResist =1, effect =Give the target a temporary +150\% boost to their maximum HP\comma{} and adds current HP to match.}
\spell{name = Break Focus, school = Maledictions, discipline = Curse, type = Instant, level =Expert, incant = adtono, duration = 5 turns,noHigh = 1, travel = Disorienting lights,resist = SPR (Endurance), resistDV = 12, effect =Disorienting noises and lights distract prevent the target from continued Focus. Afflicted beings cannot cast Focus spells for the duration of this spell \minus{}\minus{} all attempts to do so count as `failed'. A successful Resist negates this effect\comma{} but target takes check disadvantage the next time something attempts to break their Focus.}
\spell{name = Calm Being, school = Illusion, discipline = Bewitchment, type = Instant, level =Novice, incant = paxus, noDur = 1, noHigh = 1, travel = Golden mist,noResist =1, effect =Calms the target down. Remove terrified status from target.}
\spell{name = Cancel Gravity, school = Charms, discipline = Kinesis, type = Ritual (3 turns), level =Master, incant = reimannius, duration = 5 turns,noHigh = 1, noTravel = 1, noResist =1, effect =Upon the completion of a complex mathematical calculation\comma{} the force of gravity is removed from a spherical region 10m in radius (doubled for every PP added)\comma{} centred on the caster. Objects and beings no longer fall to the ground and instead remain suspended in mid\minus{}air unless otherwise acted upon. Crossing over the boundary restores gravity\comma{} until the region is re\minus{}entered. Suspended beings take a 3\minus{}point penalty to all checks.}
\spell{name = Cascading Missiles, school = Maledictions, discipline = Hex, type = Instant, level =Novice, incant = unda delor, noDur = 1, noHigh = 1, travel = Blue bolts,noResist =1, effect =Produce (3+PP) magical darts that fly towards the targets. Each dart does 1d6 force damage\comma{} and the swarm may be directed to strike multiple targets\comma{} or the same target.}
\spell{name = Caterwauling Ward, school = Recuperation, discipline = Warding, type = Ward, level =Beginner, incant = caterwaul, duration = 2 weeks,noHigh = 1, noTravel = 1, resist = FIN(Stealth), resistDV = 15, effect =Casts a ward on the area which emits a high\minus{}pitched scream when an unknown being crosses the threshold. 
Radius is (10 + $2\times$PP) metres. Ward decays after 2 weeks.}
\spell{name = Cause Confusion, school = Maledictions, discipline = Curse, type = Instant, level =Adept, incant = confundo, duration = 3 turns,higher = An Expert level caster makes the target lose the next 1d4 turns.,travel = Pink bolt,resist = SPR (Endurance), resistDV = CC, effect =If target fails to resist\comma{} they lose their next turn.}
\spell{name = Change Colour, school = Transfiguration, discipline = Alteration, type = Instant, level =Beginner, incant = pigmentus, duration = 2 days,noHigh = 1, travel = Bolt of specified colour,noResist =1, effect =Causes the colour of an object to change into the colour specified by the caster.}
\spell{name = Chaos Magic, school = Dark Arts, discipline = Occultism, type = Instant, level =Expert, incant = chaomal portis, duration = (2+PP) turns,noHigh = 1, noTravel = 1, noResist =1, effect =Open a small portal to Pand{\ae}monium\comma{} the Chaos Realm at your current location. For every turn that the portal remains open\comma{} it casts random Dark Magic at all targets outside a 2m radius of the caster. These spells increase in power as the portal remains open.}
\spell{name = Chaotic Whispers, school = Illusion, discipline = Psionics, type = Focus, level =Beginner, incant = rastarum, duration = 2 minutes,noHigh = 1, travel = Wand\minus{}tip glows purple,resist = SPR (endurance), resistDV = CC, effect =Whilst the caster maintains Focus\comma{} the target hears a voice in their ear whispering maddening words that slowly drive them insane. Target may take a minor action to perform a resist check once per turn\comma{} when one succeeds\comma{} the spell is broken. Whispers do (1+PP)d4 psychic damage per turn that the spell is active.}
\spell{name = Charge Region, school = Charms, discipline = Elemental, type = Ward, level =Adept, incant = rarnus, noDur = 1, noHigh = 1, travel = Electric arc,resist = ATH (Health), resistDV = 10, effect =Imbue a non\minus{}metallic object up to (2+PP)m in size with an enourmous electric charge. The next being to touch the object takes (3+2$\times$)d6 electric damage. Although this spell is classed as a `ward'\comma{} the threat is non\minus{}magical in nature after the spell has been cast. The charge\minus{}buildup therefore does not register to magic\minus{}only investigation.}
\spell{name = Charm Entity, school = Illusion, discipline = Bewitchment, type = Instant, level =Beginner, incant = sismeus amici, duration = 1 hour,noHigh = 1, travel = Green rays,noResist =1, effect =If target is not overtly hostile\comma{} this spell causes then to like you: persuasion checks by the caster on the individual get a (2+PP) bonus (max 5).}
\spell{name = Checkup, school = Recuperation, discipline = Healing, type = Instant, level =Novice, incant = dispungo, noDur = 1, noHigh = 1, noTravel = 1, noResist =1, effect =Enquire as to the health status of the target\comma{} find out their remaining HP\comma{} as well as any status effects they currently posses.}
\spell{name = Clean Surface, school = Charms, discipline = Kinesis, type = Focus, level =Novice, incant = pullundo, noDur = 1, higher = When cast by an expert\minus{}level caster\comma{} the rune\minus{}trigger probability is decreased by 25\% for every 3 character levels over 12th.,noTravel = 1, noResist =1, effect =Wave your wand over a surface to erase magical and mundane markings from it. Cleans 1 square metre per turn that the spell is maintained. 

When erasing magical runes\comma{} there is a chance for the rune to trigger.}
\spell{name = Commune with Nature, school = Divination, discipline = Telepathy, type = Ritual (5 turns), level =Adept, incant = naturus amicus, noDur = 1, noHigh = 1, travel = Green glow,noResist =1, effect =You tap into the consciousness that binds all living things together\comma{} and receive information about the natural order of things in the vicinity. Outdoors\comma{} the range is 3km\comma{} whilst underground it is only 100m. Spell fails in artificial environments such as towns. You instantly learn and 3 three bits of information about \begin{itemize}\item terrain and bodies of water \item nearby buildings \item abundant plants or minerals \item frequent visitors \end{itemize}}
\spell{name = Commune with the Dead, school = Divination, discipline = Temporal, type = Ritual (2 hours), level =Expert, incant = amisit amicum, noDur = 1, noHigh = 1, travel = Grey\minus{}black aura,noResist =1, effect =You may summon a spirit of the dead\comma{} and learn one piece of information from them\comma{} or temporarily borrow one of their skills and/or spells for (1+PP) turns. You must know the target\apos{}s name\comma{} and they must be willing to help you.}
\spell{name = Conceal Inscription, school = Illusion, discipline = Bewitchment, type = Instant, level =Novice, incant = occulto, duration = 1 years,noHigh = 1, noTravel = 1, noResist =1, effect =Makes a message\comma{} drawing or marking on a surface invisible to the naked eye.}
\spell{name = Confound, school = Maledictions, discipline = Curse, type = Instant, level =Beginner, incant = lombus, duration = 2 turns,noHigh = 1, travel = Blue bolt,resist = POW, resistDV = CC, effect =The target suffers a 1\minus{}point penalty to all checks for the duration of the spell.}
\spell{name = Conjure Bubble, school = Transfiguration, discipline = Conjuration, type = Instant, level =Novice, incant = ebublio, duration = 1 hour,noHigh = 1, noTravel = 1, noResist =1, effect =Conjures a large\comma{} hard\minus{}to\minus{}pop\comma{} airtight\comma{} spherical bubble radius specified by the caster (max: 2m). The bubble can use to encase enemies\comma{} or to protect the caster. The bubble has an AC of 5\comma{} and is immune to acid damage.}
\spell{name = Conjure Flowers, school = Transfiguration, discipline = Conjuration, type = Instant, level =Beginner, incant = orchideous, duration = 3 days,noHigh = 1, noTravel = 1, noResist =1, effect =Conjures flowers from thin air.}
\spell{name = Conjure Object, school = Transfiguration, discipline = Conjuration, type = Instant, level =Adept, incant = siestum, duration = 3 minutes,higher = A character above 9th level may add 1 free PP for every 3 character levels above 6th.,noTravel = 1, noResist =1, effect =Conjure a 200g inanimate\comma{} non\minus{}magical object from thin air. Each power point dedicated doubles the mass/complexity of the object that can be conjured}
\spell{name = Contagion, school = Dark Arts, discipline = Necromancy, type = Instant, level =Adept, incant = vastantes, duration = 2 weeks,higher = When cast by an Expert\minus{}level caster\comma{} all positive bonuses etc. are set to \minus{}2 for the duration.,travel = Sickly\minus{}green rays,resist = ATH (Health), resistDV = CC, effect =Target contracts a necrotic disease. All positive modifiers and proficiency bonuses are set to zero until cured. Disease is contagious and each time they touch an unafflicted individual\comma{} being must Resist\comma{} or contract the disease also. The disease is cured on a successful resist\comma{} and afflicted beings may attempt to resist once every 5 cycles.}
\spell{name = Contingency, school = Divination, discipline = Temporal, type = Instant, level =Expert, incant = fortasse, noDur = 1, noHigh = 1, travel = Green flash,noResist =1, effect =You forsee a need for defence in the future\comma{} but you can{\apos}t quite see when. The contingency charm allows you to store a spell in an alternative dimension\comma{} to be called forth instantly when you need it. After casting the contingency charm\comma{} you may then cast the spell that you wish to store. When activated\comma{} you may then use this spell as if you had declared a counterspell\comma{} in addition to your regular movements. You may have a maximum of three contingencies at any one time.}
\spell{name = Control Air 1: Coax, school = Charms, discipline = Elemental, type = Focus, level =Beginner, incant = ventepare, noDur = 1, higher = When cast by a caster greater than 3rd level\comma{} spell effects may be maintained without further FP cost. FP is only deducted when a `new’ effect is initiated\comma{} or a current effect is redirected.,travel = Wandtip glows white,noResist =1, effect =This spell is capable of summoning light breezes\comma{} up to 15mph\comma{} without too much precision. 

Every turn concentration is maintained you may add another effect\comma{} or cancel a previously utilised effect. The maximum number of effects is equal to 1 + FIN modifier (min 1). 

An example of the uses for this spell could be:
\begin{spellitemize}
\item {\it Gust}: Cause a localised light breeze within a 5m radius. This breeze is strong enough to divert the path of a light projectile by (1+PP)$\times$10 cm.
\item {\it Distract}: Summon a breeze to cause a commotion behind an opponent such that\comma{} on a failed DV 5 INT Resist\comma{} they are open to an Attack of Opportunity. 
\end{spellitemize}
This is not an exhaustive list. Be inventive!}
\spell{name = Control Air 2: Handle, school = Charms, discipline = Elemental, type = Focus, level =Novice, incant = ventepare, noDur = 1, higher = When cast by a caster greater than 6th level\comma{} spell effects may be maintained without further FP cost. FP is only deducted when a `new’ effect is initiated\comma{} or a current effect is redirected.,travel = Wandtip glows white,noResist =1, effect =This spell is capable of summoning powerful blasts (100mph +) of wind in a small area\comma{} or precision manipulation at a much lower speed.

Every turn concentration is maintained you may add another effect\comma{} or cancel a previously utilised effect. The maximum number of effects is equal to 1 + FIN modifier (min 1). 

An example of the uses for this spell could be:
\begin{spellitemize}
\item {\it Windtunnel}: direct a powerful blast of air in a line up to 5m long. On a failed Resist\comma{} any being or object in the path is blown back to the edge of the range of the spell\comma{} taking (3+PP)d4 concussive damage. 
\item {\it Airboost}: use air currents to give a target within range a boost of 1m to their base speed and allow them to jump up to 3m in a single bound. Alternatively\comma{} use this against a foe: Slow a being by 1m per turn\comma{} or cause any acrobatics to fail\comma{}  leaving them prone. Negated on a successful Resist.   
\end{spellitemize}
You may also use the effects listed in Control Air 1 with \PPDifference{\DVNovF}{\DVBegF} PP added. 
This is not an exhaustive list. Be inventive!}
\spell{name = Control Air 3: Exert, school = Charms, discipline = Elemental, type = Focus, level =Adept, incant = ventepare, noDur = 1, higher = When cast by a caster greater than 9th level\comma{} spell effects may be maintained without further FP cost. FP is only deducted when a `new’ effect is initiated\comma{} or a current effect is redirected.,travel = Wandtip glows white,noResist =1, effect =With this spell\comma{} the caster may summon very powerful blasts of air over a larger area\comma{} or summon large\minus{}scale winds to alter the weather slightly.

Every turn concentration is maintained you may add another effect\comma{} or cancel a previously utilised effect. The maximum number of effects is equal to 1 + FIN modifier (min 1). 

An example of the uses for this spell could be:

\begin{spellitemize}
\item {\it Updraft}: a powerful blast of air lifts everything within a (3+PP) radius of the caster heavier than (5+PP)$\times$ your caster level (in kg) to be thrown 10m into the air\comma{} doing 5d4 bludgeoning damage. 
\item {\it Downdraft}: a wall of air slams into everything within a cylinder  (3+PP) in radius around the caster and 5m in height. All airborne objects (but not spell\minus{}bolts) slam into the ground and take double the usual falling damage. 

\item {\it Cloudmove}: by maintaining focus for 1 minute\comma{} you may summon a brisk breeze over an area 1km in size. You may use this to move a raincloud out of the way\comma{} or to summon a mild drizzle over the targeted area. This does not work in conditions with a strong wind already present: instead you simply slow that wind down. 

\end{spellitemize}
You may also use the effects listed in Control Air 2 or below with \PPDifference{\DVAdpF}{\DVNovF} additional PP added. 
This is not an exhaustive list. Be inventive!}
\spell{name = Control Air 4: Wield, school = Charms, discipline = Elemental, type = Focus, level =Expert, incant = ventepare, noDur = 1, higher = When cast by a caster greater than 12th level\comma{} spell effects may be maintained without further FP cost. FP is only deducted when a `new’ effect is initiated\comma{} or a current effect is redirected.,travel = Wandtip glows white,noResist =1, effect =The caster may summon powerful vortices\comma{} and the ability to slightly alter the temperature of the air is gained

Every turn concentration is maintained you may add another effect\comma{} or cancel a previously utilised effect. The maximum number of effects is equal to 1 + FIN modifier (min 1). 

An example of the uses for this spell could be:


\begin{spellitemize}
\item {\it Arctic Wind}: cool the air you control to freezing. All beings affected by other effects of this spell (except large\minus{}scale weather manipulation) take (3+PP)d4 cold damage per turn in addition to any other effects. 
\item {\it Vortex Field}: summon a powerfull\comma{} swirling wall of air to act as a shield around you in a (1+PP)m radius. Physical objects and people entering the  field must Resist\comma{} or be flung 10m in a random direction. The field is opaque in both directions\comma{} so spells cast through the field must succeed an accuracy check to hit something on the other side of the field. 
\item {\it Hurricane}: twist the air into a towering column of chaos (1+PP)m in radius and (1+PP)$\times$10m high. Anything within this column takes (2+PP)d8 bludgeoning damage\comma{} and all ranged attacks passing through the require a DV (10+PP) accuracy check to pass through. 
\end{spellitemize}
You may also use the effects listed in Control Air 3 or below with \PPDifference{\DVExpF}{\DVAdpF} additional PP added. 
This is not an exhaustive list. Be inventive!}
\spell{name = Control Air 5: Master, school = Charms, discipline = Elemental, type = Focus, level =Master, incant = ventepare, noDur = 1, higher = When cast by a caster greater than 15th level\comma{} spell effects may be maintained without further FP cost. FP is only deducted when a `new’ effect is initiated\comma{} or a current effect is redirected.,travel = Wandtip glows white,noResist =1, effect =Master the element of air\comma{} and gain the ability to summon cataclsymic weather events\comma{} or use incredible precision to mimic flight.

Every turn concentration is maintained you may add another effect\comma{} or cancel a previously utilised effect. The maximum number of effects is equal to 1 + FIN modifier (min 1). 

An example of the uses for this spell could be:

\begin{spellitemize}
\item {\it Tempest}: Summon a terrifying storm over an area 1km$^2$. The storm limits visibility to 10\% and deals (6+PP)d4 bludgeoning damage\comma{} and half as much again cold damage to all targets in this radius. In addition\comma{} the caster summon PPd4 lightning bolts per turn\comma{} each of which does 3d8 electrical damage. 
\item {\it Flight}: Use precision manipulation to lift one being of up to (20$\times$PP)kg into the air and move it freely at a speed of up to 10$\times$PP mph within a radius of 200m of the caster. The caster may cast this on themselves to mimic the effects of true flight.  
\end{spellitemize}
You may also use the effects listed in Control Air 4 or below with \PPDifference{\DVMasF}{\DVBegF} PP added. 
This is not an exhaustive list. Be inventive!}
\spell{name = Control Earth 1: Coax, school = Charms, discipline = Elemental, type = Focus, level =Beginner, incant = terrapare, noDur = 1, higher = When cast by a caster greater than 3rd level\comma{} spell effects may be maintained without further FP cost. FP is only deducted when a `new’ effect is initiated\comma{} or a current effect is redirected.,travel = Wandtip glows green,noResist =1, effect =This spell is able to manipulate small stones and cause minor tremors.

Every turn concentration is maintained you may add another effect\comma{} or cancel a previously utilised effect. The maximum number of effects is equal to 1 + ATH modifier (min 1). 

An example of the uses for this spell could be:
\begin{spellitemize}
\item {\it Tremor}: Cause the ground to shake and emit a low rumble. All beings in a 10m radius must make a DV (5+PP) FIN Resist check to maintain their balance\comma{} or take check disadvantage next turn. 
\item {\it Pebbledash}: Cause a number of small stones to hurl themselves at a target within a range of 5m of the caster\comma{} doing (1+PP)d4 bludgeoning damage. 
\end{spellitemize}
This is not an exhaustive list. Be inventive!}
\spell{name = Control Earth 2: Handle, school = Charms, discipline = Elemental, type = Focus, level =Novice, incant = terrapare, noDur = 1, higher = When cast by a caster greater than 6th level\comma{} spell effects may be maintained without further FP cost. FP is only deducted when a `new’ effect is initiated\comma{} or a current effect is redirected.,travel = Wandtip glows green,noResist =1, effect =With this spell the caster may excavate and move small amounts of earth with their mind.

Every turn concentration is maintained you may add another effect\comma{} or cancel a previously utilised effect. The maximum number of effects is equal to 1 + ATH modifier (min 1). 

An example of the uses for this spell could be:
\begin{spellitemize}
\item {\it Excavate:} target an unnocupied area of loose or packed earth up to (1+PP)m in radius. You can instantly excavate it down to a depth of PP/2 metres\comma{} and move it up to 4m per turn. Excavated `packed' earth is considered `loose' after being excavated. 
\item {\it Mold:} target an area of loose earth of a cube up to PP/2 metres in length and manipulate it into taking on any shape you desire. This shape may `defy physics' during the molding\comma{} but as soon as your concentration is broken\comma{} the shape is liable to crumble. Whilst being manipulated\comma{} constructions are not strongly bound\comma{} and so cannot be weaponised\comma{} and a normal human could break them apart with ease. 
\item {\it Holdfast:} root yourself or a target into the Earth\comma{} trapping you in position\comma{} but rendering you immune to forced\minus{}movement effects. Can be broken by a DV (8 + PP) ATH (Strength) check.  
\end{spellitemize}You may also use the effects listed in Control Earth 1 with \PPDifference{\DVNovF}{\DVBegF} PP added. 
This is not an exhaustive list. Be inventive!}
\spell{name = Control Earth 3: Exert, school = Charms, discipline = Elemental, type = Focus, level =Adept, incant = terrapare, noDur = 1, higher = When cast by a caster greater than 9th level\comma{} spell effects may be maintained without further FP cost. FP is only deducted when a `new’ effect is initiated\comma{} or a current effect is redirected.,travel = Wandtip glows green,noResist =1, effect =With this spell\comma{} the caster may the caster may manipulate more tightly bound manipulated earth together\comma{} allowing for potential weaponisation and stronger constructions. 

Every turn concentration is maintained you may add another effect\comma{} or cancel a previously utilised effect. The maximum number of effects is equal to 1 + ATH modifier (min 1). 

An example of the uses for this spell could be:
\begin{spellitemize}
\item {\it Erupt}: target a region (1+PP)m in radius. A fountain of churned earth erupts vertically upwards in that region\comma{} damaging all those inside the region for (3+PP)d12 bludgeoning damage. This region is considered `loose footing' until cleared\comma{} which takes $5 \times$PP turns to do. 
\item {\it Animate Earth}: as with {\it mold}\comma{} but the earth is considered `packed' during motion and you may manipulate `packed' as well as `loose' Earth. You may clumsily animate the manipulated material\comma{} and\comma{} for example\comma{} create an animated hand or club out of the manipulated earth to strike at an enemy\comma{} doing (3+PP)d8 damage (either bludgeoning or piercing\comma{} depending on the shape of the creation). When the effect ends\comma{} the material crumbles into loose earth. 
\end{spellitemize}
You may also use the effects listed in Control Earth 2 or below with \PPDifference{\DVAdpF}{\DVNovF} additional PP added. 
This is not an exhaustive list. Be inventive!}
\spell{name = Control Earth 4: Wield, school = Charms, discipline = Elemental, type = Focus, level =Expert, incant = terrapare, noDur = 1, higher = When cast by a caster greater than 12th level\comma{} spell effects may be maintained without further FP cost. FP is only deducted when a `new’ effect is initiated\comma{} or a current effect is redirected.,travel = Wandtip glows green,noResist =1, effect =With this spell\comma{} the caster may manipulate earth on a previously unprecedented scale\comma{} or manipulate the earth into a protective shield. 

Every turn concentration is maintained you may add another effect\comma{} or cancel a previously utilised effect. The maximum number of effects is equal to 1 + ATH modifier (min 1). 

An example of the uses for this spell could be:

\begin{spellitemize}
\item {\it Grand Manipulation}: reshape dirt\comma{} clay\comma{} sand\comma{} earth of any kind\comma{} or natural rock in an area (6+PP) in radius in any fashion you choose: raise or lower the area\apos{}s elevation\comma{} raise a wall or a pillar up to (2+PP)m in height\comma{} dig or fill in a trench up to (2+PP)m in height. You may mold detailed figures and details into your creations. When concentration is broken\comma{} the material is considered `loose’\comma{} and will crumble as normal physics dictates – i.e. pillars will crumble into smaller hillocks etc. 
\item {\it Fissure}: tear the ground asunder beneath the feet of a target in range. If they fail a DV (12+PP) FIN Resist check\comma{} they fall (4+PP)m into the fissure\comma{} taking (2+PP)d4 damage. Closing a fissure on a being deals (3+PP)d12 damage and deprives them of air until they succeed a DV 10 ATH check to escape.
\item {\it Clad Being}: wrap a target in an armour of solid rock\comma{} which moves with their movements as long as concentration is maintained on this effect. Gives an AC of (20 + $2\times$PP)\comma{} and unarmed attacks deal (2+PP)d8 bludgeoning damage. 
\end{spellitemize}
You may also use the effects listed in Control Earth 3 or below with \PPDifference{\DVExpF}{\DVAdpF} additional PP added. 
This is not an exhaustive list. Be inventive!}
\spell{name = Control Earth 5: Master, school = Charms, discipline = Elemental, type = Focus, level =Master, incant = terrapare, noDur = 1, higher = When cast by a caster greater than 15th level\comma{} spell effects may be maintained without further FP cost. FP is only deducted when a `new’ effect is initiated\comma{} or a current effect is redirected.,travel = Wandtip glows green,noResist =1, effect =Master the element of earth\comma{} and gain the ability to summon devastating earthquakes\comma{} or create towering wonders.

Every turn concentration is maintained you may add another effect\comma{} or cancel a previously utilised effect. The maximum number of effects is equal to 1 + ATH modifier (min 1). 

An example of the uses for this spell could be:

\begin{spellitemize}
\item {\it Permanent Structures}: you may permanently imbue your creations with power\comma{} allowing impossible structures to be maintained permanently. This allows the caster to build anything they desire out of rock and earth. 
\item {\it Earthquake}: create a great seismic disturbance in the ground in a (30 + 10$\times$PP)m radius. For the spell\apos{} duration\comma{} All beings on the ground that are concentrating must pass a DV 15 SPR resist check to maintain focus\comma{} and all beings making a movement must pass a DV 15 ATH resist check\comma{} or be knocked to the ground. The caster may open or shut up to 2d4 fissures (see {\it Control Earth 4}) per turn\comma{} for the FP cost of opening 1 fissure. 
\end{spellitemize}
You may also use the effects listed in Control Earth 4 or below with \PPDifference{\DVMasF}{\DVBegF} PP added. 
This is not an exhaustive list. Be inventive!}
\spell{name = Control Fire 1: Coax, school = Charms, discipline = Elemental, type = Focus, level =Beginner, incant = ignipare, noDur = 1, higher = When cast by a caster greater than 3rd level\comma{} spell effects may be maintained without further FP cost. FP is only deducted when a `new’ effect is initiated\comma{} or a current effect is redirected.,travel = Wandtip glows red,noResist =1, effect =}
\spell{name = Control Fire 2: Handle, school = Charms, discipline = Elemental, type = Focus, level =Novice, incant = ignipare, noDur = 1, higher = When cast by a caster greater than 6th level\comma{} spell effects may be maintained without further FP cost. FP is only deducted when a `new’ effect is initiated\comma{} or a current effect is redirected.,travel = Wandtip glows red,noResist =1, effect =}
\spell{name = Control Fire 3: Exert, school = Charms, discipline = Elemental, type = Focus, level =Adept, incant = ignipare, noDur = 1, higher = When cast by a caster greater than 9th level\comma{} spell effects may be maintained without further FP cost. FP is only deducted when a `new’ effect is initiated\comma{} or a current effect is redirected.,travel = Wandtip glows red,noResist =1, effect =}
\spell{name = Control Fire 4: Wield, school = Charms, discipline = Elemental, type = Focus, level =Expert, incant = ignipare, noDur = 1, higher = When cast by a caster greater than 12th level\comma{} spell effects may be maintained without further FP cost. FP is only deducted when a `new’ effect is initiated\comma{} or a current effect is redirected.,travel = Wandtip glows red,noResist =1, effect =}
\spell{name = Control Fire 5: Master, school = Charms, discipline = Elemental, type = Focus, level =Master, incant = ventepare, noDur = 1, higher = When cast by a caster greater than 15th level\comma{} spell effects may be maintained without further FP cost. FP is only deducted when a `new’ effect is initiated\comma{} or a current effect is redirected.,travel = Wandtip glows red,noResist =1, effect =}
\spell{name = Control Water 1: Coax, school = Charms, discipline = Elemental, type = Focus, level =Beginner, incant = aguapare, noDur = 1, higher = When cast by a caster greater than 3rd level\comma{} spell effects may be maintained without further FP cost. FP is only deducted when a `new’ effect is initiated\comma{} or a current effect is redirected.,travel = Wandtip glows blue,noResist =1, effect =}
\spell{name = Control Water 2: Handle, school = Charms, discipline = Elemental, type = Focus, level =Novice, incant = aguapare, noDur = 1, higher = When cast by a caster greater than 6th level\comma{} spell effects may be maintained without further FP cost. FP is only deducted when a `new’ effect is initiated\comma{} or a current effect is redirected.,travel = Wandtip glows blue,noResist =1, effect =}
\spell{name = Control Water 3: Exert, school = Charms, discipline = Elemental, type = Focus, level =Adept, incant = aguapare, noDur = 1, higher = When cast by a caster greater than 9th level\comma{} spell effects may be maintained without further FP cost. FP is only deducted when a `new’ effect is initiated\comma{} or a current effect is redirected.,travel = Wandtip glows blue,noResist =1, effect =}
\spell{name = Control Water 4:  Wield, school = Charms, discipline = Elemental, type = Focus, level =Expert, incant = aguapare, noDur = 1, higher = When cast by a caster greater than 12th level\comma{} spell effects may be maintained without further FP cost. FP is only deducted when a `new’ effect is initiated\comma{} or a current effect is redirected.,travel = Wandtip glows blue,noResist =1, effect =}
\spell{name = Control Water5: Master, school = Charms, discipline = Elemental, type = Focus, level =Master, incant = aguapare, noDur = 1, higher = When cast by a caster greater than 15th level\comma{} spell effects may be maintained without further FP cost. FP is only deducted when a `new’ effect is initiated\comma{} or a current effect is redirected.,travel = Wandtip glows blue,noResist =1, effect =}
\spell{name = Countercurse, school = Recuperation, discipline = Healing, type = Instant, level =Novice, incant = finite maledictum, noDur = 1, noHigh = 1, travel = Pale\minus{}blue rays,resist = \CurCheck, resistDV = CC + 2 $\times$ PP, effect =Remove the effects of an active spell from the Curse discipline. The caster of the curse performs a resist check using the original spellcasting check dice and bonuses\comma{} if the resist fails\comma{} the spell effect is ended.}
\spell{name = Coven\apos{}s Protection, school = Dark Arts, discipline = Occultism, type = Ritual (4 hours), level =Expert, noIncant = 1, noDur = 1, noHigh = 1, noTravel = 1, noResist =1, effect =This ritual can only be performed in an isolated area\comma{} in the middle of the night. By gathering together and invoking the name of an unspeak\minus{} able\comma{} unknowable power\comma{} you bind the life forces of all participants together\comma{} to form a unified whole. While the Coven exists\comma{} any member may use a minor action to transfer HP or FP to any other member of the coven\comma{} and multiple members may send HP/FP to the same target at any given mo\minus{} ment. However\comma{} if psychic damage is inflicted on any member of the Coven in the same turn\minus{}cycle that HP or FP are being transferred\comma{} that damage is done to all members of the Coven. If this damage is also a Critical Strike\comma{} then the ritual is distrupted and the spell is ended.}
\spell{name = Create Fire, school = Charms, discipline = Elemental, type = Focus, level =Beginner, incant = incendio, noDur = 1, higher = An Adept\minus{}level caster may summon a larger gout of flame\comma{} capable of igniting even damp wood. Such a gout  gains an additional 1d4 fire damage for every Power Point dedicated to the spell.,noTravel = 1, noResist =1, effect =A small jet of fire is emitted from the tip of your wand\comma{} akin to a large lighter. 
Coming into contact with fire does 1d4 fire damage\comma{} and applies a minor Burned status effect.}
\spell{name = Create Golem, school = Transfiguration, discipline = Conjuration, type = Ritual (1 week), level =Adept, incant = lapis libiri, noDur = 1, higher = When cast by a master\minus{}level caster\comma{} the ritual only takes 2 hours.,noTravel = 1, noResist =1, effect =Breathe life into a block of inanimate matter\comma{} turning it into a hulking Golem. Basic spell produces a Weak Stone Golem. 1 power point gives a weak iron golem\comma{} 2 gives a weak crystal golem\comma{} 3 a capable stone golem and so on. Require a large amount of the golem material to cast.}
\spell{name = Create Horcrux, school = Dark Arts, discipline = Necromancy, type = Ritual (1 day), level =Master, incant = pervinco mortis, noDur = 1, noHigh = 1, noTravel = 1, noResist =1, effect =The caster places a portion of their soul into another object. Write down the horcrux on a piece of paper and keep it hidden. 

Whilst a horcrux exists\comma{} the character cannot be killed\comma{} however creating a Horcrux reduces all casting checks by 2 for each horcrux that is created.  
Can only be cast if the caster has murdered an innocent in cold blood.}
\spell{name = Create Thrall, school = Dark Arts, discipline = Necromancy, type = Focus, level =Expert, incant = Imperius, noDur = 1, noHigh = 1, noTravel = 1, resist = SPR (Endurance), resistDV = CC, effect =The target is placed under the complete control of the caster until Focus is broken.}
\spell{name = Create Trap, school = Charms, discipline = Kinesis, type = Ritual (3 turns), level =Beginner, incant = dolus, noDur = 1, higher = A character above 10th level may add free PP to the effect\minus{}spell equal to one\minus{}third their character level.,noTravel = 1, noResist =1, effect =Combine a magical ward with one of your existing spells. After casting the trap spell\comma{} cast the effect\minus{}spell to imbue the trap with that effect. 
If successful\comma{} creates a hidden magical trap of radius 50cm on any solid surface\comma{} with the effect of the original spell when triggered by an entity touching the trap. If you wish to keep a trap hidden from the GM\comma{} write down the location\comma{} spell and associated check values on a piece of paper\comma{} to be revealed when the trap is triggered.}
\spell{name = Create Water, school = Charms, discipline = Elemental, type = Focus, level =Beginner, incant = aguamente, noDur = 1, higher = An adept\minus{}level caster may summon a more powerful torrent of water\comma{} equal to a number of litres of water per second. Such a torrent does 1d4 bludgeoning damage for every power point dedicated.,noTravel = 1, noResist =1, effect =A jet of water is emitted from the tip of your wand\comma{} in a fountain approximately 30cm in length\comma{} useful for extinguishing fires small\comma{} or cleaning surfaces\comma{} however conjured water cannot be drunk.}
\spell{name = Create Zombie, school = Dark Arts, discipline = Necromancy, type = Ritual (5 minutes), level =Expert, incant = inferi exorior, noDur = 1, higher = When cast by a Master\minus{}level caster greater than 15th level\comma{} can be cast as an Instant spell.,noTravel = 1, noResist =1, effect =Breathes unlife into dead bodies\comma{} and turns them into ghastly puppets\comma{} performing your every whim: the inferi. Inferi act as golems\comma{} obeying every word of their creator. One inferi may be animated plus one \cvdv.}
\spell{name = Crippling Fatigue, school = Dark Arts, discipline = Necromancy, type = Instant, level =Novice, incant = dulcis mortem, duration = Until healed,higher = When cast by an Expert\minus{}level caster\comma{} spell gives 4th level Fatigue.,noTravel = 1, resist = SPR (Endurance), resistDV = CC, effect =Target takes 2nd level Fatigued status (negated on Resist). Target is not alerted that this spell has been cast on them.}
\spell{name = Crush Bones, school = Maledictions, discipline = Hex, type = Instant, level =Master, incant = obcillo ossium, noDur = 1, noHigh = 1, noTravel = 1, noResist =1, effect =A great force smashes into the opponent\comma{} breaking their bones. Does (8+$2\times$PP)d12 bludgeoning damage. Applies the Broken Bone status effect.}
\spell{name = Crystal Gazing, school = Divination, discipline = Temporal, type = Ritual (4 turns), level =Novice, incant = Gazing, noDur = 1, noHigh = 1, noTravel = 1, noResist =1, effect =Gaze into your crystal ball\comma{} and ask a question of the cosmos. You will receive a yes or a no answer to any question you ask.}
\spell{name = Cushion Fall, school = Charms, discipline = Kinesis, type = Instant, level =Expert, incant = sofus, noDur = 1, noHigh = 1, noTravel = 1, noResist =1, effect =Painlessly break the fall of the target from any height up to (25+ $25\times$PP) metres.}
\spell{name = Cut Object, school = Charms, discipline = Kinesis, type = Instant, level =Novice, incant = diffindo, noDur = 1, noHigh = 1, travel = Silver flash,noResist =1, effect =Cut into an object\comma{} as if you had wielded a sharp knife with a blade of up to 10cm in length.
If used on a living being\comma{} causes a deep cut\comma{} for 1d4 + 3 slashing damage.}
\spell{name = Dark Healing, school = Dark Arts, discipline = Necromancy, type = Instant, level =Novice, incant = tenebrosa sudarium, noDur = 1, noHigh = 1, travel = Black rays,noResist =1, effect =Heal for one HP for each casting point over the difficulty. Remove half of the restored HP from a willing or restrained target.}
\spell{name = Delayed Effect, school = Maledictions, discipline = Curse, type = Instant, level =Adept, incant = mora maledictus, duration = 1 hour,noHigh = 1, travel = Red bolt,noResist =1, effect =This spell may be cast as if it were a spell of any level greater than Beginner. After a target is hit by this spell\comma{} you must write down another curse that you are able to cast\comma{} of a lower level than the one chosen to cast this spell. At any point in the next hour\comma{} you may reveal the chosen curse\comma{} and the effects of that spell are then immediately applied to the target.}
\spell{name = Delusion, school = Illusion, discipline = Psionics, type = Instant, level =Expert, incant = falasarium, duration = (2 + 2$\times$ PP) hours,noHigh = 1, noTravel = 1, resist = INT (perception), resistDV = 10, effect =If target fails a Resist check\comma{} the caster may make them believe one piece of information\comma{} which they will believe to be irrefutably true. The delusion must be vaguely rational\comma{} and may not incur excessive self\minus{}harm\comma{} as judged by the GM. Delusion lasts for (2 + 2 $\times$ PP) hours.}
\spell{name = Detect Magic, school = Divination, discipline = Telepathy, type = Instant, level =Novice, incant = revelio, noDur = 1, noHigh = 1, noTravel = 1, noResist =1, effect =Reveals to the caster any active spells in the in 5m range if the casting check exceeds the hiding check. Will deactivate charms whose sole purpose is to remain hidden.}
\spell{name = Detect Thoughts, school = Divination, discipline = Telepathy, type = Focus, level =Novice, incant = psychopractum, noDur = 1, higher = An Master\minus{}level caster may subtly alter the flow of a target\apos{}s thoughts\comma{} and cause them to think about whatever the caster desires.,noTravel = 1, resist = Perception (Passive), resistDV = CC, effect =You may observe the mind of a target individual. Unlike legilimency\comma{} thought\minus{}detection is not an exact science\comma{} and you will only get a vague shape of their current thoughts – perhaps a quick flash of colour\comma{} or a feeling of fear. On a successful (passive) Resist\comma{} target becomes aware of the process.}
\spell{name = Disarm, school = Maledictions, discipline = Curse, type = Instant, level =Novice, incant = expelliarmus, noDur = 1, noHigh = 1, travel = Orange bolt,resist = ATH (Strength), resistDV = CC, effect =Target performs an ATH(strength) resist check against the casting check. If it fails\comma{} the object in the target\apos{}s hand is hurled in a random direction. If two obejcts are held\comma{} roll a d4\comma{} a 1 or 2=both objects\comma{} 3 = left hand\comma{} 4 =  right hand.}
\spell{name = Disintegrate, school = Maledictions, discipline = Hex, type = Instant, level =Master, incant = reducto, noDur = 1, noHigh = 1, noTravel = 1, resist = POW (Endurance), resistDV = 14, effect =If the spell makes contact with matter\comma{} causes it to instantly disintegrate. Living beings take 20d6 worth of force damage. Resist for half damage.}
\spell{name = Disrupt Connection, school = Divination, discipline = Telepathy, type = Instant, level =Adept, incant = ruinosus, noDur = 1, noHigh = 1, travel = Loud crack sound,resist = SPR (Endurance), resistDV = CC, effect =Sever a telepathic connection possessed by the target. This may be used to sever a link such as that caused by the {\it Telepathic Bond} spell\comma{} or the link between caster and a summoned being. An untethered summoned being no  longer accepts orders from their creator\comma{} and will potentially attack them. A successful resist negates this effect\comma{} but the target takes 2d4 psychic damage instead.}
\spell{name = Draconic Guardians, school = Transfiguration, discipline = Alteration, type = Instant, level =Expert, incant = draconifors, duration = 1 hour,higher = A master\minus{}level caster may transfigure 2d4 small objects.,noTravel = 1, noResist =1, effect =Transform 1d4 small objects into miniature dragons to fight by your side. Dragons have (10+3$\times$ PP)HP and do (2+PP)d4 fire damage.}
\spell{name = Dragon{\apos}s Breath, school = Maledictions, discipline = Hex, type = Focus, level =Expert, incant = draco flammor, noDur = 1, noHigh = 1, noTravel = 1, resist = ATH (Speed), resistDV = 12, effect =A torrent of flame erupts from the tip of your wand in a cone 3 metres in front of the caster\comma{} incinerating everything in its path. Water cannot quench this fire\comma{} and it causes 1d12 worth of fire damage for every 3 character levels the caster has above 3rd.  Resist for half damage.}
\spell{name = Drain Fortitude, school = Illusion, discipline = Psionics, type = Focus, level =Adept, incant = delcrus, noDur = 1, higher = A master\minus{}level caster drains 4d4 FP per turn.,travel = Blue Thread,resist = SPR (Endurance), resistDV = CC, effect =By imposing your will over that of your target\comma{} you do  2d4 Fatigue damage to the target per turn and add it to your own FP reserve. This spell does not cost FP to sustain\comma{} however if your Focus is broken i.e. by an attack\comma{}  then the effect is negated and no `drain' happens.}
\spell{name = Duplicate Object, school = Transfiguration, discipline = Conjuration, type = Instant, level =Expert, incant = gemino, duration = 12 hours,noHigh = 1, noTravel = 1, noResist =1, effect =Creates a copy of an object in your possession\comma{} which is identical to the first\comma{} until it disintegrates 12 hours later.}
\spell{name = Eavesdrop, school = Divination, discipline = Telepathy, type = Focus, level =Novice, incant = dumauris, noDur = 1, noHigh = 1, noTravel = 1, noResist =1, effect =Can listen in on conversations up to (10 + 2$\times$PP) meters away without the targets becoming aware of you.}
\spell{name = Eldritch Knowledge, school = Dark Arts, discipline = Occultism, type = Ritual (3 turns), level =Beginner, incant = vetitum scenticus, noDur = 1, noHigh = 1, travel = Yellow\minus{}black aura,noResist =1, effect =Attune your mind to the Eldritch Domains. The Demons of the Deep will answer one of your questions\comma{} but the answers might drive you mad.
The question must be said out loud for all to hear\comma{} but the answer may be written down and passed to your privately.}
\spell{name = Electrical Arc, school = Maledictions, discipline = Hex, type = Focus, level =Expert, incant = electrum maxima, noDur = 1, noHigh = 1, travel = Blue arc,resist = ATH (Health), resistDV = CC, effect =Whilst you maintain Focus\comma{} a bolt of energy arcs from the end of your wand\comma{} doing (4+PP)d6 electrical damage per turn\comma{} halved ona successful resist.}
\spell{name = Elemental Weapon, school = Charms, discipline = Elemental, type = Instant, level =Novice, incant = gladio subtantia, noDur = 1, higher = When cast by an adept level caster\comma{} you may choose the melee weapon that the spell forms (the damage adjusts accordingly\comma{} including any proficiencies you may have). A master\minus{}level caster forms a +3 version of that weapon.,travel = 5 minutes,noResist =1, effect =The elements are bent to your will\comma{} and a blade of nature\minus{}incarnate solidifies around your wand. You now wield a 1d6 sword made out of pure fire\comma{} ice\comma{} lightning or earth\comma{} held together by your strength of will. In addition to the physical cutting effect\comma{} this blade also imparts an elemental effect: \begin{itemize} \item Fire: additional 1d6 fire damage \item Ice: additional 1d6 cold damage \item Lightning: additional 1d6 electric damage \item Control Earth: additional 1d6 bludgeoning damage \end{itemize}
Elemental damage increases by 1d6 for every power point dedicated. You may dismiss this effect using a minor action.}
\spell{name = Enchant Animal, school = Illusion, discipline = Bewitchment, type = Instant, level =Novice, incant = nonparum, duration = ($4+2\times$PP) turns,noHigh = 1, noTravel = 1, noResist =1, effect =Commune with a non\minus{}aggressive animal: it will join you as an ally for (4+$2\times$PP) turns.}
\spell{name = Enchantment Ritual, school = Transfiguration, discipline = Alteration, type = Ritual (1 day), level =Adept, noIncant = 1, noDur = 1, noHigh = 1, noTravel = 1, noResist =1, effect =The enchantment ritual used to imbue items with magical effects. See page \pageref{S:Enchanting} for details.}
\spell{name = Endure Environment, school = Recuperation, discipline = Healing, type = Instant, level =Novice, incant = omnium, duration = 1 day,noHigh = 1, noTravel = 1, noResist =1, effect =Target is protected from the ravages of the environment\comma{} and hence can exist in temperatures in the range \minus{}40  to 50 celsius\comma{} and is unaffected by heavy rain and other weather phenomena. Does not protect against heat damage.}
\spell{name = Entrance Other, school = Illusion, discipline = Bewitchment, type = Instant, level =Adept, incant = meamicus, duration = (3 + PP) turns,noHigh = 1, noTravel = 1, resist = SPR (endurance), resistDV = 8 + PP, effect =If the target fails a resist magic check\comma{} they become hopelessly besotted with the caster for 5 turns. Besotted individuals take check double disadvantage in all negative actions relating to their beloved.}
\spell{name = Eternal Flame, school = Transfiguration, discipline = Conjuration, type = Instant, level =Novice, incant = bangala, duration = Infinite,noHigh = 1, travel = Blue flames,noResist =1, effect =Summon a minor fire spirit. If you have a glass container\comma{} you may use it trap the spirit\comma{} whereupon it acts as a dim torch (bright light 1m\comma{} dim light 2m) and as a heat source sufficient to keep one person comfortable in artcic conditions. If the spirit is not trapped\comma{} a random being within 5m radius of caster takes 2d6 fire damage.}
\spell{name = Ethereal Tag, school = Divination, discipline = Telepathy, type = Instant, level =Expert, incant = signum, duration = (2 + 2$\times$ PP) minutes,noHigh = 1, travel = Yellow bolt,resist = FIN (Stealth), resistDV = CC, effect =If the target fails to Resist\comma{} place a mystical marker on the target which enables your allies to strike more accurately at them. Target\apos{}s stealth checks fail\comma{} and evasion checks get a \minus{}5 penalty for (2+PP) turns.}
\spell{name = Extinguish Flame, school = Charms, discipline = Elemental, type = Instant, level =Novice, incant = sitim, noDur = 1, higher = An expert\minus{}level caster may cast this spell as an Expert\minus{}level spell (with the increased DV and FP associaed with that) to allow this spell to effect {\it Fiendfyre}.,noTravel = 1, noResist =1, effect =Extinguish an active fire\comma{} removing the danger and stopping any continuing damage effects. However\comma{} this spell does not preventan ongoing spell from producing more fire after it is removed\comma{} and nor does it affect the `Burned' status of a being.}
\spell{name = Fabricate Object, school = Transfiguration, discipline = Alteration, type = Instant, level =Novice, incant = facere, noDur = 1, higher = A character above 6th level may add 1 free PP for every 3 character levels above 3rd.,noTravel = 1, noResist =1, effect =Construct an object from raw materials\comma{} assembling it at a molecular level. May manipulate 500g of raw material in this fashion\comma{} doubling the mass with every power point dedicated. Consruction is permanent\comma{} and cannot be undone.}
\spell{name = False Friend, school = Illusion, discipline = Psionics, type = Instant, level =Adept, incant = amicus maxmius, duration = 1 minutes,noHigh = 1, travel = Green rays,resist = INT (perception), resistDV = , effect =Gain a (2 + 1 \cvdv) bonus on all CHR checks directed at the target for 10 minutes. At the end of the spell\comma{} or if they succeed in Resisting the target becomes aware that you have enchanted and deceived them\comma{} and will become hostile\comma{} or otherwise seek vengeance.}
\spell{name = False Moon, school = Dark Arts, discipline = Occultism, type = Focus, level =Adept, incant = lupis lunis, duration = 1 hour,higher = When cast by a character greater than 15th level\comma{} the spell works after only 1 turn.,travel = Silver Glow,noResist =1, effect =If the spell is maintained on a targeted werewolf for 3 consecutive turns\comma{} they enter their wolf\minus{}form for 1 hour.}
\spell{name = Fearsome Guardians, school = Transfiguration, discipline = Alteration, type = Instant, level =Master, incant = piertotom locomotum, duration = 1 day,noHigh = 1, noTravel = 1, noResist =1, effect =Transform nearby statues\comma{} trees and other inanimate objects into powerful guardians to fight by your side. Guardians are considered as Capable Stone Golems unless otherwise indicated.}
\spell{name = Featherweight, school = Transfiguration, discipline = Alteration, type = Instant, level =Adept, incant = pluma gravitas, duration = 1 hour,noHigh = 1, noTravel = 1, noResist =1, effect =Make the target object as light as a feather\comma{} it does not encumber you.
(Note that heavy weapons such as axes may lose their effectiveness when made featherweight)}
\spell{name = Feign Death, school = Recuperation, discipline = Healing, type = Ritual (30 minutes), level =Adept, incant = fautis, duration = (1+$2\times$PP) hours,noHigh = 1, noTravel = 1, noResist =1, effect =When cast upon a willing living being\comma{} they are placed into a state of suspended animation which perfectly replicates the outward appearance of death. Divination checks with a casting check greater than this spell may peirce the deception. The target is blinded\comma{} deafened and physically incapacitated for the duration of the spell. The caster may revive them as a minor action.}
\spell{name = Fidelius Ward, school = Recuperation, discipline = Warding, type = Ritual (1 week), level =Master, incant = onsigno scientia, noDur = 1, noHigh = 1, noTravel = 1, resist = POW (Perception), resistDV = 18, effect =Seals away all knowledge of the target inside the mind of the {\it Keeper}. The target can then only be seen\comma{} detected\comma{} and even known of by the caster\comma{} and the keeper\comma{} and those that they tell. The ultimate protective ward\comma{} since no\minus{}one even knows that the target exists. The caster cannot also be the Keeper\comma{} and if the target is a place (i.e. a house)\comma{} then the Keeper cannot reside in the region. \\ If a being knew of the target before the Fidelius spell was cast\comma{} they may Resist in order to continue to remember their existence.}
\spell{name = Fiendfyre, school = Dark Arts, discipline = Necromancy, type = Instant, level =Adept, incant = pyrkagius, duration = 1 hour,noHigh = 1, travel = Flame dragon,resist = SPR, resistDV = CC, effect =Summons a cursed fire that consumes everything that it touches\comma{} and actively seeks to destroy living beings as if it were a living being. Fire may send out up to 1d4 tendrils per turn to strike at a target\comma{} doing (1+PP)d8 fire damage to all it touches.
Attempts to extinguish the fiendfyre must succeed a Resist check. Failed extinguishing attempts trigger an attack from the fiendfyre.}
\spell{name = Fireball, school = Maledictions, discipline = Hex, type = Instant, level =Adept, incant = confringo, noDur = 1, noHigh = 1, travel = Large fiery bolt,noResist =1, effect =Launches a fireball at the target\comma{} which explodes for 5+(1+PP)d8 fire damage in a 2m radius.
Targets suffer a moderate burn.}
\spell{name = Fix Object, school = Charms, discipline = Kinesis, type = Focus, level =Novice, incant = reparo, noDur = 1, noHigh = 1, travel = Green rays,noResist =1, effect =Fix the target. Can only fix an object if it is reasonable that you could have repaired it with your bare hands (i.e. you cannot repair complex machinery without expert knowledge). Simple tasks (i.e. repairing glasses) work in a single turn\comma{} but repairing larger structures (i.e. a full stained glass window) require continued Focus.}
\spell{name = Fix Transformation, school = Transfiguration, discipline = Alteration, type = Ritual (30 seconds), level =Expert, incant = perpetuus, noDur = 1, higher = When cast by a caster greater than 15th level\comma{} this spell may be cast as an Instant spell.,travel = Golden rays,noResist =1, effect =When cast on any transfigured or conjured object\comma{} makes the transformation permanent and removes the time constraint. When a counterspell is used\comma{} this spell is removed and the countdown resumes.}
\spell{name = Floodlight, school = Charms, discipline = Elemental, type = Instant, level =Adept, incant = caecus, duration = White beam,noHigh = 1, travel = 0,resist = INT (perception), resistDV = CC, effect =Direct a brilliant beam of light from the tip of your wand\comma{} illuminating a cone 10m  in front of you with Bright light\comma{} and dim light a further 10m. If a target is illuminated by the beam and fails to resist\comma{} they are blinded for 2 turns.}
\spell{name = Foresight, school = Divination, discipline = Temporal, type = Instant, level =Adept, incant = providentia, duration = 1 turn,higher = A master\minus{}level caster may provide 1+1d4 extra actions per turn.,noTravel = 1, noResist =1, effect =By predicting the flow of time\comma{} you can give the target the ability to make moves without thinking: give the target an extra major action next turn. Each target can only get this boost once per day.}
\spell{name = Freeze, school = Charms, discipline = Elemental, type = Focus, level =Adept, incant = glacius, duration = Blue rays,noHigh = 1, travel = 0,noResist =1, effect =Lower the temperature in a cone extending up to 2m out of your wand by 40 degrees celsius\comma{} freezing the target.  When used in combat\comma{} trap the target in place and do 3d6 cold damage. The target is trapped until they are thawed out\comma{} at which point the Frostbite status is applied.}
\spell{name = Fresh Air, school = Charms, discipline = Elemental, type = Instant, level =Beginner, incant = klinneract, noDur = 1, noHigh = 1, noTravel = 1, noResist =1, effect =A gust of air refreshes the air in a sphere of radius (2 + PP) metres around the caster\comma{} removing any gaseous effects and smelling faintly of lavender.}
\spell{name = Fury, school = Illusion, discipline = Psionics, type = Instant, level =Adept, incant = irafors, duration = 1 + 1 \cvdv turns,higher = When cast by a character greater than 12th level\comma{} the DV of the Resist check is equal to the caster level if that is greater than the casting check.,travel = Red bolt,resist = EMP (willpower), resistDV = CC, effect =Target performs a Resist Magic check\comma{} if they fail\comma{} target flies into a mindless rage and begins attacking all those around them.}
\spell{name = Fury\apos{}s Fire, school = Dark Arts, discipline = Occultism, type = Beast, level =Beginner, noIncant = 1, duration = 3,higher = When cast by a Veela greater than 10th level\comma{} use a d8 dice for the damage check.,travel = Red Fireball,noResist =1, effect ={\bf This spell can only be cast by a Veela when in their Fury\minus{}form. It cannot be learned by non\minus{}Veela.}

A will\minus{}sapping fireball hurled by a Veela or their half\minus{}human brood. On contact ignites the target for (1+PP)d6 fire damage for the duration of the spell. The affected target must use one dice smaller for Resist checks than their usual dice for the next 5 turns.}
\spell{name = General Counterspell, school = Charms, discipline = Kinesis, type = Ritual (3 turns), level =Expert, incant = finite incantatem, noDur = 1, higher = When cast by a character greater than 15th level\comma{} may be cast as an instant spell.,travel = Golden rays,noResist =1, effect =End the effects of any active spell. If the spell was cast by anyone other than yourself\comma{} the counterspell check must exceed the original casting check (does not work on enemy shields\comma{} curses or spells which are still being cast).}
\spell{name = Glacial Chill, school = Maledictions, discipline = Hex, type = Instant, level =Expert, incant = gelidus, duration = 3 turns,noHigh = 1, travel = Blue Glow,resist = SPR (Endurance), resistDV = CC, effect =A cylinder of radius 5m and height 2m around the target is decreased in temperature by 50 degrees celsius. Those caught in the region take (5+PP)d4 of cold damage\comma{} and apply the mild Frostbite status effect. Resist for half damage.}
\spell{name = Glamour, school = Illusion, discipline = Bewitchment, type = Instant, level =Beginner, incant = lux stultium, duration = 1 hour,higher = When cast by a character greater than 8th level\comma{} the DV of the Resist check is equal to the caster level.,noTravel = 1, resist = INT (perception), resistDV = 5, effect =Create a superficial glamour around a person or object\comma{} a simple trick of the light. The glamour disintegrates upon physical or magical contact\comma{} and can be seen to be fake if observer succeeds on a Resist check.}
\spell{name = Glimpse Future, school = Divination, discipline = Temporal, type = Instant, level =Adept, incant = posterus, noDur = 1, noHigh = 1, noTravel = 1, noResist =1, effect =Get a fleeting glimpse into the future: Automatically dodge the next attack\comma{} in addition to your regular action\comma{} OR\comma{} your next attack always hits its target.}
\spell{name = Green Sparks, school = Maledictions, discipline = Hex, type = Instant, level =Beginner, incant = verdimillious, noDur = 1, noHigh = 1, travel = Green bolts,resist = INT (Perception), resistDV = CC, effect =Shoots (2+PP) green sparks from your wand\comma{} which can be made to strike at a single enemy. Each spark does 1d4 force damage. Resist for half damage.}
\spell{name = Halt, school = Charms, discipline = Kinesis, type = Instant, level =Beginner, incant = stabit, noDur = 1, noHigh = 1, travel = Pale blue bolt,noResist =1, effect =Stop 1 inanimate object in its tracks\comma{} if mid\minus{}air\comma{} it drops to the ground. If the target is particularly small or fast (i.e. an arrow in mid\minus{}flight) the caster must pass a FIN(precision) check (DV 12) in order to hit the target.}
\spell{name = Harden Object, school = Transfiguration, discipline = Alteration, type = Instant, level =Novice, incant = duro, duration = 2 days,noHigh = 1, travel = Grey bolt,noResist =1, effect =Freezes a non\minus{}living object into its current form\comma{} and can no longer bend or flex. Object gains an effective AC of 25.}
\spell{name = Haste, school = Charms, discipline = Kinesis, type = Instant, level =Adept, incant = silvam currere, duration = 5 minutes,noHigh = 1, noTravel = 1, noResist =1, effect =The target has their Speed proficiency increased by (1+PP) points for the duration of the spell. At the end of the effect\comma{} target must take 1 turn to rest and recover.}
\spell{name = Heal Being, school = Recuperation, discipline = Healing, type = Instant, level =Novice, incant = episkey, noDur = 1, higher = When cast by an expert\minus{}level caster\comma{} heal for 2HP + 4 \cvdv.,travel = Yellow\minus{}white rays,noResist =1, effect =Heal minor status effects like burns\comma{} bruises\comma{} broken noses and so on. If no status effect present\comma{} heal for 2HP + 2 \cvdv. If target has a serious wound (i.e. a broken bone or serious burn)\comma{} cannot heal beyond 75\% health.}
\spell{name = Heat Object, school = Maledictions, discipline = Hex, type = Instant, level =Adept, incant = flagrante, duration = 3 turns,noHigh = 1, travel = Red rays,noResist =1, effect =Causes a target object to heat up to unimaginable temperatures\comma{} doing (3+$2\times$PP)d6 fire damage every time the target object is touched\comma{} and applies a severe Burn status effect.}
\spell{name = Hoist Enemy, school = Maledictions, discipline = Curse, type = Focus, level =Novice, incant = levicorpus, noDur = 1, higher = When cast by an Expert spellcaster\comma{} the target may be moved around whilst airborn at a speed of 1m per turn.,travel = Invisible pulse,noResist =1, effect =Target is hoisted into the air. Whilst airborne\comma{} all checks by the target suffer a \minus{}2 penalty. 
Caster can then throw target up to 2 metres in any direction\comma{} with the target taking 1d6 bludgeoning damage. If spell is interrupted before they are hurled\comma{} they instead take 2 bludgeoning damage as they fall to the floor.}
\spell{name = Holy Ward, school = Recuperation, discipline = Warding, type = Ward, level =Expert, incant = pervetutem luminis, noDur = 1, noHigh = 1, noTravel = 1, noResist =1, effect =Create a region where the Unlife cannot pass. Unlife attempting to cross the barrier are ignited for 2d12 worth of holy damage per turn that they remain inside the area\comma{} and a Major Burn. 
Shield fails when (30 + 2 $\times$PP) damage has been inflicted. Radius of ward is 10m.}
\spell{name = Hovering Light, school = Charms, discipline = Elemental, type = Instant, level =Novice, incant = globus, duration = 1 minutes,higher = When cast by an Adept\minus{}level caster\comma{} this can be cast as a {\it Focus spell} (following the usual rules)\comma{} in which case the caster can direct the light to move as they desire for as long as Focus is maintained. After Focus is broken\comma{} the light is extinguished.,travel = Glowing orb,noResist =1, effect =Summons a glowing orb\comma{}around 5cm in diameter that hovers above the caster\apos{}s head\comma{} casting bright light for 4m\comma{} and dim light for a further 4m.}
\spell{name = Howl, school = Maledictions, discipline = Curse, type = Beast, level =Beginner, noIncant = 1, duration = 3 turns,noHigh = 1, noTravel = 1, resist = SPR (Endurance), resistDV = 10, effect ={\bf Werewolf Species spell. This spell can only be learned by werewolves} \\ Release an earsplitting\comma{} supernatural roar which causes all beings within 100m to perform a SPR Resist. Failure causes them to gain the {\it Terrified} status.}
\spell{name = Hunter\apos{}s Mark, school = Divination, discipline = Temporal, type = Instant, level =Beginner, incant = venari, duration = 3 days,noHigh = 1, travel = Semi\minus{}transparent arrow,resist = INT (Perception\comma{} passive), resistDV = CC, effect =Caster is aware of the location of the target for the next 3 days\comma{} or until the mark is removed by magical means. Passive resist nullifies this effect.}
\spell{name = Hypnotic Lights, school = Illusion, discipline = Bewitchment, type = Instant, level =Beginner, incant = fascum, duration = 1 minute,noHigh = 1, travel = Multicoloured Orbs,resist = SPR (Willpower), resistDV = CC, effect =Multicoloured\comma{} iridiescent orbs dance in the sky\comma{} fascinating up to 1d4 creatures that see them\comma{} if they have INT < 8. These creatures cannot remove their gaze from the orbs\comma{} and will stop all other actions for the duration of the hypnotism.}
\spell{name = Identify, school = Divination, discipline = Temporal, type = Instant, level =Beginner, incant = dicemi, noDur = 1, noHigh = 1, travel = Blue rays,noResist =1, effect =Learn the properties of the target: be it learning about the nature of the target\comma{} or the ingredients of a potion.
The more power points dedicated to the spell\comma{} the more information that is revealed.}
\spell{name = Ignite Being, school = Maledictions, discipline = Hex, type = Instant, level =Beginner, incant = bundus, duration = 2 turns,noHigh = 1, noTravel = 1, noResist =1, effect =Set a living target on fire from a distance\comma{} doing (1+PP)d4 fire damage.}
\spell{name = Illuminate Wand, school = Charms, discipline = Elemental, type = Focus, level =Beginner, incant = lumos, noDur = 1, noHigh = 1, noTravel = 1, noResist =1, effect =Causes the tip of your wand to glow\comma{} like a torch. Casts bright light for 2m radius\comma{} and dim light for 10m. Spell last indefinitely\comma{} until Focus is broken\comma{} and does not require extra FP per turn.}
\spell{name = Illusory Construction, school = Illusion, discipline = Bewitchment, type = Focus, level =Adept, incant = lux, noDur = 1, higher = When cast by a character greater than 14th level\comma{} illusion no longer must be silent.,noTravel = 1, resist = INT (perception), resistDV = 14, effect =Create an illusion\comma{} a construction of light. Illusion is silent and non\minus{}corporeal\comma{} but does not disintegrate on contact. Illusion may be manipulated and moved by the caster whilst Focus is maintained\comma{} the illusion becomes fixed after Focus is broken. An observer may determine that the illusion is not real by performing a Resist check as a major action.}
\spell{name = Illusory Disguise, school = Illusion, discipline = Bewitchment, type = Focus, level =Adept, incant = dissimulo, noDur = 1, higher = When cast by a character greater than 8th level\comma{} the caster may add free PP points equal to one quarter of the caster level.,noTravel = 1, noResist =1, effect =Causes the target to take on the exact colour and texture of the background\comma{} making them hard to spot when stationary. 
Stealth checks get a + (4+PP) bonus when stationary\comma{} and + (1+PP) when moving.}
\spell{name = Imbue Bravery, school = Illusion, discipline = Bewitchment, type = Instant, level =Beginner, incant = fortudus, duration = 1 hour,noHigh = 1, travel = Golden rays,noResist =1, effect =Imbue your target with fortitude and vigour. They gain check\minus{}advantage on all Fear\minus{}Resist checks for 1 hour.}
\spell{name = Instil Terror, school = Dark Arts, discipline = Necromancy, type = Instant, level =Beginner, incant = timeant, duration = 4 minutes,noHigh = 1, noTravel = 1, resist = SPR (Endurance), resistDV = CC, effect =Target acquires the {\it Terrified} status. Resist negates effect\comma{} but does 2 Fatigue damage.}
\spell{name = Internal Extension, school = Transfiguration, discipline = Alteration, type = Instant, level =Expert, incant = tarditia poppinia, duration = 3 minutes,noHigh = 1, noTravel = 1, noResist =1, effect =Makes the target container 2 times (+1 for each power point) larger on the inside than it is on the outside\comma{} and divides the total weight by the same factor.}
\spell{name = Inversion Zone, school = Recuperation, discipline = Warding, type = Ward, level =Expert, incant = contrarum, duration = 3 + PP minutes,noHigh = 1, noTravel = 1, noResist =1, effect =This spell creates a permanent warded area inside which all magic performs exactly the opposite to its intended purpose. Healing spells cause harm\comma{} hexes heal and shields amplify the spells passing through them.}
\spell{name = Invert Connection, school = Divination, discipline = Telepathy, type = Instant, level =Master, incant = ruinosus invertus, noDur = 1, noHigh = 1, noTravel = 1, resist = SPR (Endurance), resistDV = 12, effect =As with the {\it Disrupt Connection} spell\comma{} sever the link between two beings such as that caused by {\it Telepathic Bond}\comma{} or by summoning a being. This link is then given to you \minus{}\minus{} giving you access to the telepathic network\comma{} or giving your control over the summoned creature etc.}
\spell{name = Ironmass, school = Transfiguration, discipline = Alteration, type = Instant, level =Adept, incant = ferrus gravitas, duration = 1 hour,noHigh = 1, noTravel = 1, noResist =1, effect =Make the target object so heavy that it cannot be lifted by a single individual.}
\spell{name = Ironwall Ward, school = Recuperation, discipline = Warding, type = Ward, level =Adept, incant = ferromurrum, duration = 1 day,noHigh = 1, noTravel = 1, noResist =1, effect =Forms a shield around the warded area that absorbs (50 + 10$\times$PP) points of damage. The Ironwall is opaque and soundproof\comma{} and is two\minus{}way. Nothing can enter or leave across the threshold of the ward.}
\spell{name = Kill Target, school = Dark Arts, discipline = Necromancy, type = Instant, level =Master, incant = avada kedavra, noDur = 1, noHigh = 1, travel = Green bolt,noResist =1, effect =If the spell makes contact with the target\comma{} kills them instantly. There is no countercurse.}
\spell{name = Knockback, school = Maledictions, discipline = Hex, type = Instant, level =Beginner, incant = flipendo, noDur = 1, higher = An adept level caster may choose to summon a much larger wave\comma{} effecting all targets in a cone 3m in length.,travel = Bue pulse,resist = ATH (Speed), resistDV = CC, effect =A wave of energy strikes into the target\comma{} causing (1+PP)d4 force damage\comma{} and pushing the target backwards up to (1+PP) metres. Resist for half damage.}
\spell{name = Launch Spike, school = Transfiguration, discipline = Conjuration, type = Instant, level =Beginner, incant = voco dens, noDur = 1, higher = An adept\minus{}level caster may use a d10 dice for the damage check.,noTravel = 1, resist = ATH (Speed), resistDV = 10, effect =Conjure (1+PP) enormous spikes to transfigure itself from the surrounding walls/floor\comma{} impaling the target. Each spike does 1d6 piercing damage. Resist for half damage.}
\spell{name = Launder Clothes, school = Charms, discipline = Kinesis, type = Instant, level =Beginner, incant = savatch, noDur = 1, noHigh = 1, travel = Warm glow,noResist =1, effect =Clean and dry the targeted fabrics\comma{} leaving them comfortably warm and smelling faintly of lavender. Can be used on clothes worn by a being\comma{} or on a stack of up to 5 outfits.}
\spell{name = Leapfrog, school = Charms, discipline = Kinesis, type = Instant, level =Adept, incant = raneus, duration = 1 minute,noHigh = 1, noTravel = 1, noResist =1, effect =Target may leap up to (3+PP)m in any direction as a major action\comma{} and land safely whilst the spell is active.}
\spell{name = Lesser Ward, school = Recuperation, discipline = Warding, type = Ward, level =Novice, incant = tueor, duration = 1 day,higher = When cast by a character greater than 10th level\comma{} the ward protection is equal to twice the character level.,noTravel = 1, noResist =1, effect =Erects a ward in a cylinder around an individual. Ward is 20cm larger in radius than the individual is wide\comma{} and 20cm taller. This ward protects you from up to 15 damage of all types\comma{} before it fails. Ward may move with the target\comma{} and can be cast on self.}
\spell{name = Levitation, school = Charms, discipline = Kinesis, type = Focus, level =Beginner, incant = wingardium leviosa, noDur = 1, higher = A character above 6th level may add 1 free PP for every 3 character levels above 3rd.,noTravel = 1, noResist =1, effect =Cause an object of 500g or less to levitate\comma{} controlling the vertical distance at will. 
Each power point dedicated doubles the mass of the object that can be lifted.}
\spell{name = Lightning Bolt, school = Charms, discipline = Elemental, type = Instant, level =Novice, incant = baubilious, noDur = 1, noHigh = 1, travel = Searing\minus{}white lightning,resist = SPR (Health), resistDV = CC, effect =Releases a bolt of lightning from the end of your wand. 
Lightning can initiate fires\comma{} provide electrical current or can be used directly in combat\comma{} where it deals (2+$2\times$PP)d6 electric damage. Targets struck by lightning must succeed in a Resist check\comma{} or be blinded for 2 turns.}
\spell{name = Locate, school = Divination, discipline = Temporal, type = Instant, level =Beginner, incant = locus, noDur = 1, higher = An master\minus{}level clairvoyant may perform a SPR(willpower) check to overcome magical shields blocking this spell\apos{}s effect.,noTravel = 1, resist = INT (Stealth), resistDV = CC, effect =Learn the location of non\minus{}magical objects or an unshielded living being if it is within 1km of the caster. A being may hide from this spell by Resisting.}
\spell{name = Lock, school = Charms, discipline = Kinesis, type = Instant, level =Novice, incant = colloportus, noDur = 1, noHigh = 1, travel = Imperceptible rays,noResist =1, effect =Magically lock a door or chest. Mundane attempts to open the lock fail\comma{} and magical attempts must exceed the casting check of the locking spell.}
\spell{name = Mage Hands, school = Charms, discipline = Kinesis, type = Focus, level =Novice, incant = titillatio, noDur = 1, noHigh = 1, noTravel = 1, noResist =1, effect =The caster produces an ethereal pair of hands that lasts whilst the spell is maintained. You can use your major action to controI the hands. You can use the hand to manipulate an object\comma{} open an unlocked door or container\comma{} stow or retrieve an item from an open container\comma{} or pour the contents out of a vial\comma{} but cannot use them to attack. Hands may be moved at a speed of 2m per turn.}
\spell{name = Magical Detonation, school = Maledictions, discipline = Hex, type = Instant, level =Expert, incant = expulso, noDur = 1, noHigh = 1, noTravel = 1, resist = POW, resistDV = 10, effect =Launches a magical bolt at the target which\comma{} if it makes contact\comma{} causes the object to violently tear itself apart\comma{} doing  $4\times$(1+(1+PP)d6) force damage. Resist for half damage.}
\spell{name = Magical Shield, school = Recuperation, discipline = Warding, type = Focus, level =Beginner, incant = protego, noDur = 1, noHigh = 1, travel = Etheral Shield,noResist =1, effect =Erects an ethereal shield from your in front of you that absorbs incoming magical attacks.

Casting this spell initiates a `Brace’ action and\comma{} in addition to the Resist check\comma{} adds (1d4 +PP) to your Block stat against magical attacks. 

This shield has a health of (10 +$5\times$PP). If a spell would cause the shield to drop to 0HP\comma{} the shield fails\comma{} and half the remaining damage is dealt to the caster.}
\spell{name = Magical Stability Ward, school = Recuperation, discipline = Warding, type = Ward, level =Master, incant = victoria maximus, duration = 5 minutes,noHigh = 1, noTravel = 1, noResist =1, effect =Creates a region where magic is safer and more successful: all spell checks in the warded area get check double\minus{}advantage. Radius of ward is (4 + PP) metres.}
\spell{name = Major Healing, school = Recuperation, discipline = Healing, type = Instant, level =Expert, incant = sana, noDur = 1, noHigh = 1, travel = Yellow\minus{}white rays,noResist =1, effect =Heals the target of all status effects such as burns\comma{} frostbite\comma{} poisons and diseases\comma{} regardless of severity.
Restores HP equal to 8 + three times the total caster level.}
\spell{name = Major Ward, school = Recuperation, discipline = Warding, type = Ward, level =Expert, incant = tueormaxima, duration = 2 days,noHigh = 1, noTravel = 1, noResist =1, effect =Individual Ward (see Lesser Ward) that protects against (50+5$\times$PP) damage.}
\spell{name = Mark Surface, school = Charms, discipline = Kinesis, type = Focus, level =Beginner, incant = stylum, noDur = 1, higher = When cast by an adept\minus{}level caster\comma{} the distance between the tip of the wand and the writing surface increases to half the character level.,noTravel = 1, noResist =1, effect =Use your wand as anything from a thin marker to a thick paintbrush. The tip of the {\it brush} can be up to 0.5m away from the tip of your wand\comma{} but follows the motion of your wand exactly. The {\it paint} is a magical adhesive that sticks to any surface\comma{} and may be of any colour you choose.}
\spell{name = Mass Delusion, school = Illusion, discipline = Psionics, type = Instant, level =Master, incant = falasarium maxima, duration = (8 + 8$\times$PP) hours,noHigh = 1, noTravel = 1, resist = INT (perception), resistDV = 12, effect =Apply the {\it Delusion} spell to 2d6 targets of your choice. The delusion is the same to all targets.}
\spell{name = Mass Kinesis, school = Charms, discipline = Kinesis, type = Focus, level =Master, incant = ballatutti, noDur = 1, noHigh = 1, noTravel = 1, noResist =1, effect =Control huge numbers of objects as they levitate and move around: write a thousand books with a thousand quills\comma{} or conduct a swordfight with multiple blades at once. Can only use the objects if you would normally be able to use them without magic. You may only perform 4 unique actions with the objects\comma{} but you may duplicate those exact actions an arbitrary number of times in a 10m radius. For example\comma{} you could only copy out 4 books at a time\comma{} as each book requires a unique action\comma{} but you can copy the same book out as many times as you like\comma{} as the action is identical.}
\spell{name = Mass Suggestion, school = Illusion, discipline = Bewitchment, type = Instant, level =Master, incant = faciite maxima, duration = (8 + 8$\times$PP) hours,noHigh = 1, noTravel = 1, resist = SPR (Willpower), resistDV = 12, effect =Apply the {\it Suggestion} spell to 2d6 targets of your choice. The suggestion is the same to all targets.}
\spell{name = Mend Bones, school = Recuperation, discipline = Healing, type = Instant, level =Adept, incant = ossium emendo, noDur = 1, higher = An expert level caster may use a d20 for the healing check.,travel = Yellow\minus{}white rays,noResist =1, effect =Mends bones and other serious physical ailments. Heals for (1 + PP)d10  health points\comma{} and removes the \textit{Major Injury} status effect.}
\spell{name = Mental Burden, school = Maledictions, discipline = Curse, type = Instant, level =Novice, incant = onus, duration = 1 turns,higher = When cast by a character higher than 10th level\comma{} spells cost 8FP more than their stated value\comma{} and the Resist DV increases to 12.,noTravel = 1, resist = POW, resistDV = 8, effect =If the target fails to Resist\comma{} all spells cost 2FP more than their stated value whilst the spell lasts.}
\spell{name = Meteor Strike, school = Maledictions, discipline = Hex, type = Instant, level =Expert, incant = bothynus, duration = 2 turns,noHigh = 1, noTravel = 1, noResist =1, effect =Summon flaming rocks from the heavens\comma{} doing (3+PP)d8 bludgeoning damage\comma{} and (3+PP)d8 fire damage to all enemies in a 10m radius.}
\spell{name = Minefield Ward, school = Recuperation, discipline = Warding, type = Ward, level =Adept, incant = denarlium, duration = 1 week,noHigh = 1, noTravel = 1, resist = INT (Perception), resistDV = 10, effect =Lay magical {\it mines} in a 15m radius\comma{} with a 5m radius gap at the centre. You may designate a single safe route through the minefield (a path of width 0.5m). If a being touches any part of the minefield other than the path\comma{} the mines explode doinig (1+PP)d20 damage of a type of the caster\apos{}s choosing. Each subsequent metre travelled triggers another explosion. Explosions may be Resisted for half damage.}
\spell{name = Minor Healing, school = Recuperation, discipline = Healing, type = Focus, level =Beginner, incant = enervate, noDur = 1, noHigh = 1, travel = Yellow\minus{}white rays,noResist =1, effect =Heal for 2 points per turn. 
If the target has a serious wound\comma{} i.e. a broken bone\comma{} cannot heal beyond 50\% health. Only works on living creatures.}
\spell{name = Mirror Shield, school = Recuperation, discipline = Warding, type = Focus, level =Adept, incant = repente, noDur = 1, noHigh = 1, travel = Etheral Shield,noResist =1, effect =Erects an ethereal shield from your in front of you that absorbs incoming attacks of all kinds.

Casting this spell initiates a `Brace’ action and\comma{} in addition to the Resist check\comma{} adds (1d4 +PP) to your Block stat against magical and physical attacks. 

This shield has a health of (10 +$5\times$PP). If a spell would cause the shield to drop to 0HP\comma{} the shield fails\comma{} and half the remaining damage is dealt to the caster. 

On a successful block of a spell that the caster knows\comma{} they may perform an accuracy check to `reflect\apos{} the spell back at the original caster. This costs no additional FP to do\comma{} and on a failed attempt\comma{} the spell is reflected in a random direction. Reflected spells still drain the shield\apos{}s HP.}
\spell{name = Mists of Time, school = Divination, discipline = Temporal, type = Ritual (1 hour), level =Expert, incant = momento aeternitatis, noDur = 1, noHigh = 1, noTravel = 1, noResist =1, effect =Enter into a trance\comma{} whereby you can observe the past or the future\comma{} to uncover what was\comma{} or what will be at either your present location\comma{} or to a specific individual. 
You may observe up to (1+PP) day into the future\comma{} or (1+PP) year into the past.}
\spell{name = Modify Memory, school = Illusion, discipline = Psionics, type = Instant, level =Master, incant = obliviate, noDur = 1, noHigh = 1, noTravel = 1, resist = SPR (Willpower), resistDV = CC, effect =If target fails a Resist SPR(willpower) check\comma{} you may modify the memories of the target\comma{} even causing them to forget skills and spells that they currently know.}
\spell{name = Necrosis, school = Dark Arts, discipline = Necromancy, type = Instant, level =Novice, incant = carnes mortis, noDur = 1, noHigh = 1, travel = Sickly\minus{}green bolt,noResist =1, effect =Do 5+ (1+PP)d12 necrotic damage.}
\spell{name = Night Vision, school = Illusion, discipline = Bewitchment, type = Instant, level =Beginner, incant = aspectu, duration = 2 hours,noHigh = 1, noTravel = 1, noResist =1, effect =Give the target nightvision for one hour: dim light is as bright as daylight\comma{} and darkness is consdiered dim.}
\spell{name = Obfuscation, school = Divination, discipline = Telepathy, type = Ritual (1 hour), level =Novice, incant = obscuras, duration = 1 week,noHigh = 1, noTravel = 1, noResist =1, effect =All attempts to identify\comma{} locate\comma{} scry on\comma{} or otherwise detect the target using magical means fail.}
\spell{name = Object Swarm, school = Maledictions, discipline = Hex, type = Focus, level =Adept, incant = oppugno, noDur = 1, noHigh = 1, noTravel = 1, resist = FIN (Speed), resistDV = 10, effect =Causes (3+3$\times$PP) nearby objects to hurl themselves at the target. Each object does 1d4 bludgeoning damage. Target may perform a Resist check for each object\comma{} reducing the damage from that object by half.}
\spell{name = Occlumency, school = Divination, discipline = Telepathy, type = Ritual (5 minutes), level =Adept, incant = occlumens, duration = 1 day,noHigh = 1, noTravel = 1, noResist =1, effect =Set up barriers around your mind to defend yourself. 
Legilimency will not work on you\comma{} and all other mind\minus{}altering spells take a casting penalty equal to one third of your total level.}
\spell{name = Patronus Charm, school = Recuperation, discipline = Healing, type = Focus, level =Expert, incant = expecto patronus, noDur = 1, higher = When cast by a character higher than 15th level\comma{} the patronus takes corporeal form\comma{} and may attack Unlife directly\comma{} doing 5d8 Holy damage.,noTravel = 1, noResist =1, effect =Summon your greatest\comma{} happiest memories into physical form: your patronus. The patronus will prevent any Un\minus{}Life creatures from approaching you for the duration of the spell.}
\spell{name = Perpetual Hunger, school = Maledictions, discipline = Curse, type = Instant, level =Adept, incant = inedia, duration = 1 minutes,higher = When cast by a character higher than 14th level\comma{} spell lasts for 20 minutes.,noTravel = 1, resist = ATH (Health), resistDV = CC, effect =The afflicted feels perpetual\comma{} soul\minus{}sapping hunger. Every minute (20 turns) where at least two mouthfuls of food is not consumed\comma{} suffer necrotic damage equal to the number of minutes since food was last consumed\comma{} until the spell effect ends. Target may perform a Resist check every 3 turns\comma{} to end the effect.}
\spell{name = Piercing Wail, school = Illusion, discipline = Psionics, type = Instant, level =Beginner, incant = magnus surgerus, noDur = 1, noHigh = 1, noTravel = 1, noResist =1, effect =All targets in a 3m spherical radius of the caster take 2 points of psychic damage (+3 per PP)\comma{} and awaken if they are sleeping.}
\spell{name = Piper{\apos}s Illusion, school = Illusion, discipline = Psionics, type = Music (5 turns), level =Beginner, noIncant = 1, noDur = 1, higher = When cast by an Expert\minus{}level caster\comma{} ritual only takes 2 turns to complete.,noTravel = 1, resist = SPR (endurance), resistDV = CC, effect =This spell is performed by playing an instrument and layering it with magic. All those who hear ithe song are hypnotised if they fail a Resist check. Hyponotised individuals cannot take any actions. When the spell ends\comma{} all entranced targets take (1+PP)d10 psychic damage.
This spell is not blocked by non\minus{}specialist wards or shields.}
\spell{name = Plague of Insects, school = Dark Arts, discipline = Necromancy, type = Instant, level =Adept, incant = prorepere, duration = 5 minutes,noHigh = 1, noTravel = 1, noResist =1, effect =Summon a swarm of insects from the ground. Insect plague covers area of 2m radius\comma{} doubling with each PP (max 32 metres). All targets in radius must perform an evasion check\comma{} or take 1d4 poison damage and 1d4 piercing damage until they escape the area.}
\spell{name = Planemeld, school = Divination, discipline = Temporal, type = Ritual (1 hour), level =Master, incant = cogitosum, duration = 1 hour,noHigh = 1, noTravel = 1, noResist =1, effect =By entering into a deep trance for 1 hour\comma{} you may bring yourself into resonance with a higher power. At any point in the next 24 hours\comma{} you may use a major action to channel these energies into a warded region that surrounds you in a cylinder (10+$3\times$PP) metres in radius. The energies of the plane infuse this warded region\comma{} as if the dimension had merged into the normal one. The caster may choose if they are affected by the planemeld at first\comma{} but once they exit the region\comma{} they will feel its effects when they re\minus{}enter. You may choose from one of the planes found on page \pageref{S:Planes}\comma{} wich also details the planemeld effects for each plane.}
\spell{name = Planewalk, school = Divination, discipline = Temporal, type = Ritual (2 minutes), level =Master, incant = ambulo mundus, noDur = 1, higher = When cast by a character greater than 18th level\comma{} you may use this spell at the site of a {\it Planemeld} spell to travel to that plane of existence.,noTravel = 1, noResist =1, effect =By carefully preparing every atom in your body\comma{} you may slip effortlessly between this world and the Astral Realm\comma{} without the need for a portal. Your entire body enters into the astral realm\comma{} where you may perceive things in both the Mortal World\comma{} and the Astral Realm\comma{} but you may only interact with the astral realm. Cast this spell again to cross back over.}
\spell{name = Potion Mixing Spell, school = Transfiguration, discipline = Alteration, type = Ritual(5 turns), level =Beginner, noIncant = 1, noDur = 1, noHigh = 1, noTravel = 1, noResist =1, effect =Used to mix a potion. See page \pageref{S:Enchanting} for details.}
\spell{name = Preserve Object, school = Transfiguration, discipline = Alteration, type = Instant, level =Beginner, incant = tempocessus, duration = 1 days,noHigh = 1, travel = Silver rays,noResist =1, effect =The target is unaffected by the flow of time for the duration of the spell\comma{} and does not rot or otherwise decay.}
\spell{name = Prevent Movement, school = Maledictions, discipline = Curse, type = Focus, level =Novice, incant = impedimentia, duration = 3 turns,noHigh = 1, travel = Red bolt,noResist =1, effect =Target performs a resist magic check against the casting check\comma{} if it fails\comma{} target acquires the Trapped status effect. Arms are still free to move\comma{} and target can still speak.}
\spell{name = Privacy Ward, school = Recuperation, discipline = Warding, type = Ward, level =Beginner, incant = muffliato, duration = 1 hour,noHigh = 1, noTravel = 1, noResist =1, effect =Prevents sound from inside a region (2+PP)m in radius being heard from the outside. When inside the region\comma{} sound from both inside and outside may be heard.}
\spell{name = Psychosomatism, school = Illusion, discipline = Psionics, type = Focus, level =Expert, incant = animo materia, noDur = 1, noHigh = 1, noTravel = 1, resist = EMP (perception), resistDV = CC, effect =You produce an illusion not out of light\comma{} but in the mind of the target. If the target fails to resist\comma{} they see in their mind whatever the caster wishes\comma{} and react accordingly. No actual HP or FP is removed by the illusions\comma{} but the character acts as if they have.}
\spell{name = Receive Omen, school = Divination, discipline = Temporal, type = Ritual (3 turns), level =Beginner, noIncant = 1, noDur = 1, noHigh = 1, noTravel = 1, noResist =1, effect =Use your tea leaves to receive an omen about the future. Ask a question about the outcome of an event. The tea leaves will tell you if the outcome is positive\comma{} negative\comma{} or neutral. Takes 4 minutes to cast.}
\spell{name = Recurring Light, school = Maledictions, discipline = Hex, type = Focus, level =Adept, incant = catena, noDur = 1, higher = An expert\minus{}level caster may choose to use Celestial damage\comma{} rather than fire damage.,travel = Searing white beam,resist = INT (Perception), resistDV = CC, effect =A beam of blinding light shoots from your wand in a line up to 8m long\comma{} striking one target before moving onto the next. Targets take (2+PP)d6 of fire damage and are Blinded if they fail to Resist. Each target has the chance to avoid/counterspell this spell\comma{} the next target only receives the beam If the previous one was hit. A maximum of (3+PP) targets may be hit.}
\spell{name = Reinforce Shield, school = Recuperation, discipline = Warding, type = Focus, level =Beginner, incant = praesidium, noDur = 1, higher = When cast by an expert\minus{}level caster\comma{} you may restore a shield to 150\% of its original strength.,travel = Brick\minus{}red rays,noResist =1, effect =Restore the strength of a target shield or magical ward by (2+PP) points per turn that this spell is maintained. Cannot restore the strength to more than the original level.}
\spell{name = Release Trapped Being, school = Recuperation, discipline = Healing, type = Instant, level =Novice, incant = relashio, noDur = 1, noHigh = 1, travel = White flash,resist = ATH (Strength), resistDV = CC + 2 $\times$ PP, effect =Force physical objects and beings to release the target \comma{} and remove all impediments to moving. Does not effect magical immobiility. Resist nullifies this effect.}
\spell{name = Relive Memory, school = Illusion, discipline = Psionics, type = Instant, level =Expert, incant = legilimens, noDur = 1, noHigh = 1, travel = Green bolt,resist = SPR (endurance), resistDV = 10 + PP, effect =Target performs a resist magic check\comma{} if it fails\comma{} the caster forces the target to relive a specific memory\comma{} which they may also view.}
\spell{name = Replay Spell, school = Divination, discipline = Temporal, type = Instant, level =Beginner, incant = priori incantatem, noDur = 1, noHigh = 1, noTravel = 1, noResist =1, effect =Ghostly images of the last (2+PP) spells cast by a target wand appear\comma{} informing the caster of the target and time of the casting.}
\spell{name = Runic Shield, school = Recuperation, discipline = Warding, type = Instant, level =Novice, incant = scutum, duration = 1 hour,noHigh = 1, travel = Glowing rune,noResist =1, effect =Choose a Damage Type. Target is 10\% resistant to that damage type (+10\% for each PP) for the duration of the spell.}
\spell{name = Scramble Abilities, school = Maledictions, discipline = Curse, type = Focus, level =Adept, incant = traferus, duration = 5 + PP turns,noHigh = 1, noTravel = 1, resist = SPR (Willpower), resistDV = CC, effect =The target has their abilities scrambled for the duration of the curse if they fail to Resist. The GM randomly reassigns the character attributes.}
\spell{name = Sculpt Matter, school = Transfiguration, discipline = Alteration, type = Focus, level =Adept, incant = perseids, noDur = 1, noHigh = 1, noTravel = 1, noResist =1, effect =Sculpt a target solid object with your mind\comma{} as if it were made of soft clay. The total mass of the object must remain constant\comma{} but you can shift and scult the matter at will.}
\spell{name = Sense Humans, school = Divination, discipline = Telepathy, type = Instant, level =Adept, incant = revelio, noDur = 1, higher = When cast by a caster greater than 15th level\comma{} you also learn the name of the humans.,noTravel = 1, noResist =1, effect =Reveals the presence of humanoid life nearby. The caster gets a snapshot of the distance and direction to every humoid being within range. 
Radius of spell is (4+PP) metres.}
\spell{name = Sense Traps, school = Divination, discipline = Telepathy, type = Instant, level =Beginner, incant = antidolus, noDur = 1, noHigh = 1, noTravel = 1, noResist =1, effect =Attempt to discover any traps in your immediate vicinity. If successful\comma{} you may learn the location of the trap\comma{} and the trigger (but not the effect). Success conditions are set by the GM.}
\spell{name = Shadow Blast, school = Dark Arts, discipline = Necromancy, type = Instant, level =Beginner, incant = malusangui, noDur = 1, noHigh = 1, travel = Black bolt,noResist =1, effect =Hurl shadows at you enemy\comma{} dealing 1d2 necrotic damage per per character level\comma{} plus an additional 1d2 for every PP dedicated.}
\spell{name = Shadow Demon, school = Dark Arts, discipline = Occultism, type = Instant, level =Adept, incant = viven umbrafors, duration = (3+PP) turns,noHigh = 1, noTravel = 1, noResist =1, effect =Bring the very shadows to life: a being of pure darkness will stalk your enemies\comma{} attacking them whenever they stray near the shadows\comma{} doing (2+PP)d10 worth of necrotic damage.}
\spell{name = Shadowsight, school = Dark Arts, discipline = Occultism, type = Focus, level =Novice, incant = ivertus, noDur = 1, noHigh = 1, travel = Eyes glow white,noResist =1, effect =Invert your vision \minus{}\minus{} pure darkness is considered bright light\comma{} and bright light is considered pure darkness for as long as the spell is maintained.}
\spell{name = Shatter, school = Charms, discipline = Kinesis, type = Focus, level =Adept, incant = tootanus focum, noDur = 1, higher = When cast by a Master\minus{}level caster\comma{} you can also effect objects made of stone up to 200kg in weight.,noTravel = 1, resist = ATH (Health), resistDV = 10, effect =Focus an ultrasonic vibration into a single target object or being made of crystal\comma{} class\comma{} ceramic or porcelain\comma{} and cause it to break. The tip of your wand must touch the target for the duration of the spell\comma{} and the spell gets stronger the longer it is maintained. In the first turn shatters objects 5kg or lighter\comma{} and then doubles every subsequent turn.}
\spell{name = Shatter Illusions, school = Illusion, discipline = Psionics, type = Instant, level =Adept, incant = conlidus, noDur = 1, noHigh = 1, travel = Orange rays,noResist =1, effect =Remove all illusion spells from the target\comma{} if the casting check exceeds the casting check of the most poweful illusion.}
\spell{name = Shatterblast, school = Charms, discipline = Kinesis, type = Instant, level =Adept, incant = tootanus, noDur = 1, higher = When cast by a Master\minus{}level caster\comma{} you can also effect objects made of stone up to 10kg in weight.,travel = Shockwave,resist = ATH (Health), resistDV = 10, effect =Release a shockwave of sonic energy in a radius (1+PP)m\comma{} which causes all brittle objects to shatter. All objects made of crystal\comma{} glass\comma{} ceramic or porcelain are shattered into many hundreds of pieces unless they weigh more than your Character level (in kg). Crystalline entities take (2+PP)d6 concussive damage.}
\spell{name = Shield Breaker, school = Maledictions, discipline = Curse, type = Instant, level =Expert, incant = misericorde, duration = 2 turns,noHigh = 1, noTravel = 1, noResist =1, effect =Finds the weak point in the armour\comma{} and exploits it: if the casting check exceeds the AC/HP of the weakest defensive spell active on the target\comma{} all AC (both magical and physical) is set to 0 for two turns. Physical AC is restored at the end of this period.}
\spell{name = Shimmering Confetti, school = Transfiguration, discipline = Conjuration, type = Instant, level =Beginner, incant = chamak, duration = 3 seconds,noHigh = 1, travel = Golden particles,noResist =1, effect =Conjures a shower of golden\comma{} shimmering particles to cover every person and surface in a (2+PP)m radius. Creatures with an INT < 9 become distracted and vulnerable to critical strikes for one turn.}
\spell{name = Shockwave, school = Maledictions, discipline = Hex, type = Instant, level =Expert, incant = inpusla, noDur = 1, noHigh = 1, noTravel = 1, resist = ATH (Perception), resistDV = , effect =A shockwave emanates from the caster in every direction\comma{} for a radius of (3+PP)m\comma{} doing 8d8 concussive damage and hurling all unprotected away from the caster to the edge of the affected region. Resist for half damage.}
\spell{name = Shroud of Darkness, school = Dark Arts, discipline = Occultism, type = Instant, level =Beginner, incant = tenebrosa, duration = 2 minutes,noHigh = 1, noTravel = 1, noResist =1, effect =Extinguish all light within a (10 + 2$\times$PP) metre radius\comma{} and all attempts to create new light fail\comma{} unless caster\apos{} passive POW check exceds the casting check.}
\spell{name = Silence, school = Illusion, discipline = Psionics, type = Instant, level =Novice, incant = silencio, duration = (2+2$\times$PP) turns,higher = A master\minus{}level caster may cast this spell on 1d4 targets within range.,travel = Orange bolt,resist = SPR (Willpower), resistDV = , effect =If the target fails to Resist\comma{} they may not speak or otherwise vocalise for the duration of the spell.}
\spell{name = Silver Shield, school = Transfiguration, discipline = Conjuration, type = Instant, level =Beginner, incant = argentipus, duration = 1 hour,higher = When cast by a character above 10th level\comma{} the shield no longer degrades with each strike\comma{} and instead acts as a normal shield with an AC equal to 15 + 2$\times$PP.,travel = Silver Mist,noResist =1, effect =Conjures a floating silver shield from thin air\comma{} to defend you. Shield absorbs both physical and magical attacks for up to (10+2$\times$PP) damage points\comma{} before breaking. The caster has limited control over the shield whilst it is active\comma{} using a major action to move it up to 3m in any direction or a minor action to move it to face a different direction whilst the caster remains stationary.}
\spell{name = Sleep, school = Illusion, discipline = Bewitchment, type = Instant, level =Adept, incant = somnus, duration = (3 + 2$\times$PP) turns,noHigh = 1, noTravel = 1, resist = SPR (endurance), resistDV = CC, effect =If target fails to resist\comma{} they enter into a deep slumber for (5 + 2 $\times$ PP) turns}
\spell{name = Slip, school = Transfiguration, discipline = Alteration, type = Focus, level =Beginner, incant = glisser, noDur = 1, higher = When cast by an expert\minus{}level caster\comma{} this spell also makes staircases transform into greased chutes.,noTravel = 1, resist = FIN (Dexterity), resistDV = CC, effect =Whilst Focus is maintained\comma{} up to 1 square metre of the targeted surface becomes slippery\comma{} as if it was covered in grease. When a target touches the effected surface and fails the resist check\comma{} they fall over/ drop the item as appropriate.}
\spell{name = Smokescreen, school = Charms, discipline = Elemental, type = Instant, level =Adept, incant = fumus insterio, noDur = 1, noHigh = 1, travel = 2 minutes,noResist =1, effect =Thick white smoke issues from the end of your wand\comma{} filling a sphere 10m in radius\comma{} giving a Severe obscuration for all targets within range. In a confined area\comma{} duration is doubled.}
\spell{name = Soul Snare, school = Dark Arts, discipline = Necromancy, type = Instant, level =Master, incant = nerco decipula, noDur = 1, noHigh = 1, travel = Black aura,noResist =1, effect =Capture the soul of a recently killed enemy. 
This soul may be used to instantly cast any other spell without a casting check or fortitude cost\comma{} or alternatively; absorbed to heal the character to full health and fortitude. 
Only one soul may be trapped at any given\comma{} and no power points may be dedicated to the instant\minus{}casting.}
\spell{name = Spare the Wounded, school = Recuperation, discipline = Healing, type = Instant, level =Adept, incant = clementia, duration = 1 day,noHigh = 1, noTravel = 1, resist = EVL, resistDV = 10 + PP, effect =If the target falls below 5HP\comma{} they are considered a non\minus{}combatant and will not be targeted by beings which fail to Resist. This spell is negated (even before effect is triggered) if target engages in hostile activity.}
\spell{name = Spark, school = Charms, discipline = Elemental, type = Instant, level =Beginner, incant = electrum, noDur = 1, noHigh = 1, noTravel = 1, noResist =1, effect =Charge the tip of your wand with electrical energy. This energy is discharged when your wand\minus{}tip next touches a surface. Does 2 electrical damage on contact\comma{} and also fries any electrical equipment it comes into contact with.}
\spell{name = Speak in Tongues, school = Divination, discipline = Telepathy, type = Ritual (5 minutes), level =Beginner, incant = lingua maxima, duration = 4 minutes,noHigh = 1, noTravel = 1, noResist =1, effect =By meditating for 5 minutes\comma{} you may understand and speak the language of a willing target individual. Target must be a sapient being\comma{} or otherwise able to speak at least one language.}
\spell{name = Spider Hands, school = Charms, discipline = Kinesis, type = Instant, level =Adept, incant = aranerum fiducia, duration = 5 minutes,noHigh = 1, noTravel = 1, noResist =1, effect =Imbue the target with the ability to traverse up vertical walls using their hands and feet. Climbing movement checks are half the speed of a regular movement check.}
\spell{name = Stabilise Patient, school = Recuperation, discipline = Healing, type = Instant, level =Novice, incant = firmum, noDur = 1, noHigh = 1, travel = Yellow\minus{}white rays,noResist =1, effect =Stabilises the patient and removes the \textit{Critical Condition} status.}
\spell{name = Steelclaw, school = Transfiguration, discipline = Alteration, type = Instant, level =Beginner, incant = ferscabere, duration = 1 day,noHigh = 1, noTravel = 1, noResist =1, effect =Transfigures an animal{\apos}s claws into large steel talons\comma{} increasing their physical damage by (3 + PPd6)}
\spell{name = Stick, school = Charms, discipline = Kinesis, type = Instant, level =Novice, incant = obharesco, noDur = 1, noHigh = 1, travel = Purple flash,noResist =1, effect =Stick two objects together\comma{} as if you had fused them together at a molecular level. To break them apart requires either slicing the objects apart\comma{} or pulling them hard enough to break one (or both) of the objects.}
\spell{name = Sting, school = Maledictions, discipline = Hex, type = Instant, level =Beginner, incant = ictus, noDur = 1, noHigh = 1, travel = Green dart,noResist =1, effect =Stings the target for (2+2$\times $PP)d2 poison damage.}
\spell{name = Stoneskin, school = Transfiguration, discipline = Alteration, type = Instant, level =Novice, incant = lapis pellium, duration = 5 minutes,noHigh = 1, travel = Dark green rays,noResist =1, effect =Increase the target{\apos} AC by 1 + \cvdv by transfiguring their skin into solid stone. Spells such as `shatter’ end this effect immediately.}
\spell{name = Stopping Shield, school = Recuperation, discipline = Warding, type = Focus, level =Novice, incant = stabit vallio, noDur = 1, noHigh = 1, travel = Invisible ripple,noResist =1, effect =Erects a 1m radius shield in front of the caster\comma{} which halts any physical object that touches it. Objects in flight drop to the ground\comma{} as if the {\it Halt} spell had been cast on them.}
\spell{name = Strangle, school = Maledictions, discipline = Curse, type = Instant, level =Novice, incant = offoco, noDur = 1, higher = When cast by a character higher than 15th level\comma{} the Resist DV is equal to the character level.,travel = Grey bolt,resist = ATH (Health), resistDV = 10, effect =Target must Resist at the beginning of every turn until they succeed. Until then\comma{} they are deprived of oxygen\comma{} cannot speak\comma{} and after 6 turns\comma{} cannot take any other actions\comma{} and eventually succumb to hypoxia under the usual rules.}
\spell{name = Stunning Blast, school = Maledictions, discipline = Curse, type = Instant, level =Novice, incant = stupefy, duration = 5 turns,higher = When cast by an adept level caster\comma{} gain `free’ PP equal to half your caster level .,travel = Scarlet bolt,resist = SPR, resistDV = 6+PP, effect =The target is Stunned for 5 turns. Stunned characters cannot move or speak\comma{} but may take a major action to perform a Resist check to end the spell.}
\spell{name = Suggestion, school = Illusion, discipline = Bewitchment, type = Instant, level =Adept, incant = facite, duration = (2 + 2$\times$PP) hours,noHigh = 1, noTravel = 1, resist = SPR (Willpower), resistDV = 12, effect =Make a suggestion to a target within hearing range. The suggestion must be reasonable (i.e. no stabbing themselves) and limited to a single sentence. If target fails to resist\comma{} they must obey this suggestion for up to (2 + 2 $\times$ PP) hours.}
\spell{name = Summon Avatar, school = Transfiguration, discipline = Conjuration, type = Ritual (5 minutes), level =Expert, incant = elementos temporio, duration = (3 + 2$\times$ PP) minutes,noHigh = 1, noTravel = 1, noResist =1, effect =Summon a Capable Avatar of your choice (Storm\comma{} Ice or Fire) to be under your command for the duration of the spell\comma{} after which it dissolves.}
\spell{name = Summon Bat Bogeys, school = Maledictions, discipline = Hex, type = Instant, level =Novice, incant = vespernasum, duration = 3 turns,noHigh = 1, travel = Orange bolt,noResist =1, effect =Causes the mucus in the target{\apos}s nose to gain sentience\comma{} take the form of a (1+$2\times$PP) small bats\comma{} and attack the target. 
Each bat\minus{}bogey does 1d6 points of acid damage per turn.}
\spell{name = Summon Birds, school = Transfiguration, discipline = Conjuration, type = Focus, level =Adept, incant = avis, duration = 1 minutes,higher = A character above 9th level may add 1 free PP for every 3 character levels above 6th.,travel = Blue bolt,noResist =1, effect =The magical bolt breaks apart into a flock of (6+$4\times$PP) small birds\comma{} which do your bidding. Each bird has 3HP and can do 1d4 of piercing damage. The birds will follow the orders of the caster as long as Focus is maintained. When Focus is broken\comma{} the birds continue with their last order.}
\spell{name = Summon Daggers, school = Transfiguration, discipline = Conjuration, type = Instant, level =Expert, incant = fumus defendus, noDur = 1, noHigh = 1, travel = Black smoke,resist = ATH (Speed), resistDV = 10, effect =Causes (15+$5\times$PP) daggers to coalesce out of smoke\comma{} and fly towards the target. 
Each dagger that hits the target does 1d4 piercing damage\comma{} a successful Resist halves the damage done.}
\spell{name = Summon Object, school = Charms, discipline = Kinesis, type = Focus, level =Novice, incant = accio, noDur = 1, higher = If the caster exceeds 11th level\comma{} may cast this spell as an Instant spell.,noTravel = 1, noResist =1, effect =Summon non\minus{}shielded objects within a 500m radius. They will fly to your current position at a speed of 100m per cycle as long as Focus is maintained. Objects must be light enough that the caster could reasonably pick it up.}
\spell{name = Summon Snake, school = Transfiguration, discipline = Conjuration, type = Instant, level =Novice, incant = serpensortia, duration = 1 minutes,higher = When cast by an expert\minus{}level caster\comma{} may summon 1d4 snakes.,noTravel = 1, noResist =1, effect =Summons a venomous snake out of the tip of the caster{\apos}s wand. The snake has (8+PP)HP and does (1+PP)d6 poison damage upon biting.}
\spell{name = Summon Void, school = Dark Arts, discipline = Occultism, type = Focus, level =Adept, incant = inanis, duration = 1 minute,noHigh = 1, noTravel = 1, resist = ATH, resistDV = Half the caster level, effect =Summon a true Void anywhere within 15m of your current position\comma{} a gap in the fabric of reality that attracts all objects within a 5m radius. Everything in radius must perform an ATH Resist check to grab onto something. Objects sucked into the Void have a 25\% chance to remain there\comma{} and a75\% chance to be randomly teleported anywhere in the multiverse.\comma{} after taking 4d8 cold damage.}
\spell{name = Summoning Circle, school = Dark Arts, discipline = Occultism, type = Ward, level =Expert, noIncant = 1, noDur = 1, noHigh = 1, noTravel = 1, noResist =1, effect =By gathering together a group (only one of whom needs to be able to cast this spell)\comma{} you create a special altar from which conjuration spells are especially powerful. The radius of this region is 2m\comma{} and whilst inside it\comma{} you gain a bonus to Conjuration casting checks equal to the number of people who created the summoning circle.}
\spell{name = Sunburst, school = Recuperation, discipline = Healing, type = Instant, level =Beginner, incant = sol maxima, noDur = 1, noHigh = 1, travel = Searing\minus{}white bolt,noResist =1, effect =A bolt of magic explodes on contact with a solid {\it or} astral object\comma{} releasing a searing white light that does (2+PP)d6 Holy Damage.}
\spell{name = Suppress Intelligence, school = Illusion, discipline = Psionics, type = Instant, level =Adept, incant = romanes, duration = 2 minutes,noHigh = 1, noTravel = 1, resist = INT, resistDV = CC, effect =By touching your wand\minus{}tip to the head of the target\comma{} reduce their INT attribute by (2+PP) points for the duration of the spell.}
\spell{name = Telepathic Bond, school = Divination, discipline = Telepathy, type = Ritual (2 turns), level =Beginner, incant = conanimus, duration = 2 days,noHigh = 1, noTravel = 1, noResist =1, effect =Form a mental connection between your mind and the mind of a willing target. You may then use this connection to communicate silently. Target must be within touching distance when the spell is cast\comma{} but the bond has no distance limit after that.}
\spell{name = Teleport, school = Charms, discipline = Kinesis, type = Instant, level =Expert, incant = cruratele, noDur = 1, noHigh = 1, travel = Pink rays,noResist =1, effect =You may send a non\minus{}living object to anywhere that you have previously visited. Spell failure still teleports the object\comma{} but to an unknown location.}
\spell{name = Thick Air, school = Transfiguration, discipline = Alteration, type = Focus, level =Novice, incant = temporio, duration = 1 minute,noHigh = 1, travel = Imperceptible ripple,resist = ATH (Strength), resistDV = 12, effect =Transforms the air around the target into a thick soup\comma{} slowing their movement by (20+10 $\times$ PP) \%. Resist for half the speed reduction.}
\spell{name = Thought Extractor, school = Divination, discipline = Telepathy, type = Focus, level =Beginner, noIncant = 1, noDur = 1, noHigh = 1, travel = Silver strings,noResist =1, effect =Allows the caster to extract a specific memory from their minds\comma{} for subsequent storage\comma{} either in a glass vial\comma{} or in a pensieve. Memories that have been extracted through this method cannot be viewed by legilimency.}
\spell{name = Threshold Ward, school = Recuperation, discipline = Warding, type = Ward, level =Adept, incant = desino, duration = 1 year,noHigh = 1, noTravel = 1, noResist =1, effect =Prevents objects from passing over the edge of the ward. Usually cast on doorways and entrances. The ward is immune to all physical damage\comma{} but can only survive (8+$2\times$PP) points of spell damage.}
\spell{name = Throw Voice, school = Illusion, discipline = Bewitchment, type = Focus, level =Beginner, incant = ventrilofors, noDur = 1, noHigh = 1, noTravel = 1, noResist =1, effect =Cast your voice such that it appears to be coming from somewhere up to 5+$\times PP$ metres away.}
\spell{name = Timeslip, school = Divination, discipline = Temporal, type = Instant, level =Expert, noIncant = 1, duration = (1 + $2\times $PP ) minutes,noHigh = 1, noTravel = 1, resist = POW, resistDV = CC, effect =Create a perturbation in the temporal vortex which\comma{} on a failed Resist\comma{} catapaults the target forward in time\comma{} effectively removing them from reality for the duration of the spell. When the spell ends\comma{} the target reappears at their original location\comma{} unaware that time has passed.}
\spell{name = Torture, school = Dark Arts, discipline = Necromancy, type = Focus, level =Adept, incant = Crucio, noDur = 1, noHigh = 1, noTravel = 1, noResist =1, effect =Causes immense pain to the target\comma{} paralysing them whilst the spell is cast. 
Does PPd4 psychic damage per turn\comma{} though this spell cannot be used to reduce beings below 10\% of their maximum health.}
\spell{name = Trecherous Terrain, school = Transfiguration, discipline = Alteration, type = Instant, level =Novice, incant = transgresso, duration = 2 hours,noHigh = 1, noTravel = 1, noResist =1, effect =Transform the ground in a 5m radius around target into a deep bog\comma{} a bed of sharpened blades\comma{} or into a sticky mess\comma{} with the associated terrain costs.}
\spell{name = Trip, school = Maledictions, discipline = Curse, type = Instant, level =Beginner, incant = lubricor, noDur = 1, noHigh = 1, noTravel = 1, resist = ATH, resistDV = CC, effect =If the target is moving this turn cycle and fails to Resist\comma{} they go sprawling onto the ground taking 1d4 bludgeoning damage\comma{} and take the `Prone Position’ status.}
\spell{name = True Illusion, school = Illusion, discipline = Bewitchment, type = Ritual (10 minutes), level =Master, incant = stultuvisus, duration = 1 hours,noHigh = 1, noTravel = 1, resist = INT (perception), resistDV = 18, effect =Create a perfect illusion of an environment (up to 20m in radius) or people (up to 3)\comma{} which can be interacted with and touched by the target. Illusions can only have the knowledge that the caster has.
Illusion lasts for 10 hours. An observer may determine that the illusion is not real  by performing a Resist as a major action.}
\spell{name = True Shapeshift, school = Transfiguration, discipline = Alteration, type = Instant, level =Master, incant = muto, noDur = 1, noHigh = 1, noTravel = 1, noResist =1, effect =You assume the form of any object you wish\comma{} provided it has approximately the same size as you. You are indistinguishable from this object until you choose to break the spell.}
\spell{name = True Sight, school = Divination, discipline = Telepathy, type = Ritual (5 minutes), level =Master, incant = vidergo sumus, duration = 1 hour,noHigh = 1, noTravel = 1, noResist =1, effect =For 1 hour\comma{} you see things as they truly are. You see hidden traps\comma{} secret doors\comma{} and astral projections. You can see through illusion spells\comma{} and attempts to deceive you are in vain.}
\spell{name = Ultimate Healing, school = Recuperation, discipline = Healing, type = Ritual (2 turns), level =Master, incant = vita maxima, noDur = 1, noHigh = 1, travel = Yellow\minus{}white flash,noResist =1, effect =Restores a character to full health\comma{} and removes all negative status effects. Cannot be cast on self.}
\spell{name = Undo Transformation, school = Transfiguration, discipline = Alteration, type = Instant, level =Adept, incant = reparifarge, noDur = 1, noHigh = 1, noTravel = 1, noResist =1, effect =Transfiguration countercharm: undoes the effect of any transfiguration spell (but can not banish summoned objects). 
Spell is successful if casting check exceeds the check that cast the original spell.}
\spell{name = Unfathomable Visage, school = Dark Arts, discipline = Occultism, type = Instant, level =Novice, incant = facadus horribilis, duration = 1 minutes,noHigh = 1, noTravel = 1, noResist =1, effect =Imbue yourself with the essence of one of the Eldritch beings\comma{} giving a bonus to any one of your Attributes equal to half your caster level\comma{} in doing so\comma{} however\comma{} your face transforms into a horrifying edifice which drives all who gaze upon it to go insane: they will either fly into a murderous rage\comma{} become catatonic\comma{} or flee from you.}
\spell{name = Universal Tear, school = Dark Arts, discipline = Occultism, type = Ritual (1 week), level =Master, incant = ostium, noDur = 1, noHigh = 1, travel = Searing white flash,noResist =1, effect =Punch a hole in the fabric of reality\comma{} and establish a portal to one of the Higher Planes. This portal takes the form of a shimmering door. Stepping through the door takes you to the chosen Plane.}
\spell{name = Unlock, school = Charms, discipline = Kinesis, type = Instant, level =Novice, incant = alohomora, noDur = 1, noHigh = 1, travel = Imperceptible rays,noResist =1, effect =Unlock objects. Mundane locks will fall open for you\comma{} whilst to open magically locked objects\comma{} the unlocking must exceed the locking casting check.}
\spell{name = Use Ancient Powers, school = Dark Arts, discipline = Occultism, type = Ritual (1 hour), level =Beginner, noIncant = 1, duration = 1 day,noHigh = 1, noTravel = 1, noResist =1, effect =When at a site of ancient magic \minus{}\minus{} be it at a place where some great feat of magic was achieved\comma{} the residence of some powerful being\comma{} or simply somewhere where magic has seeped into the very walls \minus{}\minus{} you may perform this ritual to tap into those ancient powers to gain a +2 bonus to spellcasting checks whilst in this area. This ritual cannot be used at the same site for the next 7 days.}
\spell{name = Vanish Object, school = Transfiguration, discipline = Conjuration, type = Instant, level =Adept, incant = evanesco, noDur = 1, higher = A character above 9th level may add 1 free PP for every 3 character levels above 6th.,noTravel = 1, noResist =1, effect =Cause a 200g animal or object to vanish\comma{} without a trace. 
Each power point doubles the mass of objects that can be vanished.  You can only vanish a sentient creature if it has a lower POW score than you.}
\spell{name = Vicious Slash, school = Dark Arts, discipline = Necromancy, type = Instant, level =Beginner, incant = sectumsempra, noDur = 1, noHigh = 1, travel = Red slash,noResist =1, effect =Gouges at a target up to 2m away\comma{} leaving deep\comma{} cursed wounds\comma{} for (2+2$\times$PP)d4 points of slashing damage.}
\spell{name = Violent Phantasms, school = Illusion, discipline = Psionics, type = Instant, level =Novice, incant = umbra impetia, duration = (3 + PP) turns,noHigh = 1, travel = Purple bolt,resist = SPR (endurance), resistDV = CC, effect =This spell causes the target to believe that multiple phantasms are attacking them target\comma{} doing (1+PP)d6 psychic damage for every turn that the phantasms are active. Once the original spell hits the targets\comma{} phantasms exist only within the target{\apos}s mind\comma{} and so are not stopped by shields or wards (except antimagic wards).}
\end{multicols}
\fi
\if \coreMode0
	\begin{multicols}{4} \raggedbottom\subsubsection{Charms}
\textbf{Level 1 Spells}
\begin{itemize}[itemsep=0em]
\renewcommand\labelitemi{-}
\item Banshee Wail

\item Create Fire

\item Create Trap

\item Create Water

\item Fresh Air

\item Halt

\item Illuminate Wand

\item Levitation

\item Mark Surface

\item Preserve Object


\end{itemize}
\subsubsection{Dark Arts}
\textbf{Level 1 Spells}
\begin{itemize}[itemsep=0em]
\renewcommand\labelitemi{-}
\item Blight

\item Eldritch Knowledge

\item Shadow Blast

\item Shroud of Darkness

\item Vicious Slash


\end{itemize}
\vfill\null
\columnbreak\subsubsection{Divination}
\textbf{Level 1 Spells}
\begin{itemize}[itemsep=0em]
\renewcommand\labelitemi{-}
\item Astral Assistance

\item Hunter\apos{}s Mark

\item Identify

\item Locate

\item Receive Omen

\item Sense Traps

\item Speak in Tongues

\item Telepathic Bond


\end{itemize}
\subsubsection{Hexes \& Curses}
\textbf{Level 1 Spells}
\begin{itemize}[itemsep=0em]
\renewcommand\labelitemi{-}
\item Acidic Burst

\item Cause Confusion

\item Green Sparks

\item Knockback

\item Sting

\item Trip


\end{itemize}
\vfill\null
\columnbreak\subsubsection{Illusion}
\textbf{Level 1 Spells}
\begin{itemize}[itemsep=0em]
\renewcommand\labelitemi{-}
\item Blur

\item Chaotic Whispers

\item Charm Entity

\item Glamour

\item Imbue Bravery

\item Night Vision

\item Throw Voice


\end{itemize}
\subsubsection{Recuperation}
\textbf{Level 1 Spells}
\begin{itemize}[itemsep=0em]
\renewcommand\labelitemi{-}
\item Aid Charm

\item Caterwauling Ward

\item Magical Shield

\item Minor Healing

\item Privacy Ward

\item Reinforce Shield

\item Stoneskin

\item Sunburst


\end{itemize}
\vfill\null
\columnbreak\subsubsection{Transfiguration}
\textbf{Level 1 Spells}
\begin{itemize}[itemsep=0em]
\renewcommand\labelitemi{-}
\item Alter Hair

\item Basic Transmutation

\item Change Colour

\item Conjure Flowers

\item Launch Spike

\item Potion Mixing Spell

\item Silver Shield

\item Steelclaw


\end{itemize}
\end{multicols}\clearpage\begin{multicols}{3}\spell{name = Acidic Burst, incant = ambustum, school = Hexes \& Curses, type = Instant, level =Beginner, fp = 3, attribute =POW, proficiency = , noProf = 1, dv = 4, effect =Fills a 5m target area with an acidic cloud that does (1+ CV – DV) acid damage per turn. Cloud lasts for 10 cycles\comma{} unless in a confined space\comma{} where it lasts until removed by other means.}
\spell{name = Aid Charm, incant = subsidium, school = Recuperation, type = Instant, level =Beginner, fp = 2, attribute =EMP, proficiency = , noProf = 1, dv = 4, effect =Target has HP ceiling raised by 5 points for 1 hour}
\spell{name = Alter Hair, incant = crinus muto, school = Transfiguration, type = Instant, level =Beginner, fp = 2, attribute =CHR, proficiency = Deception, dv = 4, effect =Alters the colour and style of the casters hair. Useful for disguises. Degrades after 5 hours.}
\spell{name = Astral Assistance, incant = auxilio, school = Divination, type = Ritual (2 turns), level =Beginner, fp = 5, attribute =EMP, proficiency = Arcane, dv = 5, effect =By laying your hand upon a sapient being\comma{} you may channel magical energy into them. On the next check the target performs\comma{} roll 1d4\comma{} and add it to the check (+1 per PP\comma{} max 3). If the check fails\comma{} both the target and the caster take 1d6 psychic damage.}
\spell{name = Banshee Wail, incant = magnus surgerus, school = Charms, type = Instant, level =Beginner, fp = 3, attribute =SPR, proficiency = Performance, dv = 5, effect =All targets in hearing range take 2 points of psychic damage (+3 per PP)\comma{} and awaken if they are sleeping.}
\spell{name = Basic Transmutation, incant = formum mutatio, school = Transfiguration, type = Instant, level =Beginner, fp = 4, attribute =FIN, proficiency = , noProf = 1, dv = 4, effect =Transform a 200g non\minus{}sapient animal or object into a different animal or solid object. 
Each power point doubles the mass of objects that can be transformed.  Lasts for 1 hour. Objects must be simple in nature.}
\spell{name = Blight, incant = thanatos, school = Dark Arts, type = Instant, level =Beginner, fp = 4, attribute =EVL, proficiency = , noProf = 1, dv = 5, effect =A wave of necrotic energy extends outwards from you in a radius of 10m (doubled with every PP\comma{} max 1km). All plants within range die instantly\comma{} and all other living beings take 1d4 necrotic damage (+1 per PP)}
\spell{name = Blur, incant = celeritate, school = Illusion, type = Instant, level =Beginner, fp = 4, attribute =CHR, proficiency = , noProf = 1, dv = 3, effect =The target seems to become blurry around the edges\comma{} it is difficult to tell exactly where they are\comma{} and where they aren{\apos}t.
Gain check advantage on evasion checks for 3 turns.}
\spell{name = Caterwauling Ward, incant = caterwaul, school = Recuperation, type = Ward, level =Beginner, fp = 4, attribute =INT, proficiency = , noProf = 1, dv = 3, effect =Casts a ward on the area which emits a high\minus{}pitched scream when an unknown being crosses the threshold. 
Radius is (10 + $2\times$PP) metres. Ward decays after 2 weeks.}
\spell{name = Cause Confusion, incant = confundo, school = Hexes \& Curses, type = Instant, level =Beginner, fp = 5, attribute =CHR, proficiency = Deception, dv = 3, effect =Do 2 Fatigue damage.
Target performs a Resist Magic check against casting check\comma{} if it fails\comma{} then target acquires the Confused status.  If it succeeds\comma{} do 5 fatigue damage.}
\spell{name = Change Colour, incant = pigmentus, school = Transfiguration, type = Instant, level =Beginner, fp = 4, attribute =INT, proficiency = , noProf = 1, dv = 4, effect =Causes the colour of an object to change. Lasts for 2 days.}
\spell{name = Chaotic Whispers, incant = rastarum, school = Illusion, type = Concentration, level =Beginner, fp = 4, attribute =SPR, proficiency = Deception, dv = 4, effect =Target hears a voice in their ear whispering maddening words\comma{} that slowly drive them insane. Target takes (1 + PP)d4  psychic damage per turn\comma{} until they pass a SPR(endurance) Resist check with DV = casting check.}
\spell{name = Charm Entity, incant = sismeus amici, school = Illusion, type = Instant, level =Beginner, fp = 5, attribute =CHR, proficiency = Persuasion, dv = 4, effect =Causes the target to like you\comma{} persuasion checks get a (2+PP) bonus\comma{} max 5.}
\spell{name = Conjure Flowers, incant = orchideous, school = Transfiguration, type = Instant, level =Beginner, fp = 3, attribute =EMP, proficiency = , noProf = 1, dv = 5, effect =Conjures flowers from thin air. Lasts for 3 days.}
\spell{name = Create Fire, incant = incendio, school = Charms, type = Concentration, level =Beginner, fp = 3, attribute =SPR, proficiency = , noProf = 1, dv = 3, effect =A small jet of fire is emitted from the tip of your wand. 
Coming into contact with fire does 1d6 fire damage\comma{} and applies a minor Burned status effect.
(Larger jets of fire have a difficulty of 9\comma{}  do 4d6  fire damage and apply a Moderate burn)}
\spell{name = Create Trap, incant = dolus, school = Charms, type = Ritual (2 turns), level =Beginner, fp = 4, attribute =FIN, proficiency = Stealth, dv = 5, effect =Combine a magical ward with one of your existing spells. Cast the other spell first\comma{} then perform the trapping check.
If successful\comma{} creates a hidden magical trap of radius 50cm on any solid surface\comma{} with the effect of the original spell when triggered by an entity touching the trap. The effects of the trap are less than the original spell\comma{} but more power points make the trap more powerful. 
If you wish to keep a trap hidden from the GM\comma{} write down the location\comma{} spell and associated check values on a piece of paper\comma{} to be revealed when the trap is triggered.}
\spell{name = Create Water, incant = aguamente, school = Charms, type = Concentration, level =Beginner, fp = 4, attribute =INT, proficiency = , noProf = 1, dv = 4, effect =A jet of water is emitted from the tip of your wand\comma{} useful for extinguishing fires\comma{} or cleaning surfaces. 
(Larger jets of water have a difficulty of 16. Conjured water cannot be drunk)}
\spell{name = Eldritch Knowledge, incant = vetitum scenticus, school = Dark Arts, type = Ritual (3 turns), level =Beginner, fp = 6, attribute =EVL, proficiency = Arcane, dv = 3, effect =Gain access to eldritch knowledge. The Demons of the Deep will answer one of your questions\comma{} but the answers might drive you mad.
The question must be said out loud for all to hear\comma{} but the answer may be written down and passed to your privately.}
\spell{name = Fresh Air, incant = klinneract, school = Charms, type = Instant, level =Beginner, fp = 3, attribute =POW, proficiency = , noProf = 1, dv = 3, effect =A gust of air refreshes the air in a 2m radius (+1 per POW) around the caster\comma{} removing any gaseous effects and smelling faintly of lavender.}
\spell{name = Glamour, incant = lux stultium, school = Illusion, type = Instant, level =Beginner, fp = 3, attribute =INT, proficiency = Deception, dv = 3, effect =Create a superficial glamour around a person\comma{} a simple trick of the light. The glamour disintegrates upon physical or magical contact.}
\spell{name = Green Sparks, incant = verdimillious, school = Hexes \& Curses, type = Instant, level =Beginner, fp = 4, attribute =FIN, proficiency = , noProf = 1, dv = 4, effect =Emits (5+PP) green sparks from your wand\comma{} which can be made to strike at the enemy. 
Each spark does (1 + CV – DV) force damage.}
\spell{name = Halt, incant = stabit, school = Charms, type = Instant, level =Beginner, fp = 2, attribute =SPR, proficiency = , noProf = 1, dv = 4, effect =Stop 1 inanimate object (+1 for every power point dedicated) in its tracks\comma{} if mid\minus{}air\comma{} it drops to the ground.}
\spell{name = Hunter\apos{}s Mark, incant = venari, school = Divination, type = Instant, level =Beginner, fp = 3, attribute =INT, proficiency = , noProf = 1, dv = 4, effect =If casting check exceeds passive resist value\comma{} caster is aware of the location of the target for the next 3 days\comma{} or until the mark is removed by magical means.}
\spell{name = Identify, incant = dicemi, school = Divination, type = Instant, level =Beginner, fp = 6, attribute =INT, proficiency = Research, dv = 3, effect =Learn the properties of the target: be it learning about the nature of the target\comma{} or the ingredients of a potion.
The more power points dedicated to the spell\comma{} the more information that is revealed.}
\spell{name = Illuminate Wand, incant = lumos, school = Charms, type = Concentration, level =Beginner, fp = 1, attribute =INT, proficiency = , noProf = 1, dv = 2, effect =Causes the tip of your wand to glow\comma{} like a torch. Casts bright light for 2m radius\comma{} and dim light for 10m. Spell last indefinitely\comma{} until the counterspell (knox) is used. No other spells can be used whilst lumos is active.}
\spell{name = Imbue Bravery, incant = fortudus, school = Illusion, type = Instant, level =Beginner, fp = 2, attribute =SPR, proficiency = Persuasion, dv = 3, effect =Imbue your target with fortitude and vigour. They gain check\minus{}advantage on all Fear\minus{}Resist checks for 1 hour.}
\spell{name = Knockback, incant = flipendo, school = Hexes \& Curses, type = Instant, level =Beginner, fp = 3, attribute =POW, proficiency = , noProf = 1, dv = 3, effect =Causes 2 points of force damage\comma{} and knocks the target back 1 metre. Each power point adds one metre to the knockback distance and 1 damage point. May need to consider impact (see `falling’)}
\spell{name = Launch Spike, incant = voco dens, school = Transfiguration, type = Instant, level =Beginner, fp = 4, attribute =POW, proficiency = , noProf = 1, dv = 4, effect =Conjure 1 enormous spike (+ 1 for each power point) to transfigure itself from the surrounding walls/floor\comma{} impaling the target. Each spike does 1d6 piercing damage.}
\spell{name = Levitation, incant = wingardium leviosa, school = Charms, type = Concentration, level =Beginner, fp = 5, attribute =FIN, proficiency = , noProf = 1, dv = 4, effect =Cause an object of 500g or less to levitate\comma{} controlling the vertical distance at will. 
Each power point dedicated doubles the mass of the object that can be lifted.}
\spell{name = Locate, incant = locus, school = Divination, type = Instant, level =Beginner, fp = 3, attribute =EMP, proficiency = Research, dv = 4, effect =Learn the location of non\minus{}magical objects or an unshielded living being.}
\spell{name = Magical Shield, incant = protego, school = Recuperation, type = Concentration, level =Beginner, fp = 5, attribute =POW, proficiency = , noProf = 1, dv = 5, effect =Erects an ethereal shield in front of you that absorbs incoming magical attacks.
Shielding charm increases AC by 15+PP against all incoming spells\comma{} but does not protect against physical damage\comma{} or the aftereffects of magic (i.e. a nearby explosion)}
\spell{name = Mark Surface, incant = stylum, school = Charms, type = Concentration, level =Beginner, fp = 2, attribute =FIN, proficiency = Dxterity, dv = 2, effect =Use your wand as anything from a thin marker to a thick paintbrush\comma{} the {\it paint} is a magical adhesive that sticks to any surface\comma{} and may be of any colour you choose.}
\spell{name = Minor Healing, incant = enervate, school = Recuperation, type = Concentration, level =Beginner, fp = 3, attribute =EMP, proficiency = Healing, dv = 3, effect =Heal for 2 points per turn. 
If the target has a serious wound\comma{} i.e. a broken bone\comma{} cannot heal beyond 50\% health. Only works on living creatures.}
\spell{name = Night Vision, incant = aspectu, school = Illusion, type = Instant, level =Beginner, fp = 3, attribute =EMP, proficiency = Perception, dv = 3, effect =Give the target nightvision for one hour: dim light is as bright as daylight\comma{} and darkness is consdiered dim.}
\spell{name = Potion Mixing Spell, incant = , noIncant = 1, school = Transfiguration, type = Ritual(5 turns), level =Beginner, fp = 2, attribute =INT, proficiency = Arcane, dv = 0, effect =Used to mix a potion. See page \pageref{S:Enchanting} for details.}
\spell{name = Preserve Object, incant = preseritas, school = Charms, type = Instant, level =Beginner, fp = 2, attribute =FIN, proficiency = dexterity, dv = 3, effect =The target is unaffected by the flow of time for 10 days\comma{} and does not rot or otherwise decay.}
\spell{name = Privacy Ward, incant = muffliato, school = Recuperation, type = Ward, level =Beginner, fp = 6, attribute =SPR, proficiency = , noProf = 1, dv = 4, effect =A buzzing sound fills the ears of anyone trying to listen in on your conversations whilst you are in the warded area. Lasts for one hour\comma{} and has a radius of 2m.}
\spell{name = Receive Omen, incant = , noIncant = 1, school = Divination, type = Ritual (3 turns), level =Beginner, fp = 2, attribute =INT, proficiency = , noProf = 1, dv = 3, effect =Use your tea leaves to receive an omen about the future. Ask a question about the outcome of an event. The tea leaves will tell you if the outcome is positive\comma{} negative\comma{} or neutral. Takes 4 minutes to cast.}
\spell{name = Reinforce Shield, incant = praesidium, school = Recuperation, type = Concentration, level =Beginner, fp = 2, attribute =INT, proficiency = Arcane, dv = 4, effect =Restore the strength of a target shield or magical ward by (2+PP) points per turn that this spell is maintained.}
\spell{name = Sense Traps, incant = antidolus, school = Divination, type = Instant, level =Beginner, fp = 4, attribute =INT, proficiency = Understand Other, dv = 5, effect =Attempt to discover any traps in your immediate vicinity. If successful\comma{} you may learn the location of the trap\comma{} and the trigger (but not the effect). Success conditions are set by the GM.}
\spell{name = Shadow Blast, incant = malusangui, school = Dark Arts, type = Instant, level =Beginner, fp = 3, attribute =POW, proficiency = , noProf = 1, dv = 2, effect =Hurl shadows at you enemy\comma{} dealing 1 necrotic damage for every casting point over the difficulty level.}
\spell{name = Shroud of Darkness, incant = tenebrosa, school = Dark Arts, type = Instant, level =Beginner, fp = 4, attribute =EVL, proficiency = , noProf = 1, dv = 4, effect =Extinguish all light within a 10m radius (+2 for every PP to the spell)}
\spell{name = Silver Shield, incant = argentipus, school = Transfiguration, type = Instant, level =Beginner, fp = 6, attribute =INT, proficiency = , noProf = 1, dv = 5, effect =Conjures a silver shield from thin air\comma{} to defend you. Shield absorbs both physical and magical attacks for up to 15 damage points\comma{} before breaking.}
\spell{name = Speak in Tongues, incant = lingua maxima, school = Divination, type = Instant, level =Beginner, fp = 8, attribute =EMP, proficiency = Understand Other, dv = 4, effect =By meditating for 5 minutes\comma{} you may understand and speak the language of a willing target individual. Effect lasts until concentration is broken.}
\spell{name = Steelclaw, incant = ferscabere, school = Transfiguration, type = Instant, level =Beginner, fp = 4, attribute =POW, proficiency = , noProf = 1, dv = 4, effect =Transfigures an animal{\apos}s claws into large steel talons\comma{} increasing their physical damage by +5 . Each power point dedicated gives these talons + 2 damage.  Lasts for 1 day.}
\spell{name = Sting, incant = ictus, school = Hexes \& Curses, type = Instant, level =Beginner, fp = 5, attribute =SPR, proficiency = , noProf = 1, dv = 2, effect =Stings the target for (2 + CV – DV) poison damage.}
\spell{name = Stoneskin, incant = lapis pellium, school = Recuperation, type = Instant, level =Beginner, fp = 4, attribute =SPR, proficiency = Endurance, dv = 4, effect =Increase the target{\apos} AC by 10 for 5 minutes (25 combat rounds). Does not stack.}
\spell{name = Sunburst, incant = sol maxima, school = Recuperation, type = Instant, level =Beginner, fp = 4, attribute =SPR, proficiency = , noProf = 1, dv = 4, effect =A burst of bright light does 1d6 holy damage to all targets in a 5m radius.}
\spell{name = Telepathic Bond, incant = conanimus, school = Divination, type = Ritual (2 turns), level =Beginner, fp = 5, attribute =EMP, proficiency = Understand Other, dv = 5, effect =Form a mental connection between your mind and the mind of a willing target. You may then use this connection to communicate silently. 
Target must be within touching distance when the spell is cast\comma{} but the bond has no distance limit after that.
Lasts for 2 days.}
\spell{name = Throw Voice, incant = ventrilofors, school = Illusion, type = Concentration, level =Beginner, fp = 4, attribute =INT, proficiency = Deception, dv = 2, effect =Cast your voice such that it appears to be coming from somewhere up to 5+$\times PP$ metres away.}
\spell{name = Trip, incant = lubricor, school = Hexes \& Curses, type = Instant, level =Beginner, fp = 4, attribute =FIN, proficiency = , noProf = 1, dv = 4, effect =If the target is moving this turn cycle and fails an ATH Resist check\comma{} they go sprawling onto the ground taking 1d4 bludgeoning damage\comma{} and take the `Prone Position’ status.}
\spell{name = Vicious Slash, incant = sectumsempra, school = Dark Arts, type = Instant, level =Beginner, fp = 4, attribute =POW, proficiency = , noProf = 1, dv = 4, effect =Gouges at the target\comma{} leaving deep\comma{} cursed wounds\comma{} for 1d6 points of slashing damage\comma{} plus two for every PP.}
\end{multicols}
\fi
