\documentclass[../CoreRulebook.tex]{subfile}


\chapter{Types of Magic}

\newcommand\schoolDescribe[6]
{
	\subsection{#1}
	
	#2
	
	\subsubsection{{\it #3} Magic}
	
	#4
	
	\subsubsection{{\it #5} Magic}
	
	#6
}

\section{Magical Schools}\label{S:Schools}

The study of magic is a far-reaching field, which encompasses many different areas and skills -- some of which require vastly different skillsets to use. For this reason, a magical taxonomy was introduced by the Wizangemot in 1755, which divides the study of magic up into 7 `Schools', each of which contains a number of `Disciplines'. 



	\schoolDescribe{Charms}{The Charms school of magic fundamentally relies on magically manipulating the position and speed of matter, whether on a large scale, to cause objects to levitate and fly - or on a microscopic level, to excite and energise the inside of an object, causing it to burst into flame. \\Those who are proficient in Charms are known as {\it Sorcerers}.}{Elemental}{Elemental magic studies the manipulation and invocation of very primal forces -- heat, light, energy, matter, and the classical elements.}{Kinetic}{Kinetics is a discipline which relies on moving and manipulating physical objects, and often forms the basis of `everyday' magic.}

	\schoolDescribe{Divination}{The Divination school encompasses magic which taps into forces which exist beyond the physical world to discern knowledge that would have previously remained hidden - entering the domain of the senses, memory, and the spiritual realms. }{Cerebral}{Cerebral magic is the study of peering into the human mind, extending the senses beyond their normal range and detecting the undetectable.\\Those who are proficient in the field of Divination are known as {\it Clairvoyants}.}{Temporal}{One of the most mysterious disciplines, temporal magic allows one to see beyond concerns such as time and space, and observe (and perhaps manipulate) the universe at an extraplanar level}
	
	\schoolDescribe{Illusion}{The Illusion school of magic is, as the name might suggest, focussed on magic which produces false images and tweaks the mind into seeing things which are not really there. \\Witches and Wizards who excel in Illusion magics are known as {\it Magicians}. }{Bewitching}{This discipline focusses on the gentle persuasion of the mind and the manipulation and conjuring of images to convince the target of something which is not true.}{Psionic}{A darker side of illusion magics, psionics is the art of imposing your will over that of your target -- forcing your way into their mind and altering it as you see fit.}
	
	\schoolDescribe{Malediction}{The Malediction school of magic contains those spells which have the primary intent to hurt, inflict harm on and otherwise incapacitate others. \\ Those who are experts in the field of Malediction are known as {\it Battlemages}. }{Hexes}{Hexes are a field which focusses on magic that directly harms the targeted person or object.}{Curses}{Unlike hexes, curses do not directly harm the target but instead incapacitates them, inhibits their capabilities, or otherwise reduces the threat they pose.}
   
   \schoolDescribe{Recuperation}{The Recuperation school of magic is often considered unglamourous, but those who can look past that can see that the ability to heal and protect yourself and others from harm is utterly invaluable. \\ Those who are proficient in the use of Recuperation magic are known as {\it Aegistes}. }{Healing}{Healing is, unsurprisingly, the study of magic used to heal the sick and wounded, break curses and project powerful positive energies.}{Warding}{Warding magic is almost entirely defensive in nature, allowing the caster to protect themselves and others from harm by casting powerful and long lasting shields and force-fields.}
	
	\schoolDescribe{Transfiguration}{The Transfiguration school of magic is focused on the transformation of the natural order - either by altering and reshaping the form of existing objects, or by summoning entirely new matter from thin air. \\Those who excel in Transfiguration are known as {\it Thaumaturges}.}{Alteration}{The alteration discipline studies the ability to change things from one form into another.}{Conjuration}{Conjuration magic is concerned with the ability to summon new objects and beings out of thin air, or to banish objects from existence.}
	
	\schoolDescribe{Dark Arts}{The Dark Arts school of magic encompasses magic which is frowned on in polite society, either because it involves truly evil spells - those which cannot be used without leaving scars on the soul, or those which tap into the dangerous and unfathomable energies of the dark and unspeakable things which lie just out of sight - under your bed and in the corner of your eye...\\Those who wield this forbidden magic are known as {\it Warlocks}.}{Necromancy}{A taboo discipline which contains deeply unpleasant spells which can only be cast by beings corrupted by evil - torture, death and worse lie in the domain of necromancy.}{Occultism}{Occultism is a rarely studied discipline that accesses and manipulates otherworldly energies originating from the Eldritch domain -- powerful, yet highly unpredictable.}

\normalsize

Every spell is assigned to be a member of one of these disciplines, which determines the skills that are necessary to cast it. 

\newpage
\section{Spell Types} 

In addition to falling into one of the seven Schools (a taxonomy based on the spell effect), every spell can also be categorised as a {\it type}, which is based on how the spell is cast. 

\subsection{Instant}

An instant spell is `cast and forget': as soon as you complete the requisite casting checks, the spell is `launched' (usually in the form of a magical bolt of light) towards the target. These bolts travel at speeds of 100m per cycle, which means in most cases, the effect is applied between the successful casting and the beginning of the next turn cycle.

Instant spells are denoted by the symbol \instSymb.

\subsection{Focus}

A focus spell is cast like an Instant spell, but may then be continued indefinitely, repeating the initial effects once per turn as long as you keep the spell active. No further checks are needed to continue the spell, but you must remain focussed and unless stated otherwise, the FP cost is deducted once per turn as well.

Because you must remain focussed, no further spells can be cast for the duration of this spell, and all subsequent movement checks are `minor', and no other actions (such as evasion) may be taken.   

Whilst maintaining a {\it Focus} spell you are considered {\it Distracted} and take the associated status effect. This renders you vulnerable to Critical Strikes, and upon taking damage you must pass a Willpower Resist check to maintain your concentration. 

You may end the spell effect at any time without it counting as an action. 

Focus spells are denoted by the symbol \concSymb.

\subsection{Ward}
A ward is (usually) a Recuperation spell that affects a large area. A ward may be centred on a fixed point or object, or may be centred on a moving location or even a sentient being. 

Wards, however, have an unfortunate habit of interfering with each other when used in unison. If two wards have a significant overlapping region of effect and the caster does not have the {\it Multiward} skill or an equivalent feat, there is a significant chance (determined by the GM) that both wards will collapse. 

The interference only applies if the wards are similar in magnitude and intent. For example, Hogwarts castle is a heavily warded region, but a small ward could be placed in a room without problem. Interference would only become a problem when a new castle-wide ward was attempted. 

Equally, interference only applies if the effects of the ward compound each other -- if they lie in opposition, then the usual spell mechanics are applied. For example, a character with a personal shield ward touches a beartrap ward -- neither ward collapses, but the beartrap ward is triggered, and the shield will attempt to protect the character. 

Ward spells are denoted by the symbol \wardSymb.

\subsection{Ritual}

A Ritual spell is a spell that requires a large amount of preparation -- be it meditation, drawing a summoning circle upon the ground, or performing a special dance. Each Ritual spell has a designated time that the ritual takes to complete, to cast a ritual spell you must spend this length of time preparing for the spell, and after the requisite time has passed, {\it then} you perform the check, and the spell effect is activated. If you fail the check, or choose to stop the ritual, i.e. to take another action, you must restart the ritual spell from the beginning. 

As with a focus spell, concentration is key to completing a ritual, and whilst performing a ritual, you are considered {\it Distracted}. 

Ritual spells are denoted by the symbol \ritSymb.

\section{Spell Shapes}

Some spells produce bolts of energy which fly towards a target, whilst others project their energy into a given region, which are often classified via geometrical shapes: a {\it line}, a {\it cube}, a {\it sphere}, a {\it circle} a {\it cone} or a {\it cylinder}. These shapes may either originate around the caster, or from a point designated by the spell. 

\subsection{Circle}

A circular spell extends outwards from the point of origin in a 2D circular shockwave that lies parallel to the ground. The height of the shockwave above the ground is set by the point of origin, which is not included in the shockwave region (unless the caster chooses it to be). Because of its 2D nature, a circular spell can be avoided by ducking beneath it, or jumping over it -- it is only if the shockwave impacts you that the spell effect is applied. 

\subsection{Cone}

The point of origin of a cone is typically the caster's wand, and a cone extends outwards from the wand, in the direction that the wand is pointing. A cone extends forwards to the specified distance, and has a circular cross section, the radius of which is equal to the distance away from the point of origin (so it is a 45$^\circ$ cone).

 The point of origin of the cone is not considered part of the spell area. 

\subsection{Cube}

The point of origin for a cubic spell may be selected to be either the centre of the cube, or the centre of one of its 6 sides. The cube's side-length is specified by the spell effect. The cube point of origin is only affected by the spell if you choose the centre-origin.

\subsection{Cylinder}

A cylinder point of origin is specified to be a point on the ground, around which a circular cross section is drawn, and then a cylinder of energy rises up vertically to a specified height. Generally, a cylinder spell adjusts its size to an individual, and if not otherwise specified, the cylinder is 5cm wider than the target individual is wide, and 5cm taller than the target. The point of origin is affected by the spell. 

\subsection{Line}

A line extends in a straight path from the origin (a caster's wand) towards the target for a specified distance. Unless otherwise specified, the beam is considered to have the cross section equivalent to a pencil. The point of origin is not affected by the spell. 

\subsection{Sphere}

A sphere's point of origin lies at the centre, and the spell effect expands equally out in all directions from that point. Generally, the spell effect cannot penetrate into the ground or through solid objects (unless, for example, it is an explosion). The point of origin is affected by the spell. 



\chapter{Casting Spells}

Spellcasting is the process by which a witch or wizard harness the infinite, chaotic and formless power of {\it magic}, shape it through their intellect or force of will, and project it into the world around them. 

For most wizards, this is achieved through the use of an incantation, a movement of the wand, and deep concentration, though some magic spells require a ritual be conducted before the magic can be executed. 

Some powerful wizards understand that these are simply crutches, guiding tools for the weaker mind - and can cast magic both silently, and without their wand to focus the magical energies. This, however, is an advanced feat and is not to be taken lightly. 

\section{Learning Spells}

In order to cast a spell, you must be guided in how this is achieved - to learn the incantation, the wandwork and the correct patterns of thought which will channel the magical energy correctly. 

\subsection{Spellbooks and Book-Casting}

The most common source of such information is in spellbooks, such as those listed in the Items chapter. If you have a spellbook in your possession, you may be able to flip through and find a spell you would like to cast. By carefully studying the text, you may attempt to cast the spell, whilst using the book for reference. 

This is known as {\it Book-Casting}. 

Book casting is a fairly slow process - even the slightest misreading of the text could result in drastic consequences! When used in combat, book casting always takes up the entirety of your turn. 

After choosing the spell you would wish to cast, you must perform a {\it Casting Check} (see below). If the check succeeds, you must then perform an accuracy check (if relevant), and then the magic effect takes hold. 

Congratulations - you just cast your first spell!

\subsection{Memorising Spells}

After you have book-cast a spell a couple of times - you will begin to get the hang of it. Eventually, you will have comitted the spell to memory. This occurs after you have book-cast a spell a number of times equal to:
$$ N = 8 - \text{\attInt{} Modifier} ~~~\text{(min 2)} $$
These book-casts have to be in an appropriate use of the spell - you can't sit and hex a tree 5 times in a row, and expect to learn the spell. You must successfully use the spell for its intended purpose for it to be a valid learning experience. 

Alternatively, you may spend your downtime studying the {\it theory} of the spell, over the practice. Studying a spellbook, or working with a proficient teacher for 1 hours is equivalent to casting the spell once in a real-life scenario. However, knowing something is theory is not always quite enough: you can never {\it completely} learn a spell this way. After completing your research, you must book-cast the spell at least once more, before it is truly memorised. 

\subsection{Memory-Casting}

After a spell is memorised, you no longer need the spellbook to hand in order to cast the spell - instead you can {\it Memory-Cast} it. 

When you are comforable enough with the spell to memory-cast it, the casting check is assumed to succeed, unless you are trying to do something particularly out of the ordinary - such as silent casting. 

A memory cast spell therefore skips the casting check stage, and jumps straight to the accuracy check (if applicable), and then applies the specified spell effect. 


\section{Casting Checks}

When casting an unfamiliar spell there is a non-trivial chance for a spellcaster to flub some important aspect of the spellcasting - which causes the spell to fail to materialise. 

This is quantified through the {\it Casting Check}. A casting check is a normal ability check, performed with a d20 dice. The relevant ability modifier is determined by the kind of spell you are attempting to cast. Spells from different disciplines require different mental abilities in order to manifest, as shown in the table below: 

\def\xS{2}
\def\wS{2}
\begin{center}
	\begin{rndtable}{c m{\xS cm} p{\wS cm}}
	\bf School	&	\bf Discipline	&	\bf Attribute
	\\
	\school{Charms}{Elemental}{\ElCheck}{Kinesis}{\KinCheck}
	\\
	\school{Divination}{Telepathy}{\TelCheck}{Temporal}{\TemCheck}
	\\
	\school{Illusion}{Bewitchment}{\BewCheck}{Psionics}{\PsiCheck}
	\\
	\school{Malediction}{Hexes}{\HexCheck}{Curses}{\CurCheck}
   \\ 
   \school{Recuperation}{Healing}{\HeaCheck}{Warding}{\WarCheck}
	\\
	\school{Transfiguration}{Alteration}{\AltCheck}{Conjuration}{\ConCheck}
	\\
	\school{Dark Arts}{Necromancy}{\NecCheck}{Occultism}{\OccCheck}
	\end{rndtable}
\end{center}

In addition, as well as an affinity based on their attribute scores, some beings possess proficiencies in various disciplines. If a being is considered proficient in the spell-school they are attempting to cast, then they add their Expertise Bonus to the casting check. 


The difficulty of a spell is determined by the caster's own level, and the difficulty of the spell they are trying to cast. Use the table below to determine the casting DV:
\def\cc{\cellcolor{\tablecolorhead}}
\begin{rndtable}{c c c c c c c c}
~	& ~ &	\multicolumn{6}{c}{\bf Spell Level}
\\
\cc	&	\cc	&	1 &2&3&4&5&6	
\\
\cc~	&	\cc1	&	15
\\
\cc&\cc	2	&	10	&	15
\\
\cc&	\cc3	&	5	&	10	&	20
\\
\cc&	\cc4	&	5	&	10	&	15	&	20
\\
\cc&	\cc5	&	5	&	10	&	15	&	20	&	25
\\
\multirow{-6}{*}{\rotatebox[origin=c]{90}{\cc \bf Caster Level}}&	\cc 6 &	5	&	10	&	15	&20	&25	&30
\end{rndtable}

\subsection{Spell Accuracy}

Spells require an accuracy check in one of two circumstances:

\begin{itemize}
	\item The spell is classified as either {\it Blockable} or {\it Dodgeable}. 
	\item The target of the spell is far enough away, or small enough to trigger the `hard-to-hit' rules discussed on page \pageref{S:HardToHit}.
\end{itemize} 

Perform an accuracy check using the normal d20 dice. The modifier used is the same as the one that is used in the casting check - determined by the spell's discipline, plus the Expertise Bonus if applicable. 

\section{Fortitude Cost}


Casting spells is not as simple as waving your wands and saying the magic words -- it takes great mental clarity to cast, and you can become exhausted from casting difficult spells. This mental burden is enumerated through the Fortitude Points attribute. 

Each spell has an associated FP cost, which is deducted only after it is successfully cast. If the casting fails, then only half of the fortitude cost is deducted (rounded up).
 
You cannot cast a spell if it would send you into negative FP -- you must wait for your head to clear before attempting that spell.  

The fortitude cost of a given spell is determined by the spell type (Instant, Focus, Ritual etc.) and the difficulty of the spell, and if the spell is cast from memory or not. A book-cast spell has a 50\% higher FP cost than if the caster is familiar with the spell. 

The FP cost of a spell is numerically equal to the difficulty of a spell, prior to any skill modifications (i.e. a skill which reduces the difficulty of a certain spell does not reduce the FP of it, and vice versa), unless the spell is being book-cast, in which case use the bracketed values.  



{\footnotesize
\def\wFP{1}
\begin{center}
	\begin{rndtable}{p{1cm} |c |   c | c |c | c}
		~	&{\bf Beginner}	&	{\bf Novice}	&	{\bf Adept}	&	{\bf Expert}	&	{\bf Master}
		\\
		\cellcolor{\tablecolorhead} \bf Memory &	\FPEntry{\DVBeg}{1}	&	\FPEntry{\DVNov}{1}	&	\FPEntry{\DVAdp}{1}	&	\FPEntry{\DVExp}{1}	&	\FPEntry{\DVMas}{1}
		\\
		\cellcolor{\tablecolorhead} \bf Book  &	\FPEntry{\DVBeg}{2}	&	\FPEntry{\DVNov}{2}	&	\FPEntry{\DVAdp}{2}	&	\FPEntry{\DVExp}{2}	&	\FPEntry{\DVMas}{2}
	\end{rndtable}
\end{center}
}
\subsection{Casting at Higher Level}


\subsection{Resisting Spells}

Even after a spell has successfully hit a target, it is possible for them to fight against the magic, reducing the effects and sometimes negating it entirely. 

This is normaly done by performing a {\it Resist} check before the spell effect is applied, and comparing it to the spellcaster\apos{} Resist DV. If the Resist is greater than or equal to the Resist DV of the spellcaster, the spell effect is modified as the spell description states. 

The Resist DV of a cast spell is enumerated through the {\it Subjugation} statistic:

$$\text{Subjugate} = 8 + \text{Expertise bonus}  + \text{POW modifier} + \text{other bonuses}$$



\section{Spell Range} \label{S:Range}


