\documentclass[../CoreRulebook.tex]{subfile}


\chapter{Types of Magic}

\newcommand\schoolDescribe[6]
{
	\subsection{#1}
	
	#2
	
	\subsubsection{{\it #3} Magic}
	
	#4
	
	\subsubsection{{\it #5} Magic}
	
	#6
}

\section{Magical Schools}\label{S:Schools}

The study of magic is a far-reaching field, which encompasses many different areas and skills -- some of which require vastly different skillsets to use. For this reason, a magical taxonomy was introduced by the Wizangemot in 1755, which divides the study of magic up into 7 `Schools', each of which contains a number of `Disciplines'. 



	\schoolDescribe{Charms}{The Charms school of magic fundamentally relies on magically manipulating the position and speed of matter, whether on a large scale, to cause objects to levitate and fly - or on a microscopic level, to excite and energise the inside of an object, causing it to burst into flame. \\Those who are proficient in Charms are known as {\it Sorcerers}.}{Elemental}{Elemental magic studies the manipulation and invocation of very primal forces -- heat, light, energy, matter, and the classical elements.}{Kinetic}{Kinetics is a discipline which relies on moving and manipulating physical objects, and often forms the basis of `everyday' magic.}

	\schoolDescribe{Divination}{The Divination school encompasses magic which taps into forces which exist beyond the physical world to discern knowledge that would have previously remained hidden - entering the domain of the senses, memory, and the spiritual realms. }{Cerebral}{Cerebral magic is the study of peering into the human mind, extending the senses beyond their normal range and detecting the undetectable.\\Those who are proficient in the field of Divination are known as {\it Clairvoyants}.}{Temporal}{One of the most mysterious disciplines, temporal magic allows one to see beyond concerns such as time and space, and observe (and perhaps manipulate) the universe at an extraplanar level}
	
	\schoolDescribe{Illusion}{The Illusion school of magic is, as the name might suggest, focussed on magic which produces false images and tweaks the mind into seeing things which are not really there. \\Witches and Wizards who excel in Illusion magics are known as {\it Magicians}. }{Bewitching}{This discipline focusses on the gentle persuasion of the mind and the manipulation and conjuring of images to convince the target of something which is not true.}{Psionic}{A darker side of illusion magics, psionics is the art of imposing your will over that of your target -- forcing your way into their mind and altering it as you see fit.}
	
	\schoolDescribe{Malediction}{The Malediction school of magic contains those spells which have the primary intent to hurt, inflict harm on and otherwise incapacitate others. \\ Those who are experts in the field of Malediction are known as {\it Battlemages}. }{Hexes}{Hexes are a field which focusses on magic that directly harms the targeted person or object.}{Curses}{Unlike hexes, curses do not directly harm the target but instead incapacitates them, inhibits their capabilities, or otherwise reduces the threat they pose.}
   
   \schoolDescribe{Recuperation}{The Recuperation school of magic is often considered unglamourous, but those who can look past that can see that the ability to heal and protect yourself and others from harm is utterly invaluable. \\ Those who are proficient in the use of Recuperation magic are known as {\it Aegistes}. }{Healing}{Healing is, unsurprisingly, the study of magic used to heal the sick and wounded, break curses and project powerful positive energies.}{Warding}{Warding magic is almost entirely defensive in nature, allowing the caster to protect themselves and others from harm by casting powerful and long lasting shields and force-fields.}
	
	\schoolDescribe{Transfiguration}{The Transfiguration school of magic is focused on the transformation of the natural order - either by altering and reshaping the form of existing objects, or by summoning entirely new matter from thin air. \\Those who excel in Transfiguration are known as {\it Thaumaturges}.}{Alteration}{The alteration discipline studies the ability to change things from one form into another.}{Conjuration}{Conjuration magic is concerned with the ability to summon new objects and beings out of thin air, or to banish objects from existence.}
	
	\schoolDescribe{Dark Arts}{The Dark Arts school of magic encompasses magic which is frowned on in polite society, either because it involves truly evil spells - those which cannot be used without leaving scars on the soul, or those which tap into the dangerous and unfathomable energies of the dark and unspeakable things which lie just out of sight - under your bed and in the corner of your eye...\\Those who wield this forbidden magic are known as {\it Warlocks}.}{Necromancy}{A taboo discipline which contains deeply unpleasant spells which can only be cast by beings corrupted by evil - torture, death and worse lie in the domain of necromancy.}{Occultism}{Occultism is a rarely studied discipline that accesses and manipulates otherworldly energies originating from the Eldritch domain -- powerful, yet highly unpredictable.}

\normalsize

Every spell is assigned to be a member of one of these disciplines, which determines the skills that are necessary to cast it. 

\newpage
\section{Spell Types} 

In addition to falling into one of the seven Schools (a taxonomy based on the spell effect), every spell can also be categorised as a {\it type}, which is based on how the spell is cast. 

\subsection{Instant}

An instant spell is `cast and forget': as soon as you complete the requisite casting checks, the spell is `launched' (usually in the form of a magical bolt of light) towards the target. These bolts travel at speeds of 100m per cycle, which means in most cases, the effect is applied between the successful casting and the beginning of the next turn cycle.

Instant spells are denoted by the symbol \instSymb.

\subsection{Focus}

A focus spell is cast like an Instant spell, but may then be continued indefinitely, repeating the initial effects once per turn as long as you keep the spell active. No further checks are needed to continue the spell, but you must remain focussed and unless stated otherwise, the FP cost is deducted once per turn as well.

Because you must remain focussed, no further spells can be cast for the duration of this spell, and all subsequent movement checks are `minor', and no other actions (such as evasion) may be taken.   

Whilst maintaining a {\it Focus} spell you are considered {\it Distracted} and take the associated status effect. This renders you vulnerable to Critical Strikes, and upon taking damage you must pass a Willpower Resist check to maintain your concentration. 

You may end the spell effect at any time without it counting as an action. 

Focus spells are denoted by the symbol \concSymb.

\subsection{Ward}
A ward is (usually) a Recuperation spell that affects a large area. A ward may be centred on a fixed point or object, or may be centred on a moving location or even a sentient being. 

Wards, however, have an unfortunate habit of interfering with each other when used in unison. If two wards have a significant overlapping region of effect and the caster does not have the {\it Multiward} skill or an equivalent feat, there is a significant chance (determined by the GM) that both wards will collapse. 

The interference only applies if the wards are similar in magnitude and intent. For example, Hogwarts castle is a heavily warded region, but a small ward could be placed in a room without problem. Interference would only become a problem when a new castle-wide ward was attempted. 

Equally, interference only applies if the effects of the ward compound each other -- if they lie in opposition, then the usual spell mechanics are applied. For example, a character with a personal shield ward touches a beartrap ward -- neither ward collapses, but the beartrap ward is triggered, and the shield will attempt to protect the character. 

Ward spells are denoted by the symbol \wardSymb.

\subsection{Ritual}

A Ritual spell is a spell that requires a large amount of preparation -- be it meditation, drawing a summoning circle upon the ground, or performing a special dance. Each Ritual spell has a designated time that the ritual takes to complete, to cast a ritual spell you must spend this length of time preparing for the spell, and after the requisite time has passed, {\it then} you perform the check, and the spell effect is activated. If you fail the check, or choose to stop the ritual, i.e. to take another action, you must restart the ritual spell from the beginning. 

As with a focus spell, concentration is key to completing a ritual, and whilst performing a ritual, you are considered {\it Distracted}. 

Ritual spells are denoted by the symbol \ritSymb.

~


\chapter{Casting Spells}


All spells are cast by performing a `check' -- rolling a dice, and then adding on the associated skill modifiers and bonuses that apply for that spell, and comparing it to the Difficulty Value (DV) for the spell. If the Casting Check (CC) is greater than or equal to the DV and you have enough FP, then the spell is considered to be cast, and the effects are applied.  

When performing the check, you use a die of a size commensurate to your ability in that school of magic. As you become a more proficient magic-user, you get access to bigger dice, which enables you to cast more powerful spells, and increases the success rate and power of lower-level spells. 

\begin{center} \label{T:Dice}
	\begin{rndtable}{|c c c|}

	\bf Level & \bf  Name & \bf Die
	\\ 
	1 & Beginner & 1d6
	\\
	2 & Novice & 1d8
	\\
	3 & Adept & 1d10
	\\
	4 & Expert & 1d12
	\\
	5 & Master & 1d20
	\\ \hline
	\end{rndtable}
\end{center}

The size of dice you are allowed to use is determined on a school-by-school basis via the relevant skills discussed on page \pageref{S:Skills}.

\subsection{Spellbooks and Memory} \label{S:Memory}

There are two ways to cast a spell -- either by reading it from the pages of a book, or by being familiar enough with the spell that you can cast it from memory. 

For each of the 7 schools of magic, there are 5 textbooks. Each of these 35 textbooks is associated with a spell-level and a school, and contains all the spells in that school for that level. For example, the book {\it Dark Forces: A Guide to Self Protection} is a level 4 Hexes \& Curses book, and so contains all level level 4 Hexes \& Curses, but {\bf not} the 3rd level spells, for example. 

To cast a spell from a book, you must be holding a book which contains the specified spell in one hand, and your wand in another. You must then perform the checks, and the spell will be cast. Casting like this takes twice as long as normal, and you are considered {\it Distracted} whilst book-casting. Swapping books takes a minor action. 

If, however, you become familiar with a spell, then it is no longer necessary to have the book in your possession -- you can cast from memory. Spells cast from memory are almost always superior, and may be used as quickcast actions etc. Memory-casting is considered the `normal' way to cast, and all spellcasting rules discussed are assumed to apply to memory-casting. 

A spell is considered memorised when it has been cast successfully a number of times from a book in a `real life' scenario (i.e. you have to actually use the spell for its intended purpose, not just cast it wildly into thin air). People generally memorise a spell after casting it $N$ times, where:
$$ N = 6 -\text{INT modifier}~~~\text{(min 1)}$$


\subsection{Casting Checks}

The target roll of a spellcasting check (the DV) is the minimum value of the casting check (CC) which is required in order for the spell effect to be successfully initiated. It is determined by the level of the spell:


\begin{center}
	\begin{rndtable}{c c c c c c}
		~	&{\bf Beginner}	&	{\bf Novice}	&	{\bf Adept}	&	{\bf Expert}	&	{\bf Master}
		\\
		\cellcolor{\tablecolorhead} \bf DV: &	\DVBeg	&	\DVNov&	\DVAdp&	\DVExp	&	\DVMas
	\end{rndtable}
\end{center}


The check-type determines which ability modifiers are added onto the dice roll. The relevant modifier is determined by the Discipline that the spell originates from, according to the following prescription:
\def\xS{2}
\def\wS{2}
\begin{center}
	\begin{rndtable}{c m{\xS cm} p{\wS cm}}
	\bf School	&	\bf Discipline	&	\bf Attribute
	\\
	\school{Charms}{Elemental}{\ElCheck}{Kinesis}{\KinCheck}
	\\
	\school{Divination}{Telepathy}{\TelCheck}{Temporal}{\TemCheck}
	\\
	\school{Illusion}{Bewitchment}{\BewCheck}{Psionics}{\PsiCheck}
	\\
	\school{Malediction}{Hexes}{\HexCheck}{Curses}{\CurCheck}
   \\ 
   \school{Recuperation}{Healing}{\HeaCheck}{Warding}{\WarCheck}
	\\
	\school{Transfiguration}{Alteration}{\AltCheck}{Conjuration}{\ConCheck}
	\\
	\school{Dark Arts}{Necromancy}{\NecCheck}{Occultism}{\OccCheck}
	\end{rndtable}
\end{center}

You may, therefore, apply your INT modifier when casting a spell belonging to the Elemental Discipline. You may also ask your GM if it is appropriate to add on a Proficiency modifier to the check.

\subsection{Fortitude}


Casting spells is not as simple as waving your wands and saying the magic words -- it takes great mental clarity to cast, and you can become exhausted from casting difficult spells. This mental burden is enumerated through the Fortitude Points attribute. 

Each spell has an associated FP cost, which is deducted only after it is successfully cast. If the casting fails, then only half of the fortitude cost is deducted (rounded up).
 
You cannot cast a spell if it would send you into negative FP -- you must wait for your head to clear before attempting that spell.  

The fortitude cost of a given spell is determined by the spell type (Instant, Focus, Ritual etc.) and the difficulty of the spell, and if the spell is cast from memory or not. A book-cast spell has a 50\% higher FP cost than if the caster is familiar with the spell. 

The FP cost of a spell is numerically equal to the difficulty of a spell, prior to any skill modifications (i.e. a skill which reduces the difficulty of a certain spell does not reduce the FP of it, and vice versa), unless the spell is being book-cast, in which case use the bracketed values.  



{\footnotesize
\def\wFP{1}
\begin{center}
	\begin{rndtable}{p{1cm} |c |   c | c |c | c}
		~	&{\bf Beginner}	&	{\bf Novice}	&	{\bf Adept}	&	{\bf Expert}	&	{\bf Master}
		\\
		\cellcolor{\tablecolorhead} \bf Memory &	\FPEntry{\DVBeg}{1}	&	\FPEntry{\DVNov}{1}	&	\FPEntry{\DVAdp}{1}	&	\FPEntry{\DVExp}{1}	&	\FPEntry{\DVMas}{1}
		\\
		\cellcolor{\tablecolorhead} \bf Book  &	\FPEntry{\DVBeg}{2}	&	\FPEntry{\DVNov}{2}	&	\FPEntry{\DVAdp}{2}	&	\FPEntry{\DVExp}{2}	&	\FPEntry{\DVMas}{2}
	\end{rndtable}
\end{center}
}
\subsection{Power Points}

Some spells have the option to dedicate {\it Power Points} (PP) to their casting when cast from memory. Adding Power Points to a spell amplifies that spells effects, it might make it do more damage, last longer or have a wider area of effect. The effect of adding Power Points is described in the spell description. For example, a spell might state that it does (1+2$\times$PP)d4 damage. This means that, with zero PP added, the spell does d4 damage, with an additional 2d4 being added for every subsequent power point added. 

You must declare the number of power points you are dedicating to a spell before performing the check. Each power point dedicated increases the DV and FP of the casting by one.

You may only dedicate power points to a spell when casting from memory: you cannot do so when book casting. 

\subsection{Resisting Spells}

Even after a spell has successfully hit a target, it is possible for them to fight against the magic, reducing the effects and sometimes negating it entirely. 

This is normaly done by performing a {\it Resist} check before the spell effect is applied, and comparing it to the spellcaster\apos{} Resist DV. If the Resist is greater than or equal to the Resist DV of the spellcaster, the spell effect is modified as the spell description states. 

The Resist DV of a cast spell is given by:

$$\text{RDV} = 6 + \text{Expertise bonus}  + \text{POW modifier} + \text{other bonuses}$$

 ~

\chapter{Size, Shape \& Range}


\section{Spell Range} \label{S:Range}

\section{Spell Shapes}

Some spells produce bolts of energy which fly towards a target, whilst others project their energy into a given region, which are often classified via geometrical shapes: a {\it line}, a {\it cube}, a {\it sphere}, a {\it circle} a {\it cone} or a {\it cylinder}. These shapes may either originate around the caster, or from a point designated by the spell. 

\subsection{Circle}

A circular spell extends outwards from the point of origin in a 2D circular shockwave that lies parallel to the ground. The height of the shockwave above the ground is set by the point of origin, which is not included in the shockwave region (unless the caster chooses it to be). Because of its 2D nature, a circular spell can be avoided by ducking beneath it, or jumping over it -- it is only if the shockwave impacts you that the spell effect is applied. 

\subsection{Cone}

The point of origin of a cone is typically the caster's wand, and a cone extends outwards from the wand, in the direction that the wand is pointing. A cone extends forwards to the specified distance, and has a circular cross section, the radius of which is equal to the distance away from the point of origin (so it is a 45$^\circ$ cone).

 The point of origin of the cone is not considered part of the spell area. 

\subsection{Cube}

The point of origin for a cubic spell may be selected to be either the centre of the cube, or the centre of one of its 6 sides. The cube's side-length is specified by the spell effect. The cube point of origin is only affected by the spell if you choose the centre-origin.

\subsection{Cylinder}

A cylinder point of origin is specified to be a point on the ground, around which a circular cross section is drawn, and then a cylinder of energy rises up vertically to a specified height. Generally, a cylinder spell adjusts its size to an individual, and if not otherwise specified, the cylinder is 5cm wider than the target individual is wide, and 5cm taller than the target. The point of origin is affected by the spell. 

\subsection{Line}

A line extends in a straight path from the origin (a caster's wand) towards the target for a specified distance. Unless otherwise specified, the beam is considered to have the cross section equivalent to a pencil. The point of origin is not affected by the spell. 

\subsection{Sphere}

A sphere's point of origin lies at the centre, and the spell effect expands equally out in all directions from that point. Generally, the spell effect cannot penetrate into the ground or through solid objects (unless, for example, it is an explosion). The point of origin is affected by the spell. 


