\documentclass[CoreRulebook.tex]{subfile}



\chapter{Creating A Character}

The first step in playing the game is to create your own character. Your character can be whatever or whoever you want it to be -- this is your story after all. 

Your character is manifest in the game through your imagination, but in order to quantify the events occurng in the story, a character is formed from a mixture of several ingredients (of which imagination is a non-trivial part!), from which we can generate statistics and check values. 

Before you begin, it is helpful if you have an idea of the kind of character you wish to create -- your GM should tell you the rough outlines of the setting, which should help guide the type of character that will work well in the story. Do you want to play a powerfully destructive mage bent on crushing their enemies; or an investigator, pursuing the truth behind a mystery? 

You should also think about the backstory of your character -- what has led them to this point in their lives? Why are they going on this adventure?  

Once you have a good idea of the kind of character you wish to create, follow these steps to generate you character, and record the results on the Character Sheet.

\subsubsection*{1) Choose a (sub)Species}

Every character belongs to one of the Sapient races present in this world -- be they a human, a goblin, or a centaur. Some of the species (notably the humans) have several "sub-species" which take into account variation within the species. 

Belonging to a species confers your most basic characteristics: what do you look like? What magics -- if any -- do you have access to? 

Some species will also find themselves having a natural aptitude for certain skills, so it can be useful to think about how best to pair up your species and archetypes. The species available, and the abilities that they possess are discussed in Chapter \ref{C:Species}

\subsubsection*{2) Choose an Archetype}

An archetype broadly defines what your character does for a living -- but it is also much more than that. The archetype defines what role your character plays in the story, how they perceieve and interact with others and (perhaps more importantly) what skills they can develop as they progress. 

Your character recieves new skills and abilities by virtue of their archetype, so look ahead and see which skills you think will be the most useful (or, the most fun!) to develop along with your character. Archetypes are discussed in detail in Chapter \ref{C:Archetype} 

\subsubsection*{3) Determine Attribute Scores}

The 8 Attributes and 20 associated Proficiencies will be your main numerical way of interacting with the game world. These numbers encode your characters abilities. Your class and archetype will probably already have given your characters some abilities in this area, but all characters then get a choice of how to allocate some additional points. 

A low score in a given attribute will have a long-term effects on your character's abilities (though they can develop with time), so think carefully about how your abilities mesh with your character's personality and archetype. A particularly shy character, you might decide, will not be very brave, and thus will have a low Willpower. Attributes are discussed in more detail in Chapter \ref{C:Attributes}

\subsubsection*{4) Gather Your Equipment}

Your character will probably gain some supplies by virtue of their archetype, but you will also acquire some cash, as well as perhaps the most important item in your inventory: your wand. The item system is presented in chapter \ref{C:Items}.

\subsubsection*{5) Go adventuring!}

At this point, you will hopefully have a fully formed character, possibly working within a party of other characters. 

You will now be ready to set of on your adventure!

\chapter{Playable Species} \label{C:Species}
 
 Different magical races have different characteristics, abilities, and affinities with different kinds of magic. Each choice of race/species modifies your attribute values by a set amount and provides a pool of extra points which you can allocate to attributes at will, and some race-specific Abilities and Skills. 
 
It is generally impossible to switch species once a character has been created, except where it makes sense within the story (i.e. a human transitioning to a Vampire after being bitten). 

 
 \newcommand{\speciestable}[8]{
 % \begin{center}
 \begin{tabular}{|c|c|c|c|c|c|c|c|}
\hline
 ATH & FIN & SPR & CHR & INT & PER & POW & EVL
 \\
 \hline
 #1 & #2 & #3 & #4 & #5 & #6 & #7 & #8 
 \\ \hline
 \end{tabular}
 %\end{center}
 }
 
 
 
 \newcommand{\species}[6]{ 
 
	\begingroup
	\begin{tcolorbox}[  before skip=7pt plus 2pt,
			     	  boxrule=0pt,
			     	  boxsep=0pt,
			     	  toptitle=4pt,
			     	  left=0pt,
			     	  right=0pt,
			     	  bottom=5pt,
			     	  arc=0.5mm,
			     	  oversize=0pt,
			     	  colback=papyrus,
			     	 colbacktitle=titlered,
			     	colframe=titlered, title=\vspace{-4ex}
     				]
     		
	     		{\large \begin{center} \bf #1\end{center}}
	     		\vspace{-1.3ex}
	     		
	     		\dndlineLong
	     	
	     		
	     			
					
 {\bf Attribute Modifiers:}
 
 {\centering
 
 \speciestable#2
 
 }
 \begin{tabular}{l l}
 {\bf Base Speed:} & #3 metres per turn.
 \\
 {\bf Attribute points:}  & #4 extra points
 \\
 {\bf Skills:} & \parbox[t]{6cm}{\raggedright #5}
 \end{tabular}
 
 

   #6

					
	     		\normalsize
	     		\dndlineLong
     
     	\end{tcolorbox}
	\endgroup
 

 
 }
 
 %% { Pure-Blood humans get 4 extra points to spend, and two Beginner Skills to pick from those available}
 
 %ATH   FIN   SPR   CHR    INT    PER    POW      EVL     
 \species{Pure-Blood Human}{{-1}{-1}{+2}{+1}{+0}{-1}{+2}{+0}}{2}{2}{2 free skills}{ Typically the strongest magic users, pure-bloods find it  easiest to interact with other members of the magical community, whilst struggling to stay hidden amongst the muggles. Because of their lifelong reliance on magic, most pure-bloods are not very athletic or good with their hands.}


 \species{Half-Blood Human}{ {+0} {+1} {+2} {+0} {+1} {+0} {0} {0}    }{2.5}{3}{1 free skill}{Not as in-tune with magic as purebloods, nor as adept at blending in as the muggle-borns, half-bloods strike a balance between the two, matching their empathy with magical power. Being a half-blood does not inherently mean only one magical parent: it is a catchall term for those with a non-trivial amount of muggle relatives in the recent past. As a result, the vast majority of magical folk are Half-bloods.}
 
 
\species{Muggle-Born Human}{ {+1} {+0} {-1} {+1} {+0} {+1} {-1} {+0}   }{3}{3}{1 free skill}{Coming from a non-magical background, muggle-borns often lack in raw magical power. However, being brought up in a muggle household means that they are often adept at blending in. They are also used to getting by without magic, and will often find themselves more handy and athletic than those born into their magic.}
 
 \if\coreMode1
 
 
 
	\species{Metamorphmagus}{ {+0} {-2} {+1} {+3}  {-1}  {+0} {-1} {+0}  }{2}{2}{1 free skill \& {\it Morph} }{Metamorphmagi are a rare subspecies of wizard, capable of changing their shape at will. They are differentiated from animagi in that they can only mimic humanoid forms. }
	
	 \species{Half Giant}{ {+2} {-3} {+2} {+0} {-2} {+0} {-3} {0}   }{5}{2}{1}{Though rather a rare sight, the offspring of a giant and a human are not unheard of. Their magic is rather weak, but their giant blood gives them extreme strength, physical stamina and a large resistance to magical attacks. Half-giants often find it very hard to disguise themselves -- both from the muggles, and from their wizarding compatriots, who regard them with suspicion.}
	 
	 
	\species{House-Elf}{{-3} {+1}  {-2}  {+3}  {+0}  {+2}  {-3} {+0}   }{1}{2}{Behind the Scenes, Wandless Magic \& Apparate (Novice) }{Usually overlooked by all other sentient beings, house elves are in fact mischievous and quick-witted beings, with a natural propensity for illusion magic. All house-elves are born with the innate ability to apparate, and to move unseen and unheard through large crowds. Though many house elves submit themselves to a life of subservience, those who break free -- the Free Elves --  often find themselves employed in professions where stealth is a requirement.}
	
	
	\species{Goblin}{ {-2} {+4} {+0} {-2} {+5} {+0}  {-1} {0} }{1.5}{3}{Wandless Magic, Golden Touch \& Spellbinder (Novice)}{Goblins are highly intelligent non-humans, living alongside the magical world. Though viewed by many as inferior to their wizard brethren, Goblins are often far more powerful than humans expect, able to perform feats of magic without the use of a wand. They are expert artificers, able to create artefacts  and imbue them with immense powers. Goblins are also adept at the use of warding magic, with their most powerful work being displayed in the security systems at Gringott{\apos}s Bank. Goblins find it difficult (though not entirely impossible) to interact with the non-wizarding world.}
	 
	\newpage
	\species{Half-Veela}{ {+0} {+1} {+1}  {+3} {-1} {-4} {-2} {+2}   }{2}{2}{Fury\apos{}s Visage and 1 free skill}{Inheriting the enchanting beauty of the Veela, and the magical ability of humans, the half-Veela are often able to charm their way through most interactions, having a natural affinity for magic which persuades and influences others. When this does not work in their favour, however, they can call upon the Fury, transforming into a demonic form and possessing the ability to throw fireballs at their foes.}
	 
	\species{Werewolf}{ {+2} {+0} {+4} {-2} {-1}  {-1} {-1} {+5}   }{3}{2}{Wolfblood, Wolfmoon \& Corrupted Blood}{A werewolf is a human who has been afflicted by lycanthropy. At the full moon, a werewolf forgoes their human form, and takes the form of a monstrous wolf. They become a mindless killing machine, immeasurably strong and almost immune to magic, the beast within is a terrifying monster. The wolfblood dampens the magical abilities of the wizard, but gives them an increased resistance to magic in return.}
	 
	\species{Vampire}{{+0} {+0} {+5} {+3} {-2} {-4} {+3} {+7}   }{2.5}{2}{Vampric Drain, Night\apos{}s Child \& Corrupted Blood}{The corpse of an infected human, inhabited by an ancient, malevolent spirit, a vampire is a creature of the night. Vampires possess a great affinity for the dark arts, but are mortally afraid of the sun. Subsisting only on the blood of humanoids, vampires are feared and hated by all. Vampires often possess astonishingly powerful magic, but popular legends often educate mortals on their weaknesses.}
	 
	 
\fi
\subfile{Chapters/Archetypes}
\chapter{Main Attributes}\label{C:Attributes}

Attributes are the defining characteristics of your character. They enumerate how strong willed, how athletic and how popular your character is. These characteristics in turn define how good your character is at certain skills -- a character with a large willpower, for instance, will be good at combat magic, whilst a character with a low athleticism would find themselves unable to run away from threats!

\begin{itemize}
\item Athleticism (ATH):  Your character{\apos}s ability to exert themselves physically; to run, jump and deal physical attacks. Athletic characters are often harder to kill, and able to recover more quickly from wounds. 

\item     Finesse (FIN): Your character{\apos}s ability to execute actions with delicacy and precision. Picking pockets, hiding and casting spells in an unusual fashion require finesse in order to execute properly. 
  \item Spirit (SPR): Your character{\apos}s ability to face down external threats without flinching, to be sure of themselves, and to resist when the odds are against them. A character with a large spirit can often resist the effects of mind-altering spells, and can summon the strength to carry on when all others would have submitted. Typically considered the defining characteristic of Gryffindor House. 
    \item Charisma (CHR): The ability of a leader, and those who influence others. Charisma helps your character convince others of what you say, and make them like and trust you. Charisma also helps cast magic that alters their perception of reality, allowing you to convince them that it is real. A trait typically associated with Slytherin House.
   \item Intelligence (INT): Intelligence lets your character know that what they are doing is indeed the correct way forward. Though not always a substitute for raw magical power, an intelligent character learns spells more quickly, and can often be helpful in identifying threats (and their weak points). Typically considered the defining trait of Ravenclaw House.
    \item Perception (PER): Perception is your awareness of the world around you -- be it spotting a hidden tripwire, or the perception of other's emotions. Often going hand in hand with a kind and compasisonate soul, perception is considered the defining trait of Hufflepuff House. 
    \item Power (POW): Sometimes you don{\apos}t want to levitate a single brick our of a wall: you want the wall to explode. When finesse and trickery fail, throwing huge amounts of magical power at a problem can sometimes be beneficial. Some of the most spectacular magics require a large power,  but when a powerful spell goes wrong, the effects can be devastating and unforeseeable. 
   \item  Evil (EVL): Evil characters commit atrocities in the name of furthering their own goals. They will go to any lengths to get what they desire, including killing, maiming and torturing. Evil magics may grant you enormous powers, but are you willing to pay the price?
   
 \end{itemize}
 
 \subsection*{Proficiencies}
 
 Most Attributes are subdivided further into several {\it proficiencies}. These provide bonuses when the check is of a certain type, as discussed in more detail in section \ref{S:Profs}.
 \begin{itemize}[leftmargin = 0.2cm]
{\raggedright  
	\item ATH: Health, Speed, Strength

	\item FIN: Dexterity, Stealth, Precision

	\item SPR: Endurance, Willpower

	\item CHR: Deception, Performance, Persuasion

	\item INT:  Research, Arcane Knowledge, History, Flora \& Fauna

	\item PER: Surroundings, Emotions

	\item POW: (None)

	\item EVL: Chaos, Intimidation

		}
\end{itemize}



 \subsection*{Determining Abilities}
 
Perhaps the most important part of Character Creation is determining the attributes of your character. This is done by rolling a 2d6+2 ten times. This gives you 10 numbers between 4 and 14. You may then allocate 7 of these numbers to your non-EVL attributes at will. EVL defaults to zero at character creation. 
 
Generally speaking, you will want to allocate the largest of these values to the attributes which your character will rely on the most -- so a powerful magical warrior will get the largest values allotted to SPR and POW, whilst a seer would get larger values dedicated to PER. 
 
 All proficiency bonuses are set to zero at the beginning of character creation. 
\newpage 
 \section{Health \& Fortitude}
 
 Having determined your character's baseline attributes, we may now begin to see how this affects values relevant to gameplay -- namely, the Health and Fortitue of your character.
 
 \subsection*{Health}
 
Health is the physical status of your character: attacking a character lowers their health, and when the health points (HP) of a character reach zero, that character is killed. A character{\apos}s maximum health is calculated from:
$$\text{max HP} = 2 \times \text{ATH (health)} + \text{relevant bonuses}$$
When your HP limit is raised (say, by the {\it vita maxima} spell), your current HP is raised by the same amount. In contrast, when your HP ceiling is lowered, you only lose HP if the ceiling is lowered below your current health levels. It is never possible to have more than your maximum HP. 

\textbf{If your character is reduced to 0HP, then they acquire the Critical Condition status: they are completely immobilised, and will lose 1HP per turn. When you reach -10HP, you are dead, and nothing can bring you back. }

Characters regenerate health slowly as minor wounds heal, at a rate of 1HP per hour whilst not in combat (unless there is a status effect blocking the healing effect), increasing to 3HP per hour when asleep. This counter is reset every time your character takes additional damage. Status effects such as Serious Wound may impact the maximum HP which can be reached by natural healing, without external intervention. 

\subsection*{Fortitude}

Fortitude is a character{\apos}s ability to concentrate, which is necessary to cast spells and some other non-magic feats. Performing magic takes effort, and a character{\apos}s fortitude points (FP) will be slowly eroded by engaging in such mental effort.  A character{\apos}s maximum mental fortitude is calculated from:
$$\text{max FP} =  \text{SPR (willpower)} +\text{INT (arcane) }+  \text{relevant bonuses}$$
The same rules about raising/lowering the max level apply to Fortitude, as well as Health. Fortitude is used to cast spells, all spells have an associated fortitude cost written next to them -- as well as resist magic, and other actions which require intense concentration. You must subtract the relevant amount from your FP when performing such an action (plus or minus the appropriate amount for bonuses, power-boosted spells etc.)

When your FP reaches zero, your mind is exhausted, and so you will no longer be able to engage in such complex actions. Unlike HP, FP regenerates during combat; at a rate of 2FP per combat cycle where you do not cast a spell. Outside of combat, the regeneration rate is 8FP per hour, increasing to 20 per hour whilst asleep. 

Note that the maximum values of your HP and FP are dynamic values: when your ATH, SPR or INT values change, so do they. This is an important consideration when deciding which attributes to increase when levelling up. 



 
