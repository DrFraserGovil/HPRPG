\documentclass[CoreRulebook.tex]{subfile}


\chapter{Character Creation} \label{C:CharacterCreation}

The first step in playing the game is to create your own character. Your character can be whatever or whoever you want it to be. The following should serve as a guide to building a well-rounded and interesting player character. If you want to diverge from the ideas laid out here, you may be able to come to an agreement with your GM. 
\vspace{-2ex}

\section{Main Attributes}

Attributes are the defining characteristics of your character. They enumerate how strong willed, how athletic and how popular your character is. These characteristics in turn define how good your character is at certain skills -- a character with a large willpower, for instance, will be good at combat magic, whilst a character with a low athleticism would find themselves unable to run away from threats!

\begin{itemize}
\item Athleticism (ATH):  Your character{\apos}s ability to exert themselves physically; to run, jump and deal physical attacks. Athletic characters are often harder to kill, and able to recover more quickly from wounds. 

\item     Finesse (FIN): Your character{\apos}s ability to execute actions with delicacy and precision. Picking pockets, hiding and casting spells in an unusual fashion require finesse in order to execute properly. 
  \item Spirit (SPR): Your character{\apos}s ability to face down external threats without flinching, to be sure of themselves, and to resist when the odds are against them. A character with a large spirit can often resist the effects of mind-altering spells, and can summon the strength to carry on when all others would have submitted. Typically considered the defining characteristic of Gryffindor House. 
    \item Charisma (CHR): The ability of a leader, and those who influence others. Charisma helps your character convince others of what you say, and make them like and trust you. Charisma also helps cast magic that alters their perception of reality, allowing you to convince them that it is real. A trait typically associated with Slytherin House.
   \item Intelligence (INT): Intelligence lets your character know that what they are doing is indeed the correct way forward. Though not always a substitute for raw magical power, an intelligent character learns spells more quickly, and can often be helpful in identifying threats (and their weak points). Typically considered the defining trait of Ravenclaw House.
    \item Empathy (EMP): Empathy allows your character to understand other characters, to identify when something is wrong, and to be able to help. Empathy is often required for healing and protective magics. Though often mocked by dark wizards throughout history, it is empathetic magic -- love -- that has often conquered the most evil characters in history. Typically a trait associated with Hufflepuff House.
    \item Power (POW): Sometimes you don{\apos}t want to levitate a single brick our of a wall: you want the wall to explode. When finesse and trickery fail, throwing huge amounts of magical power at a problem can sometimes be beneficial. Some of the most spectacular magics require a large power,  but when a powerful spell goes wrong, the effects can be devastating and unforeseeable. 
   \item  Evil (EVL): Evil characters commit atrocities in the name of furthering their own goals. They will go to any lengths to get what they desire, including killing, maiming and torturing. Evil magics may grant you enormous powers, but are you willing to pay the price?
   
 \end{itemize}
 
 \subsection*{Proficiencies}
 
 Most Attributes are subdivided further into several {\it proficiencies}. These provide bonuses when the check is of a certain type, as discussed in more detail in section \ref{S:Profs}.
 \begin{itemize}[leftmargin = 0.2cm]
{\raggedright  
	\item ATH: Health, Speed, Strength

	\item FIN: Dexterity, Stealth, Precision

	\item SPR: Endurance, Willpower

	\item CHR: Deception, Performance, Persuasion

	\item INT:  Research, Arcane Knowledge, History, Flora \& Fauna

	\item EMP:  Percpetion, Understand Other, Healing

	\item POW: (None)

	\item EVL: Chaos, Intimidation

		}
\end{itemize}



 \subsection*{Determining Abilities}
 
Perhaps the most important part of Character Creation is determining the attributes of your character. This is done by rolling a 2d6+2 ten times. This gives you 10 numbers between 4 and 14. You may then allocate 7 of these numbers to your non-EVL attributes at will. EVL defaults to zero at character creation. 
 
Generally speaking, you will want to allocate the largest of these values to the attributes which your character will rely on the most -- so a powerful magical warrior will get the largest values allotted to SPR and POW, whilst a healer gets the largest value given to EMP. 
 
 All proficiency bonuses are set to zero at the beginning of character creation. 
\newpage 
 \section{Health \& Fortitude}
 
 Having determined your character's baseline attributes, we may now begin to see how this affects values relevant to gameplay -- namely, the Health and Fortitue of your character.
 
 \subsection*{Health}
 
Health is the physical status of your character: attacking a character lowers their health, and when the health points (HP) of a character reach zero, that character is killed. A character{\apos}s maximum health is calculated from:
$$\text{max HP} = 2 \times \text{ATH (health)} + \text{relevant bonuses}$$
When your HP limit is raised (say, by the {\it vita maxima} spell), your current HP is raised by the same amount. In contrast, when your HP ceiling is lowered, you only lose HP if the ceiling is lowered below your current health levels. It is never possible to have more than your maximum HP. 

\textbf{If your character is reduced to 0HP, then they acquire the Critical Condition status: they are completely immobilised, and will lose 1HP per turn. When you reach -10HP, you are dead, and nothing can bring you back. }

Characters regenerate health slowly as minor wounds heal, at a rate of 1HP per hour whilst not in combat (unless there is a status effect blocking the healing effect), increasing to 3HP per hour when asleep. This counter is reset every time your character takes additional damage. Status effects such as Serious Wound may impact the maximum HP which can be reached by natural healing, without external intervention. 

\subsection*{Fortitude}

Fortitude is a character{\apos}s ability to concentrate, which is necessary to cast spells and some other non-magic feats. Performing magic takes effort, and a character{\apos}s fortitude points (FP) will be slowly eroded by engaging in such mental effort.  A character{\apos}s maximum mental fortitude is calculated from:
$$\text{max FP} =  \text{SPR (willpower)} +\text{INT (arcane) }+  \text{relevant bonuses}$$
The same rules about raising/lowering the max level apply to Fortitude, as well as Health. Fortitude is used to cast spells, all spells have an associated fortitude cost written next to them -- as well as resist magic, and other actions which require intense concentration. You must subtract the relevant amount from your FP when performing such an action (plus or minus the appropriate amount for bonuses, power-boosted spells etc.)

When your FP reaches zero, your mind is exhausted, and so you will no longer be able to engage in such complex actions. Unlike HP, FP regenerates during combat; at a rate of 2FP per combat cycle where you do not cast a spell. Outside of combat, the regeneration rate is 8FP per hour, increasing to 20 per hour whilst asleep. 

Note that the maximum values of your HP and FP are dynamic values: when your ATH, SPR or INT values change, so do they. This is an important consideration when deciding which attributes to increase when levelling up. 

 \newpage
 \section{Playable Species}
 
 Different magical races have different characteristics, abilities, and affinities with different kinds of magic. Each choice of race/species modifies your attribute values by a set amount and provides a pool of extra points which you can allocate to attributes at will, and some race-specific Abilities and Skills. 
 
It is generally impossible to switch species once a character has been created, except where it makes sense within the story (i.e. a human transitioning to a Vampire after being bitten). 
 \newline
 
 \newcommand{\speciestable}[8]{
 % \begin{center}
 \begin{tabular}{|c|c|c|c|c|c|c|c|}
\hline
 ATH & FIN & SPR & CHR & INT & EMP & POW & EVL
 \\
 \hline
 #1 & #2 & #3 & #4 & #5 & #6 & #7 & #8 
 \\ \hline
 \end{tabular}
 %\end{center}
 }
 \newcommand{\species}[4]{ \vspace{1.4ex}
 \centerline{ \large \textbf{#1} } 
 \vspace{1.2ex}
 \begin{minipage}[c]{0.5 \textwidth}
  \raggedright #2 
  \end{minipage} \qquad 
  \begin{minipage}{0.5 \textwidth}
  
 \vspace{0.1ex}
#3, on top of the following basic attributes:

\small
\vspace{1.5 ex}
\speciestable#4
 \vspace{3ex}
 \end{minipage}
 }
 
 %ATH   FIN   SPR   CHR    INT    EMP     POW      EVL     
 \species{Pure-Blood Human}{ Typically the strongest magic users, pure-bloods find it  easiest to interact with other members of the magical community, whilst struggling to stay hidden amongst the muggles. Because of their lifelong reliance on magic, most pure-bloods are not very athletic or good with their hands.}{ Pure-Blood humans get 4 extra points to spend, and two Beginner Skills to pick from those available}{{-1}{-1}{+2}{+1}{+0}{-1}{+2}{+0}}


 \species{Half-Blood Human}{Not as in-tune with magic as purebloods, nor as adept at blending in as the muggle-borns, half-bloods strike a balance between the two, matching their empathy with magical power. Being a half-blood does not inherently mean only one magical parent: it is a catchall term for those with a non-trivial amount of muggle relatives in the recent past. As a result, the vast majority of magical folk are Half-bloods.}{Half-Blood humans get 3 extra points to spend, and two Beginner Skills to pick from those available}{ {+0} {+1} {+2} {+0} {+1} {+0} {-1} {0}    }
 
 
 \species{Muggle-Born Human}{Coming from a non-magical background, muggle-borns often lack in raw magical power. However, being brought up in a muggle household means that they are often adept at blending in. They are also used to getting by without magic, and will often find themselves more handy and athletic than those born into their magic.}{Muggle-Borns get 5 extra points to spend, and one Beginner  Skill to pick from those available}{ {+1} {+0} {-1} {+1} {+0} {+1} {-1} {+0}   } 
 \species{Half Giant}{Though rather a rare sight, the offspring of a giant and a human are not unheard of. Their magic is rather weak, but their giant blood gives them extreme strength, physical stamina and a large resistance to magical attacks. Half-giants often find it very hard to disguise themselves -- both from the muggles, and from their wizarding compatriots, who regard them with suspicion.}{Half-Giants get 3 extra points to spend, and one Beginner Skill to pick from those available, as well as the Enormous Size ability}{ {+2} {-3} {+2} {+0} {-2} {+0} {-1} {0}   }
 
 \species{House-Elf}{Usually overlooked by all other sentient beings, house elves are in fact mischievous and quick-witted beings, with a natural propensity for illusion magic. All house-elves are born with the innate ability to apparate, and to move unseen and unheard through large crowds. Though many house elves submit themselves to a life of subservience, those who break free -- the Free Elves --  often find themselves employed in professions where stealth is a requirement.}{House Elves get 2 extra points to spend and start with the Apparate (Adept) and Wandless Magic (Novice) skills, and the Behind the Scenes ability}{{-3} {+1}  {-2}  {+3}  {+0}  {+2}  {-3} {+0}   }
 
 \species{Goblin}{Goblins are highly intelligent non-humans, living alongside the magical world. Though viewed by many as inferior to their wizard brethren, Goblins are often far more powerful than humans expect, able to perform feats of magic without the use of a wand. They are expert artificers, able to create artefacts  and imbue them with immense powers. Goblins are also adept at the use of warding magic, with their most powerful work being displayed in the security systems at Gringott{\apos}s Bank. Goblins find it difficult (though not entirely impossible) to interact with the non-wizarding world.}{Goblins get 3 extra points to spend on attributes, as well as the Artificer (Novice), Wandless Magic (Novice) and Warder (Novice) skills}{ {-2} {+4} {+0} {-2} {+5} {+0}  {-1} {0} }
 
 \species{Half-Veela}{Inheriting the enchanting beauty of the Veela, and the magical ability of humans, the half-Veela are often able to charm their way through most interactions, having a natural affinity for magic which persuades and influences others. When this does not work in their favour, however, they can call upon the Fury, transforming into a demonic form and possessing the ability to throw fireballs at their foes.}{Half-Veela get 5 extra points to spend and start with the Fury ability}{ {+0} {+1} {+1}  {+3} {-1} {-4} {-2} {+2}   }
 
 \species{Werewolf}{A werewolf is a human who has been afflicted by lycanthropy. At the full moon, a werewolf forgoes their human form, and takes the form of a monstrous wolf. They become a mindless killing machine, immeasurably strong and almost immune to magic, the beast within is a terrifying monster. The wolfblood dampens the magical abilities of the wizard, but gives them an increased resistance to magic in return.}{Werewolves get 3 extra points to spend, as well as the WolfBlood ability, and one other Beginner skill}{ {+2} {+0} {+4} {-2} {-1}  {-1} {-1} {+5}   }
 
 \species{Vampire}{A human who has contracted the disease sanguinus vampiris, a vampire is a creature of the night, possessing a great affinity for the dark arts, but mortally afraid of the sun. Subsisting only on the blood of humanoids, vampires are feared and hated by all. Vampires often possess astonishingly powerful magic, but can be defeated by Holy Wards, wooden stakes, and garlic. It is also said that vampires cannot cross a threshold that they have not been invited over.}{Vampires get 2 extra points, as well as the Drain Life and Night{\apos}s Child abilities}{{+0} {+0} {+5} {+3} {-2} {-4} {+3} {+7}   }
 
 
 
 \onecolumn
 \section{Species Abilities}
 
 Abilities are those traits unique to a given species. 

\def\w{11}
 \begin{center}
 \begin{tabular}{|m{3.4cm}|m{2.3cm}|m{\w cm}|}
 \hline
 Name 		& Species 		& Effect
 \\ \hline \hline
 {\bf Behind the Scenes }		& House Elf 		& \parbox[t]{\w cm}{For better or for worse, you are beneath most people{\apos}s attention. You can get things done whilst nobody else is paying attention, and are able to move around without being spotted. \par  FIN (stealth) +3. You may also, once per day, perform a second action whilst another character is executing their turn (including the GMs). Apparation checks get a + 3 bonus. }
 \\
 \hline {\bf Enormous Size} 	& Half-Giant 	& \parbox[t]{\w cm}{You are enormous. You cannot fit down narrow passageways, and it is very difficult for you to go without being recognised. However, you are also enormously strong, and very hard to hurt. \\ ATH (strength) + 3, ATH (health) + 3, all FIN proficiencies: -1, SPR (endurance) + 3, CHR (deception) -2}
 \\ \hline
 {\bf Fury} 					& Half-Veela		&   \parbox[t]{\w cm}{Shed your beautiful facade and reveal the Fury within. The Fury is a powerful beast which is nearly immune to magic, and can throw powerful fireballs. \\ In human form, get FIN (persuasion) + 4.  Once per day, take a temporary stat boost, ATH: + 2, STR: +4, SPR: + 2, POW: + 4, CHR: - 5. Get a + 3 boost to resist magic checks.  Replace all active spells with Fury{\apos}s Fire. These changes revert when retaking human form. }
 \\ \hline
 {\bf Drain Life} 				& Vampire		& \parbox[t]{\w cm}{You can drain the life-force of your enemies, using it to restore your own health. \\ When within close-combat range, can deal 2d6 necrotic damage to the enemy, and restore yourself the same number of HP that you remove. Only works on living beings.  }
 \\ \hline
 {\bf Night{\apos}s Child} 			& Vampire		& \parbox[t]{\w cm}{As one of the undead, the raw sun drastically weakens your power, opens up your defences, and reduces your ability to think clearly. For every hour exposed to the sun, suffer a -1 hit to SPR, INT and POW. Magical defences are 50\% less effective. This counter is reset after feeding on a human.  \\ You also gain the ability to see in the dark. }
 \\ \hline
 {\bf Wolf Blood}			& Wolfblood	& \parbox[t]{\w cm}{When the full moon rises, you take on the form of a monstrous, mindless wolf -- unless a wolfsbane potion is applied.  For 12 hours, your character becomes the Beast Inside, and is placed under the control of the Game Master.  \\ Silver is a deadly poison to you, and wounds caused by you are infectious. Even in human form, get SPR (endurance) +3 and ATH (speed) +2}
\\ \hline
 
 \end{tabular} 
 \end{center}
 
 \newpage
 \section{Character Role}
 
 The role (also known as the {\it class}) of your character determines which of the major character archetypes your character falls into. 
 
 \twocolumn 
 \section{Wands}

All witches and wizards start off with their very own magic wand. The wand chooses the wizard, not the other way around,  so the process for selecting your wand is to roll two d6 successively. The first roll determines the wood your wand is made of, the second determines the core. 

Different materials have an affinity with different kinds of magic, and make casting those spells easier. Wood makes the spell type easier to cast (+1 to checks), and the core reduces the mental strain of casting that class of spell (-1 FP cost). 
 \footnotesize
 \begin{center}
\tablealternate
 \begin{tabular}{|c|c|c|c|}
 \tablehead
 \hline
 \bf Roll & \bf Magic School & \bf Wood& \bf Core
 \\ \hline \hline
 1 & Defensive & Apple & Pheonix feather
 \\\hline 
 2 & Hexes \& Curses & Holly & Dragon heartstring
 \\\hline 
 3 & Divination & Beech & Unicorn Tail hair
 \\\hline 
 4 & Transifguration & Oak & Thunderbird feather
 \\\hline 
 5 & Charms & Hawthorn & Kelpie hair
 \\\hline 
 6 & Illusion & Hazel & Veela hair
 \\\hline 
 - & Dark Arts & Human Bone & Dementor Robe
 \\ \hline
 \end{tabular}
 \end{center}
\normalsize
 If your original wand is destroyed or lost, you need to find someone who can sell (or make) you a new one, and perform the selection process anew. 
 
The only way to access the 7th and final category of wand is to have an EVL greater than 8. This then bypasses all other wand selection checks, and your wand is necessarily evil. It should of course be noted that wandmakers aren{\apos}t too happy to sell these evil objects -- you might have to cut a few bits off in order to sufficiently motivate them.  
 
 \section{Final Setup}
 
Once you have selected your wand, you are ready to begin the final phases of character creation: picking your skills, and then your initial spells, for which you should refer to the relevant chapters later on in this text.

Many of the playable species get a number of extra skills that they can add to their character at creation, after the Profession or House skill has been added: this is the time for you to choose them. You can only add LVL 1 skills to your character at this time. 

After picking these skills, you are ready to pick your first set of spells! Some professions get a spell or two assigned to them at first, which are automatically added to your character. On top of this, you may then pick \textbf{five spells from the schools of magic that you can cast}, and add them to you character. After this stage, learning new spells is covered by the process detailed in the Spells section of this document. 

You will also get an initial set of clothes (usually plain black robes, unless otherwise stated by your GM), as well as any of the items assigned to you by your profession, which you should add to your inventory.  All characters also get 50 gold pieces\footnote{Harry Potter technically uses Knuts, Sickles and Galleons as the currency, but with 29 Knuts to a sickle, and 17 sickles to a galleon, that just sounds like hard work. We{\apos}ll stick to just using ``gold pieces\apos\apos. } at the start of the game, except Goblins and House-Elves:  Goblins get 100, whilst House-Elves get only 25. 
\newline
~
\newline
~
\begin{center}
{\Huge \bf You are now ready to begin!}
\end{center}
 
