\documentclass[CoreRulebook.tex]{subfile}

\chapter*{Introduction \& Core Mechanic}
 \addcontentsline{toc}{part}{Introduction}
 
 
\if\coreMode0
	
	{\bf IMPORTANT:}
	
	Note that this is only the {\it Basic Rulebook}. As such, it contains only introductory information needed to get beginner players involved in the game. Significant chunks of material (notably all non-student Archetypes, and spells above Novice level) have been omitted to ensure a smooth adoption for those wanting to get to grip with the game, without being overshadowed by large amounts of information.
	
	\vspace{1 cm}
	
\fi
 
Harry Potter \& The Role Playing Game is a freeform role playing game, where you take control of a character living in the world of Harry Potter. All you need to play this game is a pen, some paper, and a handful of dice – the rest is up to your imagination. If it is reasonable for you character to do something, then you may direct them to do that – to run towards evil head on and fight injustices, to run away and save yourself, or even to become the malevolent evil itself; the world really is your oyster.  

Of course, in order to separate this from the games we all played as children, where actions were completed simply by claiming that it is so, this rulebook provides a framework for resolving the success and subsequent effects of the actions that you wish your character to undertake, as well as keeping track of the various abilities and attributes that your character possesses.

To this end, whenever the result of an action is uncertain, be it an attack, an attempt to persuade someone, or checking for clues, your character must rely on a \key{check}. This takes into account the abilities, skills, afflictions and bonuses that your character has accumulated over their lifetime: all of which determines the number of dice that you roll. \index{Check}

You then let loose with the dice, and roll away. You then compare each dice with a the \key{Difficulty Value (DV)} assigned to the task - each dice which meets or exceeds the DV counts as a \key{success}. The more successes you have, the more powerful the action is. The outcome of an action is therefore decided by the balance between the difficulty of the action, and the number of dice you are rolling. \index{DV}
 
But how is the DV of an action determined? This is where the \key{Game Master (GM)} comes into play. The GM is one of the players who has agreed to act as a referee for the story that the players wish to tell. The GM is the overseer of the narrative: they are responsible for describing the encounters, adventures and environments that the other players are taking part in. Though the GM controls the characters who oppose the players, the GM does not ‘win’ if these enemy characters prevail – the purpose of the GM is not to defeat the \key{Player Characters (PCs)}, but to drive the story and present interesting and challenging scenarios for them to overcome. \index{Game Master}

As a corollary to this, there are only completely two unassailable rules in this book:\index{Rules!Golden}

\begin{enumerate}[itemsep = 0em]
	\item There are no unassilable rules (besides these two)
	\item The Game Master's judgments are always correct and final
\end{enumerate}

The  GM has complete freedom to override the rules in this booklet, in the name of an interesting yet challenging story. Of course, if they have simply misread or misremembered a rule, they might self-correct when this is pointed out to them -- however, in a true conflict between what the rules say and what the GM says, the GM wins every time.

Of course, this is not to say that the GM should always use this power in opposition to the players. These rules are only the basic framework upon which the GM and PCs weave their narrative -- if a PC wishes to do something that is not covered in this manual, then the GM can use their power (`GM fiat') to work with the PCs to determine how best to resolve this, whilst having fun. 

With this basic set of rules in mind, the flow of the game is rather simple:

\begin{enumerate}
	\item {\bf The GM describes the environment},  they may describe the sights, sounds and smells that your PCs would experience in the situation that they find themselves in. The GM should give the basic lay of the land -- the things that every person in that situation would be able to spot. 
	\item {\bf The players decide what they would like to do}, they might decide that they'd like to investigate a certain aspect of the room more carefully, or they might decide to cast a spell, or hit somebody with a big stick. They then inform the GM of their final decisions
	\item {\bf The players and GM work together to resolve these actions}, some resolutions are simple (`you walk through the door', `you drink the potion'), others may require checks and the GM thinking carefully about the success of such an action. In some `modes of play' (i.e. combat), this resolution needs to be done in a structured fashion. Other times, it may be more fluid and conversational.
 	\item {\bf The GM narrates the result of this action}, telling the players what happened and how the success (or failure) of their actions impacted the world around them. This cycle then continues, as you build up your narrative.
\end{enumerate} 


\section{Using these Rules}

For the most part, these rules sections provide nothing more than a list of when and where to roll dice, and how many dice you can roll at any given moment -- of course -- there's rather more to it than that!

Part I of this guide details with the important act of character creation: the various routes that one takes to build and then grow a character, including the playable races, character Archetypes key statistics such as Health. Part II focusses in more detail on Actions, and the outcomes of those actions, as well as a more in-depth look at the 8 character attributes. Part III focuses on Items - physical objects that you can acquire, create and use throughout your adventure. The final part, Part IV, deals with the mystical arts of magic, spellcasting and the arcane powers that reside in this world. 

After the bulk of this rulebook, you will also find a large number of lists, tables and appendices. These contain a wide variety of important information that you may need along your journey, such as the exact details of the myriad spells and potions in this world, detailed descriptions of the professions and Archetypes that your character may fall into, and many other such important statistics. It is advised that you pick these parts up as you go along, rather than try and absorb all the knowledge at once. 

The GM also has their own rulebook, the Game Master's Guide, which contains some rules, instructions and a compedium of information which might want to be kept secret from the players so that they can discover it along with their characters, and to prevent `metagaming'. Players should only view this document with the GM's consent.


