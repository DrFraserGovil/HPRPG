\documentclass[CoreRulebook.tex]{subfile}

\chapter*{Introduction \& Core Mechanic}
 \addcontentsline{toc}{part}{Introduction}
 
 
\if\coreMode0
	
	{\bf IMPORTANT:}
	
	Note that this is only the {\it Basic Rulebook}. As such, it contains only introductory information needed to get beginner players involved in the game. Significant chunks of material (notably all non-student Archetypes, and spells above Novice level) have been omitted to ensure a smooth adoption for those wanting to get to grip with the game, without being overshadowed by large amounts of information.
	
	\vspace{1 cm}
	
\fi
 
Harry Potter \& The Role Playing Game is a freeform role playing game, where you take control of a character living in the world of Harry Potter. All you need to play this game is a pen, some paper, and a set of dice – the rest is up to your imagination. If it is reasonable for you character to do something, then you may direct them to do that – to run towards evil head on and fight injustices, to run away and save yourself, or even to become the malevolent evil itself; the world really is your oyster.  

Of course, in order to separate this from the games we all played as children, where actions were completed simply by claiming that it is so, this rulebook provides a framework for resolving the success and subsequent effects of the actions that you wish your character to undertake, as well as keeping track of the various abilities and attributes that your character possesses.

To this end, whenever the result of an action is uncertain, be it an attack, an attempt to persuade someone, or checking for clues, your character must rely on a `check'. This `check' takes into account the abilities, skills, afflictions and bonuses that your character has accumulated over their lifetime, and then adds in an element of randomness, through a dice roll, all of which are combined into a single `check value' (or CV). 
 
 If this CV surpasses a certain minimum requirement (called the ‘difficulty value’ of the action, or DV) then the action is said to succeed. If you do not meet the minimum requirement, the action fails – and you may face repercussions!
 
But how is the DV of an action determined? This is where the Game Master (GM) comes into play. The GM is one of the players who has agreed to act as a referee for the story that the players wish to tell. The GM is the overseer of the narrative: they are responsible for describing the encounters, adventures and environments that the other players are taking part in. Though the GM controls the characters who oppose the players, the GM does not ‘win’ if these enemy characters prevail – the purpose of the GM is not to defeat the player characters (PCs), but to drive the story and present interesting and challenging scenarios for them to overcome. 

As a corollary to this, the only completely unassailable rule in this book is that {\bf the Game Master's judgments are always correct and final}. The  GM has complete freedom to override the rules in this booklet, in the name of an interesting yet challenging story. Of course, if they have simply misread or misremembered a rule, they might self-correct when this is pointed out to them -- however, in a true conflict between what the rules say and what the GM says, the GM wins every time.

Of course, this is not to say that the GM should always use this power in opposition to the players. These rules are only the basic framework upon which the GM and PCs weave their narrative -- if a PC wishes to do something that is not covered in this manual, then the GM can use their power (`GM fiat') to work with the PCs to determine the outcome. Equally, if a player wants to create a PC with traits not covered in the character creation chapter, the GM may be willing to work with the PC to create the appropriate rules. 

With this basic set of rules in mind, the flow of the game is rather simple:

\begin{enumerate}
	\item {\bf The GM describes the environment},  they may describe the sights, sounds and smells that your PCs would experience in the situation that they find themselves in. The GM should give the basic lay of the land -- the things that every person in that situation would be able to spot. 
	\item {\bf The players decide what they would like to do}, they might decide that they'd like to investigate a certain aspect of the room more carefully, or they might decide to cast a spell, or hit somebody with a big stick. They then inform the GM of their final decisions
	\item {\bf The players and GM work together to resolve these actions}, some resolutions are simple (`you walk through the door', `you drink the potion'), others may require checks and the GM thinking carefully about the success of such an action. In some `modes of play' (i.e. combat), this resolution needs to be done in a structured fashion. Other times, it may be more fluid and conversational.
 	\item {\bf The GM narrates the result of this action}, telling the players what happened and how the success (or failure) of their actions impacted the world around them. 
\end{enumerate} 

This cycle then continues, as you build up your narrative.


\section{Computing Checks} \label{S:Checks}

Computing the CV of a given check is perhaps the most important mechanic for playing this game (beyond raw imagination), so it is worthwhile to consider this in more detial. 

There are three kinds of `check' that you might be asked for over the course of playing the game:

\begin{itemize}
	\item {\bf Ability Check:} this is used whenever you attempt to use your internal skills and abilities to interact with the world. Leaping over chasms, interacting with a troublesome suspect or casting a spell would all fall under the umbrella of an ability check.  
	\item {\bf Resist Check:} this is used whenever an effect is imposed on you, and you wish to try and negate or otherwise mitigate it. Leaping out of the way of a falling boulder, gritting your teeth against a mind-control spell, or outwitting your opponent at the last second would all use a Resist check. 
	\item {\bf Outcome Check:} this is used to determine the effect of a given action, most commonly in the form of computing the damage caused by an attack, though you may also perform an Outcome check to determine how many people are affected by a given spell, or how much health is restored upon drinking a potion. 
\end{itemize}

A check has two primary ingredients: the {\it roll}, and your {\it bonuses}. 

\subsection{Dice Rolls}

The roll is, as you might expect, the outcome of a dice roll. A roll can occur on one of a number of different polyhedral dice: a d4, d6, d8, d10, d12 or d20, with the number simply signifying the number of sides that the dice has (so a d6 is the usual cubic dice). You may also see the $d$ preceeded by another number, i.e. $n$d6. This tells you to roll the d6 $n$ times. More unusual checks might call for a d100 to be rolled - this can be managed by rolling 2d10 and multiplying the result of one of the dice by 10. 

Unless otherwise specified, you should generally assume that the check being asked for is using the d20 dice. This is true for all Ability Checks and Resist checks. The remaining dice are almost exclusively used during Outcome checks - weapons, spells, potions and all other effects tell you which dice to roll in their description. 


\subsection{Bonuses \& Modifiers}

Most of the time, you are not simply relying on dumb luck when taking an action, your character has some innate or learned abilities which makes them more or less likely to complete a given action. To model this, many checks use {\it bonuses} or {\it modifiers} which increase or decrease the result of the dice roll in accordance with your abilities. 

For Ability and Resist checks, each character has a number of bonuses called `Attribute Modifiers'. These numbers are derived from your character's {\it attributes}, the key defining traits of your character. There are 8 of these attributes: {\bf \attPhys, \attFin, \attSpr, \attChr, \attInt, \attPer, \attPow} and {\bf \attEvl}. They typically take values between 5 and 18. A larger attribute score will give you a larger modifier in that attribute (and hence a bonus on these checks), and a smaller value can result in a {\it negative} modifier, making these checks harder. An Ability or a Resist check is (nearly) always specified to be a check related to one of these 8 attributes, which tells you which modifier to use. 

For example, if you were attempting to escape from a ravenous Grindylow on foot, your GM would rule that this falls under the domain of your Fitness attribute. They would therefore ask for a {\it Fitness check}, you would then roll a d20 dice, and add on your Fitness modifier, plus any addiitonal bonuses you might have from spells or items in your possession:

$$\text{CV} = 1\text{d}20 + \text{intelligence modifier} + \text{other bonuses} $$

The total value is called the `Check Value' (CV). If this value meets or exceeds the limit set by the GM (the difficulty value, or DV), then you succeed on the action, and they will narrate the outcome. Conversely, if you fail the check, then the action will fail. If you fail by a significant margin, then the action will not only not happen, it might backfire on you spectacularly, and rather than blasting your opponent into oblivion, you might find yourself vomiting slugs over the school field… 

For Outcome Checks, you should be informed of the additional bonuses to be added on in the description of the item or being which is leading to the check being performed. Sometimes these checks are simple numerical bonuses (a health potion might restore 2d4 + 4 health, for example), whislt other times the bonuses are also derived from your modifiers (a axe deals additional damage equal to your fitness modifier, for example.)

\newpage
\section{Using these Rules}

For the most part, these rules sections provide nothing more than a list of when, how and under what circumstances you can aqcuire the various bonuses and penalties to plug into the above equation, although -- of course -- there's rather more to it than that!

Part I of this guide details with the important act of character creation: the various routes that one takes to build and then grow a character, including the playable races, character Archetypes key statistics such as Health. Part II focusses in more detail on Actions, and the outcomes of those actions, as well as a more in-depth look at the 8 character attributes. Part III focuses on Items - physical objects that you can acquire, create and use throughout your adventure. The final part, Part IV, deals with the mystical arts of magic, spellcasting and the arcane powers that reside in this world. 

After the bulk of this rulebook, you will also find a large number of lists, tables and appendices. These contain a wide variety of important information that you may need along your journey, such as the exact details of the myriad spells and potions in this world, detailed descriptions of the professions and Archetypes that your character may fall into, and many other such important statistics. It is advised that you pick these parts up as you go along, rather than try and absorb all the knowledge at once. 

The GM also has their own rulebook, the Game Master's Guide, which contains some rules, instructions and a compedium of information which might want to be kept secret from the players so that they can discover it along with their characters, and to prevent `metagaming'. Players should only view this document with the GM's consent.


