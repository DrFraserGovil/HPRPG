\documentclass[CoreRulebook.tex]{subfile}

\chapter{Core Mechanic}

Harry Potter \& The Role Playing Game is a freeform role playing game, where you take control of a character living in the world of Harry Potter. All you need to play this game is a pen, some paper, and a set of dice – the rest is up to your imagination. If it is reasonable for you character to do something, then you may direct them to do that – to run towards evil head on and fight injustices, to run away and save yourself, or even to become the malevolent evil itself; the world really is your oyster.  

In this game, whenever the result of an action is uncertain – be it an attack, an attempt to persuade someone, or checking for clues – your character must rely upon their abilities and skills, as well as sheer luck.
An action is deemed successful if your natural abilities, and the outcome of a dice roll (called a ‘check’)  surpasses a certain minimum requirement (called the ‘difficulty’ of the action). If you do not meet the minimum requirement, the action fails – and you may face repercussions!

A check is often specified as “ndx”. Here n refers to the number of dice to use, and “dx” refers to the type of dice to be use – x is the number of sides on the dice. A d10 is a ten-sided dice, for example.
Hence, to complete an action; for example, a “2d10 Intelligence Check”, we would roll two, ten-sided dice, and then calculate:

$$\text{check} = \text{dice roll} + \text{intelligence modifier} + \text{bonuses} $$

If the check value exceeds the minimum threshold – the difficulty – then the action is successful. If the check exceeds the threshold by a significant margin, the action might be more than just ‘successful’, and might have benefits beyond that which you originally intended!

Conversely, if you fail the check, then the action will fail. If you fail by a significant margin, then the action will not only not happen, it might backfire on you spectacularly, and rather than blasting your opponent into oblivion, you might find yourself vomiting slugs over the school field… 

The single most important rule in an RPG such as this is that the Game Master is always correct. Magic is a fickle thing, and there are rarely hard and fast rules. A spell that worked with a certain check last time might misfire a second time! A creature that fell at a single flipendo a second ago might suddenly require legions of wizards to put down – listen to your Game Master, and don’t rely on just raw numbers!

The Game Master (GM) also has complete freedom to override the rules in this booklet, in the name of an interesting yet challenging story. An example of this would be changing the requirements of a spell because of the way it is being used. For example, the Patronus Charm is usually cast using a SPR check, as it requires great strength of will to cast – but if you are using a Patronus as a diversion, the GM might decide that an intelligence check is more suitable. The GM is also the arbiter of what is allowed, and whether a tactic was successful. 

Though the GM controls the characters who oppose the players, the GM does not ‘win’ if these enemy characters prevail – the purpose of the GM is not to defeat you, but to drive the story and present interesting and challenging scenarios for you to overcome. Sometimes, that necessitates a little bit of ‘fudging’, and doing so is absolutely not against the rules!