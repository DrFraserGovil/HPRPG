
\chapter{Currency \& the Economy} \label{S:Money}

The currency used by Wizarding Britain is managed by the Goblin Clans who run Gringotts bank. The currency is broken up into 3 coins: the bronze \key{Knut} (\knut{})\comma{} the silver \key{Sickle} (\sickle{}) and the golden \key{Galleon} (\galleon{}). Because the system was designed by goblins \minus{} who have a different intrinsic idea about mental arithmetic \minus{} the coinage has an unusual exchange rate, with 1 \imp{sickle} being worth 29 \imp{knuts}, and 1 \imp{Galleon} being worth 17 \imp{sickles}, or 493 \imp{knuts}. 


\section{Abstracted Wealth}

Just as this game does not require the players to track the daily ablutions of their characters, it can become somewhat tiresome to have to keep track of the exact amount of coin that is being spent at any given moment -- especially given the mindmelting unfamiliar multiplication tables you would need to become familiar with to function in this world. 

Hence, it is assumed that characters have enough \imp{knuts} and \imp{sickles} on them to get by in daily life - and such transactions occur without needing to keep track of any exchange of currency. A character can merely note off-handedly that they visited a grocery store while passing through town, and stocked up on rations, and you may freely tip your waiter when visiting Diagon Alley, without having to worry about modifying your character sheet.

The players must only keep track of their assests on a larger scale -- represented through a more abstracted wealth system, which is measured in \key{galleons}. A \imp{Galleon} would be spent on a large, unusual purchase, something which goes far beyond everyday expenditure: purchasing unusual potion equipment, bribing your way past a guard, or gaining access to a portkey for long-distance travel. 

Though this should not be taken as a strict exchange rate, you may imagine a \imp{Galleon} as being worth approximately £100.

\subsubsection{Introducing Granularity}

Should the story lead in that direction, the GM may also decide that your group has become particularly destitute, and resources are so tight that you are {\it forced} to keep track of currency at a very granular level. In this case, you may instead track sickles on every single purchase made. 

This is an option that is always available, but should only be used for narrative reasons where the grim realism of poverty is relevant and interesting. 

\subsubsection{Abusing the System}

With a system such as this, the temptation is, therefore, that the players could try and exploit this abstracted system. By splitting up a high-cost shopping trip into a number of small individual purchases you could avoid any individual transaction requiring more than a \imp{Galleon}, and hence it would cost you nothing to acquire.

This should be avoided as violating the spirit of the abstracted wealth system which, after all, is designed to make your life as a player less complex and fiddly!

If such shenanigans are taking place, the GM may step in and decree that, cumulatively, an entire Galleon has been spent, rebalancing the scales. 

\subsubsection{Magical Currency}

It should be noted that Wizarding coins are inherently magical in nature. The Goblin Clans which run Gringotts have staked their reputation -- even their very acceptance within wizarding society -- on the security of their transactions, and the validity of their coinage. 

Wizarding currency cannot be altered, synthesised, duplicated or otherwise gained through simple magical means. Attempting to do so may draw dire consequences from the authorities. Forgery is a dangerous game within the wizarding world.


\subsection{Carrying Money}

\imp{Galleons} are pretty hefty coins, and carrying too many of them on your person is just asking to be `relieved' of them by some of the more nimble-fingered (or thuggish) members of society. 

In general, a character can carry no more than \galleon{7} on their person during day-to-day life. These are represented by the 7 \imp{Galleon} dots present in the \imp{Inventory} section of the \imp{Character Sheet}. You may carry more than this amount for very short periods of time, but doing so on more than a rare occasion could be disastrous for your finances.


\section{Vaults}
Since every witch and wizard is entitled to a small vault at Gringotts, you may expand your wealth without incurring excess risk by using their services to store your excess \imp{Galleons}. 

A \key{Vault} acts as a secure storage place for your \imp{Galleons}. You can add or remove coins from your vault whenever you have access to one of the Gringotts' branches scattered across the magical world: in addition to the main site in Diagon Alley, they run small branches in wizard-heavy locations such as Godric's Hollow and Hogsmeade.  

In addition, when making an exceptionally large purchase (a racing-quality broom, for example, can cost in excess of \galleon{100}) it is clearly not feasible to walk around with that amount of money in your pocket. If the transaction is occuring with a large, reputable business, Gringotts has a system whereby you may spend the money directly from your \imp{Vault}. 

Unless you are willing to invest in one of Gringott's more premium services, the reverse is not generally true: you cannot automatically deposit funds into your \imp{Vault} when making a transaction.  


Note that if you are currently on the run from the law, you may find your accounts have been frozen and you do not have access to your normal \imp{Vault}. You may try to set up your own secure \imp{Vault} to store money in, in which case this takes over most of the functionality of your previous vault, though you must manually attend it to remove or add money to the vault.

\section{Purchasing, Selling \& Prices}


You may purchase items if you find a willing vendor - places like \imp{Diagon Alley} are of course filled with people willing to sell you things. Hogwart's, officially, has no need for money, but you may find that the students there have set up a viable black market, and trips to Hogsmeade also allow an opportunity for spending. 


\subsection{Prices}

The cost of an item is determined by its \imp{Rarity}(see page \pageref{S:ItemRarity}), which provides an appropriate cost bracket for each item depending on how rare it is within the wizarding world. As such, the exact price you must pay depends on the details of the world that your \imp{GM} has constructed for you - if, for example, they decide that \imp{Floo Powder} is incredibly rare (perhaps due to a recent magical mishap), the price will shoot up accordingly. 

Some items in this chapter come with either an associated rarity, or in rarer cases, a suggested price. These are, however, merely suggestions for your \imp{GM}. The item lists present in this section are {\it not} a shopping list, they are merely a guide for some basic worldbuilding.

Some items may also simply be unavailable \minus{} either because you are speaking to the wrong person (don't go to a bookstore for potions!)\comma{} because of outside influences\comma{} or simply because the item is so rare that none of the available merchants possess it to sell to you. If you wish to find such a particularly rare item, you may have to do more than simply waltz into Diagon Alley! 

\subsection{Selling \& Bartering}

You may also sell your own found or manufactured items to amenable vendors. Items generally sell for 50\% of the price that you could buy one for\comma{} unless you can demonstrate your wares are of a significantly higher quality\comma{} and hence not subject to the `standard' price. 

As with purchasing your items\comma{} your ability to sell is dependent on you finding a willing (even enthusiastic) buyer\comma{} as well as the surrounding economic circumstances. 

Note that since 1692 is has been a crime in the wizarding world to allow magical items to fall into the hands of muggles \minus{} a crime which\comma{} in the most egregious of circumstances\comma{} has a punishment of death. Unless your \imp{GM} is open to conducting a `Muggle-Wizarding War' scenario, you may wish to refrain from selling magical items on the mundane marketplace...


