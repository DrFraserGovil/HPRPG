\chapter{Books}\label{S:Books}

Books are an incredibly important resource for even the most academia-averse wizard.



\section{Spellbooks}\label{S:Spellbooks}

Aside from learning from teachers (or from wild experimentation), most \imp{spells} that a magic user has learned will have come from pouring over fusty old \key{Spellbooks}, either retrieved from the limited selection in \imp{Hogwarts} library, or found or purchased on your own expense.

However, it is important to remember that \imp{Spellbooks} are not just normal books - over time, the magical secrets they hold inside them imbues them with a level of magic themselves. 

When a student reads the book, they are able to tap into this magic and use it to gain a level of understanding of the spell, beyond the simple words and descriptions on the page. Studying such a book for around 6 hours is enough for this magic to transfer into the student's brain, giving them enough of the relevant knowledge to consider the spell memorised. 

After this process is completed, however, the book is drained of this magical ability, and is considered `Depleted'.


\subsection{Depleted Spellbooks}

A depleted \imp{spellbook} slowly regains its `charge' over the course of about a year. Arcane libraries, such as the one found at Hogwarts, often have shelves which are specifically enchanted to speed up this process, reducing it to only a handful of days. 

Of course, a depleted \imp{Spellbook} is still a \imp{spellbook}, and contains at least some information on how to frame your mind, the words to utter and so on. Though it is much harder than when the \imp{spellbook} actively helps you to learn, you may still attempt to learn the spell from the words on the page.

Attempting a magical spell from a spellbook which has had its magical `teaching aura' depleted is difficult, but it is much easier than attempting to learn the spell through raw trial and error (see page \pageref{S:SpellLearn}), though the process is more or less the same: you must complete a long term studying activity: accumulating 15 successes against a DV of 7. 

\subsection{Acquiring Spellbooks}

Perhaps the easiest way for an aspiring mage to get their hands on a \imp{spellbook} is to take one out of the \imp{Hogwarts Library}. However, note that the library at \imp{Hogwarts} is somewhat limited in its selection of books - this is often deliberatly so. Spells of an unpleasant nature, or containing spells dedicated to nasty \imp{hexes} or deadly \imp{curses} are not something the school wishes to impart too freely upon their students, and in turn the world. 

If you wish to expand your repertoire beyond that which is present in the \imp{Hogwarts} library, you will need to try and purchase the spellbooks yourself: despite their absolute prevelance in wizarding society, the difficulty in producing a high-quality spellbook means that they are considered \imp{unusual}, and so can cost as much as \galleon{10}. For particularly rare spells, you may have to try and find someone who has the spellbook and convince them to part with it. 

\section{Academic Texts}

Wizards often need to learn things beyond raw magic. The (in)famous book \imp{The Monster Book of Monsters}, for example, contained some relatively detailed information on the class of creatures known as \imp{Monsters}. Whenever you wish to research something - be it the weaknesses of a foe you are facing, the complex grammar of a new \imp{enchanting rune}, recipes for a \imp{potion}, or the history of a place or peoples - you will probably need to find an academic text. 

In addition to their \imp{Spellbooks}, \imp{Hogwarts} library keeps a large selection of such academic texts, organised into the following catgeories:

\newcommand\book[2]
{
	\parbox[t]{1.8 cm}{\small \key{#1}}	&	\parbox[t]{7 cm}{\raggedright \small #2} \\
}
\begin{center}
	{\small
	\begin{rndtable}{l l}
		\key{Category}	&	\key{Description}
		\\
		\book{Astronomy}{The stars influence the present and the future in mysterious ways. A study of the stars helps reveal those mysteries.}
		\book{Enchanting}{Books on the process, and significant enchanters and enchantees throughout history. Also includes \imp{rare} and elusive \imp{runetomes}, from which new \imp{runes} can be learned.}
		\book{Herbology}{Books detailing the natural habitats and properties of magical and mundane plants, and the bast way of nuturing them.}
		\book{History}{General history texts, both of the world at large and of significant institutions such as \imp{Hogwarts} and the \imp{Ministry}. }
		\book{Magical Theory}{Musings on the nature of magic, how and why it works, and how it fits in with our understanding of the world.}
		\book{Maps}{Maps and descriptions of both large areas, such as entire countries, and smaller regions like cities and towns.}
		\book{Magical Creatures}{Books detailing information on specific types of monsters, their behaviour, habitat and any weaknesses they might possess.}
		\book{Potions}{Insights into the best recipes and ways in which ingredients can be prepared to extract their full potential.}
	\end{rndtable}
	}
\end{center}

This list is, of course, not exhaustive, but encompasses the major categories of knowledge studied at \imp{Hogwarts}.
\section{Non-Academic Texts}

Not all books contain the secrets of the universe, detailed information on the locations of mystical creatures, or the instructions on how to bend magic to your will. Some books are just books - containing stories, adventures, histories and facts. As in the \imp{Muggle} world, wizards love to immerse themselves in worlds beyond their own, and to let their imagination run wild. There are even rumours of a culture of sourcebooks for RPG games springing up....

Such books are cheap enough that, unless you're looking for a rare first-edition or bulk-buying, you probably do not need to worry about their cost when acquiring them.  
%~ \newcommand\book[2]
%~ {
	%~ \vspace{2 ex}
	%~ \small
	%~ \vbox{
	%~ {\bf #1}
	
	%~ #2
	%~ }
	%~ \normalsize
%~ }
%~ \def\spellIntro{Spellbooks contain within them the information needed to cast spells. The rules for casting from spellbooks are detailed on page \pageref{S:Memory}.

%~ For each topic\comma{} 5 books are listed in descending order. Each of these 5 books corresponds to one block of spells listed on page \pageref{S:SpellList}. {\it The Forbidden Arts}\comma{} the second Dark\minus{}Arts spellbook therefore contains all the level\minus{}2 Dark Arts spells\comma{} but not the level one spells. 
%~ }
%~ \def\normalIntro{Normal books fall into many different categories\comma{}. The list below contains an example of some of the most common topics of wizarding books\comma{} and a few examples of the most famous texts within those categories\comma{} where relevant. }

%~ A book is a compendium of knowledge\comma{} contained between two pages. As wizards\comma{} words and knowledge are power \minus{}\minus{} so all good wizards are familiar with their literature! Despite this\comma{} books can be rather heavy (classified as `medium' weight)\comma{} and hence a normal witch or wizard will struggle to carry more than 3 books on them during everyday life. 
%~ \vspace{-3 ex}
%~ \def\w{8}
%~ %%BooksBegin
%~ \subsection{Normal Books} \normalIntro \begin{center} \footnotesize \begin{rndtable}{|p{\w cm} l |}\hline \normalsize \bf Name & \normalsize \bf Cost \\ \hline	\bf Ancient Runes	&	50 \\ 
	%~ \bf Artificing	&	\\
	%~ ~~{\it From Twigs to Flight: A Broommaking Guide}	&	35\\
	%~ ~~{\it Avoiding Mishaps When Making Things}	&	20\\
	%~ ~~{\it Steel\comma{} Stone \& Sorcery: A Guide to Golems}	&	1000\\
	%~ \bf Astronomy	&	\\
	%~ ~~{\it The Stars and Why They Matter}	&	25\\
	%~ ~~{\it Galactic Dynamics\comma{} Second Edition}	&	80\\
	%~ ~~{\it The Magical Effects of Stars}	&	20\\
	%~ \bf Biographies	&	\\
	%~ ~~{\it Wizarding Biographies}	&	30\\
	%~ ~~{\it Muggle Biographies}	&	10\\
	%~ \bf Herbology	&	\\
	%~ ~~{\it One Thousand Magical Herbs and Fungi}	&	40\\
	%~ ~~{\it Flesh\minus{}Eating Trees of the World}	&	30\\
	%~ \bf History of Magic	&	\\
	%~ ~~{\it A History of Magic}	&	30\\
	%~ ~~{\it Great Wizards Through History}	&	25\\
	%~ ~~{\it Non\minus{}European Magic and its History}	&	40\\
	%~ ~~{\it Hogwarts a History}	&	15\\
	%~ ~~{\it Sites of Historical Sorcery}	&	80\\
	%~ \bf Magical Creatures Book	&	\\
	%~ ~~{\it Fantastic Beasts and Where to Find Them: A Guide to Common Magical Creatures}	&	20\\
	%~ ~~{\it Studies on Sapient Creatures}	&	20\\
	%~ ~~{\it The Unlife\comma{} and How to Avoid Them}	&	40\\
	%~ ~~{\it Monster Book of Monsters}	&	60\\
	%~ ~~{\it Rare and Dangerous Magical Creatures Around the World}	&	100\\
	%~ \bf Maps	&	\\
	%~ ~~{\it Local\minus{}Scale Maps}	&	10\\
	%~ ~~{\it Large\minus{}Scale Maps}	&	40\\
	%~ \bf Mathematics	&	10 \\ 
	%~ \bf Muggle Literature	&	5 \\ 
	%~ \bf Muggle Studies	&	25 \\ 
	%~ \bf Periodicals	&	\\
	%~ ~~{\it Daily Prophet}	&	4\\
	%~ ~~{\it The Quibbler}	&	10\\
	%~ ~~{\it Witch Weekly}	&	5\\
	%~ \bf Potions	&	\\
	%~ ~~{\it Magical Drafts and Potions}	&	30\\
	%~ ~~{\it Advanced Potion Making}	&	80\\
	%~ \bf Quidditch	&	\\
	%~ ~~{\it Quidditch Through the Ages}	&	15\\
	%~ ~~{\it Handbook of Do\minus{}It\minus{}Yourself Broomcare}	&	35\\
%~ \hline
%~ \end{rndtable}
%~ \end{center} \vfill \normalsize 
%~ \subsection{Spell Books} \spellIntro \begin{center} \footnotesize \begin{rndtable}{|p{\w cm} l |}\hline \normalsize \bf Name & \normalsize \bf Cost \\ \hline	\bf Spellbook: Charms	&	\\
	%~ ~~{\it The Standard Book of Spells}	&	30\\
	%~ ~~{\it Achievements in Charming}	&	60\\
	%~ ~~{\it The Standard Book of Spells (Grade 2)}	&	100\\
	%~ ~~{\it Charms: An Expert\apos{}s Guide}	&	200\\
	%~ ~~{\it Extreme Incantations}	&	500\\
	%~ \bf Spellbook: Dark Arts	&	\\
	%~ ~~{\it An A\minus{}Z of Spooky Spells}	&	100\\
	%~ ~~{\it The Forbidden Arts}	&	200\\
	%~ ~~{\it Necromancy: A Misunderstood Skill}	&	300\\
	%~ ~~{\it Magick Moste Evile}	&	500\\
	%~ ~~{\it Spelles Moste Vyle}	&	800\\
	%~ \bf Spellbook: Divination	&	\\
	%~ ~~{\it The Dream Oracle}	&	30\\
	%~ ~~{\it The Future is an Open Book (And So is This)}	&	60\\
	%~ ~~{\it Unfogging the Future}	&	100\\
	%~ ~~{\it Death Omens: What to Do When You Know the Worst is Coming}	&	200\\
	%~ ~~{\it Time and its Mysteries}	&	500\\
	%~ \bf Spellbook: Hexes \& Curses	&	\\
	%~ ~~{\it Basic Hexes for the Busy and Vexed}	&	30\\
	%~ ~~{\it A Compendium of Common Curses}	&	60\\
	%~ ~~{\it Curses \& Counter\minus{}Curses}	&	100\\
	%~ ~~{\it Dark Forces: A Guide to Self Protection}	&	200\\
	%~ ~~{\it An Auror\apos{}s Toolkit}	&	500\\
	%~ \bf Spellbook: Illusion	&	\\
	%~ ~~{\it Easy Spells to Fool Muggles}	&	30\\
	%~ ~~{\it Jiggery\minus{}Pokery \& Hocus\minus{}Pocus}	&	60\\
	%~ ~~{\it On the Mysteries of the Human Mind}	&	100\\
	%~ ~~{\it Merlin\apos{}s Tricks and Incantations}	&	200\\
	%~ ~~{\it Light and Perception: The Magician\apos{}s Mastery}	&	500\\
	%~ \bf Spellbook: Recuperation	&	\\
	%~ ~~{\it Self\minus{}Defensive Spellwork}	&	30\\
	%~ ~~{\it How To Not Be Killed: A Guide}	&	60\\
	%~ ~~{\it Defensive Spells to Save Your Skin}	&	100\\
	%~ ~~{\it An Anthology of Safeguarding Measures}	&	200\\
	%~ ~~{\it Life\comma{} and How to Preserve It}	&	500\\
	%~ \bf Spellbook: Transfiguration	&	\\
	%~ ~~{\it A Beginner\apos{}s Guide to Transfiguration}	&	30\\
	%~ ~~{\it Transmutation and  other Transformative Tricks}	&	60\\
	%~ ~~{\it Theories of Transubstantial Transfiguration}	&	100\\
	%~ ~~{\it Conjuring and Summoning for the Experienced Witch}	&	200\\
	%~ ~~{\it The True Art of Transfiguration}	&	500\\
%~ \hline
%~ \end{rndtable}
%~ \end{center} \vfill \normalsize 
%~ %%BooksEnd
