\chapter{Clothing \& Armour} \label{S:Armour}

The \key{clothing}, \key{armour} and protective gear you wear can have a dramatic impact on your ability to defend yourself\comma{} or run away from problems. 

Wearing thick and heavy armour provides a bonus to your \imp{Block} statistic, but will reduce your ability to \imp{Dodge} effectively. Some clothing, however, is designed to encourage freedom of movement - a simple set of running trainers, for example, can make \imp{dodging} a lot easier.  

\section{Armour Types}

\imp{Armour} is generally split into 4 categories: \key{Clothing}, \key{Light Armour}, \key{Medium Armour} and \key{Heavy Armour}.

Below are some examples of items of clothing and gear which would be classified as one of these. Your \imp{GM} is not necessarily beholden to these rules, and may allow to find, or offer for purchase a set of armour with a slightly different set of statistics.

\subsubsection{Outfits}

For the sake of simplicity\comma{} you are generally discouraged from `mix and matching' various types of armour and clothing. You do not gain an advantage from wearing a set of \imp{Running Shoes} alongside your \imp{Iron Breastplate}: you would still be considered to be \imp{Heavily Armoured}.

The classification of armour is an abstraction, to prevent having to think about the interaction of 15 different items of clothing on your game statistics. You should instead think about your armour as an entire \key{outfit}. 

You are, of course, allowed to switch out various pieces of armour for magical equivalents, or simply for a cooler aesthetic. If you have a \imp{Magical Steel Helm} of some kind, you could wear that and still be considered only lightly armoured (though unarmoured might be a stretch!).   

Ultimately, you \imp{Armour} rating is determined by whatever type of protection you are wearing {\it most} of \minus{} and if in doubt\comma{} the least favourable interpretation will be used. 

\section{Clothing}

Clothing provides virtually no protection against attacks of any kind. Whilst a combat-reliant character might be a fool to forgo protective gear, most people do not saunter through their lives behind six layers of steel: the majority of your life is spent in normal clothes. Buying a good set of \imp{Clothing} can cost around \galleon{1}.

However, even whilst \key{unarmoured}, the clothes you choose can have some benefits on your \imp{game statistics}:

\newcommand\armour[3]
{
	\vbox{
	\subsubsection{#1}
	
	#2
	
	\key{Effects:} #3
	}
}

\armour{Formalwear}
{
	Sometimes, you need to look just that little bit extra {\it cool}. Be it ballgowns for an elegant dance, suits and ties for infiltrating a swanky casino, or a suit and tie for an important business meeting: \imp{formalwear} is your friend. It is, however, somewhat impractical.
}
{
	-1d to \imp{Dodge} and \imp{Block} ratings, +2d to \imp{Eloquence}
}
\armour{Muggle Civvies}
{
	Jeans and a t-shirt, a nice summer dress, or bundled up in a cosy jumper - \imp{Muggle clothes} allow you to blend in. More and more wizards also use this as their day-to-day clothing.
}
{
	+1d to \imp{Social} checks with \imp{muggles}.
}
\armour{Sports Outfit}
{
	Well-gripped shoes, and an outfit which enables the full range of movement. Whatever sports it was designed for, these outfits enable you to move much quicker. 
}
{
	+1d to \imp{Dodge} rating and +2 to \imp{Base Dodge Bonus}
}

\armour{Wizards Robes}
{
	The standard outfit in the wizarding world, though it is slowly being considered old-fashioned and outdated. \imp{Hogwarts} requires students to wear these as their school uniform - and there is just {\it something} about the billowing of a good cloak, and the connection to wizarding history, that makes you feel more confident and sure of yourself. 
}
{
	+1 to \imp{Base Defy Bonus}
}
\section{Light Armour}

One step up from being \imp{Unarmoured}, light armour often consists of thick layers of fabric and padding which can help soften an impact or dissipate a spell before it reaches your skin. \imp{Light Armour} is often innocuous enough, or similar to normal clothing, that people do not feel threatened upon seeing someone wearing it. Though relatively common, most people do not own any \imp{Light Armour} - it would be considered \imp{Singular}, and cost between \galleon{3}-5.

\armour{Padded Fabrics}
{
	Maybe a thick motorcycle jacket, or some other suitably sturdy material. The additional layers offer some mild protection, but don't restrict your movement too much.
}
{
	+1d to \imp{Block Rating}, +1 to \imp{Base Block Bonus}
}

\armour{Warded Cloth}
{
	A more recent (and expensive) magical invention, this looks like a normal \imp{Wizard's Robe}, but the material has been enchanted. Although it provides only minor additional protection, the material holds its shape and integrity under even the most devastating attacks.
}
{
	+1d to \imp{Block} rating, even whilst \imp{fully drained}
}


\section{Medium Armour}

\imp{Medium Armour} strikes the balance between defense and movement. However, it is obviously armour - people wearing this are often given a slightly wider berth, as they appear to have come dressed for trouble.

Because they are specially designed, and not usually bought by the public, these items can be a bit pricey and difficult to get ahold of. They would be considered \imp{Unusual}, and cost between \galleon{5} and \galleon{10} each.

\armour{Battlemage Robe}
{
	When jumping into battle, \imp{Battlemages} historically wore specially hardened and warded robes which protected them from offensive magics. Magical wars have changed a lot since then, and the technology has advanced with it. 
}
{	
	+4d to \imp{Block} rating, -3d to \imp{Dodge}
}

\armour{Kevlar Vest}
{
	A \imp{Muggle} invention, designed to deflect firearms, this material serves as a surprisingly good defense against magical attacks as well. The bulk does restrict movement somewhat.
}
{
	+2d to \imp{Block} rating, +2 to \imp{Base Block Bonus}, -2d to \imp{Dodge}
}

\section{Heavy Armour}

The ultimate in defensive equipment, providing the maximum level of protection. Wearing \imp{Heavy Armour} in public is generally seen as provocative - the authorities will certainly want to know why you are walking around tooled up to the teeth. This should probably only be worn when the circumstances call for it. 

These sets of armour are also usually extremely expensive, as even the base-models are heavily enchanted. They would be considered \imp{rare}, and can cost in the region of \galleon{50} per set.
\armour{Runic Plate}
{
	It is said that modern problems require modern solutions. Runic plate is proof that\comma{} maybe\comma{} this isn't always the case. A modified version of medieval \imp{Muggle} suits of armour, imbued with runes which enforce its structural integrity. Truly a wonder of magical engineering.
}
{
	+4d to \imp{Block} rating, which cannot be reduced by drain. +2 to \imp{Base Block Bonus}, -4d to \imp{Dodge}
}

\armour{Tactical Response Suit}
{
	When \imp{Aurors} are going in against a foe capable of devastating attacks, they suit up in these specially designed outfits. 
}
{
	+3d to \imp{Block} rating, +4 to \imp{Base Block Bonus}, -3d to \imp{Dodge}
}
%~ \section{Wearing Armour}

%~ \subsection{Outfits}

%~ Wearing thicker armour protects you\comma{} by increasing your {\it Block} statistic by a specified amount. Most sets of clothing are considered to come in a `full set'\comma{} and thus cover the entire torso\comma{} arms\comma{} legs\comma{} feet \minus{} and possibly comes with some headwear. 

%~ 

%~ If Gunter the half\minus{}giant wishes to wear a full suit of knight's armour\comma{} but swap the gloves out for her cotton {\it Gloves of Pugilism}\comma{} she can do so without altering the total Block value. However\comma{} if she also swapped out the helmet for a jaunty hat\comma{} and the footwear for some running shoes\comma{} the GM may step in and decree a penalty to her Block statistic. 

%~ \def\y{1.3}
%~ \def\w{3.3}
%~ \def\x{2.6}
%~ \def\u{0.8}
%~ \newcommand\armour[4]
%~ {
%~ \\
	%~ \parbox[t]{\y cm}{\raggedright\textbf{\footnotesize\textit{#1}}} & \parbox[t]{\w cm}{\footnotesize#2} &  \parbox[t]{\x cm}{\footnotesize\raggedright #3}	&	\parbox[t]{\u cm}{\centering #4}
	
%~ }

%~ \subsection{Proficiencies}

%~ Armour comes in 4 categories: clothing\comma{} light armour\comma{} medium armour and heavy armour\comma{} in order of increasing protection. 

%~ he first two of these (clothing \& light armour) can be worn by anyone\comma{} without penalty. However\comma{} wearing medium or heavy armour requires skill to be able to do\comma{} without it becoming a severe distraction. These armours require you to be proficient (either through a class bonus\comma{} or through the relevant Skill). If you attempt to wear armour you are not proficient in\comma{} you take the {\it Encumbered} status effect and check\minus{}disadvantage on any accuracy checks made.

%~ \newpage
%~ \subsection{Clothing}

%~ Everyday clothes offer no additional protection against the attacks of malevolent forces. It is\comma{} however\comma{} comfy and easy to wear. 

%~ You require no proficiencies in order to wear clothing. 

%~ \small
%~ \begin{center}
%~ \begin{rndtable}{p{\y cm} p{\w cm} p{\x cm} p{\u cm}}
%~ \bf Type   &	\bf Description	&	\bf Effect	& \bf Cost

%~ \armour{Casual outfit}{Jeans and a t\minus{}shirt. Cheap\comma{} comfy and practical}{No effect}{\sickle{10}}
%~ \armour{Formal Wear}{Extra suave look for the discerning witch or wizard. Ball gowns and tuxedoes are impractical\comma{} but you look amazing!}{\minus{}2 Dodge\comma{} \\ +2 Charisma}{\galleon{2}}
%~ \armour{Sports clothes}{Specially designed clothing for taking part in physical activity.}{+2 Dodge}{\galleon{1}}
%~ \armour{Wizards Robes}{Once the everyday clothes of all wizardkind\comma{} now usually seen as the typical school uniform of a Hogwarts student.}{+1 to spellcasting checks}{\sickle{7}}

%~ \end{rndtable}
%~ \end{center}
%~ \normalsize
%~ \subsection{Light Armour}

%~ Light armour is the crossing point between what we typically think of as armour (knights clanking around in metal)\comma{} and everyday clothes. Light and flexible\comma{} it grants only limited protection. 

%~ You require no proficiencies in order to wear light armour. 

%~ \small
%~ \begin{center}
%~ \begin{rndtable}{p{\y cm} p{\w cm} p{\x cm} p{\u cm}}
%~ \bf Type   &	\bf Description	&	\bf Effect	& \bf Cost
%~ \armour{Padded}{Formed from multiple layers of soft fabric and padding}{+2 Block\comma{} \\\minus{}1 Dodge\comma{} \\ Conspicuous}{\sickle{25}}
%~ \armour{Leather Jacket}{A simple leather jacket offers a surprising amount of protection. Plus it looks cool.}{+1 Block}{\sickle{10}}
%~ \armour{Warded Cloth}{A recent magical invention\comma{} this expensive material hardens on impact\comma{} providing extra protection\comma{} whilst not impeding your movement.}{+2 Block}{\galleon{12}}
%~ \end{rndtable}
%~ \end{center}
%~ \normalsize

%~ \newpage
%~ \subsection{Medium Armour}

%~ \small
%~ \begin{center}
%~ \begin{rndtable}{p{\y cm} p{\w cm} p{\x cm} p{\u cm}}
%~ \bf Type   &	\bf Description	&	\bf Effect	& \bf Cost
%~ \armour{Bulletproof Vest}{A muggle invention\comma{} this weaved kevlar material offers a good amount of protection.}{+3 Block\comma{}\\ \minus{}1 Dodge\comma{}\\ Resistance to Ranged Weapon attacks}{\galleon{3}}
%~ \armour{Hardened Furs}{A primitive\minus{}appearing armour often worn by giants and other isolated peoples. Layers of hardened leather and treated hides protects against the cold\comma{} as well as from weapons.}{+2 Block\comma{}\\ \minus{}1 Dodge \\ Resistance to Cold damage}{\sickle{15}}
%~ \armour{Tactical Armour}{The armour of the Auror class\comma{} thought to strike the correct balance between hardened and fortified plates inserted between layers of flexible fabric.}{+4 Block \\ \minus{}2 Dodge\comma{} \\ Conspicuous}{\galleon{8}}
%~ \armour{Warrior Robe}{Magical armies are rare\comma{} but Battlemages often wore specially warded robes which offered improved protection\comma{} though hampered movement.}{+3 Block\comma{}\\\minus{}1 Dodge}{\galleon{3}}
%~ \end{rndtable}
%~ \end{center}
%~ \normalsize

%~ \subsection{Heavy Armour}


%~ \small
%~ \begin{center}
%~ \begin{rndtable}{p{\y cm} p{\w cm} p{\x cm} p{\u cm}}
%~ \bf Type   &	\bf Description	&	\bf Effect	& \bf Cost
%~ \armour{Bomb Suit}{Specially designed suit that one must climb inside. Used by professionals who frequently find themselves at risk of incineration or detonation}{+5 Block\comma{}\\ \minus{}6 Dodge \\ Resistance to Fire \& Concussive damage\comma{} \\Conspicuous.}{\galleon{15}}
%~ \armour{Runic Mail}{Enchanted scales of metal fit together to provide full physical and magical protection over your body\comma{}.}{+7 Block\comma{}\\ \minus{}5 Dodge\comma{} }{\galleon{100}}
%~ \armour{Steel Plate}{}{+4 Block\comma{} \\ \minus{}5 Dodge\comma{} \\Conspicuous\comma{} \\ Resistance to Piercing \& Slashing damage.}{\galleon{10}}
%~ \armour{Special Response Set}{The bigger\comma{} badder brother of the Tactical armour. Used only when overwhelming firepower needs to be withstood\comma{} as it is much more cumbersome}{+5 Block\comma{} \\ \minus{}4 Dodge \\ Conspicuous}{\galleon{12}}
%~ \end{rndtable}
%~ \end{center}
%~ \normalsize

%~ \newpage

%~ \section*{Damaging Armour}\label{S:DestroyArmour}

%~ Of course\comma{} armour is not a panacea\comma{} and it cannot protect the squishy meat inside indefinitely. 

%~ When a {\it Critical Strike} is performed with one of the damage types mentioned in the table below\comma{} the attacker may choose to forgo inflicting damage and instead damage the armour of the target. 

%~ \begin{center}
%~ \begin{rndtable}{ c c}
%~ \bf Damage Type	&	\bf Armour Damage
%~ \\
%~ Acid	&	1d4
%~ \\
%~ Bludgeoning	&	1d2
%~ \\
%~ Piercing	&	1d4
%~ \\
%~ Slashing	&	1d2
%~ \end{rndtable}
%~ \end{center}

%~ Roll the associated {\it Armour Damage Dice} for the damage type\comma{} and deduct this total from the current Block bonus provided by the being's protective layer. This is a permanent deduction in the Block statistic\comma{} until the armour is repaired. 


%~ If the block\minus{}bonus reaches zero\comma{} the armour is considered `destroyed'\comma{} and is automatically `de\minus{}equipped' as it falls to shreds around you. 

%~ \section*{Restoring Armour}

%~ Damaged Armour may be restored by spending 1 hours repairing it (with a repair kit) for one hour per {\it Block} bonus that must be restored\comma{} or by using a suitable magic spell.

%~ Armour that has been `destroyed' cannot be repaired without proficiency with a {\it repair kit}.
