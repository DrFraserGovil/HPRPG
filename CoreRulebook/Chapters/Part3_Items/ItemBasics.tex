\chapter{Item Basics}


`\key{Item}' is a catchall term for any physical object that a character has in their possession. This encompasses their clothing, their wands, their money and anything else they might own. 

Some items are important because they allow you to carryout basic tasks - writing a letter is going to be difficult without a pen or a quill to hand. If you do not have the relevant item (and you lack the necessary resources to improvise such an item), you will have to find another way around the task. This generally will be perfectly obvious to those involved, and so the rulebook does not provide rules detailing this.

On the other hand, some items provide benefits beyond their basic functionality - higher quality tools produce higher quality results, and armour protects your vital organs, as well as your modesty - in which case it is necessary to formulate a system for providing these effects. This chapter details the details and magnitudes of effects that can result from such items. 

Of course, objects in the Wizarding World are also often not what they seem - items and potions imbued with magic can have vastly powerful and varied effects. The creation of such items is known as \key{Artificing} and is detailed at the very end of this chapter.



\section{Equipped, Carried Stored Items}

An \imp{Item} is classified as being either \key{Equipped}, \key{Carried}, or \key{Stored}, in order to allow you to keep track of your possessions. All items which are currently on your person (i.e. not \imp{Stored}) are said to be in your \key{Inventory}.

\subsection{Equipped Item}

An item is \key{equipped} if you are actively carrying it on your person - clothing is equipped when you are wearing it, and items such as wands and weapons are \imp{equipped} if they are being held in your hand or holstered for quick access, smaller items are \imp{equipped} if they are being held in a pocket. 

Because these items are already on your person, they can be used easily and immediately: you can simply tell the GM that you have drawn your wand and wish to cast a spell. Most items must be \imp{equipped} before they can be used; you can't whack someone with your magical sword\comma{} if your magical sword is in your bag\comma{} after all. 

The character sheet on page \pageref{S:CharacterSheet} has a space for such important items as your wand, armour and so on on the first page. When an item is \imp{equipped}, you are encouraged to shade in the dot associated with that item, so that you may easily determine what is currently accessible. 

\subsection{Carried Items}

Items which are on your person, but that are not currently \imp{equipped} are said to be \key{Carried}. The default assumption is that an item is being \imp{carried}, unless you have stated otherwise and explicitly \imp{equipped} it. 

For the most part, these items are assumed to be in a backpack or otherwise carried in some sensible fashion. You may remove an item from your bag in order to use or \imp{equip} it, but this typically takes time, and so may not be appropriate in \imp{combat}, unless you have no other choice. You may also transfer items from one character to another, though again, this takes time and requires that characters be within passing distance of each other, unless magic is involed!

Over the course of your adventure, you are likely to pick up any number of interesting trinkets, magical or otherwise. Unless you are expecting to be using or \imp{equipping} these items on a regular basis, you might find the \imp{Full Inventory} section on the reverse of the \imp{Character Sheet} to be a useful place to store such items. 


\subsection{Stored Items}

A \key{Stored} item is an item which you own, but are not carrying on your person - having either placed it in a secure area (every student at Hogwarts has a large, secured chest in their \imp{Common Room}), or hidden away in a secret location. 

You should keep track of these items on your character sheet, but note that retrieving these items is a non-trivial task, and is not something that can generally be done in a hurry. 

Items which are \imp{stored} in an unsecure location are also at risk of theft - if you leave the priceless \imp{Staff of Merlin} in a rented room at the \imp{Leaky Cauldron}, don't be surprised if it's not there when you come back for it in three weeks time. 



\section{Item Weight}\label{S:ItemAbstraction}

Rather than keeping track of the exact weight of each individual item in your backback\comma{}, each item is instead categorised as either `Light'\comma{} `Medium'\comma{} `Heavy' or `Very Heavy'. A shorthand using `*' is used to easily mark these items:
\begin{itemize}[itemsep = 0pt]
\item A `Light item' (no * rating) can be picked up without thinking. They can typically easily fit into your pocket; a sheaf of paper\comma{} some candles and a wizard's wand are all `light'. 

\item A `medium' weight item (*) has a reasonable amount of heft to it\comma{} but can be held comfortably without strain; most weapons are categorised as `medium', as are spellbooks.

\item A `heavy' item (**) requires two hands to carry without strain; a suit of armour\comma{} as well as cumbersome objects such as the bludger are classified as `heavy'.

\item A `very heavy' item (***) cannot be carried by one person alone: multiple individuals are required. A chest full of gold and jewels would be `very heavy'. 
\end{itemize}
As a general rule, if the total weight of items being carried (measured in `*'s) exceeds half your combined \imp{Fitness + Strength} score, you have gone way past what is reasonable.  

That being said, this is a deliberately vague system, and players are to be given a certain amount of leniency in carrying items (faffing about with numbers is less fun than running around a magical castle). However at any point the GM may ask a player to justify how, for example, they were just able to execute a full sprint whilst carrying Hogwarts' entire supply of cauldrons on their person, and impose penalties and consequences as appropriate. 




\section{Item Rarity}\label{S:ItemRarity}

In order to help quantify and codify how prevalent certain items are in the magical world, a 7-tier\footnote{You may have noticed that everything in this game is based around the number `7' - this is intentional. After all, 7 is the most intrinsically powerful magical number!} system is used to classify all items.

\newcommand\enchantRow[3]{ \key{#1} & \parbox[t]{4.5 cm}{\small \raggedright #2}  & {\small #3}\\}


\begin{center}
	\begin{rndtable}{c c c}
		\key{Difficulty}	&	\key{Description}	& \key{Cost} (\galleon{})
		\\
		\enchantRow{Abundant}{Very simple, cheap and abundant objects - a normal wizard would expect to have dozens of such items in their possession.}{$\sim${0}}
		\enchantRow{Common}{Though they might not own many as many of these items, a normal witch would be familiar and comfortable with \imp{Common} items in everyday life.}{1-2}
		\enchantRow{Singular}{Verging on impressive, either in rarity or in magical power, your normal magic-user probably owns only a handful of these items, and they would be treated as prized possessions.}{3-5}
		\enchantRow{Unusual}{An \imp{Unusual} magical would usually be enough to draw inquisitive looks when displayed - a normal witch or wizard probably only owns one or two of these throughout their lifetime (if they are lucky).}{5-10}
		\enchantRow{Rare}{A rare item would be passed down through a family as a treasured heirloom, and kept under the highest security when not being used. There are probably no more than a few thousand \imp{Rare} items in the country at any one time. }{10-100}
		\enchantRow{Extraordinary}{Most witches and wizards would only ever have seen an \imp{Extraordinary} magical item, and very, very few will have owned them. An object of this power is usually either stored in \imp{Gringotts}, or on display in a museum.}{100-1000}
		\enchantRow{Mythical}{\imp{Mythical} items are unique magical items which would be remembered throughout history for their enigmatic magic and immense power. Even being able to see, let alone own, a Mythical item would be a thing to tell stories about.}{1000 +}
	\end{rndtable}
\end{center}

Note that the rarity of an item is not defined using its `power' level, or its damage output or other effects such as this, but rather in terms of how many such items your average witch or wizard would expect to encounter or own in their lifetime. When attempting to work out how `rare' an item is, it should be simple enough to work out if the entire magical world would break down if these items were cheap and abundant. 

Some examples of some common magical items broken down by rarity can be found on page \pageref{E:EnchantingExamples}


\section{Item Quality} \label{S:Quality}

Some items are of a higher quality than others (and equally, some are downright shoddy), which can make the item easier to use, and potentially allow for much greater rewards to be reaped from their usage. 

As a general rule, items are categorised into one of the following types:

\newcommand\quality[3]{\imp{#1}	&	\parbox[t]{5cm}{\raggedright #2}	&	\parbox[t]{2 cm}{\centering #3} \\}

\begin{center}
	\begin{rndtable}{r c c}
		\bf Quality	&	\bf Description	&	\bf Effect
		\\
		
		\quality{Improvised}{A version of an item that you have hurriedly assembled yourself out of scraps and other nearby items. A very poor subsitute for the real thing - you can expect to mess up badly with an improvised item.}{-2d \\+1 \imp{CL}}
		\quality{Broken}{Perhaps rusted through, with missing parts or only partially functional. A broken item is no fun at all.}{-2d}
		\quality{Poor}{A bit run-down, used and generally bad quality - still useable, however.}{-1d}
		\quality{Normal}{The normal quality you would expect from an average bit of equipment. Not bad, but not great either.}{-}
		\quality{Fine}{A high quality bit of kit. A bit more privey, but the effects are powerful.}{+1d}
		\quality{Good}{A very finely crafted item, commanding an exceptionally high price, but yielding exceptional results.}{+2d}
		\quality{Excellent}{The highest possible quality, giving truly wonderful results and making it much harder to really mess things up.}{+2d \\ -1 \imp{CL}}
	\end{rndtable}
\end{center}
\imp{CL} is the \imp{Catastrophe Level} (see page \pageref{S:Catastrophe}), which cannot be reduced to below 1.

