\chapter{Item Basics}

\section{Currency \& the Economy}

The wizarding currency is commonly broken up into 3 coins: the bronze {\bf Knut}\comma{} the silver {\bf Sickle} and the golden {\bf Galleon}. Because the system was designed by goblins \minus{} who have a different intrinsic idea about mental arithmetic \minus{} the coinage has an unusual exchange rate.

\subsubsection{Knut}

The bronze knut\comma{} denoted by the symbol \knut{}\comma{} is the lowest denomination coin in the wizarding world. Typically considered `loose change'\comma{} individual exchanges rarely occur with Knuts\comma{} though a veritable fortune in knuts is estimated to be found down the sides of sofas of the wizarding world. 

\subsubsection{Sickle}

The silver sickle (\sickle{}) is the primary currency used by most wizards. Prices for everyday items are generally listed on the order of tens of sickles. An low\minus{}skilled worker could be expected to be earning around 10 sickles for a full days work. 

There are 29\knut{} in one sickle. 

\subsubsection{Galleon}

The galleon\comma{} \galleon{}\comma{} is the largest denomination of currency\comma{} consisting as it does of 17\sickle{}\comma{} or 493\knut{}. 

Most wizards rarely handle or carry around actual galleons \minus{} purchases that occur on this scale are often directed through Gringotts \minus{} though it is not unheard of for rich wizards to flash their golden coins around town.

\subsection{Muggle Exchange Rate}

The exchange rate between muggle and wizarding currencies can be hard to pin down\comma{} as their respective economies bear very little resemblance to each other. What is scarce in one world is often common in the other. 

However\comma{} since the economic crash in 1929\comma{} Gringotts has agreed to establish a fixed exchange rate. Under the current scheme\comma{} Gringotts will purchase £20 for 50\galleon. This works out to give 1 GBP to be equal to 10 knuts\comma{} or just under £3 to a sickle.



\def\cc{\cellcolor{\tablecolorhead}}
\begin{center}
	\begin{rndtable}{c c c c c}
	~	&	\multicolumn{4}{c}{\bf Value}
	\\
	\multirow{\negTwo}{*}{\cc \bf Coin}	&	\cc \knut	&\cc	\sickle	& \cc \galleon & \cc £
	\\
	\cc	\knut		& 1		&	0.034	&	0.002	&	0.1	
	\\
	\cc\sickle		&29		&	1		&	0.059	&	2.94
	\\
	\cc\galleon		&493	&	17		&	1		&	50
	\\
	\cc  £		&	9.86	&	0.34	&	0.02	&	1
	\end{rndtable}
\end{center}

\subsection{Prices \& Availability}

Many items in this guide are listed with an associated price. This is the `standard purchase price' (SPP)\comma{} and is the price one could expect to pay for the item in a large population centre\comma{} during normal economic times\comma{} without excessive bartering. 

However\comma{} this price may increase or decrease for certain items\comma{} depending on the location and the adventure you are undergoing. 

If\comma{} for example\comma{} you had {\it accidentally} triggered a worldwide famine\comma{} then food items could become exceptionally expensive and cost far more than the SPP. Conversely\comma{} if you manage to rid a local lake of the hippocampus that had been terrorising it\comma{} you may find the bountiful fishing harvest reduces the price of fish for a few days. 

Some items may also simply be unavailable \minus{} either because you are speaking to the wrong person (don't go to a bookstore for potions!)\comma{} because of outside influences\comma{} or simply because the item is so rare that none of the available merchants possess it to sell to you. 

\subsection{Selling \& Bartering}

You may also sell your own found or manufactured items to amenable vendors. Items generally sell for 50\% of their SPP\comma{} and no amount of bartering will raise it to 100\%\comma{} unless you can demonstrate your wares are of a significantly higher quality\comma{} and hence not subject to the `standard' price. 

As with purchasing your items\comma{} your ability to sell is dependent on you finding a willing (even enthusiastic) buyer\comma{} as well as the surrounding economic circumstances. 

Note that since 1692 is has been a crime in the wizarding world to allow magical items to fall into the hands of muggles \minus{} a crime which\comma{} in the most egregious of circumstances\comma{} has a punishment of death. 


\section{Equipped Items}

An item that is equipped can be used immediately. In combat\comma{} this would count as your major action. Simply tell your GM that you are using a certain item\comma{} and you may then carry out the effect that the item has. 

Some items must be equipped before they can be used; you can't whack someone with your magical sword\comma{} if your magical sword is in your bag\comma{} after all. Generally speaking\comma{} getting items out of storage is not a major action; you may retrieve and then use a health potion in a single motion\comma{} for example. Some items\comma{} however\comma{} might take longer to equip: strapping on a suit of armour\comma{} for instance\comma{} clearly takes some time!



\section{Storing Items}

Items that are not currently equipped are stored in your backpack\comma{} which you should probably try to keep on you at all times. Losing it would be bad!

Items may be transferred between members of a party at any time\comma{} if they are within 1m (or you may use a spell such as accio). In combat\comma{} switching an item counts as a major action for both characters. 



\section{Item Weight}

Rather than keeping track of the exact weight of each individual item in your backback\comma{} this game opts for a more free\minus{}form approach to tracking item weight. Each item is categorised as either `Light'\comma{} `Medium'\comma{} `Heavy' or `Very Heavy'. 

A `Light item' can be picked up without thinking. They can typically easily fit into your pocket; a sheaf of paper\comma{} some candles and a wizard's wand are all `light'. 

A `medium' weight item has a reasonable amount of heft to it\comma{} but can be held comfortably without strain; most weapons are categorised as `medium'. 

A `heavy' item requires two hands to carry without strain; medium and heavy armour\comma{} as well as cumbersome objects such as the bludger are classified as `heavy'.

A `very heavy' item cannot be carried by one person alone: multiple individuals are required. A chest full of gold and jewels would be `very heavy'. 
