\chapter{Artificing}\label{S:Artificing}


\key{Artificing} is the art of creating new items, typically those imbued with magical powers. 

The most prominent examples of \imp{artificing} in the Wizarding world are \key{Enchanting} and \key{Alchemy}, guided by the \key{Imbuement} ability. Non-magical \imp{Artificing} is usually referred to as \key{crafting}, and controlled by the ability of the same name.  

\section{Enchanting}
\label{S:Enchanting}

\imp{Enchanting} is the process whereby magical items are made – imbuing them with extraordinary abilities. 

In order to \imp{enchant} an item to become magical, one must first create a \key{Arcane Nexus} on the item, and then funnel magical energy in order to \key{Imbue} the \imp{Nexus} with power.  

\subsection{The Nexus}

The first step in the creation of a magical item is the laborious process of \imp{arcane inscription}, in which magical \imp{runes} are drawn over the object to be enchanted using special \key{Runic Tools}. These \imp{runes} form a complicated web of magic known as the \key{nexus} of the object. 


The \imp{nexus} forms the heart of the magical enchantment, with the result that iIf the nexus of an enchanted object is destroyed (an act which normally, though not always, destroys the enchanted object) the enchantment is released and ceases to function. The \imp{Nexus} forms an arcane web which catches, stores, channels and redirects magical power in a certain way, depending on the intent of the enchanter. 


\subsection{Runes}

There are thousands of individual \imp{runes} throughout the known world, each individual culture has their own arcane symbology and methods of placing power into objects. The \imp{Runes} taught at Hogwarts are known as the `\key{Ancient Runes}', and are those most commonly used in Northern Europe since around 200 B.C.

Each rune inscribed on the surface of the object alters the nature of the nexus, and hence changes the kind of magic that can be stored in it, the way that the item is activated, and the way in which the magic is released. The most important part of the enchantment process, therefore, is selecting the runes which will produced the desired effect.

Ones these \imp{runes} are chosen, the enchanter must begin the complex task of inscribing interlinking chains of these \imp{runes} in arcane shapes and patterns over the surface of the object.  The selection of \imp{runes} is therefore of vital importance to the outcome of the enchantment. 


All of the \imp{Ancient Runes} fall into one of three categories: the \key{Control Runes}, the \key{Esoteric Runes} and the \key{Duration Runes}. For a successful echantment, you need at least one rune from each category to be inscribed into the nexus. 


\subsubsection{Learning Runes}

Whilst, for simplicity, the runes are presented as singular entities within this rulebook (the `base rune'), when a witch `learns' a rune, they are actually learning the complex `grammar' needed to conjugate and modify the base runes into the complex webs of magic required for enchanting. 

Learning a new rune, therefore, can be an arduous task, and usually requires studying a tome dedicated to that singular rune for a number of hours. Locating the relevant tome might be an adventure in and of itself - the \imp{Hogwarts Library} is unlikely to stock any tomes regarding the use of \rune{\lues}, the necromantic rune, for example.

As a rough guide, once a rune-tome has been located, a character may perform an \imp{Intelligence (Investigation)} check (DV 7). Each success reduces the required study time by 1 hour, starting from 8 hours. Once this time has elapsed, your character is considered to have memorised the use of this rune. 
\newcommand\runeRow[3]
{
	\rune{#2}	&	#1	&		\small #3 \\
}
\newcommand\esoRow[3]
{
	\runeRow{#1}{#2}{Used to contain magic associated with the \key{#3} discipline.}
}

\newcommand\runeList[3]
{
	\subsection{#1}

	#2

	\begin{center}
		\begin{rndtable}{c l p {6 cm} }
			\bf Rune	&	\bf Name	&	\bf Description
			\\
			
			#3
		\end{rndtable}
	\end{center}
}






\runeList{Control}
{
	The way in which the magical item is used and controlled is determined by the \key{Control Runes} - does the item respond to words and phrases, the approach of a foe, or does it activate when worn?  
}
{
	\runeRow{animax}{\animax}{Used for enchantments which are `sentient', and appear to be controlled at the will of a living being within the item.}

	\runeRow{fabulum}{\fabulum}{Used for triggers that rely on an arcane or magical action occuring, such as a spell being cast upon them, or placed upon an enchanted item.}
	
	\runeRow{iuxta}{a}{Induces a field which can detect the proximity of specified being or objects - useful for triggering effects when an item is approached or worn.}
	
	\runeRow{mentis}{\mentis}{Allows a wielder to control the effects of the item with their mind.}
	
	\runeRow{oculum}{\oculum}{Triggers the enchantment when a visual trigger occurs, such as a particular image appearing, or light landing on it in a specific fashion.}
	
	\runeRow{salto}{\salto}{Useful for enchantments that should trigger when a particular ritualistic motion is performed either near or with the object, or when the object is interacted with in some physical way.}
	
	\runeRow{seculum}{\seculum}{A {\it seculum} rune activates the energy within the nexus on a fixed schedule, allowing the enchanter to create a time-based trigger. }
	
	\runeRow{sessio}{\sessio}{An item enchanted with this rune is permanently active and has no trigger to speak of. The effect is considered `passive' and always active, though at the cost of a somewhat diluted effect.}
	
	\runeRow{vox}{\vox}{This rune activates the nexus when a particular command phrase is said within a certain radius of the item.}
} 

\runeList{Duration}
{
	The \key{Duration Runes} specify how long the effect of the enchanted item lasts after it is activated: does it last for only a few seconds at a time, does it release the effect incredibly quickly then halt, or is the effect permanently active? 
}
{
	\runeRow{displos}{\displos}{Used for effects that act instantaneously\comma{} releasing all their effect an energy in a split second.}
	\runeRow{velox}{\velox}{Used for effects which last for a handful of seconds –  burning a target when struck with a weapon\comma{} or activating a temporary shield.}
	\runeRow{lentus}{\lentus}{Used for effects that last on the duration of minutes to hours. The effects tend to be much more gentle, as the magic gently seeps out over time.}
\runeRow{aeternum}{\aeternum}{Used for effects which last for extended periods of time\comma{} or are constantly active. As with {\it lentus}\comma{} the effects are diluted by the need to conserve energy.}
} 


\runeList{Esoteric}
{
	The \key{Esoteric Runes} shape the nexus to accept magic from a certain discipline, and therefore determines the category of magic the enchantment is capable of reproducing.
}
{
	\esoRow{aevum}{\aevum}{Temporal}
	
	\esoRow{animus}{\animus}{Cerebral}
	
	\esoRow{basiorum}{\basiorum}{Hexes}
	
	\esoRow{canto}{\canto}{Bewitchment}
	
	\esoRow{clypus}{\clypus}{Warding}
	
	\esoRow{genero}{\genero}{Conjuration}
	
	\esoRow{lues}{\lues}{Necromancy}
	
	\esoRow{morbus}{\morbus}{Curses}
	
	\esoRow{motu}{\motu}{Kinesis}
	
	\esoRow{muto}{\muto}{Alteration}
	
	\esoRow{primum}{\primum}{Elemental}
	
	\esoRow{ritus}{\ritus}{Occultism}
	
	\esoRow{sarco}{\sarco}{Hermetics}
	
	\esoRow{vinco}{\vinco}{Psionics}
} 



\subsection{The Enchanting Ritual}

Whilst the \imp{Runes} are being inscribed, the artificer must channel magical energy into the runes, to be stored and shaped in the nexus. To do this, they must hold a very precise idea of the effect they wish to imbue into the item. 

You must describe to the GM what it is you are attempting to achieve, which the GM will use to determine the scale of the enchanting effort that you are attempting, using the tables below. The GM also uses this time to make a judgement call if you are even able to attempt this enchantment, and if the runes you have selected wouldproduce the desired effect. If the GM consents, you may begin to imbuing process. 

For every 6 hours that you spend focussed entirely on the Imbuing process, you may perform an \imp{Imbuing} check as part of a long term project (see page \pageref{S:Extended}). Every success you gain from the check is allocated into the project-pool. 

The DV of the check is determined by the GM from their assessment of the rarity of the item (see \pageref{S:ItemRarity}) and the \imp{Imbuing} ability of the enchanter. 

\newcommand\tHeader[1]{ \cc \imp{ #1} }
\newcommand\dvRow[7]{#1&#2&#3&#4&#5&#6&#7}

\newcommand\dvTable[7]
{
	\footnotesize
	\begin{center}
		\begin{rndtable}{@{} c r c c c c c c c @{}}
			~ & ~ & \multicolumn{7}{c}{\cc \small Imbuing Ability} 
			\\
			\cc & \cc	&	\tHeader{1}	&	\tHeader{2}	&	\tHeader{3}	&	\tHeader{4}	&	\tHeader{5}	&	\tHeader{6}	&	\tHeader{7}
			\\
			\cc &  \tHeader{Trivial}	&	#1
			\\
			\cc & \tHeader{Common} 	&	#2
			\\
			\cc & \tHeader{Singular}	&	#3
			\\
			\cc & \tHeader{Unusual}	&	#4
			\\
			\cc & \tHeader{Rare}	&	#5
			\\
			\cc & \tHeader{Extraordinary}	&	#6
			\\
			\multirow{-7}{*}{\rotatebox[origin=c]{90}{\cc \bf \small Item Rarity} } & \tHeader{Mythical} & #7
		\end{rndtable}
	
	\end{center}
	\normalsize
}

\dvTable
{
	\dvRow{8}{7}{6}{5}{4}{3}{2}
}
{
	\dvRow{9}{8}{7}{6}{5}{4}{3}
}
{
	\dvRow{10}{9}{8}{7}{6}{5}{4}
}
{
	\dvRow{11}{10}{9}{8}{7}{6}{5}
}
{
	\dvRow{-}{11}{10}{9}{8}{7}{6}
}
{
	\dvRow{-}{-}{11}{10}{9}{8}{7}
}
{
	\dvRow{-}{-}{-}{11}{10}{9}{8}
}


\subsubsection{Enchantment Success}

In order for the enchanting ritual to complete you must meet a number of successes determined by the difficulty of the enchanting effort your are attempting:

\newcommand\sucRow[2]{ \key{#1}  & #2 \\}
\begin{center}
\begin{rndtable}{l c}
	\bf Rarity & \bf Successes \\
	\sucRow{Trivial}{5}
	\sucRow{Common}{10}
	\sucRow{Singular}{15}
	\sucRow{Unusual}{20}
	\sucRow{Rate}{30}
	\sucRow{Extraordinary}{40}
	\sucRow{Mythical}{50+}
\end{rndtable}
\end{center}

Once you have reached the required number of successes, you have completed the enchanting process, and now possess a newly enchanted object. Your GM should finalise any remaining questions about the properties of the new object, such as any limitations or finite uses, and then you may begin to use it as you wish.


\subsubsection{Enchantment Failure}

As per the rules regarding long-term projects, if a \imp{Catastrophic Failure} ever results in the project successes to go below zero, the action fails. Unfortunately, with a process as delicate and fiddly as the construction of a \imp{Nexus}, such a failure probably ends badly. 

The exact nature of the failure of an enchantment is up for the GM to decide, but the total destruction of the original item would be considered the absolute minimum - a small explosion catching those nearby and dealing a small amount of \imp{Harm} would also be reasonable. \imp{Artificers} learn the hard way the importance of protective gear!







\subsection{The Limits of Enchanting}

Although it is possible for an unskilled indivudal to lay their hands on the \imp{runes} needed to create even the most powerful of items, the basic rules of magic still apply. For instance, one cannot enchant a magic item to produce infinite amounts of food, as this would violate the rules regarding conjuration magic. See the page \pageref{S:Laws} for more on the \imp{Laws of Magic}.

An additional note of caution is warranted regarding the use of the rune \rune{\animax}, the rune of sentience. Only in the hands of the most powerful enchanters can this replicate a true sentient mind within the enchanted item (such as that found within the \imp{Sorting Hat} ). In the hands of most enchanters this rune is much more limited, producing a more animalistic, basic mind which, though able to respond to external stimuli, is not intelligent or conscious.  


\subsection{Multiple Effects}

Sometimes you may want to layer multiple effects on a single item. 

\subsubsection{Complimentary Effects} 

If these individual effects compliment each other, and form part of a singular cohesive structure, then they can be chained together into a single enchantment. 

An enchanted cloak which lets you control nearby fire with your mind, but also bestows a resistance against fire could be enchanted with the combined runechain \rune{\mentis\velox\primum\iuxta\aeternum\clypus} - though the runes are very different, they clearly flow as part of the same enchantment (a `cloak of fire') and so the nexus can easily be modified to accomodate this additional effect. 

The individual effects would be weaker than if you had just chosen one of the effects, or the item would be classed as a significantly \imp{rarer} item, therefore requiring more time and effort to enchant, but this poses no intrinsic problems. 

\subsubsection{Non-Complimentary Effects}

However, if you attempt to enchant drastically different effects layered onto the same artefact \minus{} you may wish to have a sword which contains a vicious toxin in the blade (\rune{\sessio\displos\morbus}), but also allows you to read the minds of your enemies (\rune{\mentis\velox\animus}). Whilst undeniably a powerful weapon, these two magical effects don't necessarily `mesh' well together, and you would have to twist yourself in nots to try and come up with a narrative reason why these two effects arise from a single magical effect. 

These effects, therefore cannot be performed as part of the same enchantment ritual, and only the most powerful of \imp{Artificers} are able to enchant an already enchanted item. 


\subsection{Some Examples}\label{E:EnchantingExamples}

For the purposes of illustration, the table below contains a brief description of the enchanting profiles of some well-known magical items:

\newcommand\artefactRow[6]{\small \imp{#1} & \parbox[t]{1 cm}{\raggedright \rune{#2}\\ {\it \tiny #3\\#4\\#5}}& \parbox[t]{5.4 cm}{\raggedright \footnotesize #6}  \\} 
\newcommand\artefactList[2]
{
	\small
	\subsubsection{#1}
	\begin{center}
	\begin{rndtable}{p{2cm} c c c}
		Name	&	Runes	&	Description	 \\
		#2
	\end{rndtable}
\end{center}
	
	\normalsize
}


\artefactList{Trivial}
{
	\artefactRow{Disappearing Ink}{\vox\displos\canto}{Voice}{Instant}{Bewitchment}{A vial of ink which, when a command phrase is uttered, switches between visible and invisible.}
	\artefactRow{Enchanted Origami}{\animax\lentus\motu}{Sentient}{Long}{Kinesis}{A weakly enchanted piece of paper, folded to appear as an animal. The enchantment causes it to `come alive' for a period of time.}
	\artefactRow{Magical Lantern}{\oculum\aeternum\primum}{Visual}{Eternal}{Elemental}{A simple object which glows brightly when placed in a region of darkness.}
	\artefactRow{Swindler's Coin}{\mentis\displos\muto}{Mental}{Instant}{Change}{A small silver sickle which appears perfectly normal, when tossed, the owner can control if it lands on heads or tails with perfect accuracy.}
}

\artefactList{Common}
{
	\artefactRow{Beautifying Robes}{\iuxta\aeternum\canto}{Proximity}{Eternal}{Bewitchment}{A set of robes which make the wearer appear more physically attractive.}
	\artefactRow{Bludger}{\animax\lentus\motu}{Sentient}{Long}{Kinesis}{A strong though simple enchantment placed on an enchanted solid ball used in \imp{Quidditch}. When released, the ball seeks out players and attempts to smash them.}
	\artefactRow{Rememberall}{\sessio\aeternum\animus}{Proximity}{Eternal}{Cerebral}{A small orb which changes colour when someone nearby forgets something.}
	\artefactRow{Sneakoscope}{\fabulum\velox\animus}{Arcane}{Short}{Cerebral}{A small object which buzzes and hums when it detects the usage of magic from the \imp{Dark Arts} school.}
}
\artefactList{Singular}
{
	\artefactRow{Two-Way Mirrors}{\vox\lentus\animus}{Voice}{Long}{Cerebral}{A pair of small, handheld mirrors. When a command word is spoken, they may be used to communicate with each other akin to a muggle `video call'.}
	\artefactRow{Wound-Sealing Cloak}{\salto\displos\sarco}{Somatic}{Instant}{Hermetic}{When an attack passes through this cloak, it automatically seals itself around the wound to prevent further infection or bleeding, healing up to level 3 \imp{Harm} once per day.}
}
\artefactList{Unusual}
{
	\artefactRow{Basic Broom}{\mentis\aeternum\motu}{Mental}{Eternal}{Kinesis}{A basic broomstick allows the user to fly, though they won't be breaking any records whilst doing so.}
}

\artefactList{Rare}
{

}

\artefactList{Extraordinary}
{
	\artefactRow{Racing Broom}{\mentis\aeternum\motu}{Mental}{Eternal}{Kinesis}{A far superior enchantment when compared to the basic version, a racing broom turns tighter, responds quicker and goes like the clappers.}

}



\artefactList{Mythical}
{
	\artefactRow{Astral Cloak}{\iuxta\aeternum\canto\aevum}{Proximity}{Eternal}{Bewitchment \& Temporal}{A true cloak of invisibility, shifting the wearer partly into another realm, and thereby protecting them entirely from magical and mundane attacks. Magical effects cannot pass through this wondrous item.}
	\artefactRow{Horcrux}{\sessio\aeternum\lues}{Passive}{Eternal}{Necromancy}{A horcrux, on its own, is nothing particularly interesting: merely an exquisitely prepared vessel. When paired with a profane and disgusting ritual (the details of which are to horrifying to describe here), however, it can be used to store a part of the creator's soul, thereby tethering them to the realm of the living as long as the horcrux is intact.}
}

