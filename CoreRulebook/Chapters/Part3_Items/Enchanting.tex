\chapter{Artificing}\label{S:Artificing}


\key{Artificing} is the art of creating new items, typically those imbued with magical powers. 

The most prominent examples of \imp{artificing} in the Wizarding world are \key{Enchanting} and \key{Alchemy}, guided by the \key{Imbuement} ability. Non-magical \imp{Artificing} is usually referred to as \key{crafting}, and controlled by the ability of the same name.  

\section{Enchanting}
\label{S:Enchanting}

\imp{Enchanting} is the process whereby magical items are made – imbuing them with extraordinary abilities. 

In order to \imp{enchant} an item to become magical, one must first create a \key{Arcane Nexus} on the item, and then funnel magical energy in order to \key{Imbue} the \imp{Nexus} with power.  

\subsection{The Nexus}

The first step in the creation of a magical item is the laborious process of \imp{arcane inscription}, in which magical \imp{runes} are drawn over the object to be enchanted using special \key{Runic Tools}. These \imp{runes} form a complicated web of magic known as the \key{nexus} of the object. 


The \imp{nexus} forms the heart of the magical enchantment, with the result that iIf the nexus of an enchanted object is destroyed (an act which normally, though not always, destroys the enchanted object) the enchantment is released and ceases to function. The \imp{Nexus} forms an arcane web which catches, stores, channels and redirects magical power in a certain way, depending on the intent of the enchanter. 


\subsection{Runes}

There are thousands of individual \imp{runes} throughout the known world, each individual culture has their own arcane symbology and methods of placing power into objects. The \imp{Runes} taught at Hogwarts are known as the `\key{Ancient Runes}', and are those most commonly used in Northern Europe since around 200 B.C.

Each rune inscribed on the surface of the object alters the nature of the nexus, and hence changes the kind of magic that can be stored in it, the way that the item is activated, and the way in which the magic is released. The most important part of the enchantment process, therefore, is selecting the runes which will produced the desired effect.

Ones these \imp{runes} are chosen, the enchanter must begin the complex task of inscribing interlinking chains of these \imp{runes} in arcane shapes and patterns over the surface of the object.  The selection of \imp{runes} is therefore of vital importance to the outcome of the enchantment. 


All of the \imp{base runes} fall into one of three categories: the \key{Control Runes}, the \key{Esoteric Runes} and the \key{Duration Runes}. For a successful echantment, you need at least one rune from each category to be inscribed into the nexus. 



\newcommand\runeRow[3]
{
	\rune{#2}	&	#1	&		\small #3 \\
}
\newcommand\esoRow[3]
{
	\runeRow{#1}{#2}{Used to contain magic associated with the \key{#3} discipline.}
}

\newcommand\runeList[3]
{
	\subsection{#1}

	#2

	\begin{center}
		\begin{rndtable}{c l p {6 cm} }
			\bf Rune	&	\bf Name	&	\bf Description
			\\
			
			#3
		\end{rndtable}
	\end{center}
}






\runeList{Control}
{
	The way in which the magical item is used and controlled is determined by the \key{Control Runes} - does the item respond to words and phrases, the approach of a foe, or does it activate when worn?  
}
{
	\runeRow{animax}{\animax}{Used for enchantments which are `sentient', and appear to be controlled at the will of a living being within the item.}

	\runeRow{fabulum}{\fabulum}{Used for triggers that rely on an arcane or magical action occuring, such as a spell being cast upon them, or placed upon an enchanted item.}
	
	\runeRow{iuxta}{a}{Induces a field which can detect the proximity of specified being or objects - useful for triggering effects when an item is approached or worn.}
	
	\runeRow{mentis}{\mentis}{Allows a wielder to control the effects of the item with their mind.}
	
	\runeRow{oculum}{\oculum}{Triggers the enchantment when a visual trigger occurs, such as a particular image appearing, or light landing on it in a specific fashion.}
	
	\runeRow{salto}{\salto}{Useful for enchantments that should trigger when a particular ritualistic motion is performed either near or with the object, or when the object is interacted with in some physical way.}
	
	\runeRow{seculum}{\seculum}{A {\it seculum} rune activates the energy within the nexus on a fixed schedule, allowing the enchanter to create a time-based trigger. }
	
	\runeRow{sessio}{\sessio}{An item enchanted with this rune is permanently active and has no trigger to speak of. The effect is considered `passive' and always active, though at the cost of a somewhat diluted effect.}
	
	\runeRow{vox}{\vox}{This rune activates the nexus when a particular command phrase is said within a certain radius of the item.}
} 

\runeList{Duration}
{
	The \key{Duration Runes} specify how long the effect of the enchanted item lasts after it is activated: does it last for only a few seconds at a time, does it release the effect incredibly quickly then halt, or is the effect permanently active? 
}
{
	\runeRow{displos}{\displos}{Used for effects that act instantaneously\comma{} releasing all their effect an energy in a split second.}
	\runeRow{velox}{\velox}{Used for effects which last for a handful of seconds –  burning a target when struck with a weapon\comma{} or activating a temporary shield.}
	\runeRow{lentus}{\lentus}{Used for effects that last on the duration of minutes to hours. The effects tend to be much more gentle, as the magic gently seeps out over time.}
\runeRow{aeternum}{\aeternum}{Used for effects which last for extended periods of time\comma{} or are constantly active. As with {\it lentus}\comma{} the effects are diluted by the need to conserve energy.}
} 


\runeList{Esoteric}
{
	The \key{Esoteric Runes} shape the nexus to accept magic from a certain discipline, and therefore determines the category of magic the enchantment is capable of reproducing.
}
{
	\esoRow{aevum}{\aevum}{Temporal}
	
	\esoRow{animus}{\animus}{Cerebral}
	
	\esoRow{basiorum}{\basiorum}{Hexes}
	
	\esoRow{canto}{\canto}{Bewitchment}
	
	\esoRow{clypus}{\clypus}{Warding}
	
	\esoRow{genero}{\genero}{Conjuration}
	
	\esoRow{lues}{\lues}{Necromancy}
	
	\esoRow{morbus}{\morbus}{Curses}
	
	\esoRow{motu}{\motu}{Kinesis}
	
	\esoRow{muto}{\muto}{Alteration}
	
	\esoRow{primum}{\primum}{Elemental}
	
	\esoRow{ritus}{\ritus}{Occultism}
	
	\esoRow{sarco}{\sarco}{Hermetics}
	
	\esoRow{vinco}{\vinco}{Psionics}
} 





\subsection{Imbuing}

Whilst the \imp{Runes} are being inscribed, the artificer must channel magical energy into the runes, to be stored and shaped in the nexus. To do this, they must hold a very precise idea of the effect they wish to imbue into the item. 

You must describe to the GM what it is you are attempting to achieve, which the GM will use to determine the scale of the enchanting effort that you are attempting, using the tables below. The GM also uses this time to make a judgement call if you are even able to attempt this enchantment, and if the runes you have selected wouldproduce the desired effect. If the GM consents, you may begin to imbuing process. 

For every 6 hours that you spend focussed entirely on the Imbuing process, you may perform an \imp{Imbuing} check as part of a long term project (see page \pageref{S:Extended}). Every success you gain from the check is allocated into the project-pool. 

The DV of the check is determined by the GM from their assessment of the rarity of the item (see \pageref{S:ItemRarity}) and the \imp{Imbuing} ability of the enchanter. 

\newcommand\tHeader[1]{ \cc \imp{ #1} }
\newcommand\dvRow[7]{#1&#2&#3&#4&#5&#6&#7}

\newcommand\dvTable[7]
{
	\footnotesize
	\begin{center}
		\begin{rndtable}{@{} c r c c c c c c c @{}}
			~ & ~ & \multicolumn{7}{c}{\cc \small Imbuing Ability} 
			\\
			\cc & \cc	&	\tHeader{1}	&	\tHeader{2}	&	\tHeader{3}	&	\tHeader{4}	&	\tHeader{5}	&	\tHeader{6}	&	\tHeader{7}
			\\
			\cc &  \tHeader{Trivial}	&	#1
			\\
			\cc & \tHeader{Common} 	&	#2
			\\
			\cc & \tHeader{Singular}	&	#3
			\\
			\cc & \tHeader{Unusual}	&	#4
			\\
			\cc & \tHeader{Rare}	&	#5
			\\
			\cc & \tHeader{Extraordinary}	&	#6
			\\
			\multirow{-7}{*}{\rotatebox[origin=c]{90}{\cc \bf \small Item Rarity} } & \tHeader{Mythical} & #7
		\end{rndtable}
	
	\end{center}
	\normalsize
}

\dvTable
{
	\dvRow{8}{7}{6}{5}{4}{3}{2}
}
{
	\dvRow{9}{8}{7}{6}{5}{4}{3}
}
{
	\dvRow{10}{9}{8}{7}{6}{5}{4}
}
{
	\dvRow{11}{10}{9}{8}{7}{6}{5}
}
{
	\dvRow{-}{11}{10}{9}{8}{7}{6}
}
{
	\dvRow{-}{-}{11}{10}{9}{8}{7}
}
{
	\dvRow{-}{-}{-}{11}{10}{9}{8}
}


\subsubsection{Enchantment Success}

In order for the enchanting ritual to complete you must meet a number of successes determined by the difficulty of the enchanting effort your are attempting



%~ \newcommand\enchantRow[3]{ \key{#1} & \parbox[t]{4.5 cm}{\small \raggedright #2} & #3 \\}


%~ \begin{center}
	%~ \begin{rndtable}{c c c}
		%~ \key{Difficulty}	&	\key{Description}	&	\key{Successes}
		%~ \\
		%~ \enchantRow{Trivial}{Very simple and cheap enchanted objects - a normal wizard would expect to have dozens of such items in their possession.}{5}
		%~ \enchantRow{Common}{Though they might not own many as many of these items, a normal witch would be familiar and comfortable with \imp{Common} items in everyday life.}{10}
		%~ \enchantRow{Singular}{Verging on impressive magical power, your normal magic-user probably owns only a handful of these items, and they would be treated as prized possessions.}{15}
		%~ \enchantRow{Unusual}{An \imp{Unusual} magical item would usually be enough to draw inquisitive looks when displayed - a normal witch or wizard probably only owns one or two of these throughout their lifetime (if they are lucky).}{20}
		%~ \enchantRow{Rare}{A rare item would be passed down through a family as a treasured heirloom, and kept under the highest security when not being used. There are probably no more than a few thousand \imp{Rare} items in the country at any one time. }{20}
		%~ \enchantRow{Extraordinary}{Most witches and wizards would only ever have seen an \imp{Extraordinary} magical item, and very, very few will have owned them. An object of this power is usually either stored in \imp{Gringotts}, or on display in a museum.}{25}
		%~ \enchantRow{Mythical}{\imp{Mythical} items are unique magical items which would be remembered throughout history for their enigmatic magic and immense power. Even being able to see, let alone own, a Mythical item would be a thing to tell stories about.}{30}
	%~ \end{rndtable}
%~ \end{center}


\subsubsection{Enchantment Failure}

As per the rules regarding long-term projects, if a \imp{Catastrophic Failure} ever results in the project successes to go below zero, the action fails. Unfortunately, with a process as delicate and fiddly as the construction of a \imp{Nexus}, such a failure probably ends badly. 

The exact nature of the failure of an enchantment is up for the GM to decide, but the total destruction of the original item would be considered the absolute minimum - a small explosion catching those nearby and dealing a small amount of \imp{Harm} would also be reasonable. \imp{Artificers} learn the hard way the importance of protective gear!







%~ \subsubsection{The Limits of Enchanting}

%~ Although it is possible for an unskilled indivudal to lay their hands on a copy of even the most advanced \imp{runes}, this does not mean that you can enchant whatever you desire. 

%~ A general rule of thumb is that you cannot enchant an item which would outperform a spell of your current level. 

%~ For example, a level 5 character only has access to Novice level spells, but could have access to the runechain \rune{\displos\perdero\hominus} ({\it displos perdero hominus}, instant destroy body), and is attempting to utilise these \imp{runes} to curse an item with an effect which would cause instant death to the next person to touch it. Instant death, however, is the domain of {\it Word of Death}, a Master level necromancy spell. The GM would therefore assign this an incredibly high DV, or simply rule that this is an impossible task, far beyond your current capabilities. 

%~ Alternatively, you may be able to work with the GM to find way for the effect to be curtailed to an appropriate level \minus{} maybe this cursed object does kill, but only after prolongued contact, during which the caster suffers progressive maladies such as nosebleeds and headaches. This reduces the immediate threat (and hence game\minus{}breaking nature) of the enchantment, but keeps its fundamental essence intact. 

%~ In addition, whilst it is possible for the runechain \rune{\aeternum\cingo\sensus} to imbue items with a limited amount of sentience and ability to function independently (this runechain is found on the bludger and golden snitch, for example), it is outside the realm of most wizards to imbue an item with true sentience. Only the Artificers have discovered how to imbue an item with original thought and true, actual consciousness. 

%~ \subsubsection{Multiple Effects}

%~ Sometimes you may want to layer multiple effects on a single item. 

%~ If these individual effects compliment each other, and form part of a singular cohesive structure, then they can be chained together into a single enchantment. 

%~ An enchantment which lets you create and then manipulate fire, for example could be enchanted as part of a single runechain: \rune{\lentus\genero\ignis\lentus\imperum\ignis} (which you could probably shorten to \rune{\lentus\genero\imperum\ignis}). 

%~ The individual effects would be weaker than if you had just chosen one of the effects, or the DV might be significantly higher, but this poses no intrinsic problems, as the \imp{runes} work well together. 

%~ However, you attempt to enchant drastically different effects layered onto the same artefact \minus{} you may wish to have a sword which contains a vicious toxin in the blade (\rune{\velox\perdero\morbus}), but also allows you to read the minds of your enemies (\rune{\aeternum\discite\sensus}). These cannot be performed as part of the same enchantment ritual \minus{} you must perform the enchantment twice. 

%~ Note, however, that multiple enchantments (even if they compliment each other) can destabilise the magical nexus. The associated DV of multiply enchanted objects rises exponentially as more effects are added, and the odds of the item blowing up in your hands increases commensurately.  

\subsection{Some Examples}\label{E:EnchantingExamples}

For the purposes of illustration, the table below contains a brief description of the enchanting profiles of some well-known magical items:

\newcommand\artefactRow[6]{\small \imp{#1} & \parbox[t]{1 cm}{\raggedright \rune{#2}\\ {\it \tiny #3\\#4\\#5}}& \parbox[t]{5.4 cm}{\raggedright \footnotesize #6}  \\} 
\newcommand\artefactList[2]
{
	\small
	\subsubsection{#1}
	\begin{center}
	\begin{rndtable}{p{2cm} c c c}
		Name	&	Runes	&	Description	 \\
		#2
	\end{rndtable}
\end{center}
	
	\normalsize
}


\artefactList{Trivial}
{
	\artefactRow{Disappearing Ink}{\vox\displos\canto}{Voice}{Instant}{Bewitchment}{A vial of ink which, when a command phrase is uttered, switches between visible and invisible.}
	\artefactRow{Enchanted Origami}{\animax\lentus\motu}{Sentient}{Long}{Kinesis}{A weakly enchanted piece of paper, folded to appear as an animal. The enchantment causes it to `come alive' for a period of time.}
	\artefactRow{Magical Lantern}{\oculum\aeternum\primum}{Visual}{Eternal}{Elemental}{A simple object which glows brightly when placed in a region of darkness.}
	\artefactRow{Sneakoscope}{\fabulum\velox\animus}{Arcane}{Short}{Cerebral}{A small object which buzzes and hums when it detects the usage of magic from the \imp{Dark Arts} school.}
}

\artefactList{Common}
{
	\artefactRow{Beautifying Robes}{\iuxta\aeternum\canto}{Proximity}{Eternal}{Bewitchment}{A set of robes which make the wearer appear more physically attractive.}
	\artefactRow{Bludger}{\animax\lentus\motu}{Sentient}{Long}{Kinesis}{A strong though simple enchantment placed on an enchanted solid ball used in \imp{Quidditch}. When released, the ball seeks out players and attempts to smash them.}
	\artefactRow{Rememberall}{\sessio\aeternum\animus}{Proximity}{Eternal}{Cerebral}{A small orb which changes colour when someone nearby forgets something.}
	\artefactRow{Swindler's Coin}{\mentis\displos\muto}{Mental}{Instant}{Change}{A small silver sickle which appears perfectly normal, when tossed, the owner can control if it lands on heads or tails with perfect accuracy.}
}
\artefactList{Singular}
{
	\artefactRow{Two-Way Mirrors}{\vox\lentus\animus}{Voice}{Long}{Cerebral}{A pair of small, handheld mirrors. When a command word is spoken, they may be used to communicate with each other akin to a muggle `video call'.}
	\artefactRow{Wound-Sealing Cloak}{\salto\displos\sarco}{Somatic}{Instant}{Hermetic}{When an attack passes through this cloak, it automatically seals itself around the wound to prevent further infection or bleeding, healing up to level 3 \imp{Harm} once per day.}
}
\artefactList{Unusual}
{
	\artefactRow{Basic Broom}{\mentis\aeternum\motu}{Mental}{Eternal}{Kinesis}{A basic broomstick allows the user to fly, though they won't be breaking any records whilst doing so.}
}

\artefactList{Rare}
{

}

\artefactList{Extraordinary}
{
	\artefactRow{Racing Broom}{\mentis\aeternum\motu}{Mental}{Eternal}{Kinesis}{A far superior enchantment when compared to the basic version, a racing broom turns tighter, responds quicker and goes like the clappers.}

}



\artefactList{Mythical}
{
	\artefactRow{Astral Cloak}{\iuxta\aeternum\canto\aevum}{Proximity}{Eternal}{Bewitchment \& Temporal}{A true cloak of invisibility, shifting the wearer partly into another realm, and thereby protecting them entirely from magical and mundane attacks. Magical effects cannot pass through this wondrous item.}
	\artefactRow{Horcrux}{\sessio\aeternum\lues}{Passive}{Eternal}{Necromancy}{A horcrux, on its own, is nothing particularly interesting: merely an exquisitely prepared vessel. When paired with a profane and disgusting ritual (the details of which are to horrifying to describe here), however, it can be used to store a part of the creator's soul, thereby tethering them to the realm of the living as long as the horcrux is intact.}
}

%~ \subsubsection{Some Examples}

%~ For the purposes of an example, the list below contains the runechains that are used to enchant some of the common magical artefacts found in the wizarding world. 


%~ \def\w{2.4}
%~ \def\q{4}
%~ \newcommand\artefactRow[4]
%~ {
%~ \small #1	&	\parbox[t]{\w cm}{{\centering \rune{#2}} \\ \raggedright \it \footnotesize #3}	&	\parbox[t]{\q cm}{\footnotesize  #4} \\
%~ }
%~ \begin{rndtable}{@{} p{\w cm} p{\w cm} p {\q cm}}
%~ \bf Item 	&	\bf \imp{runes}	&	\bf	Justification
%~ \\
%~ \artefactRow{Bludger}{\lentus\cingo\sensus\lentus\imperum\pondus}{Long contain mind, long control matter}{The first string provides the bludger with a limited amount of sentience and the second allows it fly and maneouvre itself for a few hours, after being activated.}
%~ \artefactRow{Deluminator}{\velox\perdero\lux \velox\sarco\lux}{Short destroy light, short restore light}{The deluminator sucks in nearby light on activation (the first half), and then restores it on a second activation (the second half). }
%~ \artefactRow{Penseive}{\aeternum\cingo\sensus\lentus\discite\sensus}{Eternal store mind, long percieve mind }{A penseive acts as a permanent storage place for memories, and also allows the user to dive in for extended periods of time to view them. }
%~ \artefactRow{Portkey}{\displos\porto\pondus}{Instant transmit matter}{The portkey performs a single simple purpose: teleport matter instantaneously upon activation.}
%~ \artefactRow{Self\minus{}Erecting Tent}{\aeternum\genero\locus\aeternum\cingo\pondus\velox\imperum\pondus}{Eternal create space, eternal contain matter, short control matter}{The first two strings make the tent have a larger volume on the inside and to make it act as a shelter to objects inside. The final string enables the tent to assemble itself over a short period of time.}
%~ \artefactRow{Sneakoscope}{\aeternum\discite\morbus\velox\imperum\pondus}{Eternal percieve cursed, short control matter}{The primary effect of the sneakoscope is contained in the first string: the detection of evil and cursed objects. The second string merely provides the alert mechanism \minus{} the object whistles and spins of its own accord.}
%~ \end{rndtable}
