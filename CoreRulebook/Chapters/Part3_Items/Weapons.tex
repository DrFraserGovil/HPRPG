

\chapter{Weapons \& Wands}
 



\section{Wands}\label{S:Wands}

The most important tool of any witch or wizard is their wand.

Unlike with other items\comma{} you don't get to choose your wand\comma{} since it is well known that the wand chooses the wizard\comma{} not the other way around. The process for selecting your wand is to roll two d6 successively. The first roll determines the wood your wand is made of\comma{} the second determines the core. 

Different materials have an affinity with different kinds of magic\comma{} and make casting those spells easier. Wood makes the spell type easier to cast (+1 to checks)\comma{} and the core reduces the mental strain of casting that class of spell (\minus{}1 FP cost). 
 \footnotesize
 \begin{center}

 \begin{rndtable}{|c c c c|}

 \hline
 \bf Roll & \bf Magic School & \bf Wood& \bf Core
 \\
 1 & Defensive & Apple & Pheonix feather
 \\
 2 & Hexes \& Curses & Holly & Dragon heartstring
 \\ 
 3 & Divination & Beech & Unicorn Tail hair
 \\ 
 4 & Transifguration & Oak & Thunderbird feather
 \\ 
 5 & Charms & Hawthorn & Kelpie hair
 \\ 
 6 & Illusion & Hazel & Veela hair
 \\ 
 \minus{} & Dark Arts & Human Bone & Dementor Robe
 \\\hline
 \end{rndtable}
 \end{center}
 
\normalsize
 
If your original wand is destroyed or lost\comma{} you need to find someone who can sell (or make) you a new one\comma{} and perform the selection process anew. 
 
The only way to access the 7th and final category of wand is to have an EVL greater than 8. This then bypasses all other wand selection checks\comma{} and your wand is necessarily evil. It should of course be noted that wandmakers aren{\apos}t too happy to sell these evil objects \minus{}\minus{} you might have to cut a few bits off in order to sufficiently motivate them.  


\newpage


% \section*{Melee Weapons}

% Magical combat and the use of wands is covered in detail elsewhere in this guide\comma{} but what happens when you just want to hit the bad guys with big sticks? Most wizards are inexperienced in the art of physical combat\comma{} but those with the {\it Brawler} and {\it Archer} skills can attack people with their fists\comma{} with steel\comma{} or with longer ranged weapons. 

% Physical combat is underrated in the magical world\comma{} but it can be used to devastating effect. When you have moved in close enough to someone\comma{} they do not have the time or room to cast an effective counterspell\comma{} and attempts to do so trigger an `attack of opportunity'. Hence\comma{} your enemy is effectively at the mercy of you and your big stick...unless they have one of their own. In addition to this\comma{} many magical defences do not defend against physical objects\comma{} so throwing a rock through a shield charm can often be a good tactic.

% Physical weapons come in two types: melee \comma{} and ranged. Melee weapons are close\minus{}quarters weapons like swords\comma{} daggers and so on\comma{} and can only be used within a 1m radius of the target. Ranged weapons are bows and arrows and even guns\comma{} and can be used from larger distances. 

% Weapon usage does not cost any Fortitude points\comma{} and so is often a last resort if your character has no more magic spells remaining. 

% To perform a melee attack\comma{} you must have the item equipped in one of your hands (or both)\footnote{There is a 2 point penalty on any checks for weapons in your non\minus{}dominant hand} and be stood adjacent to the target. Some weapons (such as spears and battleaxes) have a longer reach. 

% Melee weapons are so simple that they are automatically assumed to hit their target\comma{} unless the target is actively dodging\comma{} in which case the usual evasion rules apply. Unarmed strikes do 1HP of damage\comma{} and strikes with weapons use a specified weapon check (usually an ATH (strength) check\comma{} with a variable die size). 

% Because a melee attack is up close and personal\comma{} it does not usually give spellcasters enough time to retaliate with a counterspell. A non\minus{}conditional spell will still be cast before you land your blow\comma{} however\comma{} though it will trigger an attack of opportunity on the spellcaster. 

% All melee weapons can be used from the beginning of the game \minus{}\minus{} however you are not considered proficient in them until you have the relevant {\it Brawler} skill. Using weapons that you are not proficient in means that you cannot apply any positive modifiers (and negative weapon modifiers are doubled) on all weapon\minus{}related checks (included evasion and anti\minus{}evasion checks)\comma{} and always open you up to attacks of opportunity. 

% The table below gives a rough overview of the weapons available\comma{} and how other effects. 

% \section*{Ranged Weapons}

% Unlike melee weapons\comma{} missing the target entirely is a rather real prospect. Ranged weapons cannot be used on any target any closer than 5m\comma{} and you need to have the Archer skill to make use of long ranged weapons. 

% After selecting your target\comma{} you must then check if the projectile hits its target. The projectile check uses a varying dice depending on the level of the Archery skill. The base level Archery skill gets you a 1d4 dice to use. The projectile hits its target if the distance to the target is \textbf{less than 5 times the dice roll} 

% Therefore if you roll a 6 to hit a target that is 30 metres away\comma{} the projectile misses\comma{} as $6 \times 5 = 30$ m\comma{} and we need the dice roll to be \textbf{larger}. If the target had been 1 metre closer\comma{} it would indeed have succeeded. 

% Increasing the Archery skill gets you access to larger dice\comma{} and hence increases the distance that you can reach\comma{} and the liklihood of success at lower distances.  If the projectile accuracy check succeeds\comma{} the relevant evasion checks are applied\comma{} and then the damage check is performed to determine how much damage is done. 

% \newpage
% \subsection*{Weapon Types \& Improvements}

% The table on the next page gives the statistics for a handful of the most common weapon types\comma{} including the generalised damage checks. 

% However\comma{} there are of course different qualities of weapons \minus{}\minus{} a finely crafted sword is going to be a more formiddable weapon than a hastily thrown together blade. Different materials can also hold an edge for longer\comma{} and hence do more damage\comma{} and last longer. 

% The weapon list is given assuming the weapon is a base\minus{}level iron weapon. Use the following table to account for better (or worse) quality weapons. Weapon damage cannot go below 0. 
% \def\y{2.6}
% \small
% \begin{center}
% \begin{rndtable}{|c c c m {\y cm}|}
% \hline
% \bf Material & \bf Damage & \bf Blunting & \bf Notes
% \\
% Wood & \minus{}3  & 10 uses  & \parbox[t]{\y cm}{\raggedright Illusion magics bind strongly to wood}
% \\
% Bone & \minus{}1  & 20 uses &  \parbox[t]{\y cm}{\raggedright Dark Arts bind strongly to bone}
% \\
% Iron & +0 & 30 uses & 
% \\
% Steel & +1 & 50 uses & 
% \\
% Meteorite\minus{}iron & +2 & 100 uses &  \parbox[t]{\y cm}{\raggedright Especially powerful enchantments can be bound to meteorite\minus{}iron.}
% \\
% Adamantium & + 3 & Does not blunt &  \parbox[t]{\y cm}{\raggedright Cannot be forged or enchanted }
% \\ 
% Silver & +1 & 30 uses & \parbox[t]{\y cm}{\raggedright Does double damage to undead}
% \\ 
% \end{rndtable}
% \end{center}

% \normalsize
% Other materials and/or bonuses may be introduced as is story appropriate. 

% Weapons may also be modified by being enchanted (see below)\comma{} or having a chemical/potion applied to them\comma{} in order to add an extra effect to the weapon. This does not generally affect the other properties of the weapon\comma{} with the exception of things such as strong acid\comma{} which would obviously impinge the integrity of a metal sword!


\onecolumn
\section{Weapon List}\label{S:WeaponList}
\small
\def\l{8}
\begin{center}
%%WeaponsBegin
\begin{rndtable}{|l l c c l |}\hline \normalsize \bf Weapon & \normalsize \bf Cost & \normalsize \bf Modifier &  \normalsize \bf Damage & \normalsize \bf Properties \\ \hline{ \it Unarmed Weapons} & & & & \\ 
\bf ~~~~~Unarmed Strike	&		&	\attPhys	&	1~Bludgeoning	&	\parbox[t]{\l cm}{}\\ 
\bf ~~~~~Improvised Weapons	&		&	?	&	1d4~	&	\parbox[t]{\l cm}{(GM fiat takes precedence: use similarity to existing weapons)}\\ 
{ \it Simple Weapons} & & & & \\ 
\bf ~~~~~Club	&	\sickle{1}~	&	\attPhys	&	1d4~Bludgeoning	&	\parbox[t]{\l cm}{}\\ 
\bf ~~~~~Dagger	&	\sickle{10}~	&	Versatile	&	1d4~Piercing	&	\parbox[t]{\l cm}{Can be thrown, range: 5m}\\ 
\bf ~~~~~Quarterstaff	&	\sickle{4}~	&	Versatile	&	1d6~Bludgeoning	&	\parbox[t]{\l cm}{Multi-handed (1d8)}\\ 
\bf ~~~~~Spear	&	\sickle{10}~	&	\attPhys	&	1d8~Piercing	&	\parbox[t]{\l cm}{Can be thrown, range: 10m}\\ 
{ \it Bladed Weapons} & & & & \\ 
\bf ~~~~~Greatsword	&	\galleon{6}~	&	\attPhys	&	2d6~Slashing	&	\parbox[t]{\l cm}{Two-handed}\\ 
\bf ~~~~~Longsword	&	\galleon{5}~	&	\attPhys	&	2d4~Slashing	&	\parbox[t]{\l cm}{}\\ 
\bf ~~~~~Rapier	&	\galleon{3}~	&	\attFin	&	1d8~Piercing	&	\parbox[t]{\l cm}{}\\ 
\bf ~~~~~Shortsword	&	\galleon{3}~	&	Versatile	&	1d6~Slashing	&	\parbox[t]{\l cm}{}\\ 
{ \it Brutish Weapons} & & & & \\ 
\bf ~~~~~Greataxe	&	\galleon{2}~	&	\attPhys	&	1d12~Slashing	&	\parbox[t]{\l cm}{Two-handed}\\ 
\bf ~~~~~Light Axe	&	\galleon{1}~	&	\attPhys	&	1d6~Slashing	&	\parbox[t]{\l cm}{Can be thrown, range: 5m}\\ 
\bf ~~~~~Mace	&	\galleon{1}~	&	\attPhys	&	1d6~Bludgeoning	&	\parbox[t]{\l cm}{}\\ 
\bf ~~~~~Warhammer	&	\galleon{3}~	&	\attPhys	&	2d4~Bludgeoning	&	\parbox[t]{\l cm}{Two-handed}\\ 
{ \it Reach Weapons} & & & & \\ 
\bf ~~~~~Glaive	&	\galleon{2}~\sickle{10}~	&	\attPhys	&	2d6~Slashing	&	\parbox[t]{\l cm}{Two-handed, reach 2m}\\ 
\bf ~~~~~Lance	&	\galleon{2}~\sickle{5}~	&	\attPhys	&	1d12~Piercing	&	\parbox[t]{\l cm}{Requires mount, reach 2m}\\ 
\bf ~~~~~Pike	&	\galleon{1}~\sickle{10}~	&	\attPhys	&	1d10~Piercing	&	\parbox[t]{\l cm}{Two-handed, reach 2m}\\ 
{ \it Exotic Weapons} & & & & \\ 
\bf ~~~~~Scythe	&	\sickle{10}~	&	Versatile	&	1d4~Slashing	&	\parbox[t]{\l cm}{}\\ 
\bf ~~~~~Trident	&	\galleon{1}~\sickle{10}~	&	Versatile	&	1d8~Piercing	&	\parbox[t]{\l cm}{}\\ 
\bf ~~~~~Whip	&	\sickle{10}~	&	\attFin	&	1d4~Slashing	&	\parbox[t]{\l cm}{Reach 5m}\\ 
\bf ~~~~~Chakram	&	\galleon{2}~	&	\attFin	&	2d4~Slashing	&	\parbox[t]{\l cm}{Max range 200m.}\\ 
\bf ~~~~~Fan	&	\sickle{8}~	&	\attFin	&	1d6~Slashing	&	\parbox[t]{\l cm}{}\\ 
\bf ~~~~~Net	&	\sickle{8}~	&	Versatile	&	~	&	\parbox[t]{\l cm}{Applies {\it Incapacitated} status on a failed DV10 Strength Resist check. Can be thrown: range 5m.}\\ 
{ \it Simple Ranged Weapons} & & & & \\ 
\bf ~~~~~Blowdart	&	\knut{5}	&	\attFin	&	1d4~Poison	&	\parbox[t]{\l cm}{Range: 10m. Ammuniion: Darts}\\ 
\bf ~~~~~Sling	&	\sickle{2}~	&	\attFin	&	1d4~Bludgeoning	&	\parbox[t]{\l cm}{Max range: 50m (rocks), 100m (lead shot). Ammunition: lead shot, or improvised.}\\ 
{ \it Ranged Weapons} & & & & \\ 
\bf ~~~~~Crossbow	&	\galleon{4}~	&	\attFin	&	1d12~Piercing	&	\parbox[t]{\l cm}{Max range 20m. Ammunition: Bolts. Reload time:1 turn.}\\ 
\bf ~~~~~Longbow	&	\galleon{2}~	&	Versatile	&	2d6~Piercing	&	\parbox[t]{\l cm}{Max range: 150m. Use a \attFinShort{} check to aim, but \attPhysShort{} for damage check. Ammunition: Arrows.}\\ 
\bf ~~~~~Shortbow	&	\galleon{1}~	&	\attFin	&	1d6~Piercing	&	\parbox[t]{\l cm}{Max range 30m, Ammunition: Arrows.}\\ 
{ \it Firearms Weapons} & & & & \\ 
\bf ~~~~~Pistol	&	\galleon{8}~	&	\attFin	&	2d12~Piercing	&	\parbox[t]{\l cm}{Max range: 30m (accurate). Ammunition: Bullets. Cartridge: 8, reload time: 1 turn.}\\ 
\bf ~~~~~Rifle	&	\galleon{12}~	&	\attFin	&	5d6~Piercing	&	\parbox[t]{\l cm}{Max range: 40m (standing), 100m (standing, 2 turn aim), 500m (prone, 3 turn aim). Ammunition: Bullets, Cartridge: 1, reload time: 1 turn.}\\ 
\bf ~~~~~Shotgun	&	\galleon{16}~	&	\attFin	&	10d4~Piercing	&	\parbox[t]{\l cm}{Max range: 10m (full damage), 1d4 removed for every subsequent metre. Ammunition: Bullets, Cartridge: 2, Reload time: 2 turns.}\\ 
\hline\end{rndtable} %%WeaponsEnd
\end{center}
\normalsize
\twocolumn
