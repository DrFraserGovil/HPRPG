

\chapter{Weapons}\label{S:Weapons}
 
\key{Weapons} are a the class of \imp{Items} which are used specifically for the act of dishing out harm and dealing damage to your foes. The mechanics of how to perform attacks is detailed on page \pageref{S:Attacks}. Each weapon has a number of properties which detail how it is used, any additional abilities, limitations or benefits that the weapon confers, as detailed below. 

This list is not meant to be exhaustive, and you may, with the assistance of the \imp{GM} introduce new weapons as you feel appropriate - equally you may wish to reflavour a weapon slightly to better fit in with the aesthetics of your character, without otherwise altering the mechanics. 

\def\brawl{\faHandRockO}
\def\skirmish{\faShield}
\def\marksman{\faBullseye}

\newcommand\weaponTable[2]
{
	\begin{strip}
	\section{#1}
	
	{
	\small
	\begin{center}
		\begin{rndtable}{l c c c c l}
			\bf Name	&	\bf Range 	&	\bf Proficiency &	\bf DV	&	\bf Damage	& \bf Additional Properties
			\\
			#2
		\end{rndtable}
	
	\end{center}
	
	}
	\end{strip}
}
\newcommand\weapon[8]
{
	\imp{#1}	&	#2	&	 \ratingB{#4} $\left( \text{#3}\right)$ & #5 & #6 \imp{#7} & \parbox[t]{4 cm}{\raggedright \imp{#8}} \\

}


\def\meleeWeapon
{
	\weapon{No Weapon}{Normal}{\brawl}{1}{5}{0}{Bashing}{}
	\weapon{Staff}{Extended}{\skirmish}{1}{6}{1}{Bashing}{Two\minus{}handed}
	\weapon{Club}{Normal}{\brawl}{2}{5}{1}{Bashing}{}
	\weapon{Knuckleduster}{Normal}{\brawl}{2}{5}{1}{Bashing}{Concealed\comma{} Poisoned}
	\weapon{Axe}{Normal}{\skirmish}{2}{6}{2}{Cutting}{Thrown}
	\weapon{Knife}{Normal}{\skirmish}{2}{6}{1}{Stabbing}{Concealed\comma{} Poisoned\comma{} Thrown}
	\weapon{Spear}{Extended}{\skirmish}{2}{6}{1}{Stabbing}{Thrown}
	\weapon{Whip}{Extended}{\skirmish}{3}{5}{1}{Cutting}{Grapple}
	\weapon{Sword}{Normal}{\skirmish}{3}{6}{2}{Stabbing/Cutting}{}
	\weapon{Greataxe}{Extended}{\skirmish}{3}{7}{3}{Cutting}{Two\minus{}handed}
	\weapon{Warhammer}{Extended}{\skirmish}{3}{7}{3}{Bashing}{Two\minus{}handed}
	\weapon{Kama}{Normal}{\skirmish}{4}{6}{2}{Cutting}{Paired}
	\weapon{Polearm}{Long}{\skirmish}{4}{6}{2}{Any Physical}{Two\minus{}handed\comma{} cumbersome}
	\weapon{Greatsword}{Extended}{\skirmish}{4}{7}{3}{Cutting}{Two\minus{}handed}
}
\def\rangedWeapon
{
	\weapon{Improvised}{Short}{\marksman}{1}{5}{0}{Bashing}{}
	\weapon{Blowdart}{Short}{\marksman}{1}{6}{1}{Stabbing}{Concealed\comma{} Poisoned}
	\weapon{Crossbow}{Normal}{\marksman}{2}{5}{2}{Stabbing}{Reload (1)}
	\weapon{Shortbow}{Extended}{\marksman}{2}{6}{1}{Stabbing}{}
	\weapon{Shotgun}{Short}{\marksman}{3}{6}{4}{Stabbing}{Reload (2)\comma{} Two\minus{}handed}
	\weapon{Shotgun (Sawn\minus{}off)}{Short}{\marksman}{3}{6}{2}{Stabbing}{Reload (2)}
	\weapon{Longbow}{Long}{\marksman}{3}{7}{2}{Stabbing}{}
	\weapon{Pistol}{Normal}{\marksman}{3}{7}{2}{Stabbing}{Reload (8)\comma{} Concealed}
	\weapon{Rifle}{Long}{\marksman}{3}{7}{3}{Stabbing}{Reload (1)\comma{} Two\minus{}handed}
	\weapon{Semi\minus{}Auto Rifle}{Long}{\marksman}{3}{7}{3}{Stabbing}{Reload (10)\comma{} Two\minus{}handed}
	\weapon{Machine Gun}{Normal}{\marksman}{4}{8}{4}{Stabbing}{Reload (5)\comma{} Burst Fire\comma{} Two\minus{}handed}
	\weapon{Sniper Rifle}{Extreme}{\marksman}{5}{10}{8}{Stabbing}{Reload (1)\comma{} Cumbersome\comma{} Two\minus{}handed}
}



\weaponTable{Melee Weapons}
{
	\meleeWeapon
}



\weaponTable{Ranged Weapons}
{
	\rangedWeapon
}


\clearpage
\section{Weapon Range}

Each weapon has an associated \imp{Range} over which it can be used, usually described as either \imp{Short}, \imp{Normal}, \imp{Extended}, \imp{Long} or \imp{Extreme}. For melee and ranged weapons, these translate into differing physical distances:

\rangetable{}

You may never use a melee weapon against a target further away than their stated distance, without some kind of magical intervention, or simply throwing the thing at your foe. For ranged weapons, you {\it may} push yourself to hit a target up to an additional 50\% further away, but the necessary reduction in accuracy and decreased velocity at this range increases the DV of the attack by +2. 

Generally, it is possible to attack a target which is closer than the stated range without penalty, unless the weapon has the \imp{Cumbersome} property.
\section{Proficiency}

Some weapons are much harder to use than others, and so require a certain degree of skill before you may use them to their maximum effictiveness. This is codified through the \key{Proficiency} attribute of a weapon. 

Each weapon has an associated \imp{Ability} (usually either \imp{Brawl} (\brawl), \imp{Skirmish} (\skirmish) and \imp{Marksmanship} (\marksman)), denoted by a the relevant symbol, as well as a \imp{Rating} in that \imp{Ability}. 

If you meet or exceed the stated rating in that skill, then you are considered \imp{Proficient} in that weapon, and may use it properly. 

If you do not meet the requirements, you are still able to use the weapon, but your inexperience gives a heavy penalty: the DV associated with the weapon is increased by 2. 


\section{Weapon Difficulties}

Each weapon has an associated \key{Difficulty Value}. When you perform an \imp{Attack Roll} using this weapon this is the value you must use to determine your degree of success. 

Some effects, abilities or magical spells might alter this value from its default value, and as stated above, using a weapon you are not proficient in increases the DV by 2. 

\section{Weapon Damage}

The \key{Damage} statistic is the {\it base} amount of damage dealt by the weapon. For every additional success above the first, the attack deals an additional level of damage above the base value.

An attack with a \imp{revolver} (base damage 2) which achieves 3 successes therefore deals \imp{Level 4 Harm}.   

The \imp{Damage} dealt by the weapon is of the specified type - if more than one damage type is stated, then you may choose to deal one or the other depending on how your attack was carried out. For example, a \imp{Sword} can be used to slash across an opponentss exposed skin (dealing \imp{Cutting} damage), or thrust through a joint in their armour (dealing \imp{Stabbing} damage). 

\section{Other Weapon Properties}

In addition to simply doling out damage, \imp{Weapons} have a variety of other properties which distinguishes them from each other. Of course, it is not possible to model this completely within a game without it beocme horrifyingly fiddly. The additional properties included in the final column of each weapon's entry is intended to provide some limit modelling of these differences. 

\begin{itemize}[leftmargin = 4 pt]
	\keyItem{Burst Attack}{A \imp{Burst Attack} weapon has the ability to `attack' multiple times at once, at the cost of discharging more ammunition/power from the weapon. For each additional `burst' applied, deduct an additional ammunition/usage from the tally, but reduce the DV of the attack by 1.}
	\keyItem{Concealed}{A weapon with the \imp{Concealed} property can easily be hidden on your person - in a pocket or inside a jacket - without it being suspicious}
	\keyItem{Cumbersome}{The \imp{Cumbersome} property is applied to weapons with a longer than normal range, but which cannot be used at a lesser range. A weapon with this property cannot be used against targets closer than the next range increment down. For example, a \imp{Long} range, \imp{Cumbersome} weapon cannot be used against a target closer than \imp{Extended} range.}
	\keyItem{Grapple}{A weapon with the \imp{Grapple} property can be used to grab onto people or things, to pull them closer or to immobilise foes.}
	\keyItem{Paired}{A \imp{paired} weapon is more effective when used in unison with another of the same type. If you have one of this weapon in each hand, all attacks deal an additional level of damage.}  
	\keyItem{Poisoned}{A \imp{Poisoned} weapon works especially well with toxins, potions and other \imp{alchemical} concoctions. Opponents suffer a 1d penalty to resist poisons delivered via these weapons.}
	\keyItem{Reload}{A weapon with the \imp{Reload} property has only a finite amount of ammunition before you must manually insert a new magazine into the weapon. The number of uses before a reload action is required is stated in the weapon description. }
	\keyItem{Thrown}{A \imp{Thrown} tag is applied to a melee weapon which can also double as a ranged weapon. When used in this fashion, they are assumed to have a \imp{Short} range, but otherwise use the same statistics.}
	\keyItem{Two-Handed}{This simple property means that both hands must be used in order to operate the weapon. These weapons can usually be carried in only one hand, but any other held items must be stowed or dropped before an attack can be performed.}
	
\end{itemize}

Note that, despite the inclusion of the \imp{Reloading} property for ranged weapons, it is \imp{not} recommended that you keep track of the total amount of ammunition in your possession, as this violates the principle of abstraction discussed on page \pageref{S:ItemAbstraction}.

\subsection{Weapon Cost}

Note that, unlike many other items in this chapter, the weapons presented here are not given an associated Rarity and/or price. This is because the availability of certain weapon technology relies incredibly heavily on the kind of game your GM is running. 

In a true-realistic model of the modern world, you would probably have to spend more to buy a weapon-quality sword than you would a simple rifle, whilst in a game set during the Witch Trials, any firearms weapons will simply not be available to the average shopper. It is encouraged that the DM therefore modifies the prices heavily in accordance with the rarity they wish each item to have. 

As a rough baseline, for a game set C. 21st century, a weapon's cost in \imp{Galleons} is equal to its base weapon damage. 

