\documentclass[../CoreRulebook.tex]{subfile}

\chapter{Performing Checks}


In general, when you want to perform an action, simply tell the GM what you wish to do. 

If it is a simple action – for example, “I walk to the shop”, then the action is completed with no further involvement. More complex actions may require a ‘check’ to be performed, to determine their success: inform the GM of what you want to do, and the GM will tell you what check to perform. 

Usually, every action you wish to perform falls into the domain of one of your 8 character attributes (where there is ambiguity, the GM's word is final). The a check to jump over a ravine, for example, would be an Athletics check, whilst a check to remember the ingredients of a potion would be an Intelligence check. Having a higher attribute score in the relevant field will make your check more likely to succeed. 

Below is a rough guide as to the kind of actions that each attribute governs: 

\begin{tabular}{l p{7cm} }
    {\bf Athletics}: & Governs physical feats: running, jumping, and swimming, as well as melee attacks
    \\~\\ {\bf Finesse}: 	& Actions that require precision and control, such as aiming, stealth and pickpocketing.
   \\~\\   {\bf Spirit}: &	Feats that require concentration, force of will and the ability to endure hardship.
    \\~\\  {\bf Charisma}:& 	Convincing and persuading others, haggling and distractions. 
    \\~\\  {\bf Intelligence}: &	Research, knowledge and memory actions. 
    \\~\\  {\bf Perception}:&	Noticing things: either as a general spatial awareness, or differences in an individual
    \\~\\  {\bf Power}:& 	Used for feats of great magical strength 
    \\~\\  {\bf Evil}: &	Used to commit atrocities and spread chaos. 
\end{tabular}

On top of these general actions, casting magic spells requires abilities in each of these fields. Each spell `discipline' is assocoiated with an attribute. Therefore casting Hexes, for example, requires a SPR check, but casting an illusion requires you to be convincing, using a CHR check. 

As always, the GM has the authority to override these general guidelines, if it is suitable to do so. For more detail on how to calculate a check, see page \pageref{S:Checks}.

\newpage

\section{Success \& Failure}

After having performed a check (the details of which are discussed below), you will end up with a single number -- a result of the (usually) d20 dice, modifiers, proficiencies and any other bonuses. It is now time to `resolve' the check, and decide if the action was successful or not. 

Generally speaking, the following table gives a rough outline of the check required to pass actions of a certain difficulty:
\begin{center}
\begin{rndtable}{|c p{3cm} c|}
\hline
Task Difficulty & 	Description & Required Check	
\\ \hline 
Very Easy & \parbox[t]{3cm}{\raggedright An everyday task that anyone could be expected to carry out first time.}	&	5
\\
Easy & \parbox[t]{3cm}{\raggedright A simple task that has only a small chance of failure.}& 10
\\
Moderate & \parbox[t]{3cm}{\raggedright A task that a normal person might require a few tries to get right} & 15
\\
Hard & \parbox[t]{3cm}{\raggedright A task that a normal person could not carry out without specialist training} &20
\\
Very Hard & \parbox[t]{3cm}{\raggedright A task that even a trained expert might struggle to complete. } & 25
\\
Legendary & \parbox[t]{3cm}{\raggedright A task that perhaps one person alive could actually complete.}	& 30
\\ \hline
\end{rndtable}
\end{center}

It is up to the GM to decide the difficulty of the task (they do not need to reveal this to you), and they may of course deviate from this table. 

If the check succeeds, then the action goes ahead -- if not, then the action fails, and there may be negative consequences, especially if the dice roll was a 1 (`natural failure'), or if your modifiers are such that the check was negative (haha). If this happens, the GM may come up with a suitable back-firing, or you may alert enemies to your presence, and so on. 

On the other hand, many GM's accept that a check which rolls a 20 on the d20 (`nat 20'), if the action succeeds, is said to be a `critical success', and may have positive effects beyond the intended. If the check was an attack, for instance, it may be considered a critical strike (see below), and do extra damage. 


\newpage

\section{Dice}

In general, you will use the 20 sided dice (d20) as the basis of all checks. You roll this dice once, and use the first value. 

There are two notable exceptions to this general rule: {\bf spellcasting checks}(used to determine if a witch or wizard managed to cast a given spell) and {\bf damage checks}(used to determine how much damage a given attack or event inflicted) often use dice other than a d20. Read more about this in the relevant chapters.  

If the value of a dice is roll indeterminate, or the dice falls off the table, it is usually best to perform the check again: though you may form your own conventions as to the etiquette in such situations.

\section{Modifiers}

After having performed the dice roll, you then modify the value by the various bonuses that your character has. 

The primary way to do this is through using the {\it attribute modifiers}. These are 8 values associated with each of your 8 attribute scores. When asked to perform a check associated with, for example, the Athletics attribute, you add your Athletics modifier on to the d20 check. 

The modifier is calculated using the following formula:
$$ \text{attribute modifier} = \frac{\text{attribute value} - 10}{2} \text{ (rounded down)} $$
The result of this formula for the usual range of values is listed below: 
 
\begin{center}
\begin{rndtable}{|c c p{0.1cm} c c|}
\hline \bf Value 	&\bf 	Modifier  & ~ & \bf Value & \bf Modifier
\\ \hline
0-1	&	-5		&	~	&	10-11	&	+ 0
\\
2-3	&	-4	&	~	&	12-13	&	+ 1
\\	
4-5	& 	-3	&	~	&	14-15	&	+ 2
\\
6-7	&	-2		&	~	&	16-17&		+ 3
\\
8-9	&	-1		&	~	&	18-19	&	+4
\\
\hline
\end{rndtable}
\end{center}

Given that an attribute value of 10 is considered `average', the attribute modifier is a way of quantifying ``how much better than average are you at this specific skill?"

For example, a Level 5 Auror wants to try and convince a ne'er-do-well to reveal the location of their boss. The GM directs her to perform a Charisma check to convince the target. The auror has a charisma value of 15, which corresponds to a +2 bonus. After rolling a 12, the total value for the check is 14, which the GM reveals was insufficient to persuade the target. 

\newpage
\section{Expertise \& Proficiencies}

\subsection{Expertise Bonus}
As a character grows and learns, they find certain skills that they excel in. The base level of expertise in the Chief Warlock of the Wizengamot is significantly larger than that of a first year Hogwarts student. When faced with a check in a field in which you are an expert, you are significantly more likely to succeed. 

This is quantified through your {\it Expertise Bonus}.  This is a single number that you may add to checks in areas which you are considered {\it proficient} in. For most characters, the proficiency is calculated from your total character level in the following fashion:
$$ \text{Expertise bonus} = \frac{\text{Character Level}}{4} + 2 ~~~\text{(rounded down)}$$
Some Archetypes, however, grant extra expertise bonus, and as such, deviate from this formua. The table representing each class-overview gives the Expertise bonus that class has at a given level. 

\subsection{Attribute Proficiencies}
The 8 attributes are further subdivided into 19 individual proficiencies. A characer usually gains proficiencies through advancement in their Archetype. These proficiencies goven subdomains within the larger attribute. For example, a powerlifter and a sprinter are both very Athletic individuals, and would have a high ATH score. However, they are athleticin very different ways: the powerlifter has the {\it Strength} proficiency and the sprinter the {\it Speed} one.

The full list of proficiencies is as follows:
%\setlength{\columnsep}{-1cm}
\begin{multicols}{2}
\begin{itemize}[leftmargin = 0.2cm]
{\raggedright  \footnotesize
	\item ATH
		\begin{itemize}[leftmargin = 0.2cm]
			\item {\bf Acrobatics}
				   {\footnotesize \it Used to leap, tumble and flip}
			\item {\bf  Speed}
				  {\footnotesize \it Used in movement and dodge checks}
			\item {\bf Strength}
				 {\footnotesize \it Used in weapon checks, and when brute force is required }
		\end{itemize}
	\item FIN
		\begin{itemize}[leftmargin = 0.2cm]
			\item {\bf  Dexterity}
			 {\footnotesize \it Checks that require a steady hand, such as pickpocketing, or sleight of hand}
			\item {\bf Stealth}
			{\footnotesize \it Stay hidden from your enemies}
			
			\item {\bf Precision}
			 {\footnotesize \it Used for long-distance shooting, or for ultra-precise spellcasting}
		\end{itemize}
	\item SPR
	\begin{itemize}[leftmargin = 0.2cm]
		\item {\bf Endurance}
		 {\footnotesize \it Checks that require resisting the effects of magic, tiredness or debilitating agony}
		\item {\bf Willpower }
		 {\footnotesize \it Checks that require dominating someone else's spirit}
	\end{itemize}
	\item CHR
		\begin{itemize}[leftmargin = 0.2cm]
			\item {\bf Deception}
			 {\footnotesize \it Lie, cheat and mislead other characters}
			\item {\bf Performance}
			 {\footnotesize \it Play music, sing, dance and otherwise entertain the masses.}
			\item {\bf Persuasion}
			 {\footnotesize \it Convince others to willingly go along with your ideas}
		\end{itemize}
	\item INT
		\begin{itemize}[leftmargin = 0.2cm]
			\item {\bf Research}
			 {\footnotesize \it Learn new information from books and other resources}
			\item {\bf Arcane knowledge}
			 {\footnotesize \it Identify magical spells, items and beings}
			\item {\bf History }
			 {\footnotesize \it Recall events from the past, and recognise important figures and items from those events}
			\item {\bf Flora \& Fauna}
			 {\footnotesize \it Identify living beingsof all shapes and sizes, and recall their properties}
		\end{itemize}
	\item PER
		\begin{itemize}[leftmargin = 0.2cm]
			\item {\bf Compassion}
			 {\footnotesize \it Empathise with those around you, and understand their feelings}
			\item {\bf Observation}
			 {\footnotesize \it Use your spatial awareness to spot dangers and threats}
		\end{itemize}
	\item EVL
		\begin{itemize}[leftmargin = 0.2cm]
			\item {\bf Chaos}
			 {\footnotesize \it Gain a bonus when spreading wanton destruction and chaos}
			\item {\bf Intimidation}
			{\footnotesize \it Scare others, and make them more likely to give you want you want}
		\end{itemize}
		}
\end{itemize}
\end{multicols}
\setlength{\columnsep}{0.7cm}


If the GM decides that an action falls under one of these categories, they may ask for, rather than a simple ``ATH check'', an ``ATH (strength) check'' in order for a character to lift a log, or an ``ATH (speed) check'' in order for the character to run away from danger. 

If your character is considered proficient in this field, then in addition to the ATH modifier, one may also add your Expertise bonus in to the check calculation to represent your character's significantly above-average skills in this field. If you do not have proficiency in the given field, you are not `punished' and must simply use your normal modifier without the additional bonus. 

In the previous example, our Level 5 auror (with 15 Charisma) rolled a 12 to try and convince a perp to talk, elevated to 14 with the +2 CHR modifier. This was insufficient to persuade the target to talk. However, if she had been proficient in the {\it persuade} subdomain, she could add her Expertise bonus (+3 for a lvl 5 character), giving a total value of 17 -- surely enough to convince anyone to talk! 


\subsubsection{Unusual Uses}

Generally speaking, proficiency bonuses will always be associated with the attribute listed above -- so Speed will usually be added on to an ATH check. However, in certain circumstances it makes sense to cross the borders. For example, if you are threatening to beat someone up, you might use an ``ATH (intimidation)'' check, or a ``POW (intimidation)'' if you are threatening them with magical violence -- even though Intimidation is an EVL proficiency. 

The GM decides what is appropriate for each moment. 

You are always allowed to ask the GM if a proficiency applies to a specific check, even if the proficiency was not explicitly asked for -- but they are always within their rights to refuse!

\subsection{Other Proficiencies}

In addition to the proficiencies associated with attributes, you may also be considered proficient in the use of various classes of weapons, and special tools. There are also some proficiencies with unusual or more nebulous domains-- for example the {\it Muggle-Lover} skill grants you proficiency in muggle-related checks, and archetypes often grant proficiency in certain spell disciplines.  

As with the attribute-proficiencies, being proficient in an area means that you may add your Expertise bonus to the associated checks. 

Weapon-proficiencies explicitly allow you to add he bonus to the {\it accuracy} check, not to the damage check. Some tools also give additional abilities with proficiency in them, as stated in the item description.

\subsection{Multiple Proficiencies} 

Occasionally, you may encounter scenarios where you may apply your Expertise bonus multiple times. For example, a character with both the {\it Muggle-Lover} skill and the {\it persuasion} proficiency attempts to persuade a muggle of something. However, you may only add your Expertise bonus once per check, unless a mechanic explicitly mentions that the bonus is doubled, or halved. 

\section{Check Advantage}

If you have the status effect {\it Check Advantage}, or are otherwise granted this ability on certain checks, then you may perform checks twice -- and take the largest value. This decreases the likelihood of a negative outcome, and increases the likelihood of a positive one. 

Conversely, a {\it Check Disadvantage} requires you to perform a check twice and take the lower of the two values. 

You may sometimes have a conflict of multiple check modifiers such as this. The GM has the ability to rule that they either cancel each other out or (in rare instances) compound each other to give {\it check double-advantage} or {\it check double-disadvantage}, in which case you must perform the check three times, rather than twice. There is no check triple-(dis)advantage.


\section{Working Together}

Occasionally two or more characters might decide that, together, they have a better chance of succeeding in a given task, and can work together. A character may only help if they could perform the action themselves (so you could only help pick a lock if you also had proficiency in lockpicking tools), or if you can provide a reasonable justification for how you are helping the action succeed (an engineer attempting to fix an engine could use somebody to hand them tools, for example). 

When working together like this, the character with the highest relevant modifier performs the check with check-advantage. 

Sometimes, you might need to complete a task where the entire group needs to succeed, but the group may help each other -- for example, if the entire group needs to jump across a ravine, or if the entire group is searching for a single hidden item. The GM may decided on the most appropriate course of action, but a general first-start is to ask all members of the group to perform the check -- if at least half of the group succeed, the entire group succeeds. 


\section{Multiple Attempts}

Sometimes, after an action fails, a character may want to try again immediately. This is generally to be discouraged - it makes the game less fun if everyone is just waiting for Mike to (finally) roll a 20. 

A general rule is that you can't repeat an action until there is a material change in circumstance that might alter the outcome. This doesn't usually apply in combat as, you are sacrificing your other combat actions each turn cycle to try anew. 

Of course, the GM may make allowances for multiple attempts. This will most commonly occur if you have some finite resource that you are burning through. If you only have 3 fragile lockpicks, there's no particular harm in giving you 3 attempts at opening the door. 

If, however, a character is attempting to `spam' a check -- i.e. just keep rolling the dice until they succeed, and it makes enough narrative sense that the GM doesn't overrule it, then they instead ask you to roll a d100 on the table found on page \pageref{S:Multi}, which will determine the number of failed attempts. 

 


\chapter{Everyday Actions}

Within the framework of the game, there are broadly two classes of actions: {\it everyday} and {\it combat}. Everyday actions are things such as traveling between two cities, getting some sleep, talking to a friend, sitting in the library and so on. Combat, however, involves things trying to hurt you, and you trying to hurt them back. 

This section is concerned with the everyday, and is by no means meant to be an exhaustive list of things you may do. Instead, it merely provides some guidelines as to how to perform some common actions, and the effects that they can have. 


\section{Movement}

Out of combat, wandering around the environment is very natural -- you simply tell the GM that you want to go over there, and you do - barring unforeseen circumstances such as traps. You needn't calculate the exact time taken for each individual movement (that would get dull), but it is generally presumed to occur on the scale of seconds to a few minutes. 


However, sometimes you might wish to travel over distances which will take more than a handful of minutes. If you are travelling by foot more than 10 minutes, then you need to decide how rapidly and carefully you are moving.

\small
\begin{center}
\begin{rndtable}{|c c c m {4 cm}|}
\hline
Pace & Speed & Duration & Effect
\\
\hline 
Slow & 2km/h & 8 hours & Can remain hidden, or draw a map
\\ 
Normal & 4 km/h & 7 hours & Can draw a map
\\ 
Rapid & 6 km/h & 5 hours & -5 penalty to all checks made whilst moving. Costs 5 FP per hour.
\\ 
Breakneck & 10km/h & 1 hour & {\raggedright -10 penalty to all checks made whilst moving. Costs 2 FP per minute and 5 HP per hour.}
\\ \hline
\end{rndtable}
\end{center}

\normalsize
If you attempt to travel for longer than the `duration' of the selected pace, you risk exhausting yourself. After the first additional kilometre travelled, all members of the party must succeed a DV 10 ATH (endurance) check. This check must be repeated after every subsequent kilometre travelled, with the DV increasing by 1 each time. After failing this check, you must halt, and take an additional level of exhaustion. 

This timer resets after a rest of more than 8 hours, after which time you can take up your pace again. 

\subsection{Vehicles \& Mounts}

Of course, the discerning wizard rarely travels too far on foot - they may prefer to use a broomstick, tame and ride a griffin or simply apparate or portkey around. 

Each of these modes of transport has their own limitations, specified by the relevant item, beast or spell effects. 

\subsection{Actions while moving}

It is possible to perform other actions whilst on the move, though unless you are travelling in a luxury carriage, you may be somewhat restricted in what exactly you can achieve. 

You may make checks to navigate, to track a foe keep or to keep an eye out for enemies (these all use variations on the PER attribute), or you may leverage your knowledge of Flora \& Fauna to forage for food and water. The faster you travel, the heavier a penalty you suffer for these checks. 

Whilst travelling at a slow pace, you may make an effort to remain hidden, the rules for which are elaborated on more on page \pageref{S:Stealth}. 

If the Slow or Normal pace is used, a member of your part may elect themselves as a map-maker, if they have the {\it Observation} proficiency. Having a map makes it impossible to get lost (unless the scenery is magically altered, of course), and you can always retrace your steps. 

\subsection{Special Movement}\label{S:SpecialMovement}

Walking and running are not the only kinds of movement out there: navigating a dangerous environment often requires other ways of exploring the space. 

\subsubsection{Climbing}

Slopes between 0 and 30 degrees are considered `gentle', and you suffer no penalty for traversing them. Between  Between 30 degrees and 50 degrees a slope is considered `steep', and you must move at half speed, but can walk without aid. 

Slopes above 50 degrees are considered `sheer', and must use an explicit climbing action to navigate. Climbing requires use of both hands and feet, as well as the existence of solid hand/foot holds, and you move and one quarter your usual speed. If you wish to use an item, or perform an action whilst climbing , you must halt, perform a DV 10 ATH (strength) check to stabilise yourself, and then use one free hand. 

Trying to navigate a sheer slope without the existence of material to hold on to requires the use of specialised tools or magic, or else you will surely fall and perish. 

\subsubsection{Swimming}

When standing in water that is up to waist deep, your movement speed is reduced to one-half of its usual value, although the presence of strong currents may increase or decrease this. 

If the water is deeper than this, you must start to swim. Swimming moves at one-quarter your usual speed and costs 1FP for every 30m travelled. If you stop moving whilst in water that is deeper than your height, you must tread water to keep your head above water. This costs 1FP per minute to maintain. If you reach 0FP, your head will fall below water, and you will drown. 

If you wish to swim under water, you may do so, referring to the rules about air found on page \pageref{S:Air}. 

\subsubsection{Jumping}

To leap over a chasm, you need to ensure that both the height and length of your jump is sufficient to clear the obstacle. Every character has a number of `jump points' equal to 1 + ATH modifier. You may spend these points to achieve either vertical or horizontal distance. You gain 2 additional points by taking a run up of at least 3 metres. 

Each horizontal metre costs 1 point, whilst each vertical metre costs 3.  It is possible to do a `pure' long jump or high jump, but note that a long jump with zero height added to it will typically leave you gripping the edge of a chasm by your fingertips, so it is important not to neglect the height of your jumps. 

The above describes a `basic' jump (DV 5 ATH). You may choose to make the jump more difficult, by adding 1 additional point, at the expense of increasing the DV of the jump by 5. 

A level 6 thief has an Athletics score of 15, and wishes to leap across a chasm that is 6m wide. By taking a runup they have 5 jump points available to them. One of these is dedicated to height so that they land on their feet, leaving only 4m of horizontal distance left. A DV 5 jump will therefore be insufficient, but a DV 15 jump, if it succeeds, clear the gap. The thief therefore decides to risk everything, and go for the more difficult jump. 




\section{Resting}

You can't spend all day, everyday doing heroic deeds, lurking in the library, or performing mighty magic: sometimes, you need to get some rest. 

Resting is an important action that can only occur when not in combat. Attempts to rest during combat are highly likely to get you killed on the spot. 

When in safe territory, you may set up camp, and get a few hours shut-eye to recover from your ordeals (see the Asleep status effect for details). But be warned, the night is dark and full of terrors, and who knows what might sneak up on you whilst you are resting…

You may take rests whilst delving deep into unfriendly territory, but note that resting after every single encounter is generally frowned upon, and the GM might start throwing more and more unpleasant random encounters at you if you begin to take things to the extremes. 

You should only rest in a place where it makes sense to rest – it does not makes sense, for example, to take a quick nap in whilst delving through the dungeons of an evil warlord, even if you have cleared the immediate area of enemies. Of course, if you kill the Warlord and claim his castle as your own, then it is a different matter...

\subsection{Short Rest}

A short rest is a period of around one hour, which allows your character to steady their mind, grab a bite to eat, read a book, and perhaps tend to some minor wounds.

During a short rest, you restore your FP to its maximum value, unless you have a status effect which prevents regeneration. In addition, you recover d10 HP per hour spent resting, and some minor status effects may be allieviated. 

However, note that no amount of rest or sleep can heal broken bones, or cure a concussion: these severe status effects limit the amount of HP that can be restored, usually limiting regeneration to 50\% of max health. 

\subsection{Long Rest}

A long rest is an extended period of respite -- upwards of 8 hours. This allows your character to sleep and recover from more serious wounds. The d10 HP recovery rate per hour continues up to the 8th hour of rest, at which point your HP is considered fully restored (again: unless a major injury prevents this). 

A long rest also allows you to recover from exhaustion: 8 hours sleep allows you to remove 1 level of exhaustion. If the long rest was preceeded by a warm, substantial meal, you may remove 2 levels of exhaustion. 

\section{Social Actions}

An adventure rarely happens in isolation, and there will be many times that your group will have to interact with other people. Characters that are part of the larger world are known as Non-Player Characters (NPCs), and interacting with them will often be key. 

\subsection{Active vs. Descriptive Roleplaying}

There are two key philosophies to RPGs, especially when it comes to social interactions. In the Dungeons and Dragons parlance, they are `active' and `descriptive'. 

Descriptive roleplaying is when a player describes what their character does -- ``Gunter goes and talks to the man at the bar, and tries to convince him to help us". 

In constrast, an active roleplayer would act out the conversation -- they may put on a voice, or echo the body language of the character, so an active roleplayer might decide that Gunter has a deep voice and an Irish accent, and would say ``hey, barkeep -- have you heard any news about the griffin attacks recently?". 

Neither approach is right or wrong, or better or worse -- the aim is for you to have as much fun as possible. 

Of course, sometimes you may have to rely on descriptive roleplaying when your character is doing something that you cannot do. Your character might be thousands of times clever than you, or charismatic beyond all human reckoning. You character doesn't have to be limited by your own experiences -- if a shy player is unsure of what an extroverted, flambouyant character would do in this scenario, you may fall back on descriptive work, though your GM will should try to help you embellish.

Of course, the converse is also true, though somewhat harder: there are many things that the players know, but the characters don't - if a merchant tries to sell you a new item for twice the price its listed in this handbook, do your characters know they're being overcharged? You might immediately recognise the inscription as being in Ancient Greek, but does you INT 7 character recognise the symbols? Try not to let such metagaming influence your character's actions. 

Finding a healthy balance between these two playstyles is key to having fun in this game, and exploring your character - feel free to experiment!

\subsection{Checks}

Of course, roleplaying is not the only factor to take into account in social interactions: you will also need to use ability checks -- after Gunter tries to convince the barkeep, the GM may ask for a Charisma (persuasion) check to see how well you made your case to the him.

Keep an eye on your skill proficiencies, and let these guide your choices when interacting with an NPC, if you are especially good at lying, or particularly intimidating, you may elect to use those skills instead of a more honest approach. Of course, you must also consider that, like in real life, social interactions can often have consequences later on. 

\subsection{Attitudes}

NPCs are generally split into 5 categories, based on their attitude towards you and your group. This helps provide a first-guess of how to approach a scenario. 

A character's attitude towards you make social interaction with them much easier, in addition to the roleplaying benefits of this, you gain a numerical bonus to charisma checks towards these characters, representing their likelihood to believe and follow you.
\begin{center}
\begin{rndtable}{l p{4cm} c}
\bf Attitude &	\bf Description	&	\bf CHR bonus
\\
Ally	&	A very close friend, whose interests align with yours almost all the time. &+5
\\
Friendly	&	Someone who likes you, and is inclined to agree with you	&	+2
\\
Indifferent	&	Someone who has no feelings for you either way. A total stranger. 	& +0
\\
Unfriendly	&	A character who dislikes you, and doesn't want you around.	& -2
\\
Enemy	&	A character that truly hates you. They would disagree with you purely out of spite. 	&	-5
\end{rndtable}
\end{center} 


\section{Downtime}

In addition to performing non-combat actions in between individual conflicts, you may occasionally find yourself with a considerable amount of time to spare -- in which you can devote entire days to activities that further your character, heal them from egregious injuries, or earn some spare cash. 

Given that extended downtime will probably be taking place in population centres, you will need to find enough resources to live a normal life -- particularly food and shelter. See page \pageref{S:Shelter} for more details. 


\subsection{Working}

Perhaps one of the mos useful things you can do is try to bolster your finances with some hard work. You may find the kind of jobs available limited by the area you are in -- a tiny village isn't going to have much call for a librarian, and a a bustling city won't have much need for a thatcher. You will need to search out clients or an employer to practice your skills. 

In general, the payment one can expect to recieve varies depending on how skilled the job is you perform, though again, the region you are in might have an economic boom in one area, or a financial collapse, which alters these wages:
\begin{center}
	\begin{rndtable}{c p{3cm} c}
	Skill 	&	Examples	&	Wage (per hour)
\\
	Unskilled	&	Manual labour, farmwork &	\sickle{}5
	\\
	Moderate	&	Shopwork, guard	&	\sickle{}10
	\\
	Skilled	&	Teacher, performer, nurse	&	\galleon{}2
	\\
	Highly skilled	&	Artificier, surgeon	&	\galleon{}4
	\end{rndtable}
\end{center}

\subsection{Crafting}

Witches, wizards and many other sentient species in the world rely on the production of magic potions and enchanted items for their day-to-day life. Downtime is a perfect time to attempt to get in on this. 

Enchanting an item usually takes around 6 hours to complete, and a potion around one hour to brew. See the rules for artificing on page \pageref{S:Artificing} for more details. 

In addition, you may also manufacture or assemble non-magical items during your downtime, if you have access to the necessary raw material, tools and machinery required. A general rule is that you can only manufacture goods up to a value of \galleon 10 per day. If you wish to exceed this value, you need to spend multiple days performing the task.

\subsection{Recuperating}

Although not a substitute for seeking genuine medical attention, a long period of rest may allow you to recover from even the most serious of injuries. 

After at least 3 days of rest, you may perform a DV 15 spirit (endurance) check to end one major injury which prevents you from regaining HP. 


\subsection{Researching}

Downtime is also the perfect time to go searching for new knowledge, whether it is to find new information about mysteries that have been partially revealed to you, to find new and interesting types of magic, or to learn about weaknesses and habits of the magical and dangerous beasts that roam nearby. You may find libraries to comb through for fusty old tomes, or go out and speak to people and try to extract local knowledge from them. 

Tell the GM what information you are looking for, and the route you will take to finding it. They will determine if the information is available, and then how long you have to spend before you hit the jackpot. 

This might also include CHR (persuasion) checks, or INT (research) checks, to determine how well your character performs their research. 
 

\subsection{Training}

You might also dedicate your time to training in a new skill: learning to use new weapons, new languages, new magic, or new tools. 

Though not nearly as useful an experience as real-life experience, this can be an important aspect of preparing yourself for the trials and tribulations you will face. 

In order to train, you will need to find an experienced person, willing to teach you. The classes cost around \galleon{}10 per day, though if the skill you are attempting to learn is particularly rare, or the teacher particularly noteworthy, the classes may cost more.  

5 weeks worth of dedicated practice (\galleon{}250) is enough to call yourself proficient in the field, and you may take up a proficiency in a tool, weapon, or language of your choice. Note that training with a weapon gives you proficiency {\it only} in that weapon, not in the entire class of weapons associated with that weapon, to learn an entire class of weapons would take 10 weeks worth of dedicated practice. 

If you find a magic teacher, they may help you memorise new spells without risking yourself. Spending two days is enough to memorise a new spell, though a teacher can only help you with spells they themselves have memorised. 

\chapter{Combat}

\section{The Combat Cycle}
Unlike most RPGs, which tend to use a turn-based system for combat, this game uses a simultaneous combat system. The reason for this is that whilst the turn-based combat fits in with how we play games (I have my turn, you have yours, etc.), it is not entirely realistic: in a fight, you don't wait patiently for everyone else to complete attacking you before finally returning fire: everybody is completing actions at once. 

After combat is initiated, a series of turn cycles occur. Each turn cycle allows every character in combat one major action, such as: a movement, casting a spell, or using an item. Before the turn is activated, there is a period of time (to be decided by your GM), during which you must decide on what you will do. Players may talk to each other during this time, but do be aware that discussing your tactics in front of the GM may give the game away, you wouldn't start shouting your plan out whilst fighting the enemy now, would you? 

After this time is up, each player writes down their action on a scrap of paper (to prevent last minute changes of heart), and then all players (including the GM) reveal their action simultaneously. 

Since all actions are considered to be simultaneous, the order in which the actions are resolved does not usually matter, recall that spells have a finite travel time, so it is entirely feasible for two players to stun each other simultaneously and it does not matter {\apos}who cast first{\apos}.

It might, of course, still be possible for actions to come into conflict with each other: if two characters attempt to occupy the same space, for example. It is up to the GM's discretion how to deal with edge cases like this - for the example given, it is recommended that this be treated as a `body slam', and both characters should recoil and take some damage. 

There might also be cases where two spells are cast simultaneously where the ordering does actually matter: for example, if you heal someone at the same time that someone casts a damaging spell that would take them below 50\% health, incurring the ``major injury{\apos \apos} status. If the healing action occurs first, then they are not taken below 50\% health, but if the damage action occurs first, then they do fall below 50\%, even if they are then brought back up over that threshold. The final health that the character ends up on might be the same, but the ordering of actions effects whether they have the {\it major injury} status at the end of the turn. 

In cases such as this it is useful to remember that it is the {\it casting} of the spell that is simultaneous: so the ordering in which the spell effects should take place can be inferred from the distance between the caster and the target. The issue above is resolved simply by looking at whoever is closest to the target. 
 
 \newpage
 
 \section{Major Actions}
 
 Major actions take an entire turn to complete, and as such are considered the main way to engage in combat. 
 
 \subsection{Attacking}
 
 Casting a spell, swinging a sword, or loosing an arrow takes (usually) a full turn to complete, and so you may decide to use your entire turn to cast a spell. Some skills and archetype abilities allow you to perform multiple such actions as part of a single major action. 
 
 Generally, to perform an attack, you first perform an accuracy check (see page \pageref{S:Accuracy}) against your targets {\it block} or {\it dodge} values, to determine if the attack hits. You must then perform the damage check, to determine the amount of damage your check does. A spell-attack has the additional step of requiringa `casting check', to determine if you can successfuly cast the spell. 
 
 \subsection{Movement}
 
 When used as a major action, movement allows you to move up to a distance given by: 
\small
$$ \text{metres travelled} = \text{Base Speed } + \frac{\text{ATH modifier + Speed Proficiency }}{2}  $$  
\normalsize
(See below for more)

All other non-standard movement (i.e. climbing, crawling etc, swimming) must be performed as a full major action. 

 \subsection{Using Items (sometimes)}
 
Obviously, some `uses' of items include using swords, wands and ranged weapons, which have already been covered by `attacking'. 

However, sometimes you might want to use an action to get something big done, outside of hitting somebody. Using a crowbar to pry open a door, changing your weapon, finding the right page of a book -- all of these take enough time to be considered major actions. 

Some uses might take multiple turns -- for instance, climbing into a full suit of armour takes more than 3 seconds to complete, and will therefore require multiple, consecutive major actions to complete. 

\subsection{Trading Items}

If two characters are standing within touching distance, they may trade items between them. 

Giving items to other people takes the major actions of both the giver and the receiver. 

\newpage
\section{Minor Actions}
You may perform two minor actions in place of a single major action, all minor movement actions occur first, but otherwise you may choose the order in which the actions are completed. Some important minor actions are listed. 

\subsection{Communicate}

Communicating vital information - such as the location of a hidden enemy or trap - to your comrades takes a minor action. Note that it is assumed that the enemy can hear you, unless you make an effort to not be understood. 

In the example of alerting a comrade to a hidden foe, the hidden enemy will hear your warning and will know it has been discovered, it may adjust its actions accordingly. 

\subsection{Using Items (sometimes)}

Generally speaking, using an item is considered a minor action. Consuming a potion, checking a rememberall, removing an item from your bag and so on would be considered `minor actions'. Anything that can be completed in around 1 second would fall into this category. 

\subsection{{\it Evade} or {\it Brace} }
 
 You may also choose to ready yourself against incoming attacks, by bolstering your ability to either {\it Dodge} or {\it Block}. This gives you a better chance of negating incoming effects.  

See page \pageref{S:Accuracy} for more details on this mechanic. 


\subsection{Quickspells}

A quickspell is a spell that is cast as a minor action. 

Spells require a clarity of focus, so casting whilst moving, or otherwise in a hurry is generally a bad idea, if you want it to actually work. However, if you are very comfortable with spell (or a very powerful spellcaster), then you may be able get away with it. 

When performing a quickspell, you {\it must} take check-disadvantage on the accuracy check, unless you have a skill which explicitly mentions {\it Quickspells}. 

~

\section{Conditional Actions}

The use of the simultaneous combat system raises some interesting opportunities with conditional actions, which are actions that depend on the actions that another character takes.

The actual action, as well as the condition, needs to be declared during the normal turn cycle -- but the action itself is not triggered until all other actions had been triggered. 

For example, it could be that you declare as your action \textit{if the troll attacks player A, then I cast a healing spell on player A}. You could also attempt to prevent the damage from being taken in the first place, by declaring \textit{if the troll attacks player A, then I cast the knockback charm on the troll}. The GM may ask for a check to determine if you are close enough and have fast enough reactions for your spell to interrupt the action, but if you pass this, then you may be able to save your friend. Please see below for more counterspell options.

You are only allowed a single conditional clause in your declaration, and if that conditional does not come to pass, then your character does not do anything: there is no \verb|if-then-else| in this game!

If a seemingly unbreakable condition-chain arises (i.e. player A says he will perform X if player B does Y, but player B says he will only perform Y if player A does X), it is up to the GM to resolve the conditionals -- in such cases the answer is usually \textit{nothing happens}, but there may be examples where the GM feels it is more appropriate that the action-chain is triggered. 


\section{Movement}

Moving is a very common action to take during combat, to avoid the enemy's attacks, or to maneouvre yourself to enable an attack on the enemy. Movement can be considered a major action, a minor action, or indeed, neither. To aid this distinction, movement is broken down into three types: minor movements, transport movements and considered movements. 




\subsection{Minor Movements}

A { minor movement} includes things such as turning to face an enemy, or taking a step out from behind cover. These actions do not constitute the entirety of a turn and you may still take a major action afterwards, \textbf{however, they are considered to happen at the very beginning of a turn cycle, and you cannot return to cover after emerging from it}. If you therefore emerge from cover to attack someone, and a character successfully guessed that this would happen and sent a spell in your direction, you will not be protected until you move back into cover in the next turn cycle. 


\subsection{Transport Movement}

{\bf Transport movements} are those designed to get you from point A to point B as quickly as possible. These actions do take up your entire turn: you can do nothing else except take a transport action. The distance that you can travel in a given trasnport action is calculated from:

\small
$$ \text{metres travelled} = \text{Base Speed } + \frac{\text{ATH modifier + Speed Proficiency }}{2}  $$  
\normalsize

This distance is rounded downwards  to the nearest half-metre, unless you are wearing ``heavy armour'' (i.e. anything more heavy than usual fabrics), in which case it is rounded downwards to the nearest integer. The direction that you are travelling in \textbf{must} be declared before performing this check. You may elect to not use all of the movement that you rolled for -- i.e. if you can move 1.5m in total, you may only use 1m, if you desire. 


\subsection{Considered Movement}

A { considered movement} is one in which your character is attempting to do something else, whilst moving. It is considered a minor action -- or `half' a major action. The check is performed exactly as above, but you then simply divi the distance by two. You may use the other half to perform another minor action, such as an evasion, or to prepare a counterspell. 

~

~

\section{Accuracy \& Instincts}\label{S:Accuracy}

When an attack is launched on a being, it is necessary to determine if the attack lands home, or if the attack goes wide, the target dodges underneath the swinging blade, or catches the attack on their armour. 

\subsection{Accuracy}

The attacker -- be they an archer, a gunman, a swordsman, or a spellcaster -- quantifies their ability to successfully hit their target through an {\it accuracy check}, a standard d20 FIN (Precision) check, plus any additional relevant bonuses or penalties. 

If the check is greater than or equal to the {\it instinct value} used by the target, then the attack lands true, and the associated effects are applied. If the accuracy check fails, then the attack misses, or is successfully blocked by the target. 

If the target is not a living being (or is restriced from moving), then hitting the target is much easier, but not totally guaranteed. The `dodge' DV of a stationary object is normally equal to 5, though modifiers may be added for targets which are particularly small, or (for ranged attacks), particularly far away. 

The additional penalty for hitting small/far away targets is:

$$ P = \frac{\text{distance}}{10 \times \text{size}} ~~~~ \text{(rounded down)}$$

Therefore, hitting a 1m target at a distance of up to 10m has a DV of 5, whilst the same target 30m away has a DV of 8, and hitting a 1cm target at a distance of of 1m has a DV of 15. 

\subsection{Instincts} \label{S:AC}

Beings either block or dodge instinctively, without having to devote conscious thought to their reaction. These two actions are therefore termed {\it instincts}, and occur inbetween turn cycles. Each action has a statistic associated with it, which is used to context the accuracy check of the attacker. The value of these statistics is:

\begin{align*} 
\text{Block} &= 8 + \text{ATH (Strength) modifier} 
\\
\text{Dodge} &= 8 + \text{FIN (Speed) modifier} 
\end{align*}

By default, characters instinctively use whichever of these values is the highest:
$$ \text{IV} = \max \left( \text{Block}, \text{Dodge} \right)$$

If a character successfully dodges, the attack whizzes by their ear and misses completely. If they successfully block the attack, then they catch the spell or weapon on a piece of armour (or, with the appropriate skill, they can {\it parry} the attack). 

Various items may improve either of these statistics. A pair of running shoes, for example, makes it easier to dodge out of the way, whilst a heavy shield makes defending yourself easier. Generally speaking, items will be a compromise: wearing heavy armour will bulk up your Block statistic, but will slow you down, reducing your Dodge value. 

\subsection{{\it Brace} and {\it Evade} }

Of course, not all defense happens instinctively -- you may make a conscious decision to brace yourself against an incoming attack, or prepare to dive out of the way. Such a decision is classified as a minor action. 

You may choose to either {\it brace} or {\it evade} (i.e. you do not have to use the highest-value statistic). 

Whichever you choose, you double your Expertise bonus (if applicable) on the chosen statistic and agressors take check-disadvantage on their accuracy checks against you this turn cycle. 


\subsection{Proficiencies}

Whether you are swinging a blade, or cowering behind a shield: you must know how to use your equipment, in order to be able to apply your Expertise to it. 

You may only add your Expertise bonus to an accuracy check, if you are proficient in the weapon you are using, or to an {\it Instinct} if you are proficient in the armour (or lack thereof) you are wearing. 

\subsection{Unblockable and Unavoidable Effects}

Some effects (usually those generated by certain spells) cannot be avoided or blocked: holding up a shield against an incoming cannonball isn't going to prevent much, and trying to dodge out of the way of a tsunami is rarely effective. 

Spells denote in their description if they can be blocked or dodged. For the (rarer) instances of non-spell effects which fall into one of these categories, the GM decides if it is reasonable to dodge or block the effect. 

If the `dominant' instinct (i.e. the one with the highest value) would be ineffective against a given effect, you may use the non-dominant one. However, if the character chose to, for example, use the evade action, they may not transfer the bonus to `block' if an evasion turns out to be ineffective. 

Note that even `unblockable' effects are stopped by `impenetrable' fields and spells which are `undodgeable' treat the target as stationary, and may still miss under those rules. 


\section{Stealth and Critical Strikes} \label{S:Stealth}

Being noticed by the enemy is generally regarded as a bad thing. It therefore often pays to be sneaky, to stay hidden from the enemy. Stealth is governed by the FIN attribute, via the Stealth proficiency. 

Every character and beast has a baseline level of awareness, even when not actively searching for hidden creatures or traps. This is your {\it passive perception}. It is calculated using an `average' dice roll (for a d20, this is 10), so: 
$$\text{Passive PER} = 10 + \text{bonuses}$$

To remain hidden, your sneak-check must exceed 

Every time you wish to take an action whilst remaining hidden, you will need to perform a FIN (stealth) check against the target, with the target performing a PE check -- if the sneak check exceeds the perception check, then you remain hidden. If it fails, then the target becomes aware of you, and probably initiates combat. 

Equally, some creatures might try to sneak up on you -- but the GM can't very well ask you to perform a perception check, as you would immediately know that something was there! In order to keep the surprise, each checktype has a `passive' value, which is simply equal to the average. Hence, for a d20 check, the passive value is 10 + relevant bonuses. The GM will use this value in private to determine if beings remain hidden or not. 

The same is true for illusion spells which are cast on you without your knowledge -- a passive SPR (endurance) check is used, with the same rules as before. The GM does not need to tell you about this spell, unless you actively perform a perception check to notice something wrong with the world. 

If you willingly choose to perform a perception check, this gets a +2 bonus. In combat, this would count as your major action. 

If you initiate combat whilst undetected (or have it initiated against you by an unseen opponent), then the attacked party must continue to attempt to percieve the enemy, until they can attack them in the usual way. You may attempt to wildly attack the enemy -- throwing a fireball {\it near} them is probably going to hurt, even if you don't know exactly where they are, but this might be a waste of resources. 

\subsection{Critical Strikes}\label{S:Sneak}

If you perform an attack on someone who is not aware that you are attacking them, or if you perform an attack on someone who has their mind elsewhere, then you have an opportunity to do large amounts of damage to the unwary target. 

A sneak attack is triggered when a character attacks another when they are not expecting it -- be it attacking someone who is not even aware that you pose a danger to them, or if you have snuck up behind an enemy whilst they are attacking someone else -- if they don't see an attack coming, you get an opportunity to surprise them!

An attack of opportunity is triggered when somebody is aware that they are in combat, but is doing something that opens them up to attack. For example, if someone was in close-quarters range and they attempt to cast a spell on you, you can quickly stab them with a knife, and there is nothing they could do about it. Equally, if they attempt to cast a spell on someone else, then their attention is not on you. If you had already commited to an attack on them, then it has a chance to be much more effective.

Whichever method is triggered, the effect is the same: you roll any (even-numbered) dice. If the result is an even number, then you multiply the damage by 2. If it is odd, then you just do the normal amount of damage\footnote{This assumes that the {\it catastrophic critical} is not in use -- if it is, use the rules detailed in that skill}. 

Critical attacks (i.e. triggered by a nat20, or otherwise through a skill) are mechanically identical to an Attack of opportunity. 

If you perform a critical {\it during} a Critical attack, then you do get to use two multipliers, but they are {\it added}. For example, a critical-opportunity attack would roll two dice, and use the following table to determine the dice:

\begin{center}
\begin{rndtable}{|c c c|}
\hline
~ & odd & even
\\ \hline 
\cellcolor{\tablecolorhead }odd & 2 & 3
\\ \hline
\cellcolor{\tablecolorhead}even & 3  & 4
\\ \hline
\end{rndtable}
\end{center}
