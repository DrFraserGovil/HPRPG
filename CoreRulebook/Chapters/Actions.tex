\documentclass[../CoreRulebook.tex]{subfile}

\chapter{Performing Checks}


In general, when you want to perform an action, simply tell the GM what you wish to do. 

If it is a simple action – for example, “I walk to the shop”, then the action is completed with no further involvement. More complex actions may require a ‘check’ to be performed, to determine their success: inform the GM of what you want to do, and the GM will tell you what check to perform. 

Usually, every action you wish to perform falls into the domain of one of your 8 character attributes (where there is ambiguity, the GM's word is final). The a check to jump over a ravine, for example, would be an Fitness check, whilst a check to remember the ingredients of a potion would be an Intelligence check. Having a higher attribute score in the relevant field will make your check more likely to succeed, via the {\it Modifier} associated with that attribute.  

As always, the GM has the authority to override these general guidelines, if it is suitable to do so. For more detail on how to calculate a check, see page \pageref{S:Checks}.



\section{Dice}

For almost every action, you will use the 20 sided dice (d20) as the basis of the check. You roll this dice once, and use the first value. 

The most notable exception to this general rule is: {\bf damage checks}, which are used to determine how much damage a given attack or event inflicted. 

If the value of a dice is roll indeterminate, or the dice falls off the table, it is usually best to perform the check again: though you may form your own conventions as to the etiquette in such situations.

\section{Modifiers}

If the GM has assigned the check to one of the Attributes, you then modify the dice roll value by the various bonuses that your character has. 

The primary way to do this is through using the {\it attribute modifiers}. These are 8 values associated with each of your 8 attribute scores. When asked to perform a check associated with, for example, the Finesse attribute, you add your Finesse modifier on to the d20 check. 

The modifier is calculated using the following formula:
$$ \text{attribute modifier} = \frac{\text{attribute value} - 10}{2} \text{ (rounded down)} $$

Given that an attribute value of 10 is considered `average', the attribute modifier is a way of quantifying ``how much better than average are you at this specific skill?"

For example, a Level 5 Auror wants to try and convince a ne'er-do-well to reveal the location of their boss. The GM directs her to perform a Charisma check to convince the target. The auror has a charisma value of 15, which corresponds to a +2 bonus. After rolling a 12, the total value for the check is 14, which the GM reveals was insufficient to persuade the target. 


\begin{center}
\begin{rndtable}{|c c p{0.1cm} c c|}
\hline \bf Value 	&\bf 	Modifier  & ~ & \bf Value & \bf Modifier
\\ \hline
0-1	&	-5		&	~	&	10-11	&	+ 0
\\
2-3	&	-4	&	~	&	12-13	&	+ 1
\\	
4-5	& 	-3	&	~	&	14-15	&	+ 2
\\
6-7	&	-2		&	~	&	16-17&		+ 3
\\
8-9	&	-1		&	~	&	18-19	&	+4
\\
\hline
\end{rndtable}
\end{center}



\section{Expertise \& Proficiencies}

\subsection{Expertise Bonus}
As a character grows and learns, they find certain skills that they excel in. The base level of expertise possessed by the Chief Warlock of the Wizengamot is significantly larger than that of a first year Hogwarts student, even on tasks they have never faced before. When faced with a check in a field in which you are an expert, you are significantly more likely to succeed. 

This is quantified through your {\it Expertise Bonus}.  This is a single number that you may add to checks in areas which you are considered {\it proficient} in. For most characters, the proficiency is calculated from your total character level in the following fashion:
$$ \text{Expertise bonus} = \frac{\text{Character Level}}{4} + 2 ~~~\text{(rounded down)}$$
Some Archetypes, however, grant extra expertise bonus, and as such, deviate from this formula. The table representing each class-overview gives the Expertise bonus that class has at a given level. 


\subsection{Proficiencies}

There are many areas in which one can be considered {\it proficient} - including the use of wands, weapons, tools and armour. In addition to this, seven of the eight Character Attributes can be broken down into several specialised subdomains: {\bf proficiencies}. Being proficient in a domain means that, when a requested action falls into that field, you may add your proficiency bonus to the resulting check. 

The profiencies are:

\newcommand\prof[2]
{
-{\bf #1}:	&	\parbox[t]{ 6 cm}{\raggedright #2} \\	
}
\begin{tabular}{l l}
\prof{Fitness}{Speed, Strength, Vitality}
\prof{Finesse}{Acrobatics, Chicanery, Stealth}
\prof{Spirit}{Conviction, Willpower}
\prof{Charisma}{Deception, Performance, Persuasion}
\prof{Intelligence}{Arcane, History, Logic,  Nature, Research, Un-nature}
\prof{Perception}{Empathy, Investigation, Observation}
\prof{Power}{Intimidation}
\end{tabular}

Your GM may therefore ask for a {\it Stealth} check, which is to be interpreted as a Finesse check with the Expertise bonus added if you posses the Stealth proficiency. If you are not proficient in Stealth, you simply perform a base Finesse check. 

The character sheet provides slots to record your total modifier for each of the listed proficiencies, for ease of use. 

\subsubsection{Unusual Uses}

Generally speaking, the proficiencies are associated with their parent attribute - so Speed will usually be added on to a Fitness check. If you are not told otherwise, you should always assume this is the case. 

However, in certain circumstances it makes sense to cross the borders. For example, if you are attempting to intimidate someone, this is usually associated with the {\it Power} attribute, but if you are threatening them with physical violence, you might be asked for a ``Fitness (Intimidation)'' check. You might also be asked for a ``Charisma (Intimidation)'' check if you are are bluffing and pretending to be more powerful than you are. 

In this case, you use the modifier of the new parent, and add the proficiency bonus if applicable. 

You are always allowed to ask the GM if a proficiency applies to a specific check, even if the proficiency was not explicitly asked for -- but they are always within their rights to refuse!

\subsection{Other Proficiencies}

In addition to the proficiencies associated with attributes, you may also be considered proficient in the use of various classes of weapons, and special tools. There are also some proficiencies with unusual or more nebulous domains-- for example the {\it Muggle-Lover} skill grants you proficiency in muggle-related checks, and archetypes often grant proficiency in certain spell disciplines.  

As with the attribute-proficiencies, being proficient in an area means that you may add your Expertise bonus to the associated checks. 

Weapon-proficiencies explicitly allow you to add the bonus to the {\it accuracy} check, not to the damage check. Some tools also give additional abilities with proficiency in them, as stated in the item description.

\subsection{Multiple Proficiencies} 

Occasionally, you may encounter scenarios where you may apply your Expertise bonus multiple times. For example, a character with both the {\it Muggle-Lover} skill and the {\it persuasion} proficiency attempts to persuade a muggle of something. However, you may only add your Expertise bonus once per check, unless a mechanic explicitly mentions that the bonus is doubled, or halved. 


\section{Success \& Failure}

After the GM has decided which ability is relevant to the task a character is trying to perform, an ability check is made. The result a single number -- the result of a dice roll and your  modifiers and bonuses. This value is the {\it Check Value} (CV). It is now time to `resolve' the check, and decide if the action was successful or not. 

The GM assigns the activity a {\it Difficulty Value} (DV). The more difficult a task is, the higher the associated DV. 

\def\w{5}
\begin{center}
\begin{rndtable}{|c p{\w cm} c|}
\hline
Task Difficulty & 	Description & DV	
\\ \hline 
Very Easy & \parbox[t]{\w cm}{\raggedright An everyday task that anyone could be expected to carry out first time.}	&	5
\\
Easy & \parbox[t]{\w cm}{\raggedright A simple task that has only a small chance of failure.}& 10
\\
Moderate & \parbox[t]{\w cm}{\raggedright A task that a normal person might require a few tries to get right} & 15
\\
Hard & \parbox[t]{\w cm}{\raggedright A task that a normal person could not carry out without specialist training} &20
\\
Very Hard & \parbox[t]{\w cm}{\raggedright A task that even a trained expert might struggle to complete. } & 25
\\
Legendary & \parbox[t]{\w cm}{\raggedright A task that perhaps one person alive could actually complete.}	& 30
\\ \hline
\end{rndtable}
\end{center}

If the CV meets, or exceeds, the assigned DV then the action is successful and the GM will describe the effects of the action. If the CV is less than the DV, the action fails. 



Many GM's accept that a check which rolls a 20 on the d20 (`nat 20'), if the action succeeds, is said to be a `critical success', and may have positive effects beyond the intended, regardless of the associated modifiers. If the check was an attack, for instance, it is considered a critical strike (page \pageref{S:Sneak}). 

\subsection{Contests}

A subset of actions are those in which the difficulty is not assigned by the GM, but by a check performed by another being. Such an action is termed a {\it Contest}. For instance, when trying to detect a being trying to stay hidden one character performs a Stealth check, whilst the other performs an Observation check. These two values are then compared directly - if the Sneak exceeds the observation, the being is hidden and vice versa. 

When the GM assigns a DV, a check which meets the DV results in a success. However, in a contest, usually only one can `win'. Therefore, {\bf the status quo is maintained on a draw}. If the stealth check equals the observation check, and the being is already hidden, then it remains unspotted. If, however, it was trying to become hidden from a being which could perceive it, then the status quo is preserved and it is not hidden. 


\section{Check Advantage}

If you have the status effect {\it Check Advantage}, or are otherwise granted this ability on certain checks, then you may perform checks twice -- and take the largest value. This decreases the likelihood of a negative outcome, and increases the likelihood of a positive one. 

Conversely, a {\it Check Disadvantage} requires you to perform a check twice and take the lower of the two values. 

Check-Advantage and Check-Disadvantage compound each other, to a limited extent. If a character already possesses check-advantage, and gets a second separate effect which also gives them check-advantage, then they are in a state of `super-advantage', in which case you roll three dice, and take the highest. Equally, two disadvantages compound into super-disadvantage. 

A disadvantage layered on an advantage cancel each other out, and a disadvantage on a super-disadvantage reduces it to normal. 

{\bf However, more than two buffs in either direction have no additional effect}. 10 disadvantages and 11 advantages are treated as 2-against-2 (i.e. a normal roll), as are 3 advantages against 10 disadvantages. 

Use the following table for reference:


\def\cc{\cellcolor{\tablecolorhead}\bf }
\begin{center}
{
\small
\renewcommand{\arraystretch}{1.4}
	\begin{rndtable}{c c c c c}
	~ & ~	&	\multicolumn{3}{c}{\bf \# Advantages}
	\\
	\cc ~	&	\cc~ & \cc 0	&\cc	1	& \cc 2+ 
	\\
	\cc~& \cc 0	&	Normal	&	Advantage	&	Super Advantage
	\\
	\cc~& \cc 1	&	Disadvantage	&	Normal	&	Advantage
	\\
	\multirow{-4}{*}{\rotatebox[origin=c]{90}{\cc \bf \# Disadvantages}} & \cc 2+	&	Super Disadvantage	&	Disadvantage	&	Normal 
	\end{rndtable}
}
\end{center}


For (dis)advantages to compound, they must arise from totally different sources - drinking two potions which both provide Advantage will not give super advantage, but being invisible {\it and} drinking a potion would. 

\section{Working Together}

Occasionally two or more characters might decide that, together, they have a better chance of succeeding in a given task, and can work together. A character may only help if they could perform the action themselves (so you could only help pick a lock if you also had proficiency in lockpicking tools), or if you can provide a reasonable justification for how you are helping the action succeed (an untrained individual could help an engineer fix an engine by passing them tools, and holding a flashlight, for example). 

When working together like this, the character with the highest relevant modifier performs the check with check-advantage. 

Sometimes, you might need to complete a task where the entire group needs to succeed, but the group may help each other -- for example, if the entire group needs to jump across a ravine, or if the entire group is searching for a single hidden item. The GM may decided on the most appropriate course of action, but a general first-start is to ask all members of the group to perform the check -- if at least half of the group succeed, the entire group succeeds. 


\section{Multiple Attempts}

Sometimes, after an action fails, a character may want to try again immediately. This is generally to be discouraged - it makes the game less fun if everyone is just waiting for Mike to (finally) roll a 20. 

A general rule is that you can't repeat an action until there is a material change in circumstance that might alter the outcome. This doesn't usually apply in combat as you are sacrificing your other combat actions each turn cycle to try anew. 

Outside of combat, however, the GM may make allowances for multiple attempts. This will most commonly occur if you have some finite resource that you are burning through. If you only have 3 fragile lockpicks, there's no particular harm in giving you 3 attempts at opening the door. 

If, however, a character is attempting to `spam' a check -- i.e. just keep rolling the dice until they succeed, and it makes enough narrative sense that the GM doesn't overrule it, then they instead ask you to roll a d100 on the table found on page \pageref{S:Multi}, which will determine the number of failed attempts. 


\section{Using Each Attribute} \label{S:Proficiencies}

Almost every task a character attempts falls into one of the 8 abilities. In this section, the kinds of actions associated with each Attribute, and the encapsulated proficiencies is elaborated on in more detail. 


\def\itdef{\renewcommand\labelitemi{-}
\itemsep-0.5em}
\newcommand\proficiency[2]
{
	\textbf{\textit{#1}}: {\raggedright #2} 
}

\subsection{Fitness}

Fitness measures your ability to exert yourself physically. 


A fitness check will be required almost every time a being attempts to do something more strenuous that break into a light jog, or lift a heavy backpack. It is used to run, jump, swim and climb, as well as wielding heavy weapons and beating down doors. 

{\it Speed}, {\it Strength} and {\it Vitality} checks generally fall under the Fitness umbrella:

\proficiency{Speed}{A Fitness (Speed) check is used in situations where you need to act and move quickly, or to exert an explosive burst of speed, such as fleeing from a ravenous beast or running down an escaping prisoner. }

\proficiency{Strength}{A Fitness (Strength) check is needed whenever you utilise the raw power of your muscles. For example: 
\begin{itemize}
\itdef
\item Attempting to break down a locked or jammed door
\item Wrestling a beast's jaws shut to prevent it from biting others
\item Move an extremely heavy object 
\item Break free of restraints
\end{itemize}
}

\proficiency{Vitality}{Your Fitness (Vitality) check measures the physical well-being and fortitude of a character. A higher value means you can stave off the effects of starvation, exhaustion and resist the effects of diseases and poisons. Vitality is mostly a passive ability, and hence will most commonly be used in the form of Resist checks to evade the harmful effects of the environment of malicious acts. }

\subsubsection{Melee Weapons}

In addition, Fitness is used as the primary attribute for most melee weapons and hand-to-hand combat. The Fitness modifier is therefore added to the Accuracy and Damage rolls for weapons such as clubs, swords and battleaxes.
\raggedbottom

\subsection{Finesse}

Finesse is the measure of a beings ability to perform acts with precision and care, and to maintain balance and poise. It also measures your ability to work with your hands - to craft intricate items, tie secure knots or steer an out of control vehicle. 

The {\it Acrobatics}, {\it Chicanery} and {\it Stealth} proficiencies measure a being's aptitude in certain types of Finesse checks. 

\proficiency{Acrobatics}{A Finesse (Acrobatics) check is used whenever a being's balance is called into question, such as maintaining balance on a rocking boat or slipping on an icy floor, as well as for more extravagant feats such as rolling, flipping, diving and somersaulting. }

\proficiency{Chicanery}{Chicanery is the trickster's and the thief's domain: a Finesse (Chicanery) check will be called for whenever you try to use duplicity, trickery, distraction or slight of hand to achieve your goal. }

\proficiency{Stealth}{A stealth check is used whenever you wish to remain hidden, and is the primary check used for the Stealth mechanic discussed on page \pageref{S:Stealth}. In addition, you may be asked for a Finesse (Stealth) check to hide an object away from prying eyes.}

\subsubsection{Ranged Weapons}

Most ranged weapons use the Finesse modifier to reflect the accuracy of the wielder. Some melee weapons which are classed as `elegant', such as rapiers, also use Finesse for their accuracy check.  In both cases, the Finesse modifier is added to the associated accuracy and damage checks.

\subsubsection{Spells}

Some spells rely on careful manipulation and high levels of precision and control: these spells belong to the {\it Kinesis} and {\it Alteration} disciplines. Spells belonging to this school use the Finesse modifier to perform Casting and Accuracy checks. 

\subsection{Spirit}

The Spirit of a character is a measure of their internal strength. 

Spirit checks are used to maintain order in your own mind, or to project that inner strength outward to dominate others. 

The {\it Conviction} and {\it Willpower} proficiencies measure your ability at certain types of Spirit checks. 

\proficiency{Conviction}{A Spirit (Conviction) check is used whenever something attempts to sway a tenet of your character - whether someone is trying to tell you that a deeply held belief is false, to persuade you that your idea is bad, or to magically influence your thoughts. Conviction measures how strongly you hold to your fundamental principles. }

\proficiency{Willpower}{A Spirit (Willpower) check is used whenever a being needs to have control over their own mind:
\begin{itemize}
\itdef
\item Enforce defences around their mind to repel intruders
\item Withstand the effects of mind-altering spell
\item Use magic which dominates the minds of others
\item Withstand terror and stand brave in the face of danger
\end{itemize}
}

\subsubsection{Spells}

Spells which rely on projecting your force of will, and an iron control of your mind use the Spirit modifier for their casting and accuracy checks. Such spells include those in the {\it Psionics}, {\it Conjuration} and the {\it Hexes} discipline.  

\subsubsection{Passive Endurance}

Your {\it Passive Endurance} is a base level of endurance that every being has when they are not even aware they are actively resisting anything. 

If an effect is inflicted on you when you are not specifically expecting it, or searching for it, then the {\it passive} score is used. This can also be used by the GM to keep the fact that an enemy is influencing your mind, for example. The passive Endurance score is calculated from the `average' dice roll, plus the usual bonuses for a Spirit (Willpower) check. 

Therefore it is calculated from a score of 10, plus the usual bonuses. If a being has advantage or disadvantage, you add or subtract 4 from the score. If you have super-advantage or disadvantage, you add or subtract a further 2. 


\subsection{Charisma}

Charisma is the social attribute - it measures a being's ability to interact with others with confidence, eloquence and panache. A high-charisma being is percieved by others as charming and friendly. 

A charisma check will be called for on almost all social interactions beyond basic introductions, services and general `how-do-you-do's. For a forthcoming individual, you may only have to ask the right questions to get the information or services you desire with no check needed, but for the more recalcitrant, you must succeed on a Charisma check to get what you want. 

The Charisma domain is divided into three proficiencies: {\it Deception}, {\it Performance}, and {\it Persuasion}. 

\proficiency{Deception}{A Charisma (Deception) check will, as the name suggests, be called for whenever you attempt to tell a convincing lie, or otherwise mislead an individual. Manipulate both your voice and your body language to give a false sense of honesty and truth to waylay the authorities, cheat an opponent out of some money, or bluff your way past a guard.}

\proficiency{Performance}{A Charisma (Performance) check is used whenever a being puts on an act to delight and impress an audience with their skills or stage presence. Perfomance is a form of {\it Deception}, with the difference usually being that the purpose is to inspire, delight or entertain, rather than mislead. } 

\proficiency{Persuasion}{A Charisma (Persuasion) check measures the ability of a being to sway others with convincing arguments, charm, and social know-how. Generally used in good faith to convince a neutral party to take a side, to persuade a guard to let you past, or to negotiate a better price for an item.}

\subsubsection{Spells}

Spells which belong to the {\it Bewitchment} discipline rely heavily on subtly altering and influencing a being's perception of reality. These spells use the Charisma modifier for their casting and accuracy checks. 

\subsection{Intelligence}

Intelligence is a being's innate mental capacity, their memory, their ability to reason and logically deduct as well as encompassing their prior education and learning.   

An intelligence check will be called for whenever a character attempts to assimilate new information, or recall information they have previously used. It may also be used to solve riddles, use logic to deduce where an item might be hidden, and so on. 

As intelligence is a wide and somewhat nebulous field, there are a number of proficiencies under this umbrella, particularly: {\it Arcane Knowledge}, {\it History}, {\it Logic}, {\it Nature}, {\it Research}, {\it Un-nature} 

\proficiency{Arcane Knowledge}{An Intelligence (Arcane Knowledge) check - often shortened to simply `Arcane' - is a measure of a being's understanding of the nature and use of magic. Used to recall or infer knowledge about spells, magical items, mystic runes and other intrinsically magical objects.}

\proficiency{History}{An Intelligence (History) check measures your ability to recall information about historical events, places and people}

\proficiency{Logic}{An Intelligence (Logic) check is used to connect the dots between disparate and incomplete information, to gain an understanding of the larger picture. When faced with riddles, mysteries and utterly unknowable forces, a high logic can be used to discern the fundamentals of the problem at hand.}

\proficiency{Nature}{Intelligence (Nature) checks are used to remember information about naturally occuring plants and beasts (both magical and mundane), the terrain or the weather. }

\proficiency{Research}{Attempting to learn new information about a known target subject falls under the domain of an Intelligence (Research) check. When faced with a library full of books and information to assimilate, Research is your friend. {\it Research} differs from {\it Investigation} in that whilst {\it Investigation} helps you find a book, only {\it Research} can help you glean knowledge from it.  }   

\proficiency{Un-nature}{The partner to the {\it Nature} proficiency, an Intelligence (Unnature) check is used to recall information and lore about unnatural, otherwordly, un-living or otherwise artifical items, creatures and constructs.}


\subsubsection{Spells}

Some spells rely on nothing more than a razor sharp mind and a deep understanding of the task at hand, and hence use the Intelligence modifier for their casting and accuracy checks. Such spells include those from the {\it Temporal}, the {\it Warding} and the {\it Occultism} disciplines.


\subsection{Perception}

The Perception attribute is your awareness and openness to the world around you - both in a material sense, and on an emotional level. 

A Perception check will be used any time you wish to take in information around you, be it to spot hidden enemies, traps or paths, search through a vault of treasures, or discern the true intentions of a being. 

To that end, the Perception attribute is split into three proficiencies: {\it Empathy}, {\it Investigation}, and {\it Observation}. 

\proficiency{Empathy}{A Perception (Empathy) check is used whenever a being needs to put themselves in another's shoes - to understand their current state of mind, understand motive and intent, and possibly glean any hint that they are lying or omitting the truth. A high Empathy check might mean that you understand an individual better than they understand themselves.}

\proficiency{Investigation}{A Perception (Investigation) check is used for in-depth scrutiny of an object, container or region. Unlike an {\it Observation} check, an Investigation is always used consciously. A high Investigation check would allow you to:
\begin{itemize}
\itdef
\item Spot a tiny inscription on the inside of a ring
\item Rifle through a chest full of nicknacks, to find a priceless object
\item Find a given book in a packed and disorganised library
\item Notice a hidden chamber hidden inside a wall, or spot the secret mechanism to trigger the door
\item Search the body of a slain enemy (or hapless victim) for useful items or clues
\end{itemize}
}

\proficiency{Observation}{A Perception (Observation) check will be called for whenever you survey your surroundings, either with sight, sound or smell - to spot an ambush waiting for you to pass, or to notice a whispered conversation. Your Observation skill denotes both your spatial awareness, and your awareness of actions occurring within that space.  }

\subsubsection{Passive Perception}

As with the Spirit attribute, Perception checks will often occur without conscious effort from the part of the individual - sneaking past bored guards is different from sneaking past guards who are actively searching from you! Your own passive perception may be used by the GM to decide whether to alert you or not to a hidden creature stalking you. In such cases you use the {\it Passive Perception} score, which is calculated from the average dice roll the being would be expected to make.

Therefore it is calculated from a score of 10, plus the usual Perception (Observation) bonuses. If a being has advantage or disadvantage, you add or subtract 4 from the score. If you have super-advantage or disadvantage, you add or subtract a further 2. 

\subsubsection{Spells}

Some spells require a deep attunement to the world around you, and the ability to notice and react to very fine details. Such spells use the Perception modifier in for both the Spellcasting and Accuracy check. This spells generally fall into the {\it Telepathy} and {\it Healing} disciplines. 



\subsection{Power}

The power attribute is a measure of the power that a being has at their disposal - usually in the form of magical power, though it may also be used as a proxy for political power, or the simply the aura of power that one projects. 

A Power check will rarely be called for outside of the context of a spellcasting context, or when resisting the effects of a spell, however you may be called on to perform a power check when performing an extraordinary feat of magic that goes beyond the normal remit of a spell's abilities. 

A powerful being may be able to use their formidable aura through the {\it Intimidation} proficiency.

\proficiency{Intimidation}{A Power (Intimidation) check will be called for whenever you attempt to leverage your superior abilities to threaten an individual into doing what you wish. }

\subsubsection{Spells}

Spells which simply require raw magical power use the Power attribute in both spellcasting and Accuracy checks. Spells which fall into this category belong to the {\it Curses} and {\it Elemental} disciplines. 

In addition, raw magical power may be leveraged into making spells more potent. 

Whenever a spell causes damage, add your Power modifier to any Damage check which is performed, unless the spell specifies otherwise. Spells which require a Resist check to be performed (both damage causing and otherwise), the DV of the Resist is set by your {\it Subjugate} value, which is calculated from:
$$ \text{Subjugate} = 8 + \text{Expertise bonus + Power modifier} $$


\subsection{Evil}

The Evil attribute is a measure of the darkness and corruption which lies in the heart of an individual. 

In a perhaps na{\"i}ve view of the world, this game system presumes people are, by default, inherently good. Comitting evil acts therefore requires conquering your inner, better nature. Slitting the throat of a incapcitated prisoner might be physically easy to do, but to actually go through with such a foul deed you must overcome this inner good - which requires passing an Evil check. 

Each time you commit such a deed, you will likely find your Evil rising in tandem with the blackening of your soul. 

Evil has no proficiencies associated with it. 

\subsubsection{Spells}

The most evil spells in existence can only be cast by those with a corrupted and wicked soul - the unforgivable curses, the animation of the dead as gruesome puppets and so on - and hence use the Evil attribute for casting and accuracy checks. This spells form the discipline known as {\it Necromancy}. 



 


\chapter{Everyday Actions}

Within the framework of the game, there are broadly two classes of actions: {\it everyday} and {\it combat}. Everyday actions are things such as traveling between two cities, getting some sleep, talking to a friend, sitting in the library and so on. Combat, however, involves things trying to hurt you, and you trying to hurt them back. 

This section is concerned with the everyday, and is by no means meant to be an exhaustive list of things you may do. Instead, it merely provides some guidelines as to how to perform some common actions, and the effects that they can have. 

\section{Roleplaying}



\section{Movement}

Out of combat, wandering around the environment is very natural -- you simply tell the GM that you want to go over there, and you do - barring unforeseen circumstances such as traps. You needn't calculate the exact time taken for each individual movement (that would get dull), but it is generally presumed to occur on the scale of seconds to a few minutes. 


However, sometimes you might wish to travel over distances which will take more than a handful of minutes. If you are travelling by foot more than 10 minutes, then you need to decide how rapidly and carefully you are moving.


\begin{center}
\begin{rndtable}{|c c c m {3 cm}|}
\hline
Pace & Speed & Duration & Effect
\\
\hline 
Slow & 2km/h & 8 hours & Can remain hidden, or draw a map
\\ 
Normal & 4 km/h & 7 hours & Can draw a map
\\ 
Rapid & 6 km/h & 5 hours & -5 penalty to all checks made whilst moving. Costs 5 FP per hour.
\\ 
Breakneck & 10km/h & 1 hour & {\raggedright- 10 penalty to all checks checks made whilst moving. Costs 2 FP per minute and 5 HP per hour.}
\\ \hline
\end{rndtable}
\end{center}

If you attempt to travel for longer than the `duration' of the selected pace, you risk exhausting yourself. After the first additional kilometre travelled, all members of the party must succeed a DV 10 ATH (endurance) check. This check must be repeated after every subsquent kilometre travelled, with the DV increasing by 1 each time. After failing this check, you must halt, and take an additional level of exhaustion. 

This timer resets after a rest of more than 8 hours, after which time you can take up your pace again. 

\subsection{Vehicles \& Mounts}

Of course, the discerning wizard rarely travels too far on foot - they may prefer to use a broomstick, tame and ride a griffin or simply apparate or portkey around. 

Each of this modes of transport has their own limitations, specified by the relevant item, beast or spell effects. 

\subsection{Actions while moving}

It is possible to perform other actions whilst on the move, though unless you are travelling in a luxury carriage, you may be somewhat restricted in what exactly you can achieve. 

You may make checks to navigate, to track a foe keep or to keep an eye out for enemies (these all use variations on the PER attribute), or you may leverage your knowledge of Flora & Fauna to forage for food and water. The faster you travel, the heavier a penalty you suffer for these checks. 

Whilst travelling at a slow pace, you may make an effort to remain hidden, the rules for which are elaborated on more on page \pageref{S:Stealth}. 

If the Slow or Normal pace is used, a member of your part may elect themselves as a map-maker, if they have the {\it Observation} proficiency. Having a map makes it impossible to get lost (unless the scenery is magically altered, of course), and you can always retrace your steps. 

\subsection{Special Movement}

Walking and running are not the only kinds of movement out there: navigating a dangerous environment often requires other ways of exploring the space. 

\subsubsection{Climbing}

You may navigate slopes of up to 20 degrees (2 in 5) without difficulty. Between 30 degrees and 50 degrees you must move at half speed, but can walk normally. For slopes between 50 degrees and 90 you must use your hands to climb. If you wish to use a hand to perform an action, you must halt, check you have a secure handhold with the other hand, and then continue. Failure to do so will lead to a (possibly fatal) fall.


\subsection{Resting}

Resting is an important action that can only occur when not in combat. Attempts to rest during combat are highly likely to get you killed on the spot. 

When in safe territory, you may set up camp, and get a few hours shut-eye to recover from your ordeals (see the Asleep status effect for details). But be warned, the night is dark and full of terrors, and who knows what might sneak up on you whilst you are resting…

You may take rests whilst delving deep into unfriendly territory, but note that resting after every single encounter is generally frowned upon, and the GM might start throwing more and more unpleasant random encounters at you if you begin to take things to the extremes. 

You should only rest in a place where it makes sense to rest – it does not makes sense, for example, to take a quick nap in whilst delving through the dungeons of an evil warlord, even if you have cleared the immediate area of enemies. Of course, if you kill the Warlord and claim his castle as your own, then it is a different matter...

\section{Social Actions}

\section{Downtime}

\chapter{Combat}

\section{The Combat Cycle}
Unlike most RPGs, which tend to use a turn-based system for combat, this game uses a simultaneous combat system. The reason for this is that whilst the turn-based combat fits in with how we play games (I have my turn, you have yours, etc.), it is not entirely realistic: in a fight, you don't wait patiently for everyone else to complete attacking you before finally returning fire: everybody is completing actions at once. 

After combat is initiated, a series of turn cycles occur. Each turn cycle allows every character in combat one major action, such as: a movement, casting a spell, or using an item. 

At the start of each turn cycle there is a period of time (to be decided by your GM), during which you must decide on what you will do. Players may talk to each other during this time, but do be aware that discussing your tactics in front of the GM may give the game away, you wouldn't start shouting your plan out whilst fighting the enemy now, would you? 

After this time is up, each player writes down their action on a scrap of paper (to prevent last minute changes of heart), and then all players (including the GM) reveal their action simultaneously. 

The GM then resolves the effects of all these actions - directing characters to perform accuracy and damage checks where appropriate - and then narrating the outcome, and the response (if any) of the remaining aggressors. 

The combat cycle then begins anew until the conflict is resolved. 

\subsection{Time}

Each combat cycle is assumed to have a duration of around 3 seconds. 

Attempting to perform actions that last significantly longer than this requires spreading the action across multiple turns -- though may choose to abort such an action if you feel your talents are better placed elsewhere. 

\subsection{Resolving Conflicts}

Since all actions are considered to be simultaneous, the order in which the actions are resolved does not usually matter. Recall that spells, arrows, and sword swings have a finite travel time, so it is entirely feasible for two players to attack each other simultaneously and it does not matter who initiated first.

It might, of course, still be possible for actions to come into conflict with each other: if two characters attempt to occupy the same space, for example. It is up to the GM's discretion how to deal with edge cases like this - for the example given, it is recommended that this be treated as a `body slam', and both characters should recoil and take some damage. 

There might also be cases where two spells are cast simultaneously where the ordering does actually matter: for example, if you heal someone at the same time that someone casts a damaging spell that would take them below 50\% health, incurring the ``major injury{\apos \apos} status. If the healing action occurs first, then they are not taken below 50\% health, but if the damage action occurs first, then they do fall below 50\%, even if they are then brought back up over that threshold. The final health that the character ends up on might be the same, but the ordering of actions effects whether they have the {\it major injury} status at the end of the turn. 

In cases such as this it is useful to remember that it is the {\it casting} of the spell that is simultaneous: so the ordering in which the spell effects should take place can be inferred from the distance between the caster and the target. The issue above is resolved simply by looking at whoever is closest to the target. 
 

 \section{Taking Actions}
 
 During each combat cycle, each character may take {\bf one} major action, or {\bf two} minor actions. In addition, your character has a number of {\it instincts} which they execute to avoid damage and brace against incoming attacks. 
 
 The list below gives some common mechanics for both major and minor actions. As usual, however, characters are free to be as inventive as they like. It is up to the GM to determine if an action is major or minor in nature, and how to resolve it. 
 
 \subsection{Major Actions}
 
 Major actions take an entire turn to complete, and as such are considered the main way to engage in combat. Some skills and archetype abilities allow you to perform multiple iterations of a single major action per turn, or may grant you multiple major actions to take. 
 
 \subsubsection{Attacking}
 
 Casting a spell, swinging a sword, or loosing an arrow takes (usually) a full turn to complete, and so you may decide to use your entire turn to cast a spell. 
 
The rules for performing attacks are elaborated on page \pageref{S:Attacks}
 \subsubsection{Movement}
 
 When used as a major action, movement allows you to move on foot up to a distance given by your {\it running speed} statistic, which is calculated from your base speed (derived from your race) and your athletics attribute:
 
\small
$$ \text{running speed} = \text{Base Speed } + \text{Athletics modifier} $$  
\normalsize

The rules concerning special movement, as discussed on page \pageref{S:SpecialMovement}, also apply in combat. 

If you possess the {\it Speed} proficiency and you made a full-turn movement last cycle, you may convert your movement into a {\it sprint}, and add your expertise bonus to your speed. You may then maintain this until you need to stop or change direction. 

Whilst moving, you need to be careful that you do not collide with other beings - either your allies or your enemies. You cannot enter space that is currently being occupied by another solid being (ghosts, however, are fair game). 

 \subsubsection{Using Items (sometimes)}
 
Some `uses' of items include using swords, wands and ranged weapons, which have already been covered by `attacking'. 

However, sometimes you might want to use an action to get something big done, outside of hitting somebody. Using a crowbar to pry open a door, changing your weapon, finding the right page of a book -- all of these take enough time to be considered major actions. 

Some uses might take multiple turns -- for instance, climbing into a full suit of armour takes more than 3 seconds to complete, and will therefore require multiple, consecutive major actions. 

In contrast, some actions (see below) are small enough to be considered minor actions. The GM has veto on which actions are major or minor. 

\subsubsection{Trading Items}

If two characters are standing within touching distance, they may trade items between them. 

Alternatively, you may attempt to throw an item to your ally, treating the item as an `improvised weapon'. If the throwing check is successful, the catcher performs a DV 10 ATH check to catch the item, and adds it to their inventory. 

Whichever method is chosen, giving items to other people takes the major actions of both the giver and the receiver. 


\subsection{Minor Actions}
You may perform two minor actions in place of a single major action. Generally, these two actions happen simultaneously: if you drink a potion and make a minor movement, then you are drinking the potion whilst moving. This places a good guide on what can be considered a minor action: is it possible to do this at the same time as I'm walking/talking/dodging? 

\subsubsection{Minor Movements}

Actions such as taking a single step, or peeking out from behind cover, do not take any time, and can be performed in the same turn as a major action. 

However, there is a middle ground between the sprint of a full-turn movement, and the zero-time of a single step. This is called a {\it minor movement}. 

During a minor movement, one moves only {\bf half as far} as during a full-turn movement, but since you are not focussed solely on moving as far as possible, you can perform other minor actions. 

\subsubsection{Quick Attack}

Just as there is a difference between a full-on sprint (a major action) and a quick jog (a minor action), so to is there a difference between a zeroed in shot on your enemy (a major action), and releasing a spray of covering fire to keep your enemies on their toes (a minor action). 

A quick attack takes only a minor action to complete. The penalty for this, however, is that you must take check-disadvantage on the associated accuracy checks, and you may not use mechanics such as Power Points or Power Attacks which increase the damage of the attack. 


\subsubsection{Communication}

Communicating vital information - such as the location of a hidden enemy or trap - to your comrades takes a minor action. Note that it is assumed that the enemy can hear you, unless you make an effort to not be understood. 

\subsubsection{Using Items (sometimes)}

Item use has already been discussed as a major action, but there are conceivably such actions that would fall into the minor action category. Consuming a potion, checking a rememberall, removing an item from your bag and so on would be considered `minor actions'. 

Any item use that can be completed in around 1 second, or which can be easily `multitasked', is considered a minor action. 

\subsubsection{Bolstering Defenses}
 
 You may also choose to ready yourself against incoming attacks, by bolstering your ability to either {\it Dodge} or {\it Block}. This gives you a better chance of negating incoming effects.  

See page \pageref{S:Accuracy} for more details on this mechanic. 



\subsection{Conditional Actions}

The use of the simultaneous combat system raises some interesting opportunities with conditional actions, which are actions that depend on the actions that another character takes.

The actual action, as well as the trigger condition, needs to be declared during the normal turn cycle -- but the action itself is not triggered until all other actions had been triggered. 

For example, it could be that you declare as your action \textit{if the troll attacks player A, then I cast a healing spell on player A}. You could also attempt to prevent the damage from being taken in the first place, by declaring \textit{if the troll attacks player A, then I cast the knockback charm on the troll}. The GM may ask for a check to determine if you are close enough and have fast enough reactions for your spell to interrupt the action, but if you pass this, then you may be able to save your friend.

You are only allowed a single conditional clause in your declaration, and if that conditional does not come to pass, then your character does not do anything: there is no \verb|if-then-else| in this game!

If a seemingly unbreakable condition-chain arises (i.e. player A says he will perform X if player B does Y, but player B says he will only perform Y if player A does X), it is up to the GM to resolve the conditionals -- in such cases the answer is usually \textit{nothing happens}, but there may be examples where the GM feels it is more appropriate that the action-chain is triggered. 


\section{Making Attacks}\label{S:Attacks}

When making an attack, either with spells, arrows, or with a blade, there are 4 key steps:
\begin{itemize}
	\item Select a target 
	\item Perform an accuracy check 
	\item See if the target defends themselves
	\item Calculate the damage inflicted
\end{itemize}

There are also some special rules regarding melee and ranged attacks.
\subsection{Target Acquisition}

You may only attack targets that are within the range of the attack you are making. For melee weapons, this is usually 1 metre, though some long weapons such as lances have additional reach. For ranged weapons, the maximum range is specified in the weapon description. Spells also have ranges associated with them, which is discussed more on page \pageref{S:Range}. 

In addition, to determining if the target is in range, you must determine if it is a valid target - you cannot shoot arrows around walls, after all. You must be able to see a target in order to attack it (see below for blindfighting rules), and you may need to consider the fact that a target has cover. 


\subsection{Melee Attacks}

A melee attack encompasses all close-range fighting, including fist-fighting, sword-swinging and whip-wrangling.

Typically, a melee attack can only be made against a target if they are within 1 metre of the attacker, with a clear line-of-reach between the two. Some weapons, as well as larger creatures, are able to perform melee attacks at a larger range.  

\subsubsection{Grappling}

If you wish to grab your opponent- either to immobilise them, or to pick them up and throw them off a cliff - you may attempt to initiate a grapple in place of a regular attack. 

To perform a grapple you need two free hands and perform an Athletics (Strength) check, which is contested by the target performing either an Athletics (Strength) or Finesse (Speed) check. If the grappling succeeds, the target acquires the trapped status. 

If the grappler is strong enough, then they move whilst carrying the target subject to the following constraint:

\begin{center}
\begin{rndtable}{c c}
\bf Weight	&	\bf Speed
\\
Heavier than $ 5 \times$ ATH(strength) value	&	Speed = 0
\\
Heavier than ATH (strength) value	&	Speed halved
\\
Lighter than ATH (strength) value & Unencumbered
\end{rndtable}
\end{center}

A grappled target may attempt to use their action to escape. Repeat the contest. 

\subsubsection{Shoving}

{\it Shoving} is considered a special form of grappling - rather than restraining the target, you may choose to push them to the ground (taking the {\it prone position} status), or push them back 1 metre. 

\subsubsection{Two-Weapon Fighting}

It is possible to have multiple one-handed weapons equipped at once -- for example, a dagger in each hand. 

If you are proficient with at least one of these weapons, you may perform a double-strike when making an attack as part of a major action. Perform the damage check with both weapons and sum them together. 

However, unless you are proficient with two-weapon fighting, you may not add your expertise bonus to either weapon check. 

\subsection{Ranged Attacks}

A ranged attack occurs over a longer distance by firing a projectile or magical effect up to the scale of hundreds of metres in some cases. 

\subsubsection{Ranged Weapons}

The description of every ranged weapon gives a maximum range at which the weapon may be fired. Some weapons have multiple ranges depending on the way in which they are used. 

Slings, for example, have a much longer reach when using aerodynamic bullets, as compare to just using rocks. Equally, hip firing a rifle has a much less accurate range than when lying in a sniper nest. 

Generally speaking, you cannot fire a projectile further than this range, as it represents the maximum distance that the projectile can reach. Some weapons (particularly the {\it firearms} class), however, the stated range is merely the range at which you can fire accurately. These weapons {\it can} be fired up to twice their stated range, but take check disadvantage on all accuracy checks beyond this point. 

In addition, you will need to ensure that you have enough ammunition to properly use your ranged weapon.
\subsubsection{Spells}

Many spells state that they have an effective range, which is discussed more on page \pageref{S:Range}. You cannot exceed this range, without skills which explicitly extend your spellcasting range. 

\subsubsection{Close-Combat Firing}

Ranged weapons and spells are significantly less effective when used on targets which are in close-quarters: aiming requires a clarity of thought that a monster trying to bite your face off denies. 

When attempting to use a ranged attack on a non-incapacitated target within melee range, take check disadvantage on the accuracy check.

\section{Accuracy}

The attacker quantifies their ability to successfully hit their target through an {\it accuracy check}. 

\subsubsection{The Accuracy Check}

An accuracy check is performed using the usual d20 die. However, the associated attribute depends on the type of attack being performed. Generally speaking the following prescription is used:

\begin{center}
\begin{rndtable}{c c}
\bf Attack Type	&	\bf Accuracy Attribute
\\
Spells	&	Finesse
\\
Melee Weapons	& Athletics
\\
Ranged Weapons	&	Finesse
\end{rndtable}
\end{center}

Some weapons diverge from this prescription, for example, a rapier is a melee weapon, but it requires great finesse to use expertly. See the item descriptions on page \pageref{S:WeaponList} for the check for each individual weapon. 

\subsubsection{Proficiency}

In addition, if you are considered proficient with the weapon (or wand) you are using to attack, you may add your proficiency bonus to the accuracy check. 

\subsubsection{Hitting the Target}

When attacking a living being, the DV of the accuracy check is determined by the {\it instinct value} used by the target. If you meet this target, then the attack lands true. If the accuracy check fails, then the attack misses, or is successfully blocked by the target. 

\subsubsection{Hitting Stationary Targets}

If the target is not a living being (or is restricted from moving), then hitting the target is much easier, but not totally guaranteed. The `dodge' DV of a stationary object is normally equal to 5. 

Some spells are classified as neither blockable nor avoidable -- but are still clearly attacks which target an enemy. For these spells, you treat the target as a stationary object. The same is true of `area effect' spells which target a region rather than a being. 

\subsubsection{Additional Difficulty}

Targeting objects which are particularly small, or (for ranged attacks) far away is more difficult.  The additional penalty for hitting such away targets is, with everything measured in metres:

$$ P = \frac{\text{distance}}{10 \times \text{size}} ~~~~ \text{(rounded down)}$$

Therefore, hitting a 1m target at a distance of up to 10m has a DV of 5, whilst the same target 30m away has a DV of 8, and hitting a 1cm target at a distance of of 1m has a DV of 15. 

\subsubsection{Blindfight}\label{S:Unseen}

If you cannot see your enemy, then you cannot select them as a target. You may, however, choose to simply start swinging your sword, or firing spells off in a random direction. You must tell the GM which direction you are attacking in, and then perform an accuracy check with a -5 penalty. 

If the target is not in the region  you are attacking, you automatically miss (though the GM will still ask for the accuracy roll, to avoid giving away where they actually are!). 

After you successfully hit an unseen attacker, you avoid the -5 penalty until your next attack misses, at which point you must retake the penalty until you find them again, or you can spot them visually. 


\section{Defence}\label{S:Accuracy}

A good fighter knows that all-out attack is rarely the path to victory: defending onself against incoming attacks is just as important. 

\subsection{Instincts} \label{S:AC}

Most beings either block or dodge, without having to devote conscious thought to their reaction. These two actions are therefore termed {\it instincts}. It is these reactions which set the difficulty of an attacker's accuracy check. A higher {\it dodge} or {\it block} statistic makes it harder for an attack to actually hit you. 

The values associated with each statistic are:

\begin{align*} 
\text{Block} &= 10 + \text{Athletics modifier} 
\\
\text{Dodge} &= 10 + \text{Finesse modifier} 
\end{align*}

By default, characters instinctively use whichever of these values is the highest:
$$ \text{IV} = \max \left( \text{Block}, \text{Dodge} \right)$$

If a character successfully dodges, the attack whizzes by their ear and misses completely. If they successfully block the attack, then they catch the spell or weapon on a piece of armour (or, with the appropriate skill, they can {\it parry} the attack with a weapon). 

\subsubsection{Clothing \& Armour}

Various items may improve either of these statistics. A pair of running shoes, for example, makes it easier to dodge out of the way, whilst a heavy shield makes defending yourself easier. 

Generally speaking, items will be a compromise: wearing heavy armour will bulk up your Block statistic, but will slow you down, reducing your Dodge value. 

Armour is discussed more in the Items chapter, on page \pageref{S:Armour}.

\subsubsection{Bolstering Defences }

Of course, not all defence happens instinctively -- you may make a conscious decision to brace yourself against an incoming attack, or prepare to dive out of the way. Such a decision is classified as a minor action. 

Though by default you automatically use whichever value is highest, when making a conscious decision, you may choose to bolster either statistic by {\it bracing} or {\it evading}. 

Whichever action is chosen, the effect of bolstering your defenses is:
\begin{itemize}
\item Double your Expertise bonus (if applicable) on the chosen statistic 
\item Aggressors take check-disadvantage on their accuracy checks against you this turn cycle. 
\end{itemize}


\subsection{Cover}

Standing out in the open is a sure-fire way to get hurt quickly. Hiding behind something, be it a tree, a low wall, or even just your ally will make you safer and harder to hit. 

A target which is concealed in this fashion is said to be {\it under cover}. It is up to the GM to determine to what extent a target is hidden from view. This can usually be achieved through the `additional difficulty' mechanics discussed in the {\it Accuracy} section above. 

If a 2m tall target is 15m away, the penalty to hit is zero. However, if they were covered such that only their head ($\sim 30$cm) could be seen, you can estimate that the penalty to hit them would be -5.

Alternatively, you may use the simpler rules that `half cover' (i.e. half of the target is concealed) gives a -2 penalty to the accuracy check, and `three-quarter cover' gives -5, in addition to any other distance penalties. 

\subsection{Undefendable Effects}

Some effects cannot be avoided or blocked: holding up a shield against an incoming cannonball isn't going to prevent much, and trying to dodge out of the way of a tsunami is rarely effective. 

Spells denote in their description if they can be blocked or dodged. For the (rarer) instances of non-spell effects which fall into one of these categories, the GM decides if it is reasonable to dodge or block the effect. 

If the `dominant' instinct (i.e. the one with the highest value) would be ineffective against a given effect, you may use the non-dominant one. However, if the character chose to, for example, use the evade action, they may not transfer the bonus to `block' if an evasion turns out to be ineffective. 

Note that even `unblockable' effects are stopped by `impenetrable' fields and spells which are `undodgeable' treat the target as stationary, and may still miss under those rules. 



\section{Doing Damage} \label{S:Damage}

If an attack lands home, and the target fails to defend themselves, then you must calculate how much damage is done.

\subsection{Calculating Damage}

Most attacks specify the amount of damage they do, either in the weapon description on page \pageref{S:WeaponList}, or in the spell effect list found on page \pageref{S:SpellList}. This is usually in the form of a dice roll, i.e. 2d6.

However, in addition to the dice, you also add a modifier on to the damage check. {\bf You never add your Expertise bonus in to a damage check}, however. 

\subsubsection{Spells}

In most cases, a spell does more or less damage depending on the {\it power} of the caster, though there are exceptions. Unless otherwise specified, you add your Power modifier to the damage check when casting spells. 

\subsubsection{Weapons}

When using a weapon, you add the same ability modifier (minus the Expertise bonus) you used in the accuracy check. 

\subsection{Group Attacks}

If a spell or other effect that attacks multiple targets at the same time, perform the damage check once, and apply the damage to all targets that were hit. 

This does not apply to a single effect which causes multiple copies of the same effect to attack multiple individuals. For instance, the {\it Cascading Missiles} may attack a number of individuals with magical darts, but as each dart is a different copy, the attack roll is unique. This contrasts with a {\it Fireball}, which is a single effect that effects a large area. 

\subsection{Damage Types}



Many effects specify what kind of damage they do (for instance, a sword does 1d8 slashing damage). This helps the players and the GM work out how the damage is done, and also how it is affected by any weaknesses and resistances possessed by the target. 

Some damage types do damage in unusual ways - draining Fortitude instead of Health, for example. 

\newcommand\damage[2]
{
\textbf{ \textit{#1}}: #2
}

\damage{Acid}{A corrosive spray of acid attacks the HP of a target, and weakens their armour.}

\damage{Bludgeoning}{The blunt-force of a hammer, or the force of falling on the ground deals bone-breaking bludgeoning HP damage.}

\damage{Celestial}{Celestial damage is dealt by pure-otherworldly energy, and damages the HP of Unliving and celestials, but does no harm to living beings.}

\damage{Cold}{Freezing temperatures seep at both your willpower and your health. Damages both the HP of a target, and half as much damage again to FP. } 

\damage{Concussive}{A concussive blast from an explosion or a shockwave causes deafening concussive HP damage.}

\damage{Electric}{Bolts of lightning, or simply touching a high-voltage wire, can lead to electrical HP damage. Electrical damage conducts through water and metal, harming all thosein contact.}

\damage{Fatigue}{A magical will-sapping force damages only your FP.}

\damage{Fire}{Fire damage burns the flesh to reduce the HP of a target, and can often lead to long-lasting burns.}

\damage{Force}{A pure magical energy that directly damages HP.}

\damage{Necrotic}{The evil energies of the undead withers your soul as it damages your body -- reducing HP and FP by equal amounts.}

\damage{Piercing}{Daggers, spears and teeth can puncture even the thickest armour to damage HP.}

\damage{Poison}{Venemous stings and poisoned weapons damage HP, and may lead to some other unpleasant side effects}

\damage{Psychic}{Damage that originates not from the body, but from the mind, but still damages your HP. You often cannot block psychic damage, you must instead rely on Resisting it.}

\damage{Slashing}{Swinging blades and flashing claws damage the HP of unprotected targets.} 


\subsection{Critical Strikes}\label{S:Sneak}

If you manage to launch an attack whilst remaining hidden, or if you roll a `natural 20' on your accuracy check, then you may trigger a {\it critical strike} on a target. 

When a critical strike happens, you double the number of dice used in the damage roll. For instance, a critical strike with a shortsword normally does 1d6 damage + modifiers. On a crticial strike, however, you would do 2d6 + modifiers. 


\section{Resisting}

Not all effects of actions are cut and dried -- some effects can be {\bf Resisted}. 

Many spells, for example, can be resisted by the target. This occurs if they have a strong enough willpower to overpower the caster; spells such as {\it confundus}, and {\it stupefy}, as well as most illusion spells. Alternatively, somebody might try to restrain you, and your character can perform a physical Resist to break free, if they are strong enough. 

Resist actions, like normal checks, are assigned an attribute (and possibly Proficiencies) that may boost the Resist check. Unless otherwise specified, the Resist check is performed using the standard d20 dice. 

This Resist check is then compared with the assigned or contested DV. If the Resist check is greater than the CV, then the action is either denied, or has a lesser effect. 

Successfully Resisting costs 2 FP. If you have fewer than 2 FP, then you cannot Resist.

You can perform multiple Resists over the course of a Turn Cycle, if multiple combatants attack you with spells that require one, for example. The only limit is when your FP runs out. However, each subsequent resist gets harder and harder: you suffer a 1 point penalty to your check for each Resist you have already performed this cycle. This counter resets at the end of the cycle.

\section{Stealth} \label{S:Stealth}

Being noticed by the enemy is generally regarded as a bad thing. It therefore often pays to be sneaky, to stay hidden from the enemy. Stealth is governed by the FIN attribute, via the Stealth proficiency. 

\subsection{Hiding}

If you are not currently being observed by a being, you may take a major action to {\it Hide}, by performing a d20 Finesse (Stealth) check. This stealth value will then be contested by any hostile beings around you. 

Whilst you are hidden you are considered an `unseen' foe, with the bonuses that come with that (see \pageref{S:Unseen}), and you are not a valid target for an attack. However, you may still take damage from area of effects that include you in their area. 

The GM may ask you to re-perform the sneak check if there is a material change in circumstance. For instance, if you performed the check in a dingy room, and suddenly the lights are turned up, then you may need to re-perform the check, in line with your character altering their strategy for the new environment. Equally, if you take damage whilst hidden, you must perform a DV 15 Spirit (Endurance) check to grit your teeth through the pain. 

You remain hidden until you to do something to give away your position: shouting to your allies, or jumping from the shadows, sword in hand. 

If an individual enemy does manage to spot you, but their allies fail to, they can use a {\it communication} action to alert everyone else to your presence. 


\subsection{Being Discovered}

Every character and beast has a baseline level of awareness, even when not actively searching for hidden creatures or traps. This is your {\it passive perception}. It is calculated using an `average' dice roll (for a d20, this is 10), so: 
$$\text{Passive PER} = 10 + \text{bonuses}$$

Alternatively, the beings might decide to take a major action to survey their surroundings, in which case they may perform an active Perception check, which may increase their perception value for this turn. 

If a being's perception value exceeds your sneak value (and it is reasonable for them to be able to percieve you), then they have spotted you, and you are no longer hidden from that creature.  




