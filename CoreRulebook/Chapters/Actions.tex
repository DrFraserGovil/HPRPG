\documentclass[../CoreRulebook.tex]{subfile}

\chapter{Performing Actions}


In general, when you want to perform an action, simply tell the GM what you wish to do. 

If it is a simple action – for example, “I walk to the shop”, then the action is completed with no further involvement. More complex actions may require a ‘check’ to be performed, to determine their success: inform the GM of what you want to do, and the GM will tell you what check to perform. 

\par
~
\par

\section{Checks, Modifiers \& Proficiencies}\label{S:Profs}

Generally speaking, the following list should serve as a guide as to what general actions require which check:

\begin{itemize}
    \item ATH: ~~Required for physical feats, i.e. running, jumping, dodging and swimming
    \item FIN: 	~~ Required for precision actions 
   \item   SPR: ~~	Feats that require concentration, force of will, and awareness of your surroundings.
    \item  CHR:~~ 	Convincing and persuading others, haggling and distractions. 
    \item  INT: ~~	Research and knowledge actions. 
    \item  EMP:~~	Calming others, requesting help, identifying when something is amiss. 
    \item  POW:~~ 	Often used to impress others, or intimidate them into helping you
    \item EVL: ~~	Intimidation, scaring. Often necessary to commit atrocities. 
\end{itemize}

On top of these general actions, each school of magic has an inclination towards one attribute or another as the required casting checkes -- Hexes \& Curses favour PWR and SPR, Transfiguration favours FIN and INT, Charms spells lean towards POW and FIN, Protective spells rely heavily on EMP checks, as does Divination, illusion obviously required CHR checks, and the Dark Arts requires extensive EVL checks. 

As always, the GM has the authority to override these general guidelines, if it is suitable to do so (there is a certain crossover between CHR and EMP checks, for example, which may require event-specific discretion). 

For more detail on how to calculate a check, see page \pageref{S:Checks}.

\subsection{Modifiers}
Each character has an associated score in each of the 8 attributes, which enables them to more reliably pass checks in these fields, via the use of the associated modifiers.  When performing an ATH check, for example, you perform the requisite check (usually a d20), and then add on your ATH modifier, which is derived from your ATH score in the following way:
$$ \text{attribute modifier} = \frac{\text{attribute level} - 10}{2} \text{ (rounded down)} $$
The table below gives an example:

  
\begin{center}
\begin{rndtable}{|c c|}
\hline \bf Attribute Value 	&\bf 	Modifier
\\ \hline
6-7	&	-2
\\
8-9	&	-1
\\
10-11	&	+0
\\
12-13	&	+1
\\
14-15	&	+2
\\
\hline
\end{rndtable}
\end{center}


Note that modifiers for attributes with a skill less than 10 are negative! You won't necessarily {\it always} be allowed to use your modifiers on every check -- if you attempt to use a weapon you are not proficient in, for example, all positive modifiers are negated.
\subsection{Proficiencies}
In addition to the 8 main attributes, checks can be further subdivided up into 20 {\it proficiencies}, as follows:
%\setlength{\columnsep}{-1cm}
\begin{multicols}{2}
\begin{itemize}[leftmargin = 0.2cm]
{\raggedright  \footnotesize
	\item ATH
		\begin{itemize}[leftmargin = 0.2cm]
			\item {\bf Health}
				   {\footnotesize \it Used in checks to determine how healthy a character is, i.e. when resisting the effects of poisons and diseased}
			\item {\bf  Speed}
				  {\footnotesize \it Used in movement and dodge checks}
			\item {\bf Strength}
				 {\footnotesize \it Used in weapon checks, and when brute force is required }
		\end{itemize}
	\item FIN
		\begin{itemize}[leftmargin = 0.2cm]
			\item {\bf  Dexterity}
			 {\footnotesize \it Checks that require a steady hand, such as pickpocketing, or sleight of hand}
			\item {\bf Stealth}
			{\footnotesize \it Stay hidden from your enemies}
			
			\item {\bf Precision}
			 {\footnotesize \it Used for long-distance shooting, or for ultra-precise spellcasting}
		\end{itemize}
	\item SPR
	\begin{itemize}[leftmargin = 0.2cm]
		\item {\bf Endurance}
		 {\footnotesize \it Checks that require resisting the effects of magic, tiredness or debilitating agony}
		\item {\bf Willpower }
		 {\footnotesize \it Checks that require dominating someone else's spirit}
	\end{itemize}
	\item CHR
		\begin{itemize}[leftmargin = 0.2cm]
			\item {\bf Deception}
			 {\footnotesize \it Lie, cheat and mislead other characters}
			\item {\bf Performance}
			 {\footnotesize \it Play music, sing, dance and otherwise entertain the masses.}
			\item {\bf Persuasion}
			 {\footnotesize \it Convince others to willingly go along with your ideas}
		\end{itemize}
	\item INT
		\begin{itemize}[leftmargin = 0.2cm]
			\item {\bf Research}
			 {\footnotesize \it Learn new information from books and other resources}
			\item {\bf Arcane knowledge}
			 {\footnotesize \it Identify magical spells, items and beings}
			\item {\bf History }
			 {\footnotesize \it Recall events from the past, and recognise important figures and items from those events}
			\item {\bf Flora \& Fauna}
			 {\footnotesize \it Identify living beingsof all shapes and sizes, and recall their properties}
		\end{itemize}
	\item EMP
		\begin{itemize}[leftmargin = 0.2cm]
			\item {\bf Perception}
			 {\footnotesize \it Recognise threats around you, spot things others might not}
			\item {\bf Understand Other}
			 {\footnotesize \it Used in checks to understand what others want -- useful in dealing with humans and beasts alike}
			\item {\bf Healing}
			{\footnotesize \it Gain bonuses when helping another being get better}
		\end{itemize}
	\item POW
		\begin{itemize}[leftmargin = 0.2cm]
			\item (None)
		\end{itemize}
	\item EVL
		\begin{itemize}[leftmargin = 0.2cm]
			\item {\bf Chaos}
			 {\footnotesize \it Gain a bonus when spreading wanton destruction and chaos}
			\item {\bf Intimidation}
			{\footnotesize \it Scare others, and make them more likely to give you want you want}
		\end{itemize}
		}
\end{itemize}
\end{multicols}
\setlength{\columnsep}{0.7cm}

Characters are provided points in a number of these areas at character creation through their racial abilities and backgrounds, and may gain more through aquiring Skills as they progress through the game. 

If the GM decides that your action falls under one of these categories, they may ask for, rather than a simple ``ATH check'', an ``ATH (strength) check'' in order for a character to lift a log, or an ``ATH (speed) check'' in order for the character to run away from danger. 

In this case, in addition to the general ATH modifier added on to the d20 check, the character would add on their points in the strength and speed proficiencies respectively (if they had any).

Generally speaking, proficiency bonuses will always be associated with the attribute listed here -- so Speed will usually be added on to an ATH check. However, in certain circumstances it makes sense to cross the borders. For example, if you are threatening to beat someone up, you might use an ``ATH (intimidation)'' check, or a ``POW (intimidation)'' if you are threatening them with magical violence -- even though Intimidation is an EVL proficiency. The GM decides what is appropriate for each moment. 

You are always allowed to ask the GM if a proficiency applies to a specific check, even if the proficiency was not explicitly asked for in the check -- but they are always within their rights ro refuse!

\subsection*{Arcane Wisdom}

A character's {\it Arcane Wisdom} is a check-modifier earned by general widsom in the magical arts. In effect, this means that it increases by one every five levels:
\begin{center}
\begin{rndtable}{|c c|}
\hline \bf Character Level 	&\bf 	Arcane Wisdom
\\ \hline
1-4	&	+0
\\
5-9	&	+1
\\
10-14	& +2
\\
15-20	& +3
\\
20+	&	+4
\\
\hline
\end{rndtable}
\end{center}
However, characters may also increase their arcane wisdom by learning the magic-school skills during the levelling-up process (see section \ref{S:Auto}).

The Arcane Wisdom stat may be used {\bf once per day} on any magic-related check. The Arcane Wisdom value is added into the CV like a normal bonus. 

The `once per day' timer resets 24 hours after the previous usage, and only if the character has had at least 7 hours sleep in the interim. 
\newpage
\subsection{Success \& Failure}

After having performed the check, you will end up with a single number -- a result of the (usually) d20 dice, modifiers, proficiencies and any other bonuses. It is now time to `resolve' the check, and decide if the action was successful or not. 

Generally speaking, the following table gives a rough outline of the check required to pass actions of a certain difficulty:
\begin{center}
\begin{rndtable}{|c c|}
\hline
Task Difficulty & Check
\\ \hline 
Very Easy & 5
\\
Easy & 10
\\
Moderate & 15
\\
Hard & 20
\\
Very Hard & 25
\\
Nigh-impossible & 30
\\ \hline
\end{rndtable}
\end{center}

It is up to the GM to decide the difficulty of the task (they do not need to reveal this to you), and they may of course deviate from this table. 

If the check succeeds, then the action goes ahead -- if not, then the action fails, and there may be negative consequences, especially if the dice roll was a 1 (`natural failure'), or if your modifiers are such that the check was negative (haha). If this happens, the GM may come up with a suitable back-firing, or you may alert enemies to your presence, and so on. 

On the other hand, many GM's accept that a check which rolls a 20 on the d20 (`nat 20'), if the action succeeds, is said to be a `critical success', and may have positive effects beyond the intended. If the check was an attack, for instance, it may be considered a critical strike (see below), and do extra damage. 


\chapter{Non-Combat Actions}

Actions can be split into two major types: combat and non-combat actions. Whilst the fundamental freeform aspect of the game remains present in both, during combat, the game necessarily becomes a bit more structured in how the moves are declared, and who gets to do what when in combat. 

When not in direct combat with the enemy, however, you have virtually free reign with what you can do. Non-combat actions tend to be a lot less strict on their turn-based nature – as there’s nobody to directly oppose you. Actions that you can undertake include casting magic, travelling, trading, creating items, and indeed, anything that you can conceive of your character doing. 


Actions such as trading, potion brewing, and enchanting can only be performed when not in combat, excepting unusual circumstances, which you should be able to justify to your GM. Some specific actions, such as resting, travelling, spell casting, potion brewing and item enchantment are covered in the following sections, everything else is left up to the game master’s discretion.

If you do not perform a spell in your turn, you character gets 2FP restored. If you are in a more ‘free form’ scenario, where moves are not rigorously kept track of, you regenerate at a rate of 2FP per minute. 

\subsection{Resting}

Resting is an important action that can only occur when not in combat. Attempts to rest during combat are highly likely to get you killed on the spot. 

When in safe territory, you may set up camp, and get a few hours shut-eye to recover from your ordeals (see the Asleep status effect for details). But be warned, the night is dark and full of terrors, and who knows what might sneak up on you whilst you are resting…

You may take rests whilst delving deep into unfriendly territory, but note that resting after every single encounter is generally frowned upon, and the GM might start throwing more and more unpleasant random encounters at you if you begin to take things to the extremes. 

You should only rest in a place where it makes sense to rest – it does not makes sense, for example, to take a quick nap in whilst delving through the dungeons of an evil warlord, even if you have cleared the immediate area of enemies. Of course, if you kill the Warlord and claim his castle as your own, then it is a different matter...

\subsection{Long-distance Movements}

The specifics of movement in combat are discussed below, this section is concerned with movement in a more general sense. 

Out of combat, wandering around the environment is very natural -- you simply tell the GM that you want to go over there, and you do (barring unforseen circumstances such as traps). The exact time taken isn't really kept track of (that would get dull), but it is presumed to occur on the scale of minutes. 

However, sometimes you might wish to travel longer distances, at which point the duration does matter. 

If you are travelling more than 1 hour, then the following table gives a variety of paces, speeds and effects:


\begin{center}
\begin{rndtable}{|c c c m {3 cm}|}
\hline
Pace & Speed & Duration & Effect
\\
\hline 
Slow & 2km/h & 8 hours & Perform FIN(stealth) check to remain hidden every 30 minutes
\\ 
Normal & 4 km/h & 7 hours & (None)
\\ 
Rapid & 6 km/h & 5 hours & -5 penalty to passive perception. 4 FP per hour.
\\ 
Breakneck & 10km/h & 1 hour & {\raggedright- 10 penalty to passive perception. 6 FP per hour. 5 HP per hour.}
\\ \hline
\end{rndtable}
\end{center}

If you attempt to travel for longer than the `duration' of the selected pace, then you aquire the `exhausted' status effect, and lose 50\% of your max HP for every time you exceed another half-duration (so at Breakneck you would lose 50\% of your max HP after 1 hour 30 minutes, at Normal, you would lose it after 10 hours 30 minutes).

This timer resets after a rest of more than 8 hours, after which time you can take up your pace again. 

\chapter{Combat Actions}

\subsection{The Combat Cycle}
Unlike most RPGs, which tend to use a turn-based system for combat, this game uses a simultaneous combat system. The reason for this is that whilst the turn-based combat fits in with how we play games (I have my turn, you have yours, etc.), it is not entirely realistic: in a fight, you don't wait patiently for everyone else to complete attacking you before finally returning fire: everybody is completing actions at once. 

After combat is initiated, a series of turn cycles occur. Each turn cycle allows every character in combat one major action, such as: a movement, casting a spell, or using an item. Before the turn is activated, there is a period of time (to be decided by your GM), during which you must decide on what you will do. Players may talk to each other during this time, but do be aware that discussing your tactics in front of the GM may give the game away, you wouldn't start shouting your plan out whilst fighting the enemy now, would you? 

After this time is up, each player writes down their action on a scrap of paper (to prevent last minute changes of heart), and then all players (including the GM) reveal their action simultaneously. 

Since all actions are considered to be simultaneous, the order in which the actions are resolved does not usually matter, recall that spells have a finite travel time, so it is entirely feasible for two players to stun each other simultaneously and it does not matter {\apos}who cast first{\apos}.

It might, of course, still be possible for actions to come into conflict with each other: if two characters attempt to occupy the same space, for example. It is up to the GM's discretion how to deal with edge cases like this - for the example given, it is recommended that this be treated as a `body slam', and both characters should recoil and take some damage. 

There might also be cases where two spells are cast simultaneously where the ordering does actually matter: for example, if you heal someone at the same time that someone casts a damaging spell that would take them below 50\% health, incurring the ``major injury{\apos \apos} status. If the healing action occurs first, then they are not taken below 50\% health, but if the damage action occurs first, then they do fall below 50\%, even if they are then brought back up over that threshold. The final health that the character ends up on might be the same, but the ordering of actions effects whether they have the {\it major injury} status at the end of the turn. 

In cases such as this it is useful to remember that it is the {\it casting} of the spell that is simultaneous: so the ordering in which the spell effects should take place can be inferred from the distance between the caster and the target. The issue above is resolved simply by looking at whoever is closest to the target. 
 
\subsection{Movement}

Moving is a very common action to take during combat, to avoid the enemy's attacks, or to maneouvre yourself to enable an attack on the enemy. Movement can be broken down into three types: minor movements, transport movements and dodges. 

Minor movements include things such as turning to face an enemy, or taking a step out from behind cover. The actions do not constitute the entirety of a turn, however \textbf{however, they are considered to happen at the very beginning of a turn cycle, and you cannot return to cover after emerging from it}. If you therefore emerge from cover to attack someone, and a character successfully guessed that this would happen and sent a spell in your direction, you will not be protected until you move back into cover in the next turn cycle. 

Transport movements are those designed to get you from point A to point B as quickly as possible. These actions do take up your entire turn: you can do nothing else except take a transport action. The distance that you can travel in a given trasnport action is calculated from:

$$ \text{metres travelled} = \frac{1\text{d}10 \text{ ATH (speed) check}}{5} $$  

This distance is rounded downwards  to the nearest half-metre, unless you are wearing ``heavy armour'' (i.e. anything more heavy than usual fabrics), in which case it is rounded downwards to the nearest integer. The direction that you are travelling in \textbf{must} be declared before performing this check. You may elect to not use all of the movement that you rolled for -- i.e. if you rolled a 1.5m movement, you may only use 1m, if you desire. If your movement check was a non-integer, and you do not continue moving in the next turn cycle, your next turn automatically uses your minor movement to move the final 0.5m and you may not use another minor movement that cycle. 

If you do elect to continue moving the next turn, you may use the `sprint' effect to continue moving at the same pace as the previous turn, without another roll. This can be continued for a maximum of 3 turns. If you are sprinting, you may not change direction -- this would require a `new movement' to be initiated. You may also simply elect to perform a new movement roll, in order to get a better value. 

 An evasion movement works exactly the same as transport movement, except that the distance roll is divided in half (rounded down to the nearest 0.5m again). This sacrifice in distance, however, allows you to attempt to evade any attacks that come your way, and negate the effects of attacks of opportunity. In addition to the movement roll, you must also immediately perform a 1d8 ATH check (or FIN, if you have the relevant skill), which denotes how skillfull your dodging was. 
Any character that attempts to cast a spell on you during this turn cycle must then perform a 1d8 EMP check to anticiapte your action. If this EMP check is less than \textbf{or equal to} your ATH check, then you succesfully dodge the spell. If not, 
then your evasion was unsuccessful, and you take the effect of the spell. 


\subsection{Other Actions}

Other actions may also be broken down into a major and a minor distinction, with major actions taking up the entirety of a turn, and minor actions being incidentals that occur before the main action of a turn cycle begins. 

Examples of minor actions could be removing an item from your bag (within reason), or crouching down. Major actions would be casting spells, equipping and using items and so on. There is, as usual, a slight grey area in what constitutes a minor action -- pulling an entire suit of armour out of your bag is clearly going to take longer than a minor action!

Some actions (i.e. putting on said suit of armour) might take a number of turn cycles to complete. You may choose to abandon the action before it is completed, but you would then need to start again from scratch to finalise it. 

Classifying these actions is up to the GM: and the GM{\apos}s word on the matter is final. 

\subsection{Conditional Actions}

The use of the simultaneous combat system raises some interesting opportunities with conditional actions, which are actions that depend on the actions that another character takes.

The actual action, as well as the condition, needs to be declared during the normal turn cycle -- but the action itself is not triggered until all other actions had been triggered. 

For example, it could be that you declare as your action \textit{if the troll attacks player A, then I cast a healing spell on player A}. You could also attempt to prevent the damage from being taken in the first place, by declaring \textit{if the troll attacks player A, then I cast the knockback charm on the troll}. The GM may ask for a check to determine if you are close enough and have fast enough reactions for your spell to interrupt the action, but if you pass this, then you may be able to save your friend. Please see below for more counterspell options.

You are only allowed a single conditional clause in your declaration, and if that conditional does not come to pass, then your character does not do anything: there is no \verb|if-then-else| in this game!

If a seemingly unbreakable condition-chain arises (i.e. player A says he will perform X if player B does Y, but player B says he will only perform Y if player A does X), it is up to the GM to resolve the conditionals -- in such cases the answer is usually \textit{nothing happens}, but there may be examples where the GM feels it is more appropriate that the action-chain is triggered. 

\subsection{Counterspells}

The only action where you can declare a conditional action without specifying exactly what that action might be is \textit{preparing a counterspell}. In this instance, you may simply decleare that you are expecting an attacking, and that you are waiting to counter that attack. 

If and when that attack comes, the GM will give you only 5 seconds to declare which spell you are going to use to counter the attack that is headed your way. If you do not declare a spell in those 5 seconds, the attack hits you as normal. However, if you do declare a spell in those 5 seconds (and the GM will choose the first spell you declare), then you may attempt to cast that spell. If the GM judges that your counterspell negates or reflects the attack that was headed your way, then you have succesfully avoided it. Otherwise, the spell continues unabated. 

If multiple attacks spells were used on you simultaneously, then it is up to the GM to decide if your counterspell affects both incoming attacks, or only one (or indeed, neither). For example, a \textit{protego} cast against two spells from the same general direction will indeed protect against both attacks. Two attacks from opposite sides, however, will not be affected by a single \textit{protego}, which may cause you problems, as you are allowed only a single counterspell per turn. 

The 5 second time limit will push you to be inventive in a split second, and you should be constantly surveying your environment to spot such situations before they arise. 

If you are preparing a counterspell, it is important that you declare this first to the GM, so that the appropriate 5 second deadline can be given. Failure to do so may result in the GM declaring your counterspell invalid!

\chapter{Stealth and Critical Strikes}

Being noticed by the enemy is generally regarded as a bad thing. It therefore often pays to be sneaky, to stay hidden from the enemy. Stealth is governed by the FIN attribute, via the Stealth proficiency. 

Every time you wish to take an action whilst remaining hidden, you will need to perform a FIN (stealth) check against the target, with the target performing an EMP(perception) check -- if the sneak check exceeds the perception check, then you remain hidden. If it fails, then the target becomes aware of you, and probably initiates combat. 

Equally, some creatures might try to sneak up on you -- but the GM can't very well ask you to perform a perception check, as you would immediately know that something was there! In order to keep the surprise, each checktype has a `passive' value, which is simply equal to the average. Hence, for a d20 check, the passive value is 10 + relevant bonuses. The GM will use this value in private to determine if beings remain hidden or not. 

The same is true for illusion spells which are cast on you without your knowledge -- a passive SPR (endurance) check is used, with the same rules as before. The GM does not need to tell you about this spell, unless you actively perform a perception check to notice something wrong with the world. 

If you willingly choose to perform a perception check, this gets a +2 bonus. In combat, this would count as your major action. 

If you initiate combat whilst undetected (or have it initiated against you by an unseen opponent), then the attacked party must continue to attempt to percieve the enemy, until they can attack them in the usual way. You may attempt to wildly attack the enemy -- throwing a fireball {\it near} them is probably going to hurt, even if you don't know exactly where they are, but this might be a waste of resources. 

\section{Sneak Attacks \& Attacks of Opportunity}

If you perform an attack on someone who is not aware that you are attacking them, or if you perform an attack on someone who has their mind elsewhere, then you have an opportunity to do large amounts of damage to the unwary target. 

A sneak attack is triggered when a character attacks another when they are not expecting it -- be it attacking someone who is not even aware that you pose a danger to them, or if you have snuck up behind an enemy whilst they are attacking someone else -- if they don't see an attack coming, you get an opportunity to surprise them!

An attack of opportunity is triggered when somebody is aware that they are in combat, but is doing something that opens them up to attack. For example, if someone was in close-quarters range and they attempt to cast a spell on you, you can quickly stab them with a knife, and there is nothing they could do about it. Equally, if they attempt to cast a spell on someone else, then their attention is not on you. If you had already commited to an attack on them, then it has a chance to be much more effective.

Whichever method is triggered, the effect is the same: you roll any (even-numbered) dice. If the result is an even number, then you multiply the damage by 2. If it is odd, then you just do the normal amount of damage\footnote{This assumes that the {\it catastrophic critical} is not in use -- if it is, use the rules detailed in that skill}. 

Critical attacks (i.e. triggered by a nat20, or otherwise through a skill) are mechanically identical to an Attack of opportunity. 

If you perform a critical {\it during} a Critical attack, then you do get to use two multipliers, but they are {\it added}. For example, a critical-opportunity attack would roll two dice, and use the following table to determine the dice:

\begin{center}
\begin{rndtable}{|c c c|}
\hline
~ & odd & even
\\ \hline 
\cellcolor{\tablecolorhead }odd & 2 & 3
\\ \hline
\cellcolor{\tablecolorhead}even & 3  & 4
\\ \hline
\end{rndtable}
\end{center}
