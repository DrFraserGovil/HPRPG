

\chapter*{Outlaw}
\addcontentsline{toc}{section}{Outlaw}


Outlaw intro

%%archBegin
\archetype{name=Outlaw, hp=8, fp=8, armour=Light Armour, tool=Lockpicking Tools, disc=Choose any two from Telepathy\comma{} Temporal\comma{} Kinesis\comma{} Bewitchment or Psionics, weapon=Simple Weapons\comma{} Simple Ranged Weapons, prof=Choose four from  Speed\comma{} Acrobatics\comma{} Chicanery\comma{} Stealth\comma{} Deception\comma{} Persuasion\comma{} Performance\comma{} Investigation and Observation., equip=A Wand\comma{} a Thief Pack\comma{} and either a) a dagger or  b) a sling., memorised=Any three from the basic spells table., listIName =Surprises, singleListMode = 1, listIIName = Surprise Attack, doubleListMode = 1, expertI = 2, maxspellI = Beginner, bonusI = Element of Surprise, listI_I= 2, listII_I= 1d8, expertII = 2, maxspellII = Beginner, bonusII = Rudimentary Recovery\comma{} Sly Action, listI_II= 3, listII_II= 1d8, expertIII = 2, maxspellIII = Beginner, bonusIII = Career Choice\comma{} Career Feature I, listI_III= 4, listII_III= 2d8, expertIV = 3, maxspellIV = Beginner, bonusIV = Artisan of the Unlawful I, listI_IV= 5, listII_IV= 2d8, expertV = 3, maxspellV = Novice, listI_V= 6, listII_V= 3d8, expertVI = 3, maxspellVI = Novice, bonusVI = Career Feature II, listI_VI= 7, listII_VI= 3d8, expertVII = 3, maxspellVII = Novice, listI_VII= 8, listII_VII= 4d8, expertVIII = 4, maxspellVIII = Novice, bonusVIII = Artisan of the Unlawful II, listI_VIII= 9, listII_VIII= 4d8, expertIX = 4, maxspellIX = Novice, bonusIX = Slippery, listI_IX= 10, listII_IX= 5d8, expertX = 4, maxspellX = Adept, bonusX = Career Feature III, listI_X= 10, listII_X= 5d8, expertXI = 4, maxspellXI = Adept, listI_XI= 10, listII_XI= 6d8, expertXII = 5, maxspellXII = Adept, bonusXII = Danger Sense, listI_XII= 12, listII_XII= 6d8, expertXIII = 5, maxspellXIII = Adept, listI_XIII= 12, listII_XIII= 7d8, expertXIV = 5, maxspellXIV = Adept, bonusXIV = Artisan of the Unlawful III, listI_XIV= 12, listII_XIV= 7d8, expertXV = 5, maxspellXV = Master, listI_XV= 15, listII_XV= 8d8, expertXVI = 6, maxspellXVI = Master, bonusXVI = Career Feature IV, listI_XVI= 15, listII_XVI= 8d8, expertXVII = 6, maxspellXVII = Master, listI_XVII= 15, listII_XVII= 9d8, expertXVIII = 6, maxspellXVIII = Master, bonusXVIII = Incredible Luck, listI_XVIII= 20, listII_XVIII= 9d8, expertXIX = 6, maxspellXIX = Master, listI_XIX= 20, listII_XIX= 10d8, expertXX = 7, maxspellXX = Ascendant, bonusXX = Career Feature V, listI_XX= 20, listII_XX= 10d8, shortmode = 0}
%%archEnd


\section*{Acquired Feats}

\feat{Element of Surprise}

At first level, you learn the primary tenet that every outlaw and rogue lives by: {\it never be predictable}. If you're predictable, you get caught, and in your line of work, if you get caught, you're dead.

To that end, you keep a number of tricks up your sleeve \minus{} your {\it surprises}. Every night when you take a long rest, you can prepare a number of surprises \minus{} the amount increases with your Outlaw level, as indicated in the Archetype table. 

At any point, you may expend one of your Surprises to reveal that you had planned for this all along: you may then choose from the list of surprises at the end of the class description. 

All Outlaws have access to the following Surprises: {\it Change of Clothes}, {\it Distraction}, {\it Hidden Knife}, {\it Secret Pockets}, {\it Surprise Attack}, {\it Unexpected Talent} and {\it Shift Weight}. You may gain access to additional surprises at higher levels. 

During a combat cycle, you may declare any number of surprises to use \minus{} limited only by the action requirements and any stipulations of the surprise itself. 

Many Surprises require that your opponent perform a Resist check. Unless otherwise specified, the DV of this check is set by your {\it Sneak Subjugation}, which is calculated from:
$$ \text{Sneak Subj.} = 8~ +~ \text{expertise bonus + \attFin{} modifier }$$ 

\feat{Rudimentary Recovery}

From 2nd level, you have learned to cobble together some additional surprises from what you have on hand. 

Upon finishing a short rest, you may choose to recover your surprises. You recover a number equal to half your maximum value (rounded up). You cannot use this feature to exceed your maximum number of surprises.  


\feat{Sly Action}{On a turn in which no beings attacked, or otherwise interacted with you, you may take an additional minor action to consume an item, move, or attempt to hide. }

\feat{Career Choice}

At 3rd level,you decide in which direction you wish to devote your energies to: you may choose to be either a Thief or an Assassin. 

Your choice of Career gives you features at 3rd, 6th, 10th, 16th and 20th levels. These are detailed after the Acquired Feats section of the class description. 

\feat{Artisan of the Unlawful}

At 4th level, and then again at 8th and 14th level, your lifetime of skullduggery enables you to learn new skills, and improve ones you already knew. 

Every time you take this feature, you may choose from the following options:
\begin{itemize}
	\item Choose 2 new proficiencies: either those associated with attributes, spellcasting, weapons or tools. You are considered proficient in these areas. 
	\item Choose 2 areas (attribute proficiencies or tools), in which you are proficient. You may double your proficiency bonus when undertaking these actions. 
\end{itemize}

\feat{Slippery}

From 9th level, you have a knack for getting out of tight spots. 

You take check advantage on any checks made to break grappes, or escape from or Resist the {\it Trapped} status effect. 

\feat{Danger Sense}

From 12th level, your senses have become accustomed to your shady lifestyle, and you have developed a 6th sense for when things are about to go wrong. 

You cannot be surprised, and sneak attacks which would normally trigger a critical strike function as normal attacks against you. 

\feat{Incredible Luck}

From 17th level, you are able to use your uncanny reflexes and years of training to save what would otherwise be a terrible failure. 

On a failed Resist, Accuracy or Attribute check, you may instead choose to have rolled a 20 (this does not trigger a critical strike, however). 

You can only use this ability once per long rest. 

\section*{Careers}
\subsection*{\bf Assassin}

\subfeat{Additional Surprises}

As an Assassin, you gain access to the following additional surprises:

{\it Poisoned Blade}

\subsection*{\bf Thief}


\subfeat{Pickpocket}

From 3rd level, you gain the ability to...borrow...items from their current owner, by stealing them directly off their person. 

Perform a Chicanery check on a being within melee range, contested with the current owner's Observation value (passive or active, depending if they are expecting you!). 

If the check succeeds, you may steal an object from their pockets or backpack - providing it is reasonably accessible. Stealing equipped items, such as a sword sheathed at their hip, is particularly difficult, and you take disadvantage on pickpocketing attempts such as this which are overly ambitious.   


\subfeat{Additional Surprises}

As a Thief, you gain access to the following additional surprises:

{\it Bag of Sand}, {\it Makeshift Tools}

\subfeat{Thief of the Mind}

From 20th level, you gain the ultimate larcenous technique: the ability to steal ideas from a target's mind.

Whenever a being targets you with a spell, you may allow the spell to effect you (either automatically hitting, or failing the Resist). In return, you force the spellcaster to perform an Willpower Resist check, against your normal Arcane Subjugation value. 

On a failed resist, you can wrestle the knowledge of this spell from their mind: for the next hour, the spellcaster cannot use that spell. If the spell belongs to a discipline in which you are proficient, you may cast it yourself as if you had memorised it.  


\section*{Surprises}
\newcommand\surprise[3]
{
{\setlength\parskip{5 pt}
{\large \textbf{\textit{#1}}}: #2


This surprise can be used by {#3}. 

}
}
%%SurpBegin
\surprise{Bag of Sand}{When performing a pickpocketing check\comma{} or otherwise attempting to steal something \minus{} you may expend a surprise to replace it with an object approximately the same size and weight\comma{} to prevent its absence from being noted. Beings take check\minus{}disadvantage on perception checks to notice your thievery.}{Thieves}
\surprise{Change of Clothes}{Maybe a simple reversible cloak\comma{} and a fake pair of glasses \minus{} or something as complex as a glamour which falls away at your command. You may expend a surprise to drastically alter your appearance. Only those who got a good look at your face are able to identify you as the same person.}{all Outlaws}
\surprise{Distraction}{At the start of your turn\comma{} you may use a surprise to distract your opponents from your true intentions. You can focus this either on an individual target  \minus{} in which case they take the {\it Distracted} status effect on a failed Observation Resist\comma{} or on yourself \minus{} in which case\comma{} you get check\minus{}advantage on a Stealth check you make this turn.}{all Outlaws}
\surprise{Hidden Knife}{As a minor action\comma{} you may expend one of your surprises to draw a secret dagger from a fold in your clothing as a minor action and either equip it\comma{} or convert this into a full\minus{}turn action and immediately make a normal melee or ranged attack using this knife.}{all Outlaws}
\surprise{Makeshift Tools}{You demonstrate a remarkable knack for improvisation. If you break your lockpicking tools\comma{} or find yourself in need of some specialist tools\comma{} you can use a surprise to cobble together a set.}{Thieves}
\surprise{Poisoned Blade}{When you land a strike on a target\comma{} you may use a surprise to reveal that the blade was coated in a deadly toxin. The victim takes poison damage equal to your Surprise Attack damage\comma{} and on a failed Vitality Resist takes the {\it Poisoned: Mild} status effect.}{Assasins}
\surprise{Secret Pockets}{You may expend a surprise to reveal a secret compartment\comma{} hidden about your person. You may store a (relatively) small item in this pocket\comma{} where it cannot be discovered except by a DV20 Investigation check.}{all Outlaws}
\surprise{Surprise Attack}{Whenever you land an attack on an opponent\comma{} you may use a surprise to twist the dagger a bit\comma{} sneak in an extra punch to the kidneys\comma{} or follow a hex with a secondary strike. In addition to the normal damage roll\comma{} you may add your Surprise Attack damage\comma{} which increases with your Outlaw level\comma{} as shown in the Archetype table. 

You may only use one Surprise Attack per combat cycle.}{all Outlaws}
\surprise{Unexpected Talent}{Whenever performing an attribute check\comma{} you may surprise everyone by revealing a hidden talent. You may expend a surprise to add your proficiency bonus to a check in an area you are not normally proficient in. You must expend the surprise {\it before} the check is performed.}{all Outlaws}
\surprise{Shift Weight}{When an enemy attempts to grapple you\comma{} when you successfully escape you may expend a surprise to use their own power against them: perform an additional attack against the target\comma{} using their Fitness modifier\comma{} rather than your own.}{all Outlaws}

%%SurpEnd 
