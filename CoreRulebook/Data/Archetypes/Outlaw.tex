

\chapter*{Outlaw}
\addcontentsline{toc}{section}{Outlaw}


Outlaw intro

%%archBegin
\archetype{name=Outlaw, hp=8, fp=8, armour=Light Armour, tool=Lockpicking Tools, disc=Choose any two from Telepathy\comma{} Temporal\comma{} Kinesis\comma{} Bewitchment or Psionics, weapon=Simple Weapons\comma{} Simple Ranged Weapons, prof=Choose five from  Speed\comma{} Acrobatics\comma{} Chicanery\comma{} Stealth\comma{} Deception\comma{} Persuasion\comma{} Performance\comma{} Investigation and Observation., equip=A Wand\comma{} a Thief Pack\comma{} and either a) a dagger or  b) a sling., memorised=Any three from the basic spells table., listIName =Surprises, singleListMode = 1, listIIName = Surprise Attack, doubleListMode = 1, expertI = 2, maxspellI = Beginner, bonusI = Element of Surprise, listI_I= 2, listII_I= 1d8, expertII = 2, maxspellII = Beginner, bonusII = Rudimentary Recovery, listI_II= 3, listII_II= 1d8, expertIII = 2, maxspellIII = Beginner, bonusIII = Career Choice, listI_III= 4, listII_III= 2d8, expertIV = 3, maxspellIV = Novice, listI_IV= 5, listII_IV= 2d8, expertV = 3, maxspellV = Novice, listI_V= 6, listII_V= 3d8, expertVI = 3, maxspellVI = Novice, listI_VI= 7, listII_VI= 3d8, expertVII = 3, maxspellVII = Novice, listI_VII= 8, listII_VII= 4d8, expertVIII = 4, maxspellVIII = Adept, listI_VIII= 9, listII_VIII= 4d8, expertIX = 4, maxspellIX = Adept, listI_IX= 10, listII_IX= 5d8, expertX = 4, maxspellX = Adept, listI_X= 10, listII_X= 5d8, expertXI = 4, maxspellXI = Expert, listI_XI= 10, listII_XI= 6d8, expertXII = 5, maxspellXII = Expert, listI_XII= 12, listII_XII= 6d8, expertXIII = 5, maxspellXIII = Expert, listI_XIII= 12, listII_XIII= 7d8, expertXIV = 5, maxspellXIV = Master, listI_XIV= 12, listII_XIV= 7d8, expertXV = 5, maxspellXV = Master, listI_XV= 15, listII_XV= 8d8, expertXVI = 6, maxspellXVI = Master, listI_XVI= 15, listII_XVI= 8d8, expertXVII = 6, maxspellXVII = Master, listI_XVII= 15, listII_XVII= 9d8, expertXVIII = 6, maxspellXVIII = Ascendant, listI_XVIII= 20, listII_XVIII= 9d8, expertXIX = 6, maxspellXIX = Ascendant, listI_XIX= 20, listII_XIX= 10d8, expertXX = 7, maxspellXX = Ascendant, bonusXX = Thief of Knowledge (thief), listI_XX= 20, listII_XX= 10d8, shortmode = 0}
%%archEnd


\section*{Acquired Feats}

\feat{Element of Surprise}

At first level, you learn the primary tenet that every outlaw and rogue lives by: {\it never be predictable}. If you're predictable, you get caught, and in your line of work, if you get caught, you're dead.

To that end, you keep a number of tricks up your sleeve \minus{} your {\it surprises}. Every night when you take a long rest, you can prepare a number of surprises \minus{} the amount increases with your Outlaw level, as indicated in the Archetype table. 

At any point, you may expend one of your Surprises to reveal that you had planned for this all along: you may then choose from the list of surprises at the end of the class description. 

All Outlaws have access to the following Surprises: {\it Change of Clothes}, {\it Distraction}, {\it Hidden Knife}, {\it Secret Pockets}, {\it Surprise Attack}, {\it Unexpected Talent} and {\it Shift Weight}. You may gain access to additional surprises at higher levels. 

During a combat cycle, you may declare any number of surprises to use \minus{} limited only by the action requirements and any stipulations of the surprise itself. 


\feat{Rudimentary Recovery}

When taking a short rest, you may choose to 

\section*{Surprises}
\newcommand\surprise[3]
{
{\setlength\parskip{5 pt}
{\large \textbf{\textit{#1}}}: #2


This surprise can be used by {#3}. 

}
}
%%SurpBegin
\surprise{Bag of Sand}{When performing a pickpocketing check\comma{} or otherwise attempting to steal something \minus{} you may expend a surprise to replace it with an object approximately the same size and weight\comma{} to prevent its absence from being noted. Beings take check\minus{}disadvantage on perception checks to notice your thievery.}{Thieves}
\surprise{Change of Clothes}{Maybe a simple reversible cloak\comma{} and a fake pair of glasses \minus{} or something as complex as a glamour which falls away at your command. You may expend a surprise to drastically alter your appearance. Only those who got a good look at your face are able to identify you as the same person.}{all Outlaws}
\surprise{Distraction}{At the start of your turn\comma{} you may use a surprise to distract your opponents from your true intentions. You can focus this either on an individual target  \minus{} in which case they take the {\it Distracted} status effect on a failed Observation Resist\comma{} or on yourself \minus{} in which case\comma{} you get check\minus{}advantage on a Stealth check you make this turn.}{all Outlaws}
\surprise{Hidden Knife}{As a minor action\comma{} you may expend one of your surprises to draw a secret dagger from a fold in your clothing as a minor action and either equip it\comma{} or convert this into a full\minus{}turn action and immediately make a normal melee or ranged attack using this knife.}{all Outlaws}
\surprise{Makeshift Tools}{You demonstrate a remarkable knack for improvisation. If you break your lockpicking tools\comma{} or find yourself in need of some specialist tools\comma{} you can use a surprise to cobble together a set.}{Thieves}
\surprise{Poisoned Blade}{When you land a strike on a target\comma{} you may use a surprise to reveal that the blade was coated in a deadly toxin. The victim takes poison damage equal to your Surprise Attack damage\comma{} and on a failed Vitality Resist takes the {\it Poisoned: Mild} status effect.}{Assasins}
\surprise{Secret Pockets}{You may expend a surprise to reveal a secret compartment\comma{} hidden about your person. You may store a (relatively) small item in this pocket\comma{} where it cannot be discovered except by a DV20 Investigation check.}{all Outlaws}
\surprise{Surprise Attack}{Whenever you land an attack on an opponent\comma{} you may use a surprise to twist the dagger a bit\comma{} sneak in an extra punch to the kidneys\comma{} or follow a hex with a secondary strike. In addition to the normal damage roll\comma{} you may add your Surprise Attack damage\comma{} which increases with your Outlaw level\comma{} as shown in the Archetype table. 

You may only use one Surprise Attack per combat cycle.}{all Outlaws}
\surprise{Unexpected Talent}{Whenever performing an attribute check\comma{} you may surprise everyone by revealing a hidden talent. You may expend a surprise to add your proficiency bonus to a check in an area you are not normally proficient in. You must expend the surprise {\it before} the check is performed.}{all Outlaws}
\surprise{Shift Weight}{When an enemy attempts to grapple you\comma{} when you successfully escape you may expend a surprise to use their own power against them: perform an additional attack against the target\comma{} using their Fitness modifier\comma{} rather than your own.}{all Outlaws}

%%SurpEnd 
