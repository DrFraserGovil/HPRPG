\documentclass[CoreRulebook.tex]{subfile}
\cleardoublepage
\begin{strip}
\setlength{\parskip}{4pt}

\section{Auror}

Her wand at the ready, poised for action, the Auror peers around the corner, looking for her targets. She has tracked these dangerous criminals halfway across the country, trying to protect the innocent from their evil goals. With a leap, she emerges from cover and with a series of casts quicker than the eye can see, manages to incapacitate her foes. The world is just a little bit safer thaks to her hard work. 

Aurors are the highly trained combat operations arm of the Office for Magical Law Enforcement. Their job is to track down and eliminate the threat posed by dark wizards and wizards, and to protect those who would otherwise come to harm. Experts in combat magic, Aurors are not to be trifled with. 

The Auror office recognises two streams of officers, the {\bf Enforcers} and the {\bf Warders}. The Enforcers form the strike capabalities of the office, specialising in infiltration, target acquisition and shock tactics, whilst the Warders specialise in area denial, and collateral limitations.
. 

%%archBegin
\archetype{Auror}{Enforcer}{Warder}{0}{featureI=Combat Training, featureII=Spellcasting Improvement, alphaFeatureIII=Intimidating Style, betaFeatureIII=Multiward, featureIV=Defence Against the Dark Arts, betaFeatureV=Runes, featureVI=Seasoned Investigator, alphaFeatureVI=Mage Slayer I, featureVII=Spellcasting Improvement, alphaFeatureVII=Ethereal Manacles, featureVIII=Combat Training II, betaFeatureVIII=Trap Expertise, alphaFeatureIX=Elegant Avoidance, betaFeatureIX=Wardbreaker, featureX=Run \apos{}n Gun, alphaFeatureXI=Fast casting, betaFeatureXI=Runes II, featureXII=Spellcasting Improvement, betaFeatureXII=Collateral Limitation, alphaFeatureXIII=Intimidating Style II, featureXV=Incredible Resilience, alphaFeatureXV=Elegant Avoidance II, betaFeatureXV=Runes III, featureXVII=Spellcasting Improvement, alphaFeatureXVII=Mage Slayer II, betaFeatureXVIII=Regenerative Shields, alphaFeatureXX=Merciless Strike, betaFeatureXX=Runes IV}%%archEnd

\end{strip}

\subsection{Starting Equipment}
\begin{itemize}[itemsep=0em]
	\item Combat Robes
	\item Wand (roll on Wand table to determine composition)
	\item 2x HP + 10 potions
	\item 4d6 $\times 5$ gold
	\item Obsidian Manacles
\end{itemize}
\subsection{Starting Spells}

Aurors may choose 2 spells from the basic spells table, and 3 spells from the following:
{\it
\begin{itemize}[itemsep=0em]
	\item Acidic Burst
	\item Confundus Charm
	\item Fire-starting spell
	\item Shielding charm
	\item Stoneskin
	\item Privacy Ward
	\item Trap Spell
	\item Glamour Charm
\end{itemize}
}
\newpage
\subsection{Archetype Features}

\feat{Combat Training}

From 1st level, your combat training allows you to re-roll the dice on any check, once per combat engagement. At 8th level, you may use this feature twice per engagement. 
\jump
\feat{Spellcasting Improvement}

At 2nd level, and then again at 7th, 12th and 17th level, you may increase the size of the dice you use to cast Hexes \& Curses {\bf or} Recuperative spells. This feat does {\it not} count when calculating the Arcane Wisdom bonuses detailed on page \pageref{S:Auto}. 
\jump
\feat{Defence Against the Dark Arts}

From 4th level, take check-advantage when performing a resist check against any Dark Arts spells cast by a wizard with an EVL less than or equal to your Auror level.

\jump
\feat{Seasoned Investigator}

From 6th level, gain a +2 bonus to Research checks. 

\jump
\feat{Run \apos{}n Gun}

From 10th level, you may ignore the dice-rolling cap when performing quickspells. The limitations on modifiers still applies. Does not apply when using the Elegant Avoidance casting feature. 
\jump
\feat{Incredible Resilience}

 From 15th level, if you pass a SPR(willpower) check (DV 30, minus 1 for each Auror level), you may ignore the resitriction on immobility from the {\it Critical Condition} and {\it Critical But Stable} conditions. 

\subsection{Enforcer Features}
\jump
\feat{Intimidating Style}

From 3rd level, gain a bonus to your Intimidation proficiency equal to one 1 + one quarter of your Enforcer level.

From 13th level, your presence is intimidating that if you are the instigator of a conflict, at the beginning of the battle, all enemies must perform a SPR(Endurance) Resist check (DV = set by a d20 POW(Intimidation) check), or take the Terrified status. 

\jump
\feat{Mage Slayer}

From 6th level, gain check advantage when casting a spell against another target performing a concentration-spell. 

At 17th level, gain check double-advantage. 

\jump
\feat{Ethereal Manacles}

From 7th level, if you are within melee range of a target and have not taken damage for 1 turn, take 1 major action to conjure a pair of locked, magical restraints around the target\apos{}s wrists. Target may resist with an SPR(arcane) Resist check (DV = Auror level) during the casting, and then subsequently may try a ATH(strength) check (DV = 18) once per cycle to break them. 

\jump
\feat{Elegant Avoidance}

From 9th level, you may use either your ATH(speed) {\bf or} your FIN(precision) skill to perform an evasion check. 

From 15th level, you may cast a quickspell whilst performing an evasive movement. You may take a three-minor-action turn (considered movement, quickspell and evasion). However, you may not apply the {\it Run \apos{}n Gun} feat to the quickspell.
\jump

\feat{Fast Casting}

From 11th level, you may cast two spells as part of your major Spellcasting action. If the first casting check fails, the second one also fails automatically (and you must therefore deduct 4FP in total).
\jump
\feat{Merciless Strike}

From 20th level, where possible, you may use a dice one larger than the prescribed one when performing damage checks. 

\jump
\jump
\subsection{Warder Features}

\feat{Multiward}

From 3rd level onwards, you may have a number of wards equal to 1 + third of your Warder level active in any given area. 
\jump
\feat{Runes}

From 5th level, you are able to recreate the basic magical runes. By painting the runes on a surface and infusing them with magical energy, you may invoke powerful ancient magic. Painting a rune takes 1 minute (5 combat rounds), though not necessarily conecutively, and requires a surface of 30cm in diameter. You may paint the rune using any material as long as
it is reasonable that it adheres to the surface. Runes are activated immediately after you complete them. Most runes lose their power after they have been triggered, unless otherwise specified. 

At 11th, 15th and 20th levels, you gain access to more powerful runes: the complex, mystifying and legendary runes respectively.


The basic runes are:
\begin{itemize}
	\item \textbf{\textit{Rune of Illusion:}} project a basic illusion onto the surface around the rune. The artist may shape the illusion to an extent, but detail is limited to basic textures and colours. Maximum area is 3m$^2$. Rune deactivates on contact with the illusion. 

	\item \textbf{\textit{Rune of Trapping:}} the next being to touch the rune must pass an ATH(Strength) Resist check (DV 14) or be paralysed for 1 turn.
 
	\item \textbf{\textit{Rune of Protection:}} when touched, casts {\it Lesser Ward} spell in a 2m radius. 
	
	\item \textbf{\textit{Rune of Blinding Light:}} when touched, casts the {\it Blinding Light} spell on all beings in a 2m radius. 
\end{itemize}

\newpage
The complex runes are:

 \begin{itemize}
 	\item \textbf{\textit{Rune of Detonation:}} the next being to touch the rune triggers an explosion which does 3d8 concussive damage to all targets in a 2m radius, and 1d8 concussive damage to all targets in a 5m radius. 
 	 \item \textbf{\textit{Rune of Suggestion:}} the next target to touch the rune has the {\it Suggestion} spell cast on them (DV 15). The Warder decides on the suggestion at the point of inscription. 
 	  \item \textbf{\textit{Rune of Amnesia:}} the next target to touch the rune must pass an INT(history) Resist, or suffer total amnesia for 2 minutes. 
 \end{itemize}
 The mystifying runes are:
 
 \begin{itemize}
	 \item \textbf{\textit{Rune of Crippling:}} the next target to touch the rune must take check-double-disadvantage on all checks for 1d4 hours. 
 	
 	\item \textbf{\textit{Rune of Transmutation:}} if the next person to touch the rune fails a POW(Endurance) Resist check, they are turned into a random non-magical beast.
 	
 	\item \textbf{\textit{Rune of Transportation:}} (requires a linked pair of runes) when a target touches one rune, they are instantly transported to the other. This rune is permanent.
 \end{itemize}
 The legendary runes are:
 
 \begin{itemize}
	 	\item \textbf{\textit{Rune of Insanity:}} when sapient being other than the inscriber views this rune, they must succeed an INT(Endurance) Resist check (DV 15) or take 6d10 psychic damage. This rune  is permanent but gradually loses power, decreasing to 4d6 damage after 1 day, 4d6 after one week, and then 2d6 after one year. 
	 	\item \textbf{\textit{Rune of Death:}} if a living being touches this rune for more than 1 minute, they must succeed an EMP(Perception) Resist (DV 14), or die. 
\item \textbf{\textit{Rune of Immortality:}} when touched, creates a warded area 5m in radius, in which it is impossible to die. You can, however, still be harmed. 
\item \textbf{\textit{Rune of the Cosmos:}} when triggered, casts the {\it Planemeld} spell to a realm of your choice. 
 \end{itemize}
\normalsize

The probability of a rune triggering when an attempt to remove it is made is found below:

\begin{center}
\begin{rndtable}{c c}
	Runes	&	Trigger Probability
	\\
	Basic	& 25\%
	\\
	Complex	&	50\%
	\\
	Mystifying	&	75\%
	\\
	Legendary & 	100\%
\end{rndtable}
\end{center}

\jump
\feat{Trap Expertise}

By 8th level, you have accumulated enough knowledge to be considered an expert in trapmaking. Checks whilst laying and looking for traps gain check-advantage. 

\jump
\feat{Wardbreaker}


From 9th level, your knowledge of wards allows you to identify their weak points. When damage is absorbed by a ward or magical AC, add half of your Warder level to the damage calculation. If the ward or shield fails, this extra damage does not affect subsequent damage calculations.  
\jump
\feat{Collateral Limitation}

From 12th level, you may spend 3 turns to cast a ward which compels civilians and non-combatants to remove themselves from the combat area, as if you had cast the {\it Beguiling Totem spell} with a casting check equal to your Warder level and 5 Power Points. 

\jump
\feat{Regenerative Wards}

From 18th level, any wards you have cast regenerate automatically, as if you were casting a permanent {\it Reinforcement Charm} on them. 
