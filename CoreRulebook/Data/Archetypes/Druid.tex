
\chapter*{Druid}
\addcontentsline{toc}{section}{Druid}
An intro bit of text 

%%archBegin
\archetype{name=Druid, hp=8, fp=10, armour=None, tool=Choose 1 from Herbology Tools\comma{} First Aid Kit\comma{} Alchemy Gear or a Musical Instrument, disc=Choose 3 from Elemental\comma{} Telepathy\comma{} Bewitchment\comma{} Healing\comma{} Warding and Alteration., weapon=None, prof=Nature and choose 2 from Speed\comma{} Strength\comma{} Vitality\comma{} Stealth\comma{} Willpower\comma{} Persuasion\comma{} Arcane\comma{} Logic\comma{} Unnature\comma{} Investigation or Observation., equip=1 weapon of your choice, memorised=3 spells from the basic spells table, expertI = 2, maxspellI = Beginner, bonusI = Nature Affinity\comma{} Druidic Aspect\comma{} Aspect Feature I, expertII = 2, maxspellII = Beginner, bonusII = Nature Senses, expertIII = 2, maxspellIII = Beginner, expertIV = 3, maxspellIV = Novice, bonusIV = Aspect Feature II, expertV = 3, maxspellV = Novice, bonusV = Reclaim Nature, expertVI = 3, maxspellVI = Novice, bonusVI = Organic Repose, expertVII = 3, maxspellVII = Novice, bonusVII = Aspect Feature III, expertVIII = 4, maxspellVIII = Adept, expertIX = 4, maxspellIX = Adept, bonusIX = Nature Affinity II, expertX = 4, maxspellX = Adept, bonusX = Nature\apos{}s Shrine, expertXI = 4, maxspellXI = Expert, bonusXI = Aspect Feature IV, expertXII = 5, maxspellXII = Expert, expertXIII = 5, maxspellXIII = Expert, bonusXIII = Ancient Powers, expertXIV = 5, maxspellXIV = Expert, expertXV = 5, maxspellXV = Master, bonusXV = Aspect Feature V, expertXVI = 6, maxspellXVI = Master, expertXVII = 6, maxspellXVII = Master, bonusXVII = Nature Affinity III, expertXVIII = 6, maxspellXVIII = Master, expertXIX = 6, maxspellXIX = Ascendant, expertXX = 7, maxspellXX = Ascendant, bonusXX = Wrath of the Wild, shortmode = 0}
%%archEnd


\section*{Acquired Feats}


\feat{Druidic Aspect}
{
	When a Druid first feels the call of the wild, the urge to protect and guide nature, they embark on a profound spiritual journey, during which they bind their soul to one of four ephemeral aspects of nature. 
	
	You may choose to follow the Tree\minus{}Loving Dryads, the Elemental Nymphs, the Beastial Satyrs or the Star\minus{}Bound Asteria. 
	
	Your choice of Druidic Aspect grants you additional abilities at 1st, 4th, 7th, 11th and 15th level. These, along with more details about the Aspects, can be found at the end of the Archetype. 
}


\feat{Nature Affinity}
{
	From 1st level, a Druid learns a deep respect for all things natural \minus{} and where possible, attempts to eschew artificial products. 
	
	Druids may only use their proficiency bonus on weapons or armour made from natural products \minus{} wood, leather and so on, but not smelted iron or synthetic fibres. In return, they are automatically proficient in all such items. 
	
	In addition, at 9th level, and again at 17th level, your affinity towards such weapons increases, allowing you to make an additional strike whenever you perform a full turn attack with a weapon meeting this criteria.
}




\feat{Nature Senses}
{
From 2nd level, a Druid's senses are deeply intertwined with their fundamental connection to nature. 

Whilst in contact with some significant aspect of the natural world \minus{} feet on untouched earth, or placing a hand on a mighty oak \minus{} a Druid gains advantage on all Observation and Investigation checks. 
}

\feat{Reclaim Nature}
{
	At 4th level, you gain the ability to imbue artificial objects with a semblance of the organic, natural energy they once possessed. 
	
	Upon touching an object manufactured from some visible natural substance, such as a wooden table, a stone statue, or a leather jacket you may imbue it with magical energy, and encourage it to retake its natural form in some way. The table might put down roots and begin to grow, whilst the statue might accumulate dust until it resembles a boulder and the leather jacket could sprout mushrooms and rot away. 

	This ability cannot bring about instantaneous, drastic change, but merely allows you to guide objects back into the natural cycle of being. 
}

\feat{Organic Repose}
{
	At 6th level, when you take a short rest in a natural area, gain additional Recovery Dice equal to one\minus{}third your Druid level. 
}

\feat{Nature\apos{}s Shrine}
{
	At 10th level, your connection to nature becomes powerful enough to imbue an entire area with natural, organic energies.
	
	By expending 3 hours work, you can create a small shrine in a natural place – maybe a stone circle, or a clearing in a wood, decorated with standing stones, feathers, crystal clear pools of water, and other such totems of nature. This shrine covers an area up to 3 metres in radius. 

	Beings which take a Long Rest inside the Shrine find that they lose an additional level of exhaustion to normal, and if they had no exhaustion to begin with, they wake with the {\it Calm Mind} status effect. 
	
	You remain aware of any being entering any Shrine you have created, as if they had a {\it Caterwauling Ward} placed upon them.
}

\feat{Ancient Powers}
{
	At 13th level, you gain learn to draw hidden magical powers from the natural corners of the world. 
	
	Choose three spells of Adept level or below from your list of memorised spells. Twice per short rest you may choose one of these spells to cast as an instantaneous, wandless and silent action.
	
	Whenever you perform a long rest you may exchange one of your selected spells for another.  
}

\feat{Wrath of the Wild}
{
	At 20th level, you can imbue your attacks with the primal fury of the wild, the infinite fury and hunger present in the cycle of life and death. 
	
	Whenever you perform an attack with a physical weapon, or a spell which targets only a single individual, you may imbue your strike with a {\it Fury} spell, even if you do not know how to cast it. The target must succeed on a Compassion Resist against your Arcane Subjugation, or become {\it enraged}.
}

\section*{Druidic Aspects}

From 1st level, all Druids must choose a Druidic Aspect to embody. 


\subsection{Aspect of the Asteria}

The Asteria are spirits of the moon and the stars, guardians of knowledge and fate, past and future. Ancient beyond reckoning, their minds have left behind the shackles of time and space. Those who follow their teachings aim to see more and farther than any living being. 

\subfeat{Channel Abilities}
{
	From first level, you learn to tap into the universal web of knowledge and understanding that the Asteria possess, and allow this unfathomable knowledge to guide you in your actions. 
	
	By spending 5 minutes in mediation, you may choose any 2 abilities, spell disciplines, tools, weapons or armour to be considered proficient in for the next hour. You must perform a Short Rest before you can use this ability again. 
}

\subfeat{Divination Prodigy}
{
	From 4th level, the power and knowledge you gain from the Asteria allows you to cast Divination spells with ease. 
	
	You may double your Expertise bonus on spellcasting and accuracy checks associated with spells from the Divination school. In addition, whenever you cast a Divination spell, roll a 1d6. If the value of this roll exceeds the level of the spell you cast, the spell costs only half the usual FP. 
}
\subfeat{Control Fate}
{
	Those who follow the Asteria, and Seers in general, are often derided and dismissed as passive observers. From 7th level, you learn that this is untrue, as you gain the ability to subtly alter the strands of fate, picking out a desirable future and making minute alterations to the present to navigate your way there. 
	
	At the end of every long rest, you gain access to a number of {\it Fate Dice}. Before any check is performed within 5m of you, you may expend one of your dice to either increase or decrease the result of the roll by the amount rolled on your Fate Die. The size and number of dice you gain is detailed below. The minimum result of any roll is a 1, and you cannot use this ability to increase the result of a dice roll above its maximum value. 
	
	\begin{center}
		\newcommand\entry[3]{#1	&	#2	&	#3	\\}
		\begin{rndtable}{c c c}
			\entry{\bf Druid Level}{\bf Number of Dice}{\bf Dice Size}	
			\entry{7-8}{2}{d6}
			\entry{9-10}{3}{d6}
			\entry{11-12}{4}{d6}
			\entry{13-14}{4}{d8}
			\entry{15-16}{5}{d8}
			\entry{17-18}{5}{d10}
			\entry{19 + }{5}{d12}
		\end{rndtable}
	\end{center} 
	
	Any unused Fate Dice are consumed at the start of a Long Rest. 
}

\subfeat{Starry-Eyed}
{
	From 11th level, you have spent so long gazing into the mystical depths of the Asteria\apos{}s domain, your very eyes have become altered and changed.  
	
	Using a major action, you may activate one of the following benefits:
	\begin{itemize}[itemsep = 0cm]
		\boldItem{Eye of the Night}{gain darkvision up to a distance of 15m}
		\boldItem{Eye of the Scribe}{You can read any written language}
		\boldItem{Eye of the Soul}{You can detect the positions of all living beings within 5m, even those hidden by invisibility}
		\boldItem{Eye of the Arcane}{You can {\it Detect Magic} in a radius of 5m.}
	\end{itemize}
	
	This ability lasts for one hour, and cannot be re-activated until you take a Short Rest. 
	
}
\subfeat{Retrocognition}{

Starting at 15th level, you can call up visions of the past that relate to an object you hold or your immediate surroundings. You must spend at least 1 minute in uninterrupted meditation, then receive a vision.  

\begin{itemize}[itemsep=0em]
\item {\bf Object Reading}: Holding an object as you meditate,
you can see visions of the object's previous owner.
After meditating for 1 minute, you learn how the owner
acquired and lost the object, as well as the most recent
significant event involving the object and that owner.
If the object was owned by another creature in the
recent past (within a number of days equal to your
\attPerShort{} score), you can spend 1 additional minute
for each owner to learn the same information about
that creature.

\item {\bf Area Reading}: As you meditate, you see visions
of recent events in your immediate vicinity (a room,
street, tunnel, clearing, or the like, up to a 10m cube),
going back a number of days equal to your \attPer{}
score. For each minute you meditate, you learn about
one significant event, beginning with the most recenl.
Significant events typically involve powerful emotions,
such as battles and betrayals, marriages and murders,
births and funerais. However, they might also include
more mundane events that are nevertheless important
in your current situation.
\end{itemize}

}


\subsection{Aspect of the Dryad}

Dryads are tree spirits, beings which reside within and protect plants. Associated with life, growth, healing and longevity, dryads are unusual and fickle creatures. 

\subfeat{Seed of Growth}{

At first level, upon choosing to follow the Dryads, many Druids report having a dream in which a Dryad offers them a choice of single seed of a number of trees, representing the first step in their path to greatness.

Choose from the following trees:

\begin{center}
	\newcommand\entry[2]{#1	&	#2	\\}
	\begin{rndtable}{p{2.4cm} p {6.2cm} }
		\entry{\bf Tree}{\bf Effect}
		\entry{Mighty Oak}{You become proficient in {\it Strength}, and have advantage on all Strength checks.}
		\entry{Healing Willow}{Whenever you take an action which restores HP to an individual, restore additional HP equal to your \attPerShort{} modifier}
		\entry{Soothing Pine}{You and allies within 5m have advantage to Resist the effects of {\it Terrified, Enraged} and {\it Charmed} status effects.}
		\entry{Piercing Holly}{Unarmed strikes and grapples by or against you deal an additional 1d6 piercing damage to your opponent}
		\entry{Protective Alder}{You gain a bonus to your block value equal to 1 + one-third your Druid level (rounded up).}
		\entry{Knowing Beech}{You may cast {\it Commune With Nature} as a wandles, silent and instantaneous action 3 times per short rest.}
	\end{rndtable}
\end{center} 

}
\subfeat{Encourage Growth}
{
	At 4th level, you gain the ability to manipulate and speed up the growth of organic processes. You may use this ability as a major action once per short rest, gaining an additional use at 6th, 12th and 17th levels respectively. 
	
	When invoking this ability, you may choose from the following effects:
	
	\begin{itemize}[itemsep=0em]
		\boldItem{Nuture Plant}{Choose a single plant that you can see within range. Over the next minute, you can cause this plant to grow in size by a number of metres equal to one-third your druid level. You may guide this growth to an extent, causing it to sprout branches in a particular direction, or produce fruit. The plant growth must be within the realms of what the plant could achieve naturally - you merely accelerate and guide the process.}
		\boldItem{Enrich Field}{Choose an area up to 100m in radius, and imbue it with natural, positive energy. All plants which grow in this region for the next year grow unusually fast and are exceptionally bountiful. }
		\boldItem{Healing Moss}{You may direct your growing, energizing powers into healing a living being by causing a magical moss to grow over their wounds, knitting them together. You may heal HP equal to 1d6 per Druid level, divided up between any creatures you can see within range.}
	\end{itemize}
	
}
\subfeat{Hybrid Seed}
{
	At 7th level, you may choose another Tree Seed from the {\it Seed of Growth} ability, or from the supplemental seeds shown below:
	
	\begin{center}
	\newcommand\entry[2]{#1	&	#2	\\}
	\begin{rndtable}{p{2.4cm} p {6.2cm} }
		\entry{\bf Tree}{\bf Effect}
		\entry{Basking Palm}{Your natural healing ability is increased. Your recovery dice increase to d8s.}
		\entry{Giant Sequoia}{Gain an additional ability for you {\it Encourage Growth} feature: you may choose to cast the {\it Alter Size} spell on yourself as a wandless, silent action. }  
		\entry{Poisonous Yew}{Once per short rest, by taking a major action, you may exude a poison from your skin. For the next 5 minutes whenever a being makes physical contact with you, it must perform a Vitality Resist against your Arcane Subjugation, taking the {\it Poisoned: Mild} status effect on a failure.}
		\entry{Tangled Ash}{Gain an additional ability for you {\it Encourage Growth} feature: you may choose to cause a large number of grasses and vines to sprout from the ground, snaring at the feet of those who pass. Designate an area up to 5m in radius, excluding any number of sub-areas you choose. Movement speeds through this region are halved. The overgrown area has a HP of 1d4 per Druid level and can be damaged by Fire, Necrotic or Slashing damage.  When HP is reduced to zero, the effect vanishes.  }
	\end{rndtable}
\end{center} 
}

\subfeat{Lay Down Roots}
{
	For a tree, their entire existence is centred around their connection to the ground: their roots. From 11th level, you gain the ability to lay down `roots' into the Astral plane.
	
	At the end of a long rest, choose from the following bonuses, which you maintain until your next long rest:
	\begin{itemize}[itemsep=0em]
		\boldItem{Groundhold}{Whilst your feet are on the ground, you cannot be moved unless you choose to be. Any magical or mundane effect which moves you through space (i.e. not telportation effects) fails unless you choose to allow it.}
		\boldItem{Draw Nutrients}{You do not need to consume food, and you gain an additional 2 recovery dice when you take a short rest.}
		\boldItem{Rootsense}{Whilst your feet are on the ground, you gain {\it tremorsense} for 5m around you. }		 
	\end{itemize}
}


\subfeat{Tree of Life}
{
	At 15th level, you are granted the power to defy death, at least for a while. 
	
	Once per long rest, if you or an ally within 100m would suffer from an effect that would normally cause them to die or enter the {\it Critical Condition} status, you may choose to expend all of your remaining FP to restore them to half their maximum HP, and remove any negative status effects they are afflicted with. You then regain 3d4 FP, or half your previous FP, whichever is lower. 
}

\subsection{Aspect of the Nymph}

The Nymphs are elemental spirits \minus{} they embody and inhabit the frigid lakes, the endless skies and the incandescent wildfires which populated the primal Earth. Though the typical image of a nymph is of a carefree spirit, they represent immense and ancient powers, and should not be trifled with.  

\subfeat{Empowered Elemental}
{
	At first level, you may add your Power modifier to the damage roll of any spell from the Elemental discipline. 
}
\subfeat{Harness the Power}
{
	Beginning at 4th level, you learn to exert your will over Elemental spells, shaping their area of effect to your will. 
	
	Whenever you cast an Elemental spell with a designated area of effect, you may nominate a number of beings up to 1 + your intelligence modifier (min 1) who are considered immune to this spell. 
}
\subfeat{Token of Primal Power}
{
	At 7th level, you gain a deeper association with one of the elements. Choose from one of the following associations:
	
	\newcommand\tableEntry[2]{ #1	&	#2 \\}
	\begin{center}
		\begin{rndtable}{c p{7.4cm}}
				\tableEntry{\bf Element}{\bf Effect}
				\tableEntry{Fire}{You are Resistant to Fire damage. }
				\tableEntry{Water}{You can breathe freely underwater. }
				\tableEntry{Earth}{You are Resistant to Bludgeoning damage. }
				\tableEntry{Air}{You are Immune to falling damage.}
				\tableEntry{Lightning}{You are Resistant to lightning damage, }
				\tableEntry{Radiance}{You gain nightvision up to 5m away, and cannot be blinded. }
		\end{rndtable}
	\end{center}
}
\subfeat{Elemental Action}
{
	Beginning at 11th level, whenever you use a major action to perform an attack, you may take an additional major action to cast a spell from the Elemental discipline. 
}


\subfeat{Overpower}
{
	At 15th level, you gain the ability to channel enormous amounts of primal power into your spells.
	
	When casting any spell which deals damage, you may perform a Power check (1d20 + Power modifier). The DV of the check is set by:
	$$ \text{DV} = 5 + 3\times\text{Spell Level}$$
	On a success, the spell deals the maximum possible damage. On a failure, perform the damage check with advantage, but also take 1d10 psychic damage (ignoring resistances or immunities) per spell level. 
	
	Each time you use this ability, the DV increases by 2. This is reset upon taking a Long Rest. 
}


\subsection{Aspect of the Satyr}

\subfeat{1}{1}
\subfeat{4}{4}
\subfeat{7}{7}
\subfeat{11}{11}
\subfeat{15}{15}


 
