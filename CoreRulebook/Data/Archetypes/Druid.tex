
\chapter*{Druid}
\addcontentsline{toc}{section}{Druid}
An intro bit of text 

%%archBegin
\archetype{name=Druid, hp=8, fp=10, armour=None, tool=Choose 1 from Herbology Tools\comma{} First Aid Kit\comma{} Alchemy Gear or a Musical Instrument, disc=, weapon=None, prof=Nature and choose 2 from Speed\comma{} Strength\comma{} Vitality\comma{} Stealth\comma{} Willpower\comma{} Persuasion\comma{} Arcane\comma{} Logic\comma{} Unnature\comma{} Investigation or Observation., equip=1 weapon of your choice, memorised=3 spells from the basic spells table, expertI = 2, maxspellI = Beginner, bonusI = Nature Affinity\comma{} Druidic Aspect\comma{} Aspect Feature I, expertII = 2, maxspellII = Beginner, bonusII = Nature Senses, expertIII = 2, maxspellIII = Beginner, expertIV = 3, maxspellIV = Beginner, bonusIV = Aspect Feature II, expertV = 3, maxspellV = Novice, bonusV = Reclaim Nature, expertVI = 3, maxspellVI = Novice, bonusVI = Organic Repose, expertVII = 3, maxspellVII = Novice, bonusVII = Aspect Feature III, expertVIII = 4, maxspellVIII = Novice, expertIX = 4, maxspellIX = Novice, bonusIX = Nature Affinity II, expertX = 4, maxspellX = Adept, bonusX = Nature\apos{}s Shrine, expertXI = 4, maxspellXI = Adept, bonusXI = Aspect Feature IV, expertXII = 5, maxspellXII = Adept, expertXIII = 5, maxspellXIII = Adept, bonusXIII = Ancient Knowledge, expertXIV = 5, maxspellXIV = Adept, expertXV = 5, maxspellXV = Master, bonusXV = Aspect Feature V, expertXVI = 6, maxspellXVI = Master, expertXVII = 6, maxspellXVII = Master, bonusXVII = Nature Affinity III, expertXVIII = 6, maxspellXVIII = Master, bonusXVIII = Aspect Feature VI, expertXIX = 6, maxspellXIX = Master, expertXX = 7, maxspellXX = Ascendant, bonusXX = Aspect Feature VII, shortmode = 0}
%%archEnd


\section*{Acquired Feats}


\feat{Druidic Aspect}
{
	When a Druid first feels the call of the wild, the urge to protect and guide nature, they embark on a profound spiritual journey, during which they bind their soul to one of four ephemeral aspects of nature. 
	
	You may choose to follow the Tree\minus{}Loving Dryads, the Elemental Nymphs, the Beastial Satyrs or the Star-Bound Asteria. 
	
	Your choice of Druidic Aspect grants you additional abilities at 1st, 4th, 7th, 11th, 15th, and 18th level. These, along with more details about the Aspects, can be found at the end of the Archetype. 
}


\feat{Nature Affinity}
{
	From 1st level, a Druid learns a deep respect for all things natural \minus{} and where possible, attempts to eschew artificial products. 
	
	Druids may only use their proficiency bonus on weapons or armour made from natural products \minus{} wood, leather and so on, but not smelted iron or synthetic fibres. In return, they are automatically proficient in all such items. 
	
	In addition, at 9th level, and again at 17th level, your affinity towards such weapons increases, allowing you to make an additional strike whenever you perform a full turn attack with a weapon meeting this criteria.
}




\feat{Nature Senses}
{
From 2nd level, a Druid's senses are deeply intertwined with their fundamental connection to nature. 

Whilst in contact with some significant aspect of the natural world \minus{} feet on untouched earth, or placing a hand on a mighty oak \minus{} a Druid gains advantage on all Perception checks. 
}

\feat{Reclaim Nature}
{
	At 4th level, you gain the ability to imbue artificial objects with a semblance of the organic, natural energy they once possessed. 
	
	Upon touching an object manufactured from some visible natural substance, such as a wooden table, a stone statue, or a leather jacket you may imbue it with magical energy, and encourage it to retake its natural form in some way. The table might put down roots and begin to grow, whilst the statue might accumulate dust until it resembles a boulder and the leather jacket could sprout mushrooms and rot away. 

	This ability cannot bring about instantaneous, drastic change, but merely allows you to guide objects back into the natural cycle of being. 
}

\feat{Organic Repose}
{
	At 6th level, when you take a short rest in a natural area, gain additional Recovery Dice equal to one\minus{}third your Druid level. 
}

\feat{Nature\apos{}s Shrine}
{
	At 10th level, your connection to nature becomes powerful enough to imbue an entire area with natural, organic energues.
	
	By expending 3 hours work, you can create a small shrine in a natural place – maybe a stone circle, or a clearing in a wood, decorated with feathers and other such totems of nature. This shrine covers an area up to 3 metres in radius. 

	Beings which take a Long Rest inside the Shrine find that they lose an additional level of exhaustion to normal, and if they had no exhaustion to begin with, they wake with the {\it Calm Mind} status effect. 
	
	You remain aware of any being entering any Shrine you have created, as if they had a {\it Caterwauling Ward} placed upon them.
}

\feat{Ancient Knowledge}
{
	At 13th level, you gain access to some hidden, secret aspect of natural life. 
	
	Choose from one of the following secrets:
	\begin{itemize}
		\emphItem{Secret of Life}{You learn a deep secret regarding the preservation of life. Once per short rest, you may touch someone with the {\it Critical Condition}, or {\it Critical But Stable} status effects, and restore them to 1HP, removing the associated status effect. }
		\emphItem{Secret of the Moon}{You learn to tap into the all-seeing nature of the Moon. Once per short rest, as an instantaneous action you may cast you soul upwards up to 30m into their air, giving you a bird\apos{}s eye view of the surrounding area, with perfect darkvision. }
		\emphItem{Secret of the Morphic Field}{You learn to tap into the web which connects all living beings together. Once per long rest, you may instantly learn the location of all living beings (i.e. not undead, constructs or devils etc.) within a 10 metre radius. This effect cannot penetrate magical shielding, though such a region would show up as a `blind spot'.}
	\end{itemize}
}

\feat{Wrath of the Wild}
{
	At 20th level, you can imbue your attacks with the primal fury of the wild, the infinite fury and hunger present in the cycle of life and death. 
	
	Whenever you perform an attack with a physical weapon, or a spell which targets only a single individual, you may imbue your strike with a {\it Fury} spell, even if you do not know how to cast it. The target must succeed on a Compassion Resist against your Arcane Subjugation, or become {\it enraged}.
}


 
