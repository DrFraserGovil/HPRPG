\documentclass[CoreRulebook.tex]{subfile}
\newpage
\begin{strip}
\setlength{\parskip}{4pt}

\section{Slytherin Student}
As a house, Slytherin has a bad repuation -- even the words of the Sorting Hat have a menacing air:
\begin{displayquote}
\it Or perhaps in Slytherin,
\\
You{\apos}ll make your real friends,
\\
Those cunning folk use any means,
\\
To achieve their ends.
\end{displayquote}
This repuation is, for the most part, undeserved. Slytherin is not a house of evil students; rather it is the house of people with ambition, charm and with lofty goals. Driven by their desire to make something of their lives, the Slytherins can indeed be deceptive, but they can also be charming and persuasive. Never underestimate a Slytherin student, for they will surely never underestimate you. 

Aside from the occaisional bad egg, the Slytherin students fall into two camps: the {\bf student politcians} and the {\it schemers}. The student politicians are those students who are heavily involved in every student initiative they can find - whether they do this for the sake of power, or out of a genuine desire to improve the lives of students is neither here nor there. The schemers, on the other hand, are students who always seem to have some kind of plan, a side-hustle or otherwise elaborate scheme. 
%%archBegin
\archetype{Slytherin Student}{Student Politician}{Schemer}{1}{featureI=Heart of Ambition, featureII=Bonus Skill, featureIII=Heart of Ambition II, alphaFeatureIII=Alliance, betaFeatureIII=Side Hustle, alphaFeatureIV=Inevitable Betrayal, betaFeatureIV=Researched Enemy, featureV=Bonus Skill, alphaFeatureV=Charm Offensive, betaFeatureV=Life Plan}%%archEnd

\end{strip}


\subsection*{Archetype Features}

\textbf{\textit{Heart of Ambition:}} 

Starting at level 1, the Heart of Ambition gives your character a +1 boost to the Persuasion and Deception proficiencies. This increases to +2 at level 4. 
\par
~
\par
\textbf{\textit{Bonus Skill:}} 

At levels 2 and 5, you may choose an additional Skill when levelling up. 




\subsection*{Student Politician Features}

\textbf{\textit{Alliance:}}

From 3rd level you may attempt to persuade any sapient creature to not only stop fighting you, but to defect and join your side. Targets must pass a SPR (willpower) Resist check against your 1d20 CHR (persuasion) check. If they fail, they will disengage from combat, and join your side as an ally. This action takes 3 turns to complete, during which time the target must be within hearing range. 
\par
~
\par
\textbf{\textit{Charm Offensive:}} 

From 4th level, you may use your charm and charisma to lower the mental defences of all non-allies within hearing range. Targets must pass a SPR (willpower) Reisst check against your 1d20 CHR (decpetion) check. Failure results in a 100\% weakness to psychic damage for 20 rounds. 
\par
~
\par
\textbf{\textit{Inevitable Betrayal:}} 

From 5th level, you may choose to betray any individual under the influence of the {\it Alliance} effect, or any genuine ally. Doing so gives you check double-advantage on all actions against them for 4 turns, and the first attack triggers a critical strike on them. 

\subsection*{Schemer Features}

\textbf{\textit{Side Hustle:}}

From 3rd level, you may set up a small business to make you a small amount of money every day. The amount of money generated every day is 5 times your Slytherin level.
\par
~
\par
\textbf{\textit{Researched Enemy:}}

From 4th level, you may research into any named species or NPC. That species or character then gets check disadvantage on any actions against you. The number of researched enemies you can have at any time is 3 less than your current Slytherin level. It takes 1 week to research a new enemy. 
\par
~
\par
\textbf{\textit{Life Plan:}}

From 5th level, when you multiclass, you may automatically start at LVL 3 in your new Archetype.