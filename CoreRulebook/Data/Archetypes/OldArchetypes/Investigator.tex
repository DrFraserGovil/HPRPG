\documentclass[CoreRulebook.tex]{subfile}
\newpage
\begin{strip}
\setlength{\parskip}{4pt}

\section{Investigator}
\blindtext[1]

%%archBegin
\archetype{Investigator}{Detective}{Journalist}{0}{featureI=Investigator's Insight, arcaneI=2, featureII=Interview, arcaneII=2, featureIII=Interrogator's Demeanor I, alphaFeatureIII=Arrest, betaFeatureIII=Connections, arcaneIII=2, featureIV=Always Alert, betaFeatureIV=Eavesdrop I\comma{} Hurtful Words, arcaneIV=2, alphaFeatureV=Police Tactics I, arcaneV=3, featureVI=Interrogator's Demeanor II, betaFeatureVI=Influence Morale (d4), arcaneVI=3, alphaFeatureVII=Gut Instinct, arcaneVII=3, alphaFeatureVIII=Good cop/Bad cop, betaFeatureVIII=Silver Tongued, arcaneVIII=3, alphaFeatureIX=Evaluate Perp, betaFeatureIX=Influence Morale (d6), arcaneIX=3, featureX=Research Team, betaFeatureX=Contact Curation I, arcaneX=4, featureXI=Interrogator's Demeanor III, alphaFeatureXI=Police Tactics II, arcaneXI=4, featureXII=Undercover, alphaFeatureXII=Taser, betaFeatureXII=Influence Morale (d8), arcaneXII=4, alphaFeatureXIII=Evaluate Perp II, arcaneXIII=4, featureXIV=Interrogator's Demeanor IV, arcaneXIV=4, alphaFeatureXV=Bent Copper / Honest Bobby, betaFeatureXV=Influence Morale (d10), arcaneXV=5, featureXVI=Illusive Deduction, betaFeatureXVI=Manipulate Truth, arcaneXVI=5, alphaFeatureXVII=Restraining Force, arcaneXVII=5, betaFeatureXVIII=Influence Morale (d12), arcaneXVIII=5, alphaFeatureXIX=Police Tactics III, betaFeatureXIX=Contact Curation II, arcaneXIX=5, alphaFeatureXX=Sherlockian Deception, betaFeatureXX=Topple Regime, arcaneXX=6}%%archEnd

\end{strip}



\subsection{Starting Equipment}

An Investigator starts swith:
\begin{itemize}[itemsep=0em]
	\item A basic pack 
	\item A notebook at Quick-Notes Quill
	\item a Wand (roll on the wand table to determine composition)
	\item 2d6 $\times 5$ gold
\end{itemize}

\subsection{Starting Spells}

An Investigator begins with 3 spells from the basic spells table, plus {\it Identify}, memorised.

\subsection{Archetype Features}

\feat{Investigator's Insight}

From 1st level, you gain a +1 bonus to Perception and Research proficiencies. 

\jump\feat{Interview}

From 2nd level, if you are able to isolate a captured (or willing) target, you may initiate an interview. Target performs a SPR (Willpower) Resist check against either a d10 CHR (Persuasion) check {\bf or} a d10 POW (Intimidation) check. If target fails, the GM must answer any question you ask truthfully (though they may lie by omission, if the question is not specific enough), if it is reasonable that the target knows the answer. If the check fails, the GM wil lie. 

You may ask up to 5 questions in an interview, before the target becomes uncooperative and refuses to answer. 
\jump
\feat{Interrogator\apos{}s Demeanour}

At 3rd level, and then again at 6th, 11th and 14th, you gain a +1 bonus to either Persuasion or Intimidation proficiencies (you may choose a different proficiency at every increase).



\jump
\feat{Always Alert}

At 4th level you gain a +1 bonus to your passive perception value, and an additional +1 for every 4 Investigator levels above 2nd.

\jump\feat{Research Team}

At 10th level, you have gathered the resources to form an in-house research team. If you are able to contact your team to ask them a question, after 1 day of research, they provide you with a +5 bonus to all research checks on an individual subject (i.e. a person, event, or location) for the next 1 hour. This ability may only be used once per week. 

\jump\feat{Undercover}

From 12th level, gain a bonus to your Stealth proficiency equal to one-third your Investigator level. 

You may also take 1 hour to completely alter your appearance and create a false identity. Whilst inhabiting this new identity, you gain an additional +2 bonus to Persuasion and Stealth proficiencies when on `undercover missions'. 


\jump\feat{Illusive Deduction}

At 16th level, your deductive skills are honed such that you can spot inconsistencies when someone attempts to alter your perception of reality. When an illusion spell is cast on you, in addition to the usual Perception checks, the GM rolls a d4. If it comes up as a 1, they must inform you that an illusion spell has been cast on you. 

\newpage

\subsection{Detective}

\feat{Arrest}

From 3rd level, if you are within touching distance of a target under the influence of the {\it Stunned}, {\it Trapped}, {\it Exhausted}, {\it Terrified} or {\it Major Injury} status effect, you may take a minor action to conjour a pair of mystic manacles around their wrists. The number of manacles you may have active at any one time is equal to half you Detective level. Targets may attempt to break out of these restraints (if they are able to take actions) through a ATH (Strength) resist check, with a DV equal to 3 + your Detective level. 

Once per combat engagement, you may take another major action to send all such manacled entities into a `holding cell', trapping them in the Astral plane. When the combat engagement has ended you may either send all trapped entities to Azkaban, or summon them individually back out of the `cell' for questioning. Entities left in the `cell' more than 5 minutes after combat has ended are returned to this plane of existence, without their manacles.  

\jump
\feat{Police Tactics}

At 5th level, and then again at 11th and 19th level, you may choose to gain one of the following bonuses:

\begin{itemize}
	\item {\bf Disarming training:} gain a bonus to the casting check of the {\it Disarm} spell equal to one-third of your Detective level
	\item {\bf Hand-to-hand Combat:} unarmed strikes do 1d4 ATH (Strength) Bludgeoning damage, and whenever contact is made, target must perform a SPR (Willpower) Resist check against a d10 POW (Intmidation) check or take the {\it Terrified} status. 
	\item {\bf Combat De-escalation:} perform a CHR (Persuasion) check on an isolated target, DV equal to target SPR attribute. If check succeeds, target lays down their weapons and raises their hands above their head (such entities are considered viable targest for the {\it Arrest} ability.
	\item {\bf Spatial Awareness:} when a pair of enemies attempt to flank you, you may perform a DV15 INT(Perception) check to automatically move up to 1m, such that only one target threatens you (if possible). This is in addition to your normal actions, but does not allow you to evade any attacks on you this turn from anyone other than the flanking opponent. 
	\item {\bf First Aid:} gain a bonus to your Healing proficiency equal to one-quarter your Detective level. 
\end{itemize}

\jump
\feat{Gut Instinct}

From 7th level if, during an {\it Interview}, the difference between your interrogation check and the resist check is greater than 2, you automatically know that the perp is lying to you.
\jump\feat{Good-cop/Bad-cop}

From 8th level, when performing an Interview, you may perform a combined d10 POW (Intimidation) and d10 CHR (Persuasion) check for the interrogation. 
\jump\feat{Evaluate Perp}

From 9th level, when encountering a new threat, your keen eye allows you to infer information about them. The GM will tell you if they are you superior, equal, or inferior in one of the following categories of your choice: ATH, CHR, INT or EVL. At 13th level, you may choose 2 categories. 

\jump\feat{Taser}

From 12th level, you gain access to a magical device similar in nature to a Taser. This device takes a minor action to use, and auto-casts the {\it Lightning Bolt} spell on a target, with a `casting check' equal to your Detective level (and hence does  2d4 + (Detective level - 5) electric damage). If the target fails the resist check, they become {\it Stunned} for 2 turns. 

This device recharges at a rate of 1 usage per hour, and holds a maximum of 5 discharges. 

\jump
\feat{Bent Copper/ Honest Bobby}

At 15th level, having risen through the ranks, you must decude what kind of officer you are going to be:
\begin{itemize}
	\item {\it Corruption:} you gain a +5 bonus to Intimidation proficiency, at the expense of a +4 increase in your EVL attribute. In addition, once per day you may perform an indimidation check on NPCs (DV 10) to curry bribes of up to 5d20 gp.
	\item {\it Idealism:} you suffer a 3-point penalty to Intimidation proficiency, but whenever the total value of a non-deception CHR check is less than your CHR attribute, you may use that value instead. 
\end{itemize}

\jump
\feat{Restraining Force}

From 16th level, when casting an explicitly non-lethal (or non-damage causing) spell from the Hexes school, you may add your Arcane Wisdom bonus to the casting check, ignoring the usual once-per-day rule. Normal usage of Arcane Wisdom is unaffected by this skill.  

\jump
\feat{Sherlockian Deception}

From 20th level, your powers of deduction have reached legendary levels, such that you can turn your enemies own plans against them. Once per day, if you trigger a trap, or otherwise fall afoul of an enemy\apos{}s nefarious scheme, you may reveal that you were aware of this all along, and that they have in fact strayed into {\it your} nefarious scheme, upon which you may either:
\begin{enumerate}
	\item Turn the effect of their trap back onto them
	\item Automatically cast any spell (memorised or not) of Expert level or below with an assumed dice roll of 20. 
	\item Teleport in a backup team of 2d4 level 15 NPC Enforcer-Aurors you had secretely waiting in the wings. 
\end{enumerate}



\subsection{Journalist}

\jump\feat{Connections}

A journalist lives or dies by their connections, so from 3rd level, for every 4 days that you spend in a given location, you may choose to gain one `contact' associated with that location. The maximum number of contacts (across all locations) is equal to twice your journalist level (at any time you may choose to lose any number of contacts from any location). 

Whilst in a location, you may call upon your local contacts to gain a +2 bonus (per invoked contact, max +10) to all research or influence checks (or checks otherwise judged to be `journalism' by your GM), or to undertake some simple task which would not normally require a check (such as getting a book out of a library for you). 

For every contact invoked, roll a d4, if the result is greater than 1, you lose that contact. Even if the contact is not `burned', you may not call upon them for one more day (you may use other connections, however).


\jump\feat{Eavesdrop}

At 4th level, you may hear all verbal communication in a 5m radius, even if whispered or behind a door, as if you had the {\it Eavesdrop} charm permanently active. 

At 17th level, this increases to 10m radius respectively. In addition, at 15th level, you become aware of all hidden or invisible creatures in this radius. 


\jump\feat{Hurtful Words}

At 4th level, your journalism has reached a level of prominence that your words can have a negative effect on your target. 

Once per week, you may state that you have submitted a hurtful article about a person or a group of people, and name a number of connections you used to do so (max = one half your Journalist level). If you encounter a target of your vicious prose, you may remind them of the hurt you did to them, which has the effect of automatically casting the {\it Chaotic Whispers} spell on them as a major action with PP equal to the number of connections dedicated. This `spell' is maintained for as long as you continue talking. The number of times you may use this action depends on how many people were targeted by your article, with more specific targets granting you more uses:

\begin{center}
\begin{rndtable}{c c c }
Size of targeted group	& 	Number of casts
\\
1-5 people	&	Half Journalist level
\\
5-10 people	&	Third Journalist level
\\
10-50 people	&	Quarter Journalist level 
\\
50+ people	& 	Once
\end{rndtable}
\end{center}
This counter is reset when you write a new article, targeting a new group. 

\jump\feat{Influence Morale}

From 6th level, you gain the ability to influence sentient beings in real time, either to boost their confidence, or destroy their self-esteem. As a minor action, you may give a d4 dice to any sentient in hearing range. 

At any point in the next 10 minutes, you may direct that being to add or subtract that dice roll from their next check. 

The size of dice you can give increases incrementally at higher levels: die increases by 2 at 9th, 12th, 15th and 18th level.




%~\jump\feat{Press Pass}

%~From 7th level, gain a number of `tokens' equal to one third your Journalist level. Each token can be pressed into a simple door lock to disable it for 1d4 minutes, without triggering any associated warnings or alarms. After the requisite amount of time, if the door is not replaced, any alarms or traps associated with the door trigger. If the door is shut when the time elapses, the lock re-engages with no hint it was ever disabled. 

%~You may restock your `tokens' once per week. 


\jump\feat{Silver Tongued}

From 8th level, when performing an interview, you may choose one of the following checks:
\begin{itemize}
	\item {\bf Convince: } 2d10 CHR (Persuasion)
	\item {\bf Blackmail: } 2d10 INT (Intimidation)
	\item {\bf Deceive:} 2d10 CHR (Deception)
\end{itemize}

You may also use your connections-bonus to aid you. 
 


\jump\feat{Contact Curation}

From 10th level, when checking to see if a connection has been `burned', you roll a d4 and the contact is only burned if the result is greater than 2 (decreasing burn chance from 75\% to 50\%).

From 19th level, the check must be equal to a 4 for the contact to be burned. 

\jump\feat{Manipulate Truth}

At 16th level, choose one of the following bonuses:

\begin{itemize}
	\item {\bf Truth-teller}: as a minor action, you may break any illusion or otherwise mind-altering spells active on an ally within touching distance.
	\item {\bf Lie-seller}: you may use your Arcane Wisdom on all Deception checks without the once-per-day rule. Other uses of the AW are unaffected.  
\end{itemize}

\jump\feat{Topple Regime}

From 20th level, you wield such influence with your words that you can bring down governments and corporations. 

By burning 20 contacts, you may learn such a damning secret about a government or business leader that they and their entire circle must immediately resign. You may use your influence to install a regime which complies with a demand that you make...just be careful that this demand is nothing {\it too} outrageous, or you may very quickly find that another journalist topples your new regime, and takes you with it!
