

\chapter*{Artificer}
\addcontentsline{toc}{section}{Artificer}
An intro bit of text 
\newcommand\itList[2]{\item {\bf #1:} #2}
%%archBegin
\archetype{name=Artificer, hp=8, fp=8, armour=Medium Armour, tool=Choose one from Runic Tools\comma{} Alchemy Gear and Tinkering tools\comma{} plus any two additional tools., disc=Choose one discipline of your choice\comma{} and a further two from Elemental\comma{} Alteration\comma{} Conjuration \& Warding., weapon=Simple Weapons, prof=Choose any two from Arcane\comma{} History\comma{} Logic\comma{} Nature\comma{} Willpower and Investigation., equip=Wand\comma{} a Scholar\apos{}s pack containing the tools selected above., memorised=Any four from the basic spells table., listIName =Ad\minus{}hocs, singleListMode = 1, expertI = 2, maxspellI = Beginner, bonusI = Artificer\apos{}s Specialisation\comma{} Ad\minus{}hoc Creations, listI_I= 2, expertII = 2, maxspellII = Beginner, bonusII = Tools of the Trade, listI_II= 2, expertIII = 2, maxspellIII = Beginner, bonusIII = Specialisation Feature I, listI_III= 3, expertIV = 3, maxspellIV = Novice, bonusIV = Tweak Effect, listI_IV= 3, expertV = 3, maxspellV = Novice, bonusV = Appraising Eyes, listI_V= 4, expertVI = 3, maxspellVI = Novice, listI_VI= 4, expertVII = 3, maxspellVII = Novice, bonusVII = Specialisation Feature II, listI_VII= 5, expertVIII = 4, maxspellVIII = Adept, bonusVIII = Artificial Ally, listI_VIII= 5, expertIX = 4, maxspellIX = Adept, listI_IX= 6, expertX = 4, maxspellX = Adept, bonusX = Mobile Laboratory, listI_X= 6, expertXI = 4, maxspellXI = Expert, listI_XI= 7, expertXII = 5, maxspellXII = Expert, bonusXII = Specialisation Feature III, listI_XII= 7, expertXIII = 5, maxspellXIII = Expert, bonusXIII = Ally Improvement, listI_XIII= 8, expertXIV = 5, maxspellXIV = Expert, listI_XIV= 8, expertXV = 5, maxspellXV = Master, bonusXV = Rapid Work, listI_XV= 10, expertXVI = 6, maxspellXVI = Master, listI_XVI= 10, expertXVII = 6, maxspellXVII = Master, bonusXVII = Specialisation Feature IV, listI_XVII= 12, expertXVIII = 6, maxspellXVIII = Master, bonusXVIII = Ally Improvement II, listI_XVIII= 12, expertXIX = 6, maxspellXIX = Ascendant, listI_XIX= 14, expertXX = 7, maxspellXX = Ascendant, bonusXX = Maximum Effectiveness, listI_XX= 15, shortmode = 0}
%%archEnd



\section*{Acquired Feats}

\feat{Artificer\apos{}s Specialisation}
{
	At first level, when you become an artificer, you must decide which path you wish to follow \minus{} your {\it Specialisation}. You may choose to become an Alchemist, an Enchanter, or a Mechanist. 
	
	Each choice grants you proficiency in the relevant tools (Alchemy Gear, Runic Tools and Tinkering Tool, respectively), as detailed in the {\it Proficiencies} section of the archetype outline. In addition your choice of Specialisation grants you additional Features at 3rd, 7th, 12th, and 17th level. These are detailed at the end of the Archetype description. 
}

\feat{Ad\minus{}hoc Creations}
{
	At first level, you gain the ability to use an Ad\minus{}hoc creation to assemble something useful on the fly. 
	
	Whilst the normal act of creation (either in an alchemical, mechanical or magical sense) requires careful, controlled actions \minus{} preferably in a laboratory or workshop, those who call themselves Artificers pride themselves on their ability to manufacture items on the fly. Not quite as refined as their normal produce, these creations are termed {\it Ad\minus{}hocs}. The number of Ad\minus{}hocs you are able to produce at each level is given by the relevant column in the Archetype table. This number resets on a Long Rest. 
	
	The type of Ad\minus{}hocs you are able to produce is determined by your specialty, and described in detail at the end of the Archetype table. 	
}

\feat{Tools of the Trade}
{
	At second level, you choose one set of tools with which you are proficient. When using these tools in future you may double your Expertise bonus.  
}

\feat{Tweak Effect}
{
	At 4th level, you gain the ability to slightly modify the effect of an existing item, potion or enchantment. 
	
	This might be a purely superficial change, such as altering the colour of a potion from red to green or changing an enchanted sword to glow with blue flames rather than red. Alternatively you may attempt to add some minor tweaks to the functioning of the object \minus{} perhaps the potion only works on Goblins, or the clockwork device discharges when turned upside down. 
	
	Tweaks should be minor in nature \minus{} attempts to alter the nature of the object too much (especially if it was not originally one of your creations) can have disastrous effects, including the destruction of the item, and collateral damage.  
}

\feat{Appriasing Eyes}
{
	At 5th level, your eyes have become accustomed to recognising the work of other artisans of crafting. 
	
	You can tell at a glance if an item is enchanted, alchemical or mechanical in nature, as well as determine the rough functioning of the object (i.e. is it intended to heal or harm, reveal or hide \minus{} but no specifics). 
	
	You also gain advantage on any Investigation checks to further discern the nature of such an object.  
}

\feat{Artificial Ally}
{
	At 8th level, you gain the ability to construct an artificial being, either alchemical, magical or mechanical in nature, to aid you on your journey. 
	
	The types of Ally you may construct are detailed in the subtype descriptions. 
	
	At 13th and 18th level, you unlock additional improvements to your Artificial Ally, as detailed in the relevant descriptions. 
}

\feat{Mobile Laboratory}
{
	At 10th level you gain the ability to turn any rudimentary work area into a functional Laboratory or Workshop for your crafting purposes. 
	
	If you have access to a sheltered area, you may spend 3 hours setting it up just so, in order to designate it as your workspace. Any artificing checks made within this space can be made with advantage. You may only have one such workspace set up at any one time. 
}

\feat{Rapid Work}
{
	At 15th level, you are able to complete even complex artificing tasks much quicker than before. 
	
	When undertaking an artificing check, roll a d6. Divide the time necessary to complete the action by the result of this roll. 
}

\feat{Maximum Effectiveness}
{
	At 20th level, your artificing abilities reach their zenith, such that all artificing you perform is always the most potent version possible. 
	
	Whenever one of your created items triggers a dice roll to determine its effectiveness, treat the result as the maximum possible value. 
}

\section*{Alchemist}

An alchemist is an Artificer who specialises in the mixing of magical potions, and chemical synthesis. 

\subfeat{Alchemical Ad\minus{}Hocs}
{
	An Alchemical Ad\minus{}hoc takes the form of a half\minus{}mixed potion or salve which the Alchemist has prepared from common ingredients. When required, they can use a minor action sprinkle in an ingredient to finalise the `mixing' of this potion. Lacking the complex brewing and steeping phases typically used in potion mixing, a potion prepared in this fashion is usually much weaker than their standardised counterparts \minus{} although they are much more flexible. 
	
	 When invoking an alchemical ad\minus{}hoc, you must tell the GM what effect you are trying to replicate. In order to maintain the simple nature of an Ad\minus{}hoc, you should limit yourself to one or two words: `heal', `explode', `hallucinate', though this is only a guideline. 
	 
	 The power of your ad\minus{}hoc creations increases with your Alchemist level. You should work with your GM to determine the exact effect, within the limits of your current powers. A rough guide is that alchemical ad\minus{}hocs cannot heal or do damage exceeding 1d6, increasing by 1d6 every three alchemist levels, and that they should not outperform standard potions which replicate this effect. If a potion requires a Resist check, the DV is set by 8 + your Expertise modifier. 
	 
	 After an Alchemical Ad\minus{}hoc is created, you may treat it as any other potion, with the exception that they lose their potency and become worthless one minute after their creation. 
}

\subfeat{Ingredient Intuition}
{
	At 3rd level, your experience with alchemical ingredients allows you to make inferences about the use of certain ingredients in your potions. Whilst handling a sample of such an ingredient, you may perform a DV 10 Nature (or Un\minus{}nature, if applicable) check in order to learn one of the following bits of information:
	
	\begin{itemize}
		\item The recipe for one potion in which this is a key ingredient
		\item One recipe (if any) which you already know, for which this is an optional ingredient
		\item The effect which would result were this ingredient added to a specific recipe
	\end{itemize}
	
	Every subsequent check increases the DV by 5. On a failure, you can learn nothing new about this ingredient for 24 hours.
}

\subfeat{Conservative Mixer}
{
	At 7th level, you have honed your mixing skills such that you can often get by without using up a full sample of the ingredients. 
	
	After a successful potion\minus{}making check, perform the same check again (a `recovery check'), and use the following table:
	
	\begin{center}
	\begin{rndtable}{c p{5 cm} }
		\bf Check	&	\bf Ingredients Recovered
		\\
		$< 10$	&	None
		\\
		10\minus{}14	&	One common, or two abundant samples
		\\
		15\minus{}19	&	One uncommon, or two uncommon samples
		\\
		20\minus{}24	&	One rare, or two uncommon samples
		\\
		25+	&	Two rare samples
	\end{rndtable}
	\end{center}
}

\subfeat{Alchemical Ally: Homunculus}
{
	At 8th level, all Artificers learn to construct an {\it Artificial Ally}. As an Alchemist, your construct takes the form of an Alchemical Homunculus, a tiny living being created during a complex potion\minus{}mixing exercise, in which the very forces of life are invoked. The exact appearance of the homunculus can be determined at the moment of creation: most alchemists prefer to go with a winged fairy\minus{}like creature, though some exotic choices include floating, semi\minus{}sentient potion vials, or simply writhing blobs of fluid. 

	The creation of a Homonculus takes 2 hours and \galleon{1} worth of supplies. Homunculi are devoted to their master and will follow all commands given to them. Outside of this, the Homunculus is considered an independent being, and takes actions with the semblence of free will. 
	
		
	\boxOnlyBeast{name = Alchemical Homunculus, mind = Semi\minus{}Sapient, category = Construct, summary = Alchemically\minus{}created being, hp = 3$\times$Alchemist level health, block = 8, dodge = 16, speed = 2m (walking)\comma{} 8m (flying), fit = 5 (\minus{}3), prs = 17 (+3), spr = 14 (+2), chr = 8 (\minus{}1), int = 7 (\minus{}2), pcp = 13 (+1), pow = 5 (\minus{}3), evl = Equals Master, rating = III, abilityBlock = 1, hasAbilities = 1, hasSkills = 1, size = 10cm, skills = Chicanery (+7)\comma{} Stealth (+7), senses=Darkvision, hasSenses = 1, hasImmune = 1, immune = Poison, hasConditionImmune = 1, conditionImmune = Poisoned, abilities = 
	
	\ability{Master\apos{}s Might}{The numerical value of all attributes (except Evil), increases by 1 for every 3 Alchemist levels above 8th possessed by the Creator. The to\minus{}hit values and damage of the creature\apos{}s attacks also increase by 1.}
	
	\ability{Free Will}{The Homunculus is able to take actions inside and outside of combat like any other sapient being.}
	
	, hasComprehend = 1, comprehend = Its Master\apos{}s language., hasActions = 1, actions = 
	\ranged{Poisonous Spittle}{+7}{4m}{A glob of toxic fluid is spat into the face of the opponent, dealing 1d6+2 poison damage.}
	
	\melee{Tiny Claws}{+2}{Deals 1d4 slashing damage}
	
	\ability{Junior Assistant}{The Homunculus may use its masters Ad\minus{}hoc slots as its own, creating a rudimentary potion following the normal rules. The Homunculus must return to its master in order to recharge this ability.}
	}
	
	At 13th level, and again at 18th, you learn to make some changes to the formula you use to create the homunculus. At each of these levels, you may choose one of the following effects to permanently imbue your Homunculus with:
	
	
	
	\begin{itemize}
		\itList{Genetic Splicing}{Upon creating a homunculus, expend a sample of an organic ingredient of Rare\minus{}level rarity or less into the mixture. The homunculus grows samples of this ingredient on its body, granting you one sample per day. Every time you create a new homunculus, you may choose a new ingredient.}
		\itList{Vicious Spittle}{Increase the damage dealth by the {\it Poisonous Spittle} attack to 2d6, and increase the damage by 1d6, rather than +1 as the Alchemist power increases.}
		\itList{Rapid Attacks}{The homunculus gains the ability to make up to three attacks per cycle.}
		\itList{Stronger Homunculus}{The HP of the homunculus is doubled.}
		\itList{Proficiencies}{Choose up to 4 attribute, tool or weapon proficiencies to grant the homunculus.}
		\itList{Lab Assistant}{Train the homunculus to act as your lab assistant, continually providing the {\it Help} action when you are performing Alchemy checks.}
	\end{itemize}
	
}

\subfeat{Toxin Tolerance}
{
	At 12th level, your continued exposure to noxious fumes has rendered you somewhat immune to them. You are considered Resistant to Poison damage, and you take Advantage on any Resist check to avoid taking the {\it Poisoned} status effects.
}

\subfeat{Pure Mixtures}
{
	At 17th level, your mastery of the potioneering arts is such that your mixtures are free from imperfections, and you can even remove imperfections from pre\minus{}brewed mixtures. 
	
	Your potions never gain the {\it Flawed Batch} status, and you may spend 5 minutes with your Alchemy Gear to remove the {\it Flawed Batch} status from any other potions you possess.
}


\subsection*{Enchanter}


\subfeat{Enchantment Ad\minus{}Hocs}
{
An Enchanter's ad\minus{}hoc takes the form of a wax seal, known as a `signum', into which the Enchanter has infused a level of their power. As a minor action, the Enchanter can then inscribe a rune\minus{}chain into this wax seal, and press it onto the surface of an item. This provides a rudimentary, temporary form of enchantment into the targeted item: an augmentation. 
 
As with the normal enchanting process, as the Enchanter begins the augmentation, they must describe how the runechain is to be interpreted, in keeping with the limited power of the augmentation, this should usually be limited to a short phrase such as `protect against fire'. As with the usual enchanting ritual, you may only invoke effects which can be described with runes that you have memorised. You may augment an item that has already been enchanted using the traditional method, but an item may not bear more than one augment at any given moment. 

The power of your ad\minus{}hoc creations increases with your Enchanter level. You should work with your GM to determine the exact effect, within the limits of your current powers. A rough guide is that enchanter ad\minus{}hocs cannot heal or do damage exceeding 1d4, increasing by 1d4 every three enchanter levels, and that they should not outperform standard spells which replicate this effect. If an augmented effect requires a Resist check, the DV is set by 8 + your Expertise modifier. 
	 
After an Enchanting Ad\minus{}hoc is created, the augmented item executes the effect as a normal enchanted item would do for the next 5 minutes, after which the effect wears off.  
}

\subfeat{Rune Experimentation}
{
	At third level, you learn to dedicate a number of hours to simple brute\minus{}force experimentation, guided by your enchanter's intuition. Upon doing so, you may learn a new rune that you do not yet know. After an hour of work, perform a DV 18 enchanting check. You may repeat this check once an hour until you succeed, for up to 6 hours in a row.

	If you succeed, you choose a new rune to memorise from those you have not yet learned. Generally, you may only choose to learn a Legendary Rune if you already know all of the Mystical runes of the same category.  

	This ability may only be used again after a Long Rest.
}

\subfeat{Enchantment Affinity}
{
	At 7th level, you allow the runes to shape your understanding of the item as you undergo the enchanting process, and allow the runes to guide you as you use the item. As a result, you are considered proficient in any weapon or armour that you have enchanted personally.
}

\subfeat{Enchanted Ally: Sentient Mind}
{
	At 8th level, all Artificers learn to construct an {\it Artificial Ally}. As an Enchanter, your ally takes the form of a True Sentience, stored inside your enchanted creations.
	
	The creation of a true Mind, a being possessing consciousness and original thought is a closely guarded secret amongst the Artificer's guilds. The creation of a mind requries 2 hours work, and an enchanted item into which the mind is to be inserted. Upon creating a new Mind, any existing minds cease to exist.
	
	All Minds are created with a positive attitude towards their creator, though they have a will of their own, and may cease to obey their master if they feel abused, neglected or betrayed.
	
	At 13th level, and again at 18th, you learn to make some changes to the enchantments used to sustain your True Mind. At each of these levels, you may choose one of the following effects to permanently imbue your ally with:
	

	\begin{itemize}
		\itList{Mind Projection}{Once per short rest, The True Mind may use its {\it item\minus{}hop} ability to project itself into the mind of a non\minus{}sapient or sapient creature. The targeted being must perform a DV 12 Willpower Resist check at the end of every turn cycle to expel the True Mind back to its original housing. Whilst the True Mind remains inside, it takes control of the creature and may use all their abilities. If the host dies whilst the Mind resides inside, the Mind is destroyed. The Mind may use its {\it item\minus{}hop} ability to return to its original housing.}
		\itList{Psychic Power}{The damage dealt by the {\it Psychic Worm} attack is increased to 4d4, increasing by 1d4 instead of +1 as the Enchanter level increases.}
		\itList{Permanent Assistance}{The Mind is always providing the {\it Assisted Use} action to those wielding the item it lives within.}
		\itList{Fortified Nexus}{The maximum HP of the mind is doubled.}
		\itList{Recharge Item}{The Mind may use a major action to recharge an item by 1 unit.}		
	\end{itemize}
	\boxOnlyBeast{name = True Mind, mind = Sapient, category = Construct, summary = Sentient mind residing inside objects, hp = 3$\times$Enchanter\apos{}s level health, block = 15, dodge = 3, speed = 0m, fit = N/A , prs = N/A, spr = 15 (+2), chr = 13 (+1), int = 17 (+3), pcp = 15 (+2), pow = 10 (+0), evl = Equals Master, rating = III, abilityBlock = 1, hasAbilities = 1, hasSkills = 1, size = 10cm, skills = Arcane (+7)\comma{} Logic (+7)\comma{} Conviction (+2)\comma{} Persuasion (+5), senses=Alien Senses, hasSenses = 1, hasConditionImmune = 1, conditionImmune = All sensory status effects\comma{} Poisoned, hasImmune = 1, immune = Poison, abilities = 
	
	\ability{Master\apos{}s Might}{The numerical value of all attributes (except Evil), increases by 1 for every 3 Enchanter levels above 8th possessed by the Creator. The DV values and damage of the Mind\apos{}s attacks also increase by 1.}

	\ability{Item\minus{}bound}{The \name{} resides within the magical nexus of an enchanted object. As such it cannot move, and so automatically fails all \attPhys{} and \attFin{} checks. Its health is tied to the object it resides within \minus{} damage to the object damages the mind. When the HP of the object/mind pair reaches zero, the item is destroyed, along with the mind. Equally, repairing the object restores the health of the Mind within.}
	
	\ability{Limited Animation}{The \name{} may subtly alter the physical form of its host in order to represent its presence, causing a face to appear in the pommel of a sword, or causing a hat to crumple into a human\minus{}like visage. This is not enough to allow locomation, however.}
	
	, hasLanguages = 1, language= Its Master\apos{}s language., hasActions = 1, actions = 
	\ability{Item\minus{}hop}{A \name{} can use a major action to transfer itself into any other item enchanted by its creator within 25m, adjusting its health appropriately. }
	
	\ability{Psychic Worm}{({\it ranged attack, range: 10m}) the Mind may force a tendril of psychic energy into a foe within range, dealing 2d4 psychic damage, halved on a successful Conviction Resist (DV 14). }
	
	\ability{Assisted Use}{As a major action, the \name{} may {\it help} the person currently using the item it resides inside. When performing i.e. an attack with a weapon, perform it with advantage. If residing inside armour or other defensive equipment, accuracy checks against the wearer take disadvantage  for this turn cycle. The mind can only help with actions relating to the item it exists within.}  
	}
	
	

}

\subfeat{Nexus of Power}
{
	From 12th level, your expertise with enchanted items allows you to discern the flow and storage of arcane power within an enchanted item, and allows you the ability to bolster or disrupt that flow. 
	
	Once per turn, when coming into contact with any Enchanted or Augmented item, you may use an instantaneous action to perform a DV 15 Arcane check. On a success, you may choose one of the following effects:
	\begin{itemize}
		\item Restore two `charges' of the item's uses
		\item Drain two `charges' of the item's uses
		\item Remove an augmentation from the item 
		\item Decrease a numerical quantity associated with the item (i.e. damage dealt, bonus provided, DV of resist) by 1 point. 
		\item Undo a previously imposed effect. 
	\end{itemize}  
	
	You may use this effect multiple times on the same item, until you fail the Arcane check, at which point the item becomes immune to further checks for 24 hours. 
}

\subfeat{Process Stabilisation}
{
	At 17th level, your enchanting abilities are such that your enchanting process is very stable and safe. Even during catastrophic failures, there is no risk of the enchanted item detonating. In addition, when an enchantment procedure fails, you always get a second chance to rescue the attempt,  which suffers none of the normal drawbacks associated with a rescued enchantment.  
}


\subsection*{Mechanist}


\subfeat{Mechanical Ad\minus{}Hocs}
{
	A Mechanical Ad\minus{}hoc takes the form of a hastily assembled {\it device}, crafted from half\minus{}constructed odds and ends you have lying about your person. Taking a major action, you may assemble these parts together into a functional whole. Formed from scraps, devices constructed in such a way are rarely very sturdy \minus{} they may execute a single well\minus{}defined function once or twice, and then will probably cease to function after that. 
	
	The purpose of a mechanical ad\minus{}hoc, and the means by which you are achieving it from the components provided must be well defined at the moment of creation \minus{} for example you could state that you cobble together a small device which has a loudspeaker attached to a timer: after a certain amount of time, the device triggers emitting an earsplitting shriek. Whilst the components you provide may be complex and electromechanical in nature, the unification of those components together must be simple and obvious from the described components.
	
	Because the components which go into creating a device are assumed to have been on your person the entire time, the invoked components of an ad\minus{}hoc must be such that they are not individually useful items you might find in your inventory. You could not include ``a gun" as a component, for example, though you could probably assemble a makeshift firearm as a separate ad\minus{}hoc action. Alternatively, you may use items from your inventory as components, though they are consumed in this process. Therefore, if you possessed a firearm in your inventory, you could sacrifice it to the creation process. 
	
	You should work with your GM to determine the limits of any such created items as your Mechanist level increases (and they retain an absolute veto on your ability to create such devices). As a general rule, mechanical ad\minus{}hocs cannot deal damage exceeding 1d6, increasing by 1d6 every three mechanist levels. If a device requires a Resist check, the DV is set by 8 + your Expertise modifier. 
	
}

\subfeat{Signature Device}
{
	At 3rd level, your engineering know\minus{}how has allowed you to construct a unique non\minus{}magical artefact, which you consider to be your magnum\minus{}opus. 
	
	Working with the GM, design a weapon, item of clothing, or other hand\minus{}held device. This item must be non\minus{}magical, though it may be imbued with abilities which are similar in nature to spells found in the spell list. You are considered proficient with this device, and any checks made with it include your Expertise modifier. 
	
	This device should follow the following basic restrictions:
	\begin{itemize}
		\item Melee weapons should not exceed 2d6 damage, or a reach beyond 1m
		\item Ranged weapons should not exceed 1d8 damage, or have a range extending beyond 30m
		\item Armour should not provide more than a +4 bonus to Block (and should include a penalty to Dodge)
		\item Items or devices should not have abilities which replicate spell effects greater than Novice level
		\item Weapons and armour should not have more than 1 additional ability, other items may have up to 2 effects
		\item Only replicate spell effects it is reasonable to be electromechanically duplicated.
	\end{itemize}
	
	Every 3 subsequent levels (i.e. at 6th, 9th, etc.) you may make improvements to your signature device, such as:
	\begin{itemize}
		\item +1 to damage
		\item Increased range
		\item +1 to Block (or reduced penalty to Dodge)
		\item More powerful abilities
	\end{itemize}
	
	You may replace your signature device by spending one week of downtime designing a new one, however this requires cannibalisation of your previous device, so you may only have one such device active at any given moment.
}

\subfeat{Singed Experience}
{
	From 7th level, you have a certain amount of experience in dealing with accidental explosions and fiery mishaps from your tinkering experience, and have become proficient in avoiding it.
	
	Whenever you successfully perform a Resist check to halve the amount of Fire or Concussive damage taken from an effect, instead you negate the damage completely.
}

\subfeat{Mechanical Ally: Clockwork Pet}
{
	At 8th level, all Artificers learn to construct an {\it Artificial Ally}. As a Mechanist, your ally takes the form of a Clockwork Pet, a mechanical recreation of an existing animal.
	
	Though non\minus{}magical in nature, the creation of such a complex device necessitates the use of magic in its creation. The creation of a Clockwork Pet requries 5 hours work and \galleon{1} of materials. Upon creating the clockwork pet, you choose what form it is, by choosing a non\minus{}sapient, non\minus{}magical creature with a MoM rating of III or less (and with a size less than 1.5 metres) to replicate.
	
	Clockwork pets are unerringly loyal to their master, and though they are exquisitely created and can mimic the behaviour of the base creature perfectly, they are entirely deterministic in nature. 
	
	\boxOnlyBeast{name = Clockwork Pet, mind = Non\minus{}sapient, category = Construct, summary = Clockwork duplicate, hp = 3$\times$Mechanist\apos{}s level health, block = ?, dodge = ?, speed = ?, fit = ? , prs = ?, spr = ?, chr = ?, int = ?, pcp = ?, pow = ?, evl = Equals Master, rating = III, abilityBlock = 1, hasAbilities = 1, hasSkills = 1, size = ?, skills = Determined by form\comma{} Logic (+4), senses=Alien Senses, hasSenses = 1, hasConditionImmune = 1, conditionImmune = All sensory status effects\comma{} Poisoned\comma{} Charmed\comma{} Terrified\comma{} Enraged, hasImmune = 1, immune = Poison\comma{} Psychic, abilities = 
	
	\ability{Mechanical Mimic}{The \name{} is constructed in the image of a base non\minus{}magical creature, and so all statistics marked with a `?' are replaced with those of the base creature. The \name{} also recreates all of the proficiencies, senses, immunities, abilities and actions of the base creature. Any abilities listed in this stat block are assumed to be {\it in addition} to those of the base creature. Where a conflict arises, the \name{}\apos{}s stats are used instead. }
	
	\ability{Master\apos{}s Might}{The numerical value of all attributes (except Evil), increases by 1 for every 3 Mechanist levels above 8th possessed by the Creator. The to\minus{}hit values of attacks, as well as DV values for Resists,  and damage of the \name{}\apos{}s attacks also increase by 1.}

	
	, hasComprehend = 1, comprehend= Its Master\apos{}s language.
	}

	
	At 13th level, and again at 18th, you learn to make some changes to the internal mechanism of your Clockwork Pet. At each of these levels, you may choose one of the following effects to permanently imbue your ally with:
	
	\begin{itemize}
		\itList{Hardy Construction}{The HP of the construct is doubled, and it takes a bonus to Block equal to one\minus{}third your Mechanist level}
		\itList{Perfect Mimicry}{Coat the outside of the construct in a realistic looking disguise \minus{} it looks exactly like the creature it is mimicking, taking advantage on all stealth checks.}
		\itList{Reinforced Claws}{The melee attacks of the creature deal an additional 1d6 slashing damage, and the damage increase associated with the {\it Master\apos{}s Might} ability is increased to 1d6.}
		\itList{Coiled Springs}{Increase the base speed of the construct by 2m, and allow it to make one additional attack per cycle.}
		\itList{Self\minus{}Destruct}{If given the order by its master, the Clockwork Pet initiates a procedure which totally destroys itself, and deals 1d8 concussive damage per Mechanist level in a 5m radius around the construct.}
	\end{itemize}
	
}

\subfeat{Mechanical Mind}
{
	From 12th level, whenever you perform a check using your Tinkering tools, or using an item which you have created, you may choose to use your \attInt{} modifier instead of the specified one. 
}

\subfeat{Master Worker}
{
	Whenever you perform an artificing check, if the result is less than your Mechanist level, you may use that value instead. 
}
