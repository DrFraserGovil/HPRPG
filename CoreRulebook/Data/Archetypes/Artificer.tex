

\chapter*{Artificer}
\addcontentsline{toc}{section}{Artificer}
An intro bit of text 

%%archBegin
\archetype{name=Artificer, hp=8, fp=8, armour=Medium Armour, tool=Choose one from Runic Tools\comma{} Alchemy Gear and Tinkering tools\comma{} plus any two additional tools., disc=Choose one discipline of your choice\comma{} and a further two from Elemental\comma{} Alteration\comma{} Conjuration \& Warding., weapon=Simple Weapons, prof=Choose any two from Arcane\comma{} History\comma{} Logic\comma{} Nature\comma{} Willpower and Investigation., equip=Wand\comma{} a Scholar\apos{}s pack containing the tools selected above., memorised=Any four from the basic spells table., listIName =Ad\minus{}hocs, singleListMode = 1, expertI = 2, maxspellI = Beginner, bonusI = Artificer\apos{}s Specialisation\comma{} Ad\minus{}hoc Creations, listI_I= 2, expertII = 2, maxspellII = Beginner, bonusII = Tools of the Trade, listI_II= 2, expertIII = 2, maxspellIII = Beginner, bonusIII = Specialisation Feature I, listI_III= 3, expertIV = 3, maxspellIV = Novice, bonusIV = Tweak Effect, listI_IV= 3, expertV = 3, maxspellV = Novice, bonusV = Appraising Eyes, listI_V= 4, expertVI = 3, maxspellVI = Novice, listI_VI= 4, expertVII = 3, maxspellVII = Novice, bonusVII = Specialisation Feature II, listI_VII= 5, expertVIII = 4, maxspellVIII = Adept, bonusVIII = Artificial Ally, listI_VIII= 5, expertIX = 4, maxspellIX = Adept, listI_IX= 6, expertX = 4, maxspellX = Adept, bonusX = Mobile Laboratory, listI_X= 6, expertXI = 4, maxspellXI = Expert, listI_XI= 7, expertXII = 5, maxspellXII = Expert, bonusXII = Specialisation Feature III, listI_XII= 7, expertXIII = 5, maxspellXIII = Expert, bonusXIII = Ally Improvement, listI_XIII= 8, expertXIV = 5, maxspellXIV = Expert, listI_XIV= 8, expertXV = 5, maxspellXV = Master, bonusXV = Rapid Work, listI_XV= 10, expertXVI = 6, maxspellXVI = Master, listI_XVI= 10, expertXVII = 6, maxspellXVII = Master, bonusXVII = Specialisation Feature IV, listI_XVII= 12, expertXVIII = 6, maxspellXVIII = Master, bonusXVIII = Ally Improvement II, listI_XVIII= 12, expertXIX = 6, maxspellXIX = Ascendant, listI_XIX= 14, expertXX = 7, maxspellXX = Ascendant, bonusXX = Specialisation Feature V, listI_XX= 15, shortmode = 0}
%%archEnd



\section*{Acquired Feats}

\feat{Artificer\apos{}s Specialisation}
{
	At first level, when you become an artificer, you must decide which path you wish to follow \minus{} your {\it Specialisation}. You may choose to become an Alchemist, an Enchanter, or a Mechanist. 
	
	Each choice grants you proficiency in the relevant tools (Alchemy Gear, Runic Tools and Tinkering Tool, respectively), as detailed in the {\it Proficiencies} section of the archetype outline. In addition your choice of Specialisation grants you additional Features at 3rd, 7th, 12th, 17th and 20th level. These are detailed at the end of the Archetype description. 
}

\feat{Ad\minus{}hoc Creations}
{
	At first level, you gain the ability to use an Ad\minus{}hoc creation to assemble something useful on the fly. 
	
	Whilst the normal act of creation (either in an alchemical, mechanical or magical sense) requires careful, controlled actions \minus{} preferably in a laboratory or workshop, those who call themselves Artificers pride themselves on their ability to manufacture items on the fly. Not quite as refined as their normal produce, these creations are termed {\it Ad\minus{}hocs}. The number of Ad\minus{}hocs you are able to produce at each level is given by the relevant column in the Archetype table. This number resets on a Long Rest. 
	
	The type of Ad\minus{}hocs you are able to produce is determined by your specialty, and described in detail at the end of the Archetype table. 	
}

\feat{Tools of the Trade}
{
	At second level, you choose one set of tools with which you are proficient. When using these tools in future you may double your Expertise bonus.  
}

\feat{Tweak Effect}
{
	At 4th level, you gain the ability to slightly modify the effect of an existing item, potion or enchantment. 
	
	This might be a purely superficial change, such as altering the colour of a potion from red to green or changing an enchanted sword to glow with blue flames rather than red. Alternatively you may attempt to add some minor tweaks to the functioning of the object \minus{} perhaps the potion only works on Goblins, or the clockwork device discharges when turned upside down. 
	
	Tweaks should be minor in nature \minus{} attempts to alter the nature of the object too much (especially if it was not originally one of your creations) can have disastrous effects, including the destruction of the item, and collateral damage.  
}

\feat{Appriasing Eyes}
{
	At 5th level, your eyes have become accustomed to recognising the work of other artisans of crafting. 
	
	You can tell at a glance if an item is enchanted, alchemical or mechanical in nature, as well as determine the rough functioning of the object (i.e. is it intended to heal or harm, reveal or hide \minus{} but no specifics). 
	
	You also gain advantage on any Investigation checks to further discern the nature of such an object.  
}

\feat{Artificial Ally}
{
	At 8th level, you gain the ability to construct an artificial being, either alchemical, magical or mechanical in nature, to aid you on your journey. 
	
	The types of Ally you may construct are detailed in the subtype descriptions. 
	
	At 13th and 18th level, you unlock additional improvements to your Artificial Ally, as detailed in the relevant descriptions. 
}

\feat{Mobile Laboratory}
{
	At 10th level you gain the ability to turn any rudimentary work area into a functional Laboratory or Workshop for your crafting purposes. 
	
	If you have access to a sheltered area, you may spend 3 hours setting it up just so, in order to designate it as your workspace. Any artificing checks made within this space can be made with advantage. You may only have one such workspace set up at any one time. 
}

\feat{Rapid Work}
{
	At 15th level, you are able to complete even complex artificing tasks much quicker than before. 
	
	When undertaking an artificing check, roll a d6. Divide the time necessary to complete the action by the result of this device. 
}

\section*{Alchemist}

An alchemist is an Artificer who specialises in the mixing of magical potions, and chemical synthesis. 

\subfeat{Alchemical Ad\minus{}Hocs}
{
	An Alchemical Ad\minus{}hoc takes the form of a half\minus{}mixed potion or salve which the Alchemist has prepared from common ingredients. When required, they can use a minor action sprinkle in an ingredient to finalise the `mixing' of this potion. Lacking the complex brewing and steeping phases typically used in potion mixing, a potion prepared in this fashion is usually much weaker than their standardised counterparts \minus{} although they are much more flexible. 
	
	 When invoking an alchemical ad\minus{}hoc, you must tell the GM what effect you are trying to replicate. In order to maintain the simple nature of an Ad\minus{}hoc, you should limit yourself to one or two words: `heal', `explode', `hallucinate', though this is only a guideline. 
	 
	 The power of your ad\minus{}hoc creations increases with your Alchemist level. You should work with your GM to determine the exact effect, within the limits of your current powers. A rough guide is that alchemical ad-hocs cannot heal or do damage exceeding 1d6, increasing by 1d6 every three alchemist levels, and that they should not outperform standard potions which replicate this effect. If a potion requires a Resist check, the DV is set by 8 + your Expertise modifier. 
	 
	 After an Alchemical Ad-hoc is created, you may treat it as any other potion, with the exception that they lose their potency and become worthless one minute after their creation. 
}

\subfeat{Ingredient Intuition}
{
	At 3rd level, your experience with alchemical ingredients allows you to make inferences about the use of certain ingredients in your potions. Whilst handling a sample of such an ingredient, you may perform a DV 10 Nature (or Un-nature, if applicable) check in order to learn one of the following bits of information:
	
	\begin{itemize}
		\item The recipe for one potion in which this is a key ingredient
		\item One recipe (if any) which you already know, for which this is an optional ingredient
		\item The effect which would result were this ingredient added to a specific recipe
	\end{itemize}
	
	Every subsequent check increases the DV by 5. On a failure, you can learn nothing new about this ingredient for 24 hours.
}

\subfeat{Conservative Mixer}
{
	At 7th level, you have honed your mixing skills such that you can often get by without using up a full sample of the ingredients. 
	
	After a successful potion-making check, perform the same check again (a `recovery check'), and use the following table:
	
	\begin{center}
	\begin{rndtable}{c p{5 cm} }
		\bf Check	&	\bf Ingredients Recovered
		\\
		$< 10$	&	None
		\\
		10-14	&	One common, or two abundant samples
		\\
		15-19	&	One uncommon, or two uncommon samples
		\\
		20-24	&	One rare, or two uncommon samples
		\\
		25+	&	Two rare samples
	\end{rndtable}
	\end{center}
}

\subfeat{Alchemical Ally: Homunculus}
{
	At 8th level, all Artificers learn to construct an {\it Artificial Ally}. As an Alchemist, your construct takes the form of an Alchemical Homunculus, a tiny living being created during a complex potion-mixing exercise, in which the very forces of life are invoked. The exact appearance of the homunculus can be determined at the moment of creation: most alchemists prefer to go with a winged fairy-like creature, though some exotic choices include floating, semi-sentient potion vials, or simply writhing blobs of fluid. 

	The creation of a Homonculus takes 2 hours and \galleon{1} worth of supplies. Homunculi are devoted to their master and will follow all commands given to them. Outside of this, the Homunculus is considered an independent being, and takes actions with the semblence of free will. 
	
		
	\boxOnlyBeast{name = Alchemical Homunculus, mind = Semi-Sapient, category = Construct, summary = Alchemically-created being, hp = 3$\times$Alchemist level health, block = 8, dodge = 16, speed = 2m (walking)\comma{} 8m (flying), fit = 5 (\minus{}3), prs = 17 (+3), spr = 14 (+2), chr = 8 (\minus{}1), int = 7 (\minus{}2), pcp = 13 (+1), pow = 5 (\minus{}3), evl = Equals Master, rating = III, abilityBlock = 1, hasAbilities = 1, hasSkills = 1, size = 10cm, skills = Chicanery (+7)\comma{} Stealth (+7), senses=Darkvision, hasSenses = 1, hasImmune = 1, immune = Poison, hasConditionImmune = 1, conditionImmune = Poisoned, abilities = 
	
	\ability{Master\apos{}s Might}{The numerical value of all attributes (except Evil), increases by 1 for every 3 Alchemist levels above 8th possessed by the Creator. The to-hit values and damage of the creature\apos{}s attacks also increase by 1.}
	
	\ability{Free Will}{The Homunculus is able to take actions inside and outside of combat like any other sapient being.}
	
	, hasComprehend = 1, comprehend = Its Master\apos{}s language., hasActions = 1, actions = 
	\ranged{Poisonous Spittle}{+7}{4m}{A glob of toxic fluid is spat into the face of the opponent, dealing 1d6+2 poison damage.}
	
	\melee{Tiny Claws}{+2}{Deals 1d4 slashing damage}
	
	\ability{Junior Assistant}{The Homunculus may use its masters Ad-hoc slots as its own, creating a rudimentary potion following the normal rules. The Homunculus must return to its master in order to recharge this ability.}
	}
	
	At 13th level, and again at 18th, you learn to make some changes to the formula you use to create the homunculus. At each of these levels, you may choose one of the following effects to permanently imbue your Homunculus with:
	
	\newcommand\itList[2]{\item {\bf #1:} #2}
	
	\begin{itemize}
		\itList{Genetic Splicing}{Upon creating a homunculus, expend a sample of an organic ingredient of Rare-level rarity or less into the mixture. The homunculus grows samples of this ingredient on its body, granting you one sample per day. Every time you create a new homunculus, you may choose a new ingredient.}
		\itList{Vicious Spittle}{Increase the damage dealth by the {\it Poisonous Spittle} attack to 2d6, and increase the damage by 1d6, rather than +1 as the Alchemist power increases.}
		\itList{Rapid Attacks}{The homunculus gains the ability to make up to three attacks per cycle.}
		\itList{Stronger Homunculus}{The HP of the homunculus is doubled.}
		\itList{Proficiencies}{Choose up to 4 attribute, tool or weapon proficiencies to grant the homunculus.}
		\itList{Lab Assistant}{Train the homunculus to act as your lab assistant, continually providing the {\it Help} action when you are performing Alchemy checks.}
	\end{itemize}
	
}

\subfeat{Toxin Tolerance}
{
	At 12th level, your continued exposure to noxious fumes has rendered you somewhat immune to them. You are considered Resistant to Poison damage, and you take Advantage on any Resist check to avoid taking the {\it Poisoned} status effects.
}

\subfeat{Pure Mixtures}
{
	At 17th level, your mastery of the potioneering arts is such that your mixtures are free from imperfections, and you can even remove imperfections from pre-brewed mixtures. 
	
	Your potions never gain the {\it Flawed Batch} status, and you may spend 5 minutes with your Alchemy Gear to remove the {\it Flawed Batch} status from any other potions you possess.
}

\subfeat{Maximum Effectiveness}
{
	At 20th level, your Potions and Alchemical Ad-hocs are always the most potent form they can be. 
	
	Whenever a potion you have created triggers a dice roll, treat the result as the maximum possible value.
}

\section*{Enchanter}


\subfeat{Ad\minus{}Hocs}{ho}

\subfeat{3}{3}

\subfeat{7}{7}

\subfeat{Ally}

\subfeat{12}{12}

\subfeat{17}{17}

\subfeat{20}{20}


\section*{Mechanist}


\subfeat{Ad\minus{}Hocs}{ho}

\subfeat{3}{3}

\subfeat{7}{7}

\subfeat{Ally}

\subfeat{12}{12}

\subfeat{17}{17}

\subfeat{20}{20}


\def\sign{{\it signum}}
\def\signs{{\it signii}}
\feat{Artificer's Augmentations}{From first level, you learn how to use {\it Augmentations}. 


An {Augmentation} is a temporary, rather weak enchantment that can be placed on an item by pressing a wax \sign{} into it. Each \sign{} has a runechain already inscribed upon it, and as the artificer begins the augmentation process they describe how those runes should be interpreted, as they would in the normal enchanting process. 

If the artificer doesn't exceed the capabilities of the \sign{}, then the item is temporarily imbued with the described effect. An augmentation typically lasts for one hour, with minor degradation to the effect quality as it wears off. 

Each \sign{} takes 12 hours to recharge, during which time it cannot be used again. You may use multiple \signs{} inscribed with the same runechain to produce different effects, provided they would fall under the same `umbrella', as described on page \pageref{S:Enchanting}. You cannot place a \sign{} on item which already possesses an augmentation, but you can augment traditionally enchanted items. 


The number of \signs{} that you may have at any one time is indicated in the \signs{} column of the class table. You may take 6 hours to produce a new set of \signs{}, with your chosen runes inscribed upon them. When you do so, your old set crumbles into dust and any active augmentations wear off. 

The effects produced by augmentations will always be less potent than those which you are able to produce through the traditional enchantment ritual. As an example, whilst a Beginner\minus{}level enchanter could use the runechain \rune{\velox\genero\ignis} to produce a Sword of Fire capable of inflicting an additional 1d6 Fire damage, a sword augmented with this runechain would be limited to 1d4 fire damage. As your spellcasting (and hence enchanting) abilities increase, so to do your augments. 
 } 


\feat{Rune Experimentation}{At second level, you learn to dedicate a number of hours to simple brute\minus{}force experimentation, guided by your enchanter's intuition. Upon doing so, you may learn a new rune that you do not yet know. After an hour of work, perform a DV 18 enchanting check. You may repeat this check once an hour until you succeed, for up to 6 hours in a row.

If you succeed, you choose a new rune to memorise from those you have not yet learned. Generally, you may only choose to learn a Legendary Rune if you already know all of the Mystical runes of the same category.  

This ability may only be used again after a Long Rest.
}

\feat{Expert Enchanter}{From 3rd level, you may double your expertise bonus on enchanting checks, and enchanting rituals take half as long as normal.

In addition, whenever you fail an enchanting check, you get the chance to `rescue' the enchanting with a second check.
 }

\feat{Renew Bindings}{AT 3rd level, you learn how to re\minus{}energise an existing nexus, effectively `recharging' a the enchantment in an item. The ritual takes approximately 1 hour, and costs 10FP to perform. 

You may `recharge' a number of items in a single ritual equal to half your enchanter level. 
}

\feat{Appraising Eyes}{At 5th level, your eyes have become accustomed to the magical aura of enchanted items. You can tell at a glance if an item is enchanted, and may take a major action to perform an investigation check. On a success, you may discern the runechain (but not the exact effect) present on an item.}

\feat{Enchantment Affinity}{At 7th level, you allow the runes to shape your understanding of the item as you undergo the enchanting process, and allow the runes to guide you a you use the item. As a result, you are considered proficient in any weapon or armour that you have enchanted personally.

At 18th level, this understanding extends even to items that others have enchanted: you are considered proficient in all enchanted weapons and armour. 
}

\feat{Tweak Enchantment}{Upon reaching 8th level, you gain the ability to place small runes at specific points along the magical nexus of an existing magical item, subtly altering the effects. 

Tweaks can be simple aesthetic changes (i.e. change the fire flickering along the blade from red to blue), provide exceptions (i.e. sleep effects do not work on blonde individuals) and other such minor effects. Attempting to alter the effects too much from their intended purpose can fragment the magical nexus, destroying the item, and probably taking out a few nearby buildings. 

This action takes 10 minutes to complete, and cannot be used again until completing a Short Rest.}

\feat{Imbue Sentience}{The creation of a true Mind, a being possessing consciousness and original thought is a closely guarded secret amongst the Artificer's guilds. By 10th level, you have learned these secrets for yourself. 

The runechains for such a feat are horrifically long and complicated, necessitating a DV20 enchanting check to produce and 24 hour of work. At the end, however, you produce an artificial sentient being. 

If you had previously created such an artefact, that consciousness transfers into the new item, with the old housing crumbling into dust. 

The consciousness within the item can percieve the outside world with perfect darkvision, and can hear as well as any human, and has a perfect memory. It also has a limited amount of control over its physical form \minus{} often manifesting a face with which to talk by selectively crumpling the fabric it resides inside, or appearing in the imperfections of a gemstone, for example. 

When manifesting a face, the sentience can communicate verbally, and it may always communicate telepathically with a being in contact with the item it resides within. The sentience is created with a friendly disposition towards it creator, and as such may dispense advice and knowledge, or warn them of unseen threats. 

 }

\feat{Runecaster}{At 12th level, you begin to realise that enchanting (and the use of runes) is no different than normal spellcasting, and you learn to utilise your knowledge of the Enchanting Runes to form a magic spell from the endless chaos of primordial magic. 

As with the usual enchanting process, you must trace the runes out in the air over the course of a major action, describing what effect you would like this runechain to have. The GM then decides the spellschool and the difficulty of the described spell, based on the magnitude of the effect you are trying to create. You must then perform a casting check to realise the runecast. 

Runecasting is almost always weaker than a spell cast using the normal methods, though it provides much greater flexibility. You cannot runecast to exactly replicate the effects of an existing spell. 

%After successfully performing a Runecast, you may spend 6 hours translating the runechain into a conventional incantation\minus{}based spell, with the discipline and DV set by the GM. You may then memory\minus{}cast this spell, and others may book\minus{}cast it, as they would any other spell.  
}

\feat{Networked Augmentations}{At 14th level, you learn how to modify your Augmentations to form a psychic network, along which both messages and Sentiences can travel. 

Any being touching an item currently bearing an active augment (anywhere in the multiverse) may communicate with you, or with any other being touching such an item, and you may communicate with them. 

If you bear an item containing a Sentience, it may also traverse this psychic network to temporarily inhabit augmented items. It will act as if it is in its `home' object, and will always return before the Augmentation wears off. Moving along the network takes a minor action. 
}

\feat{Empowered Sentience}{At 16th level, you learn how to modify your sentience\minus{}inducing runes to allow the channeling of magical power. You may teach the Sentience up to 3 Beginner or Novice\minus{}level spells which you have memorised. 

The Sentience is then able to cast these spells at will as major actions, from whichever item it is currently inhabiting. The Sentience has a +4 to accuracy checks, and has a subjugation value of 13.

Every time you re\minus{}create the sentience from scratch, it forgets the spells it learned in its previous incarnation.
}

\feat{Shared Soul}{At 20th level, you have learned one of the most dangerous secrets of all: how to cheat death.

When you create a sentience, enough of your personality and soul mingles with that of your creation for it to be viable to use the sentience to rebuild your personality after your death. 

Normally the {\it Spark of Life} spell requires a being to have been dead for less than 24 hours, and with a relatively intact body, in order for it to be viable for the soul to be clinging on. 

However, if you die whilst connected to your Sentience through the psychic network, your soul may reside in the network for a year and a day, before it moves on to the other side. As long as the enchanted item remains intact, your body can be repaired and your soul returned to it within this timeframe. 

After using this skill, your soul can become fragile. You may only use this skill to cheat death once.

Unlike Horcruxes, no dark or unspeakable magic is involved, and your soul remains intact throughout. You cannot permanently cheat death \minus{} only delay it slightly.  
 }
