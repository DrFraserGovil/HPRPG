\documentclass[CoreRulebook.tex]{subfile}
\newpage
\begin{strip}
\setlength{\parskip}{4pt}

\section{Artificer}

Their artful hands always busy, their focus on the emerging magical item in front of them, and their magics being channeled in incomprehensibly complex ways, the Artificer is the master of item creation. They are experts at using their hand to mold both material, and magic -- often mixing the two in ways that have never been done before. Some Artificers produce their items for sale, whilst some hoard their creations for their own personal use.

Artificers may choose to focus their craft into two divergent fields: the {\bf Spellbinders} focus their might into imbuing physical items with magical effects, whilst the {\bf Alchemists} use their craftsmanship to produce magnificent potions and deadly poisons. 

%%archBegin
\archetype{Artificer}{Spellbinder}{Alchemist}{0}{featureI=Artisan{\apos}s Intuition, featureII=Appraising Eyes, featureIII=Tool Proficiency, alphaFeatureIII=Alteration Runes, betaFeatureIII=Attuned Nose\comma{} Natural Knowledge, alphaFeatureIV=Renew Binding, betaFeatureV=Careful Mixing, featureVI=Reliable Supplier, alphaFeatureVI=Disenchantment Ritual, betaFeatureVII=Poison Resistance, betaFeatureVIII=Proven Recipes, featureIX=Renowned Craftsmanship, alphaFeatureIX=Enchanting Specialty, betaFeatureX=Favoured Effect, featureXI=Workshop, alphaFeatureXII=Multiple Bindings, betaFeatureXII=Purity Filters, alphaFeatureXIII=Soul Attunement, featureXIV=Artisan{\apos}s Intuition II, betaFeatureXIV=Favoured Effect II, alphaFeatureXV=Construct Mastery, betaFeatureXV=Poison Resistance II, featureXVII=Appraising Eyes II, betaFeatureXVIII=Production Line, alphaFeatureXX=Imbue Sentience, betaFeatureXX=Alchemic Construct}%%archEnd

\end{strip}

\subsection{Starting Equipment}

Artificer's start with:
\begin{itemize}[itemsep=0em]
	\item a Scholar{\apos}s pack
	\item Protective cloak (AC +4, provides 20\% immunity to airborne effects)
	\item Protective Gloves (AC +2, provides 50\% immunity to contact effects) 
	\item  a Wand (roll on the wand table to determine 
\end{itemize}
%\newpage
\subsection{Starting Spells}

In addition to choosing 1 spell from the {\it Basic Spells} set on page \pageref{S:BasicSpells}, Artificers also get the following spells:
{\it 
\begin{itemize}[itemsep=0em]
	\item Potion Mixing Spell 
	\item Enchantment Ritual
	\item Identification Charm
\end{itemize}
}
\newpage
\subsection{Archetype Features}

\feat{Artisan{\apos}s Intuition}

At 1st level, Arcane proficiency gets +1 bonus. 

From 14th level, you may use your Arcane Wisdom during artificing ignoring the once-per-day rule. Non-artificing Arcane Wisdom rules are unaffected by this feature. 
\jump
\feat{Appraising Eyes}

From 2nd level, your experience in artificing means that you can get an insight into the effects of an item. Perform a 1d20 INT (arcane) check (DV 15) to learn the major effect of a magical item or potion. 

At 17th level, your experience increases such that you now automatically detect the major effect. Perform the check to instead learn {\it all} the effects. 
\jump
\feat{Tool Proficiency}

At 3rd level you may choose a proficiency in either:
{\it 
\begin{itemize}[itemsep=0em]
	\item Runic Tools
	\item Chemistry Equipment
	\item Protective Gear
\end{itemize}
}
If you do not already posses a set of your chosen tools, acquire it.
\jump
\feat{Reliable Supplier}

From 6th level, you cultivate a relationship with a supplier of goods. Once per week, if you can deliver a message to them, they will send you up to 5 supplies for your artificing at 20\% below the stated price. 
\jump
\feat{Renowned Craftsmanship}

By 9th level, news of your skill has spread, and your reputation alone makes your work more valuable.. Get +1 bonus to Persuasion skill, and merchants will purchase your wares at 50\% above marked value. 

\jump
\feat{Workshop}

At 11th level, you have accrued enough equipment and materials to construct a high-quality workshop, and you may specify the location. Artificing checks whilst  inside your workshop get check-advantage. Your workshop may also be assumed to be stocked with common ingredients and equipment needed for your craft. 
\jump
\subsection{Spellbinder Features}

\feat{Alteration Runes}

At 3rd level, you gain the ability to use small runes placed at specific nexus points along an existing magical item, subtly altering the effects. Alterations can be aesthetic (i.e. change fire from red to blue), provide exceptions (i.e. sleep effects do not work on blonde individuals) and other such minor effects. Attempting to alter the effects too much can fragment the magical network in the item, causing an Enchanting Mishap. This action takes 5 hours. 

You may perform 1 additional alteration for every 3 levels above 3rd. 
\jump
\feat{Renew Binding}

From 4th levl, at a cost of 6FP, you may `recharge' a magical item. This is a major action. 
\jump
\feat{Disenchantment Ritual}

From 6th level, you gain the ability to disenchant a magical item. The item needs be visible during the entire ritual, which takes 2 minutes (10 combat rounds) to complete. The ritual also requires a supply of Ash to complete (this is used up). At the end of the ritual, perform a SPR (arcane) Magic Resist check (CV determined by item power) and cast the Ash over the target item.

If the check succeeds (and the ash touches it), the item has the enchantment removed. If it fails, suffer an Enchanting Mishap on one of your own enchanted items. 
\jump
\feat{Enchanting Specialty}

At 9th level, Choose any enchanting effect that you have previously used. Enchanting checks to place your chosen effect on an item get a +2 bonus. Specialty can be changed through 4 weeks of dedicated work. 
\jump
\feat{Multiple Bindings}

From 12th level, you may add more than one effect onto an enchanted item. An individual enchantment ritual must be carried out for each additional effect added on. 
\jump
\feat{Soul Attunement}
From 13th level, by infusing part of your essence into the enchanting ritual, you may be assumed to be proficient with any weapon or armour that you have enchanted. This proficiency applies only to that specific item. 
 
\jump
\feat{Construct Mastery}

Starting at 15th level, you may create Constructs by gathering the requisite parts and spending one week enchanting them. Constructs are permanent entities that can only be destroyed by physically destroying them, or a DC 20 Disenchantment Ritual. Constructs are unwaveringly loyal to their creator. 
{\it 
\begin{itemize}[itemsep=0em]
	\item Crystal Golem (600kg of diamond)
	\item Clay Golem (300kg of soil, 100kg of water and a large diamond)
	\item Clockwork Warrior (100kg of copper or bronze and 5 rubies)
	\item Flesh Golem 
	\item Iron Golem (3 tonnes of iron, heated to 6000 degrees and 1 litre of mercury)
	\item Stone Golem (2 tonnes of stone or rocks and a pogrebin shell)
	\item Spider Construct (60kg of copper or bronze and a single emerad)
\end{itemize}
}
At 15th level, you may have one construct active. This increases by one for every two Spellbinder levels taken above 15th. 

\jump
\feat{Imbue Sentience}

At 20th level, by adding Unicorn Blood into the enchantment vat, the items you create are imbued with sentience. The item may move and warp its shape at will, as well as talk. It is created with a positive attitude towards its creator, but otherwise is treated as an indepenent NPC with a personality determined by the GM.

\subsection{Alchemist Features}



\feat{Attuned Nose}

From 3rd level, when you encounter a new potion ingredient, roll a d4. Learn that effect of the ingredient (i.e. a 1 learns the first effect etc.). 
\jump
\feat{Natural Knowledge}

From 3rd Level, your Flora \& Fauna proficiency gets +1 bonus.
\jump
\feat{Careful Mixing}

From 5th level, get a bonus on all mixing checks equal to one-third your Alchemist level.
\jump
\feat{Poison Resistance}

By 7th level, you have been exposed to so many toxic fumes that you have developed an immunity to all but the most ferocious poisons. You have a 4-point `poison AC'. Any poison effect less than 4 points does zero damage. 

At 15th level, this increases to 10 points. 
\jump
\feat{Proven Recipes}

At 8th level, when you successfully mix a potion and determine its effects, you may record this recipe as `proven', and mix it again without performing a check. The number of recipes that you may have is equal to half of your Alchemist level. 
\jump
\feat{Favoured Effect}

At 10th level, and again at 14th level, you may designate one potion effect as your `specialism'. Potions with this effect are twice as effective, and can be sold for twice the market value. 
\jump
\feat{Purity Filters}
\jump
\feat{Production Line}
\jump
\feat{Alchemic Construct}















