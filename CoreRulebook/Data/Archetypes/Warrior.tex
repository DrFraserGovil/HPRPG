

\chapter*{Warrior}
\addcontentsline{toc}{section}{Warrior}
An intro bit of text 

%%archBegin
\archetype{name=Warrior, hp=10, fp=6, armour=All Armour and Shields, tool=None, disc=Choose any three from Hexes\comma{} Curses\comma{} Elemental\comma{} Necromancy\comma{} Psionics\comma{} Warding \& Conjuration., weapon=All Weapons, prof=Choose any two from Strength\comma{} Speed\comma{} Vitality\comma{} Acrobatics\comma{} Willpower\comma{} Observation \& Intimidation, equip=A Wand\comma{} a Fighter’s pack including a weapon of your choice\comma{} and a set of Warrior Robes., memorised=Any three from the basic spells table., expertI = 2, maxspellI = Beginner, bonusI = Fighting Paradigm\comma{} Paradigm Feature I, expertII = 2, maxspellII = Beginner, bonusII = Through Gritted Teeth, expertIII = 2, maxspellIII = Beginner, bonusIII = Combat Stratagem, expertIV = 3, maxspellIV = Novice, bonusIV = Paradigm Feature II, expertV = 3, maxspellV = Novice, expertVI = 3, maxspellVI = Novice, bonusVI = Stratagem Improvement I, expertVII = 3, maxspellVII = Novice, expertVIII = 4, maxspellVIII = Adept, bonusVIII = Paradigm Feature III, expertIX = 4, maxspellIX = Adept, bonusIX = Stratagem Improvement II, expertX = 4, maxspellX = Adept, expertXI = 4, maxspellXI = Expert, bonusXI = Firm Grip, expertXII = 5, maxspellXII = Expert, expertXIII = 5, maxspellXIII = Expert, bonusXIII = Paradigm Feature IV, expertXIV = 5, maxspellXIV = Master, bonusXIV = Stratagem Improvement III, expertXV = 5, maxspellXV = Master, expertXVI = 6, maxspellXVI = Master, bonusXVI = Critical Striker, expertXVII = 6, maxspellXVII = Master, expertXVIII = 6, maxspellXVIII = Ascendant, bonusXVIII = Stratagem Improvement IV, expertXIX = 6, maxspellXIX = Ascendant, expertXX = 7, maxspellXX = Ascendant, bonusXX = Paradigm Feature IV, shortmode = 0}
%%archEnd



\newpage
\section*{Acquired Feats}

\feat{Fighting Paradigm}{At first level, you select the kind of warrior you wish to be, the combat paradigm that you follow. You may choose from {\it The Way of the Blade}, {\it The Way of the Wand} and {\it The Way of the Shield}, all described at the end of the Feats description. 

Your paradigm choice grants you features at 1st, 4th, 8th, 13th and 20th level. 
}

\feat{Through Gritted Teeth}{Your years of training allow you to draw on a deep well of inner strength, to fortify your mental or physical abilities. 

As a minor action, you may draw on this well to restore either your HP or FP equal to 1d10 + Warrior level. 

You can only use this feature again after completing a short rest.  
}
\feat{Combat Strategem}{Upon reaching 3rd level, you gain access to a number of fighting techniques known as {\it stratagems}. 

A great warrior does not win a fight simply by throwing increasing amounts of firepower at the problem: they must choose when and how to strike, using a large number of tools and techniques at their disposal. 

You choice of fighting paradigm influences which stratagems are relevant, as detailed in the description of each paradigm. At 3rd level, you are experienced in the use of 3 stratagems. The list of available Stratagems is found at the end of the class description. 

Your ability to use these Stratagems is represented through your {\it strategy dice}. These are initially each a d8, and you possess 3 such strategy dice at 3rd level. Once per turn cycle, you may expend one of your strategy dice to execute a chosen strategem. Strategy dice return upon a short rest. 

Some Stratagems require your opponent to perform a Resist action against your physical prowess, which is enumerated through your {\it physical subjugation} value:
$$\text{Physical Subj.} = 8~ +~ \text{Expertise bonus + \attPhysShort{} {\bf or} \attFinShort{} modifier}$$
You may choose which of \attPhys{} or \attFin{} to use at any given moment. Some Stratagems are arcane in nature, and so use your normal Arcane Subjugation value. The Stratagem description should note which is to be used. 


\feat{Stratagem Improvement}

You receive improvements to your stratagems at 6th, 9th, 14th and 18th levels. 

Each time you take an improvement, you may learn an additional Stratagem, as well as swap out one of your existing stratagems for another. You may only learn a stratagem if it is available to a member of your paradigm. 

You may also choose one of the following bonuses:
\begin{itemize}
	\item Gain another two strategy dice
	\item Increase the size of your strategy dice (d8 improves to d10, which improves to d12). 
\end{itemize}

\feat{Firm Grip}{At 11th level, your grip is like iron. 

Gain check\minus{}advantage on all checks to Resist disarmament, and against all effects which attempt to break your grip. }

\feat{Critical Striker}{At 16th level, your ability to hit the target where it hurts improves drastically. You now trigger a critical strike when you roll a natural 18,19 or 20.}

\section*{Fighting Paradigms}

\subsection{\bf The Way of the Blade}

The Way of the Blade is a fighting style which leans much more heavily on mundane and physical weaponry, rather than on the arcane arts. Though the name implies bladed weapons alone, the Way of the Blade encompasses swords, hammers, spears and even archery.  


\subfeat{Extra Attack}{

At 1st level, your uncanny ability with physical weapons allows you to lash out, faster than any spellcaster can keep track of. 

When taking a full-round attack with a physical weapon, you may take an additional strike, and perform two attacks per round. Alternatively, you may perform a single strike as part of the `Quick Attack', without taking disadvantage. 
}

\subfeat{Blade Stratagems}{
From third level, you gain access to Stratagems. As a follower of the Way of the Blade, you may use the following Stratagems:

{\it 
%%BladeBegin
Disarming Strike\comma{} Battlefield Commander\comma{} Distraction Tactics\comma{} Fancy Footwork\comma{} Fortified Brace\comma{} Feint\comma{} Terrifying Onslaught\comma{} Lunge\comma{} Riposte\comma{} General’s Eye } and {\it Trail of Blood
%BladeEnd
}
}
\subfeat{Weapon Excellence}{
As a level 4 follower of the Way of the Blade, you have trained with all kinds of weapons, to increase your mastery. However, you find that one particular weapon type which truly fits in with your style. 

Choose a weapon (i.e. shortsword, battleaxe, longbow). You may double your proficiency bonus on any accuracy check made with that weapon-type. 

With a large amount of training, you may shift your fighting style: by spending three weeks of downtime in training, you may transfer this bonus to another weapon. 
}
\subfeat{Battle Mastery}
{
At 8th level, you continue to hone you abilities in combat, and may focus your training into one of the following areas:

\begin{itemize}
	\item {\bf Extra Attack:} take an additional extra strike when performing a major-action attack. 
	\item {\bf Mage-Killer:} whenever you attack a being in the same turn that they are casting a spell, treat them as {\it Susceptable} to your attack. 
	\item {\bf Tough Skin:} scar tissue covers your exposed skin, leaving you Resistant to slashing damage.   
	\item {\bf Weapon Focus:} you may use your weapon as a magical focus, allowing you to channel spells through it, in place of a wand. You may not use the extra attack feature on the same turn as casting a spell, but you no longer need to switch between a weapon and a wand. 
\end{itemize}
}

\subfeat{Rapid Strike}{
At 13th level, your weapon attacks continue to increase in speed, until your hands are a blur. Gain an two addional extra strikes when performing a major-action attack with a physical weapon. 

In addition, when performing a `Quick Attack', you may perform half of your total full turn-attacks (rounded up) without taking disadvantage on the accuracy checks. 
}

\subfeat{Blademaster}{
At 20th level, you are amongst the greatest warrior to have walked the path of the Blade. 

You may choose from the following abilities:
\begin{itemize}
	\item {\bf Extra Attack}: add one final extra strike into your maelstrom of attacks
	\item {\bf Magical Attack}: you may cast a Novice level or below spell with your weapon, every time you land an attack with your weapon, to augment your attack. The spell cannot be a ritual spell, and this requires you to have taken the {\it Weapon Focus} skill at 8th level. 
	\item {\bf Undefeatable}: Whenever you fall to 0HP, you may expend all your remaining FP to restore an equal amount of health to yourself. This ability can be used only once per long rest.
\end{itemize}
}

\newpage
\subsection{\bf The Way of the Wand}

The Way of the Wand is a fighting style which relies on offensive spellwork \minus{} hexes and curses, mostly  \minus{} to subdue an opponent. Often the most flashy and impressive duelists follow the way of the wand. 

\subfeat{Powerful Spells}
{
	At first level, the Way of the Wand teaches you how to focus additional power into your spells, causing them to do more harm. You may add your Power modifier to any spell's damage check.   
}

\subfeat{Wand Stratagems}


From third level, you gain access to Stratagems. As a follower of the Way of the Wand, you may use the following Stratagems:

{\it
%%WandsBegin
Battlefield Commander\comma{} Fancy Footwork\comma{} Fortified Brace\comma{} Feint\comma{} Terrifying Onslaught\comma{} Riposte\comma{} General’s Eye\comma{} Trail of Blood } and {\it Masked Incantation
%WandsEnd
}

\subfeat{Combat Magic}
{
	At fourth level, you discover that you have an affinity for some of the spells used in combat magic.  You may double your expertise for spellcasting and accuracy checks for spells in one of the following disciplines: {\it  Curses, Elemental, Hexes, Necromancy} or {\it Psionics}. 
}
\subfeat{Combat Focus}
{
	From 8th level, your mind becomes focussed when in combat, your years of training means that spells come almost reflixively to you. 
	During a combat encounter, spells cost 50\% less FP to cast. 
}
\subfeat{Hexing on the Move}
{
	From 13th level, your familiarity with combat magic means that your spellcasting efforts are unaffected when being on the move. 
	
	When using a quick attack to cast a spell, you suffer no penalties to accuracy, and your opponent gets no bonus to Resisting, as long as your other minor action is a movement. 
}
\subfeat{Duelist's Signature}
{
At 20th level, you are amongst the mightiest duellist to study the art of magical combat in the Way of the Wand. Along the way, you have become so accustomed to a certain spell, that - even though it isn't the most powerful - you feel it represent's your fighting style.  

Chooe a Novice level spell from a disicpline you are proficient in. You may cast this spell once per turn as a free action, in addition to any other non-ritual actions you have already taken.
}

\subsection{\bf The Way of the Shield}

Followers of The Way of the Shield adhere to the basic principle that you can't fight back if you are dead. Practitioners of this art therefore prioritise defensive wards and shields, and the patience to wait behind them until the opportune moment presents itself. 

\subfeat{Protective Instinct}

From 1st level, whenever an ally within 1m of you may sacrifice your own shield this turn cycle to impose disadvantage on all accuracy checks made against that ally this turn. You must possess a shield, or have an active shield ward to use this ability, and you may not use their bonuses for yourself this round. 

\subfeat{Shield Stratagems}


From third level, you gain access to Stratagems. As a follower of the Way of the Wand, you may use the following Stratagems:

{\it 
%%SieldBegin
Battlefield Commander\comma{} Distraction Tactics\comma{} Fancy Footwork\comma{} Fortified Brace\comma{} Extend Shield\comma{} Explosive Defence\comma{} General’s Eye\comma{} Trail of Blood } and {\it Masked Incantation
%SieldEnd
}

\subfeat{Shield Expert}
{
	From 4th level, you become a master of the protective arts. 	You may double your proficiency bonus when casting spells from the Warding proficiency, and physical shields provide an additional +2 bonus to your Block value. 
}
\subfeat{Combat Focus}
{
	From 8th level, your mind becomes focussed when in combat, your years of training means that spells come almost reflixively to you. During a combat encounter, spells cost 50\% less FP to cast. 
}
\subfeat{Patient Strike}
{
	From 13th level, your knowledge of when to hide, and when to attack allows you to make devastating attacks when you emerge from cover.  
	
	If you make no attacks for 3 combat cycles, you may use this ability to deactivate all wards, and drop your shields for this cycle, setting your Block value to its default value of 10. In return, all attacks you make are considered critical strikes until you either take damage, or re-establish your shields. 
}
\subfeat{Shield of the Gods}
{
	At 20th level, you are perhaps the most powerful warrior to follow the Way of the Shield. 
	
	At the beginning of every combat cycle, you may expend 5FP to nominate a being within range. That being is immune to all damage taken this turn cycle. This shield cannot be bestowed upon the same individual consecutively. 
}

\section*{Stratagems}

\newcommand\stratagem[3]
{
{\large \textbf{\textit{#1}}}: #2

Available to members of the {#3}. \\~\\ 
}

%%StratBegin
\stratagem{Disarming Strike}{When you hit a target with a physical weapon attack\comma{} you may expend a strategy dice to force the opponent to perform a Strength Resist check against your physical subjugation value. On failure\comma{} an item of your choice is sent spinning out of their hand to land 1d4 metres away.}{Blade paradigm}
\stratagem{Battlefield Commander}{Your knowledge of battlefield tactics allows you to take a minor action to give instructions to an ally in hearing range. Roll a strategy dice\comma{} and add the result to either the accuracy or the damage roll of an ally making an attack this turn cycle.}{Blade\comma{} Wand or Shield paradigms}
\stratagem{Distraction Tactics}{When you successfully hit a target with a physical weapon\comma{} or deflect a melee attack on you\comma{} you may expend a strategy dice to do something unexpected – throw dust into the air\comma{} or spit into their face. This provides a distraction\comma{} and the next accuracy roll made against the target has advantage.}{Blade or Shield paradigms}
\stratagem{Fancy Footwork}{When you take the {\it Evade} minor action\comma{} you may roll a strategy die and add this value to your Dodge statistic for this turn cycle.}{Blade\comma{} Wand or Shield paradigms}
\stratagem{Fortified Brace}{When you take the {\it Brace} minor action\comma{} you may roll a strategy die and add this value to your Block statistic for this turn cycle.}{Blade\comma{} Wand or Shield paradigms}
\stratagem{Feint}{You fake an attack this turn\comma{} to gain an advantage next turn. This turn cycle\comma{} you perform no actions besides nominating a target. Next turn\comma{} you roll a strategy dice and add the result to both the accuracy and damage checks against that target.}{Blade or Wand paradigms}
\stratagem{Terrifying Onslaught}{The fury of your attack terrifies your opponent. When you hit a creature with a physical or arcane attack\comma{} perform an Intimidation check\comma{} adding the result of your Strategy die to the roll. The creature must succeed a Willpower Resist check against this value\comma{} or become {\it Terrified} of you until the end of the next turn cycle.}{Blade or Wand paradigms}
\stratagem{Lunge}{When you make a melee attack\comma{} you may use a strategy die to lunge forward\comma{} doubling your usual reach (for normal melee weapons\comma{} this extends your reach to 2m). If the attack hits\comma{} roll the strategy die and add the value to the damage roll.}{Blade paradigm}
\stratagem{Extend Shield}{You may expend a strategy die to push magical shields and wards outwards\comma{} extending their effective range by 1m this turn cycle. You may choose to allow targets to pass through the shield\comma{} or to push them to the edge.}{Shield paradigm}
\stratagem{Riposte}{When your successfully Dodge an attack\comma{} you may expend a strategy dice to perform an additional attack action on that target.}{Blade or Wand paradigms}
\stratagem{Explosive Defence}{Whenever you successfully block an attack\comma{} either with a physical or an arcane shield\comma{} you may violently push back. The target must perform a Strength Resist check against your physical subjugation value. If they fail\comma{} they stagger backwards and take the {\it prone position}.}{Shield paradigm}
\stratagem{General’s Eye}{Your senses become heightened when in combat. You may take a minor action to perform an Observation check to spot hidden enemies and other threats or features of the environment. Roll your strategy die and add it to the result.}{Blade\comma{} Wand or Shield paradigms}
\stratagem{Trail of Blood}{When you deal piercing\comma{} slashing\comma{} or bludgeoning damage to a target you may expend a strategy die to land the blow closer to important blood vessels.  The additional bleeding isn’t enough to harm the target\comma{} but leaves a conspicuous trail of blood forcing them to take check\minus{}disadvantage on all Stealth checks.}{Blade\comma{} Wand or Shield paradigms}
\stratagem{Masked Incantation}{When casting a spell which requires a Resist check\comma{} you may expend a strategy dice to deliberately shout a different incantation\comma{} as you silently cast\comma{} or whisper the true incantation. The target takes check\minus{}disadvantage on their Resist check\comma{} due to their subverted expectations.}{Wand or Shield paradigms}

%%StratEnd

}
