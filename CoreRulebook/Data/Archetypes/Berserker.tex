
\chapter*{Berserker}
\addcontentsline{toc}{section}{Berserker}
An intro bit of text 

%%archBegin
\archetype{name=Berserker, hp=12, fp=6, armour=All Armour, tool=None, disc=Choose one spell discipline., weapon=Simple Weapons\comma{}  Bladed Weapons \& Brutish Weapons, prof=Strength and choose an additional one from : Vitality\comma{} Speed\comma{} Intimidation\comma{} Nature., equip=A wand\comma{} a fighter pack  containing a set of Hardened Furs \& either a) a greataxe\comma{} b) a greatsword  or c) 2  lightaxes., memorised=2 spells from the basic spells table., listIName =Fury, singleListMode = 1, listIIName = Endurance, doubleListMode = 1, expertI = 2, maxspellI = Beginner, bonusI = Battle Fury\comma{} Defy Exhaustion, listI_I= 1d4, listII_I= 1, expertII = 2, maxspellII = Beginner, bonusII = Berserker Ideal, listI_II= 1d4, listII_II= 2, expertIII = 2, maxspellIII = Beginner, listI_III= 1d6, listII_III= 2, expertIV = 3, maxspellIV = Beginner, bonusIV = Additional Attack, listI_IV= 1d6, listII_IV= 3, expertV = 3, maxspellV = Novice, bonusV = Honed Senses, listI_V= 1d8, listII_V= 3, expertVI = 3, maxspellVI = Novice, bonusVI = Berserker Ideal II, listI_VI= 1d8, listII_VI= 4, expertVII = 3, maxspellVII = Novice, listI_VII= 1d10, listII_VII= 4, expertVIII = 4, maxspellVIII = Novice, bonusVIII = Feral Fury, listI_VIII= 1d10, listII_VIII= 5, expertIX = 4, maxspellIX = Novice, listI_IX= 3d4, listII_IX= 5, expertX = 4, maxspellX = Adept, bonusX = Berserker Ideal III, listI_X= 3d4, listII_X= 6, expertXI = 4, maxspellXI = Adept, listI_XI= 3d4, listII_XI= 6, expertXII = 5, maxspellXII = Adept, bonusXII = Additional Attack II, listI_XII= 2d8, listII_XII= 7, expertXIII = 5, maxspellXIII = Adept, listI_XIII= 2d8, listII_XIII= 7, expertXIV = 5, maxspellXIV = Adept, bonusXIV = Berserker Ideal IV, listI_XIV= 2d8, listII_XIV= 8, expertXV = 5, maxspellXV = Master, bonusXV = Simmering Rage, listI_XV= 3d6, listII_XV= 8, expertXVI = 6, maxspellXVI = Master, listI_XVI= 3d6, listII_XVI= 9, expertXVII = 6, maxspellXVII = Master, listI_XVII= 3d6, listII_XVII= 9, expertXVIII = 6, maxspellXVIII = Master, bonusXVIII = Incredible Fitness, listI_XVIII= 1d20, listII_XVIII= 10, expertXIX = 6, maxspellXIX = Master, listI_XIX= 1d20, listII_XIX= 10, expertXX = 7, maxspellXX = Ascendant, bonusXX = Eternal Wrath, listI_XX= 2d12, listII_XX= $\infty$, shortmode = 0}
%%archEnd


\section*{Acquired Feats}

\feat{Battle Fury}
{
From 1st level, you learn to exhibit the trademark of the Beserker clans: your {\it Battle Fury}. 

You may choose to enter into your frenzied state at the beginning of your turn, at which point you roll your {\it Fury Dice}. This starts out as a 1d4, and increases in line with the `Fury' column in the Archetype table. The value of this roll is your {\it Fury Value}.


Whilst in a Frenzied State, you get the following benefits:

\begin{itemize}
	\item You add your current Fury Value to any \attPhys{} or \attSpr{} checks that you make
	\item Any damage rolls you make from a melee weapon are increased by an amount equal to your current Fury Value. 
	\item Whenever you take damage (except Psychic or Celestial), reduce the amount taken by your current Fury Value. 
\end{itemize} 

However, in such a furious state, you lose the ability to concentrate quite as effectively:
\begin{itemize}
	\item You cannot maintain concentration: you cannot cast {\it Focus} or {\it Ritual} spells whilst frenzied, you cannot book\minus{}cast spells of any kind and beings get check\minus{}advantage when performing a Resist against your Subjugation value.  
	\item Take disadvantage on Charisma checks and checks to resist the {\it Fury} effect. 
	\item Your heightened adrenaline makes aiming ranged attacks harder: take check\minus{}disadvantage on all ranged accuracy checks.
\end{itemize}


Every time you take damage (even if it was reduced to zero by your Fury), you re\minus{}roll your Fury Dice. If the value is larger than your current Fury Value, you increase your FV to this new level.

Your Battle Fury lasts until it is terminated by one of the following conditions:
\begin{itemize}
	\item Two consecutive combat rounds pass without attacking, or being attacked
	\item You fall to 0HP, or are knocked unconscious
	\item You take the {\it Charmed}, {\it Confused}, or {\it Terrified} status effects. 
\end{itemize}

When a Battle Fury is terminated, the Berserker gains an additional level of Exhaustion. 

}

\feat{Defy Exhaustion}
{
At first level, your barbarian physiology allows you to push your body further than anybody else thought possible. You gain a number of additional `Endurance Points', which increase with your Berserker level, as shown in the Archetype table. 

These Endurance points allow you to brush aside the effects of exhaustion and tiredness. Whenever you are indicated to take an additional level of exhaustion, a Berserker can instead expend an Endurance point, to mitigate this effect. 

When you run out of Endurance points, you begin to take levels of exhaustion as normal. 

Your endurance points regenerate on a long rest. You may not, however, then expend endurance points to remove any exhaustion levels you have gained, they can only be used to prevent the status from being acquired in the first place.
}

\feat{Berserker Ideal}
{

	At 2nd level, you may decide what kind of person you wish to be, by selecting a Berserker Ideal which best matches your desired path through life. 
	
	You may decide that you are a Noble Savage, a fierce warrior when enraged, but also valuing restraint and diplomacy when possible; or you may become a Primal Warrior, consumed by your eternal, unrelenting fury; or you may become a Mystic Champion, your fury bolstered by arcane energies. 
	
	Your choice of Beserker Ideal grants you abilities at 2nd level, and then again at 6th, 10th, 14th and 18th levels. 
}

\feat{Additional Attack}
{
	At 4th level, and again at 12th level, whenever you make a major\minus{}action attack you may perform an additional strike. 
}

\feat{Honed Senses}
{
	From 5th level, your senses have become accustomed to your barbarous lifestyle, and you have developed a 6th sense for when things are about to go wrong. 

	You cannot be surprised, and sneak attacks which would normally trigger a critical strike function as normal attacks against you. 
}

\feat{Feral Fury}
{
	From 8th level, if you enter into Battle Fury as soon as a combat encounter starts you gain a bonus. 
	
	If you enter into Battle Fury during the first turn of an encounter, you gain advantage on the first roll to determine your Fury Value.  
}


\feat{Simmering Rage}
{
	From 15th level, you harbour a perpetual low\minus{}level anger which lies just below the surface. 
	
	If you re\minus{}enter a frenzied state within 30 seconds of exiting it, you may retake your previous Fury Value
}

\feat{Incredible Fitness}
{
	From 18th level, your Strength, Speed and general fitness knows no bounds. 
	
	Whenever you perform a Fitness check with a result less than your \attPhys{} value, you may use that value instead. 
}

\feat{Eternal Wrath}
{
	At 20th level, the adrenaline pumping through your veins is second nature to you. You spend almost every second in a perpetual, all consuming\minus{}rage. 
	
	Your Endurance Points become truly unlimited, and you take check\minus{}advantage any time you roll your Fury Dice. 
}


\section*{Berserker Ideals}

From 2nd level, you get to choose a Berserker Ideal to follow, which grants you additional features. 

\subsection*{Noble Savage}

Though you strike a formidable, even terrifying, figure whilst in battle, you recognise that fighting is only one aspect of existence. You use violence only when necessary, preferring instead to rely on softer skills to achieve your aims. 

Most Noble Savages strive to be an innately good person, though those who seek to cross them find that this does not mean they are pacifists \minus{} often discovering this at the business end of a large axe. 


\subfeat{Controlled Fury}
{
	At 2nd level, you begin an important journey: learning to control your rage, rather than letting it control you. 
	
	As an instantaneous action you can exit your Battle Fury by choice. When doing so, if you pass a DV 15 Logic check, it does not cost you an additional point of Exhaustion.

	In addition, as a result of this rigorous training, you gain advantage on any checks to resist the {\it Fury} effect. 
}

\subfeat{Code of Honour}
{
	From 6th level, a Noble Savage subscribes to a {\it Code of Honour}, a set of ethics and ideals which dictates the form that their `nobility' takes, and acts as a focus for their rage. 
	
	You may choose from the following paths:
	
	\newcommand\code[3]
	{
		\item {\bf #1:} #2
		
		#3
		
		
	}
	
	\begin{itemize}
		\code{Protection}{You vow never to let an innocent come to harm, even when this causes difficulty for you and your friends.}{Whilst in a Battle Fury, you may use the {\it Defensive Burst} action. By halving your current Fury Value, you may use a Major action to dash over to an ally within movement range, and save them from incoming danger \minus{} negating all attacks on them this turn. }
		\code{Brotherhood}{Your circle of allies is the most important thing. You would do anything to keep them together, and prevent any harm coming to them.}{When performing an {\it assist} action, you may expend an Endurance point to provide a bonus to the check equal to half your Berserker level.}
		\code{Honour}{You prize honour in all actions that you do, and vow to face all problems face on, without the use of deceit or skulduggery.}{You are considered proficient in Persuasion.}
		\code{Heroism}{You vow to be a mighty hero of legend, never turning down an opportunity to help others, and striking down evil wherever you find it. You are uncompromising in the face of evil.}{Melee attacks against beings with an \attEvl{} value greater than your own deal an extra 1d4 celestial damage.}
	\end{itemize}
	
	If the GM rules that you have failed to live by your Code of Honour, you lose the ability granted to you by that code. In addition, you take disadvantage in all Battle Fury rolls you make. You can regain your Honour by performing some form of Penance, which should be worked out with your GM.  
	
	When you regain your honour, or even if you experience an upheaval in your life, you may change the code of honour you live by, taking a different bonus. In order to change your Code, you must complete a Penance relevant to your new Code.  
}

\subfeat{Shifting Aura}
{
	From 10th level, you have become an expert in presenting two different faces to the world \minus{} your primal, furious warrior, and the empathetic, charming being behind the rage. 

	At any given moment, you may have one of the two following bonuses active:
	\begin{itemize}
		\item {\bf Imposing Aura:} gain a bonus to all Intimidation checks equal to one\minus{}third your Berserker level. 
		\item {\bf Charming Aura:} gain a bonus to all Persuasion checks equal to one\minus{}third your Berserker level. 
	\end{itemize}
	
	You may only have one of these active at a time, changing them at will by taking one minute to prepare yourself. 
	
	If you enter into your Battle Fury, you automatically take the Imposing Aura. This effect lasts for 1 hour after your Fury ends, unless you successfully use your {\it Controlled Fury} ability to end your Battle Fury.   
}

\subfeat{Uncorrupted Soul}
{
	From 14th level, you learn to leverage the purity of your soul, which remains uncorrupted by the hustle and bustle of modern life, and connected to the innate goodness of humanity. By entering into a deep mediatative trance for at least 5 minutes, you can purge yourself of all Poisoned, Diseased, {\it Charmed}, {\it Burned} or {\it Frostbitten} status effects. 
	
	For the next 6 hours after performing this, you have advantge on any checks to Resist retaking these effects. 
}

\subsection*{Primal Warrior}

A primal warrior is utterly consumed by their rage, an entity solely defined by battle, combat and pain. Amongst the most terrifying warriors to encounter. 

\subfeat{Audacious Attack}
{
	From 2nd level, you gain the ability to make more powerful attacks, at the cost of sacrificing your own safety. 

	Whilst in a frenzied state, you can throw aside all concern for your own safety and declare that an attack is {\it Reckless}. You gain advantage on accuracy rolls this turn cycle, but all attacks made against you also get advantage.
}

\subfeat{Bloodletting}
{
	From 6th level, you learn how to bolster your fury by drawing a blade across your own skin as an instantaneous action. Doing so deals 1d6 damage to you, increasing by 1d6 at 9th, 14th, and 18th levels. This damage cannot be reduced by your Battle Fury. 
	
	Upon performing the bloodletting, you may then use the increased fury to perform one of the following actions:
	
	\newcommand\blood[2]{\item {\bf #1:} #2 }
	
	\begin{itemize}
		\blood{Primal Fury}{For the next 30 seconds, you may use a Fury dice one level higher than your current one, as indicated on the Fury column of the Archetype table. If no higher dice is listed, use 4d6.}
		\blood{Blood Smear}{Any enemy you successfully hit with a melee attack this combat round is covered in blood, which you smear into their eyes and mouth. A sapient being so effected must succeed on a Willpower Resist with a DV equal to 10 + half your Berserker level, or vomit, becoming incapacitated next round.}
		\blood{Instinctive Retribution}{If a being within melee range successfully lands an attack on you this turn, you may choose to take an additional full-turn attack, with advantage, on that being in place of you usual Instinct action. All other attacks made against you this turn succeed. You must declare this action at the beginning of the turn, or else it fails.}
		\blood{Slick Surface}{For the next minute all attempts to grapple you have disadvantage, as the blood drips down your body.}
		\blood{Bloodbourne}{You utilise an infection in your bloodstream which you have become immune to. Designate one of your weapons as that which was used to perform the bloodletting. The next three successful strikes made with that weapon have a chance inflict the {\it Poisoned: Mild} status effect on your foes, if they fail a Vitality Resist with a DV equal to 10 + half your Berserker level.}
	\end{itemize}
	
	In addition, since you took damage (albeit from yourself), you may re-roll your Battle Fury dice, following the usual procedures. Using Bloodletting also counts as an attack for the purposes of preventing a Battle Fury from wearing off.  
	
}

\subfeat{Howling Barbarian}{

From 10th level your Fury bubbles up inside of you in a earsplitting howl. 

You learn how to use a major action to release a blood\minus{}curdling howl, and perform a Fitness (Intimidation), Spirit (Intimidation) or Power (Intimidation) check (your choice). 

All beings within 5m must contest this with a Willpower Resist, or become {\it Terrified} of you. If you use this action on the same turn as performing a {\it Bloodletting} action, all effected beings take disadvantage on their resist check. A being which succeeds is immune to this effect for 24 hours. 

}

\subfeat{Persistent Fury}
{
	From 14th level, your Fury is so potent that you can sustain it, even whilst combat has died down slightly. 
	
	You can sustain a Battle Fury for up to 1 minute without taking damage, or performing an attack. This increases to 5 minutes at 17th level.
}

\subsection*{Mystic Champion}

A wand in one hand, a greataxe in the other, the Mystic Champion stalks ungodly monsters, and smites magical threats with their combined magical and physical might. The mystic champion is unusual amongst the traditional Berserkers, in that they rely on their magical abilities, rather than raw, brutal strength. Their unbridled fury, however, is unmistably Berserker in origin. 



\subfeat{Furious Spellcaster}
{
From 2nd level, you regain the ability to use some magical spells effectively, even when in the depths of your Battle Fury. 

Choose three spells that you have memorised. These spells are not effected  by the usual spell-limitations inherent in the Battle Fury. You may re-choose these 3 spells whenever you take a long rest. 

In addition, you may choose to add your Fury Value to the damage roll of a spell. Doing so, however, reduces your Fury Value to one-third your Berserker level (or to 1, if it is below this value already). 

}

\subfeat{Elemental Fury}{

From sixth level, you learn the channel your Fury through elemental means. 

When perfoming a melee attack, the additional damage provided by your Battle Fury becomes elemental in nature, rather than the same type as the weapon used. Once per turn you may choose from Fire, Cold, Bludgeoning, Concussive or Electric damage.}

\subfeat{Ancestral Warrior}{

From 10th level, you gain the ability to summon the astral form of one of your forebears, a mighty mystical warrior in their own right, to aid you in your quest. 

Once per short rest you may use a minor action to manifest this powerful spirit. When summoned, the figure appears in a free space that you can see within 5m of you.

\boxOnlyBeast{name = Ancestral Warrior, mind = Ineffable, category = Phantasm, summary = Ghostly Ancestor, hp = 3$\times$Berserker level health, block = 15, dodge = 12, speed = 8m (walking), fit = 17 (+3), prs = 12 (+1), spr = 18 (+4), chr = 11 (+0), int = 8 (-1), pcp = 10 (+0), pow = 15 (+2), evl = 0 (-5), rating = IV, abilityBlock = 1, hasAbilities = 1, hasSkills = 1, size = 2.4m, skills = Strength (+7)\comma{} Vitality (+7)\comma{} Intimidation (+6)\comma{} Willpower (+8), abilities = 

\ability{Serve the Bloodline}{The \name{} obeys instructions given to it by its summoner to the best of its abilities.}

\ability{Psychic Connection}{The \name{} is connected to its master by a psychic link, through which they can communicate. If the caster falls unconscious, the Ancestral Warrior vanishes.}

, hasComprehend = 1, comprehend = Its Master\apos{}s language., hasActions = 1, actions = 
\ability{Discorporate}{If commanded to, the \name{} can automatically end its existence, popping out of existence and ending the effect which created it. }

\ability{Channel Spell}{Though no longer possessing magic of their own, the \name{} can use a major action to act as a conduit for a spell. The Master performs any associated checks as if they were casting the spell, but the origin of the spell effect, and any viable targets, are determined by the position and senses of the Spirit Guide.}

\ability{Dispense Advice}{The \name{} can cast {\it Recieve Omen} at will, if its Master uses a minor action to ask a question.}

\melee{Spectral Axe}{+7}{The \name{} wields a weapon of pure Fury, dealing slashing damage equal to your Fury Dice to a single target.}

}
}

The Ancestral Warrior exists for one minute, until its HP reaches zero, it is dismissed by the caster, or the caster loses consciousness, at which point the spirit returns to the astral plane.

\subfeat{Siphon Fury}{

At 14th level, you learn how to siphon of your Fury into mental fortitude. 

As a major action you may reduce your Fury Value by any given amount between 1 and your current value. You then increase your FP value by this same amount. This cannot be used to exceed your current Maximum Fortitude value.  

When you perform this action, you risk losing your Fury. The more Fury you siphon off, the higher this risk is. Calculate (or ask your GM to roughly estimate) the \% of your Fury that you siphoned off, then roll a d100, and add twice your Barbarian level. If the value of the d100 is lower than the percentage you siphoned off, you end your Fury and take the associated exhaustion.  
}







