
\chapter*{Berserker}
\addcontentsline{toc}{section}{Berserker}
An intro bit of text 

%%archBegin
\archetype{name=Berserker, hp=12, fp=6, armour=None, tool=, disc=, weapon=, prof=, equip=, memorised=, expertI = 2, maxspellI = Beginner, bonusI = Test, expertII = 2, maxspellII = Beginner, expertIII = 2, maxspellIII = Beginner, expertIV = 3, maxspellIV = Beginner, expertV = 3, maxspellV = Novice, expertVI = 3, maxspellVI = Novice, expertVII = 3, maxspellVII = Novice, expertVIII = 4, maxspellVIII = Novice, expertIX = 4, maxspellIX = Novice, expertX = 4, maxspellX = Adept, expertXI = 4, maxspellXI = Adept, expertXII = 5, maxspellXII = Adept, expertXIII = 5, maxspellXIII = Adept, expertXIV = 5, maxspellXIV = Adept, expertXV = 5, maxspellXV = Master, expertXVI = 6, maxspellXVI = Master, expertXVII = 6, maxspellXVII = Master, expertXVIII = 6, maxspellXVIII = Master, expertXIX = 6, maxspellXIX = Master, expertXX = 7, maxspellXX = Ascendant, shortmode = 0}
%%archEnd


\section*{Acquired Feats}

As with a Barbarian, the berserker will be built around the `rage' feature. 

Popular idea is that you get a `Rage Die'. When you enter rage, you roll the dice. Add this result to all FIT checks etc, remove it from all damage taken. Everytime you take damage \minus{} if it is larger than current value, you increase your rage to that amount. 
