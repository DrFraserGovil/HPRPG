\documentclass[CoreRulebook.tex]{subfile}
\newpage
\begin{strip}
\setlength{\parskip}{4pt}

\section{Oathkeeper}

An oathkeeper is an individual who has dedicated their life to a cause, putting it before everything else. Their every action is geared towards fufilling the promise that they have made. Oathkeepers come in two varieties, the {\bf advocate}, who has sworn to uphold a right, or a principle, with their power coming from within. The {\bf disciple} on the other hand has sworn fealty to a powerful being, which gants them immense powers. 

%%archBegin
\archetype{Oathkeeper}{Advocate}{Disciple}{0}{featureI=Unbreakable Vow, featureII=Detect Ally, alphaFeatureIII=Oath of Fealty\comma{} Oath Perk, betaFeatureIII=Powerful Patron\comma{} Patron Boon, featureIV=Redemption Arc, alphaFeatureV=Self Improvement, betaFeatureV=Patron Boon, featureVI=Devotion to the cause, alphaFeatureVII=Oath Perk, alphaFeatureVIII=Self Improvement, betaFeatureVIII=Patron Boon, featureIX=Fanatic\apos{}s strike, alphaFeatureIX=Singular Focus, betaFeatureX=Loyalty, alphaFeatureXI=Oath Perk, betaFeatureXI=Patron Infusion, featureXII=Shield of Faith, alphaFeatureXIII=Self Improvement, betaFeatureXIII=Patron Boon, alphaFeatureXV=Oath Perk, betaFeatureXVI=Patron Boon, alphaFeatureXVII=Oath Perk, betaFeatureXVII=Temple, featureXVIII=Shield of Faith II, alphaFeatureXVIII=Self Improvement, alphaFeatureXX=Oath Perk, betaFeatureXX=Patron Boon}%%archEnd

\end{strip}


\subsection{Starting Equipment}

Oathkeeps start with:
\begin{itemize}[itemsep=0em]
	\item a basic pack
	\item a Wand (roll on the wand table to determine composition)
	\item 3d6 $\times 5$ gold
\end{itemize}
%\newpage
\subsection{Starting Spells}

Oathkeepers may choose 4 spells form the Basic Spells table. 

%\newpage
\subsection{Archetype Features}

\feat{Unbreakable Vow}

From first level, you may perform the {\it Unbreakable Vow} spell. This is a Ritual spell that takes 1 minute to complete. If willing, the participants may enter into a contract that the GM judges to be `reasonable'. The participants must then abide by the terms of this Vow, or take 15d20 psychic damage.  

This ability can only be used once per day.

\jump
\feat{Detect Ally}

At 2nd level, you may automatically detect when someone is beholden to the same promises as you are. Any Oathkeeper who shares your Oath of Fealty or Powerful Patron will glow in your vision. 

If they do not share your Patron/Oath, you may take a major action to learn to what ideals they are beholden. 

\jump
\feat{Redemption Arc}

From 4th level, if the GM judges that you have failed to uphold the cause that you have sworn to dedicate your life to, they may strip you of all bonuses granted to you by the Oathkeeper Archetype. Each Oath and Patron lists {\it Tenets of Faith}, these are the ideals to which you should adhere to. 

These bonuses can only be regained by completing a {\it Redemption Arc}. This is a quest or challenge set to you by the GM that will redeem you from yor transgressions.
\jump
\feat{Devotion to the Cause}
From 6th level, you may choose to sacrifice 80\% of your maximum health to automatically succeed in your next check. 

This ability may only be used once per day, and you can be killed by it (after completing the action you were attempting). 
\jump
\feat{Fanatic\apos{}s Strike}

From 9th level, you fanaticism towards the cause enables you to take two major actions per turn. 

\jump
\feat{Shield of Faith}

From 12th level, your devotion to the cause grants you a mythical AC of 10. This AC can be damaged in the normal way, but is restored when your FP is restored to maximum.

At 18th level, this AC increases to be equal to your Oathkeeper level. 


\subsection{Advocate Features}
\feat{Oath of Fealty}

At 3rd level you must choose a value to dedicate your life to. Oaths are detailed on page \pageref{S:Oaths}. Your oath provides you with a perk at 3rd, 7th, 11th, 15th, 17th and 20th level. 
\jump
\feat{Self Improvement}

At 5th, 8th, 13th and 18th level, choose an attribute to increase by 2. 
\jump
\feat{Singular Focus}

From 9th level, get check advantage when casting concentration-type spells, as well as on checks to maintain concentration. 

\subsection{Disciple Features}
\feat{Powerful Patron}

At 3rd level you must choose a patron to dedicate your life to. Patrons are detailed on page \pageref{S:Patrons}. Your patron provides you with a boon at 3rd, 5th, 8th, 13th, 16th and 20th level. 
\jump
\feat{Patron Infusion}

From 11th level, once per day you may take 3 turns (15 sec) to concentrate and borrow some power from your patron. if any POW check is less than your Acolyte level, you may use that value instead. 

\jump
\feat{Temple}

At 17th level, you gain access to a temple dedicated towards your patron. Checks inside the temple gain check double-advantage. 

\section{Oaths} \label{S:Oaths}

\subsection{Oath of Law}

By choosing the Oath of Law, you promise to uphold the laws, regardless of the consequences. You are a champion of Law and Order, criminals and tricksters are your foes.

\feat{Tenets of Law}

The tenets of Law are simple: you swear to never break the law, to unquestioningly follow the rules, and to challenge those who oppose this doctrine. 
\jump
\feat{Honed Senses}

From 3rd level, if anyone breaks the law in a 10m radius and fails a DV 10 FIN(stealth) check, you are immediately aware of it. 
\jump
\feat{Immutable Spirit}

From 7th level, if, under the influence of a spell such as {\it Create Thrall} or {\it Suggestion}, you are directed to take an action that would violate your oath, the spell is broken and you are immune to its effects for 24 hours. Equally, any illusions cast to disguise unlawful activity get check-disadvantage on any investigation checks you conduct. 

\jump
\feat{Interrogate}

From 11th level, if you question an individual that you have apprehended, you may take 1 hour to perform a ritual that compels them to speak the truth. 
\jump
\feat{Planemeld}

From 15th level, you may summon the spirit of the Plane of Order, and cast the {\it Planemeld} spell to summon Machina by passing a DV10 SPR(Willpower) check.
\jump
\feat{Expert Apprehension}

From 17th level, attempts to apprehend a target by immobilising them (such as the {\it Bind Target} spell) gain check double advantage. 
\jump
\feat{Summon Judiciary}

From 20th level, you may designate a lawbreaker as a target for the {\it Judiciary}. The Judiciary are a multiversal hivemind bent on bringing about justice to lawbreakers. These beings will appear and hunt down the target with a single-minded zeal.
\subsection{Oath of Righteousness}

The Oath of Righteousness means that you swear to uphold virtues that go beyond the law -- truth, courage and compassion. You swear to do what is right, even if that goes against the law. 

\feat{Tenets of Righteousness}

The Oath of righteousness requires that, wherever you see it, you defend and uphold the principles of honesty, courage, compassion, honour and duty. You must also challenge those who fail to live by these principles. 
\jump
\feat{Sword of Purity}

From 3rd level, as a major action, you may summon a sword of blinding light into an empty hand. You are considered proficient with this sword, and it does 2+2d4 Holy damage, using either an ATH(strength) check or FIN(dexterity) check.
\jump
\feat{Righteous Fury}

From 7th level, if you witness an individual harm an innocent, all damage checks on that target gain a +5 bonus. 
\jump
\feat{Excommunicate}

At 11th level, you gain the ability to excommunicate individuals whose values come into conflict with your guiding tenets. This takes a major action, and for the next 24 hours, the target is vulnerable to Holy damage. 
\jump
\feat{Angelic Impression}

From 15th level, you may take a major action to imbue your aura with a holy light. Targets within a 10m radius who live by your Tenets gain check advantage for 3 turns, whilst those who lie in opposition to it take the {\it Terrified} status. 

\subsection{Oath of Vengeance}

When taking the Oath of Vengeance, you swear to take your revenge on whatever it is that has wronged you in the past. 

\feat{Tenets of Vengeance}

When faced with doing the right thing, or extracting your revenge, a keeper of the Oath of Vengeance will always choose to take revenge. They show no mercy to the targets of their hatred, and you always initially share allyship with anyone who has also been harmed by your foe. 


\feat{Reckless Hatred}

From 3rd level, you may direct your hatred towards an individual, giving you check advantage on attack rolls, but check disadvantage on all defensive and evasion checks. 

\section{Patrons}\label{S:Patrons}

\subsection{Benevolent Deity}

\subsection{Dark Power}

\subsection{Free Spirit}

\subsection{Incomprehensible Intelligence}
