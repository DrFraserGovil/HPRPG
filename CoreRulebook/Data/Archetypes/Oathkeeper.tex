\documentclass[CoreRulebook.tex]{subfile}
\newpage
\begin{strip}
\setlength{\parskip}{4pt}

\section{Oathkeeper}

An oathkeeper is an individual who has dedicated their life to a cause, putting it before everything else. Their every action is geared towards fufilling the promise that they have made. Oathkeepers come in two varieties, the {\bf advocate}, who has sworn to uphold a right, or a principle, with their power coming from within. The {\bf disciple} on the other hand has sworn fealty to a powerful being, which gants them immense powers. 

%%archBegin
\archetype{Oathkeeper}{Advocate}{Disciple}{0}{featureI=Unbreakable Vow, arcaneI=2, featureII=Detect Ally, arcaneII=2, alphaFeatureIII=Oath of Fealty\comma{} Oath Perk, betaFeatureIII=Powerful Patron\comma{} Patron Boon, arcaneIII=2, featureIV=Redemption Arc, arcaneIV=2, alphaFeatureV=Self Improvement, betaFeatureV=Patron Boon, arcaneV=3, featureVI=Devotion to the cause, arcaneVI=3, alphaFeatureVII=Oath Perk, arcaneVII=3, alphaFeatureVIII=Self Improvement, betaFeatureVIII=Patron Boon, arcaneVIII=3, alphaFeatureIX=Singular Focus, arcaneIX=3, featureX=Shield of Faith, arcaneX=4, alphaFeatureXI=Oath Perk, betaFeatureXI=Otherworldly Infusion, arcaneXI=4, arcaneXII=4, featureXIII=Fanatic\apos{}s strike, alphaFeatureXIII=Self Improvement, betaFeatureXIII=Patron Boon, arcaneXIII=4, arcaneXIV=4, alphaFeatureXV=Oath Perk, arcaneXV=5, betaFeatureXVI=Patron Boon, arcaneXVI=5, alphaFeatureXVII=Oath Perk, betaFeatureXVII=Temple, arcaneXVII=5, featureXVIII=Shield of Faith II, alphaFeatureXVIII=Self Improvement, arcaneXVIII=5, arcaneXIX=5, alphaFeatureXX=Oath Perk, betaFeatureXX=Patron Boon, arcaneXX=6}%%archEnd

\end{strip}


\subsection{Starting Equipment}

Oathkeeps start with:
\begin{itemize}[itemsep=0em]
	\item a basic pack
	\item a Wand (roll on the wand table to determine composition)
	\item 3d6 $\times 5$ gold
\end{itemize}
%\newpage
\subsection{Starting Spells}

Oathkeepers may choose 4 spells form the Basic Spells table. 

%\newpage
\subsection{Archetype Features}

\feat{Unbreakable Vow}

From first level, you may perform the {\it Unbreakable Vow} spell. This is a Ritual spell that takes 1 minute to complete. If willing, the participants may enter into a contract that the GM judges to be `reasonable'. The participants must then abide by the terms of this Vow, or take 15d20 psychic damage.  

This ability can only be used once per day.

\jump
\feat{Detect Ally}

At 2nd level, you may automatically detect when someone is beholden to the same promises as you are. Any Oathkeeper who shares your Oath of Fealty or Powerful Patron will glow in your vision. 

If they do not share your Patron/Oath, you may take a major action to learn to what ideals they are beholden. 

\jump
\feat{Redemption Arc}

From 4th level, if the GM judges that you have failed to uphold the cause that you have sworn to dedicate your life to, they may strip you of all bonuses granted to you by the Oathkeeper Archetype. Each Oath and Patron lists {\it Tenets of Faith}, these are the ideals to which you should adhere to. 

These bonuses can only be regained by completing a {\it Redemption Arc}. This is a quest or challenge set to you by the GM that will redeem you from yor transgressions.
\jump
\feat{Devotion to the Cause}

From 6th level, you may choose to sacrifice 80\% of your maximum health to automatically succeed in your next check. 

This ability may only be used once per day, and you can be killed by it (after completing the action you were attempting). 

\jump
\feat{Shield of Faith}

From 10th level, your devotion to the cause grants you a mythical AC of 10. This AC can be damaged in the normal way, but is restored when your FP is restored to maximum.

At 18th level, this AC increases to be equal to your Oathkeeper level. 
\jump
\feat{Fanatic\apos{}s Strike}

From 13th level, you fanaticism towards the cause enables you to take two major actions per turn. 


\subsection{Advocate Features}
\feat{Oath of Fealty}

At 3rd level you must choose a value to dedicate your life to. Oaths are detailed on page \pageref{S:Oaths}. Your oath provides you with a perk at 3rd, 7th, 11th, 15th, 17th and 20th level. 
\jump
\feat{Self Improvement}

At 5th, 8th, 13th and 18th level, choose an attribute to increase by 2. 
\jump
\feat{Singular Focus}

From 9th level, get check advantage when casting concentration-type spells, as well as on checks to maintain concentration. 

\subsection{Disciple Features}
\feat{Powerful Patron}

At 3rd level you must choose a patron to dedicate your life to. Patrons are detailed on page \pageref{S:Patrons}. Your patron provides you with a boon at 3rd, 5th, 8th, 13th, 16th and 20th level. 
\jump
\feat{Otherworldly Infusion}

From 11th level, once per day you may take 3 turns (15 sec) to concentrate and borrow some power from your patron. if any POW check is less than your Acolyte level, you may use that value instead. 

\jump
\feat{Temple}

At 17th level, you gain access to a temple dedicated towards your patron. Checks inside the temple gain check double-advantage. 


\vspace{4ex}

\section{Oaths} \label{S:Oaths}

\subsection{Oath of Law}

By choosing the Oath of Law, you promise to uphold the laws, regardless of the consequences. You are a champion of Law and Order, criminals and tricksters are your foes.

\feat{Tenets of Law}

The tenets of Law are simple: you swear to never break the law, to unquestioningly follow the rules, and to challenge those who oppose this doctrine. 
\jump
\feat{Honed Senses}

From 3rd level, if anyone breaks the law in a 10m radius and fails a DV 10 FIN(stealth) check, you are immediately aware of it. 
\jump
\feat{Immutable Spirit}

From 7th level, if, under the influence of a spell such as {\it Create Thrall} or {\it Suggestion}, you are directed to take an action that would violate your oath, the spell is broken and you are immune to its effects for 24 hours. Equally, any illusions cast to disguise unlawful activity get check-disadvantage on any investigation checks you conduct. 

\jump
\feat{Interrogate}

From 11th level, if you question an individual that you have apprehended, you may take 1 hour to perform a ritual that compels them to speak the truth. 
\jump
\feat{Planemeld}

From 15th level, you may summon the spirit of the Plane of Order, and cast the {\it Planemeld} spell to summon Machina by passing a DV10 SPR(Willpower) check.
\jump
\feat{Expert Apprehension}

From 17th level, attempts to apprehend a target by immobilising them (such as the {\it Bind Target} spell) gain check double advantage. 
\jump
\feat{Summon Judiciary}

From 20th level, you may designate a lawbreaker as a target for the {\it Judiciary}. The Judiciary are a multiversal hivemind bent on bringing about justice to lawbreakers. These beings will appear and hunt down the target with a single-minded zeal.
\subsection{Oath of Righteousness}

The Oath of Righteousness means that you swear to uphold virtues that go beyond the law -- truth, courage and compassion. You swear to do what is right, even if that goes against the law. 

\feat{Tenets of Righteousness}

The Oath of righteousness requires that, wherever you see it, you defend and uphold the principles of honesty, courage, compassion, honour and duty. You must also challenge those who fail to live by these principles. 
\jump
\feat{Sword of Purity}

From 3rd level, as a major action, you may summon a sword of blinding light into an empty hand. You are considered proficient with this sword, and it does 2+2d4 Celestial damage, using either an ATH(strength) check or FIN(dexterity) check.
\jump
\feat{Aura of Protection}

From 7th level, whenever you or any ally within 3m must make a Resist check, your aura gives a bonus equal to one quarter of your Oathkeeper level. 
\jump
\feat{Righteous Fury}

From 11th level, if you witness a target harm an innocent or otherwise perform a needlessly cruel or damaging act, all damage checks on that target gain a +5 bonus. 
\jump
\feat{Excommunicate}

At 15th level, you gain the ability to excommunicate individuals whose values come into conflict with your guiding tenets. This takes a major action, and for the next 24 hours, the target is susceptible to Celestial damage. Targets that were already susceptible to Celestial damage take triple damage from this damage type.

\jump
\feat{Cleansing Touch}

At 20th level you gain the ability to purge magical effects from your allies with a touch. Once per minute you may end the effect of all negative ongoing magical effects on yourself or on a target in range. You may choose to leave some effects active if you wish (i.e. the definition of `negative' is somewhat flexible).
\subsection{Oath of Vengeance}

When taking the Oath of Vengeance, you swear to take your revenge on whatever it is that has wronged you in the past. 

\feat{Tenets of Vengeance}

When faced with doing the right thing, or extracting your revenge, a keeper of the Oath of Vengeance will always choose to take revenge. They show no mercy to the targets of their hatred, and you always initially share allyship with anyone who has also been harmed by your foe. 


\feat{Reckless Hatred}

From 3rd level, you may direct your hatred towards an individual, giving you check advantage on attack rolls, but check disadvantage on all defensive and evasion checks for 10 turns. 
\jump
\feat{Relentless Pursuit}

From 7th level, upon initiating combat you may automatically cast the {\it Hunter's Mark} spell on a subject of your rage. 

\jump
\feat{Vicious Assault}

From 11th level, if a target attempts to retreat on the same turn that you land a successful attack on them, you may perform an additional major action directed at that target. 
\jump
\feat{Aura of Hatred}

From 15th level, your aura is so tinged with rage that you may utilise it to influence others. Once per day, you may use this ability to enrage a region 5m in radius. Every being in this region acts as if the spell {\it Fury} had been cast on them (with a casting check of 20). 

\jump
\feat{Bloodthirst}

From 20th level, every time a subject of your rage hurts you or your allies, add one to a special rage-counter. If you perform a melee attack, you may add this twice the current counter value to your attack roll. 

When used, the counter value is divided in half. The counter resets to zero when all rage-foci in a given combat encounter are incapacitated. 

\newpage
\section{Patrons}\label{S:Patrons}

Patrons are powerful beings, usually residing outside of the mortal realm, who bestow upon their faithful mighty powers. Some patrons bestow this gift as part of a bargain, asking in return that the Oathkeeper enact their will upon the Earth. Other patrons might not even be aware of their champion's existence, instead the power might be gained by finding a mythical artefact containining a fragment of their power. 
 
\subsection{Benevolent Deity}

Your patron is a god-like being which prizes the beauty of mortal life, light and nature. Usually hesitant to cause harm, preferring peaceful negotiation, the deity can however be enraged when harm is done to those it protects. To maintain your gifts, your deity asks that you:
\begin{itemize}
\item  {Help those in need without prejudice}
\item Harm only those who have harmed the innocent
\item {Protect life, light and nature wherever possible}. 
\end{itemize}

\feat{Light in the Darkness}

From 3rd level, as a minor action you may invoke a {\it Flare}, a burst of sunlight, between you and a target. If the target attacks you this turn, they take check disadvantage on damage checks (targets immune to blinding are not affected by this). If the target fails a INT(perception) check, they also take 1d10 Celestial damage. This costs 5FP. You may also use this ability to create a number of hovering lights equal to half your acolyte level. These lights last for one hour, and follow a target individual around. 

\jump
\feat{Peaceful Negotiation}

From 5th level, as a major action you may perform a d20 CHR (persuasion) check (DV = 20, minus 1 for each acolyte level). If the check succeeds, you may pause combat for 5 rounds. In each of those 5 rounds, you may make one offer to the enemy combatants to end combat. If they accept the offer, the target exits combat peacefully. If they refuse the offer 5 times, combat resumes as normal. If you make an offer that you cannot deliver on, combat resumes and you take check disadvantage on attack rolls. During these 5 ceasefire-rounds, all other combatants may take non-combat actions as normal. Any actions judged as hostile, however, will resume combat early. 

\jump
\feat{Divine Intervention}

At 8th level, when in great danger or dire need, you may pray to your patron for help. Taking a minor turn, describe the assistance you seek and perform a d20 CHR(persuasion) check. The DM tells you the difficulty of the action that you are about to attempt (easy to legendary, actions that fall within your patron's sphere of influence will be judged to be easier). The table below sets the DV that you must reach:

\footnotesize
\begin{rndtable}{c c c c c c}
\hline
\bf Disciple Lvl	&	\bf Easy	&  \bf Moderate	&	\bf Hard	& 	\bf Very Hard	&	\bf Legendary
\\
$<10$	&	5	&	10	&	15	&	20	&	25
\\
10 - 13 &	4	&	8	&	12	&	16	&	21
\\
14-16	&	3	&	6	&	9	&	12	&	16
\\	
17-19 	&	2	&	4	&   6	&	10	&	13
\\
20+ 	&	1	&	2	&	5	&	7	&	10
\\
\hline
\end{rndtable}
\small

If the check succeeds, your deity will intervene and help you, and you may not then use this ability for one week. If it fails, you cannot use this ability for 1 day. At 15th level, the time delay reduces to 1 day if successful, and 1 hour if not. 
\jump
\feat{Blessed Hands}

From 13th level, you may channel the divinity of your patron through your hands. As a major action, you may grasp a living target and heal them for HP equal to twice your Acolyte level. This costs 10FP.

\jump
\feat{Angelic Host}
\jump

At 16th level, you are granted a pair of ethereal, angelic wings. These wings normally remain hidden in the astral plane, but you may summon them using a major action. You may then fly with a base speed twice your land-base speed. Dismissing the wings back to the astral plane takes a minor action. 

\jump
\feat{Biblical Wrath}

At 20th level, your benevolence melts away and you may invoke powerful destructive magic to smite those who have wronged your patron. Once per hour, as a minor action you may:

\begin{itemize}
	\item Summon a tidal wave up to 6m wide, which pushes all enemies back up to 30m, prevents breathing for 30 seconds, and deals 10d10 bludgeoning damage
	\item Call down a hail of fire, as if you had cast the {\it Meteor Strike} spell with 3PP (6d8 bludgeoning and 8d6 fire damage to all targets in 10m radius). Cannot be used indoors. 
	\item Open a chasm in the Earth, dropping 2d4 targets a distance of 50m onto a solid surface. Chasm can be made wide enough to drop a small building into.
	\item Cause all food and other non-living organic substances in a 5m radius to rot and decay, causing all those who eat it to die if they fail a DV 10 d20 SPR(Health) check.
	\item Summon a bolt of lightning to do 9d8 electric damage to a target and blinding 1d4 targets within 3m of the bolt-target.
\end{itemize}

\subsection{Dark Power}

Your patron is an evil being -- perhaps a demon from one of the lower planes, or even an exceptionally evil mortal. Their one desire is for you to spread, pain fear and suffering wherever you go. They may also call upon you to take part in specific evil acts. To maintain the favour of your patron, you should:
\begin{itemize}
	\item Obey without question any order given by your patron
	\item Spread fear and chaos wherever possible
	\item Corrupt as many individuals towards evil as you can.
\end{itemize}

\feat{Killing Joy}

From 3rd level, whenever you reduce a living being to 0HP, gain a temporary bonus to your HP ceiling equal to twice your disciple level, and increase your HP by this same amount. This lasts for 5 turns (30 seconds). This effect does not stack, you simply use the largest such bonus, and the timer resets every time you kill again. 

\jump
\feat{Evil Incarnate}

From 5th level, whenever you choose to increase an attribute as part of the levelling-up process, you may automatically increase your EVL attribute by an additional point. 
\jump
\feat{Necrotic Touch}

From 8th level, you may take a major action to grasp your enemy and channel evil energies into them. Target may attempt to break free using DV 10 ATH (strength) Resist check. This attack does 2+2d8 necrotic per turn, plus an additional 1d8 for every 3 Disciple levels above 7th.
\jump
\feat{Defer to Master}

From 13th level, if any EVL check has a total result less than your EVL attribute, you may use that value instead. 

\jump
\feat{Dark Shroud}

At 16th level, your master grants you the ability to use the {\it Dark Shroud} ability. If you are standing in shadow or a dimly lit environment, you may take 1 major turn to become completely invisible for as long as you remain in darkness. 
\jump
\feat{Planestrike}

From 20th level, when you make an unarmed strike against an opponent, you may choose to send your target to one of the more unpleasant Planes. The target instantly vanishes from this reality, and spends time either in Hades (taking 8d10 necrotic damage), Tartarus (becoming 8d10 psychic damage and acquiring the Broken Bone status effect) or Abyss (becoming completely paralysed with fear for 10 turns). They then reappear one turn later, scarred by their experiences. 

This ability can only be used once every 24 hours. 


\subsection{Free Spirit}

Your patron is a nameless, formless entity that recognises no master, and accepts no constraints. The Free Spirits will never give you direct orders, but instead will trust that you will bring freedom, and a little spark of chaos into the world. To remain in this fickle entity's good books, you should:
\begin{itemize}
	\item Never accept orders from a superior (unless you were going to do it anyway!)
	\item Not remain in one place for too long, and always be unpredictable
	\item Perform pranks, and spread mischeif, wherever possible
\end{itemize}

\feat{Fleet Foot}

From 3rd level, your base speed increases by 2m. 

\jump
\feat{Defensive Leap}

From 5th level, once per hour, you may teleport up to 10m in any direction upon taking damage. The target location must be unoccupied, and you must be able to see it. 

\jump
\feat{Make Wild}

From 8th level, once per day you may take a major action to restore all living beings in a 10m to their wild state. Trained beasts return to their animal instincts, and all sapients in range must succeed on a DV10 SPR(willpower) Resist check, or act as if the {\it Fury} spell had been cast upon them. 

\jump
\feat{Unconstrained Knowledge}

At 13th level, and again at 15th,  you may choose any other Archetype Feature designated 11th level or below, from any other Archetype. If this Feature gains additional aspects at higher levels, you may acquire them, but at 3 levels higher than the stated value. This does not include `branch-choice` features such as the {\it Powerful Patron}, but you may choose the abilities granted by these choices (i.e. you could choose {\it Immutable Spirit}, a feature granted by choosing the Oath of Law, but you could not choose {\it Oath of Fealty}, the feature that allows you to choose the Oath of Law). 

You may rechoose this ability once per week.

\jump
\feat{Chaotic Aura}

From 16th level, your patron blesses you with a powerful yet unpredictable aura. Every time you take damage, roll a d20:
\begin{itemize}
\item {\bf 1-5 }: No change (take damage as normal)
\item {\bf 6-13}: Take damage with temporary +6 AC
\item {\bf 14-15}: Attacker takes the same damage
\item {\bf 16-17}: Cast {\it Knockback} jinx on attacker.
\item {\bf 18-19}: Take 25\% damage
\item {\bf 20}: Take 0 damage
\end{itemize}

In addition, any time that a check uses the {\it Chaos} proficiency, if the total check value is less than your Disciple level, use the larger value instead. 

\jump
\feat{True Freedom}

From 20th level, all attempts to entrap, or slow you down fail. Manacles and ropes fall off you, and all impediments to movement are ineffective. You may ignore all terrain costs, and once per day you may walk through up to 1m of solid material. 

\newpage
\subsection{Incomprehensible Intelligence}

Of all the patrons, the Incomprehensible Intelligence is the most likely to be unware of your existence. Ancient beyond measure, and existing outside of the normal constraints of space and time, the Intelligence is most likely one of the Eldritch beings from the far reaches of the multiverse. When and if they do notice you, they may give you inscrutable orders, such as moving a single rock a foot to the left -- no doubt part of a millenia long plan that you cannot conceive of. 

Because of their alienness, and their indifference, it is hard to know what might anger such a being. You should therefore be wary -- and simply try to stay out of its way. Interfering with its plans is the only sure fire way to bring down its wrath. 

\feat{Terrible Secrets}

From 3rd level, your patron grants you insight into the most terrible and mind-bending facts about the universe. As a major action, you may speak one of these secrets aloud. All targets within 3m that hear these secrets must succeed in a DV 12 SPR(endurance) check, or take 2d4 psychic damage for every three Disciple levels above 0.

\jump
\feat{Arcane Grimoire}

From 5th level, you are gifted a mystic grimoire. When you gain this book, choose one Novice level (or below) spell, which you need not already know. When you have the grimoire in your possession, you may cast this spell at will as a wandless, silent minor action, as if you had performed a check equal in value to your Disciple level, or the spell DV (whichever is higher). You may not dedicate PP towards these spells. 

If you lose your grimoire, or wish to alter the chosen spell, you must perform a 2 hour ritual to recieve a new one. The old grimoire vanishes when a new one is created. 

At 10th and 15th level, you may add one more Novice level spell to your Grimoire. 

\jump
\feat{Tongues of the Ancients}

At 8th level, you gain the ability to speak and be spoken to by any being which has a coherent language. You can also read ancient runes and understand other mystical markings. 

\jump
\feat{Alien Knowledge}

At 13th level, you have spent enough time connected to your patron that your mind is suffused with knowledge that it is completely alien to most mortals. Your brain works so differently that all attempts to read your thoughts fail. Psychic damage does only 50\% damage to you, and any being which inflicts psychic damage on you, takes the same amount of damage that you do.  

\jump
\feat{Enhanced Grimoire}
At 16th level, your mastery of the grimoire improves such that you learn how to dedicate Power Points to your grimoire spell. You may dedicate one PP for every Disciple level above 14th level. 
\jump
\feat{Apotheosis}

At 20th level, you mantle a small amount of your patron\apos{}s god-like power, and gain control of a small facet of Creation itself. You find within the pages of your Grimoire the spells {\it Vanish Object} and {\it Conjure Object}, which you can cast with 10 power points. 
