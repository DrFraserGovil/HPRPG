\documentclass[CoreRulebook.tex]{subfile}
\newpage
\begin{strip}
\setlength{\parskip}{4pt}
\newpage
\section{Ravenclaw Student}

Their hearts full of a desire for learning, but their eyes blinking against the harsh light of day, Ravenclaw students often make unwilling adventurers -- preferring instead the comfort of a cosy library. Some secrets, however, cannot be found between the pages of a book -- and an expedition must be mounted, for as the Sorting Hat tells us:
\begin{displayquote}
\it Or yet in wise old Ravenclaw,
\\
If you�ve a ready mind,
\\
Where those of wit and learning,
\\
Will always find their kind.
\end{displayquote}
Ravenclaw is the house that prizes knowledge and an inquisitive mind above all other traits. Often members of this house are the most advanced spellcasters in their year, but their lack of practical experience and physical ineptitude means they're not always the best in every situation. 

Students in Ravenclaw are typically one of two breeds: the {\bf nerds} and the {\bf prodigies}. The nerds are those who love learning for learning's own sake, they latch onto a topic and devour all information related to that subject. The prodigies, on the other hand, are truly gifted individuals who have a natural talent in a specific field. 
%%archBegin
\archetype{Ravenclaw Student}{Nerd}{Prodigy}{1}{featureI=Heart of Wisdom, featureII=Arcane Knowledge Increase, arcaneII=1, alphaFeatureIII=Passion project, betaFeatureIII=Prodigy Choice, arcaneIII=1, featureIV=Heart of Wisdom II, arcaneIV=1, featureV=Arcane Knowledge Increase, alphaFeatureV=Speed Reading, arcaneV=3}%%archEnd
\end{strip}


\subsection*{Archetype Features}

\textbf{\textit{Heart of Wisdom:}}

Starting at first level, you may choose 2 of the four intelligence proficiencies to take a + 1 bonus in. Repeat this process again at 4th level (you may choose differently). 
\par
~
\par
\textbf{\textit{Arcane Knowledge Increase}}

At second level, and again at 5th level, gain a bonus point to your Arcane Wisdom. 

\subsection*{Nerd Features}

\textbf{\textit{Passion project:}}

Starting at 3rd level, you may choose one specific spell or skill (such as potionmaking) as your `passion project'. Checks to use that skill (or cast that spell) may then be performed with a die one level larger than your present one. If you use a d20, gain a +4 bonus instead. Changing your passion project takes 2 weeks of solid work. If you do not use the skill at least once a week, you become out of practice, and must start again. 
\par
~
\par
\textbf{\textit{Extraordinary Memory:}}

You may commit a book (see Items for a booklist) to memory. Memory and Knowledge checks in that field get +4. You may only have this bonus in one field at a time. 


\subsection*{Prodigy Features}

\textbf{\textit{Prodigy Choice:}}

You may choose one prodigy area: Chess, Music and Art. Your field of expertise gives you features at 3rd and 4th level. These are listed below.

\newpage
\section{Prodigy Fields}

A prodigy is typically an individual who excels in one of the three following areas: Chess, Music or Art. 

\subsection*{Art}

\textbf{\textit{Visual Clarity:}}

From 3rd level, you see things much more clearly than the average human. Perception proficiency gets +1 bonus. 

Starting at 5th level, you are also able to observe any weak spots in the armour of an enemy. 

\subsection*{Chess}

\textbf{\textit{Tactical Inference:}}

Starting from 3rd level, you may use your major action to ascertain the plans of your enemy by performing a 1d8 INT (history) check, with the DV set by the target performing an INT Resist (deception) check. 

From 5th level onwards, you may use a 1d12 to perform the check. 

Knowing their plans gives you check advantage for all actions against them, and them check disadvantage for all actions against you for 1 minute (5 combat rounds). 

\textbf{\textit{Patient Strike:}}

From 4th level onwards, for every 5 combat rounds that you do not take damage in, get a +1 bonus to all subsequent attacking checks (max 3). This counter resets when you take damage.  

\subsection*{Music}
Characters that take the Music Prodigy should choose an instrument to play. 

\textbf{\textit{Perfect Pitch:}}



\textbf{\textit{Virtuoso Performance:}}

Starting from 4th level, if you are able to play music for at least 1 minute (5 combat rounds) without taking damage or being otherwise interrupted, all targets within hearing range take one of the following effects:
\begin{itemize}
	\item Take 1d4 psychic damage
	\item Become confused for 1 round
	\item Take check disadvantage for 1 round
\end{itemize} 
For each subsequent round that you are able to maintain the performance without being interrupted, these effects repeat. 
