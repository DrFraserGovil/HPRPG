\documentclass[GMGuide.tex]{subfile}
%\documentclass{article}
\definecolor{backcol}{rgb}{0.2,0.2,0.2}
\definecolor{linecol}{rgb}{0.3,0.3,0.3}
\chapter{Beasts, Beings and Other Bad Business}

There are 3 classes of entity in this game: Beasts, Un-life and Sapients. 

Beasts are living beings without intelligence or self awareness, or with such a staggering level of violence and hatred that they cannot interact with other Sapients without imminent danger. Most beasts cannot use magic consciously, but may have innate magical abilities. 

Sapients are those creatures with intelligence, language and society. Mostly humanoids (though with a few exceptions), sapients can often wield magic with intent, unlike the mindless usage of the beasts. All player characters must be sapients. For political reasons, the word {\it Beings} is often used, though Sapients such as Centaurs and the Merpeople object to sharing this category with, for example, the hags, and hence are often classified as {\it beasts}, despite their evidently sapient nature. 

The Un-Life are those beings which cannot truly be called alive, and are often either inert matter brought to some mimicry of life by a spellcaster, or raw manifestations of pure magic. Often thought of as abominations due to creatures such as Dementors and Inferi, the Un-Life are often incredibly alien and hard to understand. 

\section{Beasts}

\newcommand{\combatdata}[3]{
\begin{tabular}{l l}
 HP: &   #1
\\
Armour: & \parbox[t]{6.9cm}{\vspace{-4.6ex}\begin{flushleft}#2\end{flushleft}\vspace{-3 ex}}
\\
Abilities: &  \parbox[t]{6.9cm}{#3}
\end{tabular}
}
 \newcommand{\beasttable}[8]{
 % \begin{center}
 %\fontsize{6}{6}\selectfont
 \begin{tabular}{|c|c|c|c|c|c|c|c|}
\hline
 ATH & FIN & SPR & CHR & INT & EMP & POW & EVL
 \\ 
 \hline
 #1 & #2 & #3 & #4 & #5 & #6 & #7 & #8 
 \\ \hline
 \end{tabular}
 %\end{center}
 }
\definecolor{titlered}{HTML}{58180D} 
\definecolor{rulered}{HTML}{9C2B1B} 
\definecolor{statblockbg}{HTML}{FDF1DC} 

\newcommand{\beastname}[2]
{
	\begin{center}
	
			{\bf \normalsize #1}
			
			\vspace{-0.2ex}
			
			{\it #2}
		\end{center}
}
\newcommand{\beastphysical}[3]
{
	{\bf Habitat:} #1       \\	{\bf #2:}   #3
}
\newcommand{\combatBlock}[2]
{
	{\bf HP:} #1 
	\\ 
	\raggedright {\bf Armour:} #2
}
\newcommand{\attackBlock}[2]
{
	{\bf Melee Attack:}  \parbox[t]{6.9cm}{#1}
	{\bf Ranged Attack:} \parbox[t]{6.9cm}{#2}
}
\newcommand{\abilitiesBlock}[3]
{
	{\bf Abilities:}  \parbox[t]{6.9cm}{\raggedright #1  #3  #2.}	
}
\newcommand{\beast}[8]
{
	\def\mode{#8}

	\if\mode0
		\definecolor{backcol}{HTML}{58180D}
		\definecolor{linecol}{HTML}{9C2B1B}
	\fi
	\if\mode1
		\definecolor{backcol}{rgb}{0.2,0.2,0.2}
		\definecolor{linecol}{rgb}{0.3,0.3,0.3}
	\fi
	\begin{tcolorbox}[  before skip=7pt plus 2pt,
	  boxrule=0pt,
	  boxsep=0pt,
	  toptitle=4pt,
	  left=7pt,
	  right=7pt,
	  bottom=11pt,
	  arc=0.5mm,
	  oversize=0pt,
	  colback=white,
	 colbacktitle=backcol,
	colframe=backcol, title=\vspace{-4ex}
	]
		
		\beastname#1
		\vspace{-3.3ex}
		\dndline
		\beastphysical#4
		\\
		\dndline
	
		{\raggedright #2}
		
		\dndline
		
		\vspace{1.4 ex}
		
		
		
		\combatBlock#5
		
		\attackBlock#6
		
		\abilitiesBlock#7
		\dndline
		
		{\begin{center} \beasttable#3 \end{center}}
	\end{tcolorbox}
}
\def\q{9.3}
\def\s{0.2}

\scriptsize

\begin{center}
%%BeastBegin
\begin{longtable}{p{\q cm} p{\s cm}  p{\q cm}}\beast{{Acromantula}{Giant Intelligent Spider \\ ~}}{A monstrous eight-eyed spider capable of human speech, its distinctive pincers produce a distinctive clicking sound when the Acromantula is excited or angry; and a poisonous secretion. The Acromantula is carnivorous and prefers large prey. It spins domeshaped webs upon the ground. Despite its near-human intellect the Acromantula is untrainable and highly dangerous to wizard and Muggle alike.}{ {25}{6}{10/13/15}{6}{18/20/24}{2}{15}{8} }{{Tropical  Rainforest}{Legspan}{4m}}{ {35 / 40 / 50}{Hard carapace on  abdomen, strength 10.} }{ {Bite: 6 / 10 / 15 + 1 d10 (piercing)}{None} }{ {Envenom (fangs, see {\it Sudden Death} potion for effect); Ensnare (2 turns, melee); Can talk;}{Exposed cephalothorax.}{50\% Magic Resistance;} }{0}\vspace{-2 cm}  & ~ & \beast{{Ashwinder}{Fire Snake}}{The Ashwinder is created when a magical fire is allowed to burn unchecked for too long. A thin, pale-grey serpent with glowing red eyes, it will rise from the embers of an unsupervised fire and slither away into the shadows of the dwelling in which it finds itself, leaving an ashy trail behind it. Incredibly delicate, the Ashwinder lives for only an hour and during that time seeks a dark and secluded spot in which to lay its eggs, after which it will collapse into dust.}{ {16}{5}{3}{3}{9}{5}{5}{0} }{{Magical Fires}{Length}{30cm}}{ {2 / 3 / 4}{None} }{ {Burn: 1d4 (fire, mild)}{Fire-spit: 1d6/8/10 (range 2m, fire damage: mild)} }{ {Ignite objects (range 1m);}{Water contact does 1d6 damage}{100\% Fire Resistance;} }{0} \vspace{-2 cm} \\ \beast{{Augrey}{Weather Forecasting Bird}}{A thin and mournfullooking bird, somewhat like a small and underfed vulture in appearance, the Augurey is greenish black. It is intensely shy, nests in bramble and thorn, eats large insects and fairies, flies only in heavy rain, and otherwise remains hidden in its tear-shaped nest. The Augurey has a distinctive low and throbbing cry, which was once believed to foretell death, research eventually revealed, however, that the Augurey merely sings at the approach of rain.}{ {2}{5}{14/16/20}{10}{8}{15}{2}{0} }{{Britain and Ireland}{Wingspan}{90cm}}{ {8}{None} }{ {Peck:  2 + 1d4 / 6 / 8 (piercing)}{None} }{ {Can sense changes in weather; ATH x10  in flight}{}{} }{0}\vspace{-2 cm}  & ~ & \beast{{Basilisk}{Giant Gorgon-Snake}}{The Basilisk is a brilliant green serpent that may reach up to fifty feet in length. The male has a scarlet plume upon its head. It has exceptionally venomous fangs but its most dangerous means of attack is the gaze of its large yellow eyes. Anyone looking directly into these will suffer instant death.}{ {16}{2}{17}{1}{1}{1}{20/21/22}{8/9/10} }{{Artificial, not found in nature}{Length}{20m}}{ {35 / 45 / 60}{Thick skin, strength 8.} }{ {Bite: 10 + 2d6 / 8 / 10  (piercing) \\ Tail Whip: 1d8 (bludegeoning)}{None} }{ {Gorgon stare (direct view: death, indirect: paralysis); Envenom (Fangs, Severe Poison status, 1d20 damage per turn)}{Exposed eyes and mouth}{70\% Magic Resistance;} }{0} \vspace{-2 cm} \\ \beast{{Bicorn}{Two-Horned Unicorn}}{The Bicorn is an equine creature similar to a unicorn and possessed of the same magical abilities. The only visible difference between the two is the Bicorn's two horns which grow one above the other on their heads. Bicorns are incredibly rare and the magic of their horns are even greater than that of a unicorn: and angering a bicorn is said to bring about nothing but bad luck.}{ {16/17/18}{4}{9}{10}{10}{10}{3}{3} }{{Indian Forests}{Length}{2m}}{ {14}{None.} }{ {Impale:  2d6 / 8 / 10 (piercing)}{None} }{ {Gore gives `Broken Wand� for 5 turns}{}{} }{0}\vspace{-2 cm}  & ~ & \beast{{Billywig}{Halluncinogenic Helicopter Insect}}{The Billywig is an insect native to Australia. It is around half an inch long and a vivid sapphire blue, although its speed is such that it is rarely noticed by Muggles and often not by wizards until they have been stung. The Billywig{\apos}s wings are attached to the top of its head and are rotated very fast so that it spins as it flies. At the bottom of the body is a long thin sting. Those who have been stung by a Billywig suffer giddiness followed by levitation.}{ {15/17/19}{1}{2}{1}{1}{3}{1}{0} }{{Austrailia}{Height}{3cm}}{ {3}{None} }{ {Sting: 1d4 (piercing)}{None} }{ {Envenom (euphoria, paralysis and levitation);}{}{} }{0} \vspace{-2 cm} \\ \beast{{Blast-Ended Skrewt}{Fire-Breathing Scorpions}}{Hybrid creatures bred from the unholy union of a manticore and a fire crab. The Blast-Ended Skrewt resemble giant, slimy scorpions with tails (and stingers) at both ends. The creatures are also able to shoot blasts of fire from either end, which they use to their advantage when hunting prey. Very difficult to harm, due to their strong shells, these creatures are not to be trifled with.}{ {7}{1}{4}{1}{2}{1}{10}{1} }{{Artificial, not found in nature}{Length}{2m}}{ {10}{Hard shell covers entire upper body, stength 25. Exposed underbelly.} }{ {Sting: 1d10 + 1 / 2 / 3 (piercing)}{Sparks: 1d6 + 2 (fire, mild)} }{ {Sparks fire in 2 directions at once;}{}{100\% Spell Resistance (Shell only)} }{0}\vspace{-2 cm}  & ~ & \beast{{Bowtruckle}{Tree Guardian}}{The Bowtruckle is a tree-guardian creature found deep in the forest. It is immensely difficult to spot, being small (maximum eight inches in height) and apparently made of bark and twigs with two small brown eyes. The Bowtruckle, which eats insects, is a peaceable and intensely shy creature but if the tree in which it lives is threatened, it has been known to leap down and gouge at their eyes of the wrongdoer with its long, sharp fingers. An offering of woodlice will placate the Bowtruckle long enough to let a witch or wizard remove wand-wood from its tree.}{ {4}{13/14/15}{3}{3}{6}{8}{2}{0} }{{Forests of Northern Europe}{Height}{15cm}}{ {5 / 6 / 7}{None} }{ {Scratch: 1d4 (slashing)}{None} }{ {Camouflage (in foliage);}{200\% Fire Damage}{} }{0} \vspace{-2 cm} \\ \beast{{Bundimun}{Living Fungus}}{Skilled at creeping under floorboards and behind skirting boards, bundimun infest houses. The presence of a Bundimun is usually announced by a foul stench of decay. The Bundimun oozes a secretion which rots away the very foundations of the dwelling in which it is found. The Bundimun at rest resembles a patch of greenish fungus with eyes, though when alarmed it will scuttle away on its 6 numerous spindly legs. It feeds on dirt.}{ {2}{1}{1}{1}{1}{1}{1}{0} }{{Worldwide}{Width}{20cm}}{ {10 / 12 / 15}{None.} }{ {Ooze: 1d4 / 6 / 8 (poison)}{Toxic Spore: 1d6 / 8 / 12} }{ {Toxic skin (contact, 1d4  poison damage per turn)}{}{} }{0}\vspace{-2 cm}  & ~ & \beast{{Chimera}{Vicious Multi-Headed Hybrid}}{The Chimeria is a rare greek monster which appears as crossbreed of a lion, a goat, a dragon, with three heads. All three heads act independently, and the dragon head can breath fire, some species (potentially the males, though none who have attempted to check the gender have survived) have a fourth head � that of a snake � as a tail. An extremely vicious and bloodthirsty animal, there is only one known instance of a wizard slaying a chimera -- and he died from exhaustion immediately afterwards.}{ {15/17/20}{2}{30}{1}{20}{2}{25}{9/11/13} }{{Greece}{Length}{4m}}{ {70 / 80 / 100}{Magical skin, strength 25} }{ {Bite:  3d 8 / 10 / 20  (piercing)}{Fire Breath: 1d 8 / 10 / 12 (fire, mild)} }{ {Flight (prolonged);}{}{100\% Magic Resistance (Below Expert)} }{0} \vspace{-2 cm} \\ \beast{{Chizpurfle}{Magical Parasite}}{Chizpurfles are small parasites up to a twentieth of an inch high, crablike in appearance, with large fangs. They are attracted by magic and may infest the fur and feathers of such creatures as Crups and Augureys. They will also enter wizard dwellings and attack magical objects such as wands, gradually gnawing their way through to the magical core, or else settle in dirty cauldrons, where they will gorge upon any lingering drops of potion, or even attack muggle electronics.}{ {6}{2}{4}{1}{2}{2}{1}{0} }{{Worldwide}{Length}{10 mm}}{ {1}{None} }{ {Bite: 1d4 (piercing)}{None} }{ {Ambient Absorb (releases unpredictably)}{}{} }{0}\vspace{-2 cm}  & ~ & \beast{{Cockatrice}{Failed Basilisk}}{A cockatrice is the result of a failed attempt at the creation of a basilisk, retaining the characteristics of the chicken from the egg it hatched from: resembling a stunted basilisk, with a chicken{\apos}s head.  he gaze of a cockatrice is not deadly like a basilisk{\apos}s, but it does cause a severe paralysis. The cockatrice is far quicker than a basilisk, and there have been recordings of a cockatrice being able to kill an adolescent basilisk, because of its superior agility and intelligence.}{ {20/22/24}{7}{14}{2}{5}{5}{14}{6} }{{Artificial, not found in nature}{Length}{10m}}{ {25 / 35 / 40}{Thick skin, strength 6. Exposed eyes and mouth.} }{ {Bite: 10 + 1d6 / 8 / 10 (piercing)\\ Tail Whip: 1d8 (bludgeoning)}{None} }{ {Gorgon Stare (paralysis);}{}{20\% Magic Resistance} }{0} \vspace{-2 cm} \\ \beast{{Demiguise}{Invisible Ape}}{The Demiguise is found in the Far East, though only with great difficulty, for this beast is able to make itself invisible when threatened, and can be seen only by wizards skilled in its capture. The Demiguise is a peaceful herbivorous beast, something like a graceful ape in appearance, with large, black, doleful eyes more often than not hidden by its hair. The whole body is covered with long, fine, silky, silvery hair. Demiguise pelts are highly valued as the hair may be spun into Invisibility Cloaks.}{ {9}{9}{8}{4}{11}{12}{3}{0} }{{Far East}{Height}{40cm}}{ {7 / 8 / 9}{None} }{ {Scratch: 1d4  (slashing) \\ Bite 1d6 + 1 / 2 / 3 (piercing)}{None} }{ {Invisibility (at will)}{}{} }{0}\vspace{-2 cm}  & ~ & \beast{{Diricawl}{Teleporting Bird}}{A plump-bodied, fluffyfeathered, flightless bird, the Diricawl is remarkable for its method of escaping danger. It can vanish in a puff of feathers and reappear elsewhere. Interestingly, Muggles were once fully aware of the existence of the Diricawl, though they knew it by the name of {\it dodo.}}{ {3}{2}{5}{3}{4}{5}{2}{0} }{{Worldwide}{Wingspan}{10cm}}{ {4 / 5 / 10}{None} }{ {Peck: 1d4 + 1 / 2 / 3 (piercing)}{None} }{ {Apparate (range: 2 / 4 / 6m );  ATH + FIN x10 when in air}{}{} }{0} \vspace{-2 cm} \\ \beast{{Doxy}{Beetle Fairy}}{The Doxy is often mistaken for a fairy (see page 16) though it is a quite separate species. Like the fairy, it has a minute human form, though in the Doxy{\apos}s case this is covered in thick black hair and has an extra pair of arms and legs. The Doxy{\apos}s wings are thick, curved, and shiny, much like a beetle{\apos}s. Doxies have double rows of sharp, venomous teeth. An antidote should be taken if bitten.}{ {6}{7}{4}{2}{3}{2}{4}{3} }{{Northern Europe and North America}{Height}{3cm}}{ {4 / 7 / 10}{None} }{ {Bite: 1d6 + 3 / 4 / 5 (piercing)}{None} }{ {Envenom (1d4 poison damage per turn for 10 turns); ATH + FIN double in flight;}{}{100\% Cold Resistance} }{0}\vspace{-2 cm}  & ~ & \beast{{Dragon}{Antipodean Opaleye}}{Perhaps the most beautiful type of dragon, it has iridescent, pearly scales and glittering, multi-coloured, pupil-less eyes, hence its name. This dragon produces a very vivid scarlet flame, though by dragon standards it is not particularly aggressive and will rarely kill unless hungry. The Antipodean opaleye is amongst the most intelligent and understanding of all the dragons.}{ {10}{3}{14/15/16}{9}{14}{10}{19}{2} }{{New Zealand}{Length}{10m}}{ {40 / 50 / 60}{Thick skin, strength 18, covers most of the body} }{ {Bite: 1d8 / 10 / 20 \\ Scratch: 1d8 + 3 / 5 / 7}{Fire Breath: 5 + 1d8 (range 10m, fire damage: moderate)} }{ {Flight (prolonged);}{}{50\% Magic Resistance (scales only)} }{0} \vspace{-2 cm} \\ \beast{{Dragon}{Chinese Fireball}}{The only Oriental (and also only flightless) dragon has a particularly striking appearance. Scarlet and smooth-scaled, it has a fringe of golden spikes around its snub-snouted face and extremely protuberant eyes, which distract from the atrophied wings. The Fireball gained its name for the mushroom-shaped flame that bursts from its nostrils when it is angered.  The Fireball is very aggressive, but more tolerant of its own species than most dragons, sometimes consenting to share its territory with up to two others. The fire of the Chinese Fireball burns particularly hot.}{ {14/15/16}{10}{11/12/13}{5}{9}{5}{17}{4} }{{China \& the Himalayas}{Length}{15m}}{ {35 / 40 / 45}{Thick skin, strength 15, covers most of the body} }{ {Bite: 1d6 / 8 / 10 (piercing)}{Fire Breath:  1/2/3 d10 (range 5m, fire damage: severe)} }{ {Underwater breathing + movement;}{}{50\% Fire and Cold Resistance} }{0}\vspace{-2 cm}  & ~ & \beast{{Dragon}{Common Welsh Green}}{The Welsh Green blends well with the lush grass of its homeland, though it nests in the higher mountains, where a reservation has been established for its preservation. This breed is among the least troublesome of the dragons, preferring, like the Opaleye, to prey on sheep and actively avoiding humans unless provoked. Fire is issued in thin jets, and is notably colder than many other species. This species is far more at home in the sky than on the ground, unlike  most species which are cumbersome in the air.}{ {7}{8}{18}{3}{8/10/12}{10}{15}{1} }{{Snowdonia}{Length}{8m}}{ {30 / 40 / 50}{Thick skin, strength 14, covers most of the body} }{ {Bite: 1d10 + 3 / 4 / 5 (piercing)\\ Scratch: 1d 6 / 8 / 10 (slashing)}{Fire Breath: 4 + 1d8 (range 10m, fire damage: mild)} }{ {Flight (prolonged); ATH + FIN attribute are doubled in flight.}{}{50\% Fire Resistance} }{0} \vspace{-2 cm} \\ \beast{{Dragon}{Hebridean Black}}{Britain{\apos}s other native dragon is more aggressive than its Welsh counterpart. It requires a territory of as much as a hundred square miles per dragon. Up to thirty feet in length, the Hebridean Black is rough-scaled, with brilliant purple eyes and a line of shallow but razor-sharp ridges along its back. The Hebridean Black is noted for its brutality and cruelty, tampered only somewhat by its immense stupidity.}{ {8}{4}{15}{2}{4}{4}{19}{5} }{{Outer Hebrides}{Length}{10m}}{ {45 / 55 / 65}{Thick skin, strength 24 / 25 / 26, covers most of the body} }{ {Bite: 1d10 / 12 / 20 (piercing)}{Fire Breath: 1d12 +5/7/10 (range 10m, fire damage: moderate)} }{ {Flight (prolonged);}{}{50\% Fire Resistance} }{0}\vspace{-2 cm}  & ~ & \beast{{Dragon}{Hungarian Horntail}}{Supposedly the most dangerous of all dragon breeds, the 13 Hungarian Horntail has black scales and is lizardlike in appearance. It has yellow eyes, bronze horns, and similarly coloured spikes that protrude from its long tail. The Horntail has one of the longest fire-breathing ranges (up to fifty feet).}{ {9}{3}{10}{2}{5}{3}{24}{4} }{{Hungarian Mountains}{Length}{12m}}{ {50 / 60 /70}{Thick skin, strength 20, covers most of the body} }{ {Bite: 1/2/3d8 (piercing)\\ Scratch: 1d10 + 3 / 4 / 5 (slashing)}{Fire Breath: 1d12 + 5/7/10 (range 20m, fire damage: moderate)} }{ {Flight (prolonged);}{}{50\% Fire Resistance} }{0} \vspace{-2 cm} \\ \beast{{Dragon}{Norwegian Ridgeback}}{The Norwegian Ridgeback resembles the Horntail in most respects, though instead of tail spikes it sports particularly prominent jet-black ridges along its back. Exceptionally aggressive to its own kind, the Ridgeback is nowadays one of the rarer dragon breeds. It has been known to attack most kinds of large land mammal and, unusually for a dragon, the Ridgeback will also feed on water-dwelling creatures.}{ {8}{4}{13}{1}{6/7/8}{6}{20/21/22}{3} }{{Scandinvia}{Length}{10m}}{ {45 / 54 / 65}{Thick skin, strength 19, covers most of the body} }{ {Bite: 1 d 8 / 10 / 20 + 5 (piercing) \\ Scratch 2d6 (slashing)}{Fire Breath: 1d10 + 4 / 8 / 12 (range 8m, fire damage: moderate)} }{ {Flight (prolonged);}{}{100\% Fire and Cold Resistance} }{0}\vspace{-2 cm}  & ~ & \beast{{Dragon}{Peruvian Vipertooth}}{This is the smallest of all known dragons, and the swiftest in flight  the Peruvian Vipertooth is smooth-scaled and copper-coloured with black ridge markings. They live perpetually in flight in the turbulent airstreams above the andes, thriving in the thin air.  Delicate yet elegant, they are the strongest fliers of all the species of dragon, and rely on their manouverability over raw strength.}{ {14}{10}{8}{2}{9/10/11}{8/9/10}{14}{1} }{{Andes}{Length}{5m}}{ {30}{Thick skin, strength 10, covers most of the body} }{ {Bite 1d8 + 3 / 4 / 5 (piercing)}{Fire Breath: 1d 6 / 8/ 10 + 5 (range 2m, fire damage: mild)} }{ {Cannot move whilst on land;  ATH + FIN attribute are doubled in flight}{}{100\% Force and Concussive Resistance} }{0} \vspace{-2 cm} \\ \beast{{Dragon}{Romanian Longhorn}}{The Longhorn has dark-green scales and long, glittering golden horns with which it gores its prey before roasting it. When powdered, these horns are highly valued as potion ingredients, resulting in the Longhorn being hunted almost to extinction.}{ {9}{6}{10/12/14}{1}{6}{4}{17}{1} }{{Romania}{Length}{8m}}{ {30 / 40 / 50}{Thick skin, strength 20, covers most of the body} }{ {Bite 1d 6  / 8 / 10 + 4 (piercing)}{Fire Breath: 1d 12 + 3 / 4 / 5 (range 10m, fire damage: moderate)} }{ {Flight (prolonged);}{}{50\% Fire Resistance} }{0}\vspace{-2 cm}  & ~ & \beast{{Dragon}{Swedish Shortsnout}}{The Swedish Short-Snout is an attractive silvery-blue dragon whose skin is sought after for the manufacture of protective gloves and shields. The flame that issues from its nostrils is a brilliant blue and can reduce timber and bone to ash in a matter of seconds.   The Short-Snout has fewer human killings to its name than most dragons, though as it prefers to live in the fjords of Scandinavia, this is unsurprising.}{ {8}{4}{9/10/11}{2}{7/9/12}{3}{18}{1} }{{Fjords of Scandinavia}{Length}{12m}}{ {45 / 50 / 55}{Thick skin, strength 16, covers most of the body} }{ {Bite: 1d 6 / 8 / 12 (piercing) \\ Scratch: 2d4/6/8 (slashing)}{Fire Breath: 3 + 2d6/8/10 (range 10m, fire damage: severe)} }{ {Flight (prolonged);}{}{50\% Fire Resistance} }{0} \vspace{-2 cm} \\ \beast{{Dragon}{Ukranian Ironbelly}}{The largest breed of dragon, the Ironbelly, has been known to achieve a weight of six tonnes. Rotund and slower in flight than the Vipertooth or the Longhorn, the Ironbelly is nevertheless extremely dangerous, capable of crushing dwellings on which it lands. The scales are metallic grey, the eyes deep red, and the talons particularly long and vicious.}{ {4}{1}{8/9/10}{1}{6}{3}{22}{2} }{{Ukraine and Crimea}{Length}{16m}}{ {80 / 90 / 100}{Thick skin, strength 25 / 28 / 35, covers most of the body. Confers 100\% resistance to fire and magic.} }{ {Bite 1d 8 + 4 / 5 / 6 (piercing)\\ Crush: 3d8 (bludgeoning) \\ Scratch: 1d8 + 4 / 5 / 6 (slashing)}{Fire Breath: 2d4 / 6 / 8 (range 15m, moderate)} }{ {Flight (prolonged);}{}{80\% Fire Resistance} }{0}\vspace{-2 cm}  & ~ & \beast{{Dugbog}{Aquatic Ambush Predator}}{The Dugbog is a marsh-dwelling creature, resembling a piece of dead wood while stationary, though closer examination will reveal finned paws and very sharp teeth. It glides and slithers through marshland, feeding mainly on small mammals, and will do severe injury to the ankles of human walkers. The Dugbog{\apos}s favourite food, however, is Mandrake.}{ {8}{3}{5}{2}{3}{2}{5}{0} }{{Throughout Europe and the Americas}{Length}{30cm}}{ {10}{Driftwood protects the back, strength 6} }{ {Bite: 1d6 + 2/3/4 (piercing)}{None} }{ {Can breath underwater;}{}{100\% Fire Resistance} }{0} \vspace{-2 cm} \\ \beast{{Erkling}{Carniverous Elf-Demon}}{Erklings are elfish creatures, three feet tall on average (making them larger than gnomes) with pointed faces, which have a particular affinity for the taste of children. Their high pitch cackles are particularly entrancing to children, and they use this to lure them away from their guardians to eat them. These creatures also enjoy shooting darts at unsuspecting victims. Originating from the Black Forest in Germany, the difference between this creature and many others is that they can speak Human-language.}{ {7}{5}{6}{15}{10}{9}{8}{4} }{{Southern Germany}{Height}{70cm}}{ {15 / 16/ 17}{None} }{ {Scratch: 1d6 + 1 / 2 / 3 (slashing)}{Dart: 1d4 + 1 /2 / 3 (poison)} }{ {Capable of human speech;}{50\% Fire Weakness}{} }{0}\vspace{-2 cm}  & ~ & \beast{{Erumpet}{Exploding-horn Rhinoceros}}{The Erumpent is a large grey African beast of great power. Weighing up to a tonne, the Erumpent may be mistaken for a rhinoceros at a distance. It has a thick hide that repels most charms and curses, a large, sharp horn upon its nose and a long, rope-like tail. Erumpents give birth to only one calf at a time. The Erumpent will not attack unless sorely provoked, but should it charge, the results are usually catastrophic. The Erumpent{\apos}s horn can pierce everything from skin to metal, and contains a deadly fluid which will cause whatever is injected with it to explode.}{ {10}{1}{9}{6}{8/10/12}{10}{14}{0} }{{African Savannahs}{Length}{3m}}{ {25 / 26 / 30}{Thick skin, strength 14, protects the entire body.} }{ {Gore: 1d8/10/12 (requires run up, piercing))}{None} }{ {Gored items explode for 1d10 concussive damage (2m radius) next turn;  Gore attack ignores all physical armour;}{}{100\% Magic Resistance (Below Expert)} }{0} \vspace{-2 cm} \\ \beast{{Fairy}{Decorative Humanoid}}{The fairy is a small and decorative beast of little intelligence. Often used or conjured by wizards for decoration, the fairy generally inhabits woodlands or glades. Ranging in height from one to five inches, the fairy has a minute humanoid body, head, and limbs but sports large insectlike wings, which may be transparent or multi-coloured, according to type. The fairy possesses a weak brand of magic that it may use to deter predators, such as the Augurey. It has a quarrelsome nature but, being excessively vain, it will become docile on any occasion when it is called to act as an ornament. Despite its humanlike appearance, the fairy cannot speak. It makes a high-pitched buzzing noise to communicate with its fellows.}{ {6}{9}{8}{10}{2}{7}{5}{0} }{{Worldwide}{Height}{5cm}}{ {5 / 6 / 7}{None} }{ {Scratch: 1d4 (slashing)}{Magical Discharge: 1d6 (range 4m, concussive)} }{ {Flight (prolonged); Can glow brightly at will;}{100\% Concussive weakness}{} }{0}\vspace{-2 cm}  & ~ & \beast{{Fire Crab}{Fire-Shooting Shelled Creature}}{Despite its name, the fire crab greatly resembles a large tortoise with a heavily jewelled shell. Despite their slow speed, they are rarely eaten by predators because of their unique defence mechanism: they can shoot flames from their rear end, hot enough to melt steel.}{ {5}{2}{7}{2}{4}{5}{10}{0} }{{Fiji}{Length}{1m}}{ {24/25/26}{Shell, strength 30, protects the back} }{ {None}{Fire shot: 1d8/10/12 (range 3m, fire damage: mild)} }{ {}{40\% Cold Weakness}{100\% Fire Resistance} }{0} \vspace{-2 cm} \\ \beast{{Flesh-Eating Slug}{Evil Carniverous Invertebrates}}{A slug which is superficially similar to the non-magical variety. As the name  suggests, however, these creatures have a penchant for human flesh. Their slime is also incredibly resistant to fire.}{ {2}{0}{3}{1}{0}{2}{2}{3} }{{Worldwide}{Length}{4cm}}{ {4 / 5 / 6}{None} }{ {Toxic Sludge: 1d6 (poison)}{None} }{ {Toxic skin (contact, 1d4  poison damage per turn);}{}{60\% Fire Resistance} }{0}\vspace{-2 cm}  & ~ & \beast{{Flobberworm}{Who Cares?}}{Possibly the most pointless animal to have ever existed, the Flobberworm lives in damp ditches. A thick brown worm reaching up to ten inches in length, the Flobberworm moves very little. One end is indistinguishable from the other, both producing the mucus from which its name is derived and which is sometimes used to thicken potions. The Flobberworm{\apos}s preferred food is lettuce, though it will eat almost any vegetation.}{ {0}{0}{0}{0}{0}{0}{0}{0} }{{Nobody cares enough to find out.}{Length}{20cm}}{ {2 / 3 / 4}{None} }{ {None}{None} }{ {Flobber}{}{} }{0} \vspace{-2 cm} \\ \beast{{Frost Salamander}{Icy-Cold Lizard}}{Whilst a normal salamander is closely associated with fire, the frost salamander has a body temperature below freezing, and hence resides where there is permafrost; although there is some debate about if Frost Salamanders live where there is permafrost, or if permafrost exists because Frost Salamanders live there.  They are incredibly sensitive to changes in temperature, getting stronger as it gets colder and vice-versa.}{ {7}{3}{4}{3}{6}{6}{9}{0} }{{Arctic circle, glaciated regions}{Length}{15cm}}{ {10 / 15 / 20}{None} }{ {Bite: 1d6 (cold damage: moderate)}{None} }{ {All contact causes Frostbite (mild) status;}{100\% Fire Weakness}{100\% Cold Resistance} }{0}\vspace{-2 cm}  & ~ & \beast{{Fwooper}{Insanity-Causing Parrot}}{A bird that may be orange, pink, lime green, or yellow. The Fwooper has long been a provider of fancy quills and also lays brilliantly patterned eggs. Though at first enjoyable, Fwooper song will eventually drive the listener to insanity.}{ {7}{7}{7}{13}{4}{9}{3}{1} }{{African Rainforests}{Wingspan}{30cm}}{ {5 / 6 / 7}{None} }{ {Peck: 1d4 + 1 / 2 / 3 (piercing)}{Piercing Song:  1d6/8/10  (fatigue, range 20m)} }{ {Flight (prolonged, light load only)}{}{} }{0} \vspace{-2 cm} \\ \beast{{Ghoul}{Dim-witted Ogre}}{A ghoul resembles a somewhat slimy, buck-toothed ogre, and generally resides in attics or barns belonging to wizards, where it eats spiders and moths. It moans and occasionally throws objects around, but is essentially simple-minded and will, at worst, growl alarmingly at anyone who stumbles across it. \\ They are, however, incredibly perceptive to the emotions of those around them, and it is said that their noisy activities correlate with the level of emotion in the house.}{ {10}{4}{8}{3}{6}{15}{6}{0} }{{Attics}{Height}{1.8m}}{ {25 / 26 / 30}{None} }{ {Chains: 1d6 + 1 / 2 / 3 (bludgeoning)}{None} }{ {None}{}{} }{0}\vspace{-2 cm}  & ~ & \beast{{Glumbumble}{Sleep-inducing Moth}}{The Glumbumble is a grey, furry-bodied flying insect that produces melancholy-inducing treacle, which is used as an antidote to the hysteria produced by eating Alihotsy leaves. It has been known to infest beehives, with disastrous effects on the honey. Glumbumbles nest in dark and secluded places such as hollow trees and caves}{ {5}{2}{3}{2}{2}{3}{5}{0} }{{Beehives in Northern Europe}{Length}{1cm}}{ {3}{None} }{ {Sting: 1d4 (piercing)}{None} }{ {Sleep (sting, 3 turns)}{}{} }{0} \vspace{-2 cm} \\ \beast{{Gnome}{Humanoid Pest}}{The gnome is a common garden pest found throughout northern Europe and North America. It may reach a foot in height, with a disproportionately large head and hard, bony feet. The gnome can be expelled from the garden by swinging it in circles until dizzy and then dropping it over the garden wall.}{ {6}{4}{8}{5}{6}{6}{5}{0} }{{Throughout Europe and the Americas}{Height}{30cm}}{ {15 / 18 / 20}{None} }{ {Bite: 1d6  + 1 / 2 / 3 (piercing)}{None} }{ {None}{}{} }{0}\vspace{-2 cm}  & ~ & \beast{{Graphorn}{Tentacle-Mouthed Quadruped}}{Large and greyish purple with a humped back, the Graphorn has two 20 very long, sharp horns, walks on large, four-thumbed feet, and has an extremely aggressive nature. Mountain trolls can occasionally be seen mounted on Graphorns, though the latter do not seem to take kindly to attempts to tame them and it is more common to see a troll covered in Graphorn scars. 
Powdered Graphorn horn is used in many potions, though it is immensely expensive owing to the difficulty in collecting it. Graphorn hide is even tougher than a dragon�s and repels most spells.}{ {10}{3}{18/19/20}{5}{9}{6}{13}{1} }{{Mountains of Europe}{Length}{3m}}{ {30}{Thick skin, strength 30/40/50, covers most of the body} }{ {Impale: 1d8 / 10 / 12 (requires run up, piercing) \\ Strangle: 1d6 (fatigue)}{None} }{ {}{}{100\% Magic Resistance (Below Expert, skin only)} }{0} \vspace{-2 cm} \\ \beast{{Griffin}{Half lion-half eagle}}{Griffin�s possess the front legs and head of a giant eagle, but the body and hind legs of a lion. Like sphinxes, griffins are often employed by wizards to guard treasure. Though griffins are fierce, a handful of skilled wizards have been known to befriend one.}{ {14}{6}{16/17/18}{7}{14}{9}{10}{1} }{{Greece and Macedonia}{Length}{2m}}{ {15 / 25 / 35}{None} }{ {Peck: 1d8 / 10 / 12 (piercing)\\ Scratch: 1d6 + 2 / 4 / 6 (slashing)}{None} }{ {Flight (prolonged)}{}{} }{0}\vspace{-2 cm}  & ~ & \beast{{Grindylow}{Water Demon}}{A horned, pale-green water demon, the Grindylow is found in lakes throughout Britain and Ireland. It feeds on small fish and is aggressive towards wizards and Muggles alike. The Grindylow has very long fingers, which, though they exert a powerful grip, are easy to break.}{ {8}{10}{6}{4}{6}{4}{9}{3} }{{European Lakes}{Height}{50cm}}{ {10 / 11 / 12}{None} }{ {Bite: 1d6  (piercing)\\ Strangle: 1d6/8/10 (fatigue)}{} }{ {Once strangle is initiated, does damage once per turn until broken; Underwater breathing;}{}{80\% Fire Resistance} }{0} \vspace{-2 cm} \\ \beast{{Gringwart Goff}{Shadow Demon}}{A demon-like creature with long horns on its head, bat-like wings and three long fingers on each hand. Creatures of shadow, they fear the light. Unusually for beasts, they seem capable of using true magic, and can cast actual spells � though they appear to have a preference for the Dark Arts. 
Since the Gringwart Goff is (apparently) immortal, it is speculated that they are the result of dark wizards attempting to live forever, and accidentally transforming themselves into monstrosities.}{ {9}{12}{14}{6}{16}{0}{14}{6} }{{Deep caves and pits worldwide}{Height}{1m}}{ {20 / 40 / 60}{None} }{ {Scratch: 1d 6 (slashing)}{None} }{ {Can cast all spells below expert level}{}{} }{0}\vspace{-2 cm}  & ~ & \beast{{Hidebehind}{Vengeful Shapeshifter}}{The result of an accidental crossbreeding between a demiguise and a ghoul, the hidebehind is a shape-shifting creature that, in its natural state, looks like a silvery-haired bipedal bear.  Residing in forests, the Hidebehind appears to seek vengeance on the humans who cruelly created it, using their shapeshifting to sneak up on them, and then ravage them with their immense strength.}{ {15/17/19}{6}{10}{14}{5}{9}{13}{3} }{{New World Forests}{Height}{2m}}{ {15 / 20 / 25}{None} }{ {Maul: 1d10 + 2 / 3 / 4 (piercing)}{None} }{ {Shapeshift (any form, at will, major action)}{}{} }{0} \vspace{-2 cm} \\ \beast{{Hippocampus}{Aquatic Equine Creature}}{The hippocampus has the head and forequarters of a horse, and the tail and hindquarters of a giant fish. The hippocampus is often domesticated by merpeople, to use for both transport, and for underwater warfare. Unlike the other equine-based magical animals, the hippocampus is generally considered rather dim, though its ability to shoot water at ultra-high velocities from its mouth can catch the unwary by surprise.}{ {15}{8}{5}{5}{4}{7}{6}{1} }{{Mediterranean}{Length}{2m}}{ {20}{Scales on rear half, strength 8} }{ {Trample: 1d6 (bludgeoning)}{Water Jet: 1d8 + 0 / 1 / 2 (concussive)} }{ {Underwater breathing; FIN =1 when out of the water;}{}{} }{0}\vspace{-2 cm}  & ~ & \beast{{Hippogriff}{Half Horse, Half Eagle}}{The hippogriff the head of a giant eagle and the body of a horse. It can be tamed, though this should be attempted only by experts. Eye contact should be maintained when approaching a Hippogriff. Bowing shows good intentions, if the gesture is returned, it is safe to draw closer. Very intelligent creatures, the hippogriff should not be underestimated.}{ {13}{7}{17}{7}{15/16/17}{14}{15}{0} }{{European Lakes and Mountains}{Length}{2m}}{ {20 / 30 / 35}{None} }{ {Peck: 1d8 / 10 / 12 (piercing)}{None} }{ {Flight (prolonged)}{}{} }{0} \vspace{-2 cm} \\ \beast{{Horklump}{Living Fungus}}{It resembles a fleshy, pinkish mushroom covered in sparse, wiry black bristles. A prodigious breeder, the Horklump will cover an average garden in a matter of days. It spreads sinewy tentacles rather than roots into the ground to search for its preferred food of earthworms. The Horklump is a favourite delicacy of gnomes but otherwise has no discernible use.}{ {2}{1}{4}{1}{6}{6}{2}{0} }{{Scandinvia}{Width}{30cm}}{ {8}{None} }{ {Vine whip: 1d6 (2m range, bludgeoning)}{None} }{ {Ground Feed (gain 4HP per turn); Breed (after Feeding 3 times, splits into 3 new horklumps)}{}{} }{0}\vspace{-2 cm}  & ~ & \beast{{Imp}{Small Humanoid Trickster}}{ometimes confused with the pixie, Imps are of similar height (between six and eight inches), though the imp cannot fly as the pixie can, nor is it as vividly coloured (the imp is usually dark brown to black). It does, however, have a similar slapstick sense of humour. Its preferred terrain is damp and marshy, and it is often found near river banks, where it will amuse itself by pushing and tripping the unwary.}{ {9}{14}{7}{10}{9}{6}{4}{2} }{{British Isles}{Height}{14cm}}{ {8 / 10 / 15}{None} }{ {Bite: 1d6 (piercing)}{None} }{ {Mimic human voices}{}{} }{0} \vspace{-2 cm} \\ \beast{{Jarvey}{Talking Ferret}}{It resembles an overgrown ferret in most respects, except for the fact that it can talk. True conversation, however, is beyond the wit of the Jarvey, which tends to confine itself to short (and often rude) phrases in an almost constant stream. Jarveys live mostly below ground, where they pursue gnomes.}{ {9}{13}{6}{14/15/16}{9}{7}{3}{0} }{{Britain and Ireland}{Length}{60cm}}{ {16}{None} }{ {Bite: 1d4 + 3 / 4 / 5 (piercing)}{None} }{ {(Limited) human speech}{}{} }{0}\vspace{-2 cm}  & ~ & \beast{{Jobberknoll}{Perfect Recollection Songbird}}{The Jobberknoll is a tiny blue, speckled bird which eats small insects. It makes no sound until the moment of its death, at which point it lets out a long scream made up of every sound it has ever heard, regurgitated backwards. Jobberknoll feathers are used in Truth Serums and Memory Potions.}{ {6}{7}{4}{5}{9}{14}{2}{0} }{{Northern Europe and North America}{Wingspan}{10cm}}{ {6}{None} }{ {Peck: 1d4 (piercing)}{None} }{ {Perfectly recollection}{}{} }{0} \vspace{-2 cm} \\ \beast{{Kappa}{Japanese Water Demon}}{The Kappa is a Japanese water demon that inhabits shallow ponds and rivers. Often said to look like a monkey with fish scales instead of fur, it has a hollow in the top of its head in which it carries water. The Kappa feeds on human blood but may be persuaded not to harm a person if it is thrown a cucumber with that person�s name carved into it. 
In confrontation, a wizard should trick the Kappa into bowing � if it does so, the water in the hollow of its head will run out, depriving it of all its strength.}{ {13}{12}{8}{6}{9/10/11}{2}{8}{4} }{{Japan}{Height}{60cm}}{ {25 / 26 / 30}{Scales, strength 6, cover entire body} }{ {Drain:  1d8 (psychic)}{None} }{ {Absorb (50\% of damage done is restored to kappa); If water in head removed, all attributes set to 1 and HP halved; Underwater breathing}{}{} }{0}\vspace{-2 cm}  & ~ & \beast{{Kelpie}{Shapeshifting Aquatic Demon}}{A water demon that can take many forms, a kelpie most often appears as a horse with bullrushes for a mane. Having lured the unwary onto its back, it will dive straight to the bottom of its river or lake and devour the rider, letting the entrails float to the surface. The correct means to overcome a kelpie is to get a bridle over its head with a Placement Charm, which renders it docile and unthreatening. It is speculated that �Nessie� is, in fact, a large Kelpie.}{ {9}{10}{13}{13}{7/9/12}{10}{4}{4} }{{Scotland}{Height:}{Variable}}{ {10 / 12 / 15}{Depends on the shape taken} }{ {Devour: 1d8/10/12 (necrotic) \\ (Form-dependent attack)}{} }{ {Devour takes 2 turns to complete \\ Attacks vary depending on the shape taken}{}{} }{0} \vspace{-2 cm} \\ \beast{{Kneazle}{Empathetic Cat}}{A small catlike creature with flecked, speckled, or spotted fur, outsize ears, and a tail like a lion�s, the Kneazle is intelligent, independent, and occasionally aggressive, though if it takes a liking to a witch or wizard, it makes an excellent pet. The Kneazle has an uncanny ability to detect unsavoury or suspicious characters and can be relied upon to guide its owner safely home if they are lost.}{ {8}{7}{11/12/13}{10}{9}{12/13/14}{6}{0} }{{Worldwide}{Length}{30cm}}{ {12}{None} }{ {Scratch: 1d6/8/10 (slashing)}{None} }{ {None}{}{} }{0}\vspace{-2 cm}  & ~ & \beast{{Lobalug}{Venemous Fish}}{The lobalug is a simple aquatic creature, ten inches long, comprising a rubbery spout and a venom sac, residing on the ocean floor. When threatened, the Lobalug contracts its venom sac, blasting the attacker with poison. Merpeople use the Lobalug as a weapon and wizards have been known to extract its poison for use in potions, though this practice is strictly controlled.}{ {3}{2}{5}{3}{2}{3}{4}{0} }{{North Sea}{Length}{25cm}}{ {10}{None} }{ {None}{Venom spit: 1d6 (poison, 1d6 damage for 5 turns)} }{ {Underwater breathing + movement;}{}{} }{0} \vspace{-2 cm} \\ \beast{{Mackled Malaclaw}{Unluckly Giant Lobster}}{The Malaclaw is a land-dwelling creature found mostly on rocky coastline. Despite its passing resemblance to the lobster, it should on no account be eaten, as its flesh is unfit for human consumption and will result in a high fever and an unsightly greenish rash. The Malaclaw can reach a length of twelve inches and is light grey with deep-green spots. It eats small crustaceans and will attempt to tackle larger prey.  The Malaclaw�s bite has the unusual side effect of making the victim highly unlucky for a period of up to a week after the injury. If you are bitten by a Malaclaw, all bets, wagers, and speculative ventures should be cancelled, as they are sure to go against the victim.}{ {5}{3}{10}{3}{10}{4}{5}{0} }{{Irish coastlines}{Length}{30cm}}{ {8}{Tough shell, strength 5, covers entire body} }{ {Pincer: 1d6/8/12 (piercing)}{None} }{ {Unlucky status on successful attack}{}{} }{0}\vspace{-2 cm}  & ~ & \beast{{Manticore}{Single-headed Chimera}}{The manticore is a highly dangerous Greek beast closeley related to the Chimera. The Manticore possesses the head of a man, the body of a lion, the wings of a (giant) bat,  and the tail of a scorpion. As dangerous as the Chimaera, and as rare, the manticore is reputed to croon softly as it devours its prey. Manticore skin repels almost all known charms and the sting causes instant death.}{ {16}{5}{25/26/27}{9}{20/22/24}{6}{25}{15} }{{Greece}{Length}{4m}}{ {50 / 60 /70}{Magical skin, strength 25} }{ {Bite: 4 + 1d8 / 12 / 20 (piercing)\\ Sting: Instant death (poison, 5 turn recharge)}{None} }{ {Flight (brief)}{}{100\% Magic Resistance (Below adept)} }{0} \vspace{-2 cm} \\ \beast{{Moke}{Size-changing lizard}}{The Moke is a silver-green lizard reaching up to ten inches in length and is found throughout Britain and Ireland. It has the ability to shrink at will and has consequently never been noticed by Muggles.  Moke skin is highly prized among wizards for use as moneybags and purses, as the scaly material will contract at the approach of a stranger, just as its owner did.}{ {6}{5}{6}{5}{8/10/12}{10}{6}{0} }{{British Isles}{Length}{20cm}}{ {8}{None} }{ {Bite: 1d? (poison, mild)}{None} }{ {Bite causes the afflicted area to swell enormously; Can shrink and grow at will: small (use 1d4), normal (1d6), large (1d12).}{}{} }{0}\vspace{-2 cm}  & ~ & \beast{{Mooncalf}{Shy Mini-Llama}}{he Mooncalf is an intensely shy creature that emerges from its burrow only at the full moon. Its body is smooth and pale grey, it has bulging round eyes on top of its head, and four spindly legs with enormous flat feet. Mooncalves perform complicated dances on their hind legs in isolated areas in the moonlight.These are believed to be a prelude to mating (and often leave intricate geometric patterns behind in wheat fields, to the great puzzlement of Muggles).  Watching Mooncalves dance by moonlight is a fascinating experience and often profitable, for if their silvery dung is collected before the sun rises and spread upon magical herb and flower beds, the plants will grow very fast and become extremely strong.}{ {7}{25}{7}{16}{9}{9}{3}{-10} }{{Worldwide}{Height}{40cm}}{ {7 / 8 / 9}{None} }{ {None}{None} }{ {Nearby animals will come to their aid; May use FIN to do evasion}{}{} }{0} \vspace{-2 cm} \\ \beast{{Murtlap}{Tentacle Rat}}{The Murtlap is a ratlike creature found in coastal areas of Britain. It has a growth upon its back resembling a sea anemone.  When pickled and eaten, these Murtlap growths promote resistance to curses and jinxes, though an overdose may cause unsightly purple ear hair. Murtlaps eat crustaceans and the feet of anyone foolish enough to step on them}{ {8}{2}{5}{3}{2}{3}{2}{0} }{{British Coastline}{Length}{10cm}}{ {11}{None} }{ {Bite: 1d4 + 2 / 3 / 4 (piercing)}{None} }{ {Underwater breathing}{}{30\% Magic resistance} }{0}\vspace{-2 cm}  & ~ & \beast{{Niffler}{Nimble-fingered Platypus}}{Fluffy, black, and long-snouted, this burrowing creature has a predilection for anything glittery. Nifflers are often kept by goblins to burrow deep into the earth for treasure. Though the Niffler is gentle and even affectionate, it can be destructive to belongings and should never be kept in a house. Nifflers live in lairs up to twenty feet below the surface.}{ {17/18/19}{15}{7}{10}{8/10/12}{10}{1}{0} }{{Britain}{Length}{20cm}}{ {15 / 16/ 17}{None} }{ {Scratch: 1d4 (slashing)}{None} }{ {Steal item (major action)}{}{} }{0} \vspace{-2 cm} \\ \beast{{Nogtail}{Devil Pig}}{Demons that resemble stunted piglets with long legs, thick, stubby tails, and narrow black eyes. The Nogtail will creep into a sty and suckle an ordinary sow alongside her own young. The longer the Nogtail is left undetected and the bigger it grows, the longer the blight on the farm into which it has entered. The Nogtail is exceptionally fast and difficult to catch, though if chased beyond the boundaries of a farm by a pure white dog, it will never return.}{ {16}{3 / 4 / 5}{15/17/20}{3}{25}{4}{25}{7} }{{Rural Europe}{Length}{40cm}}{ {15 / 25 / 35}{None} }{ {Headbutt: 1d6 (bludgeoning)}{} }{ {Curse Item (choose random item to bestow negative effect upon)}{}{} }{0}\vspace{-2 cm}  & ~ & \beast{{Nundu}{Poison Leopard}}{Perhaps the most dangerous magical beast in all of existence, the Nundu resembles a gigantic leopard that moves silently despite its size and whose breath causes disease virulent enough to eliminate entire villages. It is said that the Nundu is so powerful that one has never been subdued by less than 100 wizards working together. \\ Despite its immense power and incredible strength, the Nundu is not a vicious or cruel beast. It kills only when angered, but once pushed over the edge, it will go on a rampage.}{ {20/22/24}{15}{30/40/50}{9}{16/18/20}{8/9/10}{25}{3} }{{East Africa}{Length}{3m}}{ {100 / 120 / 150}{None} }{ {Bite: 1d20 + 5 / 10 / 15 (piercing)}{Poison Breath: 3/4/5d20 (range 500m, poison damage: 1d20 per turn)} }{ {Silent movement}{}{} }{0} \vspace{-2 cm} \\ \beast{{Occamy}{Winged Serpent}}{A plumed, winged serpent, the Occamy is a unique metamorphic beast, in that its size changes are unconscious, and it simply grows (or shrinks), to fit the available space. Very territorial, the Occamy is aggressive to all who approach it, particularly in defence of its eggs, whose shells are made of the purest, softest silver.}{ {13}{10 / 12 / 14}{8}{10}{7}{9}{4}{0} }{{Far East}{Length}{Variable}}{ {40}{Scales, strength 10 / 13 / 16, cover entire body} }{ {Peck: 1d8 + 2/ 4/ 5 (piercing)\\ Crush: 1d20 (bludgeoning)}{None} }{ {Changes size to fit the available space (crush only available when larger than 10m);  Obsessively chases insects}{}{} }{0}\vspace{-2 cm}  & ~ & \beast{{Pheonix}{Rebirthing Firebird}}{The phoenix is a magnificent, swan-sized, scarlet bird with a long golden tail, beak, and talons. The phoenix lives to an immense age as it can regenerate, bursting into flames when its body begins to fail and rising again from the ashes as a chick. The phoenix is a gentle creature that has never been known to kill and eats only herbs. Like the Diricawl, it can disappear and reappear at will. Phoenix song is magical; it is reputed to increase the courage of the pure of heart and to strike fear into the hearts of the impure. Phoenix tears have powerful healing properties.}{ {7}{6}{15/17/20}{8/9/10}{12}{20}{15}{0} }{{Mountain peaks of Egypt}{Wingspan}{1.5m}}{ {20}{None} }{ {Peck: 1d8 (piercing)}{None} }{ {Flight (prolonged); {\bf Song of Bravery}: Those with EVL < 3  get 10FP and are immune to the {\it terrified} status; {\bf Tears}: Pheonix tears heal all physical ailments + restore HP to full;  {\bf Regeneration}: If killed, regenerate in a plume of fire, restoring health to max.}{}{} }{0} \vspace{-2 cm} \\ \beast{{Pixie}{Flying Homunculus}}{Electric blue in colour, up to eight inches in height and very mischievous, the pixie delights in tricks and practical jokes of all descriptions. Although wingless, it can fly and has been known to seize unwary humans by the ears and deposit them at the tops of tall trees and buildings. Pixies produce a high-pitched jabbering intelligible only to other pixies.}{ {6}{7}{6}{4}{6}{3}{2}{2} }{{Cornwall, England}{Height}{20cm}}{ {6 / 7 / 8}{None} }{ {Scratch: 1d4 (slashing)}{None} }{ {Flight (prolonged); 20 pixies working in unison can pick up a human}{}{} }{0}\vspace{-2 cm}  & ~ & \beast{{Plimpy}{Walking Fish}}{The Plimpy is a spherical, mottled fish distinguished by its two long legs ending in webbed feet. It inhabits deep lakes where it will prowl the bottom in search of food, preferring water snails. The Plimpy is not particularly dangerous, though it will nibble the feet and clothing of swimmers.}{ {4}{2}{7}{3}{6}{9}{1}{0} }{{Rivers worldwide}{Height}{10cm}}{ {5}{} }{ {Nibble: 1d4 (piercing)}{None} }{ {Underwater breathing}{}{} }{0} \vspace{-2 cm} \\ \beast{{Pogrebin}{Rock Demon}}{The Pogrebin is a Russian demon, barely a foot tall, with a hairy body but a smooth, oversized grey head. When crouching, the Pogrebin resembles a shiny, round rock. Pogrebins are attracted to humans and enjoy tailing them, staying in their shadow and crouching quickly should the shadow�s owner turn around. If a Pogrebin is allowed to tail a human for many hours, a sense of great futility will overcome its prey, who will eventually fall into a state of lethargy and despair. When the victim stops walking and sinks to their knees to weep at the pointlessness of it all, the Pogrebin will leap upon them and attempt to devour them. However, it is easy to repulse the Pogrebin with simple hexes or Stupefying Charms. Kicking has also been found effective.}{ {5}{6}{4}{2}{10}{8/9/10}{9}{4} }{{Russian Steppes}{Height}{30cm}}{ {20}{Rocklike skin, strength 10 / 15 / 20 covers entire body} }{ {Bite: 1d8 (piercing)}{} }{ {{\bf Aura of despair} (constant effect): drains 3FP per turn (range: 30m); Shapeshift into a rock at will}{}{} }{0}\vspace{-2 cm}  & ~ & \beast{{Porlock}{Kind Horse Guardian}}{The Porlock is a horse-guardian. Covered in shaggy fur, it has a large quantity of rough hair on its head and an exceptionally large nose. It walks on two cloven feet. The arms are small and end in four stubby fingers. Fully grown Porlocks are around two feet high and feed on grass. The Porlock is shy and lives to guard horses. It may be found curled in the straw of stables or else sheltering in the midst of the herd it protects. Porlocks mistrust humans and always hide at their approach.}{ {8}{10}{6}{5}{8}{10/14/16}{6}{0} }{{Dorset \& Ireland}{Height}{60cm}}{ {15 / 16/ 17}{None} }{ {Scratch: 1d6 + 1 / 2 / 3 (slashing)}{None} }{ {Control horses telepathically}{}{} }{0} \vspace{-2 cm} \\ \beast{{Puffskein}{Lovable Fluffball}}{Spherical in shape and covered in soft, custard-coloured fur, it is a docile creature that has no objection to being cuddled or thrown about. Easy to care for, it emits a low humming noise when contented. From time to time a very long, thin, pink tongue will emerge from the depths of the Puffskein and snake through the house searching for food. The Puffskein is a scavenger that will eat anything from leftovers to spiders, but it has a particular preference for sticking its tongue up the nose of sleeping wizards and eating their bogies.
Highly prized as pets, miniature puffskeins (known as Pygmy Puffs) have been bred in recent times.}{ {3}{2}{4}{8}{2}{8}{0}{0} }{{Worldwide}{Width}{20cm}}{ {6}{None} }{ {Lick: 1d4 (poison)}{None} }{ {None}{}{} }{0}\vspace{-2 cm}  & ~ & \beast{{Quintaped}{Five-Legged Carnivore}}{The Quintaped is a highly dangerous carnivorous beast with a particular taste for humans. Its low-slung body is covered with thick reddish-brown hair, as are its five legs, each of which ends in a clubfoot.
Entirely resistant to magic, and lightening quick, the Quintaped is rumoured to be the result of an attempted massacre between warring Scottish clans, that resulted in these horrific beasts.}{ {18/20/22}{4}{15/16/17}{2}{7}{5}{24}{8} }{{Scottish Isles}{Width}{80cm}}{ {40 / 50 / 60}{None} }{ {Devour: 1d8/10/12 (necrotic)}{None} }{ {}{}{100\% Magic Resistance} }{0} \vspace{-2 cm} \\ \beast{{Ramora}{Magic Fish}}{A powerfully magical silver fish, the ramora can anchor ships and is a guardian of seafarers. Highly intelligent, and seemingly benevolent, the ramora is a friend to wizards and muggles alike.}{ {7}{8}{8}{4}{14/15/16}{8}{14}{0} }{{Indian Ocean}{Length}{20cm}}{ {10 / 11 / 14}{None} }{ {None}{None} }{ {Can manipulate water at will; Can survive 5 turns out of the water}{}{} }{0}\vspace{-2 cm}  & ~ & \beast{{Re'em}{Goliath Oxen}}{Extremely rare giant oxen with golden hides. Re�em blood gives the drinker immense strength, though the difficulty in procuring it means that supplies are negligible and rarely for sale on the open market.}{ {16}{2}{10/11/12}{6}{4}{10}{9}{0} }{{North America \& Far East}{Length}{6m}}{ {50}{Magical hide, strength 30} }{ {Charge: 1d8/10/12 (requires 3m runup, bludgeoning)}{} }{ {None}{}{} }{0} \vspace{-2 cm} \\ \beast{{Red Cap}{Battleground Dwarf}}{These dwarflike creatures live in holes on old battlegrounds or wherever human blood has been spilled. Although easily repelled by charms and hexes, they are very dangerous to solitary Muggles, whom they will attempt to bludgeon to death on dark nights.}{ {8}{4}{7}{4}{7}{2}{4}{3} }{{Northern Europe}{Height}{60cm}}{ {14 / 17 / 20}{Salvaged armour, strength 7, covers vital organs} }{ {Bludgeon: 1d6 + 1 / 2 / 3 (bludgeoning)}{None} }{ {Ambush attack x4 damage.}{}{} }{0}\vspace{-2 cm}  & ~ & \beast{{Runespoor}{Triple-Headed Giant Snake}}{A three-headed serpent, the Runespoor commonly reaches a length of six or seven feet. Livid orange with black stripes, the Runespoor is very easy to spot. 
Usually a favoured pet of a Dark Wizard, the runespoors in themselves are not particularly vicious. 
Each of the Runespoor�s heads serves a different purpose. The left head (as seen by the wizard facing the Runespoor) is the planner. It decides where the Runespoor is to go and what it is to do next. The middle head is the dreamer (Runespoors may remain stationary for days at a time, lost in glorious visions and imaginings). The right head is the critic and will evaluate the efforts of the left and middle heads with a continual irritable hissing. The right head�s fangs are extremely venomous. The Runespoor rarely reaches a great age, as the heads tend to attack each other.}{ {10/11/12}{7}{9/10/11}{9}{15}{9}{15}{5} }{{Burkina Faso}{Length}{3m}}{ {20 / 25 / 30 (per head)}{Scales, strength 4, cover entire body.} }{ {Bite: 3d6/8/10 (piercing, right head: poison, 1d8 for 3 turns)}{None} }{ {All three heads must be killed in order to kill the beast;  left head killed: FIN reduced to 1;  middle head killed: EMP reduced to 1;  right head killed, INT reduced to 1}{}{} }{0} \vspace{-2 cm} \\ \beast{{Salamander}{Fire Lizard}}{The salamander is a small fire-dwelling lizard that feeds on flame. Brilliant white, it appears blue or scarlet depending upon the heat of the fire in which it makes its appearance. Salamanders can survive up to six hours outside a fire if regularly fed pepper. They will live only as long as the fire from which they sprang burns. 
Salamander blood has powerful curative and restorative properties.}{ {6}{5}{2}{6}{6}{10}{9}{0} }{{Wherever there is fire}{Length}{10cm}}{ {7 / 8 / 10}{None} }{ {None}{Ignite: 1d6 (fire damage: mild)} }{ {Fire-based attacks restore health, rather than remove it}{}{} }{0}\vspace{-2 cm}  & ~ & \beast{{Sea Serpent}{Giant Aquatic Reptile}}{Though alarming in appearance, sea serpents are not known ever to have killed any human, despite hysterical Muggle accounts of their ferocious behaviour. Reaching lengths of up to a hundred feet, the sea serpent has a horselike head and a long snakelike body that rises in humps out of the sea.}{ {10}{4}{10/15/20}{6}{25}{14}{18}{0} }{{Atlantic, Pacific \& Mediterranean}{Length}{> 60m}}{ {100}{Scales, strength 30 cover entire body} }{ {Devour: 1/2/3d20 (concussive)}{None} }{ {None}{}{} }{0} \vspace{-2 cm} \\ \beast{{Snidget}{Golden Spherical Bird}}{The Golden Snidget is an extremely rare, protected species of bird. Completely round, with a very long, thin beak and glistening, jewel-like red eyes, the Golden Snidget is an extremely fast flier that can change direction with uncanny speed and skill, owing to the rotational joints of its win
Prior to the invention of the Golden Stich, the Snidget was the target of many a game of Quidditch.}{ {20/25/30}{20}{3}{2}{4}{3}{2}{0} }{{Arabian Deserts}{Wingspan}{8cm}}{ {6}{None} }{ {Peck: 1d6 (piercing)}{None} }{ {All actions are considered evasions}{}{} }{0}\vspace{-2 cm}  & ~ & \beast{{Streeler}{Poisonous Colour-Changing Snail}}{The Streeler is a giant snail that changes colour on an hourly basis and deposits behind it a trail so venomous that it shrivels and burns all vegetation over which it passes. It is kept as a pet by those who enjoy its kaleidoscopic colour changes, and its venom is one of the few substances known to kill Horklumps.}{ {2}{1}{5}{1}{4}{3}{4}{0} }{{African Rainforests}{Length}{1m}}{ {7}{Shell, strength 5/6/7, covers back} }{ {None}{None} }{ {Poison Aura: All living beings within 10m take 1d4 poison damage per turn}{}{} }{0} \vspace{-2 cm} \\ \beast{{Swooping Evil}{Brain-Eating Lizard Bird}}{The Swooping Evil is a blue-and-green winged magical creature. It looks like a cross between a snake and an extremely large butterfly. When it is not flying with its spiked wings, the Swooping Evil shrinks into a green spiny cocoon.
It can be quite dangerous, as it is an encephalophage � it feeds on people's brains - and its skin has the ability to deflect at least some spells. It secretes venom that, when properly diluted, can be used to erase bad memories.}{ {10}{9 / 10 / 11}{15}{3}{10/12/14}{3}{15}{2} }{{Rainforests}{Wingspan}{2m}}{ {30}{Tough skin, strength 5, covers its body.} }{ {Bite: 1d8 + 2 / 3 / 4 (poison: severe)}{None} }{ {{\bf Coccon}: recover 5HP per turn); {\bf Amnesia}:  When bitten, target forgets a random spell in their arsenal, if no spells available, forget a Skill; Flight (prolonged);}{}{} }{0}\vspace{-2 cm}  & ~ & \beast{{Tebo}{Invisible Warthog}}{The Tebo is an ash-coloured warthog found in Congo and Zaire. It has the power of invisibility, making it difficult to evade or catch, and is very dangerous. Tebo hide is highly prized by wizards for protective shields and clothing.}{ {12}{3}{10}{4}{4/5/6}{5}{7}{0} }{{Congo}{Length}{1m}}{ {15}{Magical skin, strength 10 covers entire body. Skin regenerates at a rate of 1 point per turn.} }{ {Gore: 1d8/10/12 (requires run up, piercing))}{None} }{ {Invisibility (at will)}{}{} }{0} \vspace{-2 cm} \\ \beast{{Thestral}{Death Horse}}{A close relative of the winged horse, the thestral is unique in the fact that is is totally invisible to most of the population: except for those who have seen death. 
To those who can see them, they appear as gaunt, skeletal horses with a slick, hairless skin, and batlike wings. Despite their terrifying appearance, thestrals are kind and gentle creatures}{ {12}{4}{6}{7}{5}{15}{3}{0} }{{Cemetaries Worldwide}{Length}{2m}}{ {20}{None} }{ {Trample: 1d6 (bludgeoning)}{None} }{ {Flight (prolonged); Invisible to those who have not witnessed death}{}{} }{0}\vspace{-2 cm}  & ~ & \beast{{Thunderbird}{Storm-Calling Eagle}}{The thunderbird is a large, avian creature � a close relative of the Phoenix. They are primarily eagle-like, though with three pairs of wings. Normally docile an extremely loyal, the thunderbird is generally considered a friendly animal, but when provoked, few escape its wrath.
Usually golden, they change colour as they exhibit their main ability: calling up storms, and manipulating weather.}{ {16}{7}{14/15/16}{8}{15/16/17}{9}{20/22/24}{0} }{{Arizona}{Wingspan}{3m}}{ {30 / 40 / 50}{None} }{ {Peck: 1d8/12/20 (piercing)}{Storm Call: all enemies within 30m take 1d20 damage (in flight, 5 turn recharge, electric damage)} }{ {Influences weather in 3 mile radius}{}{} }{0} \vspace{-2 cm} \\ \beast{{Troll}{Stupid Giant Humanoid}}{The troll is a fearsome creature up to twelve feet tall and weighing over a tonne. Notable for its equally prodigious strength and stupidity, the troll is often violent and unpredictable. There are three types of troll: mountain, forest, and river. The mountain troll is the largest and most vicious. It is bald, with a pale-grey skin. The forest troll has a pale-green skin and some specimens have hair, which is green or brown, thin, and straggly. The river troll has short horns and may be hairy. It has a purplish skin, and is often found lurking beneath bridges. The 3 types of troll are considered as the three levels of beast.}{ {5}{0}{6}{2}{2}{2}{14}{3} }{{Northen Europe \& Scandinavia}{Height}{4m}}{ {20 / 30 / 40}{Thick skin, strength 10 / 12 / 14, covers entire body} }{ {Bludgeon: 1d8/10/12 (bludgeoning)}{} }{ {If troll has a club, damage is doubled}{}{} }{0}\vspace{-2 cm}  & ~ & \beast{{Unicorn}{Horned Equine}}{A pure white, horned horse when fully grown, though the foals are initially golden, and turn silver before achieving maturity. The unicorn�s horn, blood, and hair all have highly magical properties. It generally avoids human contact, is more likely to allow a witch to approach it than a wizard, and is so fleet of foot that it is very difficult to capture.}{ {9}{10 / 11 / 12}{13/14/15}{14}{9}{20}{7/8/9}{0} }{{European Forests}{Length}{2m}}{ {25 / 26 / 30}{None} }{ {Gore: 1d8/10/12 (requires run up, piercing)}{None} }{ {Unprovoked attacks on unicorn applies the {\it unlucky} status effect for 1 year}{}{} }{0} \vspace{-2 cm} \\ \beast{{Wampus Cat}{Mind-Reading Panther}}{Somewhat resembling the mundane mountain lion or cougar in size and appearance, the Wampus Cat can walk on its hind legs, outrun arrows, and its yellow eyes are reputed to have the power of both hypnosis and Legilimency. The Wampus cat is fast, strong, and almost impossible to kill.}{ {15/16/17}{5}{10}{5}{7}{14}{8}{1} }{{Appalachian Mountains}{Length}{1.5m}}{ {50 / 60 /70}{None} }{ {Bite:  1d8 + 2 / 3 / 4 (piercing)\\ Scratch: 1d6/8/10 (slashing)}{Hypnosis: Attacker is fixed in place for 1 turn and cannot move (psychic)} }{ {Dodge checks get +3 bonus (negated by occlumency)}{}{} }{0}\vspace{-2 cm}  & ~ & \beast{{Winged Horse}{Winged Equine}}{As the name suggests, a winged horse is an otherwise normal horse, with a set of powerful wings on its back. They come in many different varieties across the world � from the powerful and aggressive Arabian Abraxan, to the docile British Aethonan, and the particularly fast Icelandic Granian.  It is not entirely known if the thestral is a subspecies of the Winged Horse, or a separate species altogether.}{ {12}{4}{7}{7}{4}{9}{2}{0} }{{Worldwide}{Length}{2m}}{ {20 / 24 /27}{None} }{ {Trample: 1d8 + 2/3/4 (bludgeoning)}{None} }{ {Flight (prolonged)}{}{} }{0} \vspace{-2 cm} \\ \beast{{Yeti}{Arctic Troll}}{Known as the Abominable Snowman or Bigfoot, depending on where in the world they reside, the yeti is believed to be related to the troll � and may even be a fourth category of troll. Up to o15 feet in height, it is covered from head to foot in pure white hair. The yeti devours anything that strays into its path, though it fears fire.}{ {6}{1}{7}{2}{3}{2}{16}{3} }{{Tibet \& Himalayas}{Height}{5m}}{ {25/35/45}{Thick skin, strength 11 / 13 / 15, covers entire body} }{ {Bludgeon: 1d8/10/12 (bludgeoning)}{None} }{ {If yeti has a club, damage is doubled}{100\% Fire Weakness}{100\% Cold Resistance} }{0}\vspace{-2 cm}  & ~ & \end{longtable}%%BeastEnd
\end{center}
\normalsize


\newpage
\section{Unlife}

The unlife are those entities which were never truly alive. Often called {\it amortal}, because that which is dead may never truly be said to die, many of the most unpleasant and darkest entities in the world are considered Unlife, from the soul-sucking dementors, to the manifestations of the Elder Gods themselves. There are, however, some entities which are not inherently evil - ghosts and angels are often considered friendly and helpful, in fact.

{\it Unlife} also covers inanimate objects that have been brought to life, by magic, enchanted matter given form and thought. Golems, inferi and so on are all considered {\it unlife}. 
\scriptsize
\begin{center}
%%UnlifeBegin
\begin{longtable}{p{\q cm} p{\s cm}  p{\q cm}}\beast{{Abyssal Servant}{Servant of the Elder Gods\\ ~}}{Abyssal Servants take the form of withered, ancient humanoids - however, this is simply a shell of flesh that covers the truly terrifying being beneath the surface. An aspect of the evil Elder Gods, a humanoid manifestation of the primeval chaos of the universe, they exist to serve their masters, which they do by spreading fear, chaos and death wherever they can.}{ {4}{3}{15/18/22}{14}{15}{1}{18/19/20}{25} }{{Dark Plains}{Height}{2.5m}}{ {50 / 54 / 63}{Shadows, strength 50, protect the Abyssal Servant} }{ {None}{Spells: All dark magic spells (wandless)} }{ {Aura of Darkness: All illumination in a 20m radius fails \\ Aura of Terror: All foes take the Terrified status \\ Can phase through matter \\ May subsitute EVL for a check once per turn}{}{} }{1}\vspace{-2 cm}  & ~ & \beast{{Angel}{Winged, Holy Entities}}{The winged soldiers of God himself (or whichever benevolent deity resides the other side of the mysteic veil), angels are the opposite of darkness: brilliant light exudes from them at all times, and they are champions of kindness and empathy. Created at the beginning of time itself to fight for the light in theEternal War, the angels are a force for good in this world.}{ {13}{15 / 25 / 30}{13/18/24}{15}{18}{20/25/30}{23}{0} }{{Heaven}{Height}{2m}}{ {54/68/100}{Armour, strength 15/20/30 covers vital organs} }{ {Sword: 1d10 + 5/6/7}{Holy cry: perform 1d8 check, all creatures with within hearing range with EVL greater than check have AMR set to 0 for 3 turns \\ Spells: All non-dark magic (wandless)} }{ {Can fly \\ Produce illumination in darkness \\ Deal 50\% more damage to Dark creatures}{}{} }{1} \vspace{-2 cm} \\ \beast{{Avatar of Fire}{Fire Spirit}}{It has been said since man first discovered fire that it seems fickle, like a scared animal. Muggles think that this is just their imagination, but it is in fact true. Fire is amongst the most ancient magics, and large enough fires can summon an Avatar of Flame. Ranging from 10cm tall `candles` that can be extinguished by a sneeze, to 10m tall volcanic gods that can devastate continents, the Fire Avatar is flame made flesh. Avatars of flame rejoice in burning living beings to ash, and kill wherever possible.}{ {8}{7}{7/13/25}{7}{5/8/15}{4}{6/14/27}{1/4/10} }{{Volcanoes and Forest Fires}{Height:}{Between 10cm and 10m}}{ {15 / 40 / 100}{None} }{ {Contact burn: 1/2/3d6}{Flamethrower: 1/2/3d20 fire damage} }{ {All attacks causes a mild/moderate/severe burn \\ All objects within 1m combust immediately \\ Fire attacks heal rather than hurt \\ Contact with water does 1d20 damage.}{}{} }{1}\vspace{-2 cm}  & ~ & \beast{{Avatar of Ice}{Frozen Golem}}{The Avatars of Ice are incredibly rare ice spirits,  cold embodied, matter imbued with the most ancient of magics. They emerge from their mother glaciers when the ice is threatened, or when called by an exceptionally powerful wizard. The Avatars of Ice can lower the temperature of a room by a hundred degrees in a matter of seconds, though absorbing that much heat can cause them to melt. Exceptionally strong, and with limitless endurance, these are not beings to anger.}{ {4}{2}{8/15/25}{2}{5}{2}{5/10/15}{1/2/3} }{{Glaciers worldwide}{Height:}{2m}}{ {30 / 40 / 50}{Skin strength 20} }{ {Bludgeon: 1/2/3d8}{Draw heat: freeze all water within 5m range (does 5 damage to avatar)} }{ {All attacks causes mild/moderate/severe Frostbite\\ All objects within 50cm are frozen until thawed \\ Ice attacks heal rather than hurt \\ Fire attacks do 100\% more damage}{}{} }{1} \vspace{-2 cm} \\ \beast{{Avatar of Storms}{Humanoid Thundercloud}}{The Avatars of Storms are the fury and power of a thunderstorm compacted into a physical form. Able to summon ferocious winds and bolts of lightning at will, these Avatars are capable of levelling an entire city should they want to. They can also dissipate their corporeal form and move around as nothing more than vapour, only to reappear  in physical form wherever they desire. Unlike the Avatars of Flame, which kill for the joy of it, the Avatars of Storm kill only when given no other option.}{ {5}{3}{6/7/8}{4}{8/12/17}{7}{10/15/24}{1/2/3} }{{Clouds}{Height}{Varies}}{ {40 / 50 / 65}{None} }{ {Shock touch: 1/2/3d20 (causes parlysis for 3 turns)}{Bolt: 1d12 + 2/3/4 (range 1km)} }{ {Can apparate and attack on same turn (5m range) \\ Can summon wind (range 200m,80mph limit)}{}{} }{1}\vspace{-2 cm}  & ~ & \beast{{Banshee}{Wailing Demon}}{An amortal being spawned from darkness itself, the banshee takes the form of a rotting and decaying woman, said to have a mouth up to three times as large as a normal human (though the research teams at the Ministry have revealed that they can simply unhinge their jaws at will). The cry of a banshee pierces deep into the soul, with effects ranging from insanty, to paralysis, and even instant death.}{ {6}{6}{8}{4}{10}{3}{20}{5} }{{Ireland}{Height}{2.5m}}{ {N/A}{None} }{ {Scratch: 1d8 + 1/2/3}{Wail: Drain FP and cause paralysis \\ Screech: cause instant death to ll in hearing range, if maintained for 3 turns.} }{ {Inability to die.}{Vulnerable to poo}{} }{1} \vspace{-2 cm} \\ \beast{{Boggart}{Non-corporeal Terrorform}}{A shapeshifter that has the ability to sense its target's deepest and most desperate fear, and takes that form. Boggarts are fear manifest, a pure manifestation of terror. They cannot be killed, but they can be temporarily banished by laughter.}{ {5}{20}{16}{11/12/13}{6}{13/14/15}{6}{3} }{{Worldwide}{Height:}{N/A}}{ {N/A}{Depends on the form} }{ {None}{None} }{ {Takes the form of the subject{\apos}s worst fear \\ Attacks depends on the form taken, but are weaker than the assumed form. \\ Applies `terrified' status to one target at a time}{}{} }{1}\vspace{-2 cm}  & ~ & \beast{{Clockwork Solider}{Mechanical Monstrosity}}{The Clockwork Soldier is not, strictly speaking, a magical being - it is entirely mechanical in nature, composed of springs, gears and cogs, the clockwork soldier is of such exquisite complexity that it can act like a functioning (if stupid) humanoid. However, the machinery requires such finnesse to assemble that a Clockwork Soldier can only be assembled by magical means, so no muggle has ever been able to construct one.}{ {15/16/17}{4}{1}{2}{4}{1}{15}{2} }{{Not found in nature}{Height}{2.5m}}{ {15 / 33 / 57}{Skin is metal, strength 8} }{ {Slice: 1d8/12/20}{Dart: 1d8 + 1/2/3 (range 5m)} }{ {Immune to illusion magics, poisons, diseases and curses \\ Does not need to breathe \\ Cannot be disarmed}{}{} }{1} \vspace{-2 cm} \\ \beast{{Crystal Golem}{Living Diamond}}{A core of glass, surrounded by an impenetrable diamond shell, the Crystal Golem shines like a beacon, as it redirects all light that touches its skin at will, which it can use to blind opponents. Unlike the Iron and Stone golems, which are suited to manual work, the crystal golem is created solely for battle, with arms that consist of nothing but razor-sharp diamond blades. A formiddable opponent, the crystal golem is almost impossible to kill, until its skin is penetrated, whereupon it can be shattered like glass.}{ {10}{9}{8}{3}{7}{7}{17}{3} }{{Not found in nature}{Height}{2m}}{ {5 / 10 / 15}{Skin strength 70} }{ {Slice: 3d6/10/20}{Blinding radiance: 1d4 light damage, applies Blinded status} }{ {Immune to bite, scratch and heat damage \\ Follows all orders of its creator \\ Dies if creator dies.}{}{} }{1}\vspace{-2 cm}  & ~ & \beast{{Dementor}{Soul Wraith}}{A Dementor is a gliding, wraith-like Dark creature, widely considered to be one of the foulest to inhabit the world. Dementors feed on human happiness and thus generate feelings of depression and despair in any person in close proximity to them. They can also consume a person's soul, leaving their victims in a permanent vegetative state. Though there appears to exist a humanoid form underneath their cloaks, only the victims of the Dementor�s kiss has ever born witness to it, and they are in no state to tell anyone what they saw.}{ {6}{4}{20/21/22}{0}{8}{19/20/21}{25}{7} }{{Azkaban}{Height}{2m}}{ {N/A}{None} }{ {Dementors kiss: Set FP to 0, HP to 1}{} }{ {Aura of Frozen Despair: all creatures within 30m take 1d6 chill damage, and lose 4FP per turn. Spell casting checks take a 1 point point penalty\\ Can fly. \\ Immune to all magic except the Patronus}{}{100\% Resistance to cold} }{1} \vspace{-2 cm} \\ \beast{{Demonus Temporus}{Time Travel Agents}}{In normal times, the Demonus Temporus live outside this universe, feeding off the time vortex itself. However, if a wizard misuses a time-altering spell to change the past or the future, they create ruptures and fissures in the timestream, which draw the Demonus Temporus into this universe, where they seek to fix the breach, usually by annhilating the wizard who caused it. They appear as a collection of glowing blue orbs, with each individual distinguished by a slightly different shade of blue.}{ {8}{9}{10/11/12}{5}{13/14/15}{10/11/12}{15}{0} }{{Temporal Vortex}{Height}{2m}}{ {20/40/60}{Immune to physical damage} }{ {None}{Temporal Manipulation: Makes the target either 20 years younger, or 20 years older \\ Energy blast: 1/2/3d12} }{ {Can sense the use of time travel, immune to time paradoxes}{}{} }{1}\vspace{-2 cm}  & ~ & \beast{{Ghost}{Departed Spirit}}{The non-corporeal remains of a deceased sapient, remaining in the world of the living. Nobody quite knows what causes someone to remain behind as a ghost, theories include having {\it unfinished business}, or having been cursed in life. Either way, a ghost is mostly powerlessto effect the world of the living.}{ {8}{0}{10}{10}{10}{11}{0}{0} }{{Worldwide}{Height}{1.8m}}{ {N/A}{None} }{ {None}{Terrify: drain 2FP} }{ {Immune to all damage \\Can pass through solid objects}{}{} }{1} \vspace{-2 cm} \\ \beast{{Inferius}{Zombie Flesh Puppet}}{An inferius (plural: inferi) is a hideous puppet of flesh, a dead body reanimated as a zombie by a dark witch or wizard to do their bidding. Nothing of the original being remains beside the physical shell. The inferi feel no pain, and will keep attacking even as limbs are hacked off (and so will the limbs!), the only true way to kill an inferius is to burn it to ashes. Not very dangerous on their own, the inferi are truly terrifying in large numbers.}{ {7}{2}{3/4/5}{2}{5}{1}{8}{8/9/10} }{{Not found in nature}{Height}{1.8m}}{ {10/20/40}{None} }{ {Bite: 1d6/8/10}{None} }{ {Fire does 100\% more damage \\ Immune to slashing and stabbing damage \\ Obey all commands of their creator \\ Feasting on flesh heals 10HP}{}{} }{1}\vspace{-2 cm}  & ~ & \beast{{Inhabitor}{Possessing Spirit}}{Inhabitors are formless, malevolent spirits that are drawn to areas of political turmoil, lies and deceit. They creep inside the bodies of the living, and wrestle control of the body away from the owner. The inhabitor is then able to perfectly imitate their host�s behaviour and actions, which they then use to spread chaos and strife.}{ {3}{10/11/12}{16/17/18}{7}{6}{5}{10}{5} }{{Spirit plane}{Height}{N/A}}{ {N/A}{Various} }{ {}{Inhabit: take control nearby beings (20m range)} }{ {Can take control of any living beings in range that fail a resist magic check, or inhabit inaminate objects, turning them into golems. \\Whilst inhabiting a being, inherits their stats and abilities. \\ Body returns to original owner after leaving the body.}{}{} }{1} \vspace{-2 cm} \\ \beast{{Iron Golem}{Living Metal}}{A being of living, churning metal held together by ancient magics over a core of molten iron. Stronger and more resilient than its stone bretheren, the molten core of the iron golem allows it to generate intense magnetic fields, which it uses to manipulate nearby metallic objects.}{ {7}{2}{12/14/16}{3}{4}{5}{13/19/22}{3} }{{Not found in nature}{Height}{3.5m}}{ {35 / 40 / 45}{Skin strength 40} }{ {Bludgeon: 2d10 + 3}{} }{ {Magnetism: manipulate iron objects (10m range) \\ Immune to bites and scratches. \\ Immune to electricity and lightning \\Follows all orders of its creator. \\Dies if creator dies}{}{} }{1}\vspace{-2 cm}  & ~ & \beast{{Judiciary}{Righteous Universal Entity}}{A single minded manifestation of Justice, the Judiciary are a hive mind, an extraterrestrial consciousness controlling an infinite number of iron-clad bodies, wielding ferocious whips. Their very presence damages the souls of evil beings, and the whip has the ability to banish the UnLife. The Judiciary hunts down those who have committed crimes with a single minded zeal, often tracking their target across entire nations.}{ {10}{10/11/12}{16/18/25}{6}{7}{4}{15/16/17}{0} }{{Multiverse}{Height}{3m}}{ {40 / 50 / 60}{Metal armour, strength 20/30/40} }{ {Holy whip: 1/2/3d12 (range 3m) \\ Bludgeon: 1d8}{Righteous Aura: All evil beings within 10m take 1d8 damage} }{ {Can be disarmed \\ Can apparate \\ All UnLife hit by whip are banished}{}{} }{1} \vspace{-2 cm} \\ \beast{{Lethifold}{Mindless Devourer}}{The Lethifold is a mercifully rare creature found solely in tropical climates. It resembles a black cloak perhaps half an inch thick (thicker if it has recently killed and digested a victim), which glides along the ground at night, absorbing its victims, leaving no trace. The only known spell that repels a lethifold is the Patronus Charm � which brings to light a similarity with another Dark creature: the Dementor. 
Unlike a Dementor, which appears to have some physical form underneath its cloak, a lethifold is simply a cloak. Lethifold are mindless beasts which absorb their prey, whilst Dementors seem to have a level of intelligence, Dementors also only take the soul of their victim, not the entire form. Nonetheless, it is speculated that the two creatures are related in some way.}{ {8}{3}{14/15/16}{0}{4}{0}{19}{5} }{{Tropics}{Width}{1m}}{ {20/30/40}{None} }{ {Digest: 1d8/12/20}{None} }{ {Devour the body of the dead, restoring health to full, and getting a +5 bonus to damage \\ Immune to all magic, except the Patronus charm}{}{} }{1}\vspace{-2 cm}  & ~ & \beast{{Poltergeist}{Spirit of Chaos}}{Said to be an indestructable spirit of chaos, a poltegeist haunts a specific building, rather like a ghost, though a poltergeist was never a living being, seeming to have existed since the dawn of time. Also unlike ghosts, poltergeists can manifest a physical(ish) form, and interact with the physical world, which they chiefly use to commit mischief.}{ {7}{10/12/14}{10}{11/12/13}{7}{5}{6}{1} }{{Large buildings}{Height}{1m}}{ {N/A}{None} }{ {None}{Hurl objects: 1d6} }{ {Specific objects from the environment do more damage \\ Can fly \\ Can become invisible and non-corporeal at will}{}{} }{1} \vspace{-2 cm} \\ \beast{{Shadow Demon}{Inhabitor of Dark Corners, Devourer of the Unwary}}{The Shadow Demons are the cause of the deep-seated fear of the darkness that resides in all sapient life. A being composed of pure darkness, an demonic shadow that can flicker between pockets of darkness at will, devouring those who are foolish enough to step within range. Whilst ensconced in shadow, they cannot be defeated, but they can be destroyed by bringing them into the light.}{ {15}{10/12/13}{8}{10}{13}{2}{15/16/17}{9} }{{Shadows}{Height}{2m}}{ {22 / 27 / 36}{None} }{ {Devour: 1d10 + 3/4/5 \\ Infect: Shadow curse does 5 damage per turn for 20 turns}{Shadow blast: 1d10/12/20} }{ {When in darkness, cannot be killed \\ Can teleport through shadows in 10m range}{}{} }{1}\vspace{-2 cm}  & ~ & \beast{{Stone Golem}{Living Stone}}{Inanimate stone given life through a magic ritual, a stone golem is a powerful ally for any wizard, capable of taking a beating and, more importantly, capable of giving one. Immensly strong, they follow the orders of their creator to the letter.}{ {6}{2}{10/13/16}{3}{2}{4}{10/15/20}{3} }{{Not found in nature}{Height}{3m}}{ {30 / 40 / 50}{Skin strength 35} }{ {Bludgeon: 2 + 1d8/10/12, applies Broken Bone status}{None} }{ {Immune to bite and scratch damage \\ Follows all orders of its creator  \\ Dies if creator dies.}{}{} }{1} \vspace{-2 cm} \\ \end{longtable}%%UnlifeEnd

\end{center}

\newpage
\section{Sapients}


\newcommand{\combattable}[4]{
\begin{tabular}{l l}
 HP: &   #1
 \\
 FP: & #2
 \\
Abilities: &  \parbox[t]{6.9cm}{#3}
\end{tabular}
 
\vspace{1.2ex}
{\bf Skills:}

\vspace{-5ex}

~~\parbox[t]{8.5 cm}{\begin{flushleft}#4\end{flushleft}}
\vspace{-4ex}
}

\newcommand{\sapient}[8]
{
	\definecolor{backcol}{rgb}{0,0,0}
	\definecolor{linecol}{rgb}{0,0,0}
	\def\mode{#8}
	
	\if\mode0
		\definecolor{backcol}{rgb}{0.3,0,0.3}
		\definecolor{linecol}{rgb}{0.4,0,0.4}	
	\fi
	\if\mode1
		\definecolor{backcol}{RGB}{86,36,0}
		\definecolor{linecol}{RGB}{137,57,0}
	\fi
	\if\mode2
		\definecolor{backcol}{RGB}{0,65,84}
		\definecolor{linecol}{RGB}{0,117,150}
	\fi
	\begin{tcolorbox}[  before skip=7pt plus 2pt,
	  boxrule=0pt,
	  boxsep=0pt,
	  toptitle=4pt,
	  left=7pt,
	  right=7pt,
	  bottom=11pt,
	  arc=0.5mm,
	  oversize=0pt,
	  colback=white,
	 colbacktitle=backcol,
	colframe=backcol, title=\vspace{-4ex}
	]
		\begin{center}
	
			{\bf \normalsize #1 #2}
		\end{center}
		
		\vspace{-3.3ex}
		\dndline
		\vspace{1.4 ex}
		
	
		\vspace{0.8ex}
		
		{\bf Combat stats:}
		\\
		\combattable#3
		\\
		\vspace{0ex}
		\\
		\dndline
	
		\begin{center}
			\vspace{-0.26cm}
			\beasttable#6
			\vspace{-0.26cm}
		\end{center}
		\vspace{0.5ex}
	
		\dndline
		#7
		
		\dndline
		\vspace{-0.5 cm}
		\begin{center}
			\begin{tabular}{p{3.8 cm} p{5.1cm}}
				{\bf Items:} & {\bf Spells:}
				\\
				\parbox[t]{3.8cm}{\begin{flushleft} \vspace{-3ex}#4\end{flushleft}} & \parbox[t]{5.1cm}{\begin{flushleft} \vspace{-3ex}#5\end{flushleft}}
				
			\end{tabular}
		\end{center}
		\vspace{-0.9cm}
	\end{tcolorbox}
}

Sapients are those creatures intelligent and sociable enough to either exist within the convnetional wizarding society, or to form a society of their own. A hugely diverse group, by their very nature, sapients take different roles within their society. Therefore, the threat posed by a Sapient is not only dictated by its species, but by the role that it has within a society: a soldier is more dangerous than a librarian!

The entries in this section reflect this inherent polymorphism: species have multiple entires, one for each major societal role. 

The entries are colour coded: Blue entires are those which primarily use magic to attack, brown represents sapients which rely on physical force, and purple is non-combat entities. 

(Sapients removed for maintenance)