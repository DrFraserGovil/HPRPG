\chapter{Bestiary}


In this section a number of different creatures are presented for the GM's use in building encounters. These creatures come with a set of basic canonical background information, as well as a `statblock', which contains the necessary statistics for these creatures to perform checks, and ultimately engage in combat and other character interactions.


\section{Beast Abilities}	

Whilst all \imp{Beasts} share the same 9 base \imp{Aspects} as player characters, and many of the same \imp{Abilities}. However for streamlining reasons, the number of \imp{Abilities} each individual beast has is more restricted than a player character: if an \imp{Ability} is not mentioned in the provided statblock, you may assume it has a value of \emptyCape. 

Though they have far fewer proficiencies, Beasts do have access to all of the same \imp{Abilities} as the player characters - though actions such as \imp{Imbue} and \imp{Craft} are unlikely to come up except in the most unusual of circumstances! 

In addition to the 30 base \imp{Abilities}, some beasts have additional abilities determined by their non-human and, in some cases, magical, physiology:
\newcommand\abilityRow[2]
{

	\parbox[t]{2 cm}{\key{#1}} & \parbox[t]{6.8 cm}{\raggedright #2} \\
}

\newcommand\abilityTable[1]
{
	\small
	\begin{center}
		\begin{rndtable}{r l}
		\bf Ability	& \bf Description \\
		#1
		\end{rndtable}
	\end{center}
}

\abilityTable
{
	\abilityRow{Climb}{Many beings have the ability to climb trees, and adhere to solid surfaces. A non-zero rating grants a being an inherent climb speed - the higher the rating, the faster they can climb.}
	\abilityRow{Command}{Some creatures command their lessers and may order them to do their bidding - a higher rating indicates the level of control they have over their forces.}
	\abilityRow{Elusion}{Elusion is the natural camouflaging ability of a being - morphing into the background, changing colour and even turning invisible.}
	\abilityRow{Flight}{A creature with the flying ability may defy gravity, either with wings, or innate magical levitation. A higher rating means faster flight and more elaborate maneouvers.}
	\abilityRow{Regeneration}{This ability allows a creature to heal themselves rapidly as their physical form regenerates.}
	\abilityRow{Inhuman Senses}{Many creatures have senses beyond those that humans have: the ability to sense tremors in the ground, see in the dark, as well as more arcane abilities such as the ability to detect magic.}
	\abilityRow{Shapechange}{A creature with this ability may alter their shape and form - a higher rating means more drastic changes to their appearance.}
	\abilityRow{Spellcasting}{A replacement for individual \imp{Affinities}. A creature with this ability can innately cast magic using this statistic.} 
	\abilityRow{Swimming}{Aquatic creatures have a natural affinity for moving within the water - a high \imp{Swimming} shows an ability to move quickly and navigate in 3D. }
	\abilityRow{Tunnelling}{Whilst we are most familiar with creatures which walk on land, or soar above it, some rare creatures make a living beneath it. A high \imp{Tunnel} ability allows a being to move smoothly through seemingly solid earth and rock.}
}


\section{Movement}

Some of these abilities - notably \imp{Climb}, \imp{Flight}, \imp{Swimming} and \imp{Tunnelling} - grant creatures additional means of traversing around an environment, beyond the usual walking and running that humans are used to. 

It can generally be assumed that a zero-rating in this field menas that a given mode of transport is not possible. This should, of course, be taken with a hint of salt - few creatures with a zero \imp{speed} rating are physically unable to walk, and equally, a mighty \imp{Archangel} is not going to hesitate to dive into a pool, despite not having a \imp{Swimming} rating. However, a \imp{Nogtail} is not suddenly going to be able to fly, no matter how slowly. 

If a beast is using an alternative means of transportation, their rating for that means supercedes the normal rules about movement - if a \imp{Hippogriff} is currently in flight, all checks which might normally rely on \imp{speed} are instead made using \imp{flight}.

The calculations used to determine a creature's movement speed by a given vector is as follows:

\newcommand\moveRow[2]
{
	\imp{#1}	&	#2	\\
}
\begin{center}
	\begin{rndtable}{l c}
		\key{Type}	&	\key{Speed} \\
		\moveRow{Walking}{3 + \imp{Speed} Rating}
		\moveRow{Flying}{4 $\times$ \imp{Flight} Rating}
		\moveRow{Climbing}{0.5 $\times$ \imp{Climb} Rating}
		\moveRow{Swimming}{0.5 $\times$ \imp{Swim} Rating}
		\moveRow{Tunneling}{0.25 $\times$ \imp{Tunneling} Rating}
	\end{rndtable}
\end{center}

A creature may use up to two different types of movement in a given turn, but the maximum movement distance for each mode of transport is capped at half the usual value. For instance, if a beast with a 6m walking speed and a 10m flight speed were to both walk and fly in a single turn, it could walk no more than 3m and fly no more than 5m. Using only 1m of flight does not impact the amount of walking that could be performed.

Some creatures do not follow this rule - most of those are those with a \imp{Speed} rating of 0, but who have a walking speed less than 3m. These are merely particularly slow creatures. Some others, however, have a walking speed greater than their \imp{speed} might suggest. This is usually the case with particularly large creatures such as \imp{Giants}, whose long strides grant them rapid movement, but not the nimbleness and quick reflexes that a high rating on a \imp{speed} check would imply. 
\clearpage

\newcommand\beastSpell[2]
{
	\bf #1	&	\parbox[t]{7.3  cm}{{\it #2}} \\
}
\newcommand\beastSpellList[1]
{
	
	\begin{tabular}{l l}
	#1
	\end{tabular}
}
\definecolor{statblockbg}{HTML}{FDF1DC} 

\makeatletter
\define@key{beast}{name}{\def\name{#1}}
\define@key{beast}{species}{\def\species{#1}}
\define@key{beast}{mind}{\def\mind{#1}}
\define@key{beast}{category}{\def\category{#1}}
\define@key{beast}{description}{\def\description{#1}}
\define@key{beast}{rating}{\def\rating{#1}}
\define@key{beast}{speed}{\def\speed{#1}}

\define@key{beast}{fit}{\def\fit{#1}}
\define@key{beast}{prs}{\def\prs{#1}}
\define@key{beast}{vit}{\def\vit{#1}}

\define@key{beast}{cha}{\def\cha{#1}}
\define@key{beast}{dec}{\def\dec{#1}}
\define@key{beast}{ins}{\def\ins{#1}}

\define@key{beast}{int}{\def\int{#1}}
\define@key{beast}{wil}{\def\wil{#1}}
\define@key{beast}{pcp}{\def\pcp{#1}}


\define@key{beast}{nUnharmed}{\def\nUnharmed{#1}}
\define@key{beast}{nBruised}{\def\nBruised{#1}}
\define@key{beast}{nHurt}{\def\nHurt{#1}}
\define@key{beast}{nInjured}{\def\nInjured{#1}}
\define@key{beast}{nWounded}{\def\nWounded{#1}}
\define@key{beast}{nMangled}{\def\nMangled{#1}}
\define@key{beast}{fortitude}{\def\fortitude{#1}}
\define@key{beast}{article}{\def\article{#1}}


\define@key{beast}{skills}{\def\skills{#1}}
\define@key{beast}{hasSkills}{\def\hasSkills{#1}}

\define@key{beast}{block}{\def\block{#1}}
\define@key{beast}{dodge}{\def\dodge{#1}}
\define@key{beast}{defy}{\def\defy{#1}}

\define@key{beast}{hasDamage}{\def\hasDamage{#1}}
\define@key{beast}{damage}{\def\damage{#1}}

\define@key{beast}{abilities}{\def\abilities{#1}}
\define@key{beast}{attacks}{\def\attacks{#1}}
\define@key{beast}{hasAttacks}{\def\hasAttacks{#1}}

\define@key{beast}{hasLanguages}{\def\hasLanguages{#1}}
\define@key{beast}{languages}{\def\languages{#1}}

\define@key{beast}{hasImage}{\def\hasImage{#1}}
\define@key{beast}{image}{\def\image{#1}}
\define@key{beast}{imageStack}{\def\imageStack{#1}}
%~ \define@key{beast}{hasImmune}{\def\immuneMode{#1}}
%~ \define@key{beast}{immune}{\def\immune{#1}}
%~ \define@key{beast}{hasResistance}{\def\resistanceMode{#1}}
%~ \define@key{beast}{resistance}{\def\resistance{#1}}
%~ \define@key{beast}{hasSusceptible}{\def\susceptibleMode{#1}}
%~ \define@key{beast}{susceptible}{\def\susceptible{#1}}
%~ \define@key{beast}{hasAbilities}{\def\abilitiesMode{#1}}
%~ \define@key{beast}{abilities}{\def\abilities{#1}}
%~ \define@key{beast}{hasSkills}{\def\skillsMode{#1}}
%~ \define@key{beast}{skills}{\def\skills{#1}}
%~ \define@key{beast}{hasActions}{\def\actionMode{#1}}
%~ \define@key{beast}{actions}{\def\actions{#1}}
%~ \define@key{beast}{habitat}{\def\habitat{#1}}
%~ \define@key{beast}{sizeName}{\def\sizeName{#1}}
%~ \define@key{beast}{size}{\def\size{#1}}
%~ \define@key{beast}{abilityBlock}{\def\aBlock{#1}}
%~ 
%~ \define@key{beast}{image}{\def\image{#1}}
%~ \define@key{beast}{hasImage}{\def\imageMode{#1}}
%~ \define@key{beast}{imPosition}{\def\imagePos{#1}}
%~ \define@key{beast}{hasLanguages}{\def\languageMode{#1}}
%~ \define@key{beast}{language}{\def\languages{#1}}
%~ \define@key{beast}{needsLine}{\def\needsLine{#1}}
%~ \define@key{beast}{comprehend}{\def\comprehend{#1}}
%~ \define@key{beast}{hasComprehend}{\def\comprehendMode{#1}}
%~ \define@key{beast}{imageHeight}{\def\imageHeight{#1}}
%~ \define@key{beast}{habitat}{\def\habitat{#1}}
%~ \define@key{beast}{needsPage}{\def\needsPage{#1}}
%~ 
%~ \define@key{beast}{hasHabitat}{\def\hasHabitat{#1}}
%~ \define@key{beast}{hasLair}{\def\hasLair{#1}}
%~ \define@key{beast}{lairActions}{\def\lairActions{#1}}
%~ \define@key{beast}{hasSenses}{\def\hasSenses{#1}}
%~ \define@key{beast}{senses}{\def\senses{#1}}
%~ \define@key{beast}{hasConditionImmune}{\def\hasConditionImmune{#1}}
%~ \define@key{beast}{conditionImmune}{\def\conditionImmune{#1}}
%%keyEnd    
\makeatother

\def\defaultSetter
{
	\setkeys{beast}{name = none, species = None, mind = none, rating = Unset, category = none, description = None, fit =0, prs = 0, vit =0, cha = 0, dec = 0, ins = 0, int = 0, wil =0, pcp = 0, nUnharmed =0, nBruised = 0, nHurt = 0, nInjured = 0, nWounded = 0, nMangled = 0,fortitude = 0, skills = 0, hasSkills = 0, block = 0, dodge = 0, defy = 0,hasDamage = 0, damage = , abilities = ,hasAttacks = 0, attacks = None,article = A, hasLanguages = 0, languages = None, hasImage = 0, image = None}
}
\newcounter{yCount}
\newcounter{nonZeroCount}
\def\bx{0.45}
\newcommand{\healthBox}
{%q

	\vbox{
	\key{Health}
	
    \begin{tikzpicture}
       
       \def\nTotal{ \fpeval{\nUnharmed + \nBruised + \nHurt + \nInjured + \nWounded + \nMangled + 1} }
       
       
       %\draw (0,0) rectangle ({\nTotal*\bx},{2*\bx});
       %\draw (0,\bx)--({\nTotal*\bx},{\bx});
       \foreach \c in {1,2,...,{\nTotal}}
       {
			%\draw ({\c*\bx},\bx)--({\c*\bx},{2*\bx});
			
			
			%\draw ({(\c)*\bx}, {1.5*\bx})--({(\c-0.5)*\bx}, {2*\bx})--({(\c-1)*\bx}, {1.5*\bx})--({(\c-0.5)*\bx}, {\bx})--cycle;
       }
       
       \setcounter{yCount}{0}
       \setcounter{nonZeroCount}{0}
       \foreach [count = \j from 0] \a/\b in { {Fine}/\nUnharmed, {Bruised \\(-1)}/\nBruised, {Hurt\\ (-2)}/\nHurt, {Injured\\(-3)}/\nInjured, {Harmed \\(-4)}/\nWounded, {Mangled (-5)}/\nMangled, {Critical}/1}
       {
			\def\g{1.7}
			\def\xVal{ {(\theyCount + (\b)/2)*\bx} + \thenonZeroCount*\bx/\g}
			\def\lx { {\theyCount *\bx} + \thenonZeroCount*\bx/\g}
			\if \b0
				%%
			\else
				\node[anchor = center] at (\xVal,{0}) {\parbox[t]{1cm}{\tiny\centering \a}};
				
				\foreach \j in {1,...,\b}
				{
					\def\cx{\lx + (\j-1)*\bx}
					\def\sqrtT{\bx/1.414213562373095}
					\draw[rotate around={45:({\cx+\bx/2},\bx)}] ({\cx+\bx/2-\sqrtT/2},{\bx-\sqrtT/2})rectangle ({\cx+\bx/2+\sqrtT/2},{\bx+\sqrtT/2});
					%\draw ({(\c)*\bx}, {1.5*\bx})--({(\c-0.5)*\bx}, {2*\bx})--({(\c-1)*\bx}, {1.5*\bx})--({(\c-0.5)*\bx}, {\bx})--cycle;
				}
				
				\addtocounter{nonZeroCount}{1}
				%\draw ({\xVal - \b/2*\bx},0) rectangle ({\xVal + \b/2*\bx},{2*\bx});
			\fi
			\addtocounter{yCount}{\b}
       }
       
    \end{tikzpicture}%
    }
}

\newcommand{\fortitudeBox}
{%q
	\if\fortitude0
		%who knows?
	\else
		\vbox{
		\parbox[t]{3.5 cm}{
		\key{Fortitude:}
		\\~\\
	    \begin{tikzpicture}
	      
	       \foreach \c in {1,...,{\fortitude}}
	       {
				\def\lx{ {(\c-1) * \bx}}
				
				\def\sqrtT{\bx/1.414213562373095}
				\draw[rotate around={45:({\lx},\bx)}] ({\lx-\sqrtT/2},{\bx-\sqrtT/2})rectangle ({\lx+\sqrtT/2},{\bx+\sqrtT/2});
	       }
	       
	    \end{tikzpicture}%
	    }
	    
	    }
	 \fi
}




\newcommand\understands[1]
{
	\imp{Understands: } #1
}
\newcommand\speaks[1]
{
	\imp{Speaks: } #1
}
\newcommand\skill[2]
{
	~& \imp{#1}: & \ratingB{#2} \\
		
}
\newcommand\titleBlock
{
	\definecolor{rulered}{HTML}{9C2B1B} 
		{ \huge \color{titlered}{\MakeUppercase{\name{}} }}
		
		MoM Rating: \rating{} {\it (\mind{}~\category{}) }
}

\newcommand\skillBlock
{
	\key{Abilities:}
	
	\begin{tabular}{@{} p{2cm} r c @{} }
		 \skills{}
		 
		
	\end{tabular}
}
\def\w{2.1}
\newcommand\stat[2]
{
	\parbox[t]{\w cm}{\vspace{0.3cm} \centering \key{#1} \\ \large \ratingB{#2}}
}
\newcommand\statBlock
{
	\normalsize
	
	\healthBox{}
		
	\begin{tabular}{m{4cm} m{4cm}}
	\fortitudeBox{}	&	\begin{tabular}{||l l||}
								\hline\hline
								\key{Block}	& \ratingB{\block}
								\\
								\key{Dodge}	&	\ratingB{\dodge}
								\\
								\key{Endure}	&	\ratingB{\defy}
								\\\hline\hline
						\end{tabular}
	\end{tabular}
	
	\damage{}
	
	{\centering
	\footnotesize
	\begin{tabular}{@{}  m{\w cm}  m{\w cm}  m{\w cm} @{}}
				 %\hline
				\stat{Fitness}{\fit}	&	\stat{Charm}{\cha}	&	\stat{Intelligence}{\int}
				\\ %\hline
				\stat{Precision}{\prs}	&	\stat{Deception}{\dec}	&	\stat{Willpower}{\wil}
				\\ % \hline
				\stat{Vitality}{\vit}	&	\stat{Insight}{\ins}	&	\stat{Perception}{\pcp}
				\\ %\hline
		\end{tabular}
	
	\normalsize
	}

	
}

\newcommand\ability[2]
{
\textbf{\textit{#1}}: #2
}
\newcommand\melee[5]
{
\parbox[t]{9.5cm}{\textbf{\textit{#1}}: ({\it melee attack, #2 dice, DV \fpeval{7+#3}}) \\ Effect: \imp{#4}, with Power #5}
}
\newcommand\ranged[6]
{
\parbox[t]{10cm}{\textbf{\textit{#1}}: ({\it ranged attack: #2m, #3 dice, DV \fpeval{7+#4}}) \\ Effect: \imp{#5}, with Power #6}
}

\newcommand\area[6]
{
	\parbox[t]{9.5cm}{\textbf{\textit{#1}}: ({\it area attack: #2, #3 dice, DV \fpeval{7+#4}}) \\ Effect: \imp{#5}, with Power #6}	
}


\newcommand\sideBySide
{
	\begin{minipage}{0.2 \textwidth}
	\includegraphics[width = 0.98 \textwidth, keepaspectratio=true]{../Images/\image}
	\end{minipage}
	\begin{minipage}{0.29\textwidth}
	\description{}
	\end{minipage}

}
\newcommand\verticalStack
{
	\begin{center}
	
	\includegraphics[width = 0.4 \textwidth, keepaspectratio=true]{../Images/\image}
	\end{center}
	
	\description{}

}


\newcommand\beast[1]
{
	\begingroup
	
	\defaultSetter
	\setkeys{beast}{#1}
	
	\vbox
	{
	\dndline
		\titleBlock{}
	}	
		\if\hasImage1
			\if\imageStack1
				\verticalStack
			\else
				\sideBySide
			\fi
		\fi
	

	
	\statBlock

	\if\hasSkills1
		\vspace{0.7cm}
		\skillBlock{}
	\fi

	\dndlineFade{black}
	
	\abilities{}
	
	\if\hasLanguages1
		\ability{Languages}{\languages}
	\fi
		
	\if\hasAttacks1
		\subsubsection*{Armaments \& Attacks}
		\attacks{}
	\fi
	
	
	~
	
	\endgroup
}

\newcommand\species[3]
{
	%\clearpage

		{ \Huge{\MakeUppercase{#1} }}
		
	\addcontentsline{toc}{section}{#1}
	~
	
	#2

	~
	
	#3
	
	
}


\species{Acromantula}
{
	The acromantula are an incredibly rare \minus{} and incredibly dangerous \minus{} species of gigantic\comma{} intelligent spiders. Found mainly in dense forests\comma{} where they weave their web\minus{}covered nests\comma{} they only occaisionally go out to hunt\comma{} preferring instead to let their prey come to them. 

Hatching from eggs the size of rugby balls\comma{} the oldest specimens have legspans in excess of 10 metres. Their equally enormous fangs contain a potent venom. The speed\comma{} strength and venom\comma{} however\comma{} is not what makes the Acromantula a truly awful foe. Rather\comma{} their greatest weapon is their formiddable intellect\comma{} which allows them to outthink even the greatest wizards. 

\ability{Elaborate Lairs}{A spider\apos{}s central tenet is patience: waiting for prey to come ot you. Acromantula are no different\comma{} though they work on a slightly different scale. Over their multi\minus{}decade\minus{}long lifespan\comma{} a Patriarch will build an enormous\comma{} complex labyrinth of webs and forest\comma{} in order to ensnare their unsuspecting prey}

\ability{Talking Spiders:}{Acromantula have the ability to speak the spider tongue\comma{} to command their legions of arachnid followers. As they age and their minds continue to develop\comma{} they even gain the ability to understand and eventually speak in human tongues.}

\ability{Keen Sight}{In addition to their web\minus{}enhanced senses\comma{} the 8 compound eyes of the acromantula allow them to see in incredible detail\comma{} even in dim light}

\ability{Webspinners}{As members of the spider family\comma{} all Acromantula have an affinity for spinning webs\comma{} and using them to sense and then ensnare their prey.}
}
{
\beast{name = Acromantula Hatchling, species = Acromantula, mind = Non\minus{}Sapient, category = Monstrosity, rating = III, abilities = \ability{Webwalker}{\article{} \name{} takes no movement penalty on webbed surfaces\comma{} and uses their \imp{Inhuman Senses} to sense vibrations in their webs.}

\ability{Sticky Feet}{A \name{} may use their \imp{Climbing} ability to walk on any vertical surface.}

\ability{Tiny}{\name{}s can occupy the same space as another being\comma{} climbing over them. Their small size also grants them a non\minus{}damaging terminal velocity.}, article = A, movement = \speeds{\speedrating{Walking}{5}\speedrating{Climbing}{1.5}}, fit =2, prs =3, vit =2, cha =0, dec =0, ins =1, int =1, wil =1, pcp =2, hasDamage = 1, damage =\key{Immune} to \textit{Poison\comma{} Falling Damage}, nUnharmed=1, nBruised=0, nHurt=2, nInjured=0, nWounded=0, nMangled=1, block=1, dodge=3, defy=1, fortitude=1, imageStack=0, hasSkills = 1, skills = \skill{Covert}{4}
\skill{Climb}{3}
\skill{Speed}{2}
\skill{Inhuman Senses}{2}
, hasAttacks = 1, attacks = \meleeConsequence{Poison Fangs}{3}{0}{Stabbing Damage}{1+Successes}{If the attack deals any damage\comma{} the victim takes the \imp{Poisoned} status (1 \imp{harm}\comma{} requires 5 successes)}, hasLanguages = 1, languages = \understands{Spider Tongue}, hasImage = 1, image = babyAcro, description = A newborn \imp{acromantula} is tiny when compared to their full grown counterparts – though with a legspan of up to 40cm\comma{} they are still significantly larger than almost all non\minus{}magical spiders. 

Their body is covered in a shiny\comma{} hairless and pale\minus{}grey carapace\comma{} which hardens and grows darker as they grow older – eventually they shed this skin as they enter the adult phase of their life. 

Despite their limited intelligence and diluted poison\comma{} \name{}s are often encountered in nauseating flocks of thousands upon thousands\comma{} and in such large numbers\comma{} they pose a deadly threat to even the most powerful magic user.}
\beast{name = Acromantula Adult, species = Acromantula, mind = Sapient, category = Monstrosity, rating = V, abilities = \ability{Webwalker}{\article{} \name{} takes no movement penalty on webbed surfaces\comma{} and uses their \imp{Inhuman Senses} to sense vibrations in their webs.}


\ability{Sticky Feet}{A \name{} may use their \imp{Climbing} ability to walk on any vertical surface.}, article = A, movement = \speeds{\speedrating{Walking}{8}\speedrating{Climbing}{3}}, fit =5, prs =4, vit =4, cha =0, dec =1, ins =1, int =4, wil =3, pcp =3, hasDamage = 1, damage =\key{Immune} to \textit{Poison} and \key{Resistant} to \textit{Falling Damage}, nUnharmed=3, nBruised=2, nHurt=1, nInjured=1, nWounded=1, nMangled=2, block=3, dodge=4, defy=3, fortitude=3, imageStack=0, hasSkills = 1, skills = \skill{Climb}{6}
\skill{Speed}{5}
\skill{Inhuman Senses}{5}
\skill{Covert}{4}
\skill{Strength}{3}
, hasAttacks = 1, attacks = \meleeConsequence{Poison Fangs}{6}{0}{Stabbing Damage}{1+Successes}{If the attack deals any damage\comma{} the victim takes the \imp{Poisoned} status (2 \imp{harm}\comma{} requires 8 successes)}

\ranged{Websac}{10}{6}{0}{Trapped Status}{1 + Successes}, hasLanguages = 1, languages = \understands{Human Languages}
\speaks{Spider Tongue}, hasImage = 1, image = Acromantula, description = From a nest of several thousand \imp{Hatchlings}\comma{} only one or two survive the brutal and vicious ascent to adulthood within an \imp{Acromantula} colony\comma{} shedding their final adolescant carapace to become a full\minus{}grown \imp{Acromantula}.

As a result of this violent and competitive environment\comma{} a fully grown \imp{Acromantula} is something to be greatly feared. No \imp{Acromantula} survives this long without a willingness and ability to brutally slay even their closest allies\comma{} so that only the most murderous\comma{} brutal and cunning spiders remain. 

Though  they can run incredibly quickly and they utilise a ranged web attack to ensnare their pray\comma{} the most terrifying aspect of a full\minus{}grown \imp{Acromantula} is their above\minus{}human level of intelligence\comma{} not only can they liquiefy your innards\comma{} they can counter even the most elaborate plan to outwit them.}
\beast{name = Acromantula Patriarch, species = Acromantula, mind = Sapient, category = Monstrosity, rating = VI, abilities = \ability{Webwalker}{\article{} \name{} takes no movement penalty on webbed surfaces\comma{} and uses their \imp{Inhuman Senses} to sense vibrations in their webs.}


\ability{Sticky Feet}{A \name{} may use their \imp{Climbing} ability to walk on any vertical surface.}, article = A, movement = \speeds{\speedrating{Walking}{4}\speedrating{Climbing}{0.5}}, fit =1, prs =3, vit =3, cha =2, dec =5, ins =5, int =6, wil =6, pcp =4, hasDamage = 1, damage =\key{Immune} to \textit{Poison}, nUnharmed=4, nBruised=1, nHurt=1, nInjured=1, nWounded=1, nMangled=2, block=2, dodge=2, defy=6, fortitude=5, imageStack=1, hasSkills = 1, skills = \skill{Inhuman Senses}{7}
\skill{Command}{7}
\skill{Strength}{4}
\skill{Climb}{1}
\skill{Speed}{1}
, hasAttacks = 1, attacks = \meleeConsequence{Poison Fangs}{3}{0}{Stabbing Damage}{4+Successes}{If the attack deals any damage\comma{} the victim takes the \imp{Poisoned} status (5 \imp{harm}\comma{} requires 10 successes)}

\ranged{Websac}{30}{10}{0}{Trapped}{1 + Successes}

\ability{Summon Legions}{Whilst within their lair\comma{} a \name{} may use a DV 7 \imp{Command} action to summon a number of \imp{Acromantula Adults} and \imp{Acromantula Hatchlings} to serve them.}, hasLanguages = 1, languages = \speaks{Human Languages\comma{} Spider Tongue}, hasImage = 1, image = aragog, description = If it is rare for a \imp{Hatchling} to survive to adulthood\comma{} it is even rarer for an \imp{Acromantula} to grow old\comma{} and gain the mantle of the \imp{Patriarch}. 

As the \imp{Acromantula} never stop growing\comma{} by the time they reach 40 or 50 years old\comma{} they have reached truly gargantuan sizes\comma{} with legspans up to 10m\comma{} with an exoskeleton that is so thick that almost nothing can penetrate it. 

Though they cut a truly terrifying figure\comma{} their bodies have become decrepit with age\comma{} and they do not retain the nimbleness of their younger forms\comma{} instead relying on their formidable intellect and their ability to command legions of their brood to protect them.}

}




\species{Angels}
{
	Angels are powerful\comma{} beautiful Celestial creatures\comma{} denizens of Elysium\comma{} one of the Higher Planes\comma{} though they can be found throughout the multiverse. Often perceived as powerful agents of Deities\comma{} servants of benevolent gods\comma{} it is actually unknown who or what provides these powerful creatures with their deeper purpose. 

\ability{Benevolent Fury}{Almost universally pure of heart and intrinsically ethical and good\comma{} Angels are representative of everything full of light and life in the universe. Angels will never compromise their core beliefs. They are not\comma{} however\comma{} pacifists. Angels are great and pwoerful warriors\comma{} and will strike down their enemies in the name of protecting those who cannot protect themselves.}

\ability{Angelic Host}{The Angelic society is known as the {\it Angelic Host}\comma{} a powerful seemingly omniscient society which dwells almost entirely in the Silver City found at the centre of Elysium. This society is highly structured and hierarchical\comma{} with angels being created to fill specific niches within each echelon of society. Each Angel derives their powers from their position within the angelic hierarchy\comma{} with the highest tiers wielding terrifying amounts of power.}

\ability{Holy Crusades}{Angels only leave the Silver City on two conditions\comma{} the most common of which is being directed on a holy quest by one of their superiors. Most Angels met outside of Elysium are conducting such a quest. The difficulty of the quest depends on the ranking of the angel in question: a cherubim might be sent out to conduct a blessing\comma{} or deliver a message\comma{} whilst a quest which calls for an Archangel to be sent would be a truly dire universe\minus{}ending scenario. }

\ability{Fallen Angel}{The other condition under which an Angel is refused entry into the Silver City is if they have {\it fallen}. Though Angels will never compromise their core beliefs\comma{} and are almost inherently good in nature it is possible for them to fall victim to their own pride and hubris. If this happens\comma{} an angel may act against the wishes of the Host\comma{} or inadvertently perform some great act of evil. 

If this happens\comma{} the Host will disavow them\comma{} and cast them out. Without the purpose granted to them by the rigid structure of Angelic society\comma{} many such fallen angels go entirely mad. Others sink into a deep\comma{} vengeful fury and declare war on the Host\comma{} whilst others are believed to undergo a transformation\comma{} becoming powerful demonic creatures. }

\ability{Immortal Spirit}{As a celestial being\comma{} an angel is incredibly resilient and requires neither food\comma{} drink\comma{} air or sleep (though they may enjoy the experience).}
}
{
\beast{name = Cherubim, species = Angels, mind = Ineffable, category = Celestial, rating = IV, abilities = \ability{Light in the Darkness}{If the target of an attack has more than one rating in \imp{Villainy}\comma{} or has used a \imp{Dark Arts} spell in the past 24 hours\comma{} the \name{} gets +1d for all attacks against them.}

\ability{Walk Among Mortals}{A \name{} may use their \imp{Shapechange} ability to take on a human form\comma{} hiding their wings. However\comma{} they remain almost supernaturally beautiful appearances.}

\ability{Master of Mind and Body}{A \name{} is immune to spells which would alter its mind or perception of reality\comma{} and it cannot have its form altered by magic unless it wishes to.}, article = A, movement = \speeds{\speedrating{Walking}{8}\speedrating{Flying}{16}}, fit =3, prs =4, vit =4, cha =4, dec =1, ins =4, int =5, wil =5, pcp =3, hasDamage = 1, damage =\key{Immune} to \textit{Incandescent}\comma{} \key{Resistant} to \textit{All physical damage from non\minus{}magical sources} and \key{Susceptible} to \textit{Necrotic}, nUnharmed=3, nBruised=2, nHurt=1, nInjured=0, nWounded=1, nMangled=1, block=1, dodge=4, defy=3, fortitude=5, imageStack=0, hasSkills = 1, skills = \skill{Speed}{5}
\skill{Conviction}{5}
\skill{Strength}{5}
\skill{Flight}{4}
\skill{Kindness}{4}
\skill{Spellcasting}{4}
\skill{Marksman}{4}
\skill{Covert}{3}
\skill{Skirmish}{3}
\skill{Shapechange}{2}
, hasAttacks = 1, attacks = \melee{Heavenly Sword}{6}{\minus{}1}{Slashing Damage}{3 + Successes}

\ranged{Bow \& Arrow}{50}{8}{\minus{}1}{Stabbing Damage}{2 + Successes}

\ability{Celestial Spells}{A \name{} may use their \imp{Spellcasting} ability to cast the \imp{Refine\comma{} Charm\comma{} Mirage\comma{}Communicate\comma{} Inspire\comma{} Forge\comma{} Heal\comma{} Purify\comma{} Move\comma{} Seek} and \imp{Shield} spells. }, hasLanguages = 1, languages = \speaks{All spoken languages}, hasImage = 1, image = cherubim, description = The \name{} are the lowest (and youngest) order of \imp{Angels}\comma{} and as such are typically given the least dangerous quests. It is thought that this is the reason that \imp{Muggle} art has depicted them as chubby little babies with wings.

Of course\comma{} even the most lowly angel wields immense power\comma{} and the idea of them taking on such a lowly form is considered highly insulting. Instead they appear as the typical `angel’\comma{} a beautiful\comma{} winged individual of indeterminate gender\comma{} whose voice sings out like a choir. 

The holy quests assigned to the Cherubim are those which most commonly involve mortal beings\comma{} they are often tasked with delivering important missives to Emperors\comma{} providing visions of the future to prompt heros to venture forth to vanquish evil\comma{} or to act as a guardian for an important indivual as they grow. 

With these gentle prompts\comma{} the \name{} are able to alter the course of events across the multiverse.}
\beast{name = Seraphim, species = Angels, mind = Ineffable, category = Celestial, rating = VI, abilities = \ability{Choir of Angels}{For every additional \name{} within 25m\comma{} the \name{} gains +1d on all ability checks (max +5). Each \name{} is also perfectly aware of the status of the others\comma{} and they communicate instantaneously and telepathically whilst in this radius.}

\ability{Light in the Darkness}{If the target of an attack has more than one rating in \imp{Villainy}\comma{} or has used a \imp{Dark Arts} spell in the past 24 hours\comma{} the \name{} gets +2d for all attacks against them.}

\ability{Walk Among Mortals}{A \name{} may use their \imp{Shapechange} ability to take on a human form\comma{} hiding their wings. However\comma{} they remain almost supernaturally beautiful appearances.}

\ability{Master of Mind and Body}{A \name{} is immune to spells which would alter its mind or perception of reality\comma{} and it cannot have its form altered by magic unless it wishes to.}, article = A, movement = \speeds{\speedrating{Walking}{10}\speedrating{Flying}{24}}, fit =5, prs =5, vit =5, cha =3, dec =2, ins =3, int =5, wil =5, pcp =4, hasDamage = 1, damage =\key{Immune} to \textit{Incandescent}\comma{} \key{Resistant} to \textit{All physical damage from non\minus{}magical sources} and \key{Susceptible} to \textit{Necrotic}, nUnharmed=4, nBruised=2, nHurt=2, nInjured=2, nWounded=2, nMangled=1, block=5, dodge=4, defy=5, fortitude=6, imageStack=1, hasSkills = 1, skills = \skill{Speed}{7}
\skill{Skirmish}{7}
\skill{Flight}{6}
\skill{Strength}{6}
\skill{Alertness}{5}
\skill{Bravery}{5}
\skill{Conviction}{5}
\skill{Spellcasting}{4}
\skill{Covert}{4}
\skill{Intimidation}{3}
, hasAttacks = 1, attacks = \meleeConsequence{Heavenly Smite}{6}{\minus{}1}{Slashing Damage}{3 + Successes}{A \name{} may expend a \imp{Fortitude} point to immediately perform an additional 2 sword strikes on their target.}


\ability{Celestial Spells}{A \name{} may use their \imp{Spellcasting} ability to cast the \imp{Sense\comma{} Banish\comma{} Bind\comma{} Disarm\comma{} Heal\comma{} Disintegrate\comma{} Jinx\comma{} Move\comma{} Compel} and \imp{Shield} spells\comma{} as well as any from the \imp{Elemental} school. }, hasLanguages = 1, languages = \speaks{All spoken languages}, hasImage = 1, image = seraphim, description = The most numerous class of \imp{Angelic Warrior}\comma{} the \name{} are the righteous smiters of the universe. Appearing as a magnificent armoured humanoid with multiple pairs of soft\comma{} golden wings extending from their back\comma{} they wield spears forged from pure light and a soft golden halo is ever\minus{}present above their head. 

The primary goal of the \name{} is to fight evil and slay those who would threaten others and upset the balance between good and evil across the planes.  They are often regarded amongst the mightiest warriors in existence\comma{} though this comes with a certain amount of hubris. 

Though they are powerful warriors in their own right\comma{} the \name{} are most powerful when working in unison\comma{} being sent out in large groups (a \key{Choir}) to take down evildoers.}
\beast{name = Throne, species = Angels, mind = Ineffable, category = Celestial, rating = VII, abilities = \ability{Eyes Everywhere}{A \name{} has perfect 360$^\circ$ magical vision\comma{} and cannot be snuck up on\comma{} or decieved by invisibility\comma{} mirages or other such visual deceptions.}

\ability{Light in the Darkness}{If the target of an attack has more than one rating in \imp{Villainy}\comma{} or has used a \imp{Dark Arts} spell in the past 24 hours\comma{} the \name{} gets +1d for all attacks against them.}

\ability{Master of Mind and Body}{A \name{} is immune to spells which would alter its mind or perception of reality\comma{} and it cannot have its form altered by magic unless it wishes to.}

\ability{Mindmelting Form}{Any mortal being\comma{} when seeing a \name{} for the first time\comma{} must peform a DV 10 \imp{Willpower (Conviction)} check. On a failure\comma{} they are \imp{Paralyzed}\comma{} and must repeat the check once per round until they succeed. }, article = A, movement = \speeds{\speedrating{Flying}{4}}, fit =1, prs =5, vit =2, cha =0, dec =0, ins =3, int =9, wil =7, pcp =7, hasDamage = 1, damage =\key{Immune} to \textit{Incandescent}\comma{} \key{Resistant} to \textit{All physical damage from non\minus{}magical sources} and \key{Susceptible} to \textit{Necrotic}, nUnharmed=4, nBruised=1, nHurt=1, nInjured=0, nWounded=0, nMangled=1, block=1, dodge=2, defy=7, fortitude=6, imageStack=1, hasSkills = 1, skills = \skill{Spellcasting}{8}
\skill{Logic}{7}
\skill{Knowledge abilities}{7}
\skill{Imbue}{5}
\skill{Conviction}{5}
\skill{Flight}{1}
, hasAttacks = 1, attacks = \ability{Apotheosis}{A \name{} may use their \imp{Spellcasting} ability to cast any spell except those belonging to the \imp{Dark Arts} school.}


\ability{Planar Blink}{A \name{} may expend 5 \imp{Fortitude} points to instantly travel to any known point on any other plane of existence. This bypasses any magical blocks put in place to prevent transport.}, hasLanguages = 1, languages = \speaks{All possible languages}, hasImage = 1, image = throne, description = It is incredibly rare to see a \name{} outside of the \imp{Silver City}\comma{} for they are not messengers or mighty warriors – but instead scholars\comma{} guardians of knowledge and secrets. 

Their physical form is hard for a mortal being to comprehend – the closest anyone has ever really got is {\it wheels within wheels\comma{} covered in eyes}\comma{} and even that image was enough to break the mind of the human who witnessed it. They are certainly the least humanoid of the \imp{Angels}\comma{} and their intellect is equally alien. 

The \name{}s have an almost perfect recollection of every event in history\comma{} and collect any and all knowledge they can in their vast libraries\comma{} in the hope that it will one day be useful  in the fight against the ever\minus{}present evils. The rare occasions that they venture out of their libraries\comma{} it is to find some arcane secret – either to help their own cause\comma{} or to prevent it from falling into the wrong hands.}
\beast{name = Archangel, species = Angels, mind = Ineffable, category = Celestial, rating = VII, abilities = \ability{Walk Among Mortals}{\article{} \name{} may use their \imp{Shapechange} ability to take on a human form\comma{} hiding their wings. However\comma{} they remain almost supernaturally beautiful appearances.}

\ability{Light in the Darkness}{If the target of an attack has more than one rating in \imp{Villainy}\comma{} or has used a \imp{Dark Arts} spell in the past 24 hours\comma{} the \name{} gets +3d for all attacks against them.}

\ability{Master of Mind and Body}{\article{} \name{} is immune to spells which would alter its mind or perception of reality\comma{} and it cannot have its form altered by magic unless it wishes to.}, article = A, movement = \speeds{\speedrating{Walking}{10}\speedrating{Flying}{28}}, fit =7, prs =7, vit =7, cha =6, dec =3, ins =6, int =5, wil =7, pcp =5, hasDamage = 1, damage =\key{Immune} to \textit{Incandescent}\comma{} \key{Resistant} to \textit{All physical damage from non\minus{}magical sources} and \key{Susceptible} to \textit{Necrotic}, nUnharmed=10, nBruised=1, nHurt=1, nInjured=1, nWounded=1, nMangled=3, block=7, dodge=5, defy=5, fortitude=7, imageStack=1, hasSkills = 1, skills = \skill{Strength}{8}
\skill{Flight}{7}
\skill{Speed}{7}
\skill{Bravery}{7}
\skill{Conviction}{7}
\skill{Skirmish}{6}
\skill{Spellcasting}{5}
, hasAttacks = 1, attacks = \meleeConsequence{Heavenly Smite}{13}{\minus{}1}{Slashing Damage}{3 + Successes}{A \name{} may expend a \imp{Fortitude} point to immediately perform an additional 2 sword strikes on their target.}


\area{Radiant Aura}{10m sphere around \name{}}{10}{7}{Incandesence}{1+Successes}

\ability{Celestial Spells}{An \name{} may use their \imp{Spellcasting} ability to cast the \imp{Animate\comma{} Transmute\comma{} Sense\comma{} Banish\comma{} Bind\comma{} Disarm\comma{} Heal\comma{} Disintegrate\comma{} Jinx\comma{} Compel} and \imp{Shield} spells\comma{} as well as any from the \imp{Elemental} and \imp{Kinesis} schools. }, hasLanguages = 1, languages = \speaks{All spoken languages}, hasImage = 1, image = archangel, description = An \name{} is one of the most powerful entities in existence. The mightiest\comma{} wisest and fiercest warriors in the \imp{Angelic Host}\comma{} they serve as generals in the eternal war against \imp{Abominations} and \imp{Demons}.

The existence of the \name{}s has seeped into the cultural knowledge of almost every society on every plane – seen as servents of deities\comma{} protectors of light and life – and are revered and loved by all. 

The \imp{Ministry} has only been able to gather evidence of a handful of individual \imp{Archangels}\comma{} though given the ferocity of wars in which they fight\comma{} this has lead many to speculate that their names are handed down as titles\comma{} to continue an unbroken line of Archangels throughout the history of the \imp{Host}.}

}




\species{Apparitions}
{
	Apparations are ghostly creatures \minus{} spirits and ghosts which defy the laws of life and death\comma{} and yet continue to roam the mortal realms. 

\ability{Incorporeal Form}{Almost all apparitions are merely imprints\comma{} shadows lying between the astral realm and the mortal plane\comma{} and as such are totally incapable of interacting with the physical realm. They can pass through solid objects at will\comma{} move with blatant disregard for the force of gravity\comma{} as well as being immune to all normal forms of attack. }

\ability{Unknowable Purpose}{It is not understood what drives apparaitions of any kind to remain behind on the mortal plain. Some speculate that all apparitions are manifestations of lost souls\comma{} bound to the Earth through their need to find closure\comma{} or complete some important task. Others speculate that they are glitches in the fabric of reality\comma{} whose motives even they themselves do not understand.}

\ability{Unkillable}{It is impossible to kill an apparition\comma{} though it is possible to banish them for a time. The only known way to permanently deal with an apparition is to plunge one into the Void\comma{} or help them find the closure they need\comma{} or otherwise convince them to relinquish their hold on the mortal realm. }
}
{
\beast{name = Ghost, species = Apparitions, mind = Ineffable, category = Phantasm, rating = , abilities = \ability{Incorporeal Form}{A \name{} has no physical form\comma{} and so may move through solid objects  at their flight speed\comma{} and is immune to all normal attacks.}

\ability{Wisdom of Life}{A \name{} gains additional \imp{Knowledge} abilities based on their experiences during their life. }, article = A, movement = \speeds{\speedrating{Flying}{4}}, fit =0, prs =2, vit =0, cha =3, dec =3, ins =2, int =3, wil =3, pcp =2, hasDamage = 1, damage =\key{Immune} to \textit{All damage}, nUnharmed=0, nBruised=0, nHurt=0, nInjured=0, nWounded=0, nMangled=0, block=0, dodge=7, defy=2, fortitude=1, imageStack=0, hasSkills = 1, skills = \skill{History}{3}
\skill{Intimidation}{2}
\skill{Flight}{1}
, hasAttacks = 1, attacks = \ranged{Haunting}{5}{5}{\minus{}1}{Terrified Status}{1 + Successes}, hasLanguages = 1, languages = \speaks{The languages they spoke in life}, hasImage = 1, image = ghost, description = A ghost is the imprint of the soul of a once\minus{}living wizard or witch\comma{} left to wander the material realm after their physical form has died. A ghost resembles their former selves at the moment of their death\comma{} though in a translucent\comma{} silver\minus{}grey form. 

No\minus{}one knows what causes a ghost to remain behind\comma{} though it is posited that these fleshless spirits were mortally afraid of death or have some extraordinarily strong connection to the locations they haunt.}
\beast{name = Poltergeist, species = Apparitions, mind = Ineffable, category = Phantasm, rating = II, abilities = \ability{Phaseshift}{A \name{} may use an action to shift between corporeal and incorporeal form and vice versa. Whilst in incorporeal form the \name{} is immune to all harm\comma{} can fly and can pass through solid objects.}, article = A, movement = \speeds{\speedrating{Flying}{12}}, fit =2, prs =5, vit =1, cha =2, dec =3, ins =3, int =2, wil =4, pcp =3, hasDamage = 1, damage =\key{Resistant} to \textit{All damage}, nUnharmed=1, nBruised=1, nHurt=1, nInjured=0, nWounded=0, nMangled=1, block=1, dodge=4, defy=3, fortitude=2, imageStack=0, hasSkills = 1, skills = \skill{Covert}{4}
\skill{Flight}{3}
\skill{Marksmanship}{3}
\skill{Spellcasting}{2}
, hasAttacks = 1, attacks = \ranged{Throw Objects}{10}{7}{\minus{}2}{Bashing damage}{1 + Successes}

\ability{Arcane Trickster}{A \name{} may use their \imp{Spellcasting} ability to cast the \imp{Move\comma{} Degrade\comma{} Mirage} and \imp{Bypass} spells.}, hasLanguages = 1, languages = \speaks{Human languages}, hasImage = 1, image = peeves, description = A poltergeist is an amortal\comma{} indestructable spirit of chaos and mischief. They appear as a short\comma{} childlike figure dressed in a motley jester\apos{}s garb\comma{} with glowing orange eyes\comma{} which twinkle with mischief. 

Brought into existence by a critical mass of humans\comma{} trickery and mischief\comma{} poltergeists haunt the specific place which they are tied to. 

Unusually out of apparitions and other spiritual creatures\comma{} poltergeists are able to take on physical form and cast primitive forms of magic \minus{} which they use to wreak chaos and play pranks on unsuspecting humans.}
\beast{name = Boggart, species = Apparitions, mind = Ineffable, category = Phantasm, rating = II, abilities = \ability{Phobomorph}{A \name{} can use its \imp{Shapeshift} ability to take on any form it desires (even esoteric and abstract concepts can be represented). If this ability is used to take the form of something the target fears\comma{} the DV to resist the \imp{Incite Fear} ability is increased by 3.}

\ability{Killing Joke}{A \name{} fears and hates laughter. A peal of genuine laughter instantly causes the \name{} to take the \imp{Critical Condition} status.}, article = A, movement = \speeds{\speedrating{Walking}{5}}, fit =1, prs =2, vit =0, cha =1, dec =4, ins =6, int =2, wil =3, pcp =3, hasDamage = 1, damage =\key{Immune} to \textit{All damage} and \key{Susceptible} to \textit{Genuine laughter}, nUnharmed=0, nBruised=0, nHurt=0, nInjured=0, nWounded=0, nMangled=0, block=1, dodge=2, defy=3, fortitude=3, imageStack=0, hasSkills = 1, skills = \skill{Shapeshift}{6}
\skill{Intimidation}{3}
\skill{Speed}{2}
, hasAttacks = 1, attacks = \ability{Pierce Soul}{A target within 10m of the \name{} must contest a DV 7 \imp{Insight} from the Boggart. On a faiulre\comma{} the \name{} learns a piece of information from the target\comma{} such as their deepest fear.}

\ranged{Incite Terror}{5}{7}{0}{Terrified Status}{1 + Success}, hasImage = 1, image = boggart, description = A manifestation of fear and primal terror\comma{} the shapeshifting boggart peers into the minds of humans\comma{} and takes the form of their worst nightmare. 

A boggart can never harm you\comma{} though they can be difficult to contain. The accepted trick is to transfigure them to look stupid\comma{} prompting a fit of laughter – which is fatal to a boggart.}

}




\species{Arachnid}
{
	The arachnids are a family of giant spider found throughout the wizarding world. Most members of this species are suspected to have been formed from mundane species that were experimented upon by witches and wizards throughout history\comma{} though others are known to occur in freak mutations. 

Whatever the mechanism for bringing them into this world\comma{} many have since escaped into the wild\comma{} to wreak havoc on muggles and wizardkind alike \minus{} some spinning their webs to ensnare the unwary\comma{} others prowling and hunting directly for their prey. 

\ability{Great Size}{The magical arachnids are much larger than their non\minus{}magical compatriots. Though smaller than acromantula\comma{} some species can reach legspans of up to one metre.}

\ability{Keen Sight}{In addition to their web\minus{}enhanced senses\comma{} the 8 compound eyes of arachnids allow them to see in incredible detail\comma{} even in dim light}

\ability{Webspinners}{As members of the spider family\comma{} all arachnids have an affinity for spinning webs\comma{} and using them to sense and then ensnare their prey.}
}
{
\beast{name = Great Widow, species = Arachnid, mind = Non\minus{}sapient, category = Beast, rating = III, abilities = \ability{Webwalker}{\article{} \name{} takes no movement penalty on webbed surfaces\comma{} and uses their \imp{Inhuman Senses} to sense vibrations in their webs.}

\ability{Sticky Feet}{A \name{} may use their \imp{Climbing} ability to walk on any vertical surface.}, article = A, movement = \speeds{\speedrating{Walking}{4}\speedrating{Climbing}{1.5}}, fit =2, prs =3, vit =2, cha =0, dec =1, ins =1, int =1, wil =2, pcp =4, hasDamage = 1, damage =\key{Immune} to \textit{Poison}, nUnharmed=1, nBruised=1, nHurt=1, nInjured=0, nWounded=0, nMangled=1, block=1, dodge=2, defy=1, fortitude=1, imageStack=0, hasSkills = 1, skills = \skill{Covert}{4}
\skill{Climb}{3}
\skill{Inhuman Senses}{3}
\skill{Speed}{1}
, hasAttacks = 1, attacks = \meleeConsequence{Poison Fangs}{5}{0}{Stabbing Damage}{1+Successes}{If the attack deals any damage\comma{} the victim takes the \imp{Poisoned} status (1 \imp{harm}\comma{} requires 5 successes)}

\ranged{Acid Spit}{5}{4}{\minus{}1}{Acid damage}{1 + Successes}, hasLanguages = 1, languages = \understands{Spider Tongue}, hasImage = 1, image = blackWidow, description = Magical experimentation on a {\it Black Widow} produced this grossly oversized specimen\comma{} and gave it the ability to spit acid.}
\beast{name = Howling Tick, species = Arachnid, mind = Non\minus{}sapient, category = Beast, rating = III, abilities = \ability{Webwalker}{\article{} \name{} takes no movement penalty on webbed surfaces\comma{} and uses their \imp{Inhuman Senses} to sense vibrations in their webs.}

\ability{Prolific Jumpers}{As part of their movement\comma{} a \name{} may jump a distance up to their total movement speed\comma{} in any direction.}, article = A, movement = \speeds{\speedrating{Walking}{7}}, fit =4, prs =3, vit =1, cha =0, dec =0, ins =1, int =1, wil =2, pcp =3, hasDamage = 0, nUnharmed=2, nBruised=2, nHurt=0, nInjured=0, nWounded=1, nMangled=0, block=1, dodge=3, defy=1, fortitude=1, imageStack=0, hasSkills = 1, skills = \skill{Speed}{4}
\skill{Covert}{3}
\skill{Inhuman Senses}{2}
, hasAttacks = 1, attacks = \ability{Leap Attack}{As a single action\comma{} the \name{} may jump up to 5m and perform a \imp{Bite} attack\comma{} and then jump a further 1m.}

\melee{Bite}{5}{\minus{}1}{Stabbing damage}{2 + Successes}, hasLanguages = 1, languages = \understands{Spider Tongue}, hasImage = 1, image = tick, description = The name of the Howling Tick is misleading\comma{} as it is neither a tick\comma{} and nor does it howl. Instead the name comes from its tendency to suck blood from its victims\comma{} and the howls of pain that result.

The Howling Tick has the magical ability to grow in size when it feeds\comma{} however they must continually gorge in order to maintain their size\comma{} or they quickly shrink back.}
\beast{name = Spraying Mantis, species = Arachnid, mind = Non\minus{}sapient, category = Beast, rating = III, abilities = \ability{Four Forearms}{The \name{} has two sets of arms\comma{} and so can grapple up to two individuals at a time.}

\ability{Webwalker}{\article{} \name{} takes no movement penalty on webbed surfaces\comma{} and uses their \imp{Inhuman Senses} to sense vibrations in their webs.}, article = A, movement = \speeds{\speedrating{Walking}{6}}, fit =3, prs =5, vit =2, cha =0, dec =1, ins =2, int =2, wil =1, pcp =3, hasDamage = 1, damage =\key{Immune} to \textit{Acid\comma{} Poison}, nUnharmed=3, nBruised=0, nHurt=1, nInjured=1, nWounded=0, nMangled=1, block=3, dodge=2, defy=2, fortitude=3, imageStack=0, hasSkills = 1, skills = \skill{Speed}{3}
\skill{Covert}{2}
\skill{Brawl}{2}
, hasAttacks = 1, attacks = \meleeConsequence{Hooked Arms}{5}{\minus{}1}{Stabbing Damage}{2 + Successes}{The \name{} then initiates a \imp{Grapple} action}

\area{Acid Spray}{Cone\comma{} 3m in length from \name{}’s mouth}{5}{0}{Acid damage}{1 + Successes}

\ability{Liquefaction}{The \name{} injects a \imp{Grappled} target with caustic digestive juices\comma{} dealing level 4 \imp{Acid} damage. If this reduces the target to a \imp{Critical Condition}\comma{} the target is reduced to liquid and devoured by the \name{}.}, hasLanguages = 1, languages = \understands{Spider Tongue}, hasImage = 1, image = spraying, description = A gigantic\comma{} horrifying crossbreed between a spider\comma{} and a praying mantis resulted in a grotesque monstrosity. The being appears\comma{} outwardly\comma{} to be a giant metre\minus{}long insect walking on 4 legs\comma{} with an additional 4 arms turned into hinged and hooked arms which they use to catch their prey. 

True to their name\comma{} they also spray acidic juices on their prey\comma{} to aid in their eventual digestion.}
\beast{name = Brood Mother, species = Arachnid, mind = Non\minus{}sapient, category = Beast, rating = III, abilities = \ability{Webwalker}{\article{} \name{} takes no movement penalty on webbed surfaces\comma{} and uses their \imp{Inhuman Senses} to sense vibrations in their webs.}   

\ability{Sticky Feet}{A \name{} may use their \imp{Climbing} ability to walk on any vertical surface.}

\ability{Nest Builder}{If a \name{} spends more than one day in a location\comma{} they begin to construct a nest – a region up to 5m in radius around some central point. Whilst within their nest\comma{} a \name{} gets +1d to all checks.}, article = A, movement = \speeds{\speedrating{Walking}{4}\speedrating{Climbing}{1.5}}, fit =2, prs =3, vit =2, cha =0, dec =1, ins =1, int =2, wil =2, pcp =3, hasDamage = 1, damage =\key{Resistant} to \textit{Poison}, nUnharmed=2, nBruised=1, nHurt=1, nInjured=1, nWounded=1, nMangled=1, block=2, dodge=2, defy=1, fortitude=3, imageStack=0, hasSkills = 1, skills = \skill{Inhuman Senses}{5}
\skill{Covert}{4}
\skill{Climb}{3}
\skill{Command}{3}
\skill{Speed}{1}
, hasAttacks = 1, attacks = \meleeConsequence{Poison Fangs}{4}{\minus{}1}{Stabbing damage}{1 + Successes}{If the attack deals any damage\comma{} the victim takes the \imp{Poisoned} status (1 \imp{harm}\comma{} requires 10 successes)}
\ability{Hatch Brood}{Perform a DV 7 \imp{Command} check\comma{} hatching a number of spiders equal to the successes into a space adjacent to the \name{}. Each hatchling has 1 level of health\comma{} but otherwise has the same statistics as the \name{}\comma{} without the \imp{Hatch Brood} ability..}, hasLanguages = 1, languages = \understands{Spider Tongue}, hasImage = 1, image = Brood, description = This grossly oversized spider is the result of a freak mutation which  causes them to grow to grotesque sizes and become viciously maternal. A Brood Mother will collect any and all spider eggs that it finds and nuture them as if they were her own in the dark\comma{} secluded where she has built her nest.}

}




\species{Beast Demon}
{
	Demons prowled the earth for many millenia before the dawn of human civilization\comma{} and come in many thousands of shapes and forms. The \imp{Beast Demons} are those which share – at least to a cursory glance – a visual similarity with a non\minus{}magical creature\comma{} as well as a more bestial intelligence\comma{} and reliance on primal instinct above tactics and reasoned thought. 

Typically using their more unassuming forms to get close to their prey\comma{} they unleash their demonic fury and hunger upon their prey\comma{} leaving no trace of their meal.
}
{
\beast{name = Nogtail, species = Beast Demon, mind = Ineffable, category = Demon, rating = IV, abilities = \ability{Blighting Presence}{A \name{} exudes an aura which curses the land around it for 1km in every direction from its nest. For every week the Nogtail has been nesting\comma{} all beings in this radius take a 1d penalty (max 5d) to all checks made\comma{} plants wither and die\comma{} and animals become sickly and weak. }

\ability{Nogtail Weakness}{If a pure\minus{}white dog is brought within 10m of the \imp{Nogtail}\comma{} it becomes \imp{Terrified} and must use its movement to get as far away from the creature as possible.}

\ability{Moving Target}{On any turn in which the \name{} moves more than half its movement\comma{} it gains +1d to all \imp{Dodge} checks\comma{} and incurs no drain.}, article = A, movement = \speeds{\speedrating{Walking}{11}}, fit =5, prs =2, vit =3, cha =0, dec =1, ins =1, int =2, wil =4, pcp =2, hasDamage = 1, damage =\key{Immune} to \textit{Necrotic}\comma{} \key{Resistant} to \textit{Physical damage} and \key{Susceptible} to \textit{Incandescent}, nUnharmed=3, nBruised=1, nHurt=1, nInjured=1, nWounded=2, nMangled=0, block=1, dodge=6, defy=2, fortitude=3, imageStack=1, hasSkills = 1, skills = \skill{Speed}{8}
\skill{Intimidation}{2}
, hasAttacks = 1, attacks = \melee{Bite}{7}{\minus{}2}{Stabbing damage}{1 + Successes}

\ranged{Focussed Blight}{5}{6}{0}{Necrotic damage}{1 + Successes}

\ability{Energy Reserves}{The \name{} uses some of the cursed energy it has stored in its nest: the effect from the \imp{Blighting presence} becomes one level less severe\comma{} but the \name{} makes a \imp{Focussed Blight} attack against all targets in range.}, hasImage = 1, image = nogtail, description = One of the lesser demons still native to the mortal realm\comma{} the \name{} resembles a stunted piglet\comma{} albeit with a thick stubby tail and elongated legs. 

Nogtails are known to sneak into farms to suckle from an ordinary pig\comma{} bringing with them a terrible\comma{} cursed blight which stuck to the land. Capable of reaching immense speeds on land\comma{} catching a Nogtail is therefore impossible – the only way to drive one off for good is to hunt them down and chase it away with a pure\minus{}white dog. 

The nogtail poses a threat not only because of the blight which follows it\comma{} but because of their voracious appetite\comma{} wicked teeth and willingness to take a bite out of any fool who gets too close to them.}
\beast{name = Kishi, species = Beast Demon, mind = Ineffable, category = Demon, rating = V, abilities = \ability{Carrion Hauler}{The \name{} does not have its speed halved when dragging a \imp{grappled} foe unless they are {\it significantly} heavier than the \name{}. }

\ability{Two Mouths}{The \name{} may use its \imp{Hypnotic Words} ability whilst \imp{Grappling} a foe\comma{} but not its \imp{Evicerating Bite}.}, article = A, movement = \speeds{\speedrating{Walking}{6}}, fit =4, prs =3, vit =2, cha =5, dec =5, ins =3, int =3, wil =4, pcp =2, hasDamage = 0, nUnharmed=2, nBruised=1, nHurt=0, nInjured=2, nWounded=0, nMangled=2, block=2, dodge=3, defy=3, fortitude=3, imageStack=0, hasSkills = 1, skills = \skill{Eloquence}{4}
\skill{Speed}{3}
\skill{Cover}{3}
\skill{Spellcasting}{3}
\skill{Intimidation}{3}
, hasAttacks = 1, attacks = \melee{Evicerating Bite}{6}{\minus{}1}{Stabbing damage}{3 + Successes}

\melee{Swift Scratch}{4}{\minus{}1}{Cutting damage}{1 + Successes}

\melee{Latching Bite}{4}{0}{Grappled Status}{6 + Successes}

\ability{Hypnotic Words}{The \name{} may use its \imp{Spellcasting} ability to cast the \imp{Charm} and \imp{Delude} spells. }, hasLanguages = 1, languages = \speaks{Human languages\comma{} Abyssal}, hasImage = 1, image = kishi, description = Native to the southern part of Africa\comma{} this demonic entity takes the form of a beautifully sleek and well\minus{}kept hyena\comma{} marred only by the addition of a humanoid head protruding from the back of its normal snouted face.

This human face speaks honeyed words in a calming\comma{} almost hypnotic voice and is known to lure children out into the darkness\comma{} where the hyena mouth uses its wickedly long teeth and near\minus{}unbreakable grip to maul any who cross its path.}

}




\species{Bowtruckle}
{
	Bowtruckles are a species of hand\minus{}sized\comma{} insect\minus{}eating humanoids which reside inside trees. Bowtruckles prefer to make their home in trees with wand\minus{}quality wood (or perhaps\comma{} it is the presence of a Bowtruckle which makes a tree wand\minus{}grade)\comma{} and a single tree can host up to 5 generations of the same bowtruckle clan. 

Normally peacable and shy creatures\comma{} they become territorial and violent when their home tree is threatened. 

The classification of the intelligence of the \imp{Bowtruckles} has been somewhat controversial – they are evidently intelligent and able to communicate with and understand humans\comma{} however they do not seem to possess the ability for abstract thinking or tool usage that most consider necessary for a \imp{sapient} classification. 

\ability{Camouflaged}{Bowtruckles blend in perfectly with their trees\comma{} when they wish to pass unnoticed\comma{} they appear as nothing more than a set of leafy twigs. It is only by cathcing them in motion that they can be easily spotted.}

\ability{Natural Climbers}{Living their entire life in trees\comma{} bowtruckles are natural climbers\comma{} and can move across near\minus{}sheer vertical surfaces as easily as they walk}. 

\ability{Long Fingers}{Nominally evolved to help dig insects out of the bark of a tree\comma{} the long spindly fingers of a bowtruckle can be used to perform very delicate tasks\comma{} such as picking a lock\comma{} or used offensively to poke out the eyes of those who threaten their treetop homes.}
}
{
\beast{name = Softwood Bowtruckle, species = Bowtruckle, mind = Non\minus{}sapient, category = Imp, rating = II, abilities = \ability{One with wood}{The \name{} may use their \imp{Elusion} ability to appear as a simple twig or leaf. As long as they remain still\comma{} this illusion is near\minus{}perfect.}, article = A, movement = \speeds{\speedrating{Walking}{1}}, fit =1, prs =3, vit =1, cha =3, dec =1, ins =2, int =2, wil =2, pcp =3, hasDamage = 1, damage =\key{Susceptible} to \textit{Fire}, nUnharmed=1, nBruised=0, nHurt=0, nInjured=1, nWounded=0, nMangled=0, block=1, dodge=2, defy=1, fortitude=1, imageStack=0, hasSkills = 1, skills = \skill{Elusion}{5}
\skill{Covert}{4}
\skill{Acrobatics}{3}
\skill{Kinship}{2}
\skill{Kindness}{2}
, hasAttacks = 1, attacks = \melee{Poke}{4}{\minus{}2}{Stabbing damage}{1 + Successes}

\melee{Go for the eyes}{6}{\minus{}1}{Blinded condition}{1 + Successes}, hasLanguages = 1, languages = \understands{Human language}, hasImage = 1, image = bowtruckle, description = The \name{}\comma{} as the name may suggest\comma{} reside within softwood trees\comma{} typically pine\comma{} cedars\comma{} firs\comma{} yews and redwoods\comma{} and prefer a cooler or damper environment than their hardwood cousins. They appear as green\minus{}skinned elfin creatures with leaves growing from random parts of their body\comma{} and are often said to have `kind faces’.

The \imp{Softwood} branch of the family are incredibly flexible\comma{} able to contort themselves through even the smallest of gaps as they hunt for insects\comma{} though this comes at the expense of a natural armour. 

The softwood is the most friendly of the bowtruckle species\comma{} often forming friendships with humans and other animals which they pass down through generations. However\comma{} they have also shown a tendancy to become emotional and sulk when their `friend’ gives them insufficient attention.}
\beast{name = Hardwood Bowtruckle, species = Bowtruckle, mind = Non\minus{}sapient, category = Imp, rating = III, abilities = \ability{One with wood}{The \name{} may use their \imp{Elusion} ability to appear as a simple twig or leaf. As long as they remain still\comma{} this illusion is near\minus{}perfect.}

\ability{Charcoal skin}{The first time a \name{} takes \imp{Fire} damage\comma{} it loses its \imp{Susceptibility} to fire damage for 24 hours\comma{} and the \name{} takes on a blackened appearance.}, article = A, movement = \speeds{\speedrating{Walking}{2}}, fit =2, prs =2, vit =2, cha =1, dec =1, ins =1, int =3, wil =3, pcp =3, hasDamage = 1, damage =\key{Susceptible} to \textit{Fire}, nUnharmed=1, nBruised=1, nHurt=0, nInjured=1, nWounded=1, nMangled=0, block=3, dodge=2, defy=1, fortitude=1, imageStack=0, hasSkills = 1, skills = \skill{Elusion}{5}
\skill{Covert}{3}
\skill{Intimidation}{2}
\skill{Bravery}{2}
\skill{Strength}{2}
\skill{Brawl}{2}
, hasAttacks = 1, attacks = \melee{Poke}{6}{\minus{}2}{Stabbing damage}{1 + Successes}

\melee{Go for the eyes}{7}{\minus{}1}{Blinded condition}{1 + Successes}, hasLanguages = 1, languages = \understands{Human language}, hasImage = 1, image = barktruckle, description = Residing within mighty hardwood trees such as oaks\comma{} ironwoods\comma{} mahoganies and willows\comma{} the Hardwoods are much hardier and more resilient than their softwood bretheren. 

The bodies of the hardwood bowtruckles seem to be composed almost entirely from bark\comma{} wood and twigs intertwined to form the body. Small sproutings of green may be seen from their body during spring (from which their young grow)\comma{} but otherwise they are without discernable features. 

This hardiness has evolved because life for the hardwoods is much tougher and more violent than the softwoods. Whilst softwoods are known to form friendships and only attack when provoked\comma{} the hardwoods are more likely to flee or lash out at unwanted visitors.}

}




\species{Ceratothid}
{
	The Ceratothids are a family of loosely related magical quadrupeds. Defined by their huge bulk and relatively bovine\minus{}like appearance\comma{} most Ceratothid\apos{}s have a gentle temperament until angered\comma{} at which point their great mass and inherent magic makes them dangerous foes.
}
{
\beast{name = Graphorn, species = Ceratothid, mind = Non\minus{}sapient, category = Beast, rating = IV, abilities = \ability{Graphorn Hide}{Whenever an attack \imp{Power} is reduced to zero by a \imp{Resist}\comma{} the \name{} takes no \imp{Drain}.}, article = A, movement = \speeds{\speedrating{Walking}{5}}, fit =3, prs =1, vit =5, cha =1, dec =0, ins =1, int =1, wil =2, pcp =3, hasDamage = 1, damage =\key{Resistant} to \textit{Physical damage}, nUnharmed=6, nBruised=2, nHurt=0, nInjured=2, nWounded=0, nMangled=2, block=6, dodge=1, defy=6, fortitude=1, imageStack=0, hasSkills = 1, skills = \skill{Strength}{4}
\skill{Bravery}{3}
\skill{Indimidation}{3}
\skill{Speed}{2}
, hasAttacks = 1, attacks = \meleeConsequence{Horn Gore}{7}{0}{Stabbing Damage}{2 + Successes}{If the \name{} moves at least half its full movement before taking this attack\comma{} it deals an additional three levels of harm.}

\meleeConsequence{Body Slam}{5}{\minus{}2}{Crushing Damage}{3 + Successes}{This ability leaves the \name{} \imp{Prone} }

\melee{Tail strike}{(range 3m)\comma{} 6}{\minus{}2}{Bashing damage}{1 + Successes}, hasImage = 1, image = Graphorn, description = The Graphorn is found in mountainous European regions. Large and greyish purple with a humped back\comma{} the Graphorn has a number of very long\comma{} sharp and golden horns running across its back\comma{} walks on large\comma{} four\minus{}thumbed feet\comma{} and has an extremely aggressive nature. Their mouth is surrounded by a number of prehensile tendrils\comma{} which they use both for manipulating food\comma{} and for sensing their surroundings. 

Mountain trolls can occasionally be seen mounted on Graphorns\comma{} though the latter do not seem to take kindly to attempts to tame them and it is more common to see a troll covered in Graphorn scars. Powdered Graphorn horn is used in many potions\comma{} though it is immensely expensive owing to the difficulty in collecting it. Graphorn hide is even tougher than a dragon’s and repels most spells.}
\beast{name = Erumpent, species = Ceratothid, mind = Non\minus{}sapient, category = Beast, rating = V, abilities = \ability{Erumpent Hide}{When the \imp{Block} ability is reduced to zero through \imp{Drain}\comma{} the \imp{Erumpent} loses its immunity to spells.}, article = A, movement = \speeds{\speedrating{Walking}{6}}, fit =3, prs =2, vit =4, cha =1, dec =1, ins =1, int =1, wil =1, pcp =2, hasDamage = 1, damage =\key{Immune} to \textit{\imp{Elemental spells\comma{} Hexes } and \imp{Curses} cast below \levelFour{} level.}, nUnharmed=5, nBruised=2, nHurt=3, nInjured=1, nWounded=0, nMangled=1, block=4, dodge=1, defy=3, fortitude=2, imageStack=1, hasSkills = 1, skills = \skill{Strength}{5}
\skill{Speed}{3}
, hasAttacks = 1, attacks = \meleeConsequence{Impale}{7}{\minus{}1}{Stabbing Damage}{2 + Successes}{Attempts to \imp{Block} this attack automatically fail.}

\meleeConsequence{Body Slam}{5}{\minus{}2}{Crushing Damage}{3 + Successes}{This ability leaves the \name{} \imp{Prone} }

\ability{Explosive Injection}{Select a target harmed by the \imp{Impale} attack within the last 3 rounds. A n explosion is triggered with radius 5m\comma{} centred on that target\comma{} dealing level 6 \imp{Fire} damage to all in range.}, hasImage = 1, image = erumpet, description = An enourmous\comma{} rhinocerous\minus{}like magical beast hailing from Africa\comma{} the Erumpent is an extremely dangeous beast\comma{} thanks to the gigantic horn which protrudes from its head. 

This horn pierces even the toughest armour and contains a naturally\minus{}occuring alchemical which causes whatever it is injected into to detonate in a mighty explosion. In addition to this overwhelming (literal) firepower\comma{} the Erumpent’s hide is near\minus{}immune to many powerful magics.}
\beast{name = Re'em, species = Ceratothid, mind = Non\minus{}sapient, category = Beast, rating = IV, abilities = \ability{Unstoppable}{Once it has started moving\comma{} the magic within the \name{}’s blood negates all magic which would stop it\comma{} slow it\comma{} or otherwise alter its course. }, article = A, movement = \speeds{\speedrating{Walking}{5}}, fit =7, prs =2, vit =4, cha =2, dec =0, ins =1, int =1, wil =1, pcp =2, hasDamage = 1, damage =\key{Resistant} to \textit{Physical damage}, nUnharmed=4, nBruised=0, nHurt=2, nInjured=0, nWounded=3, nMangled=2, block=5, dodge=1, defy=3, fortitude=2, imageStack=0, hasSkills = 1, skills = \skill{Strength}{9}
\skill{Speed}{2}
, hasAttacks = 1, attacks = \ability{Trampling Charge}{The \name{} moves in a straight line a distance equal to its movement speed\comma{} tracing a cylinder with a 1m radius. Any being caught in this region which does not use the \imp{Dodge} action (Or have an ally do the same to save them) takes level 5 \imp{crushing} damage. }

\meleeConsequence{Body Slam}{6}{\minus{}2}{Crushing Damage}{3 + Successes}{This ability leaves the \name{} \imp{Prone} }, hasImage = 1, image = reem, description = Once abundant across the North American continent\comma{} the mighty \name{} has been hunted to near extinction. Their mighty frames reach up to 3m\comma{} and are completely covered in a lustrous golden hide.

Though exquisite\comma{} this hide is not the reason for their desirability as prey: rather it is their blood\comma{} which acts as a powerful alchemical reagent\comma{} imparting on the drinker a fraction of the immense strength of the \name{}. 

It is said that once a \name{} has started moving\comma{} no force\comma{} physical or magical can stop them or change their path unless they choose to – researchers have found that diving out of the way of the path of a \name{} on the warpath is often the most sensible option.}

}




\speciesBeast{name = Chimera, species = Chimera, mind = Non\minus{}sapient, category = Monstrosity, rating = VII, abilities = \ability{Eyes everywhere}{A \name{} has perfect 360$^\circ$ vision\comma{} and cannot be snuck up upon or surprised\comma{} except by magical invisibility.}

\ability{Regenerative}{At the end of every round\comma{} if it is not unconscious or \imp{incapacitated}\comma{} the \name{} may perform a DV 7 \imp{Vitality (Regeneration)} check\comma{} healing itself equal to the number of successes.}, article = A, movement = \speeds{\speedrating{Walking}{8}}, fit =4, prs =4, vit =5, cha =2, dec =2, ins =2, int =4, wil =6, pcp =7, hasDamage = 1, damage =\key{Resistant} to \textit{\imp{Energetic damage}}, nUnharmed=6, nBruised=4, nHurt=3, nInjured=2, nWounded=2, nMangled=0, block=5, dodge=6, defy=5, fortitude=7, imageStack=1, hasSkills = 1, skills = \skill{Strength}{6}
\skill{Speed}{5}
\skill{Regeneration}{5}
, hasAttacks = 1, attacks = \ability{Multiheaded}{As a single \imp{Major action}\comma{} each head may make an attack of their choosing\comma{} or attempt to negate an incoming attack.}

\melee{Lion’s Bite}{12}{\minus{}2}{Stabbing damage}{1 + Successes}

\area{Lion’s Roar}{all beings within 30m who can hear}{7}{0}{Terrified status}{4 + Successes}

\ranged{Goat’s Flame}{10}{6}{\minus{}1}{FIre damage}{3 + Successes}

\melee{Goat’s Electrification}{5}{\minus{}1}{Electric damage}{4 + Successes}

\melee{Snake’s Bite}{(range 2m) 8}{\minus{}1}{Stabbing damage}{1 + Successes}

\rangedConsequence{Snake’s Poison}{10m}{7}{0}{Poison damage}{1 + Successes}{If the being takes harm from this attack\comma{} they take the \imp{Poisoned} status (2 \imp{harm}\comma{} requires 10 \imp{successes})}, hasImage = 1, image = chimera, description = One of the most dangerous\comma{} and notorious\comma{} artificial magical creatues in all of history\comma{} the \imp{Chimera} truly is a terrifying beast. Said to have been created by the mad witch \imp{Echidna}\comma{} the `mother of monsters’\comma{} the \name{} escaped into the world and began their reign of terror. 

Possessing the heads of both a lion and a goat\comma{} with a further snake\minus{}head protruding from its serpentine tail – all of which have the ability to act independently\comma{} the chimera would be terrifying enough with this alone. Alas\comma{} \imp{Echidna} was not yet done – the goat’s head has the ability to breath gouts of fire and summon bolts of lightning\comma{} whilst the lion’s head can release a howl which pierces deep into the minds of those unfortunate enough to be surrounding it. 

Only one wizard is on record as having successfully defeating a chimera – and they were killed by the sheer effort required. Humanity is lucky that the chimera are also incredibly individualistic and violent towards their own kind\comma{} only mating once a century – else chimera would probably be the dominant species on this planet.}



\species{Cloakwraiths}
{
	A tattered black cloak might not seem the most terrifying piece of attire that a being could don\comma{} though those who have encountered the \key{Cloakwraiths} might say differently. 

No\minus{}one really knows where they come from\comma{} though ancient legends say that they are the spirits of those too evil to pass to the other side. Some \imp{wraiths} appear as gaunt almost\minus{}humanoids beneath their eponymous shrouds\comma{} whilst others seem to have no corporeal form. Some believe that they are in fact a single type of being at various points in their lifecycle\comma{} whilst others believe them to be individual manifestations of primal human fears.

What they all have in common (besides the ominous cloak)\comma{} is an evil aura of terror\comma{} and a hunger for human souls.
}
{
\beast{name = Lethifold, species = Cloakwraiths, mind = Ineffable, category = Abomination, rating = IV, abilities = \ability{Soporific Aura}{Any creature within 1m of the \name{} takes 2d penalty to all attempts to \imp{Resist} sleep.}

\ability{Magical Resistance}{The \name{} gets a +3d bonus to all \imp{Resist} checks against all spells cast below a \levelFour{} level (except the \imp{Patronus})}

\ability{Digestion}{A being trapped by the \imp{Envelop} ability is trapped within the \name{} and begins to be digested\comma{} taking level one harm on the first turn it is trapped\comma{} level two on the second\comma{} and so on. If the being reaches the \imp{Critical Condition} in this fashion\comma{} it is instantly killed\comma{} and absorbed into the \name{}. }, article = A, movement = \speeds{\speedrating{Flying}{8}}, fit =1, prs =4, vit =1, cha =0, dec =0, ins =3, int =1, wil =1, pcp =1, hasDamage = 1, damage =\key{Immune} to \textit{\imp{Physical} damage} and \key{Susceptible} to \textit{Patronus charm\comma{} Incandescent Damage}, nUnharmed=1, nBruised=3, nHurt=2, nInjured=0, nWounded=0, nMangled=1, block=1, dodge=3, defy=5, fortitude=2, imageStack=1, hasSkills = 1, skills = \skill{Covert}{5}
\skill{Flight}{2}
, hasAttacks = 1, attacks = \area{Subdue Prey}{sphere 3m around \name{}}{4}{0}{Sleep Status}{1 + Successes {\it (once per day)}}

\melee{Envelop}{4}{\minus{}1}{Incapacitated Status}{3 + Successes}, hasImage = 1, image = lethifold, description = Also known as a \key{Living Shroud}\comma{} a \name{} is a carnivorous and highly dangerous magical creature. 

Unlike other \imp{Cloakwraiths}\comma{} a \name{} appears to have no physical form\comma{} appearing instead as a gently floating and flapping shroud of black fabric\comma{} which crawls out of the shadows to envelop and then devour their victim. 

When a \name{} devours a victim\comma{} the only remaining sign of their once\minus{}physical existence is a slight thickening of the lethifold\comma{} and a handful of thread\minus{}like tendrils extending from beneath its body\comma{} otherwise the lethifold leaves no trace.}
\beast{name = Dementor, species = Cloakwraiths, mind = Ineffable, category = Abomination, rating = V, abilities = \ability{Auror of Futility}{An icy\comma{} soul\minus{}sapping aura extends around a \name{} for a radius of 3m. Beings within this radius suffer a 2d penalty to resist the \imp{Terrified} status effect\comma{} and a 1d penalty to all other actions.}

\ability{Hovering Menace}{A \imp{Dementor} may move freely in 3D without hindrance.}

\ability{Magical Resistance}{The \name{} gets a +3d bonus to all \imp{Resist} checks against all spells cast below a \levelFour{} level (except the \imp{Patronus})}

\ability{Paralyzed With Fear}{If a being becomes \imp{Terrified} of the \name{}\comma{} they remain rooted in place for one turn cycle (\imp{Incapacitated})\comma{} before they can attempt to flee.}, article = A, movement = \speeds{\speedrating{Flying}{16}}, fit =2, prs =3, vit =2, cha =0, dec =1, ins =4, int =3, wil =3, pcp =2, hasDamage = 1, damage =\key{Resistant} to \textit{\imp{Physical} damage} and \key{Susceptible} to \textit{Patronus charm\comma{} Incandescent Damage}, nUnharmed=8, nBruised=0, nHurt=0, nInjured=1, nWounded=0, nMangled=0, block=2, dodge=4, defy=5, fortitude=3, imageStack=0, hasSkills = 1, skills = \skill{Intimidation}{5}
\skill{Flight}{4}
, hasAttacks = 1, attacks = \area{Intensify Aura}{sphere 15m around \name{}}{6}{\minus{}2}{Cold damage / Terrified Status}{2 + Successes\comma{} distributed between effects\comma{} {\it (once per hour)}}

\melee{Clutching Claws}{6}{\minus{}2}{Cutting Damage}{1 + Successes}

\melee{Dementor’s Kiss}{8}{0}{Necrotic Damage}{5 + Successes (Requires target to be \imp{Incapacitated})}, hasLanguages = 1, languages = \speaks{Abyssal\comma{} Human Languages}, hasImage = 1, image = dementor, description = Perhaps the most feared (and certainly the most well\minus{}known) of the \imp{Cloakwraiths}\comma{} the \name{} appear as exceptionally tall\comma{} gaunt and necrotic humanoids benath their cloak\comma{} as their float effortlessly through space. 

They use an aura of ice and terror to incapacitate their foes\comma{} before delivering the infamous `dementor’s kiss’\comma{} an act which devours the very soul of a being\comma{} leaving them a lifeless husk. 

Strangely\comma{} despite being reviled and feared throughout history\comma{} \imp{Dementors} are the only \imp{Abomination} ever employed by the \imp{Ministry} – being used as guards/torturers for \imp{Azkaban Prison} for almost 200 years\comma{} until they sided with \imp{Lord Voldemort}.}
\beast{name = Shuagh, species = Cloakwraiths, mind = Ineffable, category = Abomination, rating = VI, abilities = \ability{Magical Resistance}{The \name{} gets a +3d bonus to all \imp{Resist} checks against all spells cast below a \levelFour{} level (except the \imp{Patronus})}

\ability{Puppets of War}{Any being which is reduced to the \imp{Critical Condtion} status within 50m of the \name{} instead heals 1 level of harm\comma{} and must begin fighting another target of the \name{}s’s choosing.}

\ability{Soul Mount}{Whilst mounted\comma{} the \name{} gains +2 damage to all attacks against unmounted foes and its movement speed is doubled. Any damage dealt to the mount is dealt to the \name{} }, article = A, movement = \speeds{\speedrating{Walking}{5}}, fit =5, prs =3, vit =3, cha =0, dec =4, ins =2, int =5, wil =6, pcp =2, hasDamage = 1, damage =\key{Resistant} to \textit{\imp{Physical} damage} and \key{Susceptible} to \textit{Patronus charm\comma{} Incandescent Damage}, nUnharmed=3, nBruised=3, nHurt=3, nInjured=0, nWounded=2, nMangled=1, block=5, dodge=1, defy=6, fortitude=6, imageStack=1, hasSkills = 1, skills = \skill{Intimidation}{7}
\skill{Skirmish}{5}
\skill{Spellcasting}{4}
\skill{Conviction}{4}
\skill{Strength}{3}
\skill{Speed}{2}
, hasAttacks = 1, attacks = \melee{Sword strike}{10}{\minus{}1}{Stabbing/Cutting damage}{2 + Successes}

\ability{Innate Power}{The \name{} may use their \imp{Spellcasting} ability to cast the \imp{Corrupt}\comma{} \imp{Compel} and \imp{Delude} spells}, hasLanguages = 1, languages = \speaks{All verbal languages}, hasImage = 1, image = shuagh, description = The \name{} are a form of mounted \imp{Cloakwraith}\comma{} never seen without their sickly and skeletal horses\comma{} which produce no sound as they gallop over devastated wasteland. 

Seemingly unique amonst the other \imp{Cloakwraiths}\comma{} the \name{} use tools and weapons to achieve their goals: the formenting of war\comma{} chaos and unbridled fury. The very presence of a \name{} is enough to anger even the most passive of individuals\comma{} and so the arrival of a \name{} was often seen as the precursor to bloodshed and infighting.}

}




\species{Creations}
{
	\key{Creations} are a class of magical creature which have\comma{} either through accident or design\comma{} been created by artificial means. This is usually the result of (highly illegal) experimental breeding\comma{} though direct magical manipulation of a species has been known to occur. 

Almost every beginner \imp{Transfiguration} student has\comma{} at some point\comma{} been responsible for the creation of a pseudo\minus{}\imp{Creation}\comma{} when they accidentaly morphed their pet rat into a rat\minus{}goblet\comma{} or some other ungodly accident. What sets the creatures below apart from these short\minus{}lived accidents is that \imp{Creations} are a viable species in their own right\comma{} with the ability to reproduce and continue their species’ existence\comma{} and they have often escaped from their creators’ control\comma{} and have made their home somewhere in the magical world.
}
{
\beast{name = Hidebehind, species = Creations, mind = Sapient, category = Sprite, rating = IV, abilities = \ability{Malleable Form}{A \name{} may use their \imp{Shapeshifting} ability to warp their form as if it were made of loose rubber. They cannot alter their appearance beyond this malleability.}

\ability{Hidden Reprieve}{Whenever a \name{} manages to successfully remain hidden for a full round\comma{} and does not reveal their location by attacking\comma{} they recover a \imp{Fortitude} point.}, article = A, movement = \speeds{\speedrating{Walking}{4}}, fit =3, prs =2, vit =1, cha =1, dec =3, ins =3, int =3, wil =2, pcp =2, hasDamage = 0, nUnharmed=3, nBruised=2, nHurt=1, nInjured=1, nWounded=0, nMangled=1, block=2, dodge=2, defy=2, fortitude=0, imageStack=0, hasSkills = 1, skills = \skill{Covert}{5}
\skill{Speed}{1}
\skill{Shapeshift}{1}
, hasAttacks = 1, attacks = \melee{Bludgeon}{5}{\minus{}2}{Bashing damage}{1 + Successes}

\extendedmelee{Rubbery flail}{5}{1}{Bashing damage}{1 + Successes}{5}

\melee{Bite}{7}{0}{Stabbing damage}{2 + Successes}

\ability{Phase}{As a \imp{Minor Action}\comma{} a \name{} may expend a \imp{Fortitude} point to turn invisible at the end of the round\comma{} lasting for 3 rounds (10 seconds). This invisibility is not broken by taking actions.}, hasLanguages = 1, languages = \understands{Human languages (poorly)}, hasImage = 1, image = hidebehind, description = The Hidebehind was accidentally created when illegal trader Phineas Fletcher attempted to import a trafficked Demiguise into America\comma{} with the goal of manufacturing Invisibility cloaks. The Demiguise escaped while on board the ship and bred with a stowaway ghoul\comma{} and the offspring escaped into the forests of Massachusetts.

As a result of their \imp{Demiguise} ancestry\comma{} they are able to turn invisible\comma{} though they prefer to use their ability to contort and warp their shape like rubber\comma{} such that they can hide behind almost any object (hence the name). Those few who have glimpsed a \name{} descibe them as a bipedal bear\minus{}like creature with fine silver hair. 

\name{}s are highly intelligent\comma{} but harbour a deep hatred and resentment for humankind\comma{} probably due to a genetic memory of the cruel circumstances that led to their creation. They will use their abilities to sneak up on unwary humanoids\comma{} knock them unconscious and then feast upon them.}
\beast{name = Blast\minus{}Ended Skrewt, species = Creations, mind = Non\minus{}sapient, category = Monstrosity, rating = IV, abilities = \ability{Resistant Armour}{The slimy\comma{} disgusting shell of the \name{} reflects almost all magic. Attacks must be specifically targeted at their fleshy underbellies to be effective.}, article = A, movement = \speeds{\speedrating{Walking}{6}\speedrating{Tunnelling}{0.25}}, fit =2, prs =2, vit =3, cha =0, dec =0, ins =1, int =1, wil =2, pcp =1, hasDamage = 1, damage =\key{Immune} to \textit{Any spell not \imp{called} to their underside}, nUnharmed=3, nBruised=2, nHurt=1, nInjured=1, nWounded=0, nMangled=1, block=5, dodge=3, defy=1, fortitude=2, imageStack=1, hasSkills = 1, skills = \skill{Speed}{3}
\skill{Strength}{2}
\skill{Tunnelling}{1}
, hasAttacks = 1, attacks = \areaConsequence{Fire Blast}{Cone 2m in length}{5}{1}{Fire Damage}{2 + Successes}{The \name{} is launced 5m in the opposite direction to the attack. Can expend a \imp{Fortitude} point to extend this to up to 15m\comma{} and immediately make another attack.}

\melee{Stinger (males only)}{6}{\minus{}1}{Poison Damage}{3 + Successes}

\meleeConsequence{Sucker (females only)}{6}{\minus{}1}{Necrotic Damage}{1 + Successes}{The \name{} heals for half the amount of damage dealt.}, hasImage = 1, image = screwt, description = The first clutch of \name{}s was hatched by \imp{Rubeus Hagrid} in 1994 from an illegal interspecies mixing of \imp{Manticores} and \imp{Firecrabs}. 

The result was a set of horrifying infants which resembled 3ft long deformed\comma{} shell\minus{}less lobsters\comma{} mixed with a scorpion. As they mature\comma{} they grow up to 10ft in length\comma{} and develop a highly resistant (and equally disgusting) carapace which reflects almost all magical effects. 

The males of the species possess wicked scorpion\minus{}like tails filled with a deadly venom\comma{} whilst the females have suckers they use to drain the blood of their victims. All adult members of the species were capable of generating immense blasts of fire from their rear end – either used as a propulsion system\comma{} or as an offensive weapon. 

Due to Hagrid’s well documented and extreme negligence\comma{} a number of \name{}s were able to escape into the \imp{Forbidden Forest}\comma{} and have begun breeding. Recent attempts to quell the infestation were unsuccessful.}

}




\speciesBeast{name = Demiguise, species = Demiguise, mind = Non\minus{}sapient, category = Beast, rating = III, abilities = \ability{Invisibility}{A \name{} may use their \imp{Elusion} skill to become completely invisible to the naked eye. By expending a \imp{Fortitude} point\comma{} they may also evade magical detection.}

\ability{Precognition}{A \name{} sees slightly into the future\comma{} and so always beats all other \imp{Reflex} rolls\comma{} and gains +4d to one \imp{Resist} roll per cycle. 

In the case of multiple creatures possessing this ability\comma{} they perform their own \imp{Reflex} roll amongst themselves to determine who goes first. }, article = A, movement = \speeds{\speedrating{Walking}{4}\speedrating{Climbing}{1}}, fit =2, prs =5, vit =1, cha =3, dec =0, ins =4, int =2, wil =1, pcp =3, hasDamage = 0, nUnharmed=1, nBruised=1, nHurt=0, nInjured=2, nWounded=1, nMangled=0, block=1, dodge=3, defy=1, fortitude=2, imageStack=0, hasSkills = 1, skills = \skill{Elusion}{7}
\skill{Covert}{5}
\skill{Climb}{2}
\skill{Speed}{1}
, hasAttacks = 1, attacks = \melee{Bite}{3}{\minus{}2}{Stabbing Damage}{1 + Successes}, hasImage = 1, image = demiguise, description = Demiguise are peaceful\comma{} herbivorous magical creatures native to the Far East. These highly magical creatures have been hunted to near\minus{}extinction by wizardkind due to their incredible ability to turn invisible – the pelt of a Demiguise can therefore be used in the construction of \imp{Cloaks of Invisibility}. 

Demiguise have precious little ability to defend themselves\comma{} though they do have \imp{Precognitive Sight} which allows them to see just a few seconds into the future\comma{} and hence they excel at evading threats\comma{} if not negating them.}



\species{Electric Elemental}
{
	Within the \imp{Elemental Planes} there can be found a single\comma{} enormous mountain\comma{} surrounded at all times by a roiling\comma{} black cloud filled with crackling energy: \key{Thundertop}. Lightning and thunder are everpresent in this hostile environment\comma{} and every surface is highly charged with static electricity – the foolish explorer who sets foot on the mountain of thunder without some rubber\minus{}soled boots is liable to have a {\it very} bad time. 

Within the crackling chaos and the booming crashes of this formiddable environment\comma{} reside a number of creatures who have learned to harness\comma{} channel and consume electrical energy\comma{} using it for their own end \minus{} \key{Electric Elementals}.
}
{
\beast{name = Raiju, species = Electric Elemental, mind = Non\minus{}sapient, category = Elemental, rating = III, abilities = \ability{Crackling Aura}{When threatened\comma{} a \name{} genertes an immense static field. Any being passing within 1m of the \name{}\comma{} or making a \imp{melee} attack this cycle\comma{} is considered \imp{susceptible} to \imp{Electric} damage until the end of the next turn cycle. }, article = A, movement = \speeds{\speedrating{Walking}{7}}, fit =5, prs =4, vit =2, cha =2, dec =0, ins =2, int =1, wil =2, pcp =1, hasDamage = 1, damage =\key{Immune} to \textit{Electric Damage}, nUnharmed=1, nBruised=2, nHurt=1, nInjured=0, nWounded=0, nMangled=0, block=2, dodge=3, defy=1, fortitude=1, imageStack=0, hasSkills = 1, skills = \skill{Speed}{4}
\skill{Intimidation}{2}
, hasAttacks = 1, attacks = \melee{Static Bite}{3}{\minus{}2}{Stabbing/Electric}{2 + Successes}

\ranged{Electric Arc}{5}{7}{0}{Electric}{1 + Successes}

\area{Rumbling Roar}{Sphere 3m in radius around \name{}}{4}{\minus{}1}{Pushback}{2 + Successes}, hasImage = 1, image = raiju, description = Appearing as perfectly normal (albeit electric blue) dogs\comma{} Raiju do not seem to be magical upon first glance. However\comma{} when angered\comma{} electrical energy arcs from every surface of their body\comma{} and their growl shakes the ground like distant thunder. 

After an ambitious magical experiment went awry in 15th century Japan\comma{} a number of Raiju were stranded in this realm and promptly began to multiply – they are now considered relatively common\comma{} and many Japanese witches and wizards have been known to train them as guard dogs and family pets.}
\beast{name = Nephelai, species = Electric Elemental, mind = Non\minus{}sapient, category = Elemental, rating = V, abilities = \ability{Nephelomorph}{A \name{} may alter their form at will\comma{} creating images and shapes within themselves\comma{} or shrinking or growing themselves to any size between mere centimetres and hundreds of metres tall.}

\ability{Cloud Body}{The body of a \name{} is formed of clouds\comma{} bound together by magic. Though they may not pass through solid objects like walls and doors\comma{} objects and beings may share the same space as the \name{}. Any living being which spends part of a \imp{cycle} within the \name{} takes \imp{level 5 electric damage}.}

\ability{Pressure Sense}{Without eyes or ears to speak of\comma{} a \name{} relies on their \imp{Inhuman Senses} to detect changes in pressure. As such\comma{} they are not fooled by immaterial illusions or simple invisibility.}, article = A, movement = \speeds{\speedrating{Swimming}{0.5}\speedrating{Flying}{12}}, fit =2, prs =4, vit =3, cha =1, dec =1, ins =2, int =5, wil =5, pcp =3, hasDamage = 1, damage =\key{Immune} to \textit{Electric damage and the Poisoned\comma{} Prone\comma{} Blinded\comma{} Deafened\comma{} Frostbitten and Trapped status effects.} and \key{Resistant} to \textit{Physical damage}, nUnharmed=7, nBruised=1, nHurt=1, nInjured=0, nWounded=0, nMangled=0, block=1, dodge=6, defy=5, fortitude=3, imageStack=0, hasSkills = 1, skills = \skill{Flight}{3}
\skill{Inhuman Senses}{3}
\skill{Intimidation}{2}
\skill{Swim}{1}
, hasAttacks = 1, attacks = \ranged{Lightning Bolt}{100m}{3}{\minus{}1}{5 + Successes}

\ability{Chain Lightning}{By expending a \imp{fortitude} point\comma{} the \name{} makes successive \imp{Lightning Bolt} attacks against targets in range\comma{} with each attack originating from the previous target. The chain ends when an attack fails\comma{} or has its \imp{power} reduced to zero. }

\melee{Churning Vortex}{10}{0}{Cold/Bashing}{1 + Successes\comma{} against all targets currently inside the \name{} }, hasImage = 1, image = Nephelai, description = When one gazes upon the tumultuous\comma{} churning\comma{} sparking clouds of \imp{Thundertop}\comma{} it is easy to imagine that you are witnessing some immense\comma{} living creature. This is not as far wrong as you might wish. 

The \name{} are spirits of energy and thunder\comma{} their physical form (if you can really call it `physical’) composed of violent\comma{} turbulent stormclouds\comma{} bent into the twisted form of a humanoid. As spirits of the storm\comma{} they are almost mindless in their destructive wrath\comma{} though they are neither cruel or evil. 

Wielding the ability to generate immense bolts of lightning\comma{} and twist their near\minus{}incorporeal form into near any shape imaginable\comma{} the \name{} are agents of immense chaos when they escape from their natural habitat.}

}




\species{Elves}
{
	Modern muggle culture frequently imagines Elves to be superhuman\comma{} immortal and otherworldly creatures – outwardly appearing as impossibly beautiful humans and wielding immense\comma{} primal magic. This is primarly due to the influence of the muggle writer Tolkein (who was himself a squib). In reality\comma{} elves are much closer to those envisaged in medieval Germanic mythology – small\comma{} impish tricksters. 

Though they have a love for tricks and fun\comma{} the most common form of elf throughout history\comma{} the {\it Br{\'u}nb{\'a}su}\comma{} or {\it Common Brownie} often helped humans by performing small chores for them. In one of the more shameful acts of wizarding history\comma{} the Br{\'u}nb{\'a}su were then systematically enslaved and brainwashed into eternal servitude\comma{} leading to the creation of the race of House\minus{}Elves. Witnessing this horrific act of human cruelty\comma{} many of the other elf species retreated into relative obscurity.

\ability{Diminutive Frame}{Almost all species of Elf are tiny in stature – rarely reaching more than 3ft in height – with skinny and spindly arms and legs. They rely on magic for physical acts that their size denies them. }

\ability{Elfin Magics}{Elf magic is unlike any that is understood by humans\comma{} who often look down on it as inferior. Some scholars\comma{} however\comma{} have hypothesised that elfin magic is in fact far superior to wizard magic. It is only due to the elfin spirit and its dislike of organisation and study (and probably helped by wizard oppression) that they have not learned to harness it to its full potential. The most prominent quirk of elfin magic is their ability to ignore even the strongest magical wards and boundaries.}  

\ability{Fond of Trickery}{All elves (except perhaps the brainwashed House Elves) are fond of trickery and fun. They play endless pranks on one another\comma{} delighting in causing small amounts of chaos. Even those who are dedicated to cleaning and tidying play the occassional trick: muggles notice this in the form of missing socks\comma{} or keys never quite being where they were left.}

\ability{Holding Court}{Elven society is chaotic and unorganised. They rarely recognise a leader\comma{} though they sometimes assemble themselves into bands and groups for the purpose of safety. This all changes when one of the Elven Princes summons them to their Seelie Court. Every free\minus{}elf belongs to one of these courts\comma{} which is presided over by one of the more powerful Hulduf{\'o}lk. Court is only summoned in times of great emergency\comma{} such as conflict with another court\comma{} or when a great external threat is detected.}
}
{
\beast{name = House\minus{}Elf, species = Elves, mind = Sapient, category = Imp, rating = I, abilities = \ability{Elf Magic}{Elf\minus{}Magic is almost completely alien to wizardkind. \imp{Abjure} spells do not dispel their magic\comma{} and wards to prevent their  inherent ability to teleport do not work.}

\ability{Teleporting Step}{\name{} may use their natural magical ability to \imp{Teleport} their movement\comma{} rather than walk it. They can use this to escape from non\minus{}magical bindings holding them in place. }, article = A, movement = \speeds{\speedrating{Walking}{2}}, fit =1, prs =4, vit =1, cha =1, dec =0, ins =1, int =3, wil =2, pcp =3, hasDamage = 1, damage =\key{Susceptible} to \textit{Crushing damage}, nUnharmed=2, nBruised=2, nHurt=0, nInjured=0, nWounded=0, nMangled=0, block=1, dodge=4, defy=1, fortitude=1, imageStack=0, hasSkills = 1, skills = \skill{Spellcasting}{3}
\skill{Covert}{2}
, hasAttacks = 1, attacks = \ability{Elf Magic}{A \name{} can use their \imp{Spellcasting} ability to cast the \imp{Apparate}\comma{} \imp{Force}\comma{} \imp{Move}\comma{} \imp{Repair}\comma{} \imp{Animate} and \imp{Shield} spells}

\melee{Kitchen Knife}{4}{\minus{}1}{Stabbing/Cutting damage}{1 + Successes}, hasLanguages = 1, languages = \speaks{Human languages}, hasImage = 1, image = brownie, description = A House\minus{}Elf is a small impish humanoid creature which has been bound\comma{} or inherited a binding\comma{} to a given master. House\minus{}Elves are incredibly devoted to these masters\comma{} and will obey every order given to them. Typically treated as household servants\comma{} their magic is dedicated to the upkeep of the home and\comma{} on incredibly rare occasions\comma{} the repelling of intruders.}
\beast{name = Br{\'u}nb{\'a}su, species = Elves, mind = Sapient, category = Imp, rating = II, abilities = \ability{Elf Magic}{Elf\minus{}Magic is almost completely alien to wizardkind. \imp{Abjure} spells do not dispel their magic\comma{} and wards to prevent their  inherent ability to teleport do not work.}

\ability{Teleporting Step}{\name{} may use their natural magical ability to \imp{Teleport} their movement\comma{} rather than walk it. They can use this to escape from non\minus{}magical bindings holding them in place. }

\ability{Free Spirit}{Free from their long enslavement\comma{} the \name{} are in no hurry to lose their free will once again: they gain +4d to any check to compel\comma{} control\comma{} charm or otherwise remove their ability to think for themselves.}, article = A, movement = \speeds{\speedrating{Walking}{2}}, fit =2, prs =4, vit =2, cha =3, dec =1, ins =2, int =3, wil =4, pcp =3, hasDamage = 1, damage =\key{Susceptible} to \textit{Crushing damage}, nUnharmed=2, nBruised=2, nHurt=1, nInjured=0, nWounded=0, nMangled=0, block=2, dodge=4, defy=3, fortitude=2, imageStack=0, hasSkills = 1, skills = \skill{Spellcasting}{4}
\skill{Covert}{3}
, hasAttacks = 1, attacks = \ability{Elf Magic}{A \name{} can use their \imp{Spellcasting} ability to cast the \imp{Apparate}\comma{} \imp{Force}\comma{} \imp{Move}\comma{} \imp{Repair}\comma{} \imp{Animate}\comma{} \imp{Heal}\comma{} \imp{Slumber}\comma{} \imp{Compel} and \imp{Shield} spells}, hasLanguages = 1, languages = \speaks{Human languages\comma{} Fey}, hasImage = 1, image = dobby, description = It is said that the \name{} (or `Brownies’) were the original species that were enslaved and magically altered over generations to become the House\minus{}Elves. Long considered extinct\comma{} the freeing of Dobby the House\minus{}Elf\comma{} the first Free\minus{}Elf in 1000 years\comma{} has led to a resurgence in the \name{} as a newly re\minus{}recognised species. 

Though rejecting their former enslavement\comma{} the \name{} retain a love of order and cleanliness\comma{} and will often do chores for a household – on the condition of respect and equitable payment. They are easily offended\comma{} and if they feel denigrated or slighted\comma{} they will wreck the place\comma{} and will never return.}
\beast{name = \ellengaest, species = Elves, mind = Sapient, category = Imp, rating = II, abilities = \ability{Elf Magic}{Elf\minus{}Magic is almost completely alien to wizardkind. \imp{Abjure} spells do not dispel their magic\comma{} and wards to prevent their  inherent ability to teleport do not work.}  \ability{Teleporting Step}{\name{} may use their natural magical ability to \imp{Teleport} their movement\comma{} rather than walk it. They can use this to escape from non\minus{}magical bindings holding them in place. }, article = A, movement = \speeds{\speedrating{Walking}{3}}, fit =2, prs =5, vit =1, cha =2, dec =4, ins =3, int =2, wil =3, pcp =3, hasDamage = 1, damage =\key{Susceptible} to \textit{Crushing damage}, nUnharmed=2, nBruised=1, nHurt=1, nInjured=1, nWounded=0, nMangled=0, block=1, dodge=4, defy=2, fortitude=2, imageStack=0, hasSkills = 1, skills = \skill{Spellcasting}{4}
\skill{Covert}{3}
, hasAttacks = 1, attacks = \ability{Elf Magic}{A \name{} can use their \imp{Spellcasting} ability to cast the \imp{Apparate}\comma{} \imp{Conceal}\comma{} \imp{Distract}\comma{} \imp{Mirage}\comma{} \imp{Move}\comma{} \imp{Delude}\comma{} \imp{Bypass} and \imp{Corrupt} spells}  

\ability{Pickpocket}{When making contact with a being\comma{} a \name{} can use a \imp{Major Action} to steal a random item from the \imp{Inventory} of the target.}, hasLanguages = 1, languages = \speaks{Human languages\comma{} Fey}, hasImage = 1, image = imp, description = The archetypal impish prankster\comma{} the \name{}\comma{} also known as a `Puck’\comma{} or a `True Imp’ competes only with the Poltergeist for the mischief\minus{}maker\apos{}s crown. They adore slapstick comedy and petty theft\comma{} often building themselves a nest out of their stolen objects. 

Visually\comma{} an \name{} appears to be a small\comma{} flightless fairy; a slender humanoid\comma{} approximately a foot tall. Their mouth is filled with a disturbing number of teeth\comma{} but they’re too small to harm a human\comma{} and are used to supply their mostly\minus{}insect based diet.}
\beast{name = \dunaelf, species = Elves, mind = Sapient, category = Imp, rating = III, abilities = \ability{Elf Magic}{Elf\minus{}Magic is almost completely alien to wizardkind. \imp{Abjure} spells do not dispel their magic\comma{} and wards to prevent their  inherent ability to teleport do not work.}, article = A, movement = \speeds{\speedrating{Walking}{5}\speedrating{Climbing}{1}}, fit =3, prs =4, vit =2, cha =1, dec =2, ins =2, int =2, wil =3, pcp =4, hasDamage = 1, damage =\key{Susceptible} to \textit{Crushing damage}, nUnharmed=2, nBruised=2, nHurt=2, nInjured=0, nWounded=0, nMangled=1, block=2, dodge=5, defy=2, fortitude=3, imageStack=1, hasSkills = 1, skills = \skill{Covert}{5}
\skill{Marksmanship}{4}
\skill{Elusion}{4}
\skill{Climb}{2}
\skill{Speed}{2}
, hasAttacks = 1, attacks = \ability{Elfin Strike}{A \name{} performs a \imp{Precision (Elusion)} check against the highest passive \imp{Perception} of their foes. On a success\comma{} the \name{} uses a single major action to \imp{Move}\comma{} make an \imp{Elfshot} and then \imp{Phase}\comma{} on a failure\comma{} their turn ends.}

\ability{Phase}{A \name{} may use a \imp{MInor} action to use their \imp{Elusion} ability to become invisible}

\rangedConsequence{Elfshot}{25}{7}{\minus{}1}{Psychic/Stabbing}{1 + Successes}{For each previous \imp{Elfshot} a target has suffered damage from\comma{} the \name{} gets one additional auto\minus{}success.}, hasLanguages = 1, languages = \speaks{Human languages\comma{} Fey}, hasImage = 1, image = dunaelf, description = The warriors and guardians of the Elves\comma{} the \name{} make great use of their nimble nature and their ability to turn invisible\comma{} more than making up for their diminutive stature. The \name{} also wield magical bows which inflict agonising pain on their targets.}
\beast{name = Vaettir, species = Elves, mind = Sapient, category = Imp, rating = III, abilities = \ability{Elf Magic}{Elf\minus{}Magic is almost completely alien to wizardkind. \imp{Abjure} spells do not dispel their magic\comma{} and wards to prevent their  inherent ability to teleport do not work.}, article = A, movement = \speeds{\speedrating{Walking}{2}\speedrating{Flying}{4}}, fit =1, prs =2, vit =1, cha =2, dec =3, ins =3, int =4, wil =4, pcp =3, hasDamage = 1, damage =\key{Susceptible} to \textit{Crushing damage}, nUnharmed=2, nBruised=2, nHurt=1, nInjured=0, nWounded=1, nMangled=0, block=1, dodge=4, defy=3, fortitude=2, imageStack=0, hasSkills = 1, skills = \skill{Spellcasting}{5}
\skill{Flight}{1}
, hasAttacks = 1, attacks = \ability{Elf Magic}{A \name{} can use their \imp{Spellcasting} ability to cast all spells from the \imp{Elemental} discipline\comma{} as well as the \imp{Move}\comma{} \imp{Apparate}\comma{} \imp{Stun} and \imp{Shield} spells.}

\ability{Primal Feedback}{As a \imp{Defensive Action}\comma{} a \name{} may turn all the successes rolled on an \imp{Elemental} spellcasting effort into \imp{Catastrophes}. }, hasLanguages = 1, languages = \speaks{Human languages\comma{} Fey}, hasImage = 1, image = vaettir, description = The least human\minus{}like of the elves\comma{} the \name{} are a species of elves which have an affinity for the primal elements\comma{} and the associated magics. 

Though capable of wielding all of the elements\comma{} many \name{} find themselves drawn to one element in particular. This reflects in their outwards appearance\comma{} which slowly begins to shift and change. A fire\minus{}bound \name{} grows a fiery red beard and orange eyes\comma{} whilst a water\minus{}bound \name{}\apos{}s skin turns blue and appears permanently damp. Current arcane knowledge has found no additional abilities granted with this change\comma{} so it is unknown if there is any particular reason for this `bonding’ process to occur.}

}




\species{Giantkin}
{
	Though the \imp{True Giants} are the most prolific of the various \imp{Giantkin} around\comma{} enough to establish themselves as an independent society\comma{} there are other related creatures scattered around in isolated pockets across the globe. Through various infighting and wars with external agressors\comma{} their populations have been depleted enough that they are almost constantly nearing the edge of extinction\comma{} and most have reverted to a nomadic existence\comma{} with small groups and families searching for a safe place to call home.
}
{
\beast{name = Cyclopes, species = Giantkin, mind = Sapient, category = Gigantoid, rating = V, abilities = \ability{Tool\minus{}Users}{The \name{} are masters of tools and weapons of all shapes and sizes\comma{} and often have many dozens of weapons to hand\comma{} depending on wear and how they are located. They use their \imp{Craft} ability to determine the relevant dice pool.}

\ability{Armoured}{Unless it is caught totally unawares\comma{} a cyclops will have defensive armour and gadgets to protect it\comma{} granting +2 power to all successful \imp{Block} rolls.}

\ability{Magic Resistance}{Gain +2 power on all successful \imp{Resist} checks against magical effects. \imp{Resist} checks against magic do not incur drain.}, article = A, movement = \speeds{\speedrating{Walking}{5}}, fit =6, prs =3, vit =4, cha =3, dec =2, ins =3, int =5, wil =5, pcp =3, hasDamage = 1, damage =\key{Immune} to \textit{Electric Damage}, nUnharmed=5, nBruised=2, nHurt=2, nInjured=2, nWounded=0, nMangled=3, block=6, dodge=1, defy=5, fortitude=5, imageStack=1, hasSkills = 1, skills = \skill{Imbue}{6}
\skill{Craft}{6}
\skill{Arcane}{5}
\skill{Strength}{5}
\skill{Technology}{4}
\skill{Science}{4}
\skill{Speed}{2}
, hasAttacks = 1, attacks = \melee{Unarmed Strike}{11}{\minus{}2}{Bashing damage}{1 + successes}

\ranged{Zeus’ Spear}{100m}{3}{\minus{}1}{Electric damage}{7 + Successes}, hasLanguages = 1, languages = \speaks{Giant\comma{} Human Languages}, hasImage = 1, image = cyclops, description = The distinguishing factor of \imp{Cyclopes} in muggle mythology and popular culture is their single\comma{} central eye\comma{} beyond which they differ only very slightly from their True\minus{}Giant cousins\comma{} and remain dumb and brutish.  

Whilst the single\comma{} central eye is indeed true\comma{} the Cyclopes are fearsomely intelligent\comma{} and genius crafters\comma{} being experts in the manipulation and generation of electric currents. Of all the magical creatures in existence\comma{} only the Cyclopes have kept pace with (and often exceeded) muggles in their technological prowess. 

In aeons past\comma{} they were said to manufacture mighty magical weapons for mythical warriors\comma{} capable of bringing down the very gods themselves. Nowadays\comma{} a cyclops is  most likely to be found alone in lightning\minus{}ravaged mountaintop abodes\comma{} designing traps and defenses to keep their foes at bay.}
\beast{name = Jotun, species = Giantkin, mind = Sapient, category = Gigantoid, rating = V, abilities = \ability{Magic Resistance}{Gain +2 power on all successful \imp{Resist} checks against magical effects. \imp{Resist} checks against magic do not incur drain.}, article = A, movement = \speeds{\speedrating{Walking}{5}}, fit =5, prs =3, vit =7, cha =1, dec =1, ins =1, int =2, wil =3, pcp =4, hasDamage = 1, damage =\key{Immune} to \textit{Cold Damage}, nUnharmed=4, nBruised=2, nHurt=2, nInjured=1, nWounded=1, nMangled=0, block=6, dodge=2, defy=6, fortitude=4, imageStack=0, hasSkills = 1, skills = \skill{Survival}{6}
\skill{Strength}{6}
\skill{Speed}{2}
, hasAttacks = 1, attacks = \melee{Greatclub}{12}{\minus{}1}{Bashing Damage}{1 + Successes}

\ranged{Iceball}{25}{3}{0}{Bashing/Cold Damage}{5+Successes}, hasLanguages = 1, languages = \speaks{Giant}, hasImage = 1, image = jotun, description = Whilst many of the \imp{True Giants} have been pushed northwards by the expansion of muggle civilsation\comma{} the \name{}’s have pushed this to the extreme\comma{} and have survived in the frigid climate of the most Northern and Southern parts of the globe. Their gigantoid resilience has allowed them to adapt to this inhospitable climate\comma{} and they have become rugged survivors.}
\beast{name = Humbaba, species = Giantkin, mind = Sapient, category = Gigantoid, rating = VI, abilities = \ability{Magic Resistance}{Gain +2 power on all successful \imp{Resist} checks against magical effects. \imp{Resist} checks against magic do not incur drain.}

\ability{Corrupting Power}{The \name{} may use its \imp{Spellcasting} ability to cast the \imp{Burn}\comma{} \imp{Corrupt} and \imp{Infect} spells.}, article = A, movement = \speeds{\speedrating{Walking}{6}\speedrating{Tunnelling}{0.25}}, fit =5, prs =4, vit =4, cha =0, dec =0, ins =1, int =2, wil =2, pcp =2, hasDamage = 1, damage =\key{Immune} to \textit{Fire Damage} and \key{Resistant} to \textit{Cold Damage}, nUnharmed=3, nBruised=3, nHurt=3, nInjured=3, nWounded=3, nMangled=0, block=6, dodge=3, defy=6, fortitude=4, imageStack=1, hasSkills = 1, skills = \skill{Strength}{7}
\skill{Speed}{3}
\skill{Spellcasting}{3}
\skill{Tunnelling}{1}
, hasAttacks = 1, attacks = \melee{Deformed Claws}{12}{\minus{}2}{Cutting damage}{3 + successes}

\area{Fire Breath}{3m cone originating from \name{}}{8}{0}{Fire Damage}{2 + Successes}, hasLanguages = 1, languages = \speaks{Abyssal\comma{} Giant}, hasImage = 1, image = Humbaba, description = When the giantkin began their exodus from the rest of Giant society\comma{} some fled deep underground\comma{} into the myriad network of lava tunnels and magma pools that lie underground. Deep in the bones of the Earth\comma{} they studied ancient necromancy\comma{} and harnessed the grotesque abominations they found squirming in the darkness. 

Centuries of living with this corrupting influence has warped the \imp{Humbaba} into the most monstrous of the Giantkin. Driven almost entirely mad by the sulphurous fumes and the wailing of their slaves\comma{} Humbaba live only to further their own power\comma{} enslave those weaker than them\comma{} and exert their insane dominance over others.

Thankfully\comma{} history records only one incident of a \name{} breaking free of their self\minus{}imposed prison\comma{} in around 1000BCE. Surprisingly\comma{} this foul creature was slain by a great muggle hero\minus{}king.}
\beast{name = Basajaun, species = Giantkin, mind = Sapient, category = Gigantoid, rating = IV, abilities = \ability{Magic Resistance}{Gain +2 power on all successful \imp{Resist} checks against magical effects. \imp{Resist} checks against magic do not incur drain.}, article = A, movement = \speeds{\speedrating{Walking}{6}}, fit =4, prs =2, vit =3, cha =3, dec =0, ins =4, int =2, wil =3, pcp =5, hasDamage = 0, nUnharmed=3, nBruised=3, nHurt=2, nInjured=0, nWounded=0, nMangled=0, block=4, dodge=1, defy=4, fortitude=2, imageStack=0, hasSkills = 1, skills = \skill{Strength}{4}
\skill{Speed}{3}
, hasAttacks = 1, attacks = \melee{Staffstrike}{8}{\minus{}1}{Bashing Damage}{1 + Successes}, hasLanguages = 1, languages = \speaks{Giant}, hasImage = 1, image = basajaun, description = Now found only in scattered pockets in the most isolated forests in the world\comma{} the \name{} are a peaceful and kindly race of Giantkin\comma{} who have taken the forests and woodlands as their hiding place. 

Acting as guardians of the forests\comma{} they are devoted to the protection and nurturing of wildlife and the plants that they hold dear. A single \name{} often personally attends to thousands upon thousands of acres of woodland\comma{} and will often know the names of all the creatures who reside there.

Their fur\minus{}covered appearance\comma{} lack of agression\comma{} short stature (for giants) and general kindliness has meant that \name{} are rarely hunted or even feared by humans. Muggles have developed all sorts of strange myths around the \name{}\comma{} giving them the name `Bigfoot’\comma{} though wizards have puzzled over the fact that their feet are not particularly large or out of proportion to their bodies.}

}








\species{Light Elemental}
{
	The \key{Radiant Gardens} are one of the realms beyond our own which resonates strongly with one of the primal magical elements – in this case the element of \key{Light}. The beings native to this particular corner of the multiverse are therefore known as \imp{Light Elementals}. 

The \imp{Radiant Gardens} are an almost heaven\minus{}like domain\comma{} filled at all times with a diffuse\comma{} golden glow\comma{} interspersed with rainbows containing more colours than the human mind can conceive of. \imp{Light Elementals}\comma{} having formed from this incandescant space\comma{} are all therefore naturally able to manipulate and channel radiance in all its forms\comma{} and abhore darkness and shadows. 

Just as no two rainbows are perfectly alike\comma{} so too are \imp{Light Elements} unique and distinct creatures\comma{} with wildly varying morphology and characteristics – some appear as perfectly normal solid creatures\comma{} whilst others seem to be made up of pure\comma{} distilled light.
}
{
\beast{name = Solon, species = Light Elemental, mind = Non\minus{}Sapient, category = Elemental, rating = III, abilities = \ability{Floating}{The \name{} naturally floats in the air using its \imp{Flight} ability.}, article = A, movement = \speeds{\speedrating{Flying}{8}}, fit =2, prs =3, vit =3, cha =2, dec =0, ins =1, int =1, wil =3, pcp =2, hasDamage = 1, damage =\key{Immune} to \textit{Incandescent}\comma{} \key{Resistant} to \textit{Fire} and \key{Susceptible} to \textit{Bludgeoning}, nUnharmed=1, nBruised=1, nHurt=1, nInjured=1, nWounded=0, nMangled=0, block=4, dodge=1, defy=2, fortitude=1, imageStack=0, hasSkills = 1, skills = \skill{Spellcasting}{3}
\skill{Flight}{2}
, hasAttacks = 1, attacks = \ability{Crystal Shards}{Whenever the \name{} takes physical damage\comma{} they may choose to take an additional level of harm to deflect fragments of their body into a creatue within 2m\comma{} dealing level 5 \imp{Stabbing} damage.}

\ability{Elemental Magic}{The solon may use its \imp{Spellcasting} ability to cast spells. It knows the \imp{Illuminate}\comma{} \imp{Sense} and \imp{Move} spells.}, hasLanguages = 1, languages = \speaks{Empyrean}, hasImage = 1, image = solon, description = \name{}s are lesser elementals hailing from the Radiant Gardens\comma{} the Elemental plane of light. They are crystalline creatures\comma{} and glow with an inner radiance which shifts and refracts through their bodies as they move.}
\beast{name = Phlogiston, species = Light Elemental, mind = Ineffable, category = Elemental, rating = IV, abilities = \ability{Phaseshift}{The \name{} may use its \imp{Shapechanging} ability to morph into gaseous form\comma{} where it may use its flight speed\comma{} but loses the ability to use \imp{Evaporating Blast}. It may use this ability again to morph back into liquid form.}, article = A, movement = \speeds{\speedrating{Walking}{2}\speedrating{Flying}{40}}, fit =1, prs =5, vit =3, cha =4, dec =0, ins =3, int =5, wil =3, pcp =2, hasDamage = 1, damage =\key{Immune} to \textit{Incandescent} and \key{Resistant} to \textit{All physical damage}, nUnharmed=6, nBruised=0, nHurt=0, nInjured=0, nWounded=1, nMangled=0, block=0, dodge=4, defy=3, fortitude=4, imageStack=1, hasSkills = 1, skills = \skill{Flight}{10}
\skill{Shapechanging}{5}
\skill{Spellcasting}{5}
, hasAttacks = 1, attacks = \ranged{Evaporating Blast}{100}{6}{\minus{}1}{Incandescent}{1 + Successes}

\ability{Elemental Magic}{The \name{} may use its \imp{Spellcasting} ability to cast spells. It knows the \imp{Illuminate}\comma{} \imp{Fire} and \imp{Move} spells.}, hasLanguages = 1, languages = \speaks{Empyrean}, hasImage = 1, image = phlogiston, description = A highly unusual magical creature\comma{} originating from the Plane of Light\comma{} but subsequently spreading to all corners of reality\comma{} A phlogiston is a creature of liquid light\comma{} a sentient\comma{} glowing pool of power and warmth. 

Though normally calm and blissful creatures\comma{} when angered or afraid\comma{} the phlogiston can shift into a gaseous form\comma{} or evaporate parts of its form entirely into powerful blasts of light.}
\beast{name = Chalkydri, species = Light Elemental, mind = Non\minus{}sapient, category = Elemental, rating = V, abilities = \ability{Winged Herald}{The \name{} may use its \imp{Flight} skill to take to the skies\comma{} gaining a flying speed of 30m per round}

\ability{Spectral Mesmer}{Any creature which comes within 20m and can see the \name{} must gain at least one success on a DV 8 check (recommended \imp{Willpower (Conviction)} to tear their eyes away from the hypnotising lights emanating from the \name{}\comma{} on a failure\comma{} they must spend their entire turn doing nothing but staring at the \name{}.}

\ability{Light Siphon}{At the end of every round\comma{} if the \name{} can see a source of light\comma{} they regenerate health equal to a DV 4 \imp{Regeneration} check}, article = A, movement = \speeds{\speedrating{Walking}{5}\speedrating{Flying}{20}}, fit =4, prs =2, vit =4, cha =3, dec =0, ins =2, int =3, wil =4, pcp =3, hasDamage = 1, damage =\key{Immune} to \textit{Incandescent\comma{} Fire}, nUnharmed=3, nBruised=3, nHurt=3, nInjured=2, nWounded=1, nMangled=0, block=3, dodge=6, defy=4, fortitude=5, imageStack=1, hasSkills = 1, skills = \skill{Spellcasting}{7}
\skill{Flight}{5}
\skill{Regeneration}{4}
\skill{Speed}{2}
, hasAttacks = 1, attacks = \melee{Reptilian Jaws}{10}{0}{Stabbing}{3 + Successes}

\melee{Jagged Tail}{10}{2}{Bashing}{5 + Successes (Reach 5m)}

\area{Downdraft}{circle 10m radius below current flying position)}{8}{0}{Prone}{1 + Successes}

\ability{Bringer of Dawn}{The \name{} may use its \imp{Spellcasting} ability to cast the \imp{Illuminate} spell.}, hasLanguages = 1, languages = \speaks{Empyrean}, hasImage = 1, image = chalkydri, description = Also known as the `Heralds of Dawn’\comma{} these copper\minus{}skinned\comma{} winged serpent\minus{}like creatures were often mistaken for a species of \imp{Dragon} throughout history\comma{} and it is only recently that their true origins have been determined. 

Possessing the face and tail of a crocodile\comma{} but the body of a lion and rows upon rows of rainbow\minus{}coloured wings (up to 14 pairs on the oldest known specimen)\comma{} these creatures typically reside within the coronosphere of a sun\comma{} or near some other cosmically powerful source of light\comma{} descending planetside only for a few hours per year\comma{} and even then\comma{} they appear only during the first few moments of dawn. 

When the sunlight hits their copper\minus{}bronze skin\comma{} it splits off into a mesmerising rainbow display\comma{} and causes their entire body to hum with a resonance like a chorus of angels. Legend holds that it is this song which brings the dawn\comma{} rather than the other way around.

Though they do not seem particularly intelligent\comma{} and are currently classified as `beasts’ by the \imp{Ministry}\comma{} this does not mean that they are savage – in fact\comma{} the \name{} are often kind and gentle creatures\comma{} the few times that they have been recorded as coming into direct conflict with humans it was eventually discovered that the humans were threatening some other life form with extinction\comma{} drawing the ire of the \name{}.}

}
































\speciesBeast{name = Whomping Willow, species = Trees, mind = Sapient, category = Flora, rating = IV, abilities = \ability{Photosynthesis}{Whilst in direct sunlight\comma{} a \name{} uses rolls a DV 5 \imp{Regeneration} check at the end of each combat cycle\comma{} restoring health equal to 1 + successes}

\ability{Multi\minus{}armed}{A \name{} has 4 arms used for attacks. A \imp{called shot} against an arm forces it to perform a \imp{Vitality} check\comma{} with a DV equal to 5 + the damage dealt. On a failure\comma{} the arm is detached and cannot be used. Arms regrow after 1 week.}, article = A, movement = \speeds{}, fit =3, prs =1, vit =6, cha =0, dec =1, ins =1, int =3, wil =2, pcp =3, hasDamage = 1, damage =\key{Immune} to \textit{The \imp{Asleep\comma{} Blinded\comma{} Charmed\comma{} Confused\comma{} Deaf\comma{} Prone} and \imp{Unconscious} status effects}\comma{} \key{Resistant} to \textit{All \imp{physical} damage} and \key{Susceptible} to \textit{Fire}, nUnharmed=6, nBruised=2, nHurt=0, nInjured=0, nWounded=0, nMangled=0, block=6, dodge=0, defy=6, fortitude=2, imageStack=0, hasSkills = 1, skills = \skill{Strength}{6}
\skill{Logic}{5}
\skill{Nature}{3}
\skill{Regeneration}{2}
, hasAttacks = 1, attacks = \ability{Flail}{The \name{} makes a \imp{Whomp} attack using each of its remaining arms}

\extendedmeleeConsequence{Whomp}{3}{\minus{}1}{Bludgeoning}{4 + Successes}{Any target taking damage from this attack is knocked prone}{6}, hasImage = 1, image = whomp, description = A \name{} is an extremely rare species of articulated plant life. Though their thoughts are utterly alien and untouchable by wizardkind\comma{} it has long been recognised that these incredible plants are in fact sentient creatures – capable of independent\comma{} imaginative thought and complex processing of information. 

Further study of this unusual form of life has been hampered\comma{} however\comma{} by the fact that \name{} are\comma{} without exception\comma{} {\it incredibly} violent– even saplings of the species have been known to break an arm. 

The sole saving grace of a \name{} is that they are incapable of uprooting themselves – so simply running away is always a viable option.}




