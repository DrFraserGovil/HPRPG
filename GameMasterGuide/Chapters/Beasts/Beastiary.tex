\chapter{Bestiary}


In this section a number of different creatures are presented for the GM's use in building encounters. These creatures come with a set of basic canonical background information, as well as a `statblock', which contains the necessary statistics for these creatures to perform checks, and ultimately engage in combat and other character interactions.


\section{Beast Abilities}	

Whilst all \imp{Beasts} share the same 9 base \imp{Aspects} as player characters, and many of the same \imp{Abilities}. However for streamlining reasons, the number of \imp{Abilities} each individual beast has is more restricted than a player character: if an \imp{Ability} is not mentioned in the provided statblock, you may assume it has a value of \emptyCape. 

Though they have far fewer proficiencies, Beasts do have access to all of the same \imp{Abilities} as the player characters - though actions such as \imp{Imbue} and \imp{Craft} are unlikely to come up except in the most unusual of circumstances! 

In addition to the 30 base \imp{Abilities}, some beasts have additional abilities determined by their non-human and, in some cases, magical, physiology:
\newcommand\abilityRow[2]
{

	\parbox[t]{2 cm}{\key{#1}} & \parbox[t]{6.8 cm}{\raggedright #2} \\
}

\newcommand\abilityTable[1]
{
	\small
	\begin{center}
		\begin{rndtable}{r l}
		\bf Ability	& \bf Description \\
		#1
		\end{rndtable}
	\end{center}
}

\abilityTable
{
	\abilityRow{Climb}{Many beings have the ability to climb trees, and adhere to solid surfaces. A non-zero rating grants a being an inherent climb speed - the higher the rating, the faster they can climb.}
	\abilityRow{Command}{Some creatures command their lessers and may order them to do their bidding - a higher rating indicates the level of control they have over their forces.}
	\abilityRow{Elusion}{Elusion is the natural camouflaging ability of a being - morphing into the background, changing colour and even turning invisible.}
	\abilityRow{Flight}{A creature with the flying ability may defy gravity, either with wings, or innate magical levitation. A higher rating means faster flight and more elaborate maneouvers.}
	\abilityRow{Regeneration}{This ability allows a creature to heal themselves rapidly as their physical form regenerates.}
	\abilityRow{Inhuman Senses}{Many creatures have senses beyond those that humans have: the ability to sense tremors in the ground, see in the dark, as well as more arcane abilities such as the ability to detect magic.}
	\abilityRow{Shapechange}{A creature with this ability may alter their shape and form - a higher rating means more drastic changes to their appearance.}
	\abilityRow{Spellcasting}{A replacement for individual \imp{Affinities}. A creature with this ability can innately cast magic using this statistic.} 
	\abilityRow{Swimming}{Aquatic creatures have a natural affinity for moving within the water - a high \imp{Swimming} shows an ability to move quickly and navigate in 3D. }
	\abilityRow{Tunnelling}{Whilst we are most familiar with creatures which walk on land, or soar above it, some rare creatures make a living beneath it. A high \imp{Tunnel} ability allows a being to move smoothly through seemingly solid earth and rock.}
}


\section{Movement}

Some of these abilities - notably \imp{Climb}, \imp{Flight}, \imp{Swimming} and \imp{Tunnelling} - grant creatures additional means of traversing around an environment, beyond the usual walking and running that humans are used to. 

It can generally be assumed that a zero-rating in this field menas that a given mode of transport is not possible. This should, of course, be taken with a hint of salt - few creatures with a zero \imp{speed} rating are physically unable to walk, and equally, a mighty \imp{Archangel} is not going to hesitate to dive into a pool, despite not having a \imp{Swimming} rating. However, a \imp{Nogtail} is not suddenly going to be able to fly, no matter how slowly. 

If a beast is using an alternative means of transportation, their rating for that means supercedes the normal rules about movement - if a \imp{Hippogriff} is currently in flight, all checks which might normally rely on \imp{speed} are instead made using \imp{flight}.

This also applies to a creature 


\clearpage

\newcommand\beastSpell[2]
{
	\bf #1	&	\parbox[t]{7.3  cm}{{\it #2}} \\
}
\newcommand\beastSpellList[1]
{
	
	\begin{tabular}{l l}
	#1
	\end{tabular}
}
\definecolor{statblockbg}{HTML}{FDF1DC} 

\makeatletter
\define@key{beast}{name}{\def\name{#1}}
\define@key{beast}{species}{\def\species{#1}}
\define@key{beast}{mind}{\def\mind{#1}}
\define@key{beast}{category}{\def\category{#1}}
\define@key{beast}{description}{\def\description{#1}}
\define@key{beast}{rating}{\def\rating{#1}}
\define@key{beast}{speed}{\def\speed{#1}}

\define@key{beast}{fit}{\def\fit{#1}}
\define@key{beast}{prs}{\def\prs{#1}}
\define@key{beast}{vit}{\def\vit{#1}}

\define@key{beast}{cha}{\def\cha{#1}}
\define@key{beast}{dec}{\def\dec{#1}}
\define@key{beast}{ins}{\def\ins{#1}}

\define@key{beast}{int}{\def\int{#1}}
\define@key{beast}{wil}{\def\wil{#1}}
\define@key{beast}{pcp}{\def\pcp{#1}}


\define@key{beast}{nUnharmed}{\def\nUnharmed{#1}}
\define@key{beast}{nBruised}{\def\nBruised{#1}}
\define@key{beast}{nHurt}{\def\nHurt{#1}}
\define@key{beast}{nInjured}{\def\nInjured{#1}}
\define@key{beast}{nWounded}{\def\nWounded{#1}}
\define@key{beast}{nMangled}{\def\nMangled{#1}}
\define@key{beast}{fortitude}{\def\fortitude{#1}}
\define@key{beast}{article}{\def\article{#1}}


\define@key{beast}{skills}{\def\skills{#1}}
\define@key{beast}{hasSkills}{\def\hasSkills{#1}}

\define@key{beast}{block}{\def\block{#1}}
\define@key{beast}{dodge}{\def\dodge{#1}}
\define@key{beast}{defy}{\def\defy{#1}}

\define@key{beast}{hasDamage}{\def\hasDamage{#1}}
\define@key{beast}{damage}{\def\damage{#1}}

\define@key{beast}{abilities}{\def\abilities{#1}}
\define@key{beast}{attacks}{\def\attacks{#1}}
\define@key{beast}{hasAttacks}{\def\hasAttacks{#1}}

\define@key{beast}{hasLanguages}{\def\hasLanguages{#1}}
\define@key{beast}{languages}{\def\languages{#1}}

\define@key{beast}{hasImage}{\def\hasImage{#1}}
\define@key{beast}{image}{\def\image{#1}}
\define@key{beast}{imageStack}{\def\imageStack{#1}}

\define@key{beast}{speciesOnly}{\def\speciesOnly{#1}}
%~ \define@key{beast}{hasImmune}{\def\immuneMode{#1}}
%~ \define@key{beast}{immune}{\def\immune{#1}}
%~ \define@key{beast}{hasResistance}{\def\resistanceMode{#1}}
%~ \define@key{beast}{resistance}{\def\resistance{#1}}
%~ \define@key{beast}{hasSusceptible}{\def\susceptibleMode{#1}}
%~ \define@key{beast}{susceptible}{\def\susceptible{#1}}
%~ \define@key{beast}{hasAbilities}{\def\abilitiesMode{#1}}
%~ \define@key{beast}{abilities}{\def\abilities{#1}}
%~ \define@key{beast}{hasSkills}{\def\skillsMode{#1}}
%~ \define@key{beast}{skills}{\def\skills{#1}}
%~ \define@key{beast}{hasActions}{\def\actionMode{#1}}
%~ \define@key{beast}{actions}{\def\actions{#1}}
%~ \define@key{beast}{habitat}{\def\habitat{#1}}
%~ \define@key{beast}{sizeName}{\def\sizeName{#1}}
%~ \define@key{beast}{size}{\def\size{#1}}
%~ \define@key{beast}{abilityBlock}{\def\aBlock{#1}}
%~ 
%~ \define@key{beast}{image}{\def\image{#1}}
%~ \define@key{beast}{hasImage}{\def\imageMode{#1}}
%~ \define@key{beast}{imPosition}{\def\imagePos{#1}}
%~ \define@key{beast}{hasLanguages}{\def\languageMode{#1}}
%~ \define@key{beast}{language}{\def\languages{#1}}
%~ \define@key{beast}{needsLine}{\def\needsLine{#1}}
%~ \define@key{beast}{comprehend}{\def\comprehend{#1}}
%~ \define@key{beast}{hasComprehend}{\def\comprehendMode{#1}}
%~ \define@key{beast}{imageHeight}{\def\imageHeight{#1}}
%~ \define@key{beast}{habitat}{\def\habitat{#1}}
%~ \define@key{beast}{needsPage}{\def\needsPage{#1}}
%~ 
%~ \define@key{beast}{hasHabitat}{\def\hasHabitat{#1}}
%~ \define@key{beast}{hasLair}{\def\hasLair{#1}}
%~ \define@key{beast}{lairActions}{\def\lairActions{#1}}
%~ \define@key{beast}{hasSenses}{\def\hasSenses{#1}}
%~ \define@key{beast}{senses}{\def\senses{#1}}
%~ \define@key{beast}{hasConditionImmune}{\def\hasConditionImmune{#1}}
%~ \define@key{beast}{conditionImmune}{\def\conditionImmune{#1}}
%%keyEnd    
\makeatother

\def\defaultSetter
{
	\setkeys{beast}{name = none, species = None, mind = none, rating = Unset, category = none, description = None, fit =0, prs = 0, vit =0, cha = 0, dec = 0, ins = 0, int = 0, wil =0, pcp = 0, nUnharmed =0, nBruised = 0, nHurt = 0, nInjured = 0, nWounded = 0, nMangled = 0,fortitude = 0, skills = 0, hasSkills = 0, block = 0, dodge = 0, defy = 0,hasDamage = 0, damage = , abilities = ,hasAttacks = 0, attacks = None,article = A, hasLanguages = 0, languages = None, hasImage = 0, image = None, speciesOnly = 0}
}
\newcounter{yCount}
\newcounter{nonZeroCount}
\def\bx{0.45}
\newcommand{\healthBox}
{%q

	\vbox{
	\key{Health}
	
    \begin{tikzpicture}
       
       \def\nTotal{ \fpeval{\nUnharmed + \nBruised + \nHurt + \nInjured + \nWounded + \nMangled + 1} }
       
       
       %\draw (0,0) rectangle ({\nTotal*\bx},{2*\bx});
       %\draw (0,\bx)--({\nTotal*\bx},{\bx});
       \foreach \c in {1,2,...,{\nTotal}}
       {
			%\draw ({\c*\bx},\bx)--({\c*\bx},{2*\bx});
			
			
			%\draw ({(\c)*\bx}, {1.5*\bx})--({(\c-0.5)*\bx}, {2*\bx})--({(\c-1)*\bx}, {1.5*\bx})--({(\c-0.5)*\bx}, {\bx})--cycle;
       }
       
       \setcounter{yCount}{0}
       \setcounter{nonZeroCount}{0}
       \foreach [count = \j from 0] \a/\b in { {Fine}/\nUnharmed, {Bruised \\(-1)}/\nBruised, {Hurt\\ (-2)}/\nHurt, {Injured\\(-3)}/\nInjured, {Harmed \\(-4)}/\nWounded, {Mangled (-5)}/\nMangled, {Critical}/1}
       {
			\def\g{1.7}
			\def\xVal{ {(\theyCount + (\b)/2)*\bx} + \thenonZeroCount*\bx/\g}
			\def\lx { {\theyCount *\bx} + \thenonZeroCount*\bx/\g}
			\if \b0
				%%
			\else
				\node[anchor = center] at (\xVal,{0}) {\parbox[t]{1cm}{\tiny\centering \a}};
				
				\foreach \j in {1,...,\b}
				{
					\def\cx{\lx + (\j-1)*\bx}
					\def\sqrtT{\bx/1.414213562373095}
					\draw[rotate around={45:({\cx+\bx/2},\bx)}] ({\cx+\bx/2-\sqrtT/2},{\bx-\sqrtT/2})rectangle ({\cx+\bx/2+\sqrtT/2},{\bx+\sqrtT/2});
					%\draw ({(\c)*\bx}, {1.5*\bx})--({(\c-0.5)*\bx}, {2*\bx})--({(\c-1)*\bx}, {1.5*\bx})--({(\c-0.5)*\bx}, {\bx})--cycle;
				}
				
				\addtocounter{nonZeroCount}{1}
				%\draw ({\xVal - \b/2*\bx},0) rectangle ({\xVal + \b/2*\bx},{2*\bx});
			\fi
			\addtocounter{yCount}{\b}
       }
       
    \end{tikzpicture}%
    }
}

\newcommand{\fortitudeBox}
{%q
	\if\fortitude0
		%who knows?
	\else
		\vbox{
		\parbox[t]{3.5 cm}{
		\key{Fortitude:}
		\\~\\
	    \begin{tikzpicture}
	      
	       \foreach \c in {1,...,{\fortitude}}
	       {
				\def\lx{ {(\c-1) * \bx}}
				
				\def\sqrtT{\bx/1.414213562373095}
				\draw[rotate around={45:({\lx},\bx)}] ({\lx-\sqrtT/2},{\bx-\sqrtT/2})rectangle ({\lx+\sqrtT/2},{\bx+\sqrtT/2});
	       }
	       
	    \end{tikzpicture}%
	    }
	    
	    }
	 \fi
}




\newcommand\understands[1]
{
	\imp{Understands: } #1
}
\newcommand\speaks[1]
{
	\imp{Speaks: } #1
}
\newcommand\skill[2]
{
	~& \imp{#1}: & \ratingB{#2} \\
		
}
\newcommand\titleBlock
{
	\if\speciesOnly0
	\definecolor{rulered}{HTML}{9C2B1B} 
		\dndline
		{ \huge \color{titlered}{\MakeUppercase{\name{}} }}
		\addcontentsline{toc}{subsection}{\name{}}
	\else
		{ \Huge{\MakeUppercase{\name{}} }}
		\addcontentsline{toc}{section}{\name{}}
	\fi
	
		MoM Rating: \rating{} {\it (\mind{}~\category{}) }
}

\newcommand\skillBlock
{
	\key{Abilities:}
	
	\begin{tabular}{@{} p{2cm} r c @{} }
		 \skills{}
		 
		
	\end{tabular}
}
\def\w{2.1}
\newcommand\stat[2]
{
	\parbox[t]{\w cm}{\vspace{0.3cm} \centering \key{#1} \\ \large \ratingB{#2}}
}
\newcommand\statBlock
{
	\normalsize
	
	\healthBox{}
		
	\begin{tabular}{m{4cm} m{4cm}}
	\fortitudeBox{}	&	\begin{tabular}{||l l||}
								\hline\hline
								\key{Block}	& \ratingB{\block}
								\\
								\key{Dodge}	&	\ratingB{\dodge}
								\\
								\key{Endure}	&	\ratingB{\defy}
								\\\hline\hline
						\end{tabular}
	\end{tabular}
	
	\damage{}
	
	{\centering
	\footnotesize
	\begin{tabular}{@{}  m{\w cm}  m{\w cm}  m{\w cm} @{}}
				 %\hline
				\stat{Fitness}{\fit}	&	\stat{Charm}{\cha}	&	\stat{Intelligence}{\int}
				\\ %\hline
				\stat{Precision}{\prs}	&	\stat{Deception}{\dec}	&	\stat{Willpower}{\wil}
				\\ % \hline
				\stat{Vitality}{\vit}	&	\stat{Insight}{\ins}	&	\stat{Perception}{\pcp}
				\\ %\hline
		\end{tabular}
	
	\normalsize
	}

	
}

\newcommand\ability[2]
{
\textbf{\textit{#1}}: #2
}
\newcommand\melee[5]
{
\parbox[t]{9.5cm}{\textbf{\textit{#1}}: ({\it melee attack, #2 dice, DV \fpeval{7+#3}}) \\ Effect: \imp{#4}, with Power #5}
}

\newcommand\meleeConsequence[6]
{
	\parbox[t]{9.5cm}{\textbf{\textit{#1}}: ({\it melee attack, #2 dice, DV \fpeval{7+#3}}) \\ Effect: \imp{#4}, with Power #5 \\ #6}
}
\newcommand\ranged[6]
{
\parbox[t]{10cm}{\textbf{\textit{#1}}: ({\it ranged attack: #2m, #3 dice, DV \fpeval{7+#4}}) \\ Effect: \imp{#5}, with Power #6}
}

\newcommand\rangedConsequence[7]
{
\parbox[t]{10cm}{\textbf{\textit{#1}}: ({\it ranged attack: #2m, #3 dice, DV \fpeval{7+#4}}) \\ Effect: \imp{#5}, with Power #6. \\ #7}
}

\newcommand\area[6]
{
	\parbox[t]{9.5cm}{\textbf{\textit{#1}}: ({\it area attack: #2, #3 dice, DV \fpeval{7+#4}}) \\ Effect: \imp{#5}, with Power #6}	
}


\newcommand\sideBySide
{
	\begin{minipage}{0.2 \textwidth}
	\includegraphics[width = 0.98 \textwidth, keepaspectratio=true]{../Images/\image}
	\end{minipage}
	\begin{minipage}{0.29\textwidth}
	\description{}
	\end{minipage}

}
\newcommand\verticalStack
{
	\begin{center}
	
	\includegraphics[width = 0.4 \textwidth, keepaspectratio=true]{../Images/\image}
	\end{center}
	
	\description{}

}


\newcommand\beast[1]
{
	\begingroup
	
	\defaultSetter
	\setkeys{beast}{#1}
	
	\vbox
	{
		
		\titleBlock{}
		\if\hasImage0
			\description{}
		\fi
	}	
		\if\hasImage1
			\if\imageStack1
				\verticalStack
			\else
				\sideBySide
			\fi
		\fi
	

	
	\statBlock

	\if\hasSkills1
		\vspace{0.7cm}
		\skillBlock{}
	\fi

	\dndlineFade{black}
	
	\abilities{}
	
	\if\hasLanguages1
		\ability{Languages}{\languages}
	\fi
		
	\if\hasAttacks1
		\subsubsection*{Armaments \& Attacks}
		\attacks{}
	\fi
	
	
	~
	
	\endgroup
}

\newcommand\species[3]
{
	%\clearpage

		{ \Huge{\MakeUppercase{#1} }}
		
	\addcontentsline{toc}{section}{#1}
	~
	
	#2

	~
	
	#3
	
	
}

\newcommand\speciesBeast[1]
{
	\beast{#1, speciesOnly = 1}
}


\species{Acromantula}
{
	The acromantula are an incredibly rare \minus{} and incredibly dangerous \minus{} species of gigantic\comma{} intelligent spiders. Found mainly in dense forests\comma{} where they weave their web\minus{}covered nests\comma{} they only occaisionally go out to hunt\comma{} preferring instead to let their prey come to them. 

Hatching from eggs the size of rugby balls\comma{} the oldest specimens have legspans in excess of 10 metres. Their equally enormous fangs contain a potent venom. The speed\comma{} strength and venom\comma{} however\comma{} is not what makes the Acromantula a truly awful foe. Rather\comma{} their greatest weapon is their formiddable intellect\comma{} which allows them to outthink even the greatest wizards. 

\ability{Elaborate Lairs}{A spider\apos{}s central tenet is patience: waiting for prey to come ot you. Acromantula are no different\comma{} though they work on a slightly different scale. Over their multi\minus{}decade\minus{}long lifespan\comma{} a Patriarch will build an enormous\comma{} complex labyrinth of webs and forest\comma{} in order to ensnare their unsuspecting prey}

\ability{Talking Spiders:}{Acromantula have the ability to speak the spider tongue\comma{} to command their legions of arachnid followers. As they age and their minds continue to develop\comma{} they even gain the ability to understand and eventually speak in human tongues.}

\ability{Keen Sight}{In addition to their web\minus{}enhanced senses\comma{} the 8 compound eyes of the acromantula allow them to see in incredible detail\comma{} even in dim light}

\ability{Webspinners}{As members of the spider family\comma{} all Acromantula have an affinity for spinning webs\comma{} and using them to sense and then ensnare their prey.}
}
{
\beast{name = Acromantula Hatchling, species = Acromantula, mind = Non\minus{}Sapient, category = Monstrosity, rating = 3, nUnharmed=1, nBruised=0, nHurt=2, nInjured=0, nWounded=0, nMangled=1, block=1, dodge=3, fortitude=3, hasSkills = 1, skills = \skill{Climb}{3}
\skill{Covert}{4}
\skill{Inhuman Senses}{2}, description = A newborn \imp{acromantula} has a shiny\comma{} hairless and pale\minus{}grey carapace\comma{} covering their body which is only 1 metre across\comma{} and has a diminished intelligence compared to their full grown counterparts.}


\beast{name = Acromantula Adult, species = Acromantula, mind = Sapient, category = Monstrosity, rating = 5, nUnharmed=3, nBruised=2, nHurt=1, nInjured=1, nWounded=1, nMangled=2, block=3, dodge=4, fortitude=5, hasSkills = 1, skills = \skill{Climb}{6}
\skill{Covert}{4}
\skill{Inhuman Senses}{5}
\skill{Speed}{5}
\skill{Strength}{3}, description = A fully grown \imp{Acromantula} is something to be greatly feared. They can run incredibly quickly and they utilise a ranged web attack to ensnare their pray\comma{} capturing it for later devourment….}


\beast{name = Acromantula Patriarch, species = Acromantula, mind = Sapient, category = Monstrosity, rating = 6, nUnharmed=7, nBruised=1, nHurt=1, nInjured=1, nWounded=1, nMangled=3, block=2, dodge=2, fortitude=7, hasSkills = 1, skills = \skill{Climb}{1}
\skill{Command}{7}
\skill{Inhuman Senses}{7}
\skill{Strength}{4}, description = The eldest of the spider monstrosities is known as the \imp{Patriarch}. Though they have reached truly gargantuan sizes\comma{} their bodies have become decrepit with age. Their minds\comma{} however\comma{} are razor sharp and they have mastered human speech.}



}




\species{Angels}
{
	Angels are powerful\comma{} beautiful Celestial creatures\comma{} denizens of Elysium\comma{} one of the Higher Planes\comma{} though they can be found throughout the multiverse. Often perceived as powerful agents of Deities\comma{} servants of benevolent gods\comma{} it is actually unknown who or what provides these powerful creatures with their deeper purpose. 

\ability{Benevolent Fury}{Almost universally pure of heart and intrinsically ethical and good\comma{} Angels are representative of everything full of light and life in the universe. Angels will never compromise their core beliefs. They are not\comma{} however\comma{} pacifists. Angels are great and pwoerful warriors\comma{} and will strike down their enemies in the name of protecting those who cannot protect themselves.}

\ability{Angelic Host}{The Angelic society is known as the {\it Angelic Host}\comma{} a powerful seemingly omniscient society which dwells almost entirely in the Silver City found at the centre of Elysium. This society is highly structured and hierarchical\comma{} with angels being created to fill specific niches within each echelon of society. Each Angel derives their powers from their position within the angelic hierarchy\comma{} with the highest tiers wielding terrifying amounts of power.}

\ability{Holy Crusades}{Angels only leave the Silver City on two conditions\comma{} the most common of which is being directed on a holy quest by one of their superiors. Most Angels met outside of Elysium are conducting such a quest. The difficulty of the quest depends on the ranking of the angel in question: a cherubim might be sent out to conduct a blessing\comma{} or deliver a message\comma{} whilst a quest which calls for an Archangel to be sent would be a truly dire universe\minus{}ending scenario. }

\ability{Fallen Angel}{The other condition under which an Angel is refused entry into the Silver City is if they have {\it fallen}. Though Angels will never compromise their core beliefs\comma{} and are almost inherently good in nature it is possible for them to fall victim to their own pride and hubris. If this happens\comma{} an angel may act against the wishes of the Host\comma{} or inadvertently perform some great act of evil. 

If this happens\comma{} the Host will disavow them\comma{} and cast them out. Without the purpose granted to them by the rigid structure of Angelic society\comma{} many such fallen angels go entirely mad. Others sink into a deep\comma{} vengeful fury and declare war on the Host\comma{} whilst others are believed to undergo a transformation\comma{} becoming powerful demonic creatures. }

\ability{Immortal Spirit}{As a celestial being\comma{} an angel is incredibly resilient and requires neither food\comma{} drink\comma{} air or sleep (though they may enjoy the experience).}
}
{
\beast{name = Cherubim, species = Angels, mind = Ineffable, category = Celestial, rating = 6, nUnharmed=0, nBruised=0, nHurt=0, nInjured=0, nWounded=0, nMangled=0, block=0, dodge=0, fortitude=0, description = }


\beast{name = Seraphim, species = Angels, mind = Ineffable, category = Celestial, rating = 6, nUnharmed=0, nBruised=0, nHurt=0, nInjured=0, nWounded=0, nMangled=0, block=0, dodge=0, fortitude=0, description = }


\beast{name = Throne, species = Angels, mind = Ineffable, category = Celestial, rating = 7, nUnharmed=0, nBruised=0, nHurt=0, nInjured=0, nWounded=0, nMangled=0, block=0, dodge=0, fortitude=0, description = }


\beast{name = Archangel, species = Angels, mind = Ineffable, category = Celestial, rating = 7, nUnharmed=0, nBruised=0, nHurt=0, nInjured=0, nWounded=0, nMangled=0, block=0, dodge=0, fortitude=0, description = }



}




\species{Apparitions}
{
	Apparations are ghostly creatures \minus{} spirits and ghosts which defy the laws of life and death\comma{} and yet continue to roam the mortal realms. 

\ability{Incorporeal Form}{Almost all apparitions are merely imprints\comma{} shadows lying between the astral realm and the mortal plane\comma{} and as such are totally incapable of interacting with the physical realm. They can pass through solid objects at will\comma{} move with blatant disregard for the force of gravity\comma{} as well as being immune to all normal forms of attack. }

\ability{Unknowable Purpose}{It is not understood what drives apparaitions of any kind to remain behind on the mortal plain. Some speculate that all apparitions are manifestations of lost souls\comma{} bound to the Earth through their need to find closure\comma{} or complete some important task. Others speculate that they are glitches in the fabric of reality\comma{} whose motives even they themselves do not understand.}

\ability{Unkillable}{It is impossible to kill an apparition\comma{} though it is possible to banish them for a time. The only known way to permanently deal with an apparition is to plunge one into the Void\comma{} or help them find the closure they need\comma{} or otherwise convince them to relinquish their hold on the mortal realm. }
}
{
\beast{name = Ghost, species = Apparitions, mind = Ineffable, category = Phantasm, rating = 0, nUnharmed=0, nBruised=0, nHurt=0, nInjured=0, nWounded=0, nMangled=0, block=0, dodge=0, fortitude=0, description = A ghost is the imprint of the soul of a once\minus{}living wizard or witch\comma{} left to wander the material realm after their physical form has died. A ghost resembles their former selves at the moment of their death\comma{} though in a translucent\comma{} silver\minus{}grey form. 

No\minus{}one knows what causes a ghost to remain behind\comma{} though it is posited that these fleshless spirits were mortally afraid of death or have some extraordinarily strong connection to the locations they haunt.}


\beast{name = Poltergeist, species = Apparitions, mind = Ineffable, category = Phantasm, rating = 2, nUnharmed=0, nBruised=0, nHurt=0, nInjured=0, nWounded=0, nMangled=0, block=0, dodge=0, fortitude=0, description = A poltergeist is an amortal\comma{} indestructable spirit of chaos and mischief. They appear as a short\comma{} childlike figure dressed in a motley jester\apos{}s garb\comma{} with glowing orange eyes\comma{} which twinkle with mischief. 

Brought into existence by a critical mass of humans\comma{} trickery and mischief\comma{} poltergeists haunt the specific place which they are tied to. 

Unusually out of apparitions and other spiritual creatures\comma{} poltergeists are able to take on physical form and cast primitive forms of magic \minus{} which they use to wreak chaos and play pranks on unsuspecting humans.}


\beast{name = Boggart, species = Apparitions, mind = Ineffable, category = Phantasm, rating = 2, nUnharmed=0, nBruised=0, nHurt=0, nInjured=0, nWounded=0, nMangled=0, block=0, dodge=0, fortitude=0, description = A manifestation of fear and primal terror\comma{} the shapeshifting boggart peers into the minds of humans\comma{} and takes the form of their worst nightmare. 

A boggart can never harm you\comma{} though they can be difficult to contain. The accepted trick is to transfigure them to look stupid\comma{} prompting a fit of laughter – which is fatal to a boggart.}



}




\species{Arachnid}
{
	The arachnids are a family of giant spider found throughout the wizarding world. Most members of this species are suspected to have been formed from mundane species that were experimented upon by witches and wizards throughout history\comma{} though others are known to occur in freak mutations. 

Whatever the mechanism for bringing them into this world\comma{} many have since escaped into the wild\comma{} to wreak havoc on muggles and wizardkind alike \minus{} some spinning their webs to ensnare the unwary\comma{} others prowling and hunting directly for their prey. 

\ability{Great Size}{The magical arachnids are much larger than their non\minus{}magical compatriots. Though smaller than acromantula\comma{} some species can reach legspans of up to one metre.}

\ability{Keen Sight}{In addition to their web\minus{}enhanced senses\comma{} the 8 compound eyes of arachnids allow them to see in incredible detail\comma{} even in dim light}

\ability{Webspinners}{As members of the spider family\comma{} all arachnids have an affinity for spinning webs\comma{} and using them to sense and then ensnare their prey.}
}
{
\beast{name = Great Widow, species = Arachnid, mind = Non\minus{}sapient, category = Beast, rating = 3, nUnharmed=0, nBruised=0, nHurt=0, nInjured=0, nWounded=0, nMangled=0, block=0, dodge=0, fortitude=0, description = Magical experimentation on a {\it Black Widow} produced this grossly oversized specimen\comma{} and gave it the ability to spit acid.}


\beast{name = Howling Tick, species = Arachnid, mind = Non\minus{}sapient, category = Beast, rating = 3, nUnharmed=0, nBruised=0, nHurt=0, nInjured=0, nWounded=0, nMangled=0, block=0, dodge=0, fortitude=0, description = The name of the Howling Tick is misleading\comma{} as it is neither a tick\comma{} and nor does it howl. Instead the name comes from its tendency to suck blood from its victims\comma{} and the howls of pain that result.

The Howling Tick has the magical ability to grow in size when it feeds\comma{} however they must continually gorge in order to maintain their size\comma{} or they quickly shrink back.}


\beast{name = Spraying Mantis, species = Arachnid, mind = Non\minus{}sapient, category = Beast, rating = 3, nUnharmed=0, nBruised=0, nHurt=0, nInjured=0, nWounded=0, nMangled=0, block=0, dodge=0, fortitude=0, description = A gigantic\comma{} horrifying crossbreed between a spider\comma{} and a praying mantis resulted in a grotesque monstrosity. The being appears\comma{} outwardly\comma{} to be a giant metre\minus{}long insect walking on 4 legs\comma{} with an additional 4 arms turned into hinged and hooked arms which they use to catch their prey. 

True to their name\comma{} they also spray acidic juices on their prey\comma{} to aid in their eventual digestion.}


\beast{name = Brood Mother, species = Arachnid, mind = Non\minus{}sapient, category = Beast, rating = 3, nUnharmed=0, nBruised=0, nHurt=0, nInjured=0, nWounded=0, nMangled=0, block=0, dodge=0, fortitude=0, description = This grossly oversized spider is the result of a freak mutation which  causes them to grow to grotesque sizes and become viciously maternal. A Brood Mother will collect any and all spider eggs that it finds and nuture them as if they were her own in the dark\comma{} secluded where she has built her nest.

\blindtext{4}}



}




\species{Bowtruckle}
{
	Bowtruckles are a species of hand\minus{}sized\comma{} insect\minus{}eating humanoids which reside inside trees. Bowtruckles prefer to make their home in trees with wand\minus{}quality wood (or perhaps\comma{} it is the presence of a Bowtruckle which makes a tree wand\minus{}grade)\comma{} and a single tree can host up to 5 generations of the same bowtruckle clan. 

Normally peacable and shy creatures\comma{} they become territorial and violent when their home tree is threatened. 

\ability{Camouflaged}{Bowtruckles blend in perfectly with their trees\comma{} when they wish to pass unnoticed\comma{} they appear as nothing more than a set of leafy twigs. It is only by cathcing them in motion that they can be easily spotted.}

\ability{Natural Climbers}{Living their entire life in trees\comma{} bowtruckles are natural climbers\comma{} and can move across near\minus{}sheer vertical surfaces as easily as they walk}. 

\ability{Long Fingers}{Nominally evolved to help dig insects out of the bark of a tree\comma{} the long spindly fingers of a bowtruckle can be used to perform very delicate tasks\comma{} such as picking a lock\comma{} or used offensively to poke out the eyes of those who threaten their treetop homes.}
}
{
\beast{name = Bowtruckle Splinter, species = Bowtruckle, mind = Non\minus{}sapient, category = Imp, rating = 2, nUnharmed=0, nBruised=0, nHurt=0, nInjured=0, nWounded=0, nMangled=0, block=0, dodge=0, fortitude=0, description = hey}


\beast{name = Bowtruckle Flower, species = Bowtruckle, mind = Non\minus{}sapient, category = Imp, rating = 2, nUnharmed=0, nBruised=0, nHurt=0, nInjured=0, nWounded=0, nMangled=0, block=0, dodge=0, fortitude=0, description = hey}


\beast{name = Bowtruckle Thorn, species = Bowtruckle, mind = Non\minus{}sapient, category = Imp, rating = 2, nUnharmed=0, nBruised=0, nHurt=0, nInjured=0, nWounded=0, nMangled=0, block=0, dodge=0, fortitude=0, description = hey}



}




\species{Ceratothid}
{
	The Ceratothids are a family of loosely related magical quadrupeds. Defined by their huge bulk and relatively bovine\minus{}like appearance\comma{} most Ceratothid\apos{}s have a gentle temperament until angered\comma{} at which point their great mass and inherent magic makes them dangerous foes.
}
{
\beast{name = Graphorn, species = Ceratothid, mind = Non\minus{}sapient, category = Beast, rating = 4, nUnharmed=0, nBruised=0, nHurt=0, nInjured=0, nWounded=0, nMangled=0, block=0, dodge=0, fortitude=0, description = hey}


\beast{name = Erumpet, species = Ceratothid, mind = Non\minus{}sapient, category = Beast, rating = 5, nUnharmed=0, nBruised=0, nHurt=0, nInjured=0, nWounded=0, nMangled=0, block=0, dodge=0, fortitude=0, description = hey}


\beast{name = Re'em, species = Ceratothid, mind = Non\minus{}sapient, category = Beast, rating = 4, nUnharmed=0, nBruised=0, nHurt=0, nInjured=0, nWounded=0, nMangled=0, block=0, dodge=0, fortitude=0, description = hey}



}




\beast{name = Chimera, species = Chimera, mind = Non\minus{}sapient, category = Monstrosity, rating = 7, nUnharmed=0, nBruised=0, nHurt=0, nInjured=0, nWounded=0, nMangled=0, block=0, dodge=0, fortitude=0, description = hey}



\species{Cloaked Spirits}
{
	d
}
{
\beast{name = Dementor, species = Cloaked Spirits, mind = Ineffable, category = Abomination, rating = 6, nUnharmed=0, nBruised=0, nHurt=0, nInjured=0, nWounded=0, nMangled=0, block=0, dodge=0, fortitude=0, description = hey}


\beast{name = Lethifold, species = Cloaked Spirits, mind = Ineffable, category = Abomination, rating = 5, nUnharmed=0, nBruised=0, nHurt=0, nInjured=0, nWounded=0, nMangled=0, block=0, dodge=0, fortitude=0, description = hey}



}




\species{Creations}
{
	b
}
{
\beast{name = Hidebehind, species = Creations, mind = Sapient, category = Sprite, rating = 3, nUnharmed=0, nBruised=0, nHurt=0, nInjured=0, nWounded=0, nMangled=0, block=0, dodge=0, fortitude=0, description = hey}


\beast{name = Blast\minus{}Ended Skrewt, species = Creations, mind = Non\minus{}sapient, category = Monstrosity, rating = 3, nUnharmed=0, nBruised=0, nHurt=0, nInjured=0, nWounded=0, nMangled=0, block=0, dodge=0, fortitude=0, description = hey}



}




\beast{name = Demiguise, species = Demiguise, mind = Non\minus{}sapient, category = Beast, rating = 3, nUnharmed=0, nBruised=0, nHurt=0, nInjured=0, nWounded=0, nMangled=0, block=0, dodge=0, fortitude=0, description = hey}



\species{Dinosaurs}
{
	Long thought extinct by the muggles\comma{} dinosaurs are a class of ancient lizard and reptilian creatures which dominated life on Earth up until 75 million years ago.
}
{
\beast{name = Ankylosaurus, species = Dinosaurs, mind = Non\minus{}sapient, category = Beast, rating = 5, nUnharmed=0, nBruised=0, nHurt=0, nInjured=0, nWounded=0, nMangled=0, block=0, dodge=0, fortitude=0, description = }


\beast{name = Diplodocus, species = Dinosaurs, mind = Non\minus{}sapient, category = Beast, rating = 4, nUnharmed=0, nBruised=0, nHurt=0, nInjured=0, nWounded=0, nMangled=0, block=0, dodge=0, fortitude=0, description = }


\beast{name = Plesiosaurus, species = Dinosaurs, mind = Non\minus{}sapient, category = Beast, rating = 4, nUnharmed=0, nBruised=0, nHurt=0, nInjured=0, nWounded=0, nMangled=0, block=0, dodge=0, fortitude=0, description = }


\beast{name = Pterodactul, species = Dinosaurs, mind = Non\minus{}sapient, category = Beast, rating = 3, nUnharmed=0, nBruised=0, nHurt=0, nInjured=0, nWounded=0, nMangled=0, block=0, dodge=0, fortitude=0, description = }


\beast{name = Spinosaurus, species = Dinosaurs, mind = Non\minus{}sapient, category = Beast, rating = 6, nUnharmed=0, nBruised=0, nHurt=0, nInjured=0, nWounded=0, nMangled=0, block=0, dodge=0, fortitude=0, description = }


\beast{name = Stegosaurus, species = Dinosaurs, mind = Non\minus{}sapient, category = Beast, rating = 4, nUnharmed=0, nBruised=0, nHurt=0, nInjured=0, nWounded=0, nMangled=0, block=0, dodge=0, fortitude=0, description = }


\beast{name = Tricerotops, species = Dinosaurs, mind = Non\minus{}sapient, category = Beast, rating = 4, nUnharmed=0, nBruised=0, nHurt=0, nInjured=0, nWounded=0, nMangled=0, block=0, dodge=0, fortitude=0, description = }


\beast{name = Tyrannosaurus Rex, species = Dinosaurs, mind = Non\minus{}sapient, category = Beast, rating = 5, nUnharmed=0, nBruised=0, nHurt=0, nInjured=0, nWounded=0, nMangled=0, block=0, dodge=0, fortitude=0, description = }


\beast{name = Velociraptor, species = Dinosaurs, mind = Non\minus{}sapient, category = Beast, rating = 3, nUnharmed=0, nBruised=0, nHurt=0, nInjured=0, nWounded=0, nMangled=0, block=0, dodge=0, fortitude=0, description = }



}




\species{Dragons}
{
	c
}
{
\beast{name = Antipodean Opaleye, species = Dragons, mind = Sapient, category = Draconid, rating = 6, nUnharmed=0, nBruised=0, nHurt=0, nInjured=0, nWounded=0, nMangled=0, block=0, dodge=0, fortitude=0, description = hey}


\beast{name = Chinese Fireball, species = Dragons, mind = Non\minus{}sapient, category = Draconid, rating = 6, nUnharmed=0, nBruised=0, nHurt=0, nInjured=0, nWounded=0, nMangled=0, block=0, dodge=0, fortitude=0, description = hey}


\beast{name = Common Welsh Green, species = Dragons, mind = Non\minus{}sapient, category = Draconid, rating = 6, nUnharmed=0, nBruised=0, nHurt=0, nInjured=0, nWounded=0, nMangled=0, block=0, dodge=0, fortitude=0, description = hey}


\beast{name = Hebridean Black, species = Dragons, mind = Non\minus{}sapient, category = Draconid, rating = 6, nUnharmed=0, nBruised=0, nHurt=0, nInjured=0, nWounded=0, nMangled=0, block=0, dodge=0, fortitude=0, description = hey}


\beast{name = Hungarian Horntail, species = Dragons, mind = Non\minus{}sapient, category = Draconid, rating = 6, nUnharmed=0, nBruised=0, nHurt=0, nInjured=0, nWounded=0, nMangled=0, block=0, dodge=0, fortitude=0, description = hey}


\beast{name = Norwegian Ridgeback, species = Dragons, mind = Best, category = Draconid, rating = 6, nUnharmed=0, nBruised=0, nHurt=0, nInjured=0, nWounded=0, nMangled=0, block=0, dodge=0, fortitude=0, description = hey}


\beast{name = Peruvian Vipertooth, species = Dragons, mind = Sapient, category = Draconid, rating = 6, nUnharmed=0, nBruised=0, nHurt=0, nInjured=0, nWounded=0, nMangled=0, block=0, dodge=0, fortitude=0, description = hey}


\beast{name = Romanian Longhorn, species = Dragons, mind = Non\minus{}sapient, category = Draconid, rating = 6, nUnharmed=0, nBruised=0, nHurt=0, nInjured=0, nWounded=0, nMangled=0, block=0, dodge=0, fortitude=0, description = hey}


\beast{name = Swedish Shortsnout, species = Dragons, mind = Sapient, category = Draconid, rating = 6, nUnharmed=0, nBruised=0, nHurt=0, nInjured=0, nWounded=0, nMangled=0, block=0, dodge=0, fortitude=0, description = hey}


\beast{name = Ukranian Ironbelly, species = Dragons, mind = Sapient, category = Draconid, rating = 6, nUnharmed=0, nBruised=0, nHurt=0, nInjured=0, nWounded=0, nMangled=0, block=0, dodge=0, fortitude=0, description = hey}



}




\species{Elemental Avatar}
{
	b
}
{
\beast{name = Avatar of Fire, species = Elemental Avatar, mind = Ineffable, category = Elemental, rating = 5, nUnharmed=0, nBruised=0, nHurt=0, nInjured=0, nWounded=0, nMangled=0, block=0, dodge=0, fortitude=0, description = hey}


\beast{name = Avatar of Ice, species = Elemental Avatar, mind = Ineffable, category = Elemental, rating = 5, nUnharmed=0, nBruised=0, nHurt=0, nInjured=0, nWounded=0, nMangled=0, block=0, dodge=0, fortitude=0, description = hey}


\beast{name = Avatar of Storms, species = Elemental Avatar, mind = Ineffable, category = Elemental, rating = 5, nUnharmed=0, nBruised=0, nHurt=0, nInjured=0, nWounded=0, nMangled=0, block=0, dodge=0, fortitude=0, description = hey}



}




\species{Elf}
{
	Modern muggle culture frequently imagines Elves to be superhuman\comma{} immortal and otherworldly creatures – outwardly appearing as impossibly beautiful humans and wielding immense\comma{} primal magic. This is primarly due to the influence of the muggle writer Tolkein (who was himself a squib). In reality\comma{} elves are much closer to those envisaged in medieval Germanic mythology – small\comma{} impish tricksters. 

Though they have a love for tricks and fun\comma{} the most common form of elf throughout history\comma{} the {\it Br{\'u}nb{\'a}su}\comma{} or {\it Common Brownie} often helped humans by performing small chores for them. In one of the more shameful acts of wizarding history\comma{} the Br{\'u}nb{\'a}su were then systematically enslaved and brainwashed into eternal servitude\comma{} leading to the creation of the race of House\minus{}Elves. Witnessing this horrific act of human cruelty\comma{} many of the other elf species retreated into relative obscurity.

\ability{Diminutive Frame}{Almost all species of Elf are tiny in stature – rarely reaching more than 3ft in height – with skinny and spindly arms and legs. They rely on magic for physical acts that their size denies them. }

\ability{Elfin Magics}{Elf magic is unlike any that is understood by humans\comma{} who often look down on it as inferior. Some scholars\comma{} however\comma{} have hypothesised that elfin magic is in fact far superior to wizard magic. It is only due to the elfin spirit and its dislike of organisation and study (and probably helped by wizard oppression) that they have not learned to harness it to its full potential. The most prominent quirk of elfin magic is their ability to ignore even the strongest magical wards and boundaries.}  

\ability{Fond of Trickery}{All elves (except perhaps the brainwashed House Elves) are fond of trickery and fun. They play endless pranks on one another\comma{} delighting in causing small amounts of chaos. Even those who are dedicated to cleaning and tidying play the occassional trick: muggles notice this in the form of missing socks\comma{} or keys never quite being where they were left.}

\ability{Holding Court}{Elven society is chaotic and unorganised. They rarely recognise a leader\comma{} though they sometimes assemble themselves into bands and groups for the purpose of safety. This all changes when one of the Elven Princes summons them to their Seelie Court. Every free\minus{}elf belongs to one of these courts\comma{} which is presided over by one of the more powerful Hulduf{\'o}lk. Court is only summoned in times of great emergency\comma{} such as conflict with another court\comma{} or when a great external threat is detected.}
}
{
\beast{name = House\minus{}Elf, species = Elf, mind = Sapient, category = Imp, rating = 1, nUnharmed=0, nBruised=0, nHurt=0, nInjured=0, nWounded=0, nMangled=0, block=0, dodge=0, fortitude=0, description = A House\minus{}Elf is a small impish humanoid creature which has been bound\comma{} or inherited a binding\comma{} to a given master. House\minus{}Elves are incredibly devoted to these masters\comma{} and will obey every order given to them. Typically treated as household servants\comma{} their magic is dedicated to the upkeep of the home and\comma{} on incredibly rare occasions\comma{} the repelling of intruders.}


\beast{name = Br{\'u}nb{\'a}su, species = Elf, mind = Sapient, category = Imp, rating = 2, nUnharmed=0, nBruised=0, nHurt=0, nInjured=0, nWounded=0, nMangled=0, block=0, dodge=0, fortitude=0, description = It is said that the \name{} were the original species that were enslaved and magically altered over generations to become the House\minus{}Elves. Long considered extinct\comma{} the freeing of Dobby the House\minus{}Elf\comma{} the first Free\minus{}Elf in 1000 years\comma{} has led to a resurgence in the \name{} as a newly re\minus{}recognised species. 

Though rejecting their former enslavement\comma{} the \name{} retain a love of order and cleanliness\comma{} and will often do chores for a household – on the condition of respect and equitable payment. They are easily offended\comma{} and if they feel denigrated or slighted\comma{} they will wreck the place\comma{} and will never return.}


\beast{name = Elleng\ae{}st, species = Elf, mind = Sapient, category = Imp, rating = 2, nUnharmed=0, nBruised=0, nHurt=0, nInjured=0, nWounded=0, nMangled=0, block=0, dodge=0, fortitude=0, description = The archetypal impish prankster\comma{} the \name{}\comma{} also known as a `Puck’\comma{} or a `True Imp’ competes only with the Poltergeist for the mischief\minus{}maker\apos{}s crown. They adore slapstick comedy and petty theft\comma{} often building themselves a nest out of their stolen objects. 

Visually\comma{} an \name{} appears to be a small\comma{} flightless fairy; a slender humanoid\comma{} approximately a foot tall. Their mouth is filled with a disturbing number of teeth\comma{} but they’re too small to harm a human\comma{} and are used to supply their mostly\minus{}insect based diet.}


\beast{name = Dun\ae{}lf, species = Elf, mind = Sapient, category = Imp, rating = 3, nUnharmed=0, nBruised=0, nHurt=0, nInjured=0, nWounded=0, nMangled=0, block=0, dodge=0, fortitude=0, description = The warriors and guardians of the Elves\comma{} the \name{} make great use of their nimble nature and their ability to turn invisible\comma{} more than making up for their diminutive stature. The \name{} also wield magical bows which inflict agonising pain on their targets.}


\beast{name = V\ae{}ttir, species = Elf, mind = Sapient, category = Imp, rating = 3, nUnharmed=0, nBruised=0, nHurt=0, nInjured=0, nWounded=0, nMangled=0, block=0, dodge=0, fortitude=0, description = The least human\minus{}like of the elves\comma{} the \name{} are a species of elves which have an affinity for the primal elements\comma{} and the associated magics. 

Though capable of wielding all of the elements\comma{} many \name{} find themselves drawn to one element in particular. This reflects in their outwards appearance\comma{} which slowly begins to shift and change. A fire\minus{}bound \name{} grows a fiery red beard and orange eyes\comma{} whilst a water\minus{}bound \name{}\apos{}s skin turns blue and appears permanently damp. Current arcane knowledge has found no additional abilities granted with this change\comma{} so it is unknown if there is any particular reason for this `bonding’ process to occur.}


\beast{name = Hulduf{\'o}lk, species = Elf, mind = Sapient, category = Sprite, rating = 5, nUnharmed=0, nBruised=0, nHurt=0, nInjured=0, nWounded=0, nMangled=0, block=0, dodge=0, fortitude=0, description = The \name{} (Icelandic for {\it The Hidden People}) are powerful extraplanar beings which most closely resemble the Tolkenien elves\comma{} for they appear as $\sim$5ft humanoids of exceptional\comma{} unearthly beauty and appear to glow from within with a golden radiance. 

The \name{} reside within a group of dimensions known as the {\it Seelie Courts}\comma{} which lie close to the Astral realm. Each Seelie Court is a verdant paradise\comma{} populated entirely by elves\comma{} both those species which are found on Earth\comma{} and myriad others.

Each Court is ruled over by an Elf\minus{}Prince\comma{} the most powerful \name{} in the Court\comma{} as well as their advisors and attendants. Typically these offices are all held by other \name{}s\comma{} though rarely you may find a Dun\ae{}lf or V\ae{}ttir who has risen through the ranks of the Court to serve in these lofty positions. 

Even those elves which reside on Earth are (technically) members of a Court\comma{} and must obey the summons from their Prince when it is time for them to attend their Seelie Court\comma{} failure to do so leads to expulsion (or worse). In reality\comma{} however\comma{} the Seelie Courts are rarely called in such a manner and Earthbound elves may not even know that they belong to a Court until the summons arrives. 

The \name{} wield immense and powerful magic\comma{} which they primarily use to cause mischief\comma{} mayhem and to disrupt the plans of the other members of the courts. The most famous conflict between various courts is that between Oberon and Titania\comma{} with the conflict spilling over into the Mortal Plane on more than one occasion\comma{} even making its way into the muggle subconscious. 

On rare occasions a Court can become corrupted by evil – either due to the profane actions of the Prince themselves\comma{} or because of some external corrupting influence such as a Voidic Incursion. A corrupted court is known as an {\it Unseelie Court}\comma{} and is often a source of violence\comma{} conflict and further corruption. 

Though not all \name{}s are Elf\minus{}Princes\comma{} those which are have access to the specified {\it Lair Actions} whilst they remain in their Court.}



}




\species{Fairy}
{
	d
}
{
\beast{name = Doxy, species = Fairy, mind = Non\minus{}sapient, category = Imp, rating = 2, nUnharmed=0, nBruised=0, nHurt=0, nInjured=0, nWounded=0, nMangled=0, block=0, dodge=0, fortitude=0, description = hey}


\beast{name = Pixie, species = Fairy, mind = Non\minus{}sapient, category = Imp, rating = 2, nUnharmed=0, nBruised=0, nHurt=0, nInjured=0, nWounded=0, nMangled=0, block=0, dodge=0, fortitude=0, description = hey}


\beast{name = True Fairy, species = Fairy, mind = Non\minus{}sapient, category = Imp, rating = 2, nUnharmed=0, nBruised=0, nHurt=0, nInjured=0, nWounded=0, nMangled=0, block=0, dodge=0, fortitude=0, description = hey}



}




\species{False Dragon}
{
	b
}
{
\beast{name = Draconic Guardian, species = False Dragon, mind = Non\minus{}sapient, category = Construct, rating = 5, nUnharmed=0, nBruised=0, nHurt=0, nInjured=0, nWounded=0, nMangled=0, block=0, dodge=0, fortitude=0, description = hey}


\beast{name = Hydra, species = False Dragon, mind = Non\minus{}sapient, category = Monstrosity, rating = 6, nUnharmed=0, nBruised=0, nHurt=0, nInjured=0, nWounded=0, nMangled=0, block=0, dodge=0, fortitude=0, description = hey}



}




\species{Fish}
{
	c
}
{
\beast{name = Lobalug, species = Fish, mind = Non\minus{}sapient, category = Beast, rating = 1, nUnharmed=0, nBruised=0, nHurt=0, nInjured=0, nWounded=0, nMangled=0, block=0, dodge=0, fortitude=0, description = hey}


\beast{name = Mackled Malaclaw, species = Fish, mind = Non\minus{}sapient, category = Beast, rating = 5, nUnharmed=0, nBruised=0, nHurt=0, nInjured=0, nWounded=0, nMangled=0, block=0, dodge=0, fortitude=0, description = hey}


\beast{name = Plimpy, species = Fish, mind = Non\minus{}sapient, category = Beast, rating = 1, nUnharmed=0, nBruised=0, nHurt=0, nInjured=0, nWounded=0, nMangled=0, block=0, dodge=0, fortitude=0, description = hey}


\beast{name = Ramora, species = Fish, mind = Non\minus{}sapient, category = Phantasm, rating = 6, nUnharmed=0, nBruised=0, nHurt=0, nInjured=0, nWounded=0, nMangled=0, block=0, dodge=0, fortitude=0, description = hey}



}




\species{Flora}
{
	d
}
{
\beast{name = Bundimun, species = Flora, mind = Non\minus{}sapient, category = Flora, rating = 1, nUnharmed=0, nBruised=0, nHurt=0, nInjured=0, nWounded=0, nMangled=0, block=0, dodge=0, fortitude=0, description = hey}


\beast{name = Horklump, species = Flora, mind = Non\minus{}sapient, category = Flora, rating = 1, nUnharmed=0, nBruised=0, nHurt=0, nInjured=0, nWounded=0, nMangled=0, block=0, dodge=0, fortitude=0, description = hey}


\beast{name = Mandrake, species = Flora, mind = Non\minus{}sapient, category = Flora, rating = 3, nUnharmed=0, nBruised=0, nHurt=0, nInjured=0, nWounded=0, nMangled=0, block=0, dodge=0, fortitude=0, description = hey}



}




\species{Giants}
{
	Giants are very large bipdal beings\comma{} growing up to 8m ($\sim$25ft) in height. Many giants appear to be simply humongous humans\comma{} whilst others have disproportionate features\comma{} such as long arms which drag along the floor\comma{} or oversized heads. Rarer still are those which possess more bestial features\comma{} such as elongated fangs\comma{} thick green hide or snouted noses. There is no known rhyme or reason for how such features arise in the giant genetics\comma{} as members of even close family can vary wildly in their outward appearance. 

What is a constant amongst all giants is their phenomenal strength and their willingness to use it. Wizarding history has recorded bloody inter\minus{}giant wars going back millenia\comma{} and there have been more than several occasions of giants turning their ferocity onto wizardkind\comma{} which has led to a general distrust\comma{} fear and even hatred of the giant clans\comma{} which was only exacerbated when the Northern Clans joined forces with Voldemort during the Wizarding Wars. 

\ability{Organised Society}{Though to wizards the giants appear chaotic\comma{} violent and disorganised\comma{} they possess a highly rigid society within and between clans. The {\bf Gurg} is the warleader\comma{} who rules the clan in all matters related to combat: they decide when to send out raiding and hunting parties and when war is declared. The Gurg is the mightiest warrior in the clan – a new Gurg is chosen by another warrior defeating them in single combat\comma{} often to the death. The {\bf Council of Elders} rules the clan in all other matters\comma{} and resolves internal disputes. Membership of the council is determined by a rudimentary form of democracy in a ritual which predates even Athenian democracy. }

\ability{Intelligent Beings}{Though generally not as intelligent as humans\comma{} giants are significantly more intelligent than mere beasts\comma{} as well as other gigantoid bretheren such as Trolls. They are capable of manufacturing and using armour\comma{} weapons and tools both in battle\comma{} and in every day life. In many ways\comma{} giant technology is similar to human society $\sim$2000 years ago. }

\ability{Prideful}{Though their relative lack of intelligence is well known in the wizarding community\comma{} it is a point of some contention amongst the giants\comma{} who wish to be seen as just as intelligent as the other sapient races. Virtually all negative interactions between giants and humans has arisen because the humans condescended giants\comma{} or otherwise lorded their intelligence over the giants. When dealing with giants\comma{} it is almost always best to restrain the urge to show off – use simple sentences and ideas\comma{} give the giants time and space to think things through and\comma{} most importantly\comma{} not make a big deal out of behaving in this fashion. }

\ability{Resilience}{Full\minus{}blooded Giants are potentially even more resistant to magic than dragons\comma{} requiring exceptionally powerful magic to subdue them. Though their skin appears no more protective than normal humans\comma{} their large size means standard weaponry does not pose much of a threat to them\comma{} though it does irritate them. }

\ability{Magic Appreciation}{Though they cannot wield magic themselves\comma{} giants are aware of magic and find the idea of it fascinating (as long as it is not being used agains them that is). The best way to integrate yourself with a giant is to provide them with a spectactular show of magic\comma{} and gift them a magical device or trinket.}
}
{
\beast{name = Giant Civilian, species = Giants, mind = Sapient, category = Gigantoid, rating = 4, nUnharmed=0, nBruised=0, nHurt=0, nInjured=0, nWounded=0, nMangled=0, block=0, dodge=0, fortitude=0, description = Giant civilians\comma{} though monstrous in size and appearance\comma{} are not mighty warriors. Despite this\comma{} their immense strength and resistance to magic means that they can still pose a severe hazard to the unwary wizard.}


\beast{name = Giant Elder, species = Giants, mind = Sapient, category = Gigantoid, rating = 4, nUnharmed=0, nBruised=0, nHurt=0, nInjured=0, nWounded=0, nMangled=0, block=0, dodge=0, fortitude=0, description = Giant elders are civilian leaders\comma{} who have earned respect from their clan not merely through martial prowess (though many were mighty warriors in their youth)\comma{} but through their relatively high intellect.}


\beast{name = Giant Warrior, species = Giants, mind = Sapient, category = Gigantoid, rating = 5, nUnharmed=0, nBruised=0, nHurt=0, nInjured=0, nWounded=0, nMangled=0, block=0, dodge=0, fortitude=0, description = A \name{} dedicates their life to combat – be it hunting for supplies\comma{} or fighting in the mighty giant wars and skirmishes that saturate history.}


\beast{name = Giant Gurg, species = Giants, mind = Sapient, category = Gigantoid, rating = 6, nUnharmed=0, nBruised=0, nHurt=0, nInjured=0, nWounded=0, nMangled=0, block=0, dodge=0, fortitude=0, description = \blindtext{5}}



}




\species{Golem}
{
	b
}
{
\beast{name = Crystal Golem, species = Golem, mind = Ineffable, category = Construct, rating = 5, nUnharmed=0, nBruised=0, nHurt=0, nInjured=0, nWounded=0, nMangled=0, block=0, dodge=0, fortitude=0, description = hey}


\beast{name = Iron Golem, species = Golem, mind = Ineffable, category = Construct, rating = 5, nUnharmed=0, nBruised=0, nHurt=0, nInjured=0, nWounded=0, nMangled=0, block=0, dodge=0, fortitude=0, description = hey}


\beast{name = Stone Golem, species = Golem, mind = Ineffable, category = Construct, rating = 5, nUnharmed=0, nBruised=0, nHurt=0, nInjured=0, nWounded=0, nMangled=0, block=0, dodge=0, fortitude=0, description = hey}



}




\species{Guardian Spirit}
{
	c
}
{
\beast{name = Kneazle, species = Guardian Spirit, mind = Sapient, category = Sprite, rating = 2, nUnharmed=0, nBruised=0, nHurt=0, nInjured=0, nWounded=0, nMangled=0, block=0, dodge=0, fortitude=0, description = hey}


\beast{name = Porlock, species = Guardian Spirit, mind = Non\minus{}sapient, category = Sprite, rating = 2, nUnharmed=0, nBruised=0, nHurt=0, nInjured=0, nWounded=0, nMangled=0, block=0, dodge=0, fortitude=0, description = hey}



}




\species{Hag}
{
	d
}
{
\beast{name = Druidic Hag, species = Hag, mind = Sapient, category = Humanoid, rating = 5, nUnharmed=0, nBruised=0, nHurt=0, nInjured=0, nWounded=0, nMangled=0, block=0, dodge=0, fortitude=0, description = hey}


\beast{name = Hag, species = Hag, mind = Sapient, category = Humanoid, rating = 5, nUnharmed=0, nBruised=0, nHurt=0, nInjured=0, nWounded=0, nMangled=0, block=0, dodge=0, fortitude=0, description = hey}


\beast{name = Necrotic Hag, species = Hag, mind = Sapient, category = Humanoid, rating = 6, nUnharmed=0, nBruised=0, nHurt=0, nInjured=0, nWounded=0, nMangled=0, block=0, dodge=0, fortitude=0, description = hey}


\beast{name = Water Hag, species = Hag, mind = Sapient, category = Humanoid, rating = 5, nUnharmed=0, nBruised=0, nHurt=0, nInjured=0, nWounded=0, nMangled=0, block=0, dodge=0, fortitude=0, description = hey}



}




\species{Hellion}
{
	b
}
{
\beast{name = Hell Hound, species = Hellion, mind = Non\minus{}sapient, category = Demon, rating = 5, nUnharmed=0, nBruised=0, nHurt=0, nInjured=0, nWounded=0, nMangled=0, block=0, dodge=0, fortitude=0, description = hey}


\beast{name = Hellion, species = Hellion, mind = Sapient, category = Demon, rating = 5, nUnharmed=0, nBruised=0, nHurt=0, nInjured=0, nWounded=0, nMangled=0, block=0, dodge=0, fortitude=0, description = hey}



}




\species{Herald}
{
	c
}
{
\beast{name = Augrey, species = Herald, mind = Non\minus{}sapient, category = Sprite, rating = 3, nUnharmed=0, nBruised=0, nHurt=0, nInjured=0, nWounded=0, nMangled=0, block=0, dodge=0, fortitude=0, description = hey}


\beast{name = Pheonix, species = Herald, mind = Sapient, category = Sprite, rating = 5, nUnharmed=0, nBruised=0, nHurt=0, nInjured=0, nWounded=0, nMangled=0, block=0, dodge=0, fortitude=0, description = hey}



}




\species{Hobgoblin}
{
	d
}
{
\beast{name = Erkling, species = Hobgoblin, mind = Non\minus{}sapient, category = Imp, rating = 3, nUnharmed=0, nBruised=0, nHurt=0, nInjured=0, nWounded=0, nMangled=0, block=0, dodge=0, fortitude=0, description = hey}


\beast{name = Gnome, species = Hobgoblin, mind = Non\minus{}sapient, category = Imp, rating = 2, nUnharmed=0, nBruised=0, nHurt=0, nInjured=0, nWounded=0, nMangled=0, block=0, dodge=0, fortitude=0, description = hey}


\beast{name = Gringwart Goff, species = Hobgoblin, mind = Sapient, category = Imp, rating = 5, nUnharmed=0, nBruised=0, nHurt=0, nInjured=0, nWounded=0, nMangled=0, block=0, dodge=0, fortitude=0, description = hey}


\beast{name = Red Cap, species = Hobgoblin, mind = Sapient, category = Imp, rating = 2, nUnharmed=0, nBruised=0, nHurt=0, nInjured=0, nWounded=0, nMangled=0, block=0, dodge=0, fortitude=0, description = hey}



}




\species{Hybrid}
{
	b
}
{
\beast{name = Griffin, species = Hybrid, mind = Non\minus{}sapient, category = Beast, rating = 4, nUnharmed=0, nBruised=0, nHurt=0, nInjured=0, nWounded=0, nMangled=0, block=0, dodge=0, fortitude=0, description = hey}


\beast{name = Hippogriff, species = Hybrid, mind = Non\minus{}sapient, category = Beast, rating = 4, nUnharmed=0, nBruised=0, nHurt=0, nInjured=0, nWounded=0, nMangled=0, block=0, dodge=0, fortitude=0, description = hey}



}




\beast{name = Inhabitor, species = Inhabitor, mind = Ineffable, category = Abomination, rating = 5, nUnharmed=0, nBruised=0, nHurt=0, nInjured=0, nWounded=0, nMangled=0, block=0, dodge=0, fortitude=0, description = hey}



\species{Insectum Magicae}
{
	d
}
{
\beast{name = Billywig, species = Insectum Magicae, mind = Non\minus{}sapient, category = Beast, rating = 1, nUnharmed=0, nBruised=0, nHurt=0, nInjured=0, nWounded=0, nMangled=0, block=0, dodge=0, fortitude=0, description = hey}


\beast{name = Chizpurfle, species = Insectum Magicae, mind = Non\minus{}sapient, category = Beast, rating = 2, nUnharmed=0, nBruised=0, nHurt=0, nInjured=0, nWounded=0, nMangled=0, block=0, dodge=0, fortitude=0, description = hey}


\beast{name = Fire Crab, species = Insectum Magicae, mind = Non\minus{}sapient, category = Beast, rating = 3, nUnharmed=0, nBruised=0, nHurt=0, nInjured=0, nWounded=0, nMangled=0, block=0, dodge=0, fortitude=0, description = hey}


\beast{name = Flesh\minus{}Eating Slug, species = Insectum Magicae, mind = Non\minus{}sapient, category = Beast, rating = 3, nUnharmed=0, nBruised=0, nHurt=0, nInjured=0, nWounded=0, nMangled=0, block=0, dodge=0, fortitude=0, description = hey}


\beast{name = Flobberworm, species = Insectum Magicae, mind = Non\minus{}sapient, category = Beast, rating = 0, nUnharmed=0, nBruised=0, nHurt=0, nInjured=0, nWounded=0, nMangled=0, block=0, dodge=0, fortitude=0, description = hey}


\beast{name = Glumbumble, species = Insectum Magicae, mind = Non\minus{}sapient, category = Beast, rating = 1, nUnharmed=0, nBruised=0, nHurt=0, nInjured=0, nWounded=0, nMangled=0, block=0, dodge=0, fortitude=0, description = hey}


\beast{name = Streeler, species = Insectum Magicae, mind = Non\minus{}sapient, category = Beast, rating = 2, nUnharmed=0, nBruised=0, nHurt=0, nInjured=0, nWounded=0, nMangled=0, block=0, dodge=0, fortitude=0, description = hey}



}




\species{Lich}
{
	b
}
{
\beast{name = Et Iniquum, species = Lich, mind = Sapient, category = Humanoid, rating = 6, nUnharmed=0, nBruised=0, nHurt=0, nInjured=0, nWounded=0, nMangled=0, block=0, dodge=0, fortitude=0, description = When a powerful wizard seeks to defy death\comma{} they may tear their soul asunder\comma{} and store a portion inside a Horcrux. In doing so\comma{} the wizard stops being fully human\comma{} and becomes an {\it Et Iniquii}\comma{} the Tethered Soul. 

Outwardly human\comma{} the Et Iniquii differ only in their affinity for Dark Magic\comma{} and their inability to die.}


\beast{name = \inotator, species = Lich, mind = Sapient, category = Phantasm, rating = 4, nUnharmed=0, nBruised=0, nHurt=0, nInjured=0, nWounded=0, nMangled=0, block=0, dodge=0, fortitude=0, description = When the body of an Et Iniquum or an Aileni is destroyed\comma{} their spirit remains bound to the Mortal Plane\comma{} as long as their Horcrux(es) remain intact. 

With nowhere else to go\comma{} this spirit manifests as a dark apparition called an {\it \inotator{}}. A formless wisp\comma{} the \inotator{} seeks out living beings and whispers intheir ear\comma{} attempting to corrupt them into its service and allow them to merge together. Many Dark Wizards which go down the path of creating a Horcrux have a dedicated leiutenant who will act as a host for them until a new body can be created.}


\beast{name = Aileni, species = Lich, mind = Sapient, category = Humanoid, rating = 6, nUnharmed=0, nBruised=0, nHurt=0, nInjured=0, nWounded=0, nMangled=0, block=0, dodge=0, fortitude=0, description = When an \inotator{} regenerates itself a new body\comma{} the result is something more than human. An imprint of the true form of the being\apos{}s soul is left on the new form \minus{} resulting in a being which is more animalistic\comma{} sadistic\comma{} and much\comma{} much more powerful. 

The most famous Aileni is Lord Voldemort\comma{} who rose as an Aileni in in 1995 after 14 years as an \inotator{}. His pride in his Slytherin heritage imprinted a serpentine form on his Aileni form\comma{} with red eyes and snake\minus{}slit slits\comma{} rather than a nose. Many other deformities have been reported in Aileni throughout history \minus{} Herpo the Foul was said to be more spider than man by the time he died his final death.  

This warping of their human form allows an Aileni to wield more power\comma{} and the further they deviate from their human origins\comma{} the more powerful and deadly they become.}


\beast{name = Aberstath, species = Lich, mind = Ineffable, category = Abomination, rating = 7, nUnharmed=0, nBruised=0, nHurt=0, nInjured=0, nWounded=0, nMangled=0, block=0, dodge=0, fortitude=0, description = An Aberstath is a truly terrifying being\comma{} the result of a Dark Wizard gone too far in their quest for immortality\comma{} leaving behind a shattered shell of their former self\comma{} inhabited by a powerful and destructive demon. 

The process of creating a Horcrux requires the subject to shatter their soul\comma{} which leaves their essence unstable\comma{} pushing them closer to insanity and abject evil. If a Wizard pushes this process to its logical conclusion\comma{} they will eventually damage their soul so much that it can no longer retain cohesiveness\comma{} and it will splinter away from their body. Their quest for immortality\comma{} ironically\comma{} killing them and denying their soul any chance of eternal life beyond the veil. 

This destruction of the soul is the spiritual equivalent of splitting the atom\comma{} releasing a tremendous amount of energy into the higher planes\comma{} even to the point of weakening the barriers between worlds. 

Such a release of energy rarely goes unnoticed by the creeping\comma{} crawling horrors lying in the void – driven insane by their time spent in abject nothingness. The instant they become aware of a free vessel\comma{} they burst through the walls of reality\comma{} and attempt to make their home in the shattered remains of the unfortunate Aileni. 

This forces  the host to undergo a horrific transformation which rips away the last shreds of humanity\comma{} turning them into a dreaded Aberstath. 

\ability{Tears of the Void}{Initially\comma{} the only visible change when a being undergoes the transition from Et Iniquum or Aileni to a full\minus{}blood Aberstath is that the eyes of the Aileni will suddenly turn into inky black orbs\comma{} blacker than the darkest night. Over time\comma{} their form warps and becomes inhumane\comma{} even undead and skeletal. However\comma{} mo matter what form the Aberstath takes\comma{} they cannot hide their voidic eyes.}

\ability{Consumer of Life}{The being which now resides inside the mortal remains of the host was a resident of the Void\comma{} the inky black nothingness between dimensions. Perhaps it is native to that horrifying realm\comma{} or perhaps it was originally a powerful being which fell into the void\comma{} eons ago. Either way\comma{} an Aberstath has been driven to utter insanity by this experience – their motivations are often deeply insane and disturbed\comma{} and mostly focus on the destruction of all life.}

\ability{Memories of the Shell}{The Aberstath retains all the memories\comma{} experiences and skills of its original host.}

\ability{Souless}{With the destruction of the original host\apos{}s soul\comma{} the Aberstath loses the ability to create new Horcruxes themselves. They must rely on harvesting the souls of others to continue their existence. }}



}




\species{Mammalia Magicae}
{
	c
}
{
\beast{name = Dugbog, species = Mammalia Magicae, mind = Non\minus{}sapient, category = Beast, rating = 2, nUnharmed=0, nBruised=0, nHurt=0, nInjured=0, nWounded=0, nMangled=0, block=0, dodge=0, fortitude=0, description = hey}


\beast{name = Jarvey, species = Mammalia Magicae, mind = Non\minus{}sapient, category = Beast, rating = 2, nUnharmed=0, nBruised=0, nHurt=0, nInjured=0, nWounded=0, nMangled=0, block=0, dodge=0, fortitude=0, description = hey}


\beast{name = Mooncalf, species = Mammalia Magicae, mind = Non\minus{}sapient, category = Beast, rating = 1, nUnharmed=0, nBruised=0, nHurt=0, nInjured=0, nWounded=0, nMangled=0, block=0, dodge=0, fortitude=0, description = hey}


\beast{name = Murtlap, species = Mammalia Magicae, mind = Non\minus{}sapient, category = Beast, rating = 2, nUnharmed=0, nBruised=0, nHurt=0, nInjured=0, nWounded=0, nMangled=0, block=0, dodge=0, fortitude=0, description = hey}


\beast{name = Niffler, species = Mammalia Magicae, mind = Non\minus{}sapient, category = Beast, rating = 1, nUnharmed=0, nBruised=0, nHurt=0, nInjured=0, nWounded=0, nMangled=0, block=0, dodge=0, fortitude=0, description = hey}


\beast{name = Wampus Cat, species = Mammalia Magicae, mind = Non\minus{}sapient, category = Beast, rating = 2, nUnharmed=0, nBruised=0, nHurt=0, nInjured=0, nWounded=0, nMangled=0, block=0, dodge=0, fortitude=0, description = hey}



}




\beast{name = Manticore, species = Manticore, mind = Non\minus{}sapient, category = Monstrosity, rating = 6, nUnharmed=0, nBruised=0, nHurt=0, nInjured=0, nWounded=0, nMangled=0, block=0, dodge=0, fortitude=0, description = hey}



\species{Mundane Animal}
{
	b
}
{
\beast{name = Bluebirds, species = Mundane Animal, mind = Non\minus{}sapient, category = Beast, rating = 1, nUnharmed=0, nBruised=0, nHurt=0, nInjured=0, nWounded=0, nMangled=0, block=0, dodge=0, fortitude=0, description = hey}


\beast{name = Viper, species = Mundane Animal, mind = Non\minus{}sapient, category = Beast, rating = 2, nUnharmed=0, nBruised=0, nHurt=0, nInjured=0, nWounded=0, nMangled=0, block=0, dodge=0, fortitude=0, description = hey}



}




\beast{name = Nogtail, species = Beast Demon, mind = Ineffable, category = Demon, rating = 4, nUnharmed=0, nBruised=0, nHurt=0, nInjured=0, nWounded=0, nMangled=0, block=0, dodge=0, fortitude=0, description = hey}



\beast{name = Nundu, species = Nundu, mind = Non\minus{}sapient, category = Beast, rating = 7, nUnharmed=0, nBruised=0, nHurt=0, nInjured=0, nWounded=0, nMangled=0, block=0, dodge=0, fortitude=0, description = hey}



\species{Ogre}
{
	d
}
{
\beast{name = Ghoul, species = Ogre, mind = Non\minus{}sapient, category = Gigantoid, rating = 3, nUnharmed=0, nBruised=0, nHurt=0, nInjured=0, nWounded=0, nMangled=0, block=0, dodge=0, fortitude=0, description = hey}


\beast{name = Troll, species = Ogre, mind = Sapient, category = Gigantoid, rating = 4, nUnharmed=0, nBruised=0, nHurt=0, nInjured=0, nWounded=0, nMangled=0, block=0, dodge=0, fortitude=0, description = hey}


\beast{name = Yeti, species = Ogre, mind = Non\minus{}sapient, category = Gigantoid, rating = 4, nUnharmed=0, nBruised=0, nHurt=0, nInjured=0, nWounded=0, nMangled=0, block=0, dodge=0, fortitude=0, description = hey}



}




\species{Ophidian}
{
	c
}
{
\beast{name = Basilisk, species = Ophidian, mind = Non\minus{}sapient, category = Monstrosity, rating = 6, nUnharmed=0, nBruised=0, nHurt=0, nInjured=0, nWounded=0, nMangled=0, block=0, dodge=0, fortitude=0, description = hey}


\beast{name = Cockatrice, species = Ophidian, mind = Non\minus{}sapient, category = Monstrosity, rating = 5, nUnharmed=0, nBruised=0, nHurt=0, nInjured=0, nWounded=0, nMangled=0, block=0, dodge=0, fortitude=0, description = hey}


\beast{name = Runespoor, species = Ophidian, mind = Sapient, category = Monstrosity, rating = 4, nUnharmed=0, nBruised=0, nHurt=0, nInjured=0, nWounded=0, nMangled=0, block=0, dodge=0, fortitude=0, description = hey}



}




\species{Ornithes Magicae}
{
	b
}
{
\beast{name = Diricawl, species = Ornithes Magicae, mind = Non\minus{}sapient, category = Beast, rating = 2, nUnharmed=0, nBruised=0, nHurt=0, nInjured=0, nWounded=0, nMangled=0, block=0, dodge=0, fortitude=0, description = hey}


\beast{name = Fwooper, species = Ornithes Magicae, mind = Non\minus{}sapient, category = Beast, rating = 2, nUnharmed=0, nBruised=0, nHurt=0, nInjured=0, nWounded=0, nMangled=0, block=0, dodge=0, fortitude=0, description = hey}


\beast{name = Jobberknoll, species = Ornithes Magicae, mind = Non\minus{}sapient, category = Beast, rating = 1, nUnharmed=0, nBruised=0, nHurt=0, nInjured=0, nWounded=0, nMangled=0, block=0, dodge=0, fortitude=0, description = hey}


\beast{name = Snidget, species = Ornithes Magicae, mind = Non\minus{}sapient, category = Beast, rating = 1, nUnharmed=0, nBruised=0, nHurt=0, nInjured=0, nWounded=0, nMangled=0, block=0, dodge=0, fortitude=0, description = hey}



}




\species{Pegasus}
{
	c
}
{
\beast{name = Hippocampus, species = Pegasus, mind = Ineffable, category = Celestial, rating = 4, nUnharmed=0, nBruised=0, nHurt=0, nInjured=0, nWounded=0, nMangled=0, block=0, dodge=0, fortitude=0, description = hey}


\beast{name = Thestral, species = Pegasus, mind = Ineffable, category = Celestial, rating = 3, nUnharmed=0, nBruised=0, nHurt=0, nInjured=0, nWounded=0, nMangled=0, block=0, dodge=0, fortitude=0, description = hey}


\beast{name = Winged Horse, species = Pegasus, mind = Ineffable, category = Celestial, rating = 4, nUnharmed=0, nBruised=0, nHurt=0, nInjured=0, nWounded=0, nMangled=0, block=0, dodge=0, fortitude=0, description = hey}



}




\beast{name = Pogrebin, species = Rock Demon, mind = Sapient, category = Demon, rating = 3, nUnharmed=0, nBruised=0, nHurt=0, nInjured=0, nWounded=0, nMangled=0, block=0, dodge=0, fortitude=0, description = hey}



\species{Puffskeins}
{
	d
}
{
\beast{name = Puffskein, species = Puffskeins, mind = Non\minus{}sapient, category = Beast, rating = 1, nUnharmed=0, nBruised=0, nHurt=0, nInjured=0, nWounded=0, nMangled=0, block=0, dodge=0, fortitude=0, description = hey}


\beast{name = Pygmy Puff, species = Puffskeins, mind = Non\minus{}sapient, category = Beast, rating = 0, nUnharmed=0, nBruised=0, nHurt=0, nInjured=0, nWounded=0, nMangled=0, block=0, dodge=0, fortitude=0, description = hey}



}




\beast{name = Quintaped, species = Quintaped, mind = Sapient, category = Monstrosity, rating = 6, nUnharmed=0, nBruised=0, nHurt=0, nInjured=0, nWounded=0, nMangled=0, block=0, dodge=0, fortitude=0, description = hey}



\species{Raised Dead}
{
	b
}
{
\beast{name = Banshee, species = Raised Dead, mind = Ineffable, category = Undead, rating = 4, nUnharmed=0, nBruised=0, nHurt=0, nInjured=0, nWounded=0, nMangled=0, block=0, dodge=0, fortitude=0, description = hey}


\beast{name = Inferius, species = Raised Dead, mind = Ineffable, category = Undead, rating = 4, nUnharmed=0, nBruised=0, nHurt=0, nInjured=0, nWounded=0, nMangled=0, block=0, dodge=0, fortitude=0, description = hey}



}




\species{Salamanders}
{
	d
}
{
\beast{name = Ashwinder, species = Salamanders, mind = Non\minus{}sapient, category = Elemental, rating = 3, nUnharmed=0, nBruised=0, nHurt=0, nInjured=0, nWounded=0, nMangled=0, block=0, dodge=0, fortitude=0, description = hey}


\beast{name = Fire Salamander, species = Salamanders, mind = Non\minus{}sapient, category = Elemental, rating = 3, nUnharmed=0, nBruised=0, nHurt=0, nInjured=0, nWounded=0, nMangled=0, block=0, dodge=0, fortitude=0, description = hey}


\beast{name = Frost Salamander, species = Salamanders, mind = Non\minus{}sapient, category = Elemental, rating = 3, nUnharmed=0, nBruised=0, nHurt=0, nInjured=0, nWounded=0, nMangled=0, block=0, dodge=0, fortitude=0, description = hey}


\beast{name = Moke, species = Salamanders, mind = Non\minus{}sapient, category = Elemental, rating = 1, nUnharmed=0, nBruised=0, nHurt=0, nInjured=0, nWounded=0, nMangled=0, block=0, dodge=0, fortitude=0, description = hey}



}




\beast{name = Shadow Demon, species = Shadow Demons, mind = Ineffable, category = Demon, rating = 5, nUnharmed=0, nBruised=0, nHurt=0, nInjured=0, nWounded=0, nMangled=0, block=0, dodge=0, fortitude=0, description = hey}



\species{Soldiers of the Abyss}
{
	
}
{
\beast{name = Death Hunter, species = Soldiers of the Abyss, mind = Ineffable, category = Abomination, rating = 5, nUnharmed=0, nBruised=0, nHurt=0, nInjured=0, nWounded=0, nMangled=0, block=0, dodge=0, fortitude=0, description = A disgusting pale\minus{}white creature with 6 legs and a face which is entirely featureless except a gaping red maw. 

Incredibly fast\comma{} and with powerful alien senses the Death Hunter serves one purpose: to track and hunt down their target across dimensions. When their target is located\comma{} they release an earsplitting howl across dimensions\comma{} drawing their allies in.}


\beast{name = Flesh Ripper, species = Soldiers of the Abyss, mind = Ineffable, category = Abomination, rating = 6, nUnharmed=0, nBruised=0, nHurt=0, nInjured=0, nWounded=0, nMangled=0, block=0, dodge=0, fortitude=0, description = A hulking mass of rotting\comma{} oily flesh covered in a bony exoskeleton which seems to twist and warp in multiple dimensions. The Flesh Ripper\apos{}s role is as a heavy assault weapon. It is aided in this by it\apos{}s long\comma{} scorpion like tail\comma{} which ends in a vicious barb. This tail is used to spear beings\comma{} before dragging them\comma{} helpless\comma{} into its maw.}


\beast{name = Mind Shredder, species = Soldiers of the Abyss, mind = Ineffable, category = Abomination, rating = 6, nUnharmed=0, nBruised=0, nHurt=0, nInjured=0, nWounded=0, nMangled=0, block=0, dodge=0, fortitude=0, description = A monstrous female form\comma{} all protruding fangs and tentacles lies behind layers of illusions and deception. Created by the Abyssal Masters for the express purpose of infiltration of the minds of their foes\comma{} the Mind Shredder is weak in body\comma{} but wields immense psionic powers.}


\beast{name = Soul Devourer, species = Soldiers of the Abyss, mind = Ineffable, category = Abomination, rating = 6, nUnharmed=0, nBruised=0, nHurt=0, nInjured=0, nWounded=0, nMangled=0, block=0, dodge=0, fortitude=0, description = The Soul Devourer is every inch as terrifying as its name would suggest\comma{} with a face seemingly composed of a rotting\comma{} skeletal deer skull with enormous antlers protruding from it. Mounted atop the antlers are a series of dripping candles\comma{} said to a allow the Soul Devourer to act as a guide throught he inky darkness of the Void\comma{} as they travel to do their master\apos{}s bidding. 

As the arcane experts of the Abyssal Army\comma{} the Soul Devourers\comma{} as their name might suggest\comma{} rely heavily on the power of ritual sacrifices and the corruption and absorption of slain souls\comma{} even using the souls of their slain allies to power their assaults. Their long\comma{} spider\minus{}like hands are continually dripping with thick\comma{} congealed blood which they use in their profane rituals\comma{} and as a powerful weapon. 

Of all of the Abyssal Spawn\comma{} it seems that the Soul Devourers have the most amount of free will\comma{} and will often explicitly communicate that they are allied with their Abyssal Masters out of convenience\comma{} rather than the rabid\comma{} unthinking loyalty expressed by some of the other soldiers in the Profane Masses. 

Whispered rumours in some corners of the Multiverse indicate that some Soul Devourers have even gone rogue\comma{} and attempted to rebel against their masters\comma{} some have even had a degree of success. That the Soul Devourers remain such an integral part of the Abyssal Army indicates that this rebellious streak is in some way beneficial to their masters\comma{} though no one is quite sure why or how this might be.}


\beast{name = Wielder\minus{}of\minus{}Night, species = Soldiers of the Abyss, mind = Ineffable, category = Abomination, rating = 7, nUnharmed=0, nBruised=0, nHurt=0, nInjured=0, nWounded=0, nMangled=0, block=0, dodge=0, fortitude=0, description = In the midst of a pitched battle between the forces of good and evil\comma{} light and dark\comma{} life and death\comma{} one figure stands apart. A grotesque\comma{} scarred humanoid body is sewn onto a countless mass of tentacles which seem to boil and writhe of their own accord. Standing tall over the immense\comma{} endless battlefield\comma{} it surveys the chaos; planning and strategising. 

With a movement as fast as the eye can see\comma{} the abomination leaps into battle\comma{} dueling hand\minus{}to\minus{}hand with the most powerful servants of the light\comma{} firing bolts of dark arcane energy from a finger\comma{} and screeching orders to the untold legions of the Abyss. 

This is the Wielder\minus{}of\minus{}Night\comma{} the Commander of the Abyssal Legions\comma{} Slayer of Hope\comma{} Destroyer of Worlds\comma{} and the reason that every sane being in the multiverse is afraid of the darkness. 

It is unknown if the Wielder\minus{}of\minus{}Night is a position\comma{} akin to a general\comma{} or if there is only one of them\comma{} a single mighty champion of the Eldritch Horroers\comma{} who has executed their will throughout the eons. Most scholars hope that it is the latter\comma{} for the idea of there being multiple such beings is too horrible to even consider. 

There is only one recorded instance of a Wielder\minus{}of\minus{}Night stepping onto the Mortal Realm\comma{} at the culmination of the final battle of the Cataclysm. Merlin the Great\comma{} the most powerful Wizard who ever lived\comma{} managed to barely survive its onslaught\comma{} until he tricked it into falling through a Rift in time and space\comma{} which he sealed behind it. So harrowing was this experience\comma{} that Merlin refused to talk about it for hundreds of years after the event.}



}




\beast{name = Solon, species = Light Elemental, mind = Non\minus{}Sapient, category = Elemental, rating = 3, nUnharmed=1, nBruised=1, nHurt=1, nInjured=1, nWounded=0, nMangled=0, block=4, dodge=1, fortitude=2, description = \name{}s are lesser elementals hailing from the Radiant Gardens\comma{} the Elemental plane of light. They are crystalline creatures\comma{} and glow with an inner radiance which shifts and refracts through their bodies as they move.}



\beast{name = Swooping Evil, species = Swooping Evil, mind = Non\minus{}sapient, category = Monstrosity, rating = 6, nUnharmed=0, nBruised=0, nHurt=0, nInjured=0, nWounded=0, nMangled=0, block=0, dodge=0, fortitude=0, description = hey}



\beast{name = Thunderbird, species = Thunderbird, mind = Non\minus{}sapient, category = Elemental, rating = 6, nUnharmed=0, nBruised=0, nHurt=0, nInjured=0, nWounded=0, nMangled=0, block=0, dodge=0, fortitude=0, description = hey}



\species{Unicorns}
{
	
}
{
\beast{name = Bicorn, species = Unicorns, mind = Sapient, category = Celestial, rating = 5, nUnharmed=0, nBruised=0, nHurt=0, nInjured=0, nWounded=0, nMangled=0, block=0, dodge=0, fortitude=0, description = hey}


\beast{name = Tebo, species = Unicorns, mind = Non\minus{}sapient, category = Celestial, rating = 4, nUnharmed=0, nBruised=0, nHurt=0, nInjured=0, nWounded=0, nMangled=0, block=0, dodge=0, fortitude=0, description = hey}


\beast{name = Unicorn, species = Unicorns, mind = Sapient, category = Phantasm, rating = 6, nUnharmed=0, nBruised=0, nHurt=0, nInjured=0, nWounded=0, nMangled=0, block=0, dodge=0, fortitude=0, description = hey}



}




\species{Water Demons}
{
	
}
{
\beast{name = Grindylow, species = Water Demons, mind = Non\minus{}sapient, category = Demon, rating = 3, nUnharmed=0, nBruised=0, nHurt=0, nInjured=0, nWounded=0, nMangled=0, block=0, dodge=0, fortitude=0, description = hey}


\beast{name = Kappa, species = Water Demons, mind = Ineffable, category = Demon, rating = 4, nUnharmed=0, nBruised=0, nHurt=0, nInjured=0, nWounded=0, nMangled=0, block=0, dodge=0, fortitude=0, description = hey}


\beast{name = Kelpie, species = Water Demons, mind = Ineffable, category = Demon, rating = 5, nUnharmed=0, nBruised=0, nHurt=0, nInjured=0, nWounded=0, nMangled=0, block=0, dodge=0, fortitude=0, description = hey}



}




\species{Werewolf}
{
	
}
{
\beast{name = Human\minus{}form, species = Werewolf, mind = Sapient, category = Humanoid, rating = 4, nUnharmed=0, nBruised=0, nHurt=0, nInjured=0, nWounded=0, nMangled=0, block=0, dodge=0, fortitude=0, description = hey}


\beast{name = Wolf\minus{}form, species = Werewolf, mind = Non\minus{}sapient, category = Monstrosity, rating = 6, nUnharmed=0, nBruised=0, nHurt=0, nInjured=0, nWounded=0, nMangled=0, block=0, dodge=0, fortitude=0, description = hey}



}




\beast{name = Whomping Willow, species = Trees, mind = Non\minus{}sapient, category = Flora, rating = 4, nUnharmed=0, nBruised=0, nHurt=0, nInjured=0, nWounded=0, nMangled=0, block=0, dodge=0, fortitude=0, description = hey}



\species{Wyverns}
{
	
}
{
\beast{name = Occamy, species = Wyverns, mind = Non\minus{}sapient, category = Draconid, rating = 5, nUnharmed=0, nBruised=0, nHurt=0, nInjured=0, nWounded=0, nMangled=0, block=0, dodge=0, fortitude=0, description = hey}


\beast{name = Sea Serpent, species = Wyverns, mind = Non\minus{}sapient, category = Draconid, rating = 5, nUnharmed=0, nBruised=0, nHurt=0, nInjured=0, nWounded=0, nMangled=0, block=0, dodge=0, fortitude=0, description = hey}



}





