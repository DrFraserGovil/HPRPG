\newcommand\type[2]
{
	\key{#1:} #2
}
\newcommand\squote[1]
{
``#1"
}

\chapter{Classifying Beasts}

Throughout the centuries that wizardkind has studied the magical and mundane creatures of the world, there have been many different attempts to classify them into some kind of coherent taxonomy. Of these, two different systems have become considered the conventional mehtod of classifying beings of all kinds - though of course the exact boundaries remains a matter of intense debate. 

The debate has never been fully settled, confused even further by the introduction of a {\it third} system by the Ministry of Magic. These three systems of classification are each useful for determining different aspects of a beast, so are all presented here. These three classification systems are:

\begin{itemize}
	\boldItem{The Mind}{The level of sapience and self-awareness possessed by the creature}
	\boldItem{The Type}{A grouping based on morphological similarities and common points of origin.}
	\boldItem{The Threat}{A classification based on how dangerous the creature is and the threat it poses to the wizarding world.}
\end{itemize}

\section{Minds}

The \key{mind} possessed by a being determines how they think, behave and perceive the world.

\subsection{Sapient}

\key{Sapients} are those creatures with consciousness, and intrinsic awareness of the self. Alongside this (usually) comes intelligence, language and society. All \imp{humanoids} are considered \imp{Sapients}, though not all \imp{Sapients} are \imp{humanoids}.  

For political reasons, the word \key{Beings} is often used to describe \imp{Sapient} creatures, with \key{Beasts} reserved for \imp{Non-sapient} creatures. However this is considered a very politically charged term, and \imp{Sapients} such as Centaurs and the Merpeople object to sharing this category with, for example, the hags, and hence are often classified as {\it beasts}, despite their evidently sapient nature. 

\imp{Sapient} creatures are often able to use magic, and are capable of adapting and formulating complex tactics and plans. When controlling a \imp{Sapient} creature and deciding how they would act, the GM should bear this in mind, and allow them to strategise, coordinate and use the environment and items within it to their advantage.

\subsection{Non-sapient}

\key{Non-sapient} creatures are those which, whilst possessing a (mostly) recognisable brain, containing recognisable thoughts, do not possess a true consciousness. 

This should not be confused with a lack of intelligence: some \imp{non-sapient} creatures have analytical and problem-solving skills which far outsrip a human. However, their lack of consciousness generally means that they lack the ability to reason and make conscious decisions - they instead rely purely on their more animalistic instincts. 

Whilst generally considered to lie outside the axis of `good' and `evil', due to their intrinsic lack of morality and ethics, some \imp{non-sapient} creatures can be incredibly caring, whilst others are vicious. When a \imp{non-sapient} being is described as `good' or `evil', it should therefore be understood in these more primal terms.  

\subsection{Ineffable}

A creature which possesses an \key{Ineffable} mind has a consciousness that is beyond the realm of the humanoid mind to conceive. The very classification of sapience or not is entirely irrelevant to their being. Spirits, and abominations are generally considered `ineffable', as are the most powerful celestials.  

The term \key{unliving} is also used to apply to beings which have an \imp{ineffable} mind, due to the popular image that such creatures are not truly `alive' in the sense that we would consider them. 

\imp{Ineffable} creatures often originate from extraplanar dimensions, or were created by ancient and primal magics. They are therefore often susceptible to \imp{Celestial} attacks, which uses alien energy to strip away at whatever constitutes a soul for these creatures. 


\section{Types}

The \key{type} of a creature denotes how creatures are related to each other, and gives a hint at their intrinsic nature. Creatures which share a \imp{type} often have many characteristics in common, both visually and in terms of the magic and power that they wield.

Though often closely linked, many creatures of the same \imp{type} have a different kind of \imp{mind}. 


\type{Abomination}
{	
	An \imp{abomination} is an incomprehensibly alien creature from the depths of the \imp{Eldritch Domains}\comma{} or even the \imp{Void Beyond}. Primal\comma{} extraplanar beings\comma{} even attempting to comprehend the existence of such creatures is enough to break the minds of weaker individuals.
}

\type{Beast}
{
	A \imp{beast} is a (generally) \imp{non\minus{}sapient} creature of magical or mundane nature\comma{} which forms a natural part of the life cycle in their environment. Almost all non\minus{}magical creatures are classified as \imp{beasts}\comma{} as are many of the most common magical creatures.
}

\type{Celestial}
{
	\imp{Celestials} are natives of some of the more distant higher planes\comma{} such as the heavenly \imp{Elysium}\comma{} or the awful \imp{Tartarus}. \imp{Angels}\comma{} \imp{Devils} and other beings form the bulk of the \imp{Celestials}\comma{} normally possessing incredible power they have\comma{} throughout history\comma{} been mistaken for servants of the Gods and sometimes even for gods in their own right.
}

\type{Construct}
{
	A \imp{construct} is an artificially created being. Usually constructed from inorganic materials such as metal\comma{} stone or clay and animated using powerful magic or technological means. Though not considered {\it alive}\comma{} some rare constructs do contain a \imp{Sapient} mind.
}

\type{Demon}
{
	\imp{Demons} are malevolent magical creatures\comma{} often posessing an intrinsic affinity for the \imp{Dark Arts}\comma{} and a thirst for human flesh. Demons can take many forms\comma{} and can be found across the multiverse. Some demons\comma{} like elementals\comma{} harbour an affinity for a certain aspect of the universe\comma{} others serve powerful beings\comma{} and some demons rise to power in their own right and crown themselves \key{Demon Princes}. Over the centuries, most of the truly horrifying demons have been banished from the mortal realm\comma{} leaving behind only minor evils such as the Grindylow or the Kappa. Sometimes\comma{} however\comma{} a Dark Witch or Wizard reaches through the barriers between worlds and pulls one of the more abhorrent powers into this world.
}

\type{Draconid}
{
	A dragon or dragon\minus{}like creature would be classified as a \imp{Draconid}. Usually characterised by an enormous reptilian form and affinity for elemental flame, and often possessing both incredible physical and magical power\comma{} any member of the Draconid family should be treated with fear and respect\comma{} the True Dragons most of all.
}

\type{Elemental}
{
	\imp{Elementals} are creatures which embody one of the classical elements: fire\comma{} air\comma{} water\comma{} or earth (as well as many others). Most hail from one of the \imp{Elemental Planes}\comma{} though many magical creatures native to the Mortal Plane are considered Elementals\comma{} such as the Ashwinder Snake\comma{} or the Frost Salamander.
}

\type{Flora}
{
	Strictly speaking\comma{} \imp{flora} is a catchall term for all plant life. In this context\comma{} however\comma{} it includes a range of magical plants\comma{} imbued with a degree of ambulation\comma{} movement or other means of interacting with the outside world.
}

\type{Gigantoid}
{
	The \imp{gigantoids} are a family of oversized human\minus{}esque creatures. Though large in frame\comma{} they are often incredibly dim\minus{}witted and slow. Trolls\comma{} ogres and giants form the core of the \imp{gigantoid} family.
}

\type{Humanoid}
{
	The group of beings generally referred to as {\it people}\comma{} the \imp{humanoid} groups comprises of all the human subspecies \minus{} both wizarding and muggle \minus{} as well as the semi\minus{}human creatures such as Centaurs\comma{} Merpeople\comma{} Goblins and Veela. Half\minus{}giants often find themselves in the humanoid category\comma{} whilst their full\minus{}giant kin are considered \imp{Gigantoids}.
}

\type{Imp}
{
	The \imp{imps} are vaguely humanoid creatures\comma{} though besides the Elves\comma{} they mostly possess only limited intellect. An \imp{imp} is immediately distinguised from even the shortest dwarf by their diminutive stature (rarely reaching more than 2 feet in height)\comma{} and their innate magic which seems to operate on entirely different rules to that used by most humanoids. Elves\comma{} hobgoblins and fairies are the most prominent member of the \imp{imp} family.
}

\type{Monster}
{
	Many beings classified as \imp{monsters} could feasibly be considered \imp{beasts}\comma{} in the strictest sense of the word. However\comma{} whilst a \imp{beast} can live in harmony inside its ecological niche (even if that necessitates aggression and special abilities)\comma{} a \imp{monster} is nothing but a disruptive and lethal influence\comma{} and often form the centre of dangerous myths and legends. \imp{Monsters} are almost universally destructive\comma{} vicious and incredibly dangerous to face.
}

\type{Phantasm}
{
	A \imp{phantasm} is a non\minus{}corporeal or ghostly being\comma{} often associated with the souls of departed individuals\comma{} and manifestations of primal forces in the mortal plane.
}

\type{Sprite}
{
	\imp{Sprites} are creatures which straddle worlds\comma{} often existing as much in the Astral Realm as they do in the mortal realm. Sometimes corporeal\comma{} and sometimes ghostly\comma{} the \imp{sprites} are united in their overarching goodness and fondness for living beings. Often considered by muggles to be guardian spirits\comma{} the Sprites often choose an area or a domain to protect\comma{} and their rage when their protection is violated can be potent.
}

\type{Undead}
{
	The \imp{undead} are profane creations\comma{} the mortal remains of a once\minus{}living creature reanimated by powerful necromantic magic\comma{} or posessed by an evil spirit. The Walking Corpses\comma{} as well as Vampires\comma{} fall into this category.
}



%\begin{strip}
\section{Rating}

The Department of Magical Beasts, an important part of the Ministry of Magic, maintains a classification scheme to determine the threat posed by individual magical creatures, labelling creatures between 0 and VII. A creature with a low rating can be dealt with easily, whilst a rating of V or above is an immediate cause for concern.


\newcommand\vRow[3]{
\key{#1}		&	#3	\\
}
\begin{center}
\begin{rndtable}{c p{7.7 cm}}
\bf Category  & \bf Description
\\
	\vRow{0}{0}{Utterly harmless, incapable of inflicting harm}
	\vRow{I}{2}{\squote{Boring}, capable of inflicting only tiny injuries}
	\vRow{II}{10}{Mostly harmless, commonly domesticated}
	\vRow{III}{25}{Poses only minimal danger to a capable individual}
	\vRow{IV}{50}{A group of competent individuals can handle, though an individual would face serious harm.}
	\vRow{V}{75}{Requires specialist knowledge, or a group of highly trained individuals to defeat}
	\vRow{VI}{100}{Known Wizard-Killer, impossible to control or train. Requires a large group of exceptionally trained warriors to defeat}
	\vRow{VII}{150}{Lethal, poses a viable extinction-level threat to population centres if left unchecked. Few-to-no examples in recorded history of wizards successfully defeating them.}
\end{rndtable}
\end{center}
%\end{strip}


